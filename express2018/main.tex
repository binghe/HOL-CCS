%%%% -*- Mode: LaTeX -*-
%%
%% This is the draft of the 2nd part of EXPRESS/SOS 2018 paper, co-authored by
%% Prof. Davide Sangiorgi and Chun Tian.

\documentclass[submission]{eptcs} % required template (EPTCS)
\providecommand{\event}{EXPRESS/SOS 2018} % Name of the event you are submitting to

\usepackage[T1]{fontenc}
\usepackage{lmodern} % Latin Modern fonts (better than CM-Super)
\usepackage[utf8]{inputenc} % no need for XeTeX, which always uses UTF-8

%% Math symbol packages
\usepackage{amsmath}
\usepackage{amssymb}
\usepackage{mathrsfs}

\usepackage{stmaryrd}
\SetSymbolFont{stmry}{bold}{U}{stmry}{m}{n}
% \cupdot (the best one)
\newcommand{\cupdot}{\mathbin{\mathaccent\cdot\cup}}

\usepackage{amsthm}

\newtheorem{definition}{Definition}[section]
\newtheorem{example}[definition]{Example}
\newtheorem{lemma}[definition]{Lemma}
\newtheorem{theorem}[definition]{Theorem}
\newtheorem{corollary}[definition]{Corollary}
\newtheorem{proposition}[definition]{Proposition} 
\newtheorem{remark}[definition]{Remark}

% HOL theorem embedding support
\usepackage{holindex}
\usepackage{alltt}
\include{tokens}

\usepackage{listings}
\renewcommand{\ttdefault}{cmtt} % {pcr}
\lstset{tabsize=8,language=ML,basicstyle=\small\ttfamily\bfseries}

%
% LaTeX math formula spacing:
%
% \enskip	leave a horizontal space of respectively half an em
% \quad		space equal to the current font size (= 18 mu), one em
% \qquad	twice of \quad (= 36 mu), i.e. two ems
% \enspace	it's inherited from Plain TeX and is almost the same
%                      as \enskip, but technically it is a kern, rather than a skip.
% \,	3/18 of \quad (= 3 mu)
% \:	4/18 of \quad (= 4 mu)
% \;	5/18 of \quad (= 5 mu)
% \!	-3/18 of \quad (= -3 mu)
% \ (space after backslash!) equivalent of space in normal text

\usepackage{graphicx}
\usepackage[all,cmtip]{xy}

% NAMES and VARIABLES

\def\Names{{\cal N}}            % set of all names
\def\fn#1{\rmsf{fn}(#1)}         % free names
\def\fv#1{\rmsf{fv}(#1)}         % free variables
\def\bv#1{\rmsf{bv}(#1)}         % bound variables
\def\bn#1{\rmsf{bn}(#1)}         % bound names

\newcommand{\dom}[1]{{\rmsf{dom}}(#1)} % the domain of something  

% FOR PROCESSES 

\def\nil{{\boldsymbol 0}} % nil
\def\res#1{{\boldsymbol \nu} #1\:}   % restriction
%% the following definitions allow us to use the symbols ! . and | 
%directly, for the replication, prefix and parallel compoisition
%operators in math mode 
\mathcode`\!="4021 % `!' as prefix operator
\mathcode`\.="602E  % prefix 
\mathcode`\|="326A % `|' as relation operator

\newcommand{\outC}[1]{\overline{#1}}      % CCS  output
\newcommand{\inpC}[1]{#1}                 % CCS  input

\newcommand{\out}[2]{\overline{#1}\langle{#2}\rangle} % output with value    
\newcommand{\inp}[2]{#1(#2)}  % input with value
\newcommand{\inpW}[2]{#1(#2). }  % input with value and dot

\newcommand{\iae}[2]{{#1}\langle{#2}\rangle} % input with value    


\newcommand{\cond}[3]{\myif\ #1\ \mythen\ #2\ \myelse\ #3} % if-then-else  
\newcommand{\myif}{\myspace{\rmtt{if}}\myspace}            % ``if''
\newcommand{\mythen}{\myspace{\rmtt{then}}\myspace}        % ``then''
\newcommand{\myelse}{\myspace{\rmtt{else}}\myspace}        % ``else''

\newcommand{\true}{\rmsf{true}} %boolean true
\newcommand{\false}{\rmsf{false}} %boolean false


% substitutions  (used as a postfixed operator)
\def\sub#1#2{\{\raisebox{.5ex}{\small$#1$}\! / \!\mbox{\small$#2$}\}} 



% TRANSITIONS (arrows)

\newcommand{\racap}{\mathrel{\stackrel{{\;\; {{\wedge}} \;\;}}{\mbox{\rightarrowfill}}}} 

\newcommand{\arr}[1]{\mathrel{\stackrel{{\;\;#1\;\;}}{\mbox{\rightarrowfill}}}}
                                %  strong labelled transition 

\newcommand{\Arr}[1]{\mathrel{\stackrel{{\;\;#1\;\;}}{\mbox{\rightarrowfillWEAK}}}} 
                                    %weak  labelled transitions
                                    
\newcommand{\arcap}[1]{\mathrel{\stackrel{{\;\; {\widehat{#1}} \;\;}}{\mbox{\rightarrowfill}}}} 
                                    %strong labelled transitions with hat
\newcommand{\Arcap}[1]{\mathrel{\stackrel{{\;\;{\widehat{#1}}\;\;}}{\mbox{\rightarrowfillWEAK}}}}
                                    %weak labelled transitions with   hat
                                    
% FOR CONTEXTS

\newcommand{\contexthole}{ [ \cdot  ] }      %hole of a context
\newcommand{\ct}[1]{ C \brac{#1} }   %filled context
\newcommand{\qct}{ C  }              %unfilled context  
\newcommand{\brac}[1]{[#1] }   % brackets for the context hole



%% SOME STYLE COMMANDS
\newcommand{\rmtt}[1]{{\rm\tt{#1}}} % for keywords like ``if'', ``then'' ... 
\newcommand{\rmsf}[1]{{{\rm\sf{#1}}}}


%BEHAVIOURAL  EQUIVALENCES and relations

\def\R{{\cal R}}          % R without spaces around
\def\RR{\mathrel{\cal R}} % R with some space around
\def\S{{\cal S}}          % S without spaces around
\def\SS{\mathrel{\cal S}} % S with some space around


\newcommand{\equival}{=} %\mathrel{\equiv_\alpha} 
                          % equality up to alpha conversion 





%SPECIAL SYMBOLS


\def\midd{\; \; \mbox{\Large{$\mid$}}\;\;}
               %separation symbol in  grammars               

\def\st{\; \mid \;} % ``such that'' in formulas
\def\DSdefi{\stackrel{\mbox{\scriptsize def}}{=}} % definition equal

\def\Defi{\stackrel{\mbox{\scriptsize $\triangle$}}{=}} % definition equal


% \def\qed{}
%  {\unskip\nobreak\hfil\penalty50\hskip1em\null\nobreak\hfil
%   $\Box$\parfillskip=0pt\finalhyphendemerits=0\endgraf}
%                    % end of proofs (or theorems,results without proofs)

\newcommand{\myspace}{\:}  % some spacing abbreviation

% rename tilde to widetilde, to be used for tuples
\renewcommand{\tilde}{\widetilde}


% % environment for proofs (for CUP style file)
% \newenvironment{proof}{\noindent {\bf Proof }}{\qed \bigskip}


\newcommand{\finish}[1]{ \vskip .2cm  {\bf #1} \vskip .2cm   \marginpar{{\bf $DS$}}}


% NEW THINGS 

\newcommand{\mylabel}[1]{{\rm (#1).}}
\newcommand{\MYsketch}{[Sketch] }


\newcommand{\behav}{equation}
\newcommand{\behavC}{contraction}


\newcommand{\hb}{\hskip .5cm}

\newcommand{\ArrN}[2]{\mathrel{\stackrel{{\;\;#1\;\;}}{\mbox{\rightarrowfillWEAK}}_{#2}
  }} 
                                    %weak  labelled transitions weighted

\newcommand{\Var}{{\cal X}}

\newcommand{\AL}{{\cal RL}}
\newcommand{\PL}{{\cal L}}

\newcommand{\sign}{\Sigma}
\newcommand{\prsign}{\pr_\sign}


% possible actions in Abnients:
\newcommand{\qina}{{\ccc{in}} } 
\newcommand{\qout}{{\ccc{out}} }
\newcommand{\qopen}{{\ccc{open}} }
\newcommand{\capa}{{\ccc{capa}} }
\newcommand{\amb}[2]{#1 [\, #2 \,] } % ambients

\newcommand{\HOAMB}{\mbox{\rm{HO$\pi$Amb}}}
\newcommand{\SeqCCS}{\mbox{\rm{SeqCCS}}}


\newcommand{\SE}{{E\!S}}
\newcommand{\SL}{{L\!S}}
\newcommand{\SEp}{{E\!S'}}
\newcommand{\SEpU}{{E\!S'_1}}


%\newcommand{\mcontrP}[1]{\mathrel{\stackrel{{\footnotesize{\mbox{$\succ$}}}}{\footnotesize{\mbox{$#1$}}}}}
\newcommand{\mexpaP}[1]{\mathrel{\stackrel{{\footnotesize{\mbox{$\prec$}}}}{\footnotesize{\mbox{$#1$}}}}}

\newcommand{\wc}{\mathrel{\approx^{\rm c}}}
\newcommand{\wb}{\approx}
\newcommand{\contr}{\mathrel{\succeq_{\rm{e}}}}
\newcommand{\expa}{\mathrel{\preceq_{\rm{e}}}} 


\newcommand{\mcontr}{\mathrel{\succeq}}
\newcommand{\mexpa}{\mathrel{\preceq}}

\newcommand{\mcontrmay}{\mathrel{\succeq_{\rm{ctx}}}}
\newcommand{\mexpamay}{\mathrel{\preceq_{\rm{ctx}}}}


\newcommand{\mcontrTE}{\mathrel{\succeq_{\rm{tr}}}}
\newcommand{\TE}{\approx_{\rm tr}}       %trace equivalence


\newcommand{\mcontrBIS}{\mathrel{\succeq_{\rm{bis}}}}
\newcommand{\mexpaBIS}{\mathrel{\preceq_{\rm{bis}}}}


\newcommand{\rcontr}{\mathrel{\succeq^{\rm{c}}_{\rm{bis}}}}
\newcommand{\rexpa}{\mathrel{\preceq^{\rm{c}}_{\rm{bis}}}}


\newcommand{\til}{\tilde}

\newcommand{\ctx}[1]{#1^{\rm{c}}}

\newcommand{\Dwaleq}[1]{\Dwa^{\leq #1}}
\newcommand{\Dwageq}[1]{\Dwa^{\geq #1}}


\newcommand{\holeDS}{ [ \cdot  ] }
\newcommand{\holei}[1]{[\cdot]_{#1}}
\newcommand{\ctp}[1]{ C' \brac{#1} }  %primed context       
\newcommand{\qctp}{ C'  }
\newcommand{\ctpp}[1]{ C'' \brac{#1} }  %primed context       
\newcommand{\ctppp}[1]{ C''' \brac{#1} }  %primed context       

\newcommand{\qctpp}{ C''  }
\newcommand{\qctppp}{ C'''  }


\newcommand{\may}{\approx_{\rm ctx}}       %may equivalence

\newcommand{\mayHA}{\approx^{\rm{HAmb}}_{\rm ctx}}       %may equivalence

\newcommand{\hk}{\hskip .2cm }
\newcommand{\mysp}{10pt}
\newcommand{\tkp}{10pt}
\newcommand{\tkpS}{6pt}
\newcommand{\tkpSS}{3pt}
\newcommand{\tkpP}{15pt}

\newcommand{\smay}{\mathrel{\sim_{\rm ctx}}}       %may equivalence
\newcommand{\smayHA}{\mathrel{\sim^{\rm{HAmb}}_{\rm ctx}}}       %may equivalence

\newcommand{\holE}{\contexthole}  % hole

\newcommand{\murule}{\fortherules\mu} % mu rule of $\lambda$-calculus
\newcommand{\nurule}{\fortherules\nu} % nu rule of $\lambda$-calculus
\newcommand{\nuvrule}{\fortherules\nuv} % nuv rule of $\lambda$-calculus
\newcommand{\xirule}{\fortherules\xi} % xi rule of $\lambda$-calculus
\newcommand{\betarule}{\fortherules\beta} %beta rule of $\lambda$-calculus
\newcommand{\betavrule}{\fortherules\betav} % beta_v rule of $\lambda$-calculus
\newcommand{\etarule}{\fortherules\eta} % eta rule of $\lambda$-calculus
\newcommand{\nuv}{\nu_{\myrm v}} %beta rule of $\lambda$-calculus
\newcommand{\betav}{\mbox{$\beta_{\myrm{v}}$}} %beta rule of $\lambda$-calculus
\newcommand{\alpharule}{\fortherules\alpha} %alpha rule 
\newcommand{\fortherules}[1]{\mbox{$#1$}} %auxiliary def for the rules



\newcommand{\barbedbis}
{\mathrel{\stackrel{\bfcdotB}{\approx}}}

%OLD:
%{\mbox{ $\approx \! \! \! \!\! \!\!        
%\raisebox{1.15ex}[0ex][0ex]{\bfcdot} \; \,$}}



\newcommand{\wbb}{\mathrel{\approx_{\rm{bar}}}}
\newcommand{\wbc}{\mathrel{\approx^{\rm{c}}_{\rm{bar}}}}
% \newcommand{\wbb}{\mathrel{\barbedbis}}
% \newcommand{\wbc}{\cong}


\newcommand{\bfcdotB}{ {\mbox{\boldmath $.$}}  }         

\newcommand{\bcontra}
{\mathrel{\mcontr_{\rm{bar}}}}
%{\mathrel{\stackrel{\bfcdotB}{\mcontr}}}

\newcommand{\cbc}
{\mathrel{\mcontr^{\rm{c}}_{\rm{bar}}}}
%{\mathrel{{\mcontr_{\rm{bc}}}}}


\newcommand{\mypt}{2pt} 

% --------------



\newenvironment{myquote}
               {\list{}{\rightmargin\leftmargin}%
                \item\relax}
               {\endlist}


% \newenvironment{proofEx}{
% \begin{myquote}
% %\trivlist\parindent=0pt
% %      \item[\hskip \labelsep{\bf Answer: }]}
% \noindent %{\bf Answer to}
% }
% {\qed%\endtrivlist
% \end{myquote}}

%SPECIAL SYMBOLS

\newcommand{\EXX}[1]{{\bf Exercise~\ref{#1}}} % for answers to exercises 
\newcommand{\EXXpa}[2]{{\bf Exercise~\ref{#1}(#2)}} % for answers to exercises 


\newcommand{\Mybar}{\hrulefill} % separation rule

% \def\finish#1{\vskip.2cm\noindent{\em #1}%
%   \marginpar{$\longleftarrow$}\vskip.2cm}

\newcommand{\spaceD}{\,}


\newcommand{\bulletD}{\diamond}
%\mathrel{\lozenge} %\bowtie %blacktriangle %\minuso %\blacktriangleup %filleddiamond %\blackdiamond}

\newcommand{\ccc}{\rmtt} % {\rm\tt{#1}}}  %{\mbox{{\tt #1}}}
\newcommand{\cccTT}{\tt} % {\rm\tt{#1}}}  %{\mbox{{\tt #1}}}

\newcommand{\rr}{\RR}  
\newcommand{\Id}{{\cal I}} % identity relation  


\newcommand{\Prop}{{\cal P}} %
\newcommand{\FF}{F} % a function
\newcommand{\FFbis}{F_{\sim}} 
\newcommand{\FFbisW}{F_{\approx}} 
\newcommand{\finLISTS}{\mbox{{\tt FinLists}$_{A}$}} %   
\newcommand{\fininfLISTS}{\mbox{{\tt FinInfLists}$_{A}$}} %   
\newcommand{\finLISTSi}[1]{\mbox{\tt FinLists}_{#1}} %   
\newcommand{\fininfLISTSi}[1]{\mbox{\tt FinInfLists}_{#1}} %   
\newcommand{\nilLISTS}{\mbox{\tt nil}} %   
\newcommand{\mapLISTS}[2]{\mbox{\tt map}\: #1\: #2 } %   
\newcommand{\qmapLISTS}{\mbox{\tt map}} %   
\newcommand{\iterate}[2]{\mbox{\tt iterate}\: #1\: #2 } %   
\newcommand{\qiterate}{\mbox{\tt iterate}} %   
\newcommand{\qnats}{\mbox{\tt nats}} %   
\newcommand{\qfrom}{\mbox{\tt from}\: } %   
\newcommand{\fibs}{\mbox{\tt fibs}} %   
\newcommand{\qplus}{\mbox{\tt plus}\:} %   
\newcommand{\qtail}{\mbox{\tt tail}\,} %   
\newcommand{\plusU}{+_1} %   
\newcommand{\FFlist}{\Phi_{A\tt list}} 

\newcommand{\consLISTS}{\mbox{\tt cons}} %   
\newcommand{\consLISTSnew}[2]{\langle #1\rangle \bullet #2} %   
\newcommand{\consLISTSnewB}[2]{ #1 \bullet #2} %   


\newcommand{\simList}{\sim}  %_{A\tt list}}} % bisimilarity on lists 


\newcommand{\tree}[1]{\mbox{{\tt Tree}$(#1)$}} %   
%\newcommand{\root}[1]{\mbox{{\tt root}$(#1)$}} %   
\newcommand{\T}{{\cal T}}
\newcommand{\Vp}{{\tt V}}
\newcommand{\Rp}{{\tt R}}
\newcommand{\Gin}[2]{{\cal G}^{\tt ind}(#1,#2)} %   
\newcommand{\Gco}[2]{{\cal G}^{\tt coind}(#1,#2)} %   



\newcommand{\pws}[1]{\wp (#1)} % powerset
\newcommand{\lfp}[1]{\qqlfp(#1)} % least fixed point 
\newcommand{\gfp}[1]{\qqgfp(#1)} % greatest fixed point
\newcommand{\qqlfp}{{\tt lfp}} % least fixed point abbrv.
\newcommand{\qqgfp}{{\tt gfp}} % greatest fixed point abbrv.
\newcommand{\qlfp}{least fixed point}
\newcommand{\qgfp}{greatest fixed point} 


\newcommand{\Fcoin}{F_{\tt coind}} % coind. def. set
\newcommand{\Fin}{F_{\tt ind}}     % ind. def. set

\newcommand{\Lao}{\Lambda^0}  %closed $\lambda$-terms



  %convergence

\newcommand{\Dwa}{\Downarrow}           % plain convergence
\newcommand{\DwaP}[2]{#1 \Downarrow #2} % plain convergence, in rules


\newcommand{\EQsin}[2]{#1 = #2} % syn. equality in rules

\newcommand{\dwa}{\downarrow}           % plain convergence
\newcommand{\Up}{\Uparrow}   % divergence
\newcommand{\UpP}{\Uparrow}   % divergence in rules

\newcommand{\Reach}{\Downarrow}           % reachability
\newcommand{\Termi}{\downharpoonright}        % can terminate
\newcommand{\TermiP}{\Termi}        % can terminate, in rules
\newcommand{\Adiv}{\upharpoonright_\mu}           % \mu-divergence
\newcommand{\Adiva}{\upharpoonright_a}           % \mu-divergence

\newcommand{\LL}{{\cal L}}  % set of terms
\newcommand{\States}[1]{{\tt St}^{#1}} %states of an LTS


\newcommand{\myemptyItem}{\mbox{$ $ }} % utile per il xy package
\newcommand{\NONemptyItem}[1]{\mbox{$#1$ }} % utile per il xy package

\newcommand{\LongrightarrowN}[1]{\Longrightarrow_{#1}}


% for imperative programs
\newcommand{\XX}{\spaceD{\ccc{X}}}
\newcommand{\YY}{\spaceD{\ccc{Y}}}

% for LTSs
\newcommand{\Act}{\mbox{\it Act}} 
\newcommand{\pr}{\mbox{\it Pr}} % \mbox{\it Pr}}  %{{\mathbb P}} %{{\cal P}r} 
\newcommand{\power}{\wp} 
\renewcommand{\Pr}{\pr} % \mbox{\it Pr}}  %{{\mathbb P}} %{{\cal P}r} 


% membership stuff among relations
\newcommand{\memb}[3]{ #1 #3 #2 } 
\newcommand{\rmemb}[3]{ {(#1 , #3)} \in { #2} } 
%\newcommand{\Rmemb}[3]{ \tobr{#1 , #3} \in  #2 } 

% symbols
%\newcommand{\vv}{P} 
%\newcommand{\ww}{Q} 
\newcommand{\pp}{P} 
\newcommand{\qq}{Q} 

\newcommand{\rmm}[1]{\mbox{\rm #1}} %labels of transitions in figure 


%% for inference rules


\newcommand{\infrule}[3]{\[
{\trans{#1}\quadrule \displaystyle{#2 \over #3} } %\\[10pt]
\]}
\newcommand{\infruleSIDE}[4]{\[
{\trans{#1}\quadrule\displaystyle{#2 \over #3}\;\; #4 } %\\[10pt]
\]}  % inf rule with a side condition
\newcommand{\shortinfrule}[3]{ {\trans{#1}} \quadrule
     \displaystyle{#2 \over #3}}
\newcommand{\shortinfruleSIDE}[4]{ {\trans{#1}} \quadrule
     \displaystyle{#2 \over #3}\;\; #4}

\newcommand{\shortaxiom}[2]{{\trans{#1}}\quadrule
\displaystyle{ \over #2}}

\newcommand{\myinf}[3]{{\rn{#1}}\quadrule \displaystyle{#2 \over #3} }
    % for  plain  inference rules

\def\trans#1{\rn{#1}}   % for the names of transition rules
\newcommand{\rn}[1]{%
  \ifmmode 
    \mathchoice
      {\mbox{\sc #1}}
      {\mbox{\sc #1}}
      {\mbox{\small\sc #1}}
      {\mbox{\tiny\uppercase{#1}}}%
  \else
    {\sc #1}%
  \fi}

\newcommand{\quadrule}{\hskip .2cm }

\newcommand{\andalso}{\quad\quad}


% references

\def\reff#1{(\ref{#1})}       %references between brackets



\newcommand{\enco}[1]{[\! [ #1 ] \! ]  }


\newcommand{\mydots}{,\ldots,}

% for not\sim_n

\newcommand{\notsimN}[1]{\mathrel{\not\!{\sim_{#1}}} }



\newcommand{\cti}[2]{ C_{#1} \brac{#2} }   %filled context
\newcommand{\ctD}[1]{ D \brac{#1} }   %filled context
\newcommand{\qcti}[1]{ C_{#1}  }   % context

\newcommand{\ctDp}[1]{ D' \brac{#1} }   %filled context
\newcommand{\qctD}{D}   % context
\newcommand{\qctDp}{D'}   % context

%\newcommand{\qctp}{C'}   % context


\newcommand{\beginlongtable}{
 \begin{longtable}{l@{\extracolsep{\fill}}p{76mm}@{\extracolsep{\fill}}r}
%{|l@{\extracolsep{\fill}}p{80mm}@{\extracolsep{\fill}}r|}
%% READ THIS !!!
%% ho cancellato sotto altrimenti mi fa una entry nella list of tables
%\caption{ffff} \\
%\hline
%symbol          & description                & page \\
%\hline
}
\newcommand{\ENDlongtable}{% \hline
\end{longtable}}

\newcommand{\GLSbeg}[1]{\noindent %\underline
{\bf \large #1}}

\newcommand{\GLS}[1]{
\multicolumn{3}{l}{%\noindent 
%\underline
{\bf \large #1}}\\ }
\newcommand{\GLSb}[1]{
\multicolumn{3}{l}{%\noindent 
%\underline
{\it \large #1}}\\ }

\newcommand{\beginlongtableIN}{\\}
\newcommand{\ENDlongtableIN}{\\}

% % to add or remove a line to a page use these 
% \newcommand{\longpage}{\enlargethispage{\baselineskip}} 
% \newcommand{\shortpage}{\enlargethispage{-\baselineskip}} 





\usepackage{DSarrow} % this is something to use extensible arrows in transitions

% \usepackage{pgf,tikz}
% \usetikzlibrary{arrows}

\title{The coarsest precongruence contained in bisimilarity 
contraction and its unique solution theorem}
\author{Davide Sangiorgi
\institute{Universit\`a di Bologna and INRIA\\Bologna, Italy}
\email{davide.sangiorgi@unibo.it}
\and Chun Tian
\institute{Fondazione Bruno Kessler\thanks{Part of this work was
    carried out when the author was studying in University of
    Bologna.}\\Trento, Italy}
\email{ctian@fbk.eu}
}
\def\titlerunning{Coarsest precongruence contained in bisimilarity contraction}
\def\authorrunning{D. Sangiorgi \& C. Tian}

\begin{document}
\maketitle

\begin{abstract}
Milner's ``unique solution of equations'' theorem for (rooted) weak
bisimilarity have severe syntactical limitations, i.e.\;the
equations must be both strongly guarded and sequential. By replacing
equations with special inequations called ``contraction'', Sangiorgi
has proven the ``unique solution of contractions'' theorem for weak
bisimialarity, which requires only weakly guarded equations. However,
contraction is not (pre)congruence under direct sums of processes.
To overcome this difficulty, we further moved to ``rooted
contraction'', which is the coarsest precongruence contained in
contraction bisimilarity. Using rooted contraction one obtains an
unique-solution theorem that is valid for 
\emph{rooted bisimilarity} (hence also for bisimilarity itself) while
requires true weak guardness (with direct sums). All results were
formalized in HOL theorem prover (HOL4), as part of a rather complete
formalization of process algebra CCS. We show some highlights of this
formalization work.
\end{abstract}

% part 1 (Sangiorgi)
\section{Introduction}

A prominent proof method for bisimulation, put forward by Robin Milner and widely used in his
landmark CCS book \cite{Mil89} is the
\emph{unique solution of equations}, whereby two tuples of processes are
componentwise bisimilar if they are solutions 
of the same system of equations.
This method is important in verification techniques and tools
based on algebraic reasoning \cite{BaeBOOK,theoryAndPractice,RosUnder10}. 

In the versions of Milner's unique solution theorems for proving that all
solutions are weakly (or rooted) bisimilar (in practice these are the most
relevant cases), however,
Milner's proof method has severe syntactical limitations, such that
the equations must be ``guarded and sequential,'' that is, the
variables of the equations may only be used underneath a visible
prefix and preceed, in the syntax tree, only by the sum and prefix operators.
One way of overcoming such limitations is to replace equations
 with special inequations called
\emph{contractions} \cite{sangiorgi2015equations,sangiorgi2017equations}. Contraction is a
preorder that, roughly, places some efficiency
constraints on processes.  The uniqueness of solutions of a system of contractions
is defined as with systems of equations: any two solutions must be bisimilar.
The difference with equations is in the meaning of a solution:
in the case of contractions the solution is evaluated with respect to
the contraction preorder, rather than bisimilarity. 
With contractions, most syntactic limitations of the unique-solution theorem can be
removed.  One constraint that still remains in
\cite{sangiorgi2017equations} (in which the issue is bypassed using a more
restrictive CCS syntax)
is the occurrences of direct sums, due to the failure of the
substitutivity of contraction under direct sums.

The main goal of the work described in this paper is a rather
comprehensive formalisation of the core of the theory of CCS in the HOL
theorem prover (HOL4),  with a focus on the theory of unique solutions of contractions.
The formalisation, however, is not confined to the theory of unique
solutions of equations, but embraces a significant portion the theory of CCS \cite{Mil89}
(mostly because the theory of unique solutions relies on a large number of more fundamental results).
Indeed the formalisation encompasses the basic properties of strong and weak
bisimilarity (e.g. the fixed-point and substitutivity properties), the
basic properties of
rooted bisimilarity (the congruence induced by weak
bisimilarity, also called observation congruence), and
their algebraic laws. Further extensions (beyond Nesi
\cite{Nesi:1992ve}) include four versions of ``bisimulation up to''
techniques (e.g., bisimulation up-to bisimilarity) \cite{Mil89,sangiorgi1992problem}, and the
expansion and contraction preorder (two
efficiency-like refinements of weak bisimilarity). Concerning rooted bisimilarity, the formalisation
includes Hennessy Lemma and Deng Lemma (Lemma 4.1 and 4.2 of
\cite{Gorrieri:2015jt}),
 and two long proofs saying the rooted bisimilarity is the coarsest (largest)
 congruence contained in (weak) bisimilarity: one following Milner's
 book \cite{Mil89}, with the hypothesis that no processes can use up
 all labels;
the other without such hypothesis, essentially formalising van Glabbeek's paper \cite{van2005characterisation}.
Similar theorems are proved for the rooted contraction preorder.
In this respect, the work is an extensive experiment with the use of the HOL theorem prover and its
most recent developments, including a package for expressing coinductive definitions.

From the view of CCS theory, this formalisation has offered us the possibility of
further refining the theory of unique solutions of
equations, as formally proving previous previously known results gives us a
chance to see \emph{what's really needed} for establishing that results.
In particular, the existing theory has placed limitations on the body of the contractions due to the
substitutivity problems of weak bisimilarity and other behavioural relations with respect
to the sum operator.
We have thus refined the contraction-based proof technique, by moving to  
\emph{rooted contraction}, that is, the coarsest precongruence contained in the contraction
preorder. The resulting unique-solution theorem is now valid for
\emph{rooted bisimilarity} (hence also for bisimilarity itself), and places no 
constraints on the occurrences of sums.

% \finish{ I had to remove the reference \cite{Tian:2017wrba} here
% since it would further 
%   weaken this paper. } 

Another advantage of the formalisation is 
that we can take advantage of results about different 
equivalences or preorders that share a similar  proof structure. 
Examples are: the results that rooted bisimilarity and rooted contraction are,
respectively, the coarsest congruence contained in weak bisimilarity 
and the coarsest precongruence contained in the contraction  preorder; 
the result about unique solution of equations for weak bisimilarity that uses the
contraction preorder as an auxiliary relation, and other unique solution results (e.g., 
the one for rooted in which
the auxiliary relation is rooted contraction); various forms of enhancements of the bisimulation
proof method (the `up-to' techniques).
In these cases,  moving between proofs there are only a few places in
which the HOL scripts have to be modified.
Then the successful termination of the scripts  gives us a guarantee that the proof is
completed,  removing the risk 
of overlooking or missing details as in hand-written proofs.

% to describe
% The purpose of this paper is twofold. 
% On the one hand, 
% On the other hand, we provide a  
%  comprehensive formalisation  of the core of the theory of CCS 
%  in the HOL
% theorem prover (HOL4). The formalisation  includes the proofs of
% Milner's 3 ``unique solution of equations'' theorems and
% contractions discussed in the present paper, but is not limited to it (partly because such
% theorems rely on a number of more fundamental results):
% indeed the formalisation encompasses the basic properties of strong and weak
% bisimilarity (e.g. the fixed-point and substitutivity properties), 
% their algebraic theory, various versions of ``bisimulation up to''
% techniques (e.g., bisimulation up-to expansion),
% the basic properties  of rooted bisimilarity. 
% Thus the work is an extensive experiment with the use of the HOL theorem provers and its
% most recent developments, including a facility  for expressing coinductive definitions.

% % Considering the relationship between bisimilarity and rooted
% % bisimilarity, the formalisation includes the proof that the latter is the coarsest
% % congruence included in the former, for which two proofs are formalized: one as in Milner's
% % book,  requiring the hypothesis that  no processes can use all labels; the other without
% % such hypothesis, essentially formalising van Glabbek's paper \cite{van2005characterisation}.
% % Similar theorems are proved for rooted contractions wrt the contraction preorder.

\paragraph{Structure of the paper} 
Section~\ref{ss:ccs} presents basic background materials on CCS,
including its syntax, operational semantics, bisimilarity and rooted
bisimilarity. Section~\ref{s:eq} discussed equations and contractions.
 Section~\ref{ss:new} presents rooted contraction and the related
 unique-solution result for rooted bisimilarity. Section~\ref{s:for}
 \hl{highlights} our formalisation in HOL4. Finally,  Section~\ref{s:rel} and
 \ref{s:concl} discuss related work, conclusions,  and  a few
 directions for future work.
% \section{Background}
% \label{s:back}

\section{CCS}
\label{ss:ccs}

We assume a possibly infinite set of names $\mathscr{L} = \{a, b,
\ldots\}$ forming input and $\overline{\mbox{output}}$ actions, plus a special invisible
action $\tau \notin \mathscr{L}$, and a set of variables $A, B,
\ldots$ for defining recursive behaviours.
Given a deadlock $\nil$, the class of CCS processes is inductively defined from $\nil$ by the operators
of prefixing, parallel composition, summation (binary choice), restriction, recursion and relabeling:
\begin{equation*}
\begin{array}{cccl}
\mu  & := & \tau\!\!\!\! & \midd \; a  \; \midd \;  \outC a  \\
P  & := & \nil\!\!\!\! & \midd \;  \mu . P \; \midd \;  P_1 |  P_2 \; \midd  \;
P_1 + P_2 \; \midd % \; \mu . P\; \midd  \; 
  (\res a\!)\, P  \;  \midd \;  A \; \midd \; \recu A  P
\; \midd \; P\; [r\!f]  % relabelling
\end{array}
\end{equation*}
\hl{We sometimes omit the trailing $\nil$, e.g., writing $a|b$ for $a.\nil |b .\nil$.}
The operational semantics of CCS is then given by means of
a Labeled Transition System (LTS), shown in Fig.~\ref{f:LTSCCS} as
\hl{Structural Operational Semantics (SOS)}
rules (the symmetric version of the two rules for
parallel composition and the rule for sum are omitted).
A CCS expression uses only \emph{weakly\hl{ }guarded sums} if all occurrences of
the sum operator are of the form $\mu_1.P_1 + \mu_2.P_2 + \ldots
+ \mu_n.P_n$, for some $n \geq 2$.
 The \emph{immediate derivatives} of a
process $P$ are the elements of the set $\{P' \st P \arr\mu P' \mbox{
  for some $\mu$}\}$.
 \hl{We use $\ell$ to range over
  visible actions (i.e.~inputs or outputs, excluding  $\tau$).}
\begin{figure*}[t]
\begin{center}
\vskip .1cm
 $\displaystyle{  \over  \mu.  P    \arr\mu
P } $  $ \hb$   
\hskip .5cm
 $\displaystyle{   P \arr\mu   P' \over   P + Q   \arr\mu
P'  } $  $ \hb$   
\hskip .5cm
 $\displaystyle{   P \arr\mu   P' \over   P | Q   \arr\mu
P' | Q } $  $ \hb$   
\hskip .3cm
  $\; \;$  $\displaystyle{ P \arr{ a}P' \hk \hk  Q
\arr{\outC a }Q'  \over     P|  Q \arr{ \tau} P'
|  Q'  }$ 
\\
\vspace{.2cm}
$\displaystyle{ P \arr{\mu}P' \over
 (\res a\!)\, P   \arr{\mu} (\res a\!)\, P'} $ $ \mu \neq a, \outC a$
$ \hb$
%
$\displaystyle{ P \sub {\recu A P} A \arr{\mu}P' \over
 \recu A P   \arr{ \mu} P'  } $
\hskip .5cm  
$\displaystyle{ P \arr{\mu} P' \over
 P \;[r\!f] \arr{r\!f(\mu)} P' \;[r\!f]} $ $\forall a.\, r\!f(\outC a) = \overline{r\!f(a)}$
$ \hb$ %  &
\end{center}
\caption{Structural Operational Semantics of CCS}
\label{f:LTSCCS}
\end{figure*}
Some standard notations for transitions: $\Arr\epsilon$ is the 
reflexive and transitive closure of $\arr\tau$, and 
$\Arr \mu $ is $\Arr\epsilon \arr\mu \Arr\epsilon$ (the
composition of the three relations).
Moreover,   
$ P \arcap \mu P'$ holds if $P \arr\mu P'$ or ($\mu =\tau$ and
$P=P'$); similarly 
$ P \Arcap \mu P'$ holds if $P \Arr\mu P'$ or ($\mu =\tau$ and
$P=P'$).
We write $P \:(\arr\mu)^n P'$ if $P$ can become $P'$ after performing
$n$ $\mu$-transitions. Finally, $P \arr\mu$ holds if there is $P'$
with $P \arr\mu P'$, and similarly for other forms of transitions.

\paragraph{Further notations}
\hl{Let variables} $\R$, $\S$ range over binary relations%
\footnote{We use infix notation for relations, e.g., 
$P \RR Q$ means that $(P,Q) \in \R$.} of CCS processes.
We use tilded letters \hl{like $\til P$} to denote tuples \hl{of processes} with
countably many elements; (Thus the tuple may also be infinite.)
All notations are extended to tuples componentwise,
e.g., $\til P \RR \til Q$ means that $P_i \RR Q_i$, for each  
component $i$ of the tuples $\til P$ and $\til Q$.
And $\ct{\til P}$ is the process obtained by replacing each hole
$\holei i$ of the  context $\qct$ with $P_i$.  
We write $
\ctx \R$ for the closure of a relation under contexts. Thus $P\: \ctx \R\: Q$
means that there are context $\qct$ and tuples $\til P,\til Q$ with
$P =  \ct{\til P}, Q =  \ct{\til Q}$ and $\til P \RR \til Q$.
We use the symbol $\DSdefi$ for abbreviations. For instance, $P \DSdefi G $, where
$G$ is some expression, means that $P$ stands for the expression $G$.
If $\leq$ is a preorder, then  $\geq$  is its inverse (and
conversely).

% next file: expansion.tex

\subsection{Bisimilarity and rooted bisimilarity}
\label{ss:BiEx}

The equivalences we consider here are mainly \emph{weak} ones, in that they
abstract from the number of internal steps being performed:
\begin{definition}%[bisimilarity]
\label{d:wb}
A process relation ${\R}$ is a \emph{bisimulation} if, whenever
 $P\RR Q$, for all $\mu$ we have:
\begin{enumerate}
\item $P \arr\mu P'$ implies that there is $Q'$ such that $Q \Arcap \mu Q'$ and $P' \RR Q'$;
\item $Q \arr\mu Q'$,implies that there is $P'$ such that $P \Arcap
  \mu P'$ and $P' \RR Q'$\enspace.
\end{enumerate}
 $P$ and $Q$ are \emph{bisimilar},
written as $P \wb Q$, if $P \RR Q$ for some bisimulation $\R$.
\end{definition}

\hl{We sometimes call the bisimulation and bisimilarity defined above
the \emph{weak} bisimulation (and weak bisimilarity), to
distinguish it from the \emph{strong} bisimulation and strong bisimilarity} ($\sim$),
obtained by replacing the weak transition $Q\Arcap\mu Q'$ in the above
definition with the strong transition $Q \arr \mu Q'$ (same for the
other case).
Weak bisimilarity is not preserved by the sum operator (except for
guarded sums). For this, Milner introduced \emph{observational congruence}, also called \emph{rooted
  bisimilarity} \cite{Gorrieri:2015jt,Sangiorgi:2011ut}:
\begin{definition}%[rooted bisimilarity]
\label{d:rootedBisimilarity}
Two processes $P$ and $Q$ are \emph{rooted bisimilar}, written as $P
\rapprox Q$, if
for all $\mu$:
%  for all $\mu\in \mathscr{L}\cup\{\tau\}$
\begin{enumerate}
 \item  $P \arr\mu P'$ implies that there is $Q'$ such that $Q
   \Arr\mu Q'$ and $P' \wb Q'$;
 \item  $Q \arr\mu Q'$ implies that there is $P'$ such that $P
   \Arr\mu P'$ and $P' \wb Q'$\enspace.
\end{enumerate}
\end{definition}

Besides reducing the rooted \hl{bisimilarity} of two processes to
the bisimilarities of their immediate derivatives, the
above definition also brings up a proof technique for proving rooted
\hl{bisimilarity} results by constructing a bisimulation.
\begin{lemma}[Rooted bisimilarity by bisimulation]
\label{l:obsCongrByWeakBisim}
Given a (weak) bisimulation $\RR$, if two processes $P$ and $Q$
satisfies the following properties:
\begin{enumerate}
\item $P \arr\mu P'$ implies that there is $Q'$ such that $Q
   \Arr\mu Q'$ and $P' \RR Q'$;
\item $Q \arr\mu Q'$ \hl{implies that there is} $P'$ such that $P
   \Arr\mu P'$ and $P' \RR Q'$.
\end{enumerate}
\hl{Then} $P$ and $Q$ are rooted bisimilar, i.e.~$P \approx^c Q$.
\end{lemma}

\begin{theorem}
\label{t:rapproxCongruence}
$\rapprox$ is a congruence in CCS, and it is the
coarsest congruence contained in $\approx$ \cite{van2005characterisation}.
\end{theorem}

\section{Equations and contractions}
\label{s:eq}



\subsection{Systems of  equations}
\label{ss:SysEq}

             
Uniqueness of  solutions of equations \cite{Mil89} intuitively says that if  a context $\qct$ obeys
certain  conditions, 
then all processes $P$  that satisfy the equation $ P \wb \ct P$ are
bisimilar with each other.
We need variables to write equations. We  use
 capital
letters  $X,Y,Z$
 for  these variables and call them \emph{\behav\  variables}.
 The body of an equation is a CCS expression
possibly containing \behav\  variables. Thus such expressions, ranged
over by $E$, live in the CCS
grammar extended with \behav\  variables.
% , which we  call 
%  \emph{extended CCS}. 

  
\begin{definition}
Assume that, for each $i$ of 
 a countable indexing set $I$, we have variables $X_i$, and expressions
$E_i$ possibly containing  such variables. 
Then 
$$\{  X_i = E_i\}_{i\in I}$$
is 
  a \emph{system of equations}. (There is one equation for each variable $X_i$.)
\end{definition}

We write $E[\til P]$ for the expression resulting from $E$ by
replacing each variable $X_i$   with the process $P_i$, assuming
$\til P$ and $\til X$ have the same length. (This is syntactic
replacement.) 
% The components of $\til P$ need not be
%  different from each other, as it must be for the variables $\til X$.
% If the system has infinitely many equations,
% the  tuples $\til P$ and $\til X$
%  are infinite too.
\begin{definition}
Suppose  $\{  X_i = E_i\}_{i\in I}$ is a system of equations: 
\begin{itemize}
\item
 $\til P$ is a \emph{solution of the 
system of equations  for $\wb$} 
if for each $i$ it holds
that $P_i \wb E_i [\til P]$;

\item it %the  system
 has \emph{a unique solution for $\wb$}  if whenever 
 $\til P$ and $\til Q$ are both solutions for $\wb$, then $\til P \wb
 \til Q$. 
\end{itemize} 
 \end{definition} 




% Examples of systems with a  unique solution for $\wb$ are: 
% \begin{enumerate}
% \item
% $ X = a. X$ 

% \item 
% $ X_1 = a.  X_2$, $ X_2 = b.  X_1$  

% \end{enumerate}

For instance, the solution of the equation 
$ X = a. X$ 
is  the process
$R \DSdefi \recu A { a. A}$, and   for any other solution $P$ we have $P \wb R$.
In contrast, the equation 
 $X = a|  X$ has solutions that may be quite different, namely any process capable of
continuously  performing $a$ action (but that could behave arbitrarily  on other actions). 

 
% The unique solution of the system (1), modulo $\wb$,  is the constant $K \Defi a
% . K$:  for any other solution $P$ we have $P \wb K$.
% The unique solution of (2), modulo $\wb$, are the constants $K_1 , K_2$
% with $K_1 \Defi a . K_2$ and $K_2 \Defi b. K_1$; again, for any other
% pair of solutions $P_1,P_2$ we have $K_1 \wb P_1$ and $K_2 \wb P_2$.
% Examples of systems that do not have a unique solution are: 
% \begin{enumerate}
% \item $X = X $ 

% \item $X = \tau . X$
% \item $X = a | X$

% \end{enumerate} 
% All processes are solutions of (1) and (2); examples of solutions for
% (3) are $K$ and $K | b$, for $K \Defi a
% . K$.

\begin{definition}
A system of equations 
$\{  X_i = E_i\}_{i\in I}$
 is 
\begin{itemize}
\item
\emph{(strongly) guarded} if,  in each    $E_i$, each occurrence of
an \behav\  variable is underneath a visible prefix;

\item 
 \emph{sequential} if,  in each    $E_i$, each occurrence of
an \behav\  variable only appears  underneath prefixes and sums.
\end{itemize}
 \end{definition}

% In other words,  
% if the system is sequential, then 
%  for
% every expression $E_i$, any subexpression of $E_i$ in which $X_j $ 
% appears, apart from $X_j$ itself,  is a sum (of prefixed terms). 
% For instance, 
% \begin{itemize}
% \item $X = \tau. X + \mu . \nil$ is sequential but not 
%  guarded, because the guarding prefix for the variable
% is not visible.

% \item $X =  \ell . X | P$ is  guarded but not sequential.

% \item $X =  \ell . X + \tau. \res a (a .\outC b | a.\nil)$, as well as 
% $X = \tau . (a. X + \tau . b .X + \tau  )$
% are both 
%  guarded and sequential.
% \end{itemize} 


\begin{theorem}[unique solution of equations, \cite{Mil89}]
\label{t:Mil89}
A system of guarded and sequential equations
% $\{  X_i = E_i\}_{i\in I}$ 
   has 
a unique solution
 for $\wb$.
\end{theorem} 

% The proof exploits an invariance property on immediate transitions for
% guarded and sequential expressions, and then extracts a bisimulation
% (up-to bisimilarity) out
% of the solutions of the system.  
To see the  need of the
 sequentiality  condition, 
  consider
 the equation (from \cite{Mil89}) $X = \res a (a. X | \outC a)$
where $X$ is guarded but not sequential. Any processes that does
not use  $a$
 is a solution.

\subsection{Expansions and Contractions}
\label{s:mcontr}

Milner's ``unique solution of equations'' theorem for $\wb$
(Theorem~\ref{t:Mil89})
brings a new proof technique for proving (weak) bisimilarities. However, it has
 limitations: the equations must be guarded and sequential. (\hl{Moreover,}
all sums where equation variables appear must be guarded sums.)
This limits the usefulness of the technique, since
the occurrences of other operators \hl{using equation} variables, such as parallel
composition and restriction,
in general cannot be \hl{eliminated}. % without changing the meaning of equations
The constraints \hl{in} Theorem~\ref{t:Mil89}, however, can be
weakened if we move from equations to a special kind of inequations called
  \emph{contractions}.

Intuitively, the bisimilarity contraction $\mcontrBIS$ is a preorder
\hlD{in which} $P \mcontrBIS Q$ holds if $P \wb Q$ and, in addition, 
\emph{$Q$ has the possibility of being at least as efficient as $P$} (as far as
$\tau$-actions are performed).
The process $Q$, however, may be nondeterministic and may have other ways
\hl{to do} the same work, \hl{ways} which could be \hl{slower} (i.e., involving
more $\tau$-actions than those performed by $P$).
% Thus, in contrast with expansion,  we cannot really say that `$Q$ is more efficient than
% $P$'.

\begin{definition}[contraction]
\label{d:BisCon}
A process relation ${\R}$ 
 is a \emph{(bisimulation) contraction} if, whenever $P\RR Q$,

\begin{enumerate}
\item $P \arr\mu P'$ implies \hl{that} there is $Q'$ \hl{with} $Q \arcap \mu
  Q'$ and $P' \RR Q'$;
\item $Q \arr\mu Q'$ implies \hl{that} there is $P'$ \hl{with} $P \Arcap \mu
 P'$ and $P' \wb Q'$.
\end{enumerate}
Two processes $P$ and $Q$ are in the \emph{bisimilarity
contraction}, written as $P \mcontrBIS Q$,
if $P\ \R\ Q$ for some contraction $\R$.
Sometimes we write $\mexpaBIS$ for the inverse of $\mcontrBIS$.
\end{definition}
In clause (1) \hl{of the above definition}, $Q$ is required to match the challenge
transition \hl{of $P$} with at most one transition.
This makes sure that $Q$ is capable of mimicking % verb: mimic
$P$'s work at least as efficiently as $P$. 
In contrast, \hl{clause (2) entirely ignores efficiency on the challenges from $Q$:}
the final derivatives are required to be related by $\wb$, rather than by $\R$.

Bisimilarity contraction is coarser than bisimilarity expansion
$\expa$~\cite{arun1992efficiency,sangiorgi2015equations}, \hl{one of the
most useful auxiliary relations in up-to techniques}:
\begin{definition}[expansion]
\label{d:expa}
A process relation ${\R}$
  is an \emph{expansion} if, whenever $P\RR Q$,
 \begin{enumerate}
 \item   $P \arr\mu P'$ implies that there is $Q'$ with $Q \arcap \mu  Q'$
  and $P' \RR Q'$;
 \item $Q \arr\mu Q'$ implies that there is $P'$ with $P \Arr \mu P'$ and $P' \RR Q'$.
 \end{enumerate}
Two processes $P$ and $Q$ are in the \emph{bisimilarity
  expansion}, written as $P \expa Q$, if $P \RR Q$ for some expansion $\R$.
 \end{definition}
\hl{Bisimilarity expansion} is widely used in proof techniques for bisimilarity.
\hl{It} intuitively refines bisimilarity by 
formalising the idea of ``efficiency'' between processes.
Clause (1) is the same in the both preorders, while in clause (2) expansion \hl{requires}
$P \Arr \mu P'$, rather than $P \Arcap \mu P'$.
\hl{Moreover,} in clause (2) of Def.~\ref{d:BisCon} the final derivatives
are simply required to be bisimilar ($P' \wb Q'$).
Intuitively, $P \expa Q$ holds if $P\wb Q$ and, in addition, \emph{$Q$
  is always at least as efficient as $P$}.

\begin{example}
\label{exa:contr}
We have %\mcontrBIS a + \tau^n . a $
 $ a \not  \mcontrBIS \tau. a$. However,
$a+ \tau . a \mcontrBIS a$, as well as its converse, 
$  a \mcontrBIS a +
\tau . a $. Indeed, if $P \wb Q$ then 
$  P  \mcontrBIS P +Q$. The last two relations do not hold with 
$\expa$, which explains the strictness of the inclusion
 ${\expa} \subset {\mcontrBIS}$. 
% The inclusion is strict: for instance
% $a+ \tau . a \mcontrBIS a$, where $\mcontrBIS$ cannot be replaced by
%  $\contr$. Also the converse of  $a+ \tau . a \mcontrBIS a$ holds, namely
% $  a \mcontrBIS a +
% \tau . a $. However, we have %\mcontrBIS a + \tau^n . a $
%  $ a \not  \mcontrBIS \tau. a$
\end{example} 

\hlD{Bisimilarity expansion and bisimilarity contraction are both
preorders.}
Similarily with (weak) bisimilarity, both the expansion and the
contraction preorders are preserved by all CCS operators except the
summation. The proofs are similar \hlD{to those for bisimilarity},
see, e.g.~\cite{sangiorgi2017equations} \hl{for details.}

% next file: unique.tex

\subsection{Systems of contractions}
\label{ss:SysContr}

A \emph{system of contractions} is defined as a system of equations,
except that the contraction symbol $\mcontr$ is used in the place of
the equality symbol $=$. Thus a system of contractions is a set 
$\{  X_i \mcontr E_i\}_{i\in I}$
where $I$ is an  indexing set and expressions
$E_i$  may contain the  \behavC\  variables 
$\{  X_i\}_{i\in I}$.

\begin{definition}
\label{d:uniContra}
Given a system of contractions 
$\{  X_i \mcontr E_i\}_{i\in I}$, 
 we say that:
\begin{itemize}
\item $\til P$ is a \emph{solution (for $\mcontrBIS$) of the 
 system of contractions} if $\til P \mcontrBIS \til E [\til P]$;
\item the system has \emph{a unique solution (for $\approx$)}
if $\til P \approx \til Q$ whenever $\til P$ and $\til Q$ are both solutions.
\end{itemize}
\end{definition}

The guardedness of contractions follows Def.~\ref{def:guardness} (for equations).
% \begin{definition}
% \label{d:guarded}
% A system of contractions $\{  X_i \mcontr E_i\}_{i\in I}$
%  is
% \emph{weakly guarded}
% if,  in each    $E_i$, each occurrence of
% a \behavC\ variable is underneath a prefix.

% The system use \emph{weakly-guarded sums} if 
% each $E_i$ only makes use of guarded sums.
% \end{definition}

\begin{lemma}
\label{l:uptocon}
Suppose $\til P$ and $\til Q$ are solutions  for $\mcontrBIS$
 of a system of weakly-guarded contractions that uses 
weakly-guarded sums.
For any context $\qct$  that uses 
weakly-guarded sums,
if  $\ct{\til P}\Arr{\mu}  R$,
 then 
there is a context $\qctp$  that uses 
weakly-guarded sums
such that $R \mcontrBIS \ctp{\til P}$ and $\ct{\til Q} \Arcap{\mu}
 \wb \ctp{\til Q}$.\footnote{There's no typo here: $\ct{\til Q} \Arcap{\mu} \wb \ctp{\til
     Q}$ means $\exists {\til R}.\; \ct{\til Q} \Arcap{\mu} {\til R}
   \wb \ctp{\til Q}$. Same as in Lemma~\ref{l:ruptocon}.}
\end{lemma}

\begin{proof}{(sketch from \cite{sangiorgi2017equations})}
Let $n$ be the length of the transition $\ct{\til P}\Arr\mu R$  (the
number of `strong steps' of which it is composed), and  
let $\ctpp {\til P}$ and $\ctpp {\til Q}$  be the processes obtained
from  $\ct {\til P}$ and $\ct {\til Q}$ by unfolding the definitions
of the contractions $n$ times. Thus in $\qctpp$ each hole is
underneath at least $n$ prefixes, and cannot contribute to an action
in the first $n$ transitions; moreover all the contexts have only
weakly-guarded sums.

We have $\ct{\til P} \mcontrBIS \ctpp{\til P}$, and 
$\ct{\til Q} \mcontrBIS \ctpp{\til Q}$, 
 by the substitutivity  properties of $\mcontrBIS$ (we exploit here
 the syntactic constraints on sums). Moreover,
 since each hole of the  context $\qctpp$ is underneath at least $n$
 prefixes, applying  
the definition
 of $ \mcontrBIS$ on the transition 
 $\ct{\til P}\Arr{\mu}  R$, we infer the existence
 of $\qctp$ such that 
$
\ctpp{\til P}\Arcap{\mu} \ctp{\til P} \mexpaBIS R
$
and 
$
\ctpp{\til Q}\Arcap{\mu}  \ctp{\til Q} 
. $
Finally, again applying the definition of $\mcontrBIS$ on 
$\ct{\til Q} \mcontrBIS \ctpp{\til Q}$, 
we derive 
$
\ct{\til Q}\Arcap{\mu}  \wb \ctp{\til Q} 
.$
\end{proof}

\begin{theorem}[unique solution of contractions for $\wb$]
\label{t:contraBisimulationU}
A system of weakly-guarded contractions
having only weakly-guarded sums, has a unique solution for $\wb$.
\end{theorem}

\begin{proof}{(sketch from \cite{sangiorgi2017equations})}
Suppose $\til P$ and $\til Q$ are two such solutions (for $\wb$) and consider
the relation
\begin{equation*}
\R \DSdefi \{ 
(R,S) \st R \wb \ct{\til P}, S \wb \ct{\til Q} \mbox{ for some context
$\qct$ (having only weakly-guarded sums)} \} \enspace.
\end{equation*}
We show that $\R$ is a bisimulation. Suppose $R\ \R\ S$ vis the context
$C$, and $R \arr{\mu} R'$. We have to find $S'$ with $S \Arcap{\mu}
S'$ and $R'\ \R\ S'$. From $R \wb C[{\til P}]$, we derive $C[{\til P}]
\Arcap{\mu} R'' \wb R'$ for some $R''$. By Lemma~\ref{l:uptocon},
there is $C'$ with $R'' \mcontrBIS C'[{\til P}]$ and $C[{\til Q}]
\Arcap{\mu} \wb C'[{\til Q}]$. Hence, by definition of $\wb$, there is
also $S'$ with $S \Arcap{\mu} S' \wb C'[{\til Q}]$. This closes the
proof, as we have $R' \wb C'[{\til P}]$ and $S' \wb C'[{\til Q}]$.
\end{proof}

\section{Rooted contraction}
\label{ss:new}

The unique solution theorem of Section~\ref{ss:SysContr} requires a
constrained syntax for sums, due to the congruence and precongruence
problems of bisimilarity and contraction with such operator. 
We show here that the constraints can be
removed by moving to the induced congruence and precongruence, the
latter 
called \emph{rooted contraction}.
\begin{definition}[rooted contraction]
\label{d:rcontra}
Two processes $P$ and $Q$ are in \emph{rooted contraction}, written 
 $P\rcontr Q$, if
\begin{enumerate}
\item $P \arr\mu P'$ implies that there is $Q'$ with $Q \arr \mu Q'$
 and $P'\mcontrBIS Q'$;
\item $Q \arr\mu Q'$   implies that there is $P'$ with $P \Arr \mu
 P'$ and $P' \wb Q'$.
\end{enumerate}
\end{definition}

%Above definition adapts the definition of rooted
%bisimilarity on top of that of the  contraction preorder
%$\mcontrBIS$.  %% Reviewer said this sentance is unclear. I too think so.

\hl{The discovery of this definition is with help of HOL theorem
  prover and
the following two principles:} 1) Its definition must not be recursive,
instead it should resemble the definition of rooted bisimilarity
$\approx^c$ in Def.~\ref{d:rootedBisimilarity};
2) It must be built on top of existing \emph{contracts}
relation $\mcontrBIS$, which we believe it's the \emph{right} one
because of its completeness. \hl{Multiple candicates were quickly tested,
finally only above definition is proven to be a precongruence, as the
following theorem states:}

\begin{theorem}
\label{t:rcontrPrecongruence}
$\rcontr$ is a precongruence in CCS, and it is the
coarsest precongruence contained in $\contr$.
\end{theorem}  
The proof of this result is along the lines of the analogous result
for rooted bisimilarity with respect to bisimilarity. 
%Similarly one can define and handle the rooted variant of the
%expansion preorder. (not sure)

For a system of contractions, the meaning of 
``solution for $\rcontr$'' and of 
 \emph{a unique 
solution for $\rapprox$}
is the expected one --- just replace in Definition~\ref{d:uniContra}  the preorder 
$\contr$ with $\rcontr$, and the equivalence 
$\approx$ with $\rapprox$.

For the rooted relations, the analogous of Lemma~\ref{l:uptocon} and of
Theorem~\ref{t:contraBisimulationU} can now be stated without constraints on the sum
operator.  
The schema of the proofs is also the same; the substitutivity
properties of 
$\rcontr$ and $\rapprox$ allow us to avoid issues with the sum
operator. Some care is needed in Theorem~\ref{t:rcontraBisimulationU}
to make sure that the final result is about  
$\rapprox$, rather than (the weaker) $\approx$.

\begin{lemma}
\label{l:ruptocon}
Suppose $\til P$ and $\til Q$ are solutions  for $\rcontr$ 
 of a system of weakly-guarded
contractions.
For any context $\qct$, 
if  $\ct{\til P}\Arr{\mu}  R$,
 then 
there is a  context $\qctp$
such that $R \mcontrBIS \ctp{\til P}$ and  $\ct{\til Q} \Arr{\mu}
 \wb \ctp{\til Q}$.
\end{lemma}

\begin{theorem}[unique solution of contractions for $\rapprox$]
\label{t:rcontraBisimulationU}
A system of weakly-guarded contractions has a unique solution 
 for $\rapprox$.
\end{theorem} 



% part 2 (Tian)
\section{Formalisation}
\label{s:for}
We highlight here a formalisation of CCS
in the HOL theorem
prover (HOL4) \cite{slind2008brief},
including the new concepts and theorems proposed in the first half of
this paper.
%  The main purpose is to convince the readers that, there's no flaw
%  in the informal proofs. 
The whole formalisation can be found 
in \cite{Tian:2017wrba}  or in 
 in the official example folder of HOL4 source
code\footnote{\url{https://github.com/HOL-Theorem-Prover/HOL}}. The
work consists of about 20,000 lines of proof scripts in Standard ML
(and that follows the pioneering work by Nesi \cite{Nesi:1992ve}, in
1992-1995). 

HOL4 is written in Standard ML (ML for abbreviation), which plays three roles here:
\begin{enumerate}
\item It serves as the underlying implementation language for the core HOL engine;
\item it is used to implement tactics (and tacticals) for writing proofs;
\item it is used as the command language of the HOL interactive shell.
\end{enumerate}
HOL4 users can write arbitrary applications which leverage
the theorem proving facility. 
\iflong
The HOL logic language, on the other
 side, is more close to Classic ML, in which the early HOL systems were built.
\DS{removed the sentence because  it did not
  seem important} 
\fi

Higher Order Logic (or ``HOL Logic'') \cite{hollogic} stands for simple-typed $\lambda$-calculus plus Hibert
choice operator, axiom of infinity, and rank-1 polymorphism (type
variables). HOL4 implements the original HOL logics, 
in contrast with 
 other theorem provers of the HOL family (e.g. Isabelle/HOL) that have
made extensions.
%  (they made the formal language more powerful,
% but they also bring the possibilities that the entire logic becomes
% inconsistent). 
Indeed the HOL Logic has considerable simpler logic
foundations than most other theorem provers. %, e.g. Coq. 
As a result,
formal theories in HOL can be easily migrated (sometimes even
automatically) into other theorem provers.

% \finish{below: to be reformulated (by Davide) later} 
% We will not repeat all definitions and theorems of CCS again by their
% formalized versions. Instead, we just focus on several highlights in
% this work, i.e.
% \begin{enumerate}
% \item The use of HOL's new coinductive relation package for defining bisimilarity;
% \item The formalisation of context by $\lambda$-expression and the theory of
%   (pre)congruence for CCS;
% \item The definition and uses of trace in the proof of unique solution of
%   contractions theorem;
% \end{enumerate}\finish{TODO: need updates}

In the
formalisation we consider only single equations/contractions. 
This considerably simplifies the  reading of the HOL scripts.
%  (on 
% the extension  to multiple equations/contractions
% see also  Section~\ref{s:concl}).

% \subsection{Simplifications and Limitations}

% We have only formally verified simplified forms of the unique
% solution theorems introduced in the first part. This is to not only
% save the formalisation efforts, but also to make us focusing on the most
% important part of the proof. The most important simplification is
% that, we have only proved all unique solution theorems with only single-variable equations
% (contractions).

% But such a simplification doesn't hurt the rigorousness of our entire
% work, as once the single-equation version of the theorem is
% proven, the general case is just a routine
% adaptation for paper proofs.
% It's not easy to formalize multi-variable CCS equations with
% introducing a large number of definitions and intermediate
% results. It's easy, however, to directly use the semantics context to
% represent single-variable equations without the need of formalizing
% the concept of ``equation'' at all.

% There're two major limitations in this work. One is due to HOL, we only support Finitary CCS, i.e.
% no infinite sum nor infinite parallel composition.
% The other limitation is, equation variables (or holes in semantics contexts)
% cannot appear inside recursion operator closure. However, these limitations do not
% cause any difference in the proof of main results represented in this paper.

%%%% -*- Mode: LaTeX -*-
%%
%% This is the draft of the 2nd part of EXPRESS/SOS 2018 paper, co-authored by
%% Prof. Davide Sangiorgi and Chun Tian.

\subsection{CCS processes and their transitions}

In our CCS formalisation, the type ``\HOLinline{\ensuremath{\beta} \HOLTyOp{Label}}'' (\texttt{'b} or
$\beta$ is the type variable for actions) accounts for visible actions, divided into input
and output actions, defined by HOL's Datatype package:
\begin{lstlisting}
val _ = Datatype `Label = name 'b | coname 'b`;
\end{lstlisting}
The type ``\HOLinline{\ensuremath{\beta} \HOLTyOp{Action}}'' is the
union of all visible actions, plus invisible action $\tau$ (now based on
HOL's \texttt{option} theory). The cardinality of
``\HOLinline{\ensuremath{\beta} \HOLTyOp{Action}}'' (and therefore of all
CCS types built on top of it)
 depends on the choice (or \emph{type-instantiation}) of type variable $\beta$.

The type ``\HOLinline{\ensuremath{(}\ensuremath{\alpha}, \ensuremath{\beta}\ensuremath{)} \HOLTyOp{CCS}}'', accounting for the CCS
syntax\footnote{The order of type variables $\alpha$ and $\beta$
    is arbitrary. Our choice is aligned with other CCS literals.
$\mathrm{CCS}(h,k)$ is the CCS subcalculus that can use at most $h$ constants
and $k$ actions \cite{gorrieri2017ccs}. Thus, to formalize theorems on
such a calculus, the needed CCS type can be retrieved by instantiating the type
variables $\alpha$ and $\beta$ in ``\HOLinline{\ensuremath{(}\ensuremath{\alpha}, \ensuremath{\beta}\ensuremath{)} \HOLTyOp{CCS}}'' with types
having the corresponding cardinalities $h$ and $k$. Monica Nesi goes
farther, by adding another type variable $\gamma$ for value-passing CCS
\cite{Nesi:2017wo}.}, is then defined inductively:
(\texttt{'a} or $\alpha$ is the type variable for recursion variables,
``\HOLinline{\ensuremath{\beta} \HOLTyOp{Relabeling}}'' is the type of all relabeling functions,
\mbox{\color{blue}{\texttt{`}}} is for backquotes of HOL terms):
\begin{lstlisting}
val _ = Datatype `CCS = nil
		      | var 'a
		      | prefix ('b Action) CCS
		      | sum CCS CCS
		      | par CCS CCS
		      | restr (('b Label) set) CCS
		      | relab CCS ('b Relabeling)
		      | rec 'a CCS`;
\end{lstlisting}

We have added some grammar support,
 using HOL's powerful pretty printer, to represent CCS
processes in more readable forms (c.f. the column ``HOL (abbrev.)''
of Table \ref{tab:ccsoperator}, which summarizes 
the main syntactic notations of CCS). For the restriction
operator, we have chosen to allow a  set of names as a parameter, rather than a
  single name as in the ordinary  CCS syntax; this simplifies 
the manipulation of 
 processes with different orders of
  nested restrictions.
% Also, we do not assume that the uses of \texttt{var} are
%  guarded by \texttt{rec} of the same variable.

%  (Notice the use of
% recursion operator for representing process constants)
\begin{table}[h]
\begin{center}
\begin{tabular}{|c|c|c|c|}
\hline
\hl{\textbf{CCS concept}} & \hl{\textbf{Notation}} & \textbf{HOL term} &
                                                                \textbf{\TeX{} output} \\
\hline
\hl{deadlock} & $\textbf{0}$ & \texttt{nil} & \HOLinline{\HOLConst{\ensuremath{\mathbf{0}}}} \\
prefix & $u.P$ & \texttt{prefix u P} & \HOLinline{\HOLFreeVar{u}\HOLSymConst{\ensuremath{\ldotp}}\HOLFreeVar{P}} \\
sum & $P + Q$ & \texttt{sum P Q} & \HOLinline{\HOLFreeVar{P} \HOLSymConst{\ensuremath{+}} \HOLFreeVar{Q}} \\
parallel & $P \,\mid\, Q$ & \texttt{par P Q} & \HOLinline{\HOLFreeVar{P} \HOLSymConst{\ensuremath{\mid}} \HOLFreeVar{Q}} \\
restriction & $(\nu\;L)\;P$ & \texttt{restr L P} & \HOLinline{\ensuremath{(\nu}\HOLFreeVar{L}\ensuremath{)} \HOLFreeVar{P}}  \\
recursion & $\recu A P$ & \texttt{rec A P} & \HOLinline{\HOLConst{rec} \HOLFreeVar{A} \HOLFreeVar{P}}  \\
relabeling & $P\;[r\!f]$ & \texttt{relab P rf} & \HOLinline{\HOLConst{relab} \HOLFreeVar{P} \HOLFreeVar{rf}}  \\
\hline
\hl{variable} & $A$ & \texttt{var A} & \HOLinline{\HOLConst{var} \HOLFreeVar{A}} \\
invisible action & $\tau$ & \texttt{tau} & \HOLinline{\HOLSymConst{\ensuremath{\tau}}} \\
input action & $a$ & \texttt{label (name a)} & \HOLinline{\HOLConst{In} \HOLFreeVar{a}} \\
output action & $\outC a$ & \texttt{label (coname a)} & \HOLinline{\HOLConst{Out} \HOLFreeVar{a}} \\
\hline
\end{tabular}
\end{center}
%\vspace{-1em}
   \caption{Syntax of CCS operators, constant and actions}
   \label{tab:ccsoperator}
\end{table}

The transition semantics of CCS processes follows Structural
Operational Semantics (SOS) in Fig.~\ref{f:LTSCCS}:
\begin{alltt}
\HOLTokenTurnstile{} \HOLFreeVar{u}\HOLSymConst{\ensuremath{\ldotp}}\HOLFreeVar{E} \HOLTokenTransBegin\HOLFreeVar{u}\HOLTokenTransEnd \HOLFreeVar{E}\hfill\texttt{[PREFIX]}
\HOLTokenTurnstile{} \HOLFreeVar{E} \HOLTokenTransBegin\HOLFreeVar{u}\HOLTokenTransEnd \HOLFreeVar{E\sb{\mathrm{1}}} \HOLSymConst{\HOLTokenImp{}} \HOLFreeVar{E} \HOLSymConst{\ensuremath{+}} \HOLFreeVar{E\sp{\prime}} \HOLTokenTransBegin\HOLFreeVar{u}\HOLTokenTransEnd \HOLFreeVar{E\sb{\mathrm{1}}}\hfill\texttt{[SUM1]}
\HOLTokenTurnstile{} \HOLFreeVar{E} \HOLTokenTransBegin\HOLFreeVar{u}\HOLTokenTransEnd \HOLFreeVar{E\sb{\mathrm{1}}} \HOLSymConst{\HOLTokenImp{}} \HOLFreeVar{E\sp{\prime}} \HOLSymConst{\ensuremath{+}} \HOLFreeVar{E} \HOLTokenTransBegin\HOLFreeVar{u}\HOLTokenTransEnd \HOLFreeVar{E\sb{\mathrm{1}}}\hfill\texttt{[SUM2]}
\HOLTokenTurnstile{} \HOLFreeVar{E} \HOLTokenTransBegin\HOLFreeVar{u}\HOLTokenTransEnd \HOLFreeVar{E\sb{\mathrm{1}}} \HOLSymConst{\HOLTokenImp{}} \HOLFreeVar{E} \HOLSymConst{\ensuremath{\mid}} \HOLFreeVar{E\sp{\prime}} \HOLTokenTransBegin\HOLFreeVar{u}\HOLTokenTransEnd \HOLFreeVar{E\sb{\mathrm{1}}} \HOLSymConst{\ensuremath{\mid}} \HOLFreeVar{E\sp{\prime}}\hfill\texttt{[PAR1]}
\HOLTokenTurnstile{} \HOLFreeVar{E} \HOLTokenTransBegin\HOLFreeVar{u}\HOLTokenTransEnd \HOLFreeVar{E\sb{\mathrm{1}}} \HOLSymConst{\HOLTokenImp{}} \HOLFreeVar{E\sp{\prime}} \HOLSymConst{\ensuremath{\mid}} \HOLFreeVar{E} \HOLTokenTransBegin\HOLFreeVar{u}\HOLTokenTransEnd \HOLFreeVar{E\sp{\prime}} \HOLSymConst{\ensuremath{\mid}} \HOLFreeVar{E\sb{\mathrm{1}}}\hfill\texttt{[PAR2]}
\HOLTokenTurnstile{} \HOLFreeVar{E} \HOLTokenTransBegin\HOLConst{label} \HOLFreeVar{l}\HOLTokenTransEnd \HOLFreeVar{E\sb{\mathrm{1}}} \HOLSymConst{\HOLTokenConj{}} \HOLFreeVar{E\sp{\prime}} \HOLTokenTransBegin\HOLConst{label} \ensuremath{(}\HOLConst{COMPL} \HOLFreeVar{l}\ensuremath{)}\HOLTokenTransEnd \HOLFreeVar{E\sb{\mathrm{2}}} \HOLSymConst{\HOLTokenImp{}} \HOLFreeVar{E} \HOLSymConst{\ensuremath{\mid}} \HOLFreeVar{E\sp{\prime}} \HOLTokenTransBegin\HOLSymConst{\ensuremath{\tau}}\HOLTokenTransEnd \HOLFreeVar{E\sb{\mathrm{1}}} \HOLSymConst{\ensuremath{\mid}} \HOLFreeVar{E\sb{\mathrm{2}}}\hfill\texttt{[PAR3]}
\HOLTokenTurnstile{} \HOLFreeVar{E} \HOLTokenTransBegin\HOLFreeVar{u}\HOLTokenTransEnd \HOLFreeVar{E\sp{\prime}} \HOLSymConst{\HOLTokenConj{}} \ensuremath{(}\HOLFreeVar{u} \HOLSymConst{\ensuremath{=}} \HOLSymConst{\ensuremath{\tau}} \HOLSymConst{\HOLTokenDisj{}} \HOLFreeVar{u} \HOLSymConst{\ensuremath{=}} \HOLConst{label} \HOLFreeVar{l} \HOLSymConst{\HOLTokenConj{}} \HOLFreeVar{l} \HOLSymConst{\HOLTokenNotIn{}} \HOLFreeVar{L} \HOLSymConst{\HOLTokenConj{}} \HOLConst{COMPL} \HOLFreeVar{l} \HOLSymConst{\HOLTokenNotIn{}} \HOLFreeVar{L}\ensuremath{)} \HOLSymConst{\HOLTokenImp{}}
   \ensuremath{(\nu}\HOLFreeVar{L}\ensuremath{)} \HOLFreeVar{E} \HOLTokenTransBegin\HOLFreeVar{u}\HOLTokenTransEnd \ensuremath{(\nu}\HOLFreeVar{L}\ensuremath{)} \HOLFreeVar{E\sp{\prime}}\hfill\texttt{[RESTR]}
\HOLTokenTurnstile{} \HOLFreeVar{E} \HOLTokenTransBegin\HOLFreeVar{u}\HOLTokenTransEnd \HOLFreeVar{E\sp{\prime}} \HOLSymConst{\HOLTokenImp{}} \HOLConst{relab} \HOLFreeVar{E} \HOLFreeVar{rf} \HOLTokenTransBegin\HOLConst{relabel} \HOLFreeVar{rf} \HOLFreeVar{u}\HOLTokenTransEnd \HOLConst{relab} \HOLFreeVar{E\sp{\prime}} \HOLFreeVar{rf}\hfill\texttt{[RELABELING]}
\HOLTokenTurnstile{} \ensuremath{[}\HOLConst{rec} \HOLFreeVar{X} \HOLFreeVar{E}\ensuremath{/}\HOLFreeVar{X}\ensuremath{]} \HOLFreeVar{E} \HOLTokenTransBegin\HOLFreeVar{u}\HOLTokenTransEnd \HOLFreeVar{E\sb{\mathrm{1}}} \HOLSymConst{\HOLTokenImp{}} \HOLConst{rec} \HOLFreeVar{X} \HOLFreeVar{E} \HOLTokenTransBegin\HOLFreeVar{u}\HOLTokenTransEnd \HOLFreeVar{E\sb{\mathrm{1}}}\hfill\texttt{[REC]}
\end{alltt}

The last rule \texttt{REC} (Recursion)
 says that if we substitute all appearances of variable $A$ in $P$ to
$(\recu A P)$ and the resulting process has a transition to $P'$
with action $u$, then $(\recu A P)$ has the same
transition. In its definition, \texttt{CCS_Subst} is a recursive substutiion function
with the following long definition:
\begin{alltt}
\ensuremath{[}\HOLFreeVar{E}\ensuremath{/}\HOLFreeVar{X}\ensuremath{]} \HOLConst{\ensuremath{\mathbf{0}}} \HOLSymConst{\HOLTokenDefEquality{}} \HOLConst{\ensuremath{\mathbf{0}}}
\ensuremath{[}\HOLFreeVar{E\sp{\prime}}\ensuremath{/}\HOLFreeVar{X}\ensuremath{]} \ensuremath{(}\HOLFreeVar{u}\HOLSymConst{\ensuremath{\ldotp}}\HOLFreeVar{E}\ensuremath{)} \HOLSymConst{\HOLTokenDefEquality{}} \HOLFreeVar{u}\HOLSymConst{\ensuremath{\ldotp}}\ensuremath{[}\HOLFreeVar{E\sp{\prime}}\ensuremath{/}\HOLFreeVar{X}\ensuremath{]} \HOLFreeVar{E}
\ensuremath{[}\HOLFreeVar{E\sp{\prime}}\ensuremath{/}\HOLFreeVar{X}\ensuremath{]} \ensuremath{(}\HOLFreeVar{E\sb{\mathrm{1}}} \HOLSymConst{\ensuremath{+}} \HOLFreeVar{E\sb{\mathrm{2}}}\ensuremath{)} \HOLSymConst{\HOLTokenDefEquality{}} \ensuremath{[}\HOLFreeVar{E\sp{\prime}}\ensuremath{/}\HOLFreeVar{X}\ensuremath{]} \HOLFreeVar{E\sb{\mathrm{1}}} \HOLSymConst{\ensuremath{+}} \ensuremath{[}\HOLFreeVar{E\sp{\prime}}\ensuremath{/}\HOLFreeVar{X}\ensuremath{]} \HOLFreeVar{E\sb{\mathrm{2}}}
\ensuremath{[}\HOLFreeVar{E\sp{\prime}}\ensuremath{/}\HOLFreeVar{X}\ensuremath{]} \ensuremath{(}\HOLFreeVar{E\sb{\mathrm{1}}} \HOLSymConst{\ensuremath{\mid}} \HOLFreeVar{E\sb{\mathrm{2}}}\ensuremath{)} \HOLSymConst{\HOLTokenDefEquality{}} \ensuremath{[}\HOLFreeVar{E\sp{\prime}}\ensuremath{/}\HOLFreeVar{X}\ensuremath{]} \HOLFreeVar{E\sb{\mathrm{1}}} \HOLSymConst{\ensuremath{\mid}} \ensuremath{[}\HOLFreeVar{E\sp{\prime}}\ensuremath{/}\HOLFreeVar{X}\ensuremath{]} \HOLFreeVar{E\sb{\mathrm{2}}}
\ensuremath{[}\HOLFreeVar{E\sp{\prime}}\ensuremath{/}\HOLFreeVar{X}\ensuremath{]} \ensuremath{(}\ensuremath{(\nu}\HOLFreeVar{L}\ensuremath{)} \HOLFreeVar{E}\ensuremath{)} \HOLSymConst{\HOLTokenDefEquality{}} \ensuremath{(\nu}\HOLFreeVar{L}\ensuremath{)} \ensuremath{(}\ensuremath{[}\HOLFreeVar{E\sp{\prime}}\ensuremath{/}\HOLFreeVar{X}\ensuremath{]} \HOLFreeVar{E}\ensuremath{)}
\ensuremath{[}\HOLFreeVar{E\sp{\prime}}\ensuremath{/}\HOLFreeVar{X}\ensuremath{]} \ensuremath{(}\HOLConst{relab} \HOLFreeVar{E} \HOLFreeVar{rf}\ensuremath{)} \HOLSymConst{\HOLTokenDefEquality{}} \HOLConst{relab} \ensuremath{(}\ensuremath{[}\HOLFreeVar{E\sp{\prime}}\ensuremath{/}\HOLFreeVar{X}\ensuremath{]} \HOLFreeVar{E}\ensuremath{)} \HOLFreeVar{rf}
\ensuremath{[}\HOLFreeVar{E}\ensuremath{/}\HOLFreeVar{X}\ensuremath{]} \ensuremath{(}\HOLConst{var} \HOLFreeVar{Y}\ensuremath{)} \HOLSymConst{\HOLTokenDefEquality{}} \HOLKeyword{if} \HOLFreeVar{Y} \HOLSymConst{\ensuremath{=}} \HOLFreeVar{X} \HOLKeyword{then} \HOLFreeVar{E} \HOLKeyword{else} \HOLConst{var} \HOLFreeVar{Y}
\ensuremath{[}\HOLFreeVar{E\sp{\prime}}\ensuremath{/}\HOLFreeVar{X}\ensuremath{]} \ensuremath{(}\HOLConst{rec} \HOLFreeVar{Y} \HOLFreeVar{E}\ensuremath{)} \HOLSymConst{\HOLTokenDefEquality{}} \HOLKeyword{if} \HOLFreeVar{Y} \HOLSymConst{\ensuremath{=}} \HOLFreeVar{X} \HOLKeyword{then} \HOLConst{rec} \HOLFreeVar{Y} \HOLFreeVar{E} \HOLKeyword{else} \HOLConst{rec} \HOLFreeVar{Y} \ensuremath{(}\ensuremath{[}\HOLFreeVar{E\sp{\prime}}\ensuremath{/}\HOLFreeVar{X}\ensuremath{]} \HOLFreeVar{E}\ensuremath{)}\hfill{[CCS_Subst_def]}
\end{alltt}

From HOL's viewpoint, these
SOS rules are \emph{inductive 
  definitions} on the tenary relation \HOLinline{\HOLConst{TRANS}} of type ``\HOLinline{\ensuremath{(}\ensuremath{\alpha}, \ensuremath{\beta}\ensuremath{)} \HOLTyOp{CCS} \HOLTokenTransEnd \ensuremath{\beta} \HOLTyOp{Action} \HOLTokenTransEnd \ensuremath{(}\ensuremath{\alpha}, \ensuremath{\beta}\ensuremath{)} \HOLTyOp{CCS} \HOLTokenTransEnd \HOLTyOp{bool}}'', generated by HOL's 
\texttt{Hol_reln} function.

A useful facility exploiting the interplay
between HOL4 and Standard ML (that follows an idea by Nesi \cite{Nesi:1992ve})
 is a complex Standard ML function
  taking a CCS process and returning a theorem indicating all
  direct transitions of the process.\footnote{If the input process were
 infinite branching, due to the use of recursion or
    relabeling operators, the program will loop forever.}
For instance, we know that the process $(a.0 | \bar{a}.0)$ has three
possible transitions: $(a.0 | \bar{a}.0) \overset{a}{\longrightarrow}
(0 | \bar{a}.0)$, $(a.0 | \bar{a}.0)
\overset{\bar{a}}{\longrightarrow} (a.0 | 0)$ and $(a.0 | \bar{a}.0)
\overset{\tau}{\longrightarrow} (0 | 0)$.
To completely describe all possible transitions of a process, if done manually, the
following facts should be proved: (1) there exists transitions from
$(a.0 | \bar{a}.0)$ (optional); (2) the correctness for each of the
transitions; and (3) the non-existence of other transitions.

For large processes it may be surprisingly hard to manually prove the
non-existence of transitions.  Hence the usefulness of appealing to 
the new  function \texttt{CCS\_TRANS\_CONV}. 
For instance this function
is called on the  process $(a.0 | \bar{a}.0)$ thus:
(\mbox{\color{blue}{\texttt{``}}} is for double-backquotes of HOL
  terms, \mbox{\color{blue}{\texttt{>}}} is HOL's prompt)
\begin{lstlisting}
> CCS_TRANS_CONV ``par (prefix (label (name "a")) nil)
                       (prefix (label (coname "a")) nil)``
\end{lstlisting}
This returns the following theorem, indeed describing all immediate
transitions of the process:
\begin{alltt}
\HOLTokenTurnstile{} \HOLSymConst{\HOLTokenForall{}}\HOLBoundVar{u} \HOLBoundVar{E}.
       \HOLConst{In} \HOLStringLit{a}\HOLSymConst{\ensuremath{\ldotp}}\HOLConst{\ensuremath{\mathbf{0}}} \HOLSymConst{\ensuremath{\mid}} \HOLConst{Out} \HOLStringLit{a}\HOLSymConst{\ensuremath{\ldotp}}\HOLConst{\ensuremath{\mathbf{0}}} \HOLTokenTransBegin\HOLBoundVar{u}\HOLTokenTransEnd \HOLBoundVar{E} \HOLSymConst{\HOLTokenEquiv{}}
       \ensuremath{(}\HOLBoundVar{u} \HOLSymConst{\ensuremath{=}} \HOLConst{In} \HOLStringLit{a} \HOLSymConst{\HOLTokenConj{}} \HOLBoundVar{E} \HOLSymConst{\ensuremath{=}} \HOLConst{\ensuremath{\mathbf{0}}} \HOLSymConst{\ensuremath{\mid}} \HOLConst{Out} \HOLStringLit{a}\HOLSymConst{\ensuremath{\ldotp}}\HOLConst{\ensuremath{\mathbf{0}}} \HOLSymConst{\HOLTokenDisj{}} \HOLBoundVar{u} \HOLSymConst{\ensuremath{=}} \HOLConst{Out} \HOLStringLit{a} \HOLSymConst{\HOLTokenConj{}} \HOLBoundVar{E} \HOLSymConst{\ensuremath{=}} \HOLConst{In} \HOLStringLit{a}\HOLSymConst{\ensuremath{\ldotp}}\HOLConst{\ensuremath{\mathbf{0}}} \HOLSymConst{\ensuremath{\mid}} \HOLConst{\ensuremath{\mathbf{0}}}\ensuremath{)} \HOLSymConst{\HOLTokenDisj{}}
       \HOLBoundVar{u} \HOLSymConst{\ensuremath{=}} \HOLSymConst{\ensuremath{\tau}} \HOLSymConst{\HOLTokenConj{}} \HOLBoundVar{E} \HOLSymConst{\ensuremath{=}} \HOLConst{\ensuremath{\mathbf{0}}} \HOLSymConst{\ensuremath{\mid}} \HOLConst{\ensuremath{\mathbf{0}}}\hfill{[Example.ex_A]}
\end{alltt}

% next file: bisim.htex

%%%% -*- Mode: LaTeX -*-
%%
%% This is the draft of the 2nd part of EXPRESS/SOS 2018 paper, coauthored by
%% Prof. Davide Sangiorgi and Chun Tian.

\subsection{Bisimulation and Bisimilarity}
\label{ss:bb}
\hl{One} highlight of this formalization project is the simplified definitions of
  bisimilarities using the new coinductive \hl{relation-defining} package of HOL4.
Without it, the bisimilaries can still be defined,  but
  proving their basic properties would be more complicated.

The definition of strong bisimulation (\texttt{STRONG_BISIM})  follows
  its textbook definitions. Essentially it is a predicate (i.e.~unary relation) stating
what kind of binary \hl{relation on CCS processes is a (strong) bisimulation}:
\begin{alltt}
\HOLConst{STRONG_BISIM} \HOLFreeVar{Bsm} \HOLSymConst{\HOLTokenDefEquality{}}
  \HOLSymConst{\HOLTokenForall{}}\HOLBoundVar{E} \HOLBoundVar{E\sp{\prime}}.
      \HOLFreeVar{Bsm} \HOLBoundVar{E} \HOLBoundVar{E\sp{\prime}} \HOLSymConst{\HOLTokenImp{}}
      \HOLSymConst{\HOLTokenForall{}}\HOLBoundVar{u}.
          \ensuremath{(}\HOLSymConst{\HOLTokenForall{}}\HOLBoundVar{E\sb{\mathrm{1}}}. \HOLBoundVar{E} \HOLTokenTransBegin\HOLBoundVar{u}\HOLTokenTransEnd \HOLBoundVar{E\sb{\mathrm{1}}} \HOLSymConst{\HOLTokenImp{}} \HOLSymConst{\HOLTokenExists{}}\HOLBoundVar{E\sb{\mathrm{2}}}. \HOLBoundVar{E\sp{\prime}} \HOLTokenTransBegin\HOLBoundVar{u}\HOLTokenTransEnd \HOLBoundVar{E\sb{\mathrm{2}}} \HOLSymConst{\HOLTokenConj{}} \HOLFreeVar{Bsm} \HOLBoundVar{E\sb{\mathrm{1}}} \HOLBoundVar{E\sb{\mathrm{2}}}\ensuremath{)} \HOLSymConst{\HOLTokenConj{}}
          \HOLSymConst{\HOLTokenForall{}}\HOLBoundVar{E\sb{\mathrm{2}}}. \HOLBoundVar{E\sp{\prime}} \HOLTokenTransBegin\HOLBoundVar{u}\HOLTokenTransEnd \HOLBoundVar{E\sb{\mathrm{2}}} \HOLSymConst{\HOLTokenImp{}} \HOLSymConst{\HOLTokenExists{}}\HOLBoundVar{E\sb{\mathrm{1}}}. \HOLBoundVar{E} \HOLTokenTransBegin\HOLBoundVar{u}\HOLTokenTransEnd \HOLBoundVar{E\sb{\mathrm{1}}} \HOLSymConst{\HOLTokenConj{}} \HOLFreeVar{Bsm} \HOLBoundVar{E\sb{\mathrm{1}}} \HOLBoundVar{E\sb{\mathrm{2}}}\hfill{[STRONG_BISIM]}
\end{alltt}
With \hl{the} above definition it is easy to prove that the identity
relation is indeed a bisimulation,  that bisimulation is preserved by
inversion, composition, and union operations.

Without the coinductive relation package for HOL,
the definition \hl{of (strong) bisimilarity} would be the following one:
\begin{alltt}
  \HOLFreeVar{E} \HOLSymConst{\HOLTokenStrongEQ} \HOLFreeVar{E\sp{\prime}} \HOLSymConst{\HOLTokenDefEquality{}} \HOLSymConst{\HOLTokenExists{}}\HOLBoundVar{Bsm}. \HOLBoundVar{Bsm} \HOLFreeVar{E} \HOLFreeVar{E\sp{\prime}} \HOLSymConst{\HOLTokenConj{}} \HOLConst{STRONG_BISIM} \HOLBoundVar{Bsm}\hfill{[STRONG_EQUIV]}
\end{alltt}
This is indeed the way followed by Nesi \cite{Nesi:1992ve} at  the
  time of HOL88. With the above definition, deriving, e.g.,
 the  property (*) below \citep[p.~91]{Mil89}  is tedious:
\begin{quote}
  $P\sim Q$ iff, for all $\mu$,\hfill{(*)}
  \begin{enumerate}[(i)]
    \item Whenever $P\overset{\mu}{\rightarrow}P'$ then, for some
      $Q'$, $Q\overset{\mu}{\rightarrow}Q'$ and $P'\sim Q'$
    \item Whenever $Q\overset{\mu}{\rightarrow}Q'$ then, for some
      $P'$, $P\overset{\mu}{\rightarrow}P'$ and $P'\sim Q'$
  \end{enumerate}
\end{quote}

%However we know that bisimiliarity is a coinductive relation, now 
With HOL's
new coinductive relation package (\texttt{Hol_coreln} (since \hl{the} Kananaskis-11 release),
it is possible to define (strong) bisimilarity in a more convienent way, which
essentially amounts to defining bisimilarity as the greatest
fixed-point of the appropriate functional on relations. Precisely we
call \hl{the} \texttt{Hol_coreln} command in the following way:%
\footnote{Here {\tt !} and {\tt ?} stand for universal and
existential quantifiers \hl{respectively} in HOL's ASCII-based term syntax.}
\begin{lstlisting}
val (STRONG_EQUIV_rules, STRONG_EQUIV_coind, STRONG_EQUIV_cases) = Hol_coreln `
    (!(E :('a, 'b) CCS) (E' :('a, 'b) CCS).
       (!u.
         (!E1. TRANS E u E1 ==>
               (?E2. TRANS E' u E2 /\ STRONG_EQUIV E1 E2)) /\
         (!E2. TRANS E' u E2 ==>
               (?E1. TRANS E u E1 /\ STRONG_EQUIV E1 E2))) ==> STRONG_EQUIV E E')`;
\end{lstlisting}
\texttt{Hol_coreln} returns 3 theorems: the first one,
\texttt{STRONG_EQUIV_rules}, is the
same as the input term\footnote{Our mixing of HOL notation and mathematical
  notation in this paper is not arbitrary.
We have  pasted here the
    original proof scripts,  written in HOL ASCII term
  notation (c.f.\  \cite{holdesc} for more details; HOL4 also supports writing Unicode symbols directly in
  proof scripts). All formal definitions and
  theorems in the paper are automatically generated from HOL4. We have
  tried to  generate
  Unicode and TeX outputs as easy as possible to read.}, but now promoted into a theorem.
The second and third theorems, namely \texttt{STRONG_EQUIV_coind} and \texttt{STRONG_EQUIV_cases},
represent the coinduction proof method for bisimilarity 
(i.e.~any bisimulation is contained in bisimilarity)
and the fixed-point property of bisimilarity
(bisimilarity itself is a bisimulation, thus the largest
bisimulation):
\begin{enumerate}
\item \begin{alltt}
\HOLTokenTurnstile{} \HOLSymConst{\HOLTokenForall{}}\HOLBoundVar{p} \HOLBoundVar{q}.
       \ensuremath{(}\HOLSymConst{\HOLTokenForall{}}\HOLBoundVar{l}.
            \ensuremath{(}\HOLSymConst{\HOLTokenForall{}}\HOLBoundVar{p\sp{\prime}}. \HOLBoundVar{p} \HOLTokenTransBegin\HOLBoundVar{l}\HOLTokenTransEnd \HOLBoundVar{p\sp{\prime}} \HOLSymConst{\HOLTokenImp{}} \HOLSymConst{\HOLTokenExists{}}\HOLBoundVar{q\sp{\prime}}. \HOLBoundVar{q} \HOLTokenTransBegin\HOLBoundVar{l}\HOLTokenTransEnd \HOLBoundVar{q\sp{\prime}} \HOLSymConst{\HOLTokenConj{}} \HOLBoundVar{p\sp{\prime}} \HOLSymConst{\HOLTokenStrongEQ} \HOLBoundVar{q\sp{\prime}}\ensuremath{)} \HOLSymConst{\HOLTokenConj{}}
            \HOLSymConst{\HOLTokenForall{}}\HOLBoundVar{q\sp{\prime}}. \HOLBoundVar{q} \HOLTokenTransBegin\HOLBoundVar{l}\HOLTokenTransEnd \HOLBoundVar{q\sp{\prime}} \HOLSymConst{\HOLTokenImp{}} \HOLSymConst{\HOLTokenExists{}}\HOLBoundVar{p\sp{\prime}}. \HOLBoundVar{p} \HOLTokenTransBegin\HOLBoundVar{l}\HOLTokenTransEnd \HOLBoundVar{p\sp{\prime}} \HOLSymConst{\HOLTokenConj{}} \HOLBoundVar{p\sp{\prime}} \HOLSymConst{\HOLTokenStrongEQ} \HOLBoundVar{q\sp{\prime}}\ensuremath{)} \HOLSymConst{\HOLTokenImp{}}
       \HOLBoundVar{p} \HOLSymConst{\HOLTokenStrongEQ} \HOLBoundVar{q}\hfill{[STRONG_EQUIV_rules]}
\end{alltt}
\item \begin{alltt}
\HOLTokenTurnstile{} \HOLSymConst{\HOLTokenForall{}}\HOLBoundVar{BISIM\HOLTokenUnderscore{}REL\sp{\prime}}.
       \ensuremath{(}\HOLSymConst{\HOLTokenForall{}}\HOLBoundVar{a\sb{\mathrm{0}}} \HOLBoundVar{a\sb{\mathrm{1}}}.
            \HOLBoundVar{BISIM\HOLTokenUnderscore{}REL\sp{\prime}} \HOLBoundVar{a\sb{\mathrm{0}}} \HOLBoundVar{a\sb{\mathrm{1}}} \HOLSymConst{\HOLTokenImp{}}
            \HOLSymConst{\HOLTokenForall{}}\HOLBoundVar{l}.
                \ensuremath{(}\HOLSymConst{\HOLTokenForall{}}\HOLBoundVar{p\sp{\prime}}. \HOLBoundVar{a\sb{\mathrm{0}}} \HOLTokenTransBegin\HOLBoundVar{l}\HOLTokenTransEnd \HOLBoundVar{p\sp{\prime}} \HOLSymConst{\HOLTokenImp{}} \HOLSymConst{\HOLTokenExists{}}\HOLBoundVar{q\sp{\prime}}. \HOLBoundVar{a\sb{\mathrm{1}}} \HOLTokenTransBegin\HOLBoundVar{l}\HOLTokenTransEnd \HOLBoundVar{q\sp{\prime}} \HOLSymConst{\HOLTokenConj{}} \HOLBoundVar{BISIM\HOLTokenUnderscore{}REL\sp{\prime}} \HOLBoundVar{p\sp{\prime}} \HOLBoundVar{q\sp{\prime}}\ensuremath{)} \HOLSymConst{\HOLTokenConj{}}
                \HOLSymConst{\HOLTokenForall{}}\HOLBoundVar{q\sp{\prime}}. \HOLBoundVar{a\sb{\mathrm{1}}} \HOLTokenTransBegin\HOLBoundVar{l}\HOLTokenTransEnd \HOLBoundVar{q\sp{\prime}} \HOLSymConst{\HOLTokenImp{}} \HOLSymConst{\HOLTokenExists{}}\HOLBoundVar{p\sp{\prime}}. \HOLBoundVar{a\sb{\mathrm{0}}} \HOLTokenTransBegin\HOLBoundVar{l}\HOLTokenTransEnd \HOLBoundVar{p\sp{\prime}} \HOLSymConst{\HOLTokenConj{}} \HOLBoundVar{BISIM\HOLTokenUnderscore{}REL\sp{\prime}} \HOLBoundVar{p\sp{\prime}} \HOLBoundVar{q\sp{\prime}}\ensuremath{)} \HOLSymConst{\HOLTokenImp{}}
       \HOLSymConst{\HOLTokenForall{}}\HOLBoundVar{a\sb{\mathrm{0}}} \HOLBoundVar{a\sb{\mathrm{1}}}. \HOLBoundVar{BISIM\HOLTokenUnderscore{}REL\sp{\prime}} \HOLBoundVar{a\sb{\mathrm{0}}} \HOLBoundVar{a\sb{\mathrm{1}}} \HOLSymConst{\HOLTokenImp{}} \HOLBoundVar{a\sb{\mathrm{0}}} \HOLSymConst{\HOLTokenStrongEQ} \HOLBoundVar{a\sb{\mathrm{1}}}\hfill{[STRONG_EQUIV_coind]}
\end{alltt}
\item \begin{alltt}
\HOLTokenTurnstile{} \HOLSymConst{\HOLTokenForall{}}\HOLBoundVar{a\sb{\mathrm{0}}} \HOLBoundVar{a\sb{\mathrm{1}}}.
       \HOLBoundVar{a\sb{\mathrm{0}}} \HOLSymConst{\HOLTokenStrongEQ} \HOLBoundVar{a\sb{\mathrm{1}}} \HOLSymConst{\HOLTokenEquiv{}}
       \HOLSymConst{\HOLTokenForall{}}\HOLBoundVar{l}.
           \ensuremath{(}\HOLSymConst{\HOLTokenForall{}}\HOLBoundVar{p\sp{\prime}}. \HOLBoundVar{a\sb{\mathrm{0}}} \HOLTokenTransBegin\HOLBoundVar{l}\HOLTokenTransEnd \HOLBoundVar{p\sp{\prime}} \HOLSymConst{\HOLTokenImp{}} \HOLSymConst{\HOLTokenExists{}}\HOLBoundVar{q\sp{\prime}}. \HOLBoundVar{a\sb{\mathrm{1}}} \HOLTokenTransBegin\HOLBoundVar{l}\HOLTokenTransEnd \HOLBoundVar{q\sp{\prime}} \HOLSymConst{\HOLTokenConj{}} \HOLBoundVar{p\sp{\prime}} \HOLSymConst{\HOLTokenStrongEQ} \HOLBoundVar{q\sp{\prime}}\ensuremath{)} \HOLSymConst{\HOLTokenConj{}}
           \HOLSymConst{\HOLTokenForall{}}\HOLBoundVar{q\sp{\prime}}. \HOLBoundVar{a\sb{\mathrm{1}}} \HOLTokenTransBegin\HOLBoundVar{l}\HOLTokenTransEnd \HOLBoundVar{q\sp{\prime}} \HOLSymConst{\HOLTokenImp{}} \HOLSymConst{\HOLTokenExists{}}\HOLBoundVar{p\sp{\prime}}. \HOLBoundVar{a\sb{\mathrm{0}}} \HOLTokenTransBegin\HOLBoundVar{l}\HOLTokenTransEnd \HOLBoundVar{p\sp{\prime}} \HOLSymConst{\HOLTokenConj{}} \HOLBoundVar{p\sp{\prime}} \HOLSymConst{\HOLTokenStrongEQ} \HOLBoundVar{q\sp{\prime}}\hfill{[STRONG_EQUIV_cases]}
\end{alltt}
\end{enumerate}
The last theorem, \texttt{STRONG_EQUIV_cases}, is indeed \hl{the} property (*)
mentioned earlier.
% , and these theorems completely capture the definition of
% , because now we can easily derive the original definition of
% \texttt{STRONG_EQUIV} as a theorem from \texttt{STRONG_EQUIV_coind}
% and \texttt{STRONG_EQUIV_cases}.

To define (weak) bisimilarity, we first need to define weak
transitions of CCS processes. 
Following  Nesi \cite{Nesi:1992ve},
we define a (possibly empty) sequence of $\tau$-transitions between
two processes as
a new relation \texttt{EPS}
($\overset{\epsilon}{\Longrightarrow}$), which is the
reflexive transitive closure (RTC, denoted by \mbox{\color{blue}{$^*$}} in
HOL4) of ordinary $\tau$-transitions of CCS processes:
\begin{alltt}
\HOLConst{EPS} \HOLSymConst{\HOLTokenDefEquality{}} \ensuremath{(}\HOLTokenLambda{}\HOLBoundVar{E} \HOLBoundVar{E\sp{\prime}}. \HOLBoundVar{E} \HOLTokenTransBegin\HOLSymConst{\ensuremath{\tau}}\HOLTokenTransEnd \HOLBoundVar{E\sp{\prime}}\ensuremath{)}\HOLSymConst{\HOLTokenSupStar{}}\hfill{[EPS_def]}
\end{alltt}
Then we can define a weak transition as an ordinary transition wrapped by
two $\epsilon$-transitions:
\begin{alltt}
\HOLFreeVar{E} \HOLTokenWeakTransBegin\HOLFreeVar{u}\HOLTokenWeakTransEnd \HOLFreeVar{E\sp{\prime}} \HOLSymConst{\HOLTokenDefEquality{}} \HOLSymConst{\HOLTokenExists{}}\HOLBoundVar{E\sb{\mathrm{1}}} \HOLBoundVar{E\sb{\mathrm{2}}}. \HOLFreeVar{E} \HOLSymConst{\HOLTokenEPS} \HOLBoundVar{E\sb{\mathrm{1}}} \HOLSymConst{\HOLTokenConj{}} \HOLBoundVar{E\sb{\mathrm{1}}} \HOLTokenTransBegin\HOLFreeVar{u}\HOLTokenTransEnd \HOLBoundVar{E\sb{\mathrm{2}}} \HOLSymConst{\HOLTokenConj{}} \HOLBoundVar{E\sb{\mathrm{2}}} \HOLSymConst{\HOLTokenEPS} \HOLFreeVar{E\sp{\prime}}\hfill{[WEAK_TRANS]}
\end{alltt}

\hl{The definition of weak bisimulation is based on weak and $\epsilon$--transitions:}
\begin{alltt}
\HOLConst{WEAK_BISIM} \HOLFreeVar{Wbsm} \HOLSymConst{\HOLTokenDefEquality{}}
  \HOLSymConst{\HOLTokenForall{}}\HOLBoundVar{E} \HOLBoundVar{E\sp{\prime}}.
      \HOLFreeVar{Wbsm} \HOLBoundVar{E} \HOLBoundVar{E\sp{\prime}} \HOLSymConst{\HOLTokenImp{}}
      \ensuremath{(}\HOLSymConst{\HOLTokenForall{}}\HOLBoundVar{l}.
           \ensuremath{(}\HOLSymConst{\HOLTokenForall{}}\HOLBoundVar{E\sb{\mathrm{1}}}. \HOLBoundVar{E} \HOLTokenTransBegin\HOLConst{label} \HOLBoundVar{l}\HOLTokenTransEnd \HOLBoundVar{E\sb{\mathrm{1}}} \HOLSymConst{\HOLTokenImp{}} \HOLSymConst{\HOLTokenExists{}}\HOLBoundVar{E\sb{\mathrm{2}}}. \HOLBoundVar{E\sp{\prime}} \HOLTokenWeakTransBegin\HOLConst{label} \HOLBoundVar{l}\HOLTokenWeakTransEnd \HOLBoundVar{E\sb{\mathrm{2}}} \HOLSymConst{\HOLTokenConj{}} \HOLFreeVar{Wbsm} \HOLBoundVar{E\sb{\mathrm{1}}} \HOLBoundVar{E\sb{\mathrm{2}}}\ensuremath{)} \HOLSymConst{\HOLTokenConj{}}
           \HOLSymConst{\HOLTokenForall{}}\HOLBoundVar{E\sb{\mathrm{2}}}. \HOLBoundVar{E\sp{\prime}} \HOLTokenTransBegin\HOLConst{label} \HOLBoundVar{l}\HOLTokenTransEnd \HOLBoundVar{E\sb{\mathrm{2}}} \HOLSymConst{\HOLTokenImp{}} \HOLSymConst{\HOLTokenExists{}}\HOLBoundVar{E\sb{\mathrm{1}}}. \HOLBoundVar{E} \HOLTokenWeakTransBegin\HOLConst{label} \HOLBoundVar{l}\HOLTokenWeakTransEnd \HOLBoundVar{E\sb{\mathrm{1}}} \HOLSymConst{\HOLTokenConj{}} \HOLFreeVar{Wbsm} \HOLBoundVar{E\sb{\mathrm{1}}} \HOLBoundVar{E\sb{\mathrm{2}}}\ensuremath{)} \HOLSymConst{\HOLTokenConj{}}
      \ensuremath{(}\HOLSymConst{\HOLTokenForall{}}\HOLBoundVar{E\sb{\mathrm{1}}}. \HOLBoundVar{E} \HOLTokenTransBegin\HOLSymConst{\ensuremath{\tau}}\HOLTokenTransEnd \HOLBoundVar{E\sb{\mathrm{1}}} \HOLSymConst{\HOLTokenImp{}} \HOLSymConst{\HOLTokenExists{}}\HOLBoundVar{E\sb{\mathrm{2}}}. \HOLBoundVar{E\sp{\prime}} \HOLSymConst{\HOLTokenEPS} \HOLBoundVar{E\sb{\mathrm{2}}} \HOLSymConst{\HOLTokenConj{}} \HOLFreeVar{Wbsm} \HOLBoundVar{E\sb{\mathrm{1}}} \HOLBoundVar{E\sb{\mathrm{2}}}\ensuremath{)} \HOLSymConst{\HOLTokenConj{}}
      \HOLSymConst{\HOLTokenForall{}}\HOLBoundVar{E\sb{\mathrm{2}}}. \HOLBoundVar{E\sp{\prime}} \HOLTokenTransBegin\HOLSymConst{\ensuremath{\tau}}\HOLTokenTransEnd \HOLBoundVar{E\sb{\mathrm{2}}} \HOLSymConst{\HOLTokenImp{}} \HOLSymConst{\HOLTokenExists{}}\HOLBoundVar{E\sb{\mathrm{1}}}. \HOLBoundVar{E} \HOLSymConst{\HOLTokenEPS} \HOLBoundVar{E\sb{\mathrm{1}}} \HOLSymConst{\HOLTokenConj{}} \HOLFreeVar{Wbsm} \HOLBoundVar{E\sb{\mathrm{1}}} \HOLBoundVar{E\sb{\mathrm{2}}}\hfill{[WEAK_BISIM]}
\end{alltt}

Again, we can prove that the identity
relation is a bisimulation, that bisimulation is preserved by inversion,
 composition, and union. 
The definition of weak bisimilarity can be
 generated with the following call to
\texttt{Hol_coreln}:
\begin{lstlisting}
val (WEAK_EQUIV_rules, WEAK_EQUIV_coind, WEAK_EQUIV_cases) = Hol_coreln `
    (!(E :('a, 'b) CCS) (E' :('a, 'b) CCS).
       (!l.
         (!E1. TRANS E  (label l) E1 ==>
               (?E2. WEAK_TRANS E' (label l) E2 /\ WEAK_EQUIV E1 E2)) /\
         (!E2. TRANS E' (label l) E2 ==>
               (?E1. WEAK_TRANS E  (label l) E1 /\ WEAK_EQUIV E1 E2))) /\
       (!E1. TRANS E  tau E1 ==> (?E2. EPS E' E2 /\ WEAK_EQUIV E1 E2)) /\
       (!E2. TRANS E' tau E2 ==> (?E1. EPS E  E1 /\ WEAK_EQUIV E1 E2))
      ==> WEAK_EQUIV E E')`;
\end{lstlisting}
This returns the following 3 theorems defining \texttt{WEAK_EQUIV}:
\begin{enumerate}
\item \begin{alltt}
\HOLTokenTurnstile{} \HOLSymConst{\HOLTokenForall{}}\HOLBoundVar{E} \HOLBoundVar{E\sp{\prime}}.
       \ensuremath{(}\HOLSymConst{\HOLTokenForall{}}\HOLBoundVar{l}.
            \ensuremath{(}\HOLSymConst{\HOLTokenForall{}}\HOLBoundVar{E\sb{\mathrm{1}}}. \HOLBoundVar{E} \HOLTokenTransBegin\HOLConst{label} \HOLBoundVar{l}\HOLTokenTransEnd \HOLBoundVar{E\sb{\mathrm{1}}} \HOLSymConst{\HOLTokenImp{}} \HOLSymConst{\HOLTokenExists{}}\HOLBoundVar{E\sb{\mathrm{2}}}. \HOLBoundVar{E\sp{\prime}} \HOLTokenWeakTransBegin\HOLConst{label} \HOLBoundVar{l}\HOLTokenWeakTransEnd \HOLBoundVar{E\sb{\mathrm{2}}} \HOLSymConst{\HOLTokenConj{}} \HOLBoundVar{E\sb{\mathrm{1}}} \HOLSymConst{\HOLTokenWeakEQ} \HOLBoundVar{E\sb{\mathrm{2}}}\ensuremath{)} \HOLSymConst{\HOLTokenConj{}}
            \HOLSymConst{\HOLTokenForall{}}\HOLBoundVar{E\sb{\mathrm{2}}}. \HOLBoundVar{E\sp{\prime}} \HOLTokenTransBegin\HOLConst{label} \HOLBoundVar{l}\HOLTokenTransEnd \HOLBoundVar{E\sb{\mathrm{2}}} \HOLSymConst{\HOLTokenImp{}} \HOLSymConst{\HOLTokenExists{}}\HOLBoundVar{E\sb{\mathrm{1}}}. \HOLBoundVar{E} \HOLTokenWeakTransBegin\HOLConst{label} \HOLBoundVar{l}\HOLTokenWeakTransEnd \HOLBoundVar{E\sb{\mathrm{1}}} \HOLSymConst{\HOLTokenConj{}} \HOLBoundVar{E\sb{\mathrm{1}}} \HOLSymConst{\HOLTokenWeakEQ} \HOLBoundVar{E\sb{\mathrm{2}}}\ensuremath{)} \HOLSymConst{\HOLTokenConj{}}
       \ensuremath{(}\HOLSymConst{\HOLTokenForall{}}\HOLBoundVar{E\sb{\mathrm{1}}}. \HOLBoundVar{E} \HOLTokenTransBegin\HOLSymConst{\ensuremath{\tau}}\HOLTokenTransEnd \HOLBoundVar{E\sb{\mathrm{1}}} \HOLSymConst{\HOLTokenImp{}} \HOLSymConst{\HOLTokenExists{}}\HOLBoundVar{E\sb{\mathrm{2}}}. \HOLBoundVar{E\sp{\prime}} \HOLSymConst{\HOLTokenEPS} \HOLBoundVar{E\sb{\mathrm{2}}} \HOLSymConst{\HOLTokenConj{}} \HOLBoundVar{E\sb{\mathrm{1}}} \HOLSymConst{\HOLTokenWeakEQ} \HOLBoundVar{E\sb{\mathrm{2}}}\ensuremath{)} \HOLSymConst{\HOLTokenConj{}}
       \ensuremath{(}\HOLSymConst{\HOLTokenForall{}}\HOLBoundVar{E\sb{\mathrm{2}}}. \HOLBoundVar{E\sp{\prime}} \HOLTokenTransBegin\HOLSymConst{\ensuremath{\tau}}\HOLTokenTransEnd \HOLBoundVar{E\sb{\mathrm{2}}} \HOLSymConst{\HOLTokenImp{}} \HOLSymConst{\HOLTokenExists{}}\HOLBoundVar{E\sb{\mathrm{1}}}. \HOLBoundVar{E} \HOLSymConst{\HOLTokenEPS} \HOLBoundVar{E\sb{\mathrm{1}}} \HOLSymConst{\HOLTokenConj{}} \HOLBoundVar{E\sb{\mathrm{1}}} \HOLSymConst{\HOLTokenWeakEQ} \HOLBoundVar{E\sb{\mathrm{2}}}\ensuremath{)} \HOLSymConst{\HOLTokenImp{}}
       \HOLBoundVar{E} \HOLSymConst{\HOLTokenWeakEQ} \HOLBoundVar{E\sp{\prime}}\hfill{[WEAK_EQUIV_rules]}
\end{alltt}
\item \begin{alltt}
\HOLTokenTurnstile{} \HOLSymConst{\HOLTokenForall{}}\HOLBoundVar{WEAK\HOLTokenUnderscore{}EQUIV\sp{\prime}}.
       \ensuremath{(}\HOLSymConst{\HOLTokenForall{}}\HOLBoundVar{a\sb{\mathrm{0}}} \HOLBoundVar{a\sb{\mathrm{1}}}.
            \HOLBoundVar{WEAK\HOLTokenUnderscore{}EQUIV\sp{\prime}} \HOLBoundVar{a\sb{\mathrm{0}}} \HOLBoundVar{a\sb{\mathrm{1}}} \HOLSymConst{\HOLTokenImp{}}
            \ensuremath{(}\HOLSymConst{\HOLTokenForall{}}\HOLBoundVar{l}.
                 \ensuremath{(}\HOLSymConst{\HOLTokenForall{}}\HOLBoundVar{E\sb{\mathrm{1}}}.
                      \HOLBoundVar{a\sb{\mathrm{0}}} \HOLTokenTransBegin\HOLConst{label} \HOLBoundVar{l}\HOLTokenTransEnd \HOLBoundVar{E\sb{\mathrm{1}}} \HOLSymConst{\HOLTokenImp{}}
                      \HOLSymConst{\HOLTokenExists{}}\HOLBoundVar{E\sb{\mathrm{2}}}. \HOLBoundVar{a\sb{\mathrm{1}}} \HOLTokenWeakTransBegin\HOLConst{label} \HOLBoundVar{l}\HOLTokenWeakTransEnd \HOLBoundVar{E\sb{\mathrm{2}}} \HOLSymConst{\HOLTokenConj{}} \HOLBoundVar{WEAK\HOLTokenUnderscore{}EQUIV\sp{\prime}} \HOLBoundVar{E\sb{\mathrm{1}}} \HOLBoundVar{E\sb{\mathrm{2}}}\ensuremath{)} \HOLSymConst{\HOLTokenConj{}}
                 \HOLSymConst{\HOLTokenForall{}}\HOLBoundVar{E\sb{\mathrm{2}}}.
                     \HOLBoundVar{a\sb{\mathrm{1}}} \HOLTokenTransBegin\HOLConst{label} \HOLBoundVar{l}\HOLTokenTransEnd \HOLBoundVar{E\sb{\mathrm{2}}} \HOLSymConst{\HOLTokenImp{}}
                     \HOLSymConst{\HOLTokenExists{}}\HOLBoundVar{E\sb{\mathrm{1}}}. \HOLBoundVar{a\sb{\mathrm{0}}} \HOLTokenWeakTransBegin\HOLConst{label} \HOLBoundVar{l}\HOLTokenWeakTransEnd \HOLBoundVar{E\sb{\mathrm{1}}} \HOLSymConst{\HOLTokenConj{}} \HOLBoundVar{WEAK\HOLTokenUnderscore{}EQUIV\sp{\prime}} \HOLBoundVar{E\sb{\mathrm{1}}} \HOLBoundVar{E\sb{\mathrm{2}}}\ensuremath{)} \HOLSymConst{\HOLTokenConj{}}
            \ensuremath{(}\HOLSymConst{\HOLTokenForall{}}\HOLBoundVar{E\sb{\mathrm{1}}}. \HOLBoundVar{a\sb{\mathrm{0}}} \HOLTokenTransBegin\HOLSymConst{\ensuremath{\tau}}\HOLTokenTransEnd \HOLBoundVar{E\sb{\mathrm{1}}} \HOLSymConst{\HOLTokenImp{}} \HOLSymConst{\HOLTokenExists{}}\HOLBoundVar{E\sb{\mathrm{2}}}. \HOLBoundVar{a\sb{\mathrm{1}}} \HOLSymConst{\HOLTokenEPS} \HOLBoundVar{E\sb{\mathrm{2}}} \HOLSymConst{\HOLTokenConj{}} \HOLBoundVar{WEAK\HOLTokenUnderscore{}EQUIV\sp{\prime}} \HOLBoundVar{E\sb{\mathrm{1}}} \HOLBoundVar{E\sb{\mathrm{2}}}\ensuremath{)} \HOLSymConst{\HOLTokenConj{}}
            \HOLSymConst{\HOLTokenForall{}}\HOLBoundVar{E\sb{\mathrm{2}}}. \HOLBoundVar{a\sb{\mathrm{1}}} \HOLTokenTransBegin\HOLSymConst{\ensuremath{\tau}}\HOLTokenTransEnd \HOLBoundVar{E\sb{\mathrm{2}}} \HOLSymConst{\HOLTokenImp{}} \HOLSymConst{\HOLTokenExists{}}\HOLBoundVar{E\sb{\mathrm{1}}}. \HOLBoundVar{a\sb{\mathrm{0}}} \HOLSymConst{\HOLTokenEPS} \HOLBoundVar{E\sb{\mathrm{1}}} \HOLSymConst{\HOLTokenConj{}} \HOLBoundVar{WEAK\HOLTokenUnderscore{}EQUIV\sp{\prime}} \HOLBoundVar{E\sb{\mathrm{1}}} \HOLBoundVar{E\sb{\mathrm{2}}}\ensuremath{)} \HOLSymConst{\HOLTokenImp{}}
       \HOLSymConst{\HOLTokenForall{}}\HOLBoundVar{a\sb{\mathrm{0}}} \HOLBoundVar{a\sb{\mathrm{1}}}. \HOLBoundVar{WEAK\HOLTokenUnderscore{}EQUIV\sp{\prime}} \HOLBoundVar{a\sb{\mathrm{0}}} \HOLBoundVar{a\sb{\mathrm{1}}} \HOLSymConst{\HOLTokenImp{}} \HOLBoundVar{a\sb{\mathrm{0}}} \HOLSymConst{\HOLTokenWeakEQ} \HOLBoundVar{a\sb{\mathrm{1}}}\hfill{[WEAK_EQUIV_coind]}
\end{alltt}
\item \begin{alltt}
\HOLTokenTurnstile{} \HOLSymConst{\HOLTokenForall{}}\HOLBoundVar{a\sb{\mathrm{0}}} \HOLBoundVar{a\sb{\mathrm{1}}}.
       \HOLBoundVar{a\sb{\mathrm{0}}} \HOLSymConst{\HOLTokenWeakEQ} \HOLBoundVar{a\sb{\mathrm{1}}} \HOLSymConst{\HOLTokenEquiv{}}
       \ensuremath{(}\HOLSymConst{\HOLTokenForall{}}\HOLBoundVar{l}.
            \ensuremath{(}\HOLSymConst{\HOLTokenForall{}}\HOLBoundVar{E\sb{\mathrm{1}}}. \HOLBoundVar{a\sb{\mathrm{0}}} \HOLTokenTransBegin\HOLConst{label} \HOLBoundVar{l}\HOLTokenTransEnd \HOLBoundVar{E\sb{\mathrm{1}}} \HOLSymConst{\HOLTokenImp{}} \HOLSymConst{\HOLTokenExists{}}\HOLBoundVar{E\sb{\mathrm{2}}}. \HOLBoundVar{a\sb{\mathrm{1}}} \HOLTokenWeakTransBegin\HOLConst{label} \HOLBoundVar{l}\HOLTokenWeakTransEnd \HOLBoundVar{E\sb{\mathrm{2}}} \HOLSymConst{\HOLTokenConj{}} \HOLBoundVar{E\sb{\mathrm{1}}} \HOLSymConst{\HOLTokenWeakEQ} \HOLBoundVar{E\sb{\mathrm{2}}}\ensuremath{)} \HOLSymConst{\HOLTokenConj{}}
            \HOLSymConst{\HOLTokenForall{}}\HOLBoundVar{E\sb{\mathrm{2}}}. \HOLBoundVar{a\sb{\mathrm{1}}} \HOLTokenTransBegin\HOLConst{label} \HOLBoundVar{l}\HOLTokenTransEnd \HOLBoundVar{E\sb{\mathrm{2}}} \HOLSymConst{\HOLTokenImp{}} \HOLSymConst{\HOLTokenExists{}}\HOLBoundVar{E\sb{\mathrm{1}}}. \HOLBoundVar{a\sb{\mathrm{0}}} \HOLTokenWeakTransBegin\HOLConst{label} \HOLBoundVar{l}\HOLTokenWeakTransEnd \HOLBoundVar{E\sb{\mathrm{1}}} \HOLSymConst{\HOLTokenConj{}} \HOLBoundVar{E\sb{\mathrm{1}}} \HOLSymConst{\HOLTokenWeakEQ} \HOLBoundVar{E\sb{\mathrm{2}}}\ensuremath{)} \HOLSymConst{\HOLTokenConj{}}
       \ensuremath{(}\HOLSymConst{\HOLTokenForall{}}\HOLBoundVar{E\sb{\mathrm{1}}}. \HOLBoundVar{a\sb{\mathrm{0}}} \HOLTokenTransBegin\HOLSymConst{\ensuremath{\tau}}\HOLTokenTransEnd \HOLBoundVar{E\sb{\mathrm{1}}} \HOLSymConst{\HOLTokenImp{}} \HOLSymConst{\HOLTokenExists{}}\HOLBoundVar{E\sb{\mathrm{2}}}. \HOLBoundVar{a\sb{\mathrm{1}}} \HOLSymConst{\HOLTokenEPS} \HOLBoundVar{E\sb{\mathrm{2}}} \HOLSymConst{\HOLTokenConj{}} \HOLBoundVar{E\sb{\mathrm{1}}} \HOLSymConst{\HOLTokenWeakEQ} \HOLBoundVar{E\sb{\mathrm{2}}}\ensuremath{)} \HOLSymConst{\HOLTokenConj{}}
       \HOLSymConst{\HOLTokenForall{}}\HOLBoundVar{E\sb{\mathrm{2}}}. \HOLBoundVar{a\sb{\mathrm{1}}} \HOLTokenTransBegin\HOLSymConst{\ensuremath{\tau}}\HOLTokenTransEnd \HOLBoundVar{E\sb{\mathrm{2}}} \HOLSymConst{\HOLTokenImp{}} \HOLSymConst{\HOLTokenExists{}}\HOLBoundVar{E\sb{\mathrm{1}}}. \HOLBoundVar{a\sb{\mathrm{0}}} \HOLSymConst{\HOLTokenEPS} \HOLBoundVar{E\sb{\mathrm{1}}} \HOLSymConst{\HOLTokenConj{}} \HOLBoundVar{E\sb{\mathrm{1}}} \HOLSymConst{\HOLTokenWeakEQ} \HOLBoundVar{E\sb{\mathrm{2}}}\hfill{[WEAK_EQUIV_cases]}
\end{alltt}
\end{enumerate}

The coinduction principle \texttt{WEAK_EQUIV_coind} says that any
bisimulation is contained in the resulting relation (i.e.~it is
largest), but it didn't constrain the resulting relation in the set of
fixed points (e.g.~even the universal relation---the set of all
pairs---would fit with this theorem); the
purpose of \texttt{WEAK_EQUIV_cases} is to
further assert that the resulting relation is indeed a
fixed point. Thus \texttt{WEAK_EQUIV_coind} and \texttt{WEAK_EQUIV_cases}
together make sure that bisimilarity is the greatest
fixed point, as
the former contributes to ``greatest'' while the latter
contributes to ``fixed point''.
%
Without HOL's coinductive relation package, (weak) bisimilarity
would have to be defined by following literally
Def.~\ref{d:wb};  then other properties of bisimilarity, such
as the fixed-point property in \texttt{WEAK_EQUIV_cases}, would have to be
derived manually.

Finally, the original definition of \texttt{WEAK_EQUIV}
becomes a theorem:
\begin{alltt}
\HOLTokenTurnstile{} \HOLFreeVar{E} \HOLSymConst{\HOLTokenWeakEQ} \HOLFreeVar{E\sp{\prime}} \HOLSymConst{\HOLTokenEquiv{}} \HOLSymConst{\HOLTokenExists{}}\HOLBoundVar{Wbsm}. \HOLBoundVar{Wbsm} \HOLFreeVar{E} \HOLFreeVar{E\sp{\prime}} \HOLSymConst{\HOLTokenConj{}} \HOLConst{WEAK_BISIM} \HOLBoundVar{Wbsm}\hfill{[WEAK_EQUIV]}
\end{alltt}

% \finish{I have removed other things as i fear they would confuse a
%   reader and I think the main point we wanted to say are now clearly
%   expressed}  (I don't buy this any more, sorry)

The formal definition of rooted bisimilarity ($\rapprox$, \texttt{OBS_CONGR}) 
follows Definition~\ref{d:rootedBisimilarity}:
\begin{alltt}
\HOLFreeVar{E} \HOLSymConst{\HOLTokenObsCongr} \HOLFreeVar{E\sp{\prime}} \HOLSymConst{\HOLTokenDefEquality{}}
  \HOLSymConst{\HOLTokenForall{}}\HOLBoundVar{u}.
      \ensuremath{(}\HOLSymConst{\HOLTokenForall{}}\HOLBoundVar{E\sb{\mathrm{1}}}. \HOLFreeVar{E} \HOLTokenTransBegin\HOLBoundVar{u}\HOLTokenTransEnd \HOLBoundVar{E\sb{\mathrm{1}}} \HOLSymConst{\HOLTokenImp{}} \HOLSymConst{\HOLTokenExists{}}\HOLBoundVar{E\sb{\mathrm{2}}}. \HOLFreeVar{E\sp{\prime}} \HOLTokenWeakTransBegin\HOLBoundVar{u}\HOLTokenWeakTransEnd \HOLBoundVar{E\sb{\mathrm{2}}} \HOLSymConst{\HOLTokenConj{}} \HOLBoundVar{E\sb{\mathrm{1}}} \HOLSymConst{\HOLTokenWeakEQ} \HOLBoundVar{E\sb{\mathrm{2}}}\ensuremath{)} \HOLSymConst{\HOLTokenConj{}}
      \HOLSymConst{\HOLTokenForall{}}\HOLBoundVar{E\sb{\mathrm{2}}}. \HOLFreeVar{E\sp{\prime}} \HOLTokenTransBegin\HOLBoundVar{u}\HOLTokenTransEnd \HOLBoundVar{E\sb{\mathrm{2}}} \HOLSymConst{\HOLTokenImp{}} \HOLSymConst{\HOLTokenExists{}}\HOLBoundVar{E\sb{\mathrm{1}}}. \HOLFreeVar{E} \HOLTokenWeakTransBegin\HOLBoundVar{u}\HOLTokenWeakTransEnd \HOLBoundVar{E\sb{\mathrm{1}}} \HOLSymConst{\HOLTokenConj{}} \HOLBoundVar{E\sb{\mathrm{1}}} \HOLSymConst{\HOLTokenWeakEQ} \HOLBoundVar{E\sb{\mathrm{2}}}\hfill{[OBS_CONGR]}
\end{alltt}
Below is the formal version of Lemma~\ref{l:obsCongrByWeakBisim}, which is needed in the proof
of unique-solution Theorem~\ref{t:rcontraBisimulationU}:
\begin{alltt}
\HOLTokenTurnstile{} \HOLConst{WEAK_BISIM} \HOLFreeVar{Wbsm} \HOLSymConst{\HOLTokenImp{}}
   \HOLSymConst{\HOLTokenForall{}}\HOLBoundVar{E} \HOLBoundVar{E\sp{\prime}}.
       \ensuremath{(}\HOLSymConst{\HOLTokenForall{}}\HOLBoundVar{u}.
            \ensuremath{(}\HOLSymConst{\HOLTokenForall{}}\HOLBoundVar{E\sb{\mathrm{1}}}. \HOLBoundVar{E} \HOLTokenTransBegin\HOLBoundVar{u}\HOLTokenTransEnd \HOLBoundVar{E\sb{\mathrm{1}}} \HOLSymConst{\HOLTokenImp{}} \HOLSymConst{\HOLTokenExists{}}\HOLBoundVar{E\sb{\mathrm{2}}}. \HOLBoundVar{E\sp{\prime}} \HOLTokenWeakTransBegin\HOLBoundVar{u}\HOLTokenWeakTransEnd \HOLBoundVar{E\sb{\mathrm{2}}} \HOLSymConst{\HOLTokenConj{}} \HOLFreeVar{Wbsm} \HOLBoundVar{E\sb{\mathrm{1}}} \HOLBoundVar{E\sb{\mathrm{2}}}\ensuremath{)} \HOLSymConst{\HOLTokenConj{}}
            \HOLSymConst{\HOLTokenForall{}}\HOLBoundVar{E\sb{\mathrm{2}}}. \HOLBoundVar{E\sp{\prime}} \HOLTokenTransBegin\HOLBoundVar{u}\HOLTokenTransEnd \HOLBoundVar{E\sb{\mathrm{2}}} \HOLSymConst{\HOLTokenImp{}} \HOLSymConst{\HOLTokenExists{}}\HOLBoundVar{E\sb{\mathrm{1}}}. \HOLBoundVar{E} \HOLTokenWeakTransBegin\HOLBoundVar{u}\HOLTokenWeakTransEnd \HOLBoundVar{E\sb{\mathrm{1}}} \HOLSymConst{\HOLTokenConj{}} \HOLFreeVar{Wbsm} \HOLBoundVar{E\sb{\mathrm{1}}} \HOLBoundVar{E\sb{\mathrm{2}}}\ensuremath{)} \HOLSymConst{\HOLTokenImp{}}
       \HOLBoundVar{E} \HOLSymConst{\HOLTokenObsCongr} \HOLBoundVar{E\sp{\prime}}\hfill{[OBS_CONGR_BY_WEAK_BISIM]}
\end{alltt}

Both strong bisimilarity ($\sim$) and
rooted bisimilarity ($\approx^c$) are  congruence relations:
\begin{alltt}
\HOLTokenTurnstile{} \HOLConst{congruence} \HOLConst{STRONG_EQUIV}\hfill{[STRONG_EQUIV_congruence]}
\HOLTokenTurnstile{} \HOLConst{congruence} \HOLConst{OBS_CONGR}\hfill{[OBS_CONGR_congruence]}
\end{alltt}

Although weak bisimilarity ($\approx$) is \emph{not} a congruence
  with respect to~\texttt{CONTEXT}, it is indeed substitutive
  with respect to~\texttt{GCONTEXT} 
% (or if the CCS syntax were defined with
%   only guarded sum operator \cite{sangiorgi2015equations}) 
as $\approx$ is indeed preserved by guarded sums.

On the relationship between (weak) bisimilarity and rooted bisimilarity, 
we have proved \hl{Deng's Lemma and Hennessy's Lemma}
(Lemma 4.1 and 4.2 of~\citep[p.~176,~178]{Gorrieri:2015jt}):
\begin{alltt}
\HOLTokenTurnstile{} \HOLFreeVar{p} \HOLSymConst{\HOLTokenWeakEQ} \HOLFreeVar{q} \HOLSymConst{\HOLTokenImp{}} \ensuremath{(}\HOLSymConst{\HOLTokenExists{}}\HOLBoundVar{p\sp{\prime}}. \HOLFreeVar{p} \HOLTokenTransBegin\HOLSymConst{\ensuremath{\tau}}\HOLTokenTransEnd \HOLBoundVar{p\sp{\prime}} \HOLSymConst{\HOLTokenConj{}} \HOLBoundVar{p\sp{\prime}} \HOLSymConst{\HOLTokenWeakEQ} \HOLFreeVar{q}\ensuremath{)} \HOLSymConst{\HOLTokenDisj{}} \ensuremath{(}\HOLSymConst{\HOLTokenExists{}}\HOLBoundVar{q\sp{\prime}}. \HOLFreeVar{q} \HOLTokenTransBegin\HOLSymConst{\ensuremath{\tau}}\HOLTokenTransEnd \HOLBoundVar{q\sp{\prime}} \HOLSymConst{\HOLTokenConj{}} \HOLFreeVar{p} \HOLSymConst{\HOLTokenWeakEQ} \HOLBoundVar{q\sp{\prime}}\ensuremath{)} \HOLSymConst{\HOLTokenDisj{}} \HOLFreeVar{p} \HOLSymConst{\HOLTokenObsCongr} \HOLFreeVar{q}\hfill{[DENG_LEMMA]}
  
\HOLTokenTurnstile{} \HOLFreeVar{p} \HOLSymConst{\HOLTokenWeakEQ} \HOLFreeVar{q} \HOLSymConst{\HOLTokenEquiv{}} \HOLFreeVar{p} \HOLSymConst{\HOLTokenObsCongr} \HOLFreeVar{q} \HOLSymConst{\HOLTokenDisj{}} \HOLFreeVar{p} \HOLSymConst{\HOLTokenObsCongr} \HOLSymConst{\ensuremath{\tau}}\HOLSymConst{\ensuremath{\ldotp}}\HOLFreeVar{q} \HOLSymConst{\HOLTokenDisj{}} \HOLSymConst{\ensuremath{\tau}}\HOLSymConst{\ensuremath{\ldotp}}\HOLFreeVar{p} \HOLSymConst{\HOLTokenObsCongr} \HOLFreeVar{q}\hfill{[HENNESSY_LEMMA]}
\end{alltt}

\subsection{Algebraic Laws}

Having formalised  the definitions of strong bisimulation and strong bisimilarity,
we can derive  \emph{algebraic laws} for the 
 bisimilarities. We only report a few laws for 
 the sum operator:
\begin{alltt}
STRONG_SUM_IDEMP:          \HOLTokenTurnstile{} \HOLFreeVar{E} \HOLSymConst{\ensuremath{+}} \HOLFreeVar{E} \HOLSymConst{\HOLTokenStrongEQ} \HOLFreeVar{E}  
STRONG_SUM_COMM:           \HOLTokenTurnstile{} \HOLFreeVar{E} \HOLSymConst{\ensuremath{+}} \HOLFreeVar{E\sp{\prime}} \HOLSymConst{\HOLTokenStrongEQ} \HOLFreeVar{E\sp{\prime}} \HOLSymConst{\ensuremath{+}} \HOLFreeVar{E}
STRONG_SUM_IDENT_L:        \HOLTokenTurnstile{} \HOLConst{\ensuremath{\mathbf{0}}} \HOLSymConst{\ensuremath{+}} \HOLFreeVar{E} \HOLSymConst{\HOLTokenStrongEQ} \HOLFreeVar{E}
STRONG_SUM_IDENT_R:        \HOLTokenTurnstile{} \HOLFreeVar{E} \HOLSymConst{\ensuremath{+}} \HOLConst{\ensuremath{\mathbf{0}}} \HOLSymConst{\HOLTokenStrongEQ} \HOLFreeVar{E}
STRONG_SUM_ASSOC_R:        \HOLTokenTurnstile{} \HOLFreeVar{E} \HOLSymConst{\ensuremath{+}} \HOLFreeVar{E\sp{\prime}} \HOLSymConst{\ensuremath{+}} \HOLFreeVar{E\sp{\prime\prime}} \HOLSymConst{\HOLTokenStrongEQ} \HOLFreeVar{E} \HOLSymConst{\ensuremath{+}} \ensuremath{(}\HOLFreeVar{E\sp{\prime}} \HOLSymConst{\ensuremath{+}} \HOLFreeVar{E\sp{\prime\prime}}\ensuremath{)}
STRONG_SUM_ASSOC_L:        \HOLTokenTurnstile{} \HOLFreeVar{E} \HOLSymConst{\ensuremath{+}} \ensuremath{(}\HOLFreeVar{E\sp{\prime}} \HOLSymConst{\ensuremath{+}} \HOLFreeVar{E\sp{\prime\prime}}\ensuremath{)} \HOLSymConst{\HOLTokenStrongEQ} \HOLFreeVar{E} \HOLSymConst{\ensuremath{+}} \HOLFreeVar{E\sp{\prime}} \HOLSymConst{\ensuremath{+}} \HOLFreeVar{E\sp{\prime\prime}}
STRONG_SUM_MID_IDEMP:      \HOLTokenTurnstile{} \HOLFreeVar{E} \HOLSymConst{\ensuremath{+}} \HOLFreeVar{E\sp{\prime}} \HOLSymConst{\ensuremath{+}} \HOLFreeVar{E} \HOLSymConst{\HOLTokenStrongEQ} \HOLFreeVar{E\sp{\prime}} \HOLSymConst{\ensuremath{+}} \HOLFreeVar{E}
STRONG_LEFT_SUM_MID_IDEMP: \HOLTokenTurnstile{} \HOLFreeVar{E} \HOLSymConst{\ensuremath{+}} \HOLFreeVar{E\sp{\prime}} \HOLSymConst{\ensuremath{+}} \HOLFreeVar{E\sp{\prime\prime}} \HOLSymConst{\ensuremath{+}} \HOLFreeVar{E\sp{\prime}} \HOLSymConst{\HOLTokenStrongEQ} \HOLFreeVar{E} \HOLSymConst{\ensuremath{+}} \HOLFreeVar{E\sp{\prime\prime}} \HOLSymConst{\ensuremath{+}} \HOLFreeVar{E\sp{\prime}}
\end{alltt}

% Not all above theorems are primitive (in the sense of providing a
% minimal axiomatization set for proving all other strong algebraic
% laws). 
The
first 5 laws are proved
 by constructing bisimulation
relations, and their formal proofs are written in
a goal-directed manner. Instead, the
last 3 laws are derived in a forward manner by applications of
previous proven laws (without directly using the SOS
inference rules and the definition of bisimulation).
 
\hl{These algebraic laws also hold for weak bisimilarity and rooted
  bisimilarity, as these are coarser than strong bisimilarity.}

\subsection{Expansion, Contraction and Rooted Contraction}

To formally define bisimulation expansion and contraction (and their preorders), we have
followed the same ways as in the case of strong and weak bisimilarities:
\begin{alltt}
\HOLConst{EXPANSION} \HOLFreeVar{Exp} \HOLSymConst{\HOLTokenDefEquality{}}
  \HOLSymConst{\HOLTokenForall{}}\HOLBoundVar{E} \HOLBoundVar{E\sp{\prime}}.
      \HOLFreeVar{Exp} \HOLBoundVar{E} \HOLBoundVar{E\sp{\prime}} \HOLSymConst{\HOLTokenImp{}}
      \ensuremath{(}\HOLSymConst{\HOLTokenForall{}}\HOLBoundVar{l}.
           \ensuremath{(}\HOLSymConst{\HOLTokenForall{}}\HOLBoundVar{E\sb{\mathrm{1}}}. \HOLBoundVar{E} \HOLTokenTransBegin\HOLConst{label} \HOLBoundVar{l}\HOLTokenTransEnd \HOLBoundVar{E\sb{\mathrm{1}}} \HOLSymConst{\HOLTokenImp{}} \HOLSymConst{\HOLTokenExists{}}\HOLBoundVar{E\sb{\mathrm{2}}}. \HOLBoundVar{E\sp{\prime}} \HOLTokenTransBegin\HOLConst{label} \HOLBoundVar{l}\HOLTokenTransEnd \HOLBoundVar{E\sb{\mathrm{2}}} \HOLSymConst{\HOLTokenConj{}} \HOLFreeVar{Exp} \HOLBoundVar{E\sb{\mathrm{1}}} \HOLBoundVar{E\sb{\mathrm{2}}}\ensuremath{)} \HOLSymConst{\HOLTokenConj{}}
           \HOLSymConst{\HOLTokenForall{}}\HOLBoundVar{E\sb{\mathrm{2}}}. \HOLBoundVar{E\sp{\prime}} \HOLTokenTransBegin\HOLConst{label} \HOLBoundVar{l}\HOLTokenTransEnd \HOLBoundVar{E\sb{\mathrm{2}}} \HOLSymConst{\HOLTokenImp{}} \HOLSymConst{\HOLTokenExists{}}\HOLBoundVar{E\sb{\mathrm{1}}}. \HOLBoundVar{E} \HOLTokenWeakTransBegin\HOLConst{label} \HOLBoundVar{l}\HOLTokenWeakTransEnd \HOLBoundVar{E\sb{\mathrm{1}}} \HOLSymConst{\HOLTokenConj{}} \HOLFreeVar{Exp} \HOLBoundVar{E\sb{\mathrm{1}}} \HOLBoundVar{E\sb{\mathrm{2}}}\ensuremath{)} \HOLSymConst{\HOLTokenConj{}}
      \ensuremath{(}\HOLSymConst{\HOLTokenForall{}}\HOLBoundVar{E\sb{\mathrm{1}}}. \HOLBoundVar{E} \HOLTokenTransBegin\HOLSymConst{\ensuremath{\tau}}\HOLTokenTransEnd \HOLBoundVar{E\sb{\mathrm{1}}} \HOLSymConst{\HOLTokenImp{}} \HOLFreeVar{Exp} \HOLBoundVar{E\sb{\mathrm{1}}} \HOLBoundVar{E\sp{\prime}} \HOLSymConst{\HOLTokenDisj{}} \HOLSymConst{\HOLTokenExists{}}\HOLBoundVar{E\sb{\mathrm{2}}}. \HOLBoundVar{E\sp{\prime}} \HOLTokenTransBegin\HOLSymConst{\ensuremath{\tau}}\HOLTokenTransEnd \HOLBoundVar{E\sb{\mathrm{2}}} \HOLSymConst{\HOLTokenConj{}} \HOLFreeVar{Exp} \HOLBoundVar{E\sb{\mathrm{1}}} \HOLBoundVar{E\sb{\mathrm{2}}}\ensuremath{)} \HOLSymConst{\HOLTokenConj{}}
      \HOLSymConst{\HOLTokenForall{}}\HOLBoundVar{E\sb{\mathrm{2}}}. \HOLBoundVar{E\sp{\prime}} \HOLTokenTransBegin\HOLSymConst{\ensuremath{\tau}}\HOLTokenTransEnd \HOLBoundVar{E\sb{\mathrm{2}}} \HOLSymConst{\HOLTokenImp{}} \HOLSymConst{\HOLTokenExists{}}\HOLBoundVar{E\sb{\mathrm{1}}}. \HOLBoundVar{E} \HOLTokenWeakTransBegin\HOLSymConst{\ensuremath{\tau}}\HOLTokenWeakTransEnd \HOLBoundVar{E\sb{\mathrm{1}}} \HOLSymConst{\HOLTokenConj{}} \HOLFreeVar{Exp} \HOLBoundVar{E\sb{\mathrm{1}}} \HOLBoundVar{E\sb{\mathrm{2}}}\hfill{[EXPANSION]}

\HOLTokenTurnstile{} \HOLFreeVar{P} \HOLSymConst{\HOLTokenExpands{}} \HOLFreeVar{Q} \HOLSymConst{\HOLTokenEquiv{}} \HOLSymConst{\HOLTokenExists{}}\HOLBoundVar{Exp}. \HOLBoundVar{Exp} \HOLFreeVar{P} \HOLFreeVar{Q} \HOLSymConst{\HOLTokenConj{}} \HOLConst{EXPANSION} \HOLBoundVar{Exp}\hfill{[expands_thm]}
\end{alltt}

\begin{alltt}
\HOLConst{CONTRACTION} \HOLFreeVar{Con} \HOLSymConst{\HOLTokenDefEquality{}}
  \HOLSymConst{\HOLTokenForall{}}\HOLBoundVar{E} \HOLBoundVar{E\sp{\prime}}.
      \HOLFreeVar{Con} \HOLBoundVar{E} \HOLBoundVar{E\sp{\prime}} \HOLSymConst{\HOLTokenImp{}}
      \ensuremath{(}\HOLSymConst{\HOLTokenForall{}}\HOLBoundVar{l}.
           \ensuremath{(}\HOLSymConst{\HOLTokenForall{}}\HOLBoundVar{E\sb{\mathrm{1}}}. \HOLBoundVar{E} \HOLTokenTransBegin\HOLConst{label} \HOLBoundVar{l}\HOLTokenTransEnd \HOLBoundVar{E\sb{\mathrm{1}}} \HOLSymConst{\HOLTokenImp{}} \HOLSymConst{\HOLTokenExists{}}\HOLBoundVar{E\sb{\mathrm{2}}}. \HOLBoundVar{E\sp{\prime}} \HOLTokenTransBegin\HOLConst{label} \HOLBoundVar{l}\HOLTokenTransEnd \HOLBoundVar{E\sb{\mathrm{2}}} \HOLSymConst{\HOLTokenConj{}} \HOLFreeVar{Con} \HOLBoundVar{E\sb{\mathrm{1}}} \HOLBoundVar{E\sb{\mathrm{2}}}\ensuremath{)} \HOLSymConst{\HOLTokenConj{}}
           \HOLSymConst{\HOLTokenForall{}}\HOLBoundVar{E\sb{\mathrm{2}}}. \HOLBoundVar{E\sp{\prime}} \HOLTokenTransBegin\HOLConst{label} \HOLBoundVar{l}\HOLTokenTransEnd \HOLBoundVar{E\sb{\mathrm{2}}} \HOLSymConst{\HOLTokenImp{}} \HOLSymConst{\HOLTokenExists{}}\HOLBoundVar{E\sb{\mathrm{1}}}. \HOLBoundVar{E} \HOLTokenWeakTransBegin\HOLConst{label} \HOLBoundVar{l}\HOLTokenWeakTransEnd \HOLBoundVar{E\sb{\mathrm{1}}} \HOLSymConst{\HOLTokenConj{}} \HOLBoundVar{E\sb{\mathrm{1}}} \HOLSymConst{\HOLTokenWeakEQ} \HOLBoundVar{E\sb{\mathrm{2}}}\ensuremath{)} \HOLSymConst{\HOLTokenConj{}}
      \ensuremath{(}\HOLSymConst{\HOLTokenForall{}}\HOLBoundVar{E\sb{\mathrm{1}}}. \HOLBoundVar{E} \HOLTokenTransBegin\HOLSymConst{\ensuremath{\tau}}\HOLTokenTransEnd \HOLBoundVar{E\sb{\mathrm{1}}} \HOLSymConst{\HOLTokenImp{}} \HOLFreeVar{Con} \HOLBoundVar{E\sb{\mathrm{1}}} \HOLBoundVar{E\sp{\prime}} \HOLSymConst{\HOLTokenDisj{}} \HOLSymConst{\HOLTokenExists{}}\HOLBoundVar{E\sb{\mathrm{2}}}. \HOLBoundVar{E\sp{\prime}} \HOLTokenTransBegin\HOLSymConst{\ensuremath{\tau}}\HOLTokenTransEnd \HOLBoundVar{E\sb{\mathrm{2}}} \HOLSymConst{\HOLTokenConj{}} \HOLFreeVar{Con} \HOLBoundVar{E\sb{\mathrm{1}}} \HOLBoundVar{E\sb{\mathrm{2}}}\ensuremath{)} \HOLSymConst{\HOLTokenConj{}}
      \HOLSymConst{\HOLTokenForall{}}\HOLBoundVar{E\sb{\mathrm{2}}}. \HOLBoundVar{E\sp{\prime}} \HOLTokenTransBegin\HOLSymConst{\ensuremath{\tau}}\HOLTokenTransEnd \HOLBoundVar{E\sb{\mathrm{2}}} \HOLSymConst{\HOLTokenImp{}} \HOLSymConst{\HOLTokenExists{}}\HOLBoundVar{E\sb{\mathrm{1}}}. \HOLBoundVar{E} \HOLSymConst{\HOLTokenEPS} \HOLBoundVar{E\sb{\mathrm{1}}} \HOLSymConst{\HOLTokenConj{}} \HOLBoundVar{E\sb{\mathrm{1}}} \HOLSymConst{\HOLTokenWeakEQ} \HOLBoundVar{E\sb{\mathrm{2}}}\hfill{[CONTRACTION]}

\HOLTokenTurnstile{} \HOLFreeVar{P} \HOLSymConst{\HOLTokenContracts{}} \HOLFreeVar{Q} \HOLSymConst{\HOLTokenEquiv{}} \HOLSymConst{\HOLTokenExists{}}\HOLBoundVar{Con}. \HOLBoundVar{Con} \HOLFreeVar{P} \HOLFreeVar{Q} \HOLSymConst{\HOLTokenConj{}} \HOLConst{CONTRACTION} \HOLBoundVar{Con}\hfill{[contracts_thm]}
\end{alltt}

We can prove that the contraction preorder is contained in weak bisimilarity,
and contains the expansion preorder.
\begin{proposition}{(Relationships between contraction, expansion and weak
    bisimilarity)}
\begin{enumerate}
\item (`expansion' implies `contraction')
\begin{alltt}
\HOLTokenTurnstile{} \HOLFreeVar{P} \HOLSymConst{\HOLTokenExpands{}} \HOLFreeVar{Q} \HOLSymConst{\HOLTokenImp{}} \HOLFreeVar{P} \HOLSymConst{\HOLTokenContracts{}} \HOLFreeVar{Q}\hfill[expands_IMP_contracts]
\end{alltt}
\item (`contraction' implies weak  bisimilarity)
\begin{alltt}
\HOLTokenTurnstile{} \HOLFreeVar{P} \HOLSymConst{\HOLTokenContracts{}} \HOLFreeVar{Q} \HOLSymConst{\HOLTokenImp{}} \HOLFreeVar{P} \HOLSymConst{\HOLTokenWeakEQ} \HOLFreeVar{Q}\hfill[contracts_IMP_WEAK_EQUIV]
\end{alltt}
\end{enumerate}
\end{proposition}
In general a proof of a property for the contraction \hl{(and the contraction preorder)} is
harder than that for the cases of expansion: \hl{this is mostly due to the surprising fact
that, although the contraction preorder is contained in weak
bisimilarity, contraction does not imply weak bisimulation (while
expansion does so),}
i.e. the following proposition does not hold:
\begin{alltt}
\HOLinline{\HOLSymConst{\HOLTokenForall{}}\HOLBoundVar{Con}. \HOLConst{CONTRACTION} \HOLBoundVar{Con} \HOLSymConst{\HOLTokenImp{}} \HOLConst{WEAK_BISIM} \HOLBoundVar{Con}}
\end{alltt}
\hl{For instance, in the proof of the following lemma (contraction preorder implies $\wb$)}:
\begin{alltt}
\HOLTokenTurnstile{} \HOLFreeVar{P} \HOLSymConst{\HOLTokenContracts{}} \HOLFreeVar{Q} \HOLSymConst{\HOLTokenImp{}} \HOLFreeVar{P} \HOLSymConst{\HOLTokenWeakEQ} \HOLFreeVar{Q}
\end{alltt}
We prove it by constructing a bisimulation containing two processes
$P$ and $Q$, given that they're in a contraction $Con$:
\begin{lstlisting}
        ?Wbsm. Wbsm P Q /\ WEAK_BISIM Wbsm
   ------------------------------------
    0.  Con P Q
    1.  CONTRACTION Con
   
   : proof
\end{lstlisting}
We cannot show that $Con$ itself is a bisimulation, but rather
that the union of $Con$ and $\wb$ is a bisimulation.
\hl{The corresponding lemma for the expansion preorder is rather
straightforward (just use $Con$)
and does not require any non-trivial construction.}

The rooted contraction ($\rcontr$, \texttt{OBS_contracts}) is formally
defined as follows, with its precongruence result:
\begin{alltt}
\HOLFreeVar{E} \HOLSymConst{\HOLTokenObsContracts} \HOLFreeVar{E\sp{\prime}} \HOLSymConst{\HOLTokenDefEquality{}}
  \HOLSymConst{\HOLTokenForall{}}\HOLBoundVar{u}.
      \ensuremath{(}\HOLSymConst{\HOLTokenForall{}}\HOLBoundVar{E\sb{\mathrm{1}}}. \HOLFreeVar{E} \HOLTokenTransBegin\HOLBoundVar{u}\HOLTokenTransEnd \HOLBoundVar{E\sb{\mathrm{1}}} \HOLSymConst{\HOLTokenImp{}} \HOLSymConst{\HOLTokenExists{}}\HOLBoundVar{E\sb{\mathrm{2}}}. \HOLFreeVar{E\sp{\prime}} \HOLTokenTransBegin\HOLBoundVar{u}\HOLTokenTransEnd \HOLBoundVar{E\sb{\mathrm{2}}} \HOLSymConst{\HOLTokenConj{}} \HOLBoundVar{E\sb{\mathrm{1}}} \HOLSymConst{\HOLTokenContracts{}} \HOLBoundVar{E\sb{\mathrm{2}}}\ensuremath{)} \HOLSymConst{\HOLTokenConj{}}
      \HOLSymConst{\HOLTokenForall{}}\HOLBoundVar{E\sb{\mathrm{2}}}. \HOLFreeVar{E\sp{\prime}} \HOLTokenTransBegin\HOLBoundVar{u}\HOLTokenTransEnd \HOLBoundVar{E\sb{\mathrm{2}}} \HOLSymConst{\HOLTokenImp{}} \HOLSymConst{\HOLTokenExists{}}\HOLBoundVar{E\sb{\mathrm{1}}}. \HOLFreeVar{E} \HOLTokenWeakTransBegin\HOLBoundVar{u}\HOLTokenWeakTransEnd \HOLBoundVar{E\sb{\mathrm{1}}} \HOLSymConst{\HOLTokenConj{}} \HOLBoundVar{E\sb{\mathrm{1}}} \HOLSymConst{\HOLTokenWeakEQ} \HOLBoundVar{E\sb{\mathrm{2}}}\hfill{[OBS_contracts]}

\HOLTokenTurnstile{} \HOLConst{precongruence} \HOLConst{OBS_contracts}\hfill{[OBS_contracts_precongruence]}
\end{alltt}

%% TODO: move upto techniques inside "unique solution" theorems

\subsection{The formalisation of ``bisimulation up to bisimilarity''}

``Bisimulation up to'' is a family of  powerful proof techniques,
used to reduce the size of a relation needed to define a bisimulation.
By definition, two processes are bisimilar if there exists a
bisimulation relation containing them as a pair. However, in practice
this definition is hardly ever followed plainly; instead, to reduce
the size of the relations exhibited one prefers to define relations
which are bisimulations only when closed up under some specific and
privileged relation, so to relieve the proof work needed. These are  called
 an \emph{``up-to'' techniques}. 
% It is a pretty general device
% which allows a great variety of prssibilities.


% Following Milner \cite{Mil89}, the concept of ``bisimulation up to
% $\sim$''  begins  with a generalization of the notion of strong
% bisimulation, which is often more useful in applications.
% The following definition and proposition put the idea on a firm
% basis. 
We recall that we often write \hl{$P \;\R\; Q$ to denote}
$(P, Q) \in \R$ for any binary relation $\R$. 
Moreover, 
 $\sim \mathcal{S} \sim$ is the composition of \hl{three binary
relations: $\sim$, $\S$ and $\sim$.} Hence $P \sim \S \sim Q$ means that,
\hl{there exist $P'$ and $Q'$ such that} $P \sim P'$, $P' \;\S\; Q'$ and $Q' \sim Q$.
\begin{definition}[Bisimulation up to $\sim$]
  \label{def:bisimUptoSim}
$\mathcal{S}$ is a ``\emph{bisimulation up to $\sim$}'' if $P
  \mathcal{S} Q$ implies, for all $\mu$,
\begin{enumerate}
\item Whenever $P \overset{\mu}{\rightarrow} P'$ then, for some
  $Q'$, $Q \overset{\mu}{\rightarrow} Q'$ and $P' \sim \mathcal{S}
  \sim Q'$,
\item Whenever $Q \overset{\mu}{\rightarrow} Q'$ then, for some
  $P'$, $P \overset{\mu}{\rightarrow} P'$ and $P' \sim \mathcal{S}
  \sim Q'$.
\end{enumerate}
\end{definition}

\begin{theorem}
If $\mathcal{S}$ is a ``bisimulation up to $\sim$'', then
$\mathcal{S} \subseteq\;\sim$:
\begin{alltt}
\HOLTokenTurnstile{} \HOLConst{STRONG_BISIM_UPTO} \HOLFreeVar{Bsm} \HOLSymConst{\HOLTokenConj{}} \HOLFreeVar{Bsm} \HOLFreeVar{P} \HOLFreeVar{Q} \HOLSymConst{\HOLTokenImp{}} \HOLFreeVar{P} \HOLSymConst{\HOLTokenStrongEQ} \HOLFreeVar{Q}\hfill{[STRONG_EQUIV_BY_BISIM_UPTO]}
\end{alltt}
\end{theorem}
Hence, to prove $P \sim Q$, one only needs to find a bisimulation
up to $\sim$ that contains $(P, Q)$.

For weak bisimilarity, \hl{the \emph{naive} weak bisimulation up to weak bisimilarity
is unsound:} if one simply replaces all $\sim$ in
Def.~\ref{def:bisimUptoSim} with $\wb$, the resulting ``weak
bisimulation up`` is not contained in $\wb$.~\cite{sangiorgi1992problem}
\hl{There are a few ways to fix this problem, one is the following:}

\begin{definition}{(Bisimulation up to $\approx$)}
$\mathcal{S}$ is a ``\emph{bisimulation up to $\approx$}'' if $P\;
  \mathcal{S}\; Q$ implies, for all $\mu$,
\begin{enumerate}
\item Whenever $P \arr{\mu} P'$ then, for some
  $Q'$, $Q \Arcap{\mu} Q'$ and $P' \sim \S \approx Q'$,
\item Whenever $Q \arr{\mu} Q'$ then, for some
  $P'$, $P \Arcap{\mu} P'$ and $P' \approx \S \sim Q'$.
\end{enumerate}
or formally (for illustrating purposes below),
\begin{alltt}
    \HOLConst{WEAK_BISIM_UPTO} \HOLFreeVar{Wbsm} \HOLSymConst{\HOLTokenDefEquality{}}
      \HOLSymConst{\HOLTokenForall{}}\HOLBoundVar{E} \HOLBoundVar{E\sp{\prime}}.
          \HOLFreeVar{Wbsm} \HOLBoundVar{E} \HOLBoundVar{E\sp{\prime}} \HOLSymConst{\HOLTokenImp{}}
          \ensuremath{(}\HOLSymConst{\HOLTokenForall{}}\HOLBoundVar{l}.
               \ensuremath{(}\HOLSymConst{\HOLTokenForall{}}\HOLBoundVar{E\sb{\mathrm{1}}}.
                    \HOLBoundVar{E} \HOLTokenTransBegin\HOLConst{label} \HOLBoundVar{l}\HOLTokenTransEnd \HOLBoundVar{E\sb{\mathrm{1}}} \HOLSymConst{\HOLTokenImp{}}
                    \HOLSymConst{\HOLTokenExists{}}\HOLBoundVar{E\sb{\mathrm{2}}}.
                        \HOLBoundVar{E\sp{\prime}} \HOLTokenWeakTransBegin\HOLConst{label} \HOLBoundVar{l}\HOLTokenWeakTransEnd \HOLBoundVar{E\sb{\mathrm{2}}} \HOLSymConst{\HOLTokenConj{}}
                        \ensuremath{(}\HOLConst{WEAK_EQUIV} \HOLSymConst{\HOLTokenRCompose{}} \HOLFreeVar{Wbsm} \HOLSymConst{\HOLTokenRCompose{}} \HOLConst{STRONG_EQUIV}\ensuremath{)} \HOLBoundVar{E\sb{\mathrm{1}}} \HOLBoundVar{E\sb{\mathrm{2}}}\ensuremath{)} \HOLSymConst{\HOLTokenConj{}}
               \HOLSymConst{\HOLTokenForall{}}\HOLBoundVar{E\sb{\mathrm{2}}}.
                   \HOLBoundVar{E\sp{\prime}} \HOLTokenTransBegin\HOLConst{label} \HOLBoundVar{l}\HOLTokenTransEnd \HOLBoundVar{E\sb{\mathrm{2}}} \HOLSymConst{\HOLTokenImp{}}
                   \HOLSymConst{\HOLTokenExists{}}\HOLBoundVar{E\sb{\mathrm{1}}}.
                       \HOLBoundVar{E} \HOLTokenWeakTransBegin\HOLConst{label} \HOLBoundVar{l}\HOLTokenWeakTransEnd \HOLBoundVar{E\sb{\mathrm{1}}} \HOLSymConst{\HOLTokenConj{}}
                       \ensuremath{(}\HOLConst{STRONG_EQUIV} \HOLSymConst{\HOLTokenRCompose{}} \HOLFreeVar{Wbsm} \HOLSymConst{\HOLTokenRCompose{}} \HOLConst{WEAK_EQUIV}\ensuremath{)} \HOLBoundVar{E\sb{\mathrm{1}}} \HOLBoundVar{E\sb{\mathrm{2}}}\ensuremath{)} \HOLSymConst{\HOLTokenConj{}}
          \ensuremath{(}\HOLSymConst{\HOLTokenForall{}}\HOLBoundVar{E\sb{\mathrm{1}}}.
               \HOLBoundVar{E} \HOLTokenTransBegin\HOLSymConst{\ensuremath{\tau}}\HOLTokenTransEnd \HOLBoundVar{E\sb{\mathrm{1}}} \HOLSymConst{\HOLTokenImp{}}
               \HOLSymConst{\HOLTokenExists{}}\HOLBoundVar{E\sb{\mathrm{2}}}. \HOLBoundVar{E\sp{\prime}} \HOLSymConst{\HOLTokenEPS} \HOLBoundVar{E\sb{\mathrm{2}}} \HOLSymConst{\HOLTokenConj{}} \ensuremath{(}\HOLConst{WEAK_EQUIV} \HOLSymConst{\HOLTokenRCompose{}} \HOLFreeVar{Wbsm} \HOLSymConst{\HOLTokenRCompose{}} \HOLConst{STRONG_EQUIV}\ensuremath{)} \HOLBoundVar{E\sb{\mathrm{1}}} \HOLBoundVar{E\sb{\mathrm{2}}}\ensuremath{)} \HOLSymConst{\HOLTokenConj{}}
          \HOLSymConst{\HOLTokenForall{}}\HOLBoundVar{E\sb{\mathrm{2}}}.
              \HOLBoundVar{E\sp{\prime}} \HOLTokenTransBegin\HOLSymConst{\ensuremath{\tau}}\HOLTokenTransEnd \HOLBoundVar{E\sb{\mathrm{2}}} \HOLSymConst{\HOLTokenImp{}}
              \HOLSymConst{\HOLTokenExists{}}\HOLBoundVar{E\sb{\mathrm{1}}}. \HOLBoundVar{E} \HOLSymConst{\HOLTokenEPS} \HOLBoundVar{E\sb{\mathrm{1}}} \HOLSymConst{\HOLTokenConj{}} \ensuremath{(}\HOLConst{STRONG_EQUIV} \HOLSymConst{\HOLTokenRCompose{}} \HOLFreeVar{Wbsm} \HOLSymConst{\HOLTokenRCompose{}} \HOLConst{WEAK_EQUIV}\ensuremath{)} \HOLBoundVar{E\sb{\mathrm{1}}} \HOLBoundVar{E\sb{\mathrm{2}}}
\end{alltt}
\end{definition}
\hl{Notice that, the formal represention of $\sim \S \approx$ is}
``\HOLinline{\HOLConst{WEAK_EQUIV} \HOLSymConst{\HOLTokenRCompose{}} \HOLFreeVar{Wbsm} \HOLSymConst{\HOLTokenRCompose{}} \HOLConst{STRONG_EQUIV}}'' where \hl{the order of
$\sim$ and $\approx$ seems reverted.} This is because, in HOL's
notation, the rightmost relation (\HOLinline{\HOLConst{STRONG_EQUIV}} or $\sim$) in the relational composition is
applied first.

\begin{theorem}
If $\mathcal{S}$ is a bisimulation up to $\approx$, then
$\mathcal{S} \subseteq\;\approx$:
\begin{alltt}
\HOLTokenTurnstile{} \HOLConst{WEAK_BISIM_UPTO} \HOLFreeVar{Bsm} \HOLSymConst{\HOLTokenConj{}} \HOLFreeVar{Bsm} \HOLFreeVar{P} \HOLFreeVar{Q} \HOLSymConst{\HOLTokenImp{}} \HOLFreeVar{P} \HOLSymConst{\HOLTokenWeakEQ} \HOLFreeVar{Q}
\end{alltt}
\end{theorem}

%% TODO: move these after ``unique solution of equations'' theorem:
The above version of ``bisimulation up to $\wb$'' 
is not powerful to prove Milner's ``unique solution of equations''
theorem for $\wb$ (c.f.~\cite{sangiorgi1992problem} for more details).
To complete the proof, the following ``double-weak'' version is
actually used:

\begin{definition}{(Bisimulation up to $\approx$, the ``double-weak''
    version)}
  \label{def:doubleweak}
$\mathcal{S}$ is a ``\emph{bisimulation up to $\approx$}'' if $P \;
  \mathcal{S} \; Q$ implies, for all $\mu$,
\begin{enumerate}
\item Whenever $P \overset{\mu}{\rightarrow} P'$ then, for some
  $Q'$, $Q \overset{\hat{\mu}}{\rightarrow} Q'$ and $P' \approx \S \approx Q'$,
\item Whenever $Q \overset{\mu}{\rightarrow} Q'$ then, for some
  $P'$, $P \overset{\hat{\mu}}{\rightarrow} P'$ and $P' \approx \S \approx Q'$.
\end{enumerate}
\end{definition}

\begin{theorem}
\hl{If $\mathcal{S}$ is a bisimulation up to $\approx$} (w.r.t. Definition~\ref{def:doubleweak}), then
$\mathcal{S} \subseteq\;\approx$:
\begin{alltt}
\HOLTokenTurnstile{} \HOLConst{WEAK_BISIM_UPTO_ALT} \HOLFreeVar{Bsm} \HOLSymConst{\HOLTokenConj{}} \HOLFreeVar{Bsm} \HOLFreeVar{P} \HOLFreeVar{Q} \HOLSymConst{\HOLTokenImp{}} \HOLFreeVar{P} \HOLSymConst{\HOLTokenWeakEQ} \HOLFreeVar{Q}
\end{alltt}
\end{theorem}

%  next file: context.htex


%%%% -*- Mode: LaTeX -*-
%%
%% This is the draft of the 2nd part of EXPRESS/SOS 2018 paper, co-authored by
%% Prof. Davide Sangiorgi and Chun Tian.

\subsection{Unique solution of contractions}

The major difficulties in the formalisation of the results about unique solution of
contractions are in the proof of lemma~\ref{l:ruptocon}; in particular, it
uses an induction on the length of weak transitions. 
For this, one could introduced a refined form of weak transition relation
enriched with its length, but such a non-standard relation finds no
other uses beside proving our target theorem. Another way is to use 
traces instead, as it shows more clearly all passing actions inside a
trace, making formal reasoning easier.

We represent a trace by the initial process, the final derivative, and
the list of actions performed. 
To formalise this, 
we first introduce 
the Reflexive Transitive Closure with a
List (LRTC);

A trace is represented by the beginning and ending CCS processes, plus
a list of action it passes. Insteading of defining it directly, we
have first defined a new concept called Reflexive Transitive Closure with a
List (LRTC):
given a  labelled transition relation $R$, \HOLinline{\HOLConst{LRTC}
  \HOLFreeVar{R}} builds 
the possible traces derived from $R$.
\begin{alltt}
\HOLTokenTurnstile{} \HOLConst{LRTC} \HOLFreeVar{R} \HOLFreeVar{a} \HOLFreeVar{l} \HOLFreeVar{b} \HOLSymConst{\HOLTokenEquiv{}}
   \HOLSymConst{\HOLTokenForall{}}\HOLBoundVar{P}.
       (\HOLSymConst{\HOLTokenForall{}}\HOLBoundVar{x}. \HOLBoundVar{P} \HOLBoundVar{x} [] \HOLBoundVar{x}) \HOLSymConst{\HOLTokenConj{}}
       (\HOLSymConst{\HOLTokenForall{}}\HOLBoundVar{x} \HOLBoundVar{h} \HOLBoundVar{y} \HOLBoundVar{t} \HOLBoundVar{z}. \HOLFreeVar{R} \HOLBoundVar{x} \HOLBoundVar{h} \HOLBoundVar{y} \HOLSymConst{\HOLTokenConj{}} \HOLBoundVar{P} \HOLBoundVar{y} \HOLBoundVar{t} \HOLBoundVar{z} \HOLSymConst{\HOLTokenImp{}} \HOLBoundVar{P} \HOLBoundVar{x} (\HOLBoundVar{h}\HOLSymConst{::}\HOLBoundVar{t}) \HOLBoundVar{z}) \HOLSymConst{\HOLTokenImp{}}
       \HOLBoundVar{P} \HOLFreeVar{a} \HOLFreeVar{l} \HOLFreeVar{b}\hfill{[LRTC_DEF]}
\end{alltt}
 Then the traces for the  CCS processes
are obtained
 by combining LRTC with the (strong) CCS transition
relation:
\begin{alltt}
\HOLConst{TRACE} \HOLSymConst{=} \HOLConst{LRTC} \HOLConst{TRANS}\hfill{[TRACE_def]}
\end{alltt}

If there's at most one visible action (label) in the list of actions of a trace,
then the trace is also a weak transition. We divided this observation
into two cases: no label and unique label. The definition of ``no
label'' in an action list is easy and clear (\texttt{MEM} is for
testing if an element is \emph{member} of a list):
\begin{alltt}
\HOLTokenTurnstile{} \HOLConst{NO_LABEL} \HOLFreeVar{L} \HOLSymConst{\HOLTokenEquiv{}} \HOLSymConst{\HOLTokenNeg{}}\HOLSymConst{\HOLTokenExists{}}\HOLBoundVar{l}. \HOLConst{MEM} (\HOLConst{label} \HOLBoundVar{l}) \HOLFreeVar{L}\hfill{[NO_LABEL_def]}
\end{alltt}
while the definition of ``unique label'' can be done in many ways, in
which we have chosen to use the version learnt from Robert Beers in
a private discussion which prevented counting or filtering in the list:
\begin{alltt}
\HOLTokenTurnstile{} \HOLConst{UNIQUE_LABEL} \HOLFreeVar{u} \HOLFreeVar{L} \HOLSymConst{\HOLTokenEquiv{}}
   \HOLSymConst{\HOLTokenExists{}}\HOLBoundVar{L\sb{\mathrm{1}}} \HOLBoundVar{L\sb{\mathrm{2}}}. (\HOLBoundVar{L\sb{\mathrm{1}}} \HOLSymConst{\HOLTokenDoublePlus} [\HOLFreeVar{u}] \HOLSymConst{\HOLTokenDoublePlus} \HOLBoundVar{L\sb{\mathrm{2}}} \HOLSymConst{=} \HOLFreeVar{L}) \HOLSymConst{\HOLTokenConj{}} \HOLConst{NO_LABEL} \HOLBoundVar{L\sb{\mathrm{1}}} \HOLSymConst{\HOLTokenConj{}} \HOLConst{NO_LABEL} \HOLBoundVar{L\sb{\mathrm{2}}}\hfill{[UNIQUE_LABEL_def]}
\end{alltt}
It says, a label is unique in an action list iff it there's no other
labels in the rest part of the list.

The final relationship between traces and weak transitions is stated
and proved in the following lemma: (here $acts$ as a single variable means
a list of actions)
\begin{lemma}
A weak transition $P\overset{u}{\Rightarrow}P'$ is a just trace with non
empty action list: 1) without any visible label, if $u = \tau$, or 2)
$u$ is the unique label in the list, if $u \neq \tau$.
\begin{alltt}
\HOLTokenTurnstile{} \HOLFreeVar{P} \HOLTokenWeakTransBegin\HOLFreeVar{u}\HOLTokenWeakTransEnd \HOLFreeVar{P\sp{\prime}} \HOLSymConst{\HOLTokenEquiv{}}
   \HOLSymConst{\HOLTokenExists{}}\HOLBoundVar{acts}.
       \HOLConst{TRACE} \HOLFreeVar{P} \HOLBoundVar{acts} \HOLFreeVar{P\sp{\prime}} \HOLSymConst{\HOLTokenConj{}} \HOLSymConst{\HOLTokenNeg{}}\HOLConst{NULL} \HOLBoundVar{acts} \HOLSymConst{\HOLTokenConj{}}
       \HOLKeyword{if} \HOLFreeVar{u} \HOLSymConst{=} \HOLSymConst{\ensuremath{\tau}} \HOLKeyword{then} \HOLConst{NO_LABEL} \HOLBoundVar{acts} \HOLKeyword{else} \HOLConst{UNIQUE_LABEL} \HOLFreeVar{u} \HOLBoundVar{acts}\hfill{[WEAK_TRANS_AND_TRACE]}
\end{alltt}
\end{lemma}

To finish the proof of Lemma~\ref{l:ruptocon}, we have used the  above
result to freely switch between weak transitions and traces
during the induction argument.
% , which either keep its
% length or become shorter after passing weak bisimilations, then we
% used above lemma again to convert the traces back to weak transitions.
Below is the formally proved version of
Theorem~\ref{t:contraBisimulationU}: (Here \texttt{WGS} stands for weakly guarded context without direct
sums)
\begin{alltt}
\HOLTokenTurnstile{} \HOLConst{WGS} \HOLFreeVar{E} \HOLSymConst{\HOLTokenImp{}} \HOLSymConst{\HOLTokenForall{}}\HOLBoundVar{P} \HOLBoundVar{Q}. \HOLBoundVar{P} \HOLSymConst{\HOLTokenContracts{}} \HOLFreeVar{E} \HOLBoundVar{P} \HOLSymConst{\HOLTokenConj{}} \HOLBoundVar{Q} \HOLSymConst{\HOLTokenContracts{}} \HOLFreeVar{E} \HOLBoundVar{Q} \HOLSymConst{\HOLTokenImp{}} \HOLBoundVar{P} \HOLSymConst{\HOLTokenWeakEQ} \HOLBoundVar{Q}
\hfill{[UNIQUE_SOLUTION_OF_CONTRACTIONS]}
\end{alltt}

\subsection{Unique solution of rooted contractions}

The formal proof of ``unique solution of rooted contractions theorem''
(Theorem~\ref{t:rcontraBisimulationU}) has the
same initial proof steps as Theorem~\ref{t:contraBisimulationU} plus a
little more steps to handle the rooted bisimilarity at conclusion. The
two proofs are quite similar, mostly because the only property we need
from (rooted) contraction is its precongruence. Once we have proved
the precongruence of rooted contracion, we can naturally use the
normal version of weakly guarded expressions with direct sums
included. Below is the formally verified version of
Theorem~\ref{t:rcontraBisimulationU}:
\begin{alltt}
\HOLTokenTurnstile{} \HOLConst{WG} \HOLFreeVar{E} \HOLSymConst{\HOLTokenImp{}} \HOLSymConst{\HOLTokenForall{}}\HOLBoundVar{P} \HOLBoundVar{Q}. \HOLBoundVar{P} \HOLSymConst{\HOLTokenObsContracts} \HOLFreeVar{E} \HOLBoundVar{P} \HOLSymConst{\HOLTokenConj{}} \HOLBoundVar{Q} \HOLSymConst{\HOLTokenObsContracts} \HOLFreeVar{E} \HOLBoundVar{Q} \HOLSymConst{\HOLTokenImp{}} \HOLBoundVar{P} \HOLSymConst{\HOLTokenObsCongr} \HOLBoundVar{Q}
\hfill{[UNIQUE_SOLUTION_OF_ROOTED_CONTRACTIONS]}
\end{alltt}

See how similar it is with Milner's original ``unique solution of
equations'' theorem for strong bisimilarity ($\sim$) in which the same
weak guardness (\texttt{WG}) is required:
\begin{alltt}
\HOLTokenTurnstile{} \HOLConst{WG} \HOLFreeVar{E} \HOLSymConst{\HOLTokenImp{}} \HOLSymConst{\HOLTokenForall{}}\HOLBoundVar{P} \HOLBoundVar{Q}. \HOLBoundVar{P} \HOLSymConst{\HOLTokenStrongEQ} \HOLFreeVar{E} \HOLBoundVar{P} \HOLSymConst{\HOLTokenConj{}} \HOLBoundVar{Q} \HOLSymConst{\HOLTokenStrongEQ} \HOLFreeVar{E} \HOLBoundVar{Q} \HOLSymConst{\HOLTokenImp{}} \HOLBoundVar{P} \HOLSymConst{\HOLTokenStrongEQ} \HOLBoundVar{Q}\hfill{[STRONG_UNIQUE_SOLUTION]}
\end{alltt}

Or our Theorem~\ref{t:rcontraBisimulationU} can be seen as a more
applicable version of Milner's ``unique solution of
equations'' theorem for rooted bisimilarity ($\rapprox$), which has more
restrictions on equations:
\begin{alltt}
\HOLTokenTurnstile{} \HOLConst{SG} \HOLFreeVar{E} \HOLSymConst{\HOLTokenConj{}} \HOLConst{SEQ} \HOLFreeVar{E} \HOLSymConst{\HOLTokenImp{}} \HOLSymConst{\HOLTokenForall{}}\HOLBoundVar{P} \HOLBoundVar{Q}. \HOLBoundVar{P} \HOLSymConst{\HOLTokenObsCongr} \HOLFreeVar{E} \HOLBoundVar{P} \HOLSymConst{\HOLTokenConj{}} \HOLBoundVar{Q} \HOLSymConst{\HOLTokenObsCongr} \HOLFreeVar{E} \HOLBoundVar{Q} \HOLSymConst{\HOLTokenImp{}} \HOLBoundVar{P} \HOLSymConst{\HOLTokenObsCongr} \HOLBoundVar{Q}
\hfill{[OBS_UNIQUE_SOLUTION]}
\end{alltt}


\section{Related work on formalisation}
\label{s:rel}

Monica Nesi did the first CCS formalisations for both pure and
value-passing CCS \cite{Nesi:1992ve,Nesi:2017wo} using early versions of the HOL
theorem prover.\footnote{Part of this work can now be found at
  \url{https://github.com/binghe/HOL-CCS/tree/master/CCS-Nesi}.}
Her main focus was on implementing decision procedures (as a ML
program, e.g.~\cite{cleaveland1993concurrency}) for
automatically proving bisimilarities of CCS processes.
Her work has been a basis for ours~\cite{Tian:2017wrba}.
However, the differences are substantial, especially in the way of defining
bisimilarities. We greatly benefited from \hl{new} features and standard
libraries in recent versions of HOL4, and our formalisation has
covered a  larger spectrum of the (pure) CCS theory.

Bengtson, Parrow and Weber did a substantial formalisation work
on CCS, $\pi$-calculi
and $\psi$-calculi 
using Isabelle/HOL and its nominal logic, with the main focus on the handling of
name binders \cite{bengtson2007completeness,parrow2009formalising}.
More details can be found in Bengtson's PhD thesis~\cite{bengtson2010formalising}. For CCS, 
he has formalized basic properties for strong/weak equivalence (congruence, basic
 algebraic laws); \hl{the CCS syntax does not have constants
or recursion, using instead replication.}
% The formalisation effort has then been continued 
%
% On the other side, Jesper
% Bengtson and Joachim Parrow made great progress on $\pi$-calculus
% formalization and 
% proved that the algebraic axiomatization of bisimulation
% equivalence in the $\pi$-calculi is sound and
% complete. \cite{bengtson2007completeness} 
Other formalisations in this area include the early work of T.F.~Melham
\cite{melham1994mechanized} and O.A.~Mohamed
\cite{mohamed1995mechanizing} in HOL, Compton
\cite{compton2005embedding} in Isabelle/HOL,
Solange\footnote{\url{https://github.com/coq-contribs/ccs}} in Coq
and Chaudhuri et al.\;\cite{chaudhuri2015lightweight} in Abella, the latter
focuses on `bisimulation up-to' techniques (for strong bisimilarity)
for CCS and $\pi$-calculus.
Damien Pous \cite{pous2007new} also formalised up-to techniques and some CCS examples in
Coq.
Formalisations less related to ours
include Kahsai and Miculan \cite{kahsai2008implementing} for the spi
calculus in Isabelle/HOL, and Hirschkoff \cite{hirschkoff1997full} for the $\pi$-calculus in Coq.

% next file: conclusions.htex

% \section{To Do}
% \label{s:todo}

% \finish{discuss multiple equations/contractions} 

% The formalisation of multi-variable context (or expressions) and the version of
% ``unique solutions'' theorems with multiple equations/contractions, is
% a relative hard problem. To the best of our knowledge, so far there's no formalization (in any kind of
% process calculi) touching these areas yet. However, we've made a few initial steps
% of this interesting problem.

\section{Conclusions and future work}
\label{s:concl}

We have highlighted a formalisation of the theory of CCS in the 
HOL4 theorem prover (for lack of space we have not discussed 
the formalisation of some basic algebraic theory, of the basic
properties of the expansion preorder,   and of a few
 versions of `bisimulation up to'
techniques such as  bisimulation up-to bisimilarity). 
The formalisation has focused on the theory of
unique solution of equations and contractions. 
The formalisation has also allowed us to further develop the theory,
notably the basic properties of rooted contraction, and the unique
solution theorem for it with respect to rooted bisimilarity. 
The formalisation brings up and exploits similarities between results
and proofs for different equivalences and preorders. 
We think that the statements in the formalisation are easy to read and
understand, as they remain very close  to  the ordinary  statements of
the theory of CCS. 
Moreover, 
the logic foundations of HOL makes the whole work (or individual
parts) easily portable to other theorem provers.


As future work,  it would be worth extending the current work
to multiple equations and contractions; and considering other equivalences
and preorders, notably trace relations.  
On another line, one could  
examine the formalisation of  a different approach to unique
solutions, by Durier et al.\ \cite{DurierHS17}, in which the use of contraction is
replaced by semantic conditions on process transitions such as
divergence.  

% Further plan on the formalisation mainly includes: 1) the extension to
% multi-variable equation/contractions. 2) the support of recursion
% operators in 
% CCS context and expressions.


% We believe that, beside the discovery of the rooted contraction $\rcontr$
% with a more elegant unique solution theorem,
% our CCS formalisation in HOL4 has also
% provided a solid formal framework for future theoretical developments in
% Concurrency Theory, particularily for process algebras like CCS. It's
% easily understandable, with statements extremely close to the original
% textbook. The logic foundations of HOL makes the whole work (or individual
% parts) easily portable to other theorem provers.

\subsection*{Acknowledgements}

We would like to thank many people from the HOL community
for their help in the past two years, in particular 
Michael Norrish, Thomas Tuerk, 
Ramana Kumar, Konrad Slind, Robert Beers, and Jeremy Dawson. 
% The second half of this
% paper was written in memory of  Mike J. Gordon.


\bibliographystyle{eptcs}
\bibliography{generic}
\end{document}
