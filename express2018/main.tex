%%%% -*- Mode: LaTeX -*-
%%
%% This is the draft of the 2nd part of EXPRESS/SOS 2018 paper, co-authored by
%% Prof. Davide Sangiorgi and Chun Tian.

\documentclass[submission]{eptcs} % required template (EPTCS)
\providecommand{\event}{EXPRESS/SOS 2018} % Name of the event you are submitting to

\usepackage[T1]{fontenc}
\usepackage{lmodern} % Latin Modern fonts (better than CM-Super)
\usepackage[utf8]{inputenc} % no need for XeTeX, which always uses UTF-8

%% Math symbol packages
\usepackage{amsmath}
\usepackage{amssymb}
\usepackage{mathrsfs}

\usepackage{stmaryrd}
\SetSymbolFont{stmry}{bold}{U}{stmry}{m}{n}
% \cupdot (the best one)
\newcommand{\cupdot}{\mathbin{\mathaccent\cdot\cup}}

\usepackage{amsthm}

\newtheorem{definition}{Definition}[section]
\newtheorem{example}[definition]{Example}
\newtheorem{lemma}[definition]{Lemma}
\newtheorem{theorem}[definition]{Theorem}
\newtheorem{corollary}[definition]{Corollary}
\newtheorem{proposition}[definition]{Proposition} 
\newtheorem{remark}[definition]{Remark}

% HOL theorem embedding support
\usepackage{holindex}
\usepackage{alltt}
%% TeX commands needed for generating terms and theorems of our CCS theories:

\newcommand{\HOLTokenStrongEQ}{$\sim{}$}
\newcommand{\HOLTokenWeakEQ}{$\approx{}$}
\newcommand{\HOLTokenObsCongr}{$\approx^{\mathrm{c}}\!$}
\newcommand{\HOLTokenEPS}{$\overset{\epsilon}{\Rightarrow}$}
\newcommand{\HOLTokenTransBegin}{$-$}
\newcommand{\HOLTokenTransEnd}{$\rightarrow$}
\newcommand{\HOLTokenWeakTransBegin}{$=$}
\newcommand{\HOLTokenWeakTransEnd}{$\Rightarrow$}
\newcommand{\HOLTokenExpands}{$\succeq_{\mathrm{e}}\!$}
\newcommand{\HOLTokenContracts}{$\succeq_{\mathrm{bis}}\!$}
\newcommand{\HOLTokenObsContracts}{$\succeq^{\mathrm{c}}_{\mathrm{bis}}\!$}
%\renewcommand{\HOLTokenImp}{\ensuremath{\Longrightarrow}}

% Experimental environments
\usepackage{environ}
\NewEnviron{HOLTrans}{\overset{\BODY}{\longrightarrow}}
\NewEnviron{HOLWeakTrans}{\overset{\BODY}{\Longrightarrow}}


%
% LaTeX math formula spacing:
%
% \enskip	leave a horizontal space of respectively half an em
% \quad		space equal to the current font size (= 18 mu), one em
% \qquad	twice of \quad (= 36 mu), i.e. two ems
% \enspace	it's inherited from Plain TeX and is almost the same
%                      as \enskip, but technically it is a kern, rather than a skip.
% \,	3/18 of \quad (= 3 mu)
% \:	4/18 of \quad (= 4 mu)
% \;	5/18 of \quad (= 5 mu)
% \!	-3/18 of \quad (= -3 mu)
% \ (space after backslash!) equivalent of space in normal text

\usepackage{graphicx}
\usepackage[all,cmtip]{xy}

% NAMES and VARIABLES

\def\Names{{\cal N}}            % set of all names
\def\fn#1{\rmsf{fn}(#1)}         % free names
\def\fv#1{\rmsf{fv}(#1)}         % free variables
\def\bv#1{\rmsf{bv}(#1)}         % bound variables
\def\bn#1{\rmsf{bn}(#1)}         % bound names

\newcommand{\dom}[1]{{\rmsf{dom}}(#1)} % the domain of something  

% FOR PROCESSES 

\def\nil{{\boldsymbol 0}} % nil
\def\res#1{{\boldsymbol \nu} #1\:}   % restriction
%% the following definitions allow us to use the symbols ! . and | 
%directly, for the replication, prefix and parallel compoisition
%operators in math mode 
\mathcode`\!="4021 % `!' as prefix operator
\mathcode`\.="602E  % prefix 
\mathcode`\|="326A % `|' as relation operator

\newcommand{\outC}[1]{\overline{#1}}      % CCS  output
\newcommand{\inpC}[1]{#1}                 % CCS  input

\newcommand{\out}[2]{\overline{#1}\langle{#2}\rangle} % output with value    
\newcommand{\inp}[2]{#1(#2)}  % input with value
\newcommand{\inpW}[2]{#1(#2). }  % input with value and dot

\newcommand{\iae}[2]{{#1}\langle{#2}\rangle} % input with value    


\newcommand{\cond}[3]{\myif\ #1\ \mythen\ #2\ \myelse\ #3} % if-then-else  
\newcommand{\myif}{\myspace{\rmtt{if}}\myspace}            % ``if''
\newcommand{\mythen}{\myspace{\rmtt{then}}\myspace}        % ``then''
\newcommand{\myelse}{\myspace{\rmtt{else}}\myspace}        % ``else''

\newcommand{\true}{\rmsf{true}} %boolean true
\newcommand{\false}{\rmsf{false}} %boolean false


% substitutions  (used as a postfixed operator)
\def\sub#1#2{\{\raisebox{.5ex}{\small$#1$}\! / \!\mbox{\small$#2$}\}} 



% TRANSITIONS (arrows)

\newcommand{\racap}{\mathrel{\stackrel{{\;\; {{\wedge}} \;\;}}{\mbox{\rightarrowfill}}}} 

\newcommand{\arr}[1]{\mathrel{\stackrel{{\;\;#1\;\;}}{\mbox{\rightarrowfill}}}}
                                %  strong labelled transition 

\newcommand{\Arr}[1]{\mathrel{\stackrel{{\;\;#1\;\;}}{\mbox{\rightarrowfillWEAK}}}} 
                                    %weak  labelled transitions
                                    
\newcommand{\arcap}[1]{\mathrel{\stackrel{{\;\; {\widehat{#1}} \;\;}}{\mbox{\rightarrowfill}}}} 
                                    %strong labelled transitions with hat
\newcommand{\Arcap}[1]{\mathrel{\stackrel{{\;\;{\widehat{#1}}\;\;}}{\mbox{\rightarrowfillWEAK}}}}
                                    %weak labelled transitions with   hat
                                    
% FOR CONTEXTS

\newcommand{\contexthole}{ [ \cdot  ] }      %hole of a context
\newcommand{\ct}[1]{ C \brac{#1} }   %filled context
\newcommand{\qct}{ C  }              %unfilled context  
\newcommand{\brac}[1]{[#1] }   % brackets for the context hole



%% SOME STYLE COMMANDS
\newcommand{\rmtt}[1]{{\rm\tt{#1}}} % for keywords like ``if'', ``then'' ... 
\newcommand{\rmsf}[1]{{{\rm\sf{#1}}}}


%BEHAVIOURAL  EQUIVALENCES and relations

\def\R{{\cal R}}          % R without spaces around
\def\RR{\mathrel{\cal R}} % R with some space around
\def\S{{\cal S}}          % S without spaces around
\def\SS{\mathrel{\cal S}} % S with some space around


\newcommand{\equival}{=} %\mathrel{\equiv_\alpha} 
                          % equality up to alpha conversion 





%SPECIAL SYMBOLS


\def\midd{\; \; \mbox{\Large{$\mid$}}\;\;}
               %separation symbol in  grammars               

\def\st{\; \mid \;} % ``such that'' in formulas
\def\DSdefi{\stackrel{\mbox{\scriptsize def}}{=}} % definition equal

\def\Defi{\stackrel{\mbox{\scriptsize $\triangle$}}{=}} % definition equal


% \def\qed{}
%  {\unskip\nobreak\hfil\penalty50\hskip1em\null\nobreak\hfil
%   $\Box$\parfillskip=0pt\finalhyphendemerits=0\endgraf}
%                    % end of proofs (or theorems,results without proofs)

\newcommand{\myspace}{\:}  % some spacing abbreviation

% rename tilde to widetilde, to be used for tuples
\renewcommand{\tilde}{\widetilde}


% % environment for proofs (for CUP style file)
% \newenvironment{proof}{\noindent {\bf Proof }}{\qed \bigskip}


\newcommand{\finish}[1]{ \vskip .2cm  {\bf #1} \vskip .2cm   \marginpar{{\bf $DS$}}}

\newcommand{\recu}[2]{{\tt rec}\: #1 . #2}

\newcommand{\rapprox}{\mathrel{\approx^{\rm{c}}}}


% NEW THINGS 

\newcommand{\mylabel}[1]{{\rm (#1).}}
\newcommand{\MYsketch}{[Sketch] }


\newcommand{\behav}{equation}
\newcommand{\behavC}{contraction}


\newcommand{\hb}{\hskip .5cm}

\newcommand{\ArrN}[2]{\mathrel{\stackrel{{\;\;#1\;\;}}{\mbox{\rightarrowfillWEAK}}_{#2}
  }} 
                                    %weak  labelled transitions weighted

\newcommand{\Var}{{\cal X}}

\newcommand{\AL}{{\cal RL}}
\newcommand{\PL}{{\cal L}}

\newcommand{\sign}{\Sigma}
\newcommand{\prsign}{\pr_\sign}


% possible actions in Abnients:
\newcommand{\qina}{{\ccc{in}} } 
\newcommand{\qout}{{\ccc{out}} }
\newcommand{\qopen}{{\ccc{open}} }
\newcommand{\capa}{{\ccc{capa}} }
\newcommand{\amb}[2]{#1 [\, #2 \,] } % ambients

\newcommand{\HOAMB}{\mbox{\rm{HO$\pi$Amb}}}
\newcommand{\SeqCCS}{\mbox{\rm{SeqCCS}}}


\newcommand{\SE}{{E\!S}}
\newcommand{\SL}{{L\!S}}
\newcommand{\SEp}{{E\!S'}}
\newcommand{\SEpU}{{E\!S'_1}}


%\newcommand{\mcontrP}[1]{\mathrel{\stackrel{{\footnotesize{\mbox{$\succ$}}}}{\footnotesize{\mbox{$#1$}}}}}
\newcommand{\mexpaP}[1]{\mathrel{\stackrel{{\footnotesize{\mbox{$\prec$}}}}{\footnotesize{\mbox{$#1$}}}}}

\newcommand{\wc}{\mathrel{\approx^{\rm c}}}
\newcommand{\wb}{\approx}
\newcommand{\contr}{\mathrel{\succeq_{\rm{bis}}}}
\newcommand{\expa}{\mathrel{\succeq_{\rm{e}}}}


\newcommand{\mcontr}{\mathrel{\succeq}}
\newcommand{\mexpa}{\mathrel{\preceq}}

\newcommand{\mcontrmay}{\mathrel{\succeq_{\rm{ctx}}}}
\newcommand{\mexpamay}{\mathrel{\preceq_{\rm{ctx}}}}


\newcommand{\mcontrTE}{\mathrel{\succeq_{\rm{tr}}}}
\newcommand{\TE}{\approx_{\rm tr}}       %trace equivalence


\newcommand{\mcontrBIS}{\mathrel{\succeq_{\rm{bis}}}}
\newcommand{\mexpaBIS}{\mathrel{\preceq_{\rm{bis}}}}


\newcommand{\rcontr}{\mathrel{\succeq^{\rm{c}}_{\rm{bis}}}}
\newcommand{\rexpa}{\mathrel{\preceq^{\rm{c}}_{\rm{bis}}}}


\newcommand{\til}{\tilde}

\newcommand{\ctx}[1]{#1^{\rm{c}}}

\newcommand{\Dwaleq}[1]{\Dwa^{\leq #1}}
\newcommand{\Dwageq}[1]{\Dwa^{\geq #1}}


\newcommand{\holeDS}{ [ \cdot  ] }
\newcommand{\holei}[1]{[\cdot]_{#1}}
\newcommand{\ctp}[1]{ C' \brac{#1} }  %primed context       
\newcommand{\qctp}{ C'  }
\newcommand{\ctpp}[1]{ C'' \brac{#1} }  %primed context       
\newcommand{\ctppp}[1]{ C''' \brac{#1} }  %primed context       

\newcommand{\qctpp}{ C''  }
\newcommand{\qctppp}{ C'''  }


\newcommand{\may}{\approx_{\rm ctx}}       %may equivalence

\newcommand{\mayHA}{\approx^{\rm{HAmb}}_{\rm ctx}}       %may equivalence

\newcommand{\hk}{\hskip .2cm }
\newcommand{\mysp}{10pt}
\newcommand{\tkp}{10pt}
\newcommand{\tkpS}{6pt}
\newcommand{\tkpSS}{3pt}
\newcommand{\tkpP}{15pt}

\newcommand{\smay}{\mathrel{\sim_{\rm ctx}}}       %may equivalence
\newcommand{\smayHA}{\mathrel{\sim^{\rm{HAmb}}_{\rm ctx}}}       %may equivalence

\newcommand{\holE}{\contexthole}  % hole

\newcommand{\murule}{\fortherules\mu} % mu rule of $\lambda$-calculus
\newcommand{\nurule}{\fortherules\nu} % nu rule of $\lambda$-calculus
\newcommand{\nuvrule}{\fortherules\nuv} % nuv rule of $\lambda$-calculus
\newcommand{\xirule}{\fortherules\xi} % xi rule of $\lambda$-calculus
\newcommand{\betarule}{\fortherules\beta} %beta rule of $\lambda$-calculus
\newcommand{\betavrule}{\fortherules\betav} % beta_v rule of $\lambda$-calculus
\newcommand{\etarule}{\fortherules\eta} % eta rule of $\lambda$-calculus
\newcommand{\nuv}{\nu_{\myrm v}} %beta rule of $\lambda$-calculus
\newcommand{\betav}{\mbox{$\beta_{\myrm{v}}$}} %beta rule of $\lambda$-calculus
\newcommand{\alpharule}{\fortherules\alpha} %alpha rule 
\newcommand{\fortherules}[1]{\mbox{$#1$}} %auxiliary def for the rules



\newcommand{\barbedbis}
{\mathrel{\stackrel{\bfcdotB}{\approx}}}

%OLD:
%{\mbox{ $\approx \! \! \! \!\! \!\!        
%\raisebox{1.15ex}[0ex][0ex]{\bfcdot} \; \,$}}



\newcommand{\wbb}{\mathrel{\approx_{\rm{bar}}}}
\newcommand{\wbc}{\mathrel{\approx^{\rm{c}}_{\rm{bar}}}}
% \newcommand{\wbb}{\mathrel{\barbedbis}}
% \newcommand{\wbc}{\cong}


\newcommand{\bfcdotB}{ {\mbox{\boldmath $.$}}  }         

\newcommand{\bcontra}
{\mathrel{\mcontr_{\rm{bar}}}}
%{\mathrel{\stackrel{\bfcdotB}{\mcontr}}}

\newcommand{\cbc}
{\mathrel{\mcontr^{\rm{c}}_{\rm{bar}}}}
%{\mathrel{{\mcontr_{\rm{bc}}}}}


\newcommand{\mypt}{2pt} 

% --------------



\newenvironment{myquote}
               {\list{}{\rightmargin\leftmargin}%
                \item\relax}
               {\endlist}


% \newenvironment{proofEx}{
% \begin{myquote}
% %\trivlist\parindent=0pt
% %      \item[\hskip \labelsep{\bf Answer: }]}
% \noindent %{\bf Answer to}
% }
% {\qed%\endtrivlist
% \end{myquote}}

%SPECIAL SYMBOLS

\newcommand{\EXX}[1]{{\bf Exercise~\ref{#1}}} % for answers to exercises 
\newcommand{\EXXpa}[2]{{\bf Exercise~\ref{#1}(#2)}} % for answers to exercises 


\newcommand{\Mybar}{\hrulefill} % separation rule

% \def\finish#1{\vskip.2cm\noindent{\em #1}%
%   \marginpar{$\longleftarrow$}\vskip.2cm}

\newcommand{\spaceD}{\,}


\newcommand{\bulletD}{\diamond}
%\mathrel{\lozenge} %\bowtie %blacktriangle %\minuso %\blacktriangleup %filleddiamond %\blackdiamond}

\newcommand{\ccc}{\rmtt} % {\rm\tt{#1}}}  %{\mbox{{\tt #1}}}
\newcommand{\cccTT}{\tt} % {\rm\tt{#1}}}  %{\mbox{{\tt #1}}}

\newcommand{\rr}{\RR}  
\newcommand{\Id}{{\cal I}} % identity relation  


\newcommand{\Prop}{{\cal P}} %
\newcommand{\FF}{F} % a function
\newcommand{\FFbis}{F_{\sim}} 
\newcommand{\FFbisW}{F_{\approx}} 
\newcommand{\finLISTS}{\mbox{{\tt FinLists}$_{A}$}} %   
\newcommand{\fininfLISTS}{\mbox{{\tt FinInfLists}$_{A}$}} %   
\newcommand{\finLISTSi}[1]{\mbox{\tt FinLists}_{#1}} %   
\newcommand{\fininfLISTSi}[1]{\mbox{\tt FinInfLists}_{#1}} %   
\newcommand{\nilLISTS}{\mbox{\tt nil}} %   
\newcommand{\mapLISTS}[2]{\mbox{\tt map}\: #1\: #2 } %   
\newcommand{\qmapLISTS}{\mbox{\tt map}} %   
\newcommand{\iterate}[2]{\mbox{\tt iterate}\: #1\: #2 } %   
\newcommand{\qiterate}{\mbox{\tt iterate}} %   
\newcommand{\qnats}{\mbox{\tt nats}} %   
\newcommand{\qfrom}{\mbox{\tt from}\: } %   
\newcommand{\fibs}{\mbox{\tt fibs}} %   
\newcommand{\qplus}{\mbox{\tt plus}\:} %   
\newcommand{\qtail}{\mbox{\tt tail}\,} %   
\newcommand{\plusU}{+_1} %   
\newcommand{\FFlist}{\Phi_{A\tt list}} 

\newcommand{\consLISTS}{\mbox{\tt cons}} %   
\newcommand{\consLISTSnew}[2]{\langle #1\rangle \bullet #2} %   
\newcommand{\consLISTSnewB}[2]{ #1 \bullet #2} %   


\newcommand{\simList}{\sim}  %_{A\tt list}}} % bisimilarity on lists 


\newcommand{\tree}[1]{\mbox{{\tt Tree}$(#1)$}} %   
%\newcommand{\root}[1]{\mbox{{\tt root}$(#1)$}} %   
\newcommand{\T}{{\cal T}}
\newcommand{\Vp}{{\tt V}}
\newcommand{\Rp}{{\tt R}}
\newcommand{\Gin}[2]{{\cal G}^{\tt ind}(#1,#2)} %   
\newcommand{\Gco}[2]{{\cal G}^{\tt coind}(#1,#2)} %   



\newcommand{\pws}[1]{\wp (#1)} % powerset
\newcommand{\lfp}[1]{\qqlfp(#1)} % least fixed point 
\newcommand{\gfp}[1]{\qqgfp(#1)} % greatest fixed point
\newcommand{\qqlfp}{{\tt lfp}} % least fixed point abbrv.
\newcommand{\qqgfp}{{\tt gfp}} % greatest fixed point abbrv.
\newcommand{\qlfp}{least fixed point}
\newcommand{\qgfp}{greatest fixed point} 


\newcommand{\Fcoin}{F_{\tt coind}} % coind. def. set
\newcommand{\Fin}{F_{\tt ind}}     % ind. def. set

\newcommand{\Lao}{\Lambda^0}  %closed $\lambda$-terms



  %convergence

\newcommand{\Dwa}{\Downarrow}           % plain convergence
\newcommand{\DwaP}[2]{#1 \Downarrow #2} % plain convergence, in rules


\newcommand{\EQsin}[2]{#1 = #2} % syn. equality in rules

\newcommand{\dwa}{\downarrow}           % plain convergence
\newcommand{\Up}{\Uparrow}   % divergence
\newcommand{\UpP}{\Uparrow}   % divergence in rules

\newcommand{\Reach}{\Downarrow}           % reachability
\newcommand{\Termi}{\downharpoonright}        % can terminate
\newcommand{\TermiP}{\Termi}        % can terminate, in rules
\newcommand{\Adiv}{\upharpoonright_\mu}           % \mu-divergence
\newcommand{\Adiva}{\upharpoonright_a}           % \mu-divergence

\newcommand{\LL}{{\cal L}}  % set of terms
\newcommand{\States}[1]{{\tt St}^{#1}} %states of an LTS


\newcommand{\myemptyItem}{\mbox{$ $ }} % utile per il xy package
\newcommand{\NONemptyItem}[1]{\mbox{$#1$ }} % utile per il xy package

\newcommand{\LongrightarrowN}[1]{\Longrightarrow_{#1}}


% for imperative programs
\newcommand{\XX}{\spaceD{\ccc{X}}}
\newcommand{\YY}{\spaceD{\ccc{Y}}}

% for LTSs
\newcommand{\Act}{\mbox{\it Act}} 
\newcommand{\pr}{\mbox{\it Pr}} % \mbox{\it Pr}}  %{{\mathbb P}} %{{\cal P}r} 
\newcommand{\power}{\wp} 
\renewcommand{\Pr}{\pr} % \mbox{\it Pr}}  %{{\mathbb P}} %{{\cal P}r} 


% membership stuff among relations
\newcommand{\memb}[3]{ #1 #3 #2 } 
\newcommand{\rmemb}[3]{ {(#1 , #3)} \in { #2} } 
%\newcommand{\Rmemb}[3]{ \tobr{#1 , #3} \in  #2 } 

% symbols
%\newcommand{\vv}{P} 
%\newcommand{\ww}{Q} 
\newcommand{\pp}{P} 
\newcommand{\qq}{Q} 

\newcommand{\rmm}[1]{\mbox{\rm #1}} %labels of transitions in figure 


%% for inference rules


\newcommand{\infrule}[3]{\[
{\trans{#1}\quadrule \displaystyle{#2 \over #3} } %\\[10pt]
\]}
\newcommand{\infruleSIDE}[4]{\[
{\trans{#1}\quadrule\displaystyle{#2 \over #3}\;\; #4 } %\\[10pt]
\]}  % inf rule with a side condition
\newcommand{\shortinfrule}[3]{ {\trans{#1}} \quadrule
     \displaystyle{#2 \over #3}}
\newcommand{\shortinfruleSIDE}[4]{ {\trans{#1}} \quadrule
     \displaystyle{#2 \over #3}\;\; #4}

\newcommand{\shortaxiom}[2]{{\trans{#1}}\quadrule
\displaystyle{ \over #2}}

\newcommand{\myinf}[3]{{\rn{#1}}\quadrule \displaystyle{#2 \over #3} }
    % for  plain  inference rules

\def\trans#1{\rn{#1}}   % for the names of transition rules
\newcommand{\rn}[1]{%
  \ifmmode 
    \mathchoice
      {\mbox{\sc #1}}
      {\mbox{\sc #1}}
      {\mbox{\small\sc #1}}
      {\mbox{\tiny\uppercase{#1}}}%
  \else
    {\sc #1}%
  \fi}

\newcommand{\quadrule}{\hskip .2cm }

\newcommand{\andalso}{\quad\quad}


% references

\def\reff#1{(\ref{#1})}       %references between brackets



\newcommand{\enco}[1]{[\! [ #1 ] \! ]  }


\newcommand{\mydots}{,\ldots,}

% for not\sim_n

\newcommand{\notsimN}[1]{\mathrel{\not\!{\sim_{#1}}} }



\newcommand{\cti}[2]{ C_{#1} \brac{#2} }   %filled context
\newcommand{\ctD}[1]{ D \brac{#1} }   %filled context
\newcommand{\qcti}[1]{ C_{#1}  }   % context

\newcommand{\ctDp}[1]{ D' \brac{#1} }   %filled context
\newcommand{\qctD}{D}   % context
\newcommand{\qctDp}{D'}   % context

%\newcommand{\qctp}{C'}   % context


\newcommand{\beginlongtable}{
 \begin{longtable}{l@{\extracolsep{\fill}}p{76mm}@{\extracolsep{\fill}}r}
%{|l@{\extracolsep{\fill}}p{80mm}@{\extracolsep{\fill}}r|}
%% READ THIS !!!
%% ho cancellato sotto altrimenti mi fa una entry nella list of tables
%\caption{ffff} \\
%\hline
%symbol          & description                & page \\
%\hline
}
\newcommand{\ENDlongtable}{% \hline
\end{longtable}}

\newcommand{\GLSbeg}[1]{\noindent %\underline
{\bf \large #1}}

\newcommand{\GLS}[1]{
\multicolumn{3}{l}{%\noindent 
%\underline
{\bf \large #1}}\\ }
\newcommand{\GLSb}[1]{
\multicolumn{3}{l}{%\noindent 
%\underline
{\it \large #1}}\\ }

\newcommand{\beginlongtableIN}{\\}
\newcommand{\ENDlongtableIN}{\\}

% % to add or remove a line to a page use these 
% \newcommand{\longpage}{\enlargethispage{\baselineskip}} 
% \newcommand{\shortpage}{\enlargethispage{-\baselineskip}} 





\usepackage{DSarrow} % this is something to use extensible arrows in transitions

% \usepackage{pgf,tikz}
% \usetikzlibrary{arrows}

\title{The coarsest precongruence contained in bisimilarity 
contraction and its unique solution theorem}
\author{Davide Sangiorgi
\institute{Universit\`a di Bologna and INRIA\\Bologna, Italy}
\email{davide.sangiorgi@unibo.it}
\and Chun Tian
\institute{Fondazione Bruno Kessler\thanks{Part of this work was
    carried out when the author was studying in University of
    Bologna.}\\Trento, Italy}
\email{ctian@fbk.eu}
}
\def\titlerunning{Coarsest precongruence contained in bisimilarity contraction}
\def\authorrunning{D. Sangiorgi \& C. Tian}

\begin{document}
\maketitle

\begin{abstract}
Milner's ``unique solution of equations'' theorem for (rooted) weak
bisimilarity have severe syntactical limitations, i.e.\;the
equations must be both strongly guarded and sequential. By replacing
equations with special inequations called ``contraction'', Sangiorgi
has proven the ``unique solution of contractions'' theorem for weak
bisimialarity, which requires only weakly guarded equations. However,
contraction is not (pre)congruence under direct sums of processes.
To overcome this difficulty, we further moved to ``rooted
contraction'', which is the coarsest precongruence contained in
contraction bisimilarity. Using rooted contraction one obtains an
unique-solution theorem that is valid for 
\emph{rooted bisimilarity} (hence also for bisimilarity itself) while
requires true weak guardness (with direct sums). All results were
formalized in HOL theorem prover (HOL4), as part of a rather complete
formalization of process algebra CCS. We show some highlights of this
formalization work.
\end{abstract}

\section{Introduction}

A prominent proof method for bisimulation, put forward by Milner and widely used in his
landmark CCS book \cite{Mil89} is the
\emph{unique solution of equations}, whereby two tuples of processes are
componentwise bisimilar if they are solutions 
of the same system of equations.
This method  is important in verification techniques and tools
based on algebraic reasoning \cite{theoryAndPractice,RosUnder10,BaeBOOK}. 

In the   \emph{weak} case (when  behavioural equivalences abstract from internal moves,
which practically is the most relevant case), however, 
Milner's proof method has severe syntactical limitations. 
To overcome such limitations, Sangiorgi proposes to replace
equations with  special inequations called
\emph{contractions} \cite{sangiorgi2015equations}. Contraction is a
preorder that, roughly, places some efficiency
constraints on processes.  Uniqueness of the solutions of a system of contractions
 is defined as with systems of equations:  
any two solutions must be bisimilar.
The difference with equations is in the meaning of solution:
in the case of contractions
the solution is evaluated with respect to
the contraction preorder, rather than bisimilarity. 
With contractions, most syntactic limitations of the unique-solution theorem can be
removed.  One constraint that still remains in
\cite{sangiorgi2015equations} is on occurrences of the sum operator,
due to the failure of substitutivity of contraction w.r.t. such operator.


The main goal  of the work described in this paper is 
a rather  
 comprehensive formalisation  of the core of the theory of CCS 
 in the HOL
theorem prover (HOL4),  with a focus on the theory of unique solutions of contractions.
The formalisation however is not confined to the theory of  unique solutions, but embraces 
most of the 
core of the theory of CCS \cite{Mil89}
(partly because the theory of unique solutions
 relies on a number of more fundamental results):
indeed the formalisation encompasses the basic properties of strong and weak
bisimilarity (e.g. the fixed-point and substitutivity properties), 
their algebraic theory, various versions of ``bisimulation up to''
techniques (e.g., bisimulation up-to bisimilarity),
the main properties  of the rooted bisimilarity (the congruence induced by weak
bisimilarity, also called observation congruence). Concerning rooed bisimilarity, the formalisation
includes Hennessy and Deng lemmas, and two proofs that rooted bisimilarity is the largest
congruence included in bisimilarity:
one as in Milner's
book,  requiring the hypothesis that  no processes can use all labels; the other without
such hypothesis, essentially formalising van Glabbek's paper \cite{vanGlabbeek:2005ur}
(such proof however follows the structure of the ordinal numbers, which cannot be handled
in HOL, and therefore it is restricted to finite-state processes ).
Similar theorems are proved for the rooted contraction preorder.

  
In this respect,  the work is an extensive experiment with the use of the HOL theorem prover and its
most recent developments, including a package  for expressing coinductive definitions.
The
work consists of about 20,000 lines of proof scripts in Standard ML.



From the CCS theory viewpoint, the formalisation has offered us the possibility of 
further refining the theory of unique solutions. 
In particular, the existing theory  placed limitations on the body of the contractions due to the
substitutivity problems of weak bisimilarity and other behavioural relations with respect
to the sum operator.  
We have thus refined the proof  technique based on contractions by moving to the 
\emph{rooted contraction}, that is, the coarsest (pre)congruence contained in the contraction
preorder.  Using rooted contraction one obtains a unique-solution theorem that is valid for
\emph{rooted bisimilarity} (hence also for bisimilarity itself), and without syntactic
constraints on the occurrences of sums.   

\finish{ I had to remove the reference \cite{Tian:2017wrba} here since it would further
  weaken this paper. } 

Another advantage of the formalisation is 
that we can take advantage of results about different 
equivalences or preorders that share a similar  proof structure. 
Examples are: the results that rooted bisimilarity and rooted contraction are,
respectively, the coarsest congruence contained in weak bisimilarity 
and the coarsest precongruence contained in the contraction  preorder; 
the result about unique solution of equations for weak bisimilarity that uses the
contraction preorder as an auxiliary relation, and other unique solution results (e.g., 
the one for rooted in which
the auxiliary relation is rooted contraction); various forms of enhancement of the bisimulation
proof method (the `up-to' techniques).  
In these cases, there are only a few places in which the HOL scripts have to be modified.
Then the succesful termination of the scripts  gives us a guarantee that the proof is
completed,  removing the risk 
of overlooking or missing details as in hand-written proofs. 


% to describe
% The purpose of this paper is twofold. 
% On the one hand, 
% On the other hand, we provide a  
%  comprehensive formalisation  of the core of the theory of CCS 
%  in the HOL
% theorem prover (HOL4). The formalisation  includes the proofs of
% Milner's 3 ``unique solution of equations'' theorems and
% contractions discussed in the present paper, but is not limited to it (partly because such
% theorems rely on a number of more fundamental results):
% indeed the formalisation encompasses the basic properties of strong and weak
% bisimilarity (e.g. the fixed-point and substitutivity properties), 
% their algebraic theory, various versions of ``bisimulation up to''
% techniques (e.g., bisimulation up-to expansion),
% the basic properties  of rooted bisimilarity. 
% Thus the work is an extensive experiment with the use of the HOL theorem provers and its
% most recent developments, including a facility  for expressing coinductive definitions.

% % Considering the relationship between bisimilarity and rooted
% % bisimilarity, the formalisation includes the proof that the latter is the coarsest
% % congruence included in the former, for which two proofs are formalized: one as in Milner's
% % book,  requiring the hypothesis that  no processes can use all labels; the other without
% % such hypothesis, essentially formalising van Glabbek's paper \cite{vanGlabbeek:2005ur}.
% % Similar theorems are proved for rooted contractions wrt the contraction preorder. 


\paragraph{Structure of the paper}....
% \section{Background}
% \label{s:back}

\section{CCS}
\label{ss:ccs}


We assume  a possibly infinite set of \emph{names} $\mathscr{L} = \{a, b,
\ldots\}$ forming input and output actions, plus a special invisible
action $\tau$ not in $\mathscr{L}$, and a set of variables $A,B,
\ldots$ for defining recursive behaviours.
The class  of the CCS processes is inductively defined from $\nil$ by the operators
of prefixing, parallel composition, sum (binary choice), restriction, recursion and \hl{relabeling}:
\begin{equation*}
\begin{array}{ccl}
\mu  & := &  \tau \hspace{.3pt} \; \midd \; a  \; \midd \;  \outC a  \\
P  & := &  \nil \; \midd \;  \mu . P \; \midd \;  P_1 |  P_2 \; \midd  \;
P_1 + P_2 \; \midd % \; \mu . P\; \midd  \; 
  (\res a\!)\, P  \;  \midd \;  A \; \midd \; \recu A  P
\; \midd \; P\; [r\!\!f]  % relabelling
\end{array}
\end{equation*}
%We sometimes omit trailing $\nil$, e.g., writing $a|b$ for $a.\nil |b .\nil $ .
The operational semantics of CCS is given by means of
a Labeled Transition System (LTS), shown in Fig.~\ref{f:LTSCCS} as SOS
rules (the symmetric version of the two rules for
parallel composition and the rule for sum are omitted).
A CCS expression uses only \emph{weakly-guarded sums} if all occurrences of
the sum operator are of the form $\mu_1.P_1 + \mu_2.P_2 + \ldots
+ \mu_n.P_n$, for some $n \geq 2$.
 The \emph{immediate derivatives} of a
process $P$ are the elements of the set $\{P' \st P \arr\mu P' \mbox{
  for some $\mu$}\}$.
% We use $\ell$ to range over
%  visible actions (i.e.~inputs or outputs, excluding  $\tau$).
\begin{figure*}
\begin{center}
\vskip .1cm
 $\displaystyle{  \over  \mu.  P    \arr\mu
P } $  $ \hb$   
\hskip .5cm
 $\displaystyle{   P \arr\mu   P' \over   P + Q   \arr\mu
P'  } $  $ \hb$   
\hskip .5cm
 $\displaystyle{   P \arr\mu   P' \over   P | Q   \arr\mu
P' | Q } $  $ \hb$   
\hskip .3cm
  $\; \;$  $\displaystyle{ P \arr{ a}P' \hk \hk  Q
\arr{\outC a }Q'  \over     P|  Q \arr{ \tau} P'
|  Q'  }$ 
\\
\vspace{.2cm}
$\displaystyle{ P \arr{\mu}P' \over
 (\res a\!)\, P   \arr{\mu} (\res a\!)\, P'} $ $ \mu \neq a, \outC a$
$ \hb$
%
$\displaystyle{ P \sub {\recu A P} A \arr{\mu}P' \over
 \recu A P   \arr{ \mu} P'  } $
\hskip .5cm  
$\displaystyle{ P \arr{\mu} P' \over
 P \;[r\!\!f] \arr{r\!\!f(\mu)} P' \;[r\!\!f]} $ $\forall a.\, r\!\!f(\outC a) = \overline{r\!\!f(a)}$
$ \hb$ %  &
\end{center}
\caption{\hl{Structural Operational Semantics} of CCS}
\label{f:LTSCCS}
\end{figure*}
Some standard notations for transitions:  $\Arr\epsilon$ is the 
reflexive and transitive closure of $\arr\tau$, and 
$\Arr \mu $ is $\Arr\epsilon \arr\mu \Arr\epsilon$ (the
composition of the three relations).
Moreover,   
$ 
P \arcap \mu P'$ holds if $P \arr\mu P'$ or ($\mu =\tau$ and
$P=P'$); similarly 
$ 
P \Arcap \mu P'$ holds if $P \Arr\mu P'$ or ($\mu =\tau$ and
$P=P'$).
We write $P \:(\arr\mu)^n P'$ if $P$ can become $P'$ after performing
$n$ $\mu$-transitions. Finally, $P \arr\mu$ holds if there is $P'$
with $P \arr\mu P'$, and similarly for other forms of transitions.




\paragraph{Further notations}
Letters  $\R$, $\S$ range over relations.
We use infix notation for relations, e.g., 
$P \RR Q$ means that $(P,Q) \in \R$.
We use a tilde to denote a tuple, with countably many elements; thus
the tuple may also be infinite.
 All
notations  are  extended to tuples componentwise;
e.g., $\til P \RR \til Q$ means that $P_i \RR Q_i$, for  each  
component $i$  of the tuples $\til P$ and $\til Q$.
And $\ct{\til P}$ is the process obtained by replacing each hole
$\holei i$ of the  context $\qct$ with $P_i$.  
We write $
\ctx \R$ for the closure of a relation under contexts. Thus $P\: \ctx \R\: Q$
means that there are context $\qct$ and tuples $\til P,\til Q$ with
$P =  \ct{\til P}, Q =  \ct{\til Q}$ and    $\til P \RR \til Q$.
We use  symbol 
$\DSdefi$ for abbreviations. For instance, $P \DSdefi G $, where
$G$ is some expression, means that  $P$ stands
for the  expression
$G$.
If $\leq$ is a preorder, then  $\geq$  is its inverse (and
conversely).


\subsection{Bisimilarity and rooted bisimilarity}
\label{ss:BiEx}

The equivalences we consider here are mainly \emph{weak} ones, in that they
abstract from the number of internal steps being performed:
\begin{definition}%[bisimilarity]
\label{d:wb}
A process relation ${\R}$ is a \textbf{bisimulation} if, whenever
 $P\RR Q$, \hl{for all $\mu\in \mathscr{L}\cup\{\tau\}$} we have:
\begin{enumerate}
\item $P \arr\mu P'$ implies that there is $Q'$ such that $Q \Arcap \mu Q'$ and $P' \RR Q'$;
\item $Q \arr\mu Q'$,implies that there is $P'$ such that $P \Arcap
  \mu P'$ and $P' \RR Q'$\enspace.
\end{enumerate}
 $P$ and $Q$ are \textbf{bisimilar},
written as $P \wb Q$, if $P \RR Q$ for some bisimulation $\R$.
\end{definition}

We sometimes call bisimilarity the \emph{weak} one, to
distinguish it from \emph{strong} bisimilarity ($\sim$),
obtained by replacing in the above definition   the weak answer $
Q\Arcap\mu Q'$ with the strong  $Q \arr \mu Q'$.
Weak bisimilarity is not preserved by the sum operator (except for
guarded sums). For this, Milner introduced \emph{observational congruence}, also called \emph{rooted
  bisimilarity} \cite{Gorrieri:2015jt,Sangiorgi:2011ut}:
\begin{definition}%[rooted bisimilarity]
\label{d:rootedBisimilarity}
Two processes $P$ and $Q$ are \textbf{rooted bisimilar}, written as $P
\rapprox Q$, iff \hl{for all $\mu\in \mathscr{L}\cup\{\tau\}$}
\begin{enumerate}
 \item  $P \arr\mu P'$ implies that there is $Q'$ such that $Q
   \Arr\mu Q'$ and $P' \wb Q'$;
 \item  $Q \arr\mu Q'$ implies that there is $P'$ such that $P
   \Arr\mu P'$ and $P' \wb Q'$\enspace.
\end{enumerate}
\end{definition}
% Besides reducing the rooted bisimiarity of two processes to
% the bisimilarities of their first-step transition ends, this definition also brings a proof technique for proving the
% rooted bisimiarity by constructing a bisimulation:
% \begin{lemma}{(Rooted bisimilarity by constructing a bisimulation)}
% \label{l:obsCongrByWeakBisim}
% Given a (weak) bisimulation $\RR$, if two processes $P$ and $Q$
% satisfies the following properties:
% \begin{enumerate}
%  \item  $P \arr\mu P'$ implies that there is $Q'$ such that $Q
%    \Arr\mu Q'$ and $P' \RR Q'$;
% \item the converse of (1) on the actions from $Q$.
% \end{enumerate}
% then $P$ and $Q$ are rooted bisimilar, i.e.~$P \approx^c Q$.
% \end{lemma}

\begin{theorem}
\label{t:rapproxCongruence}
$\rapprox$ is a congruence in CCS, and it is the
coarsest congruence contained in $\approx$ \cite{van2005characterisation}.
\end{theorem}

\subsection{Expansions}
\label{s:expa}

\hl{The bisimulation proof method can be enhanced by means of \emph{up-to
techniques}. One of the most useful auxiliary relations in up-to
techniques is the \emph{expansion} relation} $\expa$ \cite{arun1992efficiency,sangiorgi2015equations}.
This is an asymmetric version
of $\wb$ where $P \expa Q$ means that $P \wb Q$,
but also that $Q$ achieves the same as $P$
with no more work, i.e.~with no more $\tau$ actions.
Intuitively, if $P \expa Q$, we can think of $Q$ as being
at least as fast as $P$
or, more generally, we can think that $P$ uses at least as many resources as $Q$.
\begin{definition}%[expansion]
\label{d:expa}
A process relation ${\R}$
  is an \textbf{expansion} if, whenever
we have $P\RR Q$, for all $\mu$
 \begin{enumerate}
 \item   $P \arr\mu P'$ implies that there is $Q'$ with $Q \arcap \mu
   Q'$
  and $P' \RR Q'$;
 \item
     $Q \arr\mu Q'$   implies that there is $P'$ with $P \Arr \mu
  P'$ and $P'
 \RR Q'$.
 \end{enumerate}
  $P$  {\em expands} $Q$, written as
 $P  \expa Q$,
 if $P \RR Q$ for some expansion $\R$.
 \end{definition}

Same as bisimilarity, the expansion preorder is preserved by all operators but (direct) sums.

% next file: contraction.tex

%\section{Equations and contractions}
\label{s:eq}

In the CCS syntax, 
a recursion $\recu A  P$ acts as a binder for $A$ in the body $P$. 
This gives rise, in the expected manner, to the notions of 
\emph{free} and \emph{bound} recursion variables in a CCS expression. 
For instance,  $X$ is free in $a.X + \recu Y (b.Y)$ while $Y$
is bound; \hl{And} in $a.X + \recu X (b.X)$, $X$ is both free and bound.
A \hl{term} without free variables is \hl{called} a \emph{process}.
% This setting does not cause any ambiguity, but sometimes leads to more
% complex proofs.} (see Section~\ref{sec:multivariate} for more details.)

In this paper (and the formalisation work), we use the agent
variables also as \emph{equation variables}. This eliminates the need of
another type for  equations, and we can reuse the existing
variable substitution operation (cf.~the SOS rule for the Recursion in
Fig.~\ref{f:LTSCCS}) for the substitution of equation variables.
For example, the result of substituting variable $X$ with $\nil$ in $a.X +
\recu X (b.X)$,  written $(a.X + \recu X (b.X)) \sub {\nil} X$, is
$a.\nil + \recu X (b.X)$ (with the part $\recu X (b.X)$
untouched). \Multivariate substitutions are written in the same syntax,
e.g. $E \sub {\til P} {\til X}$. Whenever $\til X$ is clear from the
context, we may also write $E[\til P]$ instead of $E \sub {\til P} {\til
  X}$ (and $E[P]$ for $E \sub P X$ if there is a single equation
variable $X$). 

% In fact, free agent variables have the same transitional behavior
% as the deadlock $\nil$ according to the SOS rules, as there is
% no rule for their transitions at all. In fact, most CCS theorems still hold if
% the involved CCS terms contain free variables. (The most notable
% exceptions are all versions of unique solution of equations/contracts,
% where all solutions must be \emph{pure} processes (i.e. no free variable).)
 
\subsection{Systems of equations}
\label{ss:SysEq}
When discussing equations it is standard to talk about `context'. This is a 
 a CCS expression  possibly containing  free variables that, however, may not occur within
the body of recursive definitions. 
Milner's ``unique solution of equations'' theorems~\cite{Mil89} intuitively
say that, if a context $C$
%  \hl{(i.e., a CCS expression with possibly
% free variables)}\footnote{\hl{Rigorously speaking, under our setting
% (i.e.~reusing free agent variables as equation variables) not all
% valid CCS expressions
% are valid contexts: those where free variables occur inside
% recusion operators must be all excluded. For instace, $\recu X (a.X +
% b.Y)$ is not a valid context with the variable $Y$. This special requirement
% is perfectly aligned with CCS literature using process constants, as any
% equation variable cannot occur inside the definition of any
% constant. In fact, all versions of ``unique solution of
% equations/contractions'' theorems do not hold if equation variables
% are allowed to occur inside any recursion operator.}} 
obeys certain conditions,
then all processes $P$ that satisfy the equation $P \wb \ct P$ are
bisimilar with each other.

\begin{definition}[equations] % Def 3.1
  \label{def:equation}
Assume that, for each $i$ of 
 a countable indexing set $I$, we have variables $X_i$, and expressions
$E_i$ possibly containing such variables $\cup_i \{ X_i\}$. Then 
$\{ X_i = E_i\}_{i\in I}$ is 
  a \emph{system of equations}. (There is one equation $E_i$ for each variable $X_i$.)
\end{definition}

The components of $\til P$ need not be
different from each other, as it must be for the variables $\til X$.
% If the system has infinitely many equations, the  tuples $\til P$
% and $\til X$ are infinite too. 

\begin{definition}[solutions and unique solutions]
  \label{def:solution}
Suppose $\{ X_i = E_i\}_{i\in I}$ is a system of equations: 
\begin{itemize}
\item
 $\til P$ is a \emph{solution of the system of equations (for $\wb$)} 
if for each $i$ it holds that $P_i \wb E_i [\til P]$;
\item The system has \emph{a unique solution for $\wb$}  if whenever 
 $\til P$ and $\til Q$ are both solutions then $\til P \wb \til Q$. 
\end{itemize} 
 \end{definition}
Similarily, the \emph{(unique) solution of a system of equations for $\sim$}
(or for $\rapprox$) can be obtained by replacing all occurrences of $\wb$
in above definition with $\sim$ and $\rapprox$, respectively.

For instance, the solution of the equation $X = a. X$ 
is the process
$R \DSdefi \recu A {\, (a. A)}$, and for any other solution $P$ we have $P \wb R$.
In contrast, the equation 
 $X = a|  X$ has solutions that may be quite different, for instance,
 $K$ and $K | b$, for $K \DSdefi \recu K {\, (a. K)}$. (Actually any process capable of
continuously performing $a$--actions is a solution of $X = a|  X$.)

%
% The unique solution of the system (1), modulo $\wb$,  is the constant $K \Defi a
% . K$:  for any other solution $P$ we have $P \wb K$.
% The unique solution of (2), modulo $\wb$, are the constants $K_1 , K_2$
% with $K_1 \Defi a . K_2$ and $K_2 \Defi b. K_1$; again, for any other
% pair of solutions $P_1,P_2$ we have $K_1 \wb P_1$ and $K_2 \wb P_2$.
%
Examples of systems that do not have unique solutions are: $X = X$, $X
= \tau . X$ and $X = a | X$.

\begin{definition}[guardedness of equations]
\label{def:guardness}
A system of equations $\{ X_i = E_i\}_{i\in I}$ is 
\begin{itemize}
\item \emph{weakly guarded} if, in each $E_i$, each occurrence of
  each $X_i$ is underneath a prefix;

\item \emph{guarded} if, in each $E_i$, each occurrence of
  each $X_i$ is underneath a \emph{visible} prefix;

\item \emph{sequential} if, in each $E_i$, each
  occurrence of each $X_i$ is only underneath prefixes and sums.
\end{itemize}
\end{definition}

In other words, if a system of equations is sequential, then for
each  $E_i$, any subexpression of $E_i$ in which $X_j $
appears, apart from $X_j$ itself, is a sum of prefixed expressions.
For instance,
\begin{itemize}
\item $X = \tau. X + \mu . \nil$ is sequential but not 
 guarded, because the guarding prefix for the variable
is not visible;
\item $X =  \ell . X | P$ is guarded but not sequential;
\item $X =  \ell . X + \tau. \res a (a .\outC b | a.\nil)$, as well as
$X = \tau . (a. X + \tau . b .X + \tau  )$ are both guarded and sequential.
\end{itemize}

Milner has  three versions of  ``unique solution of equations''
theorems, for $\sim$, $\wb$ and $\rapprox$, respectively, though only the
following two versions are explicitly mentioned in~\citep[p.~103, 158]{Mil89}:
\begin{theorem}[unique solution of equations for $\sim$]
\label{t:Mil89s1}
Let $E_i$ be weakly guarded with free variables in $\til X$,
and let ${\til P} \sim {\til E}\{\til P /\til X\}$,
  ${\til Q} \sim {\til E}\{\til Q /\til X\}$. Then ${\til P} \sim {\til Q}$.
\end{theorem}

\begin{theorem}[unique solution of equations for $\rapprox$]
\label{t:Mil89s3}
Let $E_i$ be guarded and sequential with free
variables in $\til X$, and let ${\til P} \rapprox {\til E}\{\til P /\til X\}$,
  ${\til Q} \rapprox {\til E}\{\til Q /\til X\}$. Then ${\til P} \rapprox {\til Q}$.
\end{theorem}

The version of Milner's unique-solution theorem for $\wb$ further requires
that all sums are guarded:
% \footnote{But if the CCS syntax were defined with only guarded
%   sums, i.e., $\sum_{i\in I} \mu_i.P_i$ as
%   in~\cite{sangiorgi2015equations}, this addition
%   requirement disappears automatically, and we can even say $\wb$ is indeed a
%   congruence.}:
\begin{theorem}[unique solution of equations for $\wb$]
\label{t:Mil89}
Let $E_i$ be guarded and sequential with free
variables in $\til X$, and let ${\til P} \wb {\til E}\{\til P /\til X\}$,
  ${\til Q} \wb {\til E}\{\til Q /\til X\}$. Then ${\til P} \wb {\til Q}$.
\end{theorem}

The proof of the theorem above exploits an invariance
property on immediate derivatives
of guarded and sequential expressions, and then extracts a bisimulation
(up to bisimilarity) out
of the solutions of the system.
To see the need of the sequentiality  condition, consider
 the equation (from \cite{Mil89}) $X = \res a (a. X | \outC a)$
where $X$ is guarded but not sequential. Any process that does not use
$a$ is a solution, e.g. $\nil$ and $b.\nil$.

For more details of above three theorems, see Section~\ref{ss:part2}
for the \univariate case and
Section~\ref{sec:multivariate} for the \multivariate case.

%% next file: contraction.tex
 % Systems of  equations
\subsection{Expansions and Contractions}
\label{s:mcontr}

Milner's ``unique solution of equations'' theorem for $\wb$
(Theorem~\ref{t:Mil89})
brings a new proof technique for proving (weak) bisimilarities. However, it has
 limitations: the equations must be guarded and sequential. (\hl{Moreover,}
all sums where equation variables appear must be guarded sums.)
This limits the usefulness of the technique, since
the occurrences of other operators \hl{using equation} variables, such as parallel
composition and restriction,
in general cannot be \hl{eliminated}. % without changing the meaning of equations
The constraints \hl{in} Theorem~\ref{t:Mil89}, however, can be
weakened if we move from equations to a special kind of inequations called
  \emph{contractions}.

Intuitively, the bisimilarity contraction $\mcontrBIS$ is a preorder
\hlD{in which} $P \mcontrBIS Q$ holds if $P \wb Q$ and, in addition, 
\emph{$Q$ has the possibility of being at least as efficient as $P$} (as far as
$\tau$-actions are performed).
The process $Q$, however, may be nondeterministic and may have other ways
\hl{to do} the same work, \hl{ways} which could be \hl{slower} (i.e., involving
more $\tau$-actions than those performed by $P$).
% Thus, in contrast with expansion,  we cannot really say that `$Q$ is more efficient than
% $P$'.

\begin{definition}[contraction]
\label{d:BisCon}
A process relation ${\R}$ 
 is a \emph{(bisimulation) contraction} if, whenever $P\RR Q$,

\begin{enumerate}
\item $P \arr\mu P'$ implies \hl{that} there is $Q'$ \hl{with} $Q \arcap \mu
  Q'$ and $P' \RR Q'$;
\item $Q \arr\mu Q'$ implies \hl{that} there is $P'$ \hl{with} $P \Arcap \mu
 P'$ and $P' \wb Q'$.
\end{enumerate}
Two processes $P$ and $Q$ are in the \emph{bisimilarity
contraction}, written as $P \mcontrBIS Q$,
if $P\ \R\ Q$ for some contraction $\R$.
Sometimes we write $\mexpaBIS$ for the inverse of $\mcontrBIS$.
\end{definition}
In clause (1) \hl{of the above definition}, $Q$ is required to match the challenge
transition \hl{of $P$} with at most one transition.
This makes sure that $Q$ is capable of mimicking % verb: mimic
$P$'s work at least as efficiently as $P$. 
In contrast, \hl{clause (2) entirely ignores efficiency on the challenges from $Q$:}
the final derivatives are required to be related by $\wb$, rather than by $\R$.

Bisimilarity contraction is coarser than bisimilarity expansion
$\expa$~\cite{arun1992efficiency,sangiorgi2015equations}, \hl{one of the
most useful auxiliary relations in up-to techniques}:
\begin{definition}[expansion]
\label{d:expa}
A process relation ${\R}$
  is an \emph{expansion} if, whenever $P\RR Q$,
 \begin{enumerate}
 \item   $P \arr\mu P'$ implies that there is $Q'$ with $Q \arcap \mu  Q'$
  and $P' \RR Q'$;
 \item $Q \arr\mu Q'$ implies that there is $P'$ with $P \Arr \mu P'$ and $P' \RR Q'$.
 \end{enumerate}
Two processes $P$ and $Q$ are in the \emph{bisimilarity
  expansion}, written as $P \expa Q$, if $P \RR Q$ for some expansion $\R$.
 \end{definition}
\hl{Bisimilarity expansion} is widely used in proof techniques for bisimilarity.
\hl{It} intuitively refines bisimilarity by 
formalising the idea of ``efficiency'' between processes.
Clause (1) is the same in the both preorders, while in clause (2) expansion \hl{requires}
$P \Arr \mu P'$, rather than $P \Arcap \mu P'$.
\hl{Moreover,} in clause (2) of Def.~\ref{d:BisCon} the final derivatives
are simply required to be bisimilar ($P' \wb Q'$).
Intuitively, $P \expa Q$ holds if $P\wb Q$ and, in addition, \emph{$Q$
  is always at least as efficient as $P$}.

\begin{example}
\label{exa:contr}
We have %\mcontrBIS a + \tau^n . a $
 $ a \not  \mcontrBIS \tau. a$. However,
$a+ \tau . a \mcontrBIS a$, as well as its converse, 
$  a \mcontrBIS a +
\tau . a $. Indeed, if $P \wb Q$ then 
$  P  \mcontrBIS P +Q$. The last two relations do not hold with 
$\expa$, which explains the strictness of the inclusion
 ${\expa} \subset {\mcontrBIS}$. 
% The inclusion is strict: for instance
% $a+ \tau . a \mcontrBIS a$, where $\mcontrBIS$ cannot be replaced by
%  $\contr$. Also the converse of  $a+ \tau . a \mcontrBIS a$ holds, namely
% $  a \mcontrBIS a +
% \tau . a $. However, we have %\mcontrBIS a + \tau^n . a $
%  $ a \not  \mcontrBIS \tau. a$
\end{example} 

\hlD{Bisimilarity expansion and bisimilarity contraction are both
preorders.}
Similarily with (weak) bisimilarity, both the expansion and the
contraction preorders are preserved by all CCS operators except the
summation. The proofs are similar \hlD{to those for bisimilarity},
see, e.g.~\cite{sangiorgi2017equations} \hl{for details.}

% next file: unique.tex

\subsection{Systems of contractions}
\label{ss:SysContr}

A \emph{system of contractions} is defined as a system of equations,
except that the contraction symbol $\mcontr$ is used in the place of
the equality symbol $=$. Thus a system of contractions is a set 
$\{  X_i \mcontr E_i\}_{i\in I}$
where $I$ is an  indexing set and expressions
$E_i$  may contain the  \behavC\  variables 
$\{  X_i\}_{i\in I}$.

\begin{definition}
\label{d:uniContra}
Given a system of contractions 
$\{  X_i \mcontr E_i\}_{i\in I}$, 
 we say that:
\begin{itemize}
\item $\til P$ is a \emph{solution (for $\mcontrBIS$) of the 
 system of contractions} if $\til P \mcontrBIS \til E [\til P]$;
\item the system has \emph{a unique solution (for $\approx$)}
if $\til P \approx \til Q$ whenever $\til P$ and $\til Q$ are both solutions.
\end{itemize}
\end{definition}

The guardedness of contractions follows Def.~\ref{def:guardness} (for equations).
% \begin{definition}
% \label{d:guarded}
% A system of contractions $\{  X_i \mcontr E_i\}_{i\in I}$
%  is
% \emph{weakly guarded}
% if,  in each    $E_i$, each occurrence of
% a \behavC\ variable is underneath a prefix.

% The system use \emph{weakly-guarded sums} if 
% each $E_i$ only makes use of guarded sums.
% \end{definition}

\begin{lemma}
\label{l:uptocon}
Suppose $\til P$ and $\til Q$ are solutions  for $\mcontrBIS$
 of a system of weakly-guarded contractions that uses 
weakly-guarded sums.
For any context $\qct$  that uses 
weakly-guarded sums,
if  $\ct{\til P}\Arr{\mu}  R$,
 then 
there is a context $\qctp$  that uses 
weakly-guarded sums
such that $R \mcontrBIS \ctp{\til P}$ and $\ct{\til Q} \Arcap{\mu}
 \wb \ctp{\til Q}$.\footnote{There's no typo here: $\ct{\til Q} \Arcap{\mu} \wb \ctp{\til
     Q}$ means $\exists {\til R}.\; \ct{\til Q} \Arcap{\mu} {\til R}
   \wb \ctp{\til Q}$. Same as in Lemma~\ref{l:ruptocon}.}
\end{lemma}

\begin{proof}{(sketch from \cite{sangiorgi2017equations})}
Let $n$ be the length of the transition $\ct{\til P}\Arr\mu R$  (the
number of `strong steps' of which it is composed), and  
let $\ctpp {\til P}$ and $\ctpp {\til Q}$  be the processes obtained
from  $\ct {\til P}$ and $\ct {\til Q}$ by unfolding the definitions
of the contractions $n$ times. Thus in $\qctpp$ each hole is
underneath at least $n$ prefixes, and cannot contribute to an action
in the first $n$ transitions; moreover all the contexts have only
weakly-guarded sums.

We have $\ct{\til P} \mcontrBIS \ctpp{\til P}$, and 
$\ct{\til Q} \mcontrBIS \ctpp{\til Q}$, 
 by the substitutivity  properties of $\mcontrBIS$ (we exploit here
 the syntactic constraints on sums). Moreover,
 since each hole of the  context $\qctpp$ is underneath at least $n$
 prefixes, applying  
the definition
 of $ \mcontrBIS$ on the transition 
 $\ct{\til P}\Arr{\mu}  R$, we infer the existence
 of $\qctp$ such that 
$
\ctpp{\til P}\Arcap{\mu} \ctp{\til P} \mexpaBIS R
$
and 
$
\ctpp{\til Q}\Arcap{\mu}  \ctp{\til Q} 
. $
Finally, again applying the definition of $\mcontrBIS$ on 
$\ct{\til Q} \mcontrBIS \ctpp{\til Q}$, 
we derive 
$
\ct{\til Q}\Arcap{\mu}  \wb \ctp{\til Q} 
.$
\end{proof}

\begin{theorem}[unique solution of contractions for $\wb$]
\label{t:contraBisimulationU}
A system of weakly-guarded contractions
having only weakly-guarded sums, has a unique solution for $\wb$.
\end{theorem}

\begin{proof}{(sketch from \cite{sangiorgi2017equations})}
Suppose $\til P$ and $\til Q$ are two such solutions (for $\wb$) and consider
the relation
\begin{equation}
\label{eq:R}
\R \DSdefi \{ 
(R,S) \st R \wb \ct{\til P}, S \wb \ct{\til Q} \mbox{ for some context
$\qct$ (having only weakly-guarded sums)} \} \enspace.
\end{equation}
We show that $\R$ is a bisimulation. \hl{Suppose $R\ \R\ S$ vis the context
$C$}, and $R \arr{\mu} R'$. We have to find $S'$ with $S \Arcap{\mu}
S'$ and $R'\ \R\ S'$. From $R \wb C[{\til P}]$, we derive $C[{\til P}]
\Arcap{\mu} R'' \wb R'$ for some $R''$. By Lemma~\ref{l:uptocon},
there is $C'$ with $R'' \mcontrBIS C'[{\til P}]$ and $C[{\til Q}]
\Arcap{\mu} \wb C'[{\til Q}]$. Hence, by definition of $\wb$, there is
also $S'$ with $S \Arcap{\mu} S' \wb C'[{\til Q}]$. This closes the
proof, as we have $R' \wb C'[{\til P}]$ and $S' \wb C'[{\til Q}]$.
\end{proof}

\section{Rooted contraction}
\label{ss:new}

The unique solution theorem of Section~\ref{ss:SysContr} requires a
constrained syntax for sums, due to the congruence and precongruence
problems of bisimilarity and contraction with such operator. 
We show here that the constraints can be
removed by moving to the induced congruence and precongruence, the
latter called \emph{rooted contraction}:
\begin{definition}
\label{d:rcontra}
Two processes $P$ and $Q$ are in \emph{rooted contraction}, written as
 $P\rcontr Q$, if
\begin{enumerate}
\item $P \arr\mu P'$ implies that there is $Q'$ with $Q \arr \mu Q'$
 and $P'\mcontrBIS Q'$;
\item $Q \arr\mu Q'$   implies that there is $P'$ with $P \Arr \mu
 P'$ and $P' \wb Q'$.
\end{enumerate}
\end{definition}

%Above definition adapts the definition of rooted
%bisimilarity on top of that of the  contraction preorder
%$\mcontrBIS$.  %% Reviewer said this sentance is unclear. I too think so.

\hl{The discovery of this definition is with help of HOL theorem
  prover and
the following two principles:} (1) Its definition must not be recursive,
instead it should resemble the definition of rooted bisimilarity
$\approx^c$ in Def.~\ref{d:rootedBisimilarity};
(2) It must be built on top of existing \emph{contracts}
relation $\mcontrBIS$, which we believe it's the \emph{right} one
because of its completeness. \hl{Multiple candicates were quickly tested,
finally only above definition is proven to be a precongruence, as the
following theorem states.} (The proof of this result is along the lines of the analogous result
for rooted bisimilarity with respect to bisimilarity.)

\begin{theorem}
\label{t:rcontrPrecongruence}
$\rcontr$ is a precongruence in CCS, and it is the
coarsest precongruence contained in $\contr$.
\end{theorem}  

For a system of rooted contractions, the meaning of 
``solution for $\rcontr$'' and of \emph{a unique solution for $\rapprox$}
is the expected one --- just replace in Definition~\ref{d:uniContra}  the preorder 
$\contr$ with $\rcontr$, and the equivalence 
$\approx$ with $\rapprox$.
%
For this new relation, the analogous of Lemma~\ref{l:uptocon} and of
Theorem~\ref{t:contraBisimulationU} can now be stated without constraints on the sum
operator.
The schema of the proofs is almost the same, because all needed
properties of $\rcontr$ in the proof is its precongruence, which is
now true for unrestricted contexts using direct sums:

\begin{lemma}
\label{l:ruptocon}
Suppose $\til P$ and $\til Q$ are solutions  for $\rcontr$ 
 of a system of weakly-guarded
contractions.
For any context $\qct$, 
if  $\ct{\til P}\Arr{\mu}  R$,
 then 
there is a  context $\qctp$
such that $R \mcontrBIS \ctp{\til P}$ and  $\ct{\til Q} \Arr{\mu}
 \wb \ctp{\til Q}$.
\end{lemma}

\begin{theorem}[unique solution of contractions for $\rapprox$]
\label{t:rcontraBisimulationU}
A system of weakly-guarded contractions has a unique solution 
 for $\rapprox$. (thus also for $\wb$)
\end{theorem} 

\begin{proof}
We first follow the same steps as in the proof of Theorem~\ref{t:contraBisimulationU} to show the relation $\R$ (now
with $\rcontr$ and unrestrict context $C$) in (\ref{eq:R}) is bisimulation,
exploting Lemma~\ref{l:ruptocon}. \hl{Then it remains to show that,} for
any two process $P$ and $Q$ with action $\mu$, if $P \arr{\mu} P'$ then
there is $Q'$ such that $Q \Arr{\mu} Q'$ (not $Q \Arcap{\mu} Q'$!) and
$P'\ \R\ Q'$, and also for the converse direction, exploting Lemma
4.13 of \cite{Mil89} \hl{(unexpected!)}. \hl{By definition of
\emph{bisimulation} (not $\wb$!) and $\approx^c$, we actually proved $P
\approx^c Q$ instead of $P \wb Q$.}
\end{proof}

%%%% -*- Mode: LaTeX -*-
%%
%% This is the draft of the 2nd part of EXPRESS/SOS 2018 paper, co-authored by
%% Prof. Davide Sangiorgi and Chun Tian.

\subsection{Unique solution of contractions}

A delicate point in the formalisation of the results about unique solution of
contractions are the proof of Lemma~\ref{l:ruptocon} and lemmas alike;
in particular, there is
 an induction on the length of weak transitions. 
For this, rather than 
 introducing a refined form of weak transition relation
enriched with its length, 
we found it more elegant  to  work with traces
(a motivation for this is to set the ground for extensions of this
formalisation work to trace equivalence in place of bisimilarity).

% but such a non-standard relation finds no
% other uses beside proving our target theorem. Another way is to use 
% traces instead, as it shows more clearly all passing actions inside a
% trace, making formal reasoning easier.

% We represent a trace by the initial process, the final derivative, and
% the list of actions performed. 
% To formalise this, 
% we first introduce 
% the Reflexive Transitive Closure with a
% List (LRTC);

A trace is represented by the initial and final processes, plus
a list of actions  so performed.
For this, we first 
 define \hl{the concept of label-accumulated reflexive transitive closure
 (\texttt{LRTC})}.
Given a labeled transition relation \texttt{R} on CCS, \texttt{LRTC R} is
a label-accumulated relation representing the trace of transitions:
\begin{alltt}
\HOLConst{LRTC} \HOLFreeVar{R} \HOLFreeVar{a} \HOLFreeVar{l} \HOLFreeVar{b} \HOLSymConst{\HOLTokenEquiv{}}
\HOLSymConst{\HOLTokenForall{}}\HOLBoundVar{P}.
    (\HOLSymConst{\HOLTokenForall{}}\HOLBoundVar{x}. \HOLBoundVar{P} \HOLBoundVar{x} [] \HOLBoundVar{x}) \HOLSymConst{\HOLTokenConj{}}
    (\HOLSymConst{\HOLTokenForall{}}\HOLBoundVar{x} \HOLBoundVar{h} \HOLBoundVar{y} \HOLBoundVar{t} \HOLBoundVar{z}. \HOLFreeVar{R} \HOLBoundVar{x} \HOLBoundVar{h} \HOLBoundVar{y} \HOLSymConst{\HOLTokenConj{}} \HOLBoundVar{P} \HOLBoundVar{y} \HOLBoundVar{t} \HOLBoundVar{z} \HOLSymConst{\HOLTokenImp{}} \HOLBoundVar{P} \HOLBoundVar{x} (\HOLBoundVar{h}\HOLSymConst{::}\HOLBoundVar{t}) \HOLBoundVar{z}) \HOLSymConst{\HOLTokenImp{}}
    \HOLBoundVar{P} \HOLFreeVar{a} \HOLFreeVar{l} \HOLFreeVar{b}\hfill{[LRTC_DEF]}
\end{alltt}
\hl{The trace relation for CCS can be then obtained
 by calling \texttt{LRTC} on the (strong, or single-step) labeled transition
 relation \texttt{TRANS} ($\overset{\mu}{\rightarrow}$) defined by SOS rules}:
\begin{alltt}
\HOLConst{TRACE} \HOLSymConst{=} \HOLConst{LRTC} \HOLConst{TRANS}\hfill{[TRACE_def]}
\end{alltt}

\hl{If the list of actions is empty, that means that there is no transition and hence,}
if there is at most one visible action (i.e., a label) in the list of actions,
then the trace is also a weak transition. Here
we have to distinguish between two cases: no label and unique label (in
the list of actions). The definition of ``no
label'' in an action list is easy (here \texttt{MEM} tests if a given element is a member of a list):
\begin{alltt}
\HOLConst{NO_LABEL} \HOLFreeVar{L} \HOLSymConst{\HOLTokenEquiv{}} \HOLSymConst{\HOLTokenNeg{}}\HOLSymConst{\HOLTokenExists{}}\HOLBoundVar{l}. \HOLConst{MEM} (\HOLConst{label} \HOLBoundVar{l}) \HOLFreeVar{L}\hfill{[NO_LABEL_def]}
\end{alltt}

The definition of ``unique label'' \hl{can be done in many ways, the
following definition (a suggestion from Robert Beers)
avoides any counting or filtering in the list.}
It says that a label is unique in a list of actions if and only if there is no
label in the rest of list:
\begin{alltt}
\HOLConst{UNIQUE_LABEL} \HOLFreeVar{u} \HOLFreeVar{L} \HOLSymConst{\HOLTokenEquiv{}}
\HOLSymConst{\HOLTokenExists{}}\HOLBoundVar{L\sb{\mathrm{1}}} \HOLBoundVar{L\sb{\mathrm{2}}}. (\HOLBoundVar{L\sb{\mathrm{1}}} \HOLSymConst{\HOLTokenDoublePlus} [\HOLFreeVar{u}] \HOLSymConst{\HOLTokenDoublePlus} \HOLBoundVar{L\sb{\mathrm{2}}} \HOLSymConst{=} \HOLFreeVar{L}) \HOLSymConst{\HOLTokenConj{}} \HOLConst{NO_LABEL} \HOLBoundVar{L\sb{\mathrm{1}}} \HOLSymConst{\HOLTokenConj{}} \HOLConst{NO_LABEL} \HOLBoundVar{L\sb{\mathrm{2}}}\hfill{[UNIQUE_LABEL_def]}
\end{alltt}

The final relationship between traces and weak transitions is stated
and proved in the following theorem
(where the  variable $acts$ stands for
a list of actions); 
it says, a weak transition $P\overset{u}{\Rightarrow}P'$ is also a
trace $P\overset{acts}{\longrightarrow}P'$ with a
 non-empty action list $acts$, in which either there is no label (for $u = \tau$), or 
$u$ is the unique label (for $u \neq \tau$):
%\begin{lemma}
% A weak transition $P\overset{u}{\Rightarrow}P'$ is a just trace with non
% empty action list: 1) without any visible label, if $u = \tau$, or 2)
% $u$ is the unique label in the list, if $u \neq \tau$.
\begin{alltt}
\HOLTokenTurnstile{} \HOLFreeVar{P} \HOLTokenWeakTransBegin\HOLFreeVar{u}\HOLTokenWeakTransEnd \HOLFreeVar{P\sp{\prime}} \HOLSymConst{\HOLTokenEquiv{}}
   \HOLSymConst{\HOLTokenExists{}}\HOLBoundVar{acts}.
       \HOLConst{TRACE} \HOLFreeVar{P} \HOLBoundVar{acts} \HOLFreeVar{P\sp{\prime}} \HOLSymConst{\HOLTokenConj{}} \HOLSymConst{\HOLTokenNeg{}}\HOLConst{NULL} \HOLBoundVar{acts} \HOLSymConst{\HOLTokenConj{}}
       \HOLKeyword{if} \HOLFreeVar{u} \HOLSymConst{=} \HOLSymConst{\ensuremath{\tau}} \HOLKeyword{then} \HOLConst{NO_LABEL} \HOLBoundVar{acts} \HOLKeyword{else} \HOLConst{UNIQUE_LABEL} \HOLFreeVar{u} \HOLBoundVar{acts}\hfill{[WEAK_TRANS_AND_TRACE]}
\end{alltt}
%\end{lemma}

Now the formalised version of Lemma~\ref{l:uptocon}:
\hfill{\texttt{[UNIQUE_SOLUTION_OF_CONTRACTIONS_LEMMA]}\vspace{-1em}
\begin{alltt}
\begin{small}
\HOLTokenTurnstile{} (\HOLSymConst{\HOLTokenExists{}}\HOLBoundVar{E}. \HOLConst{WGS} \HOLBoundVar{E} \HOLSymConst{\HOLTokenConj{}} \HOLFreeVar{P} \HOLSymConst{\HOLTokenContracts{}} \HOLBoundVar{E} \HOLFreeVar{P} \HOLSymConst{\HOLTokenConj{}} \HOLFreeVar{Q} \HOLSymConst{\HOLTokenContracts{}} \HOLBoundVar{E} \HOLFreeVar{Q}) \HOLSymConst{\HOLTokenImp{}}
   \HOLSymConst{\HOLTokenForall{}}\HOLBoundVar{C}.
       \HOLConst{GCONTEXT} \HOLBoundVar{C} \HOLSymConst{\HOLTokenImp{}}
       (\HOLSymConst{\HOLTokenForall{}}\HOLBoundVar{l} \HOLBoundVar{R}.
            \HOLBoundVar{C} \HOLFreeVar{P} \HOLTokenWeakTransBegin\HOLConst{label} \HOLBoundVar{l}\HOLTokenWeakTransEnd \HOLBoundVar{R} \HOLSymConst{\HOLTokenImp{}}
            \HOLSymConst{\HOLTokenExists{}}\HOLBoundVar{C\sp{\prime}}.
                \HOLConst{GCONTEXT} \HOLBoundVar{C\sp{\prime}} \HOLSymConst{\HOLTokenConj{}} \HOLBoundVar{R} \HOLSymConst{\HOLTokenContracts{}} \HOLBoundVar{C\sp{\prime}} \HOLFreeVar{P} \HOLSymConst{\HOLTokenConj{}}
                (\HOLConst{WEAK_EQUIV} \HOLSymConst{\HOLTokenRCompose{}} (\HOLTokenLambda{}\HOLBoundVar{x} \HOLBoundVar{y}. \HOLBoundVar{x} \HOLTokenWeakTransBegin\HOLConst{label} \HOLBoundVar{l}\HOLTokenWeakTransEnd \HOLBoundVar{y})) (\HOLBoundVar{C} \HOLFreeVar{Q})
                  (\HOLBoundVar{C\sp{\prime}} \HOLFreeVar{Q})) \HOLSymConst{\HOLTokenConj{}}
       \HOLSymConst{\HOLTokenForall{}}\HOLBoundVar{R}.
           \HOLBoundVar{C} \HOLFreeVar{P} \HOLTokenWeakTransBegin\HOLSymConst{\ensuremath{\tau}}\HOLTokenWeakTransEnd \HOLBoundVar{R} \HOLSymConst{\HOLTokenImp{}}
           \HOLSymConst{\HOLTokenExists{}}\HOLBoundVar{C\sp{\prime}}.
               \HOLConst{GCONTEXT} \HOLBoundVar{C\sp{\prime}} \HOLSymConst{\HOLTokenConj{}} \HOLBoundVar{R} \HOLSymConst{\HOLTokenContracts{}} \HOLBoundVar{C\sp{\prime}} \HOLFreeVar{P} \HOLSymConst{\HOLTokenConj{}}
               (\HOLConst{WEAK_EQUIV} \HOLSymConst{\HOLTokenRCompose{}} \HOLConst{EPS}) (\HOLBoundVar{C} \HOLFreeVar{Q}) (\HOLBoundVar{C\sp{\prime}} \HOLFreeVar{Q})
\end{small}
\end{alltt}
\vspace{-1em}
Traces are actually used in the proof of above lemma via 
the following ``unfolding lemma'':\vspace{-1em}
\begin{alltt}
\begin{small}
\HOLTokenTurnstile{} \HOLConst{GCONTEXT} \HOLFreeVar{C} \HOLSymConst{\HOLTokenConj{}} \HOLConst{WGS} \HOLFreeVar{E} \HOLSymConst{\HOLTokenConj{}} \HOLConst{TRACE} ((\HOLFreeVar{C} \HOLSymConst{\HOLTokenCompose} \HOLConst{FUNPOW} \HOLFreeVar{E} \HOLFreeVar{n}) \HOLFreeVar{P}) \HOLFreeVar{xs} \HOLFreeVar{P\sp{\prime}} \HOLSymConst{\HOLTokenConj{}}
   \HOLConst{LENGTH} \HOLFreeVar{xs} \HOLSymConst{\HOLTokenLeq{}} \HOLFreeVar{n} \HOLSymConst{\HOLTokenImp{}}
   \HOLSymConst{\HOLTokenExists{}}\HOLBoundVar{C\sp{\prime}}.
       \HOLConst{GCONTEXT} \HOLBoundVar{C\sp{\prime}} \HOLSymConst{\HOLTokenConj{}} (\HOLFreeVar{P\sp{\prime}} \HOLSymConst{=} \HOLBoundVar{C\sp{\prime}} \HOLFreeVar{P}) \HOLSymConst{\HOLTokenConj{}}
       \HOLSymConst{\HOLTokenForall{}}\HOLBoundVar{Q}. \HOLConst{TRACE} ((\HOLFreeVar{C} \HOLSymConst{\HOLTokenCompose} \HOLConst{FUNPOW} \HOLFreeVar{E} \HOLFreeVar{n}) \HOLBoundVar{Q}) \HOLFreeVar{xs} (\HOLBoundVar{C\sp{\prime}} \HOLBoundVar{Q})\hfill{[unfolding_lemma4]}
\end{small}
\end{alltt}
\vspace{-1em}
It roughly says, for any context $C$ and weakly-guarded context
$E$, if $C [\, E^n[P]\,] \overset{xs}{\Longrightarrow} P'$ and the length
of actions $xs \leqslant n$, then $P$ has the form of $C'[P]$ (meaning
that $P$ is not touched during the transitions). Traces are used for
reasoning about the \hl{number} of intermediate actions in weak
transitions. For instance, from Def.~\ref{d:BisCon}, \hl{it is easy
to see that, a weak transition either becomes shorter
or remains the same when moving between $\mcontrBIS$-related processes}.
\hl{This property is essential} in the proof of
Lemma~\ref{l:uptocon}. We show only one such lemma, for the case of
$\tau$-transitions passing into $\mcontrBIS$ (from left to right):
\begin{alltt}
\HOLTokenTurnstile{} \HOLFreeVar{P} \HOLSymConst{\HOLTokenContracts{}} \HOLFreeVar{Q} \HOLSymConst{\HOLTokenImp{}}
   \HOLSymConst{\HOLTokenForall{}}\HOLBoundVar{xs} \HOLBoundVar{P\sp{\prime}}.
       \HOLConst{TRACE} \HOLFreeVar{P} \HOLBoundVar{xs} \HOLBoundVar{P\sp{\prime}} \HOLSymConst{\HOLTokenConj{}} \HOLConst{NO_LABEL} \HOLBoundVar{xs} \HOLSymConst{\HOLTokenImp{}}
       \HOLSymConst{\HOLTokenExists{}}\HOLBoundVar{xs\sp{\prime}} \HOLBoundVar{Q\sp{\prime}}.
           \HOLConst{TRACE} \HOLFreeVar{Q} \HOLBoundVar{xs\sp{\prime}} \HOLBoundVar{Q\sp{\prime}} \HOLSymConst{\HOLTokenConj{}} \HOLBoundVar{P\sp{\prime}} \HOLSymConst{\HOLTokenContracts{}} \HOLBoundVar{Q\sp{\prime}} \HOLSymConst{\HOLTokenConj{}} \HOLConst{LENGTH} \HOLBoundVar{xs\sp{\prime}} \HOLSymConst{\HOLTokenLeq{}} \HOLConst{LENGTH} \HOLBoundVar{xs} \HOLSymConst{\HOLTokenConj{}}
           \HOLConst{NO_LABEL} \HOLBoundVar{xs\sp{\prime}}\hfill{[contracts_AND_TRACE_tau]}
\end{alltt}

\hl{With all above lemmas, we can thus finally prove Theorem~\ref{t:contraBisimulationU}:}
\begin{alltt}
\HOLTokenTurnstile{} \HOLConst{WGS} \HOLFreeVar{E} \HOLSymConst{\HOLTokenImp{}} \HOLSymConst{\HOLTokenForall{}}\HOLBoundVar{P} \HOLBoundVar{Q}. \HOLBoundVar{P} \HOLSymConst{\HOLTokenContracts{}} \HOLFreeVar{E} \HOLBoundVar{P} \HOLSymConst{\HOLTokenConj{}} \HOLBoundVar{Q} \HOLSymConst{\HOLTokenContracts{}} \HOLFreeVar{E} \HOLBoundVar{Q} \HOLSymConst{\HOLTokenImp{}} \HOLBoundVar{P} \HOLSymConst{\HOLTokenWeakEQ} \HOLBoundVar{Q}
\hfill{[UNIQUE_SOLUTION_OF_CONTRACTIONS]}
\end{alltt}
\vspace{-2ex}

\subsection{Unique solution of rooted contractions}

The formal proof of ``unique solution of rooted contractions theorem''
(Theorem~\ref{t:rcontraBisimulationU}) has the
same initial proof steps as Theorem~\ref{t:contraBisimulationU}; 
it then requires a
few more steps to handle  rooted bisimilarity in the conclusion. 
Overall  the
two proofs are very similar, mostly because the only property we need
from (rooted) contraction is its precongruence. 
 Below is the formally verified version of
Theorem~\ref{t:rcontraBisimulationU}
(having proved
the precongruence of rooted contraction, 
we can use  weakly-guarded expressions,   without constraints on  sums;
that is, \texttt{WG} in place of \texttt{WGS}):
\begin{alltt}
\HOLTokenTurnstile{} \HOLConst{WG} \HOLFreeVar{E} \HOLSymConst{\HOLTokenImp{}} \HOLSymConst{\HOLTokenForall{}}\HOLBoundVar{P} \HOLBoundVar{Q}. \HOLBoundVar{P} \HOLSymConst{\HOLTokenObsContracts} \HOLFreeVar{E} \HOLBoundVar{P} \HOLSymConst{\HOLTokenConj{}} \HOLBoundVar{Q} \HOLSymConst{\HOLTokenObsContracts} \HOLFreeVar{E} \HOLBoundVar{Q} \HOLSymConst{\HOLTokenImp{}} \HOLBoundVar{P} \HOLSymConst{\HOLTokenObsCongr} \HOLBoundVar{Q}
\hfill{[UNIQUE_SOLUTION_OF_ROOTED_CONTRACTIONS]}
\end{alltt}

Having removed the  constraints on sums, the result is
 similar to Milner's original `unique solution of
equations' theorem for \emph{strong} bisimilarity ($\sim$)~--- 
the same weakly guarded context (\texttt{WG}) is required:
\begin{alltt}
\HOLTokenTurnstile{} \HOLConst{WG} \HOLFreeVar{E} \HOLSymConst{\HOLTokenImp{}} \HOLSymConst{\HOLTokenForall{}}\HOLBoundVar{P} \HOLBoundVar{Q}. \HOLBoundVar{P} \HOLSymConst{\HOLTokenStrongEQ} \HOLFreeVar{E} \HOLBoundVar{P} \HOLSymConst{\HOLTokenConj{}} \HOLBoundVar{Q} \HOLSymConst{\HOLTokenStrongEQ} \HOLFreeVar{E} \HOLBoundVar{Q} \HOLSymConst{\HOLTokenImp{}} \HOLBoundVar{P} \HOLSymConst{\HOLTokenStrongEQ} \HOLBoundVar{Q}\hfill{[STRONG_UNIQUE_SOLUTION]}
\end{alltt}
In contrast, Milner's ``unique solution of
equations'' theorem for rooted bisimilarity ($\rapprox$)
has more severe constraints (both strongly guarded and sequential):
% Or our Theorem~\ref{t:rcontraBisimulationU} can be seen as a more
% applicable version of Milner's ``unique solution of
% equations'' theorem for rooted bisimilarity ($\rapprox$), which has more
% restrictions on equations:
\begin{alltt}
\HOLTokenTurnstile{} \HOLConst{SG} \HOLFreeVar{E} \HOLSymConst{\HOLTokenConj{}} \HOLConst{SEQ} \HOLFreeVar{E} \HOLSymConst{\HOLTokenImp{}} \HOLSymConst{\HOLTokenForall{}}\HOLBoundVar{P} \HOLBoundVar{Q}. \HOLBoundVar{P} \HOLSymConst{\HOLTokenObsCongr} \HOLFreeVar{E} \HOLBoundVar{P} \HOLSymConst{\HOLTokenConj{}} \HOLBoundVar{Q} \HOLSymConst{\HOLTokenObsCongr} \HOLFreeVar{E} \HOLBoundVar{Q} \HOLSymConst{\HOLTokenImp{}} \HOLBoundVar{P} \HOLSymConst{\HOLTokenObsCongr} \HOLBoundVar{Q}
\hfill{[OBS_UNIQUE_SOLUTION]}
\end{alltt}
\vspace{-4ex}


\section{Conclusions}



\bibliographystyle{eptcs}
\bibliography{generic}
\end{document}
