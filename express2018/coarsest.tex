
\subsection{Coarsest (pre)congruence contained in $\approx$ (or $\succeq_{\mathrm{bis}}$)}

The ``coarsest congruence contained in weak bisimilarity ($\approx$)''
theorem in CCS is somehow special, as its current known proofs either
rely on quite restricted conditions, or have an extremely complicated proof
(c.f. van Glabbek's paper) in
which ordinal theory is required.  Actually even the relationship
between its name and statement is not well explained in many CCS
textbooks. But van Glabbek's paper has given the so far clearest
explainion, here we briefly repeat his arguments:

As we know weak bisimilarity ($\approx$) is not (real) congruence, as
it doesn't satisfy subsitutivity on direct sums (but if the CCS syntax
is non-standard, i.e. has only prefixed sums, $\approx$ is indeed a
congruence). The purpose is to find a coarsest congruence contained in
weak bisimilarity. (``coarsest'' means, any other congruence finer than it must be contained in it)
There're two ways to build a congruence from weak bisimilarity, one
way is the standard definition for observational congruence (rooted
weak bisimilarity) $\approx^c$ in CCS textbooks, but even it's proven to be a
congruence we don't know if it's coarsest one.  The other way is to
build a (pre)congruence closure (Def ??) directly upon
the original weak bisimilarity relation, we call the resulting
relation ``Weak bisimilarity congruence'' ($[\approx]$):
\begin{alltt}
\HOLConst{WEAK_CONGR} \HOLSymConst{=} \HOLConst{CC} \HOLConst{WEAK_EQUIV}
\HOLConst{CC} \HOLFreeVar{R} \HOLSymConst{=} (\HOLTokenLambda{}\HOLBoundVar{g} \HOLBoundVar{h}. \HOLSymConst{\HOLTokenForall{}}\HOLBoundVar{c}. \HOLConst{CONTEXT} \HOLBoundVar{c} \HOLSymConst{\HOLTokenImp{}} \HOLFreeVar{R} (\HOLBoundVar{c} \HOLBoundVar{g}) (\HOLBoundVar{c} \HOLBoundVar{h}))
\end{alltt}
It can be shown that any such (pre)congruence closure is automatically coarset.

Now it remains to prove that, the congrunce relation built by above
two quite different approaches actually coincide. To achive this goal,
we first noticed that, all other operators beside sums used in
semantic context doesn't matter, because they're already substituible
for weak bisimilarity. The only important operator is the sum
operator. To focus on this important operator, we can temporily
introduce another concept called \emph{sum equivalence}:
\begin{alltt}
\HOLConst{SUM_EQUIV} \HOLSymConst{=} (\HOLTokenLambda{}\HOLBoundVar{p} \HOLBoundVar{q}. \HOLSymConst{\HOLTokenForall{}}\HOLBoundVar{r}. \HOLBoundVar{p} \HOLSymConst{+} \HOLBoundVar{r} \HOLSymConst{\HOLTokenWeakEQ} \HOLBoundVar{q} \HOLSymConst{+} \HOLBoundVar{r})
\end{alltt}
It can be shown that, weak bisimilarity congruence implies this sum
equivalence. Thus if we can further prove that the sum
equivalence imples observational congrence, all the three relations
must coincide together, as shown in the following diagram:
\begin{displaymath}
\xymatrix{
{\textrm{Weak bisimilarity } (\approx)} & {\textrm{Sum
    equivalence } (\approx^+)} \ar@/^3ex/[ldd]^{\subseteq ?}\\
{\textrm{Weak bisim. congruence } ([\approx])}
\ar[u]^{\subseteq} \ar[ru]^{\subseteq} \\
{\textrm{Rooted bisimilarity } (\approx^c)} \ar[u]^{\subseteq}
}
\end{displaymath}
In another words, observational congruence is the coarsest congruence
contained in weak bisimilary if and only if
\begin{equation}
\forall p\; q.\; p \approx^c\! q \Longleftrightarrow \forall r.\; p+r \approx
q+r
\end{equation}

The classical assumption for proving above theorem requires that $p$
and $q$ do not use up all available visible actions,
i.e. $\mathrm{fn}(p) \cup \mathrm{fn}(q) \neq \mathscr{L}$. But it's
not easy to formalize and use such an assumption without a detailed
treatment on free and bound names (visible actions) of CCS
processes.\footnote{There're totally four such concepts: 1) free names
are all visible actions appearing in a CCS term without surrounding
$\nu$ (restriction) operator on the same action; 2) bound names are
the set of all actions ever used by $\nu$ (restriction) operator; 3)
free variables (or equation variables) are those variables without a
definition given by recursion
operator; 4) bound variables (process constants) are variables with
definitions given by recursion operator. All CCS results using these
concepts are not touched so far, although these four concepts are
successfully defined using HOL's set and list theories.} However, by
analyzing the proof steps, we found that, what's really required is to
not use up all available labels in those weak transitions directly
lead from $p$ and $q$. In another words, even they have used all
available labels, as long as their first weak transitions didn't, the
whole proof can still be finished.\footnote{Further more, $p$ and $q$
  can be considered separately: the proof can be finished as long as
  \emph{each} of them didn't use up all labels on first weak
  transition, while the union of these labels are all labels.}
We have formalized this property of
CCS process and call it ``free action'' property:
\begin{alltt}
\HOLTokenTurnstile{} \HOLConst{free_action} \HOLFreeVar{p} \HOLSymConst{\HOLTokenEquiv{}} \HOLSymConst{\HOLTokenExists{}}\HOLBoundVar{a}. \HOLSymConst{\HOLTokenForall{}}\HOLBoundVar{p\sp{\prime}}. \HOLSymConst{\HOLTokenNeg{}}(\HOLFreeVar{p} \HOLTokenWeakTransBegin\HOLConst{label} \HOLBoundVar{a}\HOLTokenWeakTransEnd \HOLBoundVar{p\sp{\prime}})
\end{alltt}
With this property, the classical form of this theorem that we have
formally proved is:
\begin{alltt}
\HOLTokenTurnstile{} \HOLConst{free_action} \HOLFreeVar{p} \HOLSymConst{\HOLTokenConj{}} \HOLConst{free_action} \HOLFreeVar{q} \HOLSymConst{\HOLTokenImp{}}
   (\HOLFreeVar{p} \HOLSymConst{\HOLTokenObsCongr} \HOLFreeVar{q} \HOLSymConst{\HOLTokenEquiv{}} \HOLSymConst{\HOLTokenForall{}}\HOLBoundVar{r}. \HOLFreeVar{p} \HOLSymConst{+} \HOLBoundVar{r} \HOLSymConst{\HOLTokenWeakEQ} \HOLFreeVar{q} \HOLSymConst{+} \HOLBoundVar{r})
\end{alltt}

If we drop this classical assumption, then the proof becomes much
harder, as given a van Glabbek's paper. The most important
intermediate result we have proved here, is the following lemma:
\begin{alltt}
\HOLTokenTurnstile{} (\HOLSymConst{\HOLTokenExists{}}\HOLBoundVar{k}.
        \HOLConst{STABLE} \HOLBoundVar{k} \HOLSymConst{\HOLTokenConj{}} (\HOLSymConst{\HOLTokenForall{}}\HOLBoundVar{p\sp{\prime}} \HOLBoundVar{u}. \HOLFreeVar{p} \HOLTokenWeakTransBegin\HOLBoundVar{u}\HOLTokenWeakTransEnd \HOLBoundVar{p\sp{\prime}} \HOLSymConst{\HOLTokenImp{}} \HOLSymConst{\HOLTokenNeg{}}(\HOLBoundVar{p\sp{\prime}} \HOLSymConst{\HOLTokenWeakEQ} \HOLBoundVar{k})) \HOLSymConst{\HOLTokenConj{}}
        \HOLSymConst{\HOLTokenForall{}}\HOLBoundVar{q\sp{\prime}} \HOLBoundVar{u}. \HOLFreeVar{q} \HOLTokenWeakTransBegin\HOLBoundVar{u}\HOLTokenWeakTransEnd \HOLBoundVar{q\sp{\prime}} \HOLSymConst{\HOLTokenImp{}} \HOLSymConst{\HOLTokenNeg{}}(\HOLBoundVar{q\sp{\prime}} \HOLSymConst{\HOLTokenWeakEQ} \HOLBoundVar{k})) \HOLSymConst{\HOLTokenImp{}}
   (\HOLSymConst{\HOLTokenForall{}}\HOLBoundVar{r}. \HOLFreeVar{p} \HOLSymConst{+} \HOLBoundVar{r} \HOLSymConst{\HOLTokenWeakEQ} \HOLFreeVar{q} \HOLSymConst{+} \HOLBoundVar{r}) \HOLSymConst{\HOLTokenImp{}}
   \HOLFreeVar{p} \HOLSymConst{\HOLTokenObsCongr} \HOLFreeVar{q}
\end{alltt}
It roughly says, for any two processes $p$ and $q$, if we can find
another stable (no $\tau$-transitions) process $k$ which is not weak bisimilar to any transition of
$p$ and $q$, then the hard part of our main theorem is proved. In
practice, once two processes were given, it's not hard to find
such a process, but the arbitrariness of $p$ and $q$ made this result
extremely hard to prove. The method given by van Glabbek requires a
construction of arbitrarily non-bisimilar processes called ``Klop
processes'':
\begin{definition}{(Klop processes)}
For each ordinal $\lambda$, and an arbitrary chosen non-$\tau$ action $a$,
define a CCS process $k_\lambda$ as follows:
\begin{enumerate}
\item $k_0 = 0$,
\item $k_{\lambda+1} = k_\lambda + a.k_\lambda$ and
\item for $\lambda$ a limit ordinal, $k_\lambda = \sum_{\mu < \lambda}
  k_\mu$, meaning that $k_\lambda$ is constructed from all graphs
  $k_\mu$ for $\mu < \lambda$ by identifying their root.
\end{enumerate}
\end{definition}
It's not hard to prove that, all $k_i$ are non-bisimilar (not only
weakly but also strongly). The idea is, for any two processes, the union of sets of their all
transitions cannot be arbitrarily large: it has to be limited by an
ordinal number. But above contruction can be arbitrarily large, so
there always exists a Klop process which can be used to satisfy above
lemma (and finish the proof).

However, this goes beyond HOL's expressivity to define above process, mostly
because there's no way to express infinite ``sums'' in CCS
datatype.\footnote{There're actually several ways to modify the
  datatype to support infinite sums, but none of these infinity can be
arbitrary. The full version of ordinal theory cannot be expressed in HOL.}
What we can do is to eliminate the last branch with limiting
ordinals, and define Klop processes only on finite cases:
\begin{alltt}
\HOLConst{KLOP} \HOLFreeVar{a} \HOLNumLit{0} \HOLSymConst{=} \HOLConst{nil}
\HOLConst{KLOP} \HOLFreeVar{a} (\HOLConst{SUC} \HOLFreeVar{n}) \HOLSymConst{=} \HOLConst{KLOP} \HOLFreeVar{a} \HOLFreeVar{n} \HOLSymConst{+} \HOLConst{label} \HOLFreeVar{a}\HOLSymConst{..}\HOLConst{KLOP} \HOLFreeVar{a} \HOLFreeVar{n}\hfill[KLOP_def]
\end{alltt}
But this means we much assume the two processes $p$ and $q$ are
finite-state, i.e. their corresponding LTS graphs have only finite
states. Choosing an element from a countable infinite set of processes
to make sure it's not bisimiar with a given set of processes, this is
essentially a pure set-theoretic problem, as formalized and proved in
the following lemma:
\begin{alltt}
\HOLTokenTurnstile{} \HOLConst{equivalence} \HOLFreeVar{R} \HOLSymConst{\HOLTokenImp{}}
   \HOLConst{FINITE} \HOLFreeVar{A} \HOLSymConst{\HOLTokenConj{}} \HOLConst{INFINITE} \HOLFreeVar{B} \HOLSymConst{\HOLTokenConj{}}
   (\HOLSymConst{\HOLTokenForall{}}\HOLBoundVar{x} \HOLBoundVar{y}. \HOLBoundVar{x} \HOLSymConst{\HOLTokenIn{}} \HOLFreeVar{B} \HOLSymConst{\HOLTokenConj{}} \HOLBoundVar{y} \HOLSymConst{\HOLTokenIn{}} \HOLFreeVar{B} \HOLSymConst{\HOLTokenConj{}} \HOLBoundVar{x} \HOLSymConst{\HOLTokenNotEqual{}} \HOLBoundVar{y} \HOLSymConst{\HOLTokenImp{}} \HOLSymConst{\HOLTokenNeg{}}\HOLFreeVar{R} \HOLBoundVar{x} \HOLBoundVar{y}) \HOLSymConst{\HOLTokenImp{}}
   \HOLSymConst{\HOLTokenExists{}}\HOLBoundVar{k}. \HOLBoundVar{k} \HOLSymConst{\HOLTokenIn{}} \HOLFreeVar{B} \HOLSymConst{\HOLTokenConj{}} \HOLSymConst{\HOLTokenForall{}}\HOLBoundVar{n}. \HOLBoundVar{n} \HOLSymConst{\HOLTokenIn{}} \HOLFreeVar{A} \HOLSymConst{\HOLTokenImp{}} \HOLSymConst{\HOLTokenNeg{}}\HOLFreeVar{R} \HOLBoundVar{n} \HOLBoundVar{k}
\end{alltt}

With all these results, finally we can prove the ``coarsest congruence
contained in $\approx$'' theorem for finite-state CCS:
\begin{alltt}
\HOLTokenTurnstile{} \HOLConst{finite_state} \HOLFreeVar{p} \HOLSymConst{\HOLTokenConj{}} \HOLConst{finite_state} \HOLFreeVar{q} \HOLSymConst{\HOLTokenImp{}}
   (\HOLFreeVar{p} \HOLSymConst{\HOLTokenObsCongr} \HOLFreeVar{q} \HOLSymConst{\HOLTokenEquiv{}} \HOLSymConst{\HOLTokenForall{}}\HOLBoundVar{r}. \HOLFreeVar{p} \HOLSymConst{+} \HOLBoundVar{r} \HOLSymConst{\HOLTokenWeakEQ} \HOLFreeVar{q} \HOLSymConst{+} \HOLBoundVar{r})
\end{alltt}

For contraction and rooted contraction, the situation (and proof
steps) is exactly the same:
\begin{displaymath}
\xymatrix{
{\textrm{Contraction } (\succeq_{\mathrm{bis}})} & {\textrm{Sum
    contraction } (\succeq_{\mathrm{bis}}^+)} \ar@/^3ex/[ldd]^{\subseteq ?}\\
{\textrm{Contraction precongruence } ([\succeq_{\mathrm{bis}}])}
\ar[u]^{\subseteq} \ar[ru]^{\subseteq} \\
{\textrm{Rooted contraction } (\succeq_{\mathrm{bis}}^c)} \ar[u]^{\subseteq}
}
\end{displaymath}

And we got two versions of theorem saying rooted contraction is the
coarsest precongruence of (bisimilarity) contraction:
\begin{alltt}
\HOLTokenTurnstile{} \HOLConst{free_action} \HOLFreeVar{p} \HOLSymConst{\HOLTokenConj{}} \HOLConst{free_action} \HOLFreeVar{q} \HOLSymConst{\HOLTokenImp{}}
   (\HOLFreeVar{p} \HOLSymConst{\HOLTokenObsContracts} \HOLFreeVar{q} \HOLSymConst{\HOLTokenEquiv{}} \HOLSymConst{\HOLTokenForall{}}\HOLBoundVar{r}. \HOLFreeVar{p} \HOLSymConst{+} \HOLBoundVar{r} \HOLSymConst{\HOLTokenContracts{}} \HOLFreeVar{q} \HOLSymConst{+} \HOLBoundVar{r})
\HOLTokenTurnstile{} \HOLConst{finite_state} \HOLFreeVar{p} \HOLSymConst{\HOLTokenConj{}} \HOLConst{finite_state} \HOLFreeVar{q} \HOLSymConst{\HOLTokenImp{}}
   (\HOLFreeVar{p} \HOLSymConst{\HOLTokenObsContracts} \HOLFreeVar{q} \HOLSymConst{\HOLTokenEquiv{}} \HOLSymConst{\HOLTokenForall{}}\HOLBoundVar{r}. \HOLFreeVar{p} \HOLSymConst{+} \HOLBoundVar{r} \HOLSymConst{\HOLTokenContracts{}} \HOLFreeVar{q} \HOLSymConst{+} \HOLBoundVar{r})
\end{alltt}
