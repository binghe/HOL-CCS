%%%% -*- Mode: LaTeX -*-
%%
%% This is the draft of the 2nd part of EXPRESS/SOS 2018 paper, co-authored by
%% Prof. Davide Sangiorgi and Chun Tian.

\subsection{Coarsest (pre)congruence contained in $\approx$ (or $\succeq_{\mathrm{bis}}$)}



% The ``coarsest congruence contained in weak bisimilarity ($\approx$)''
% theorem in CCS is somehow special, as its current known proofs either
% rely on quite restricted conditions, or have an extremely complicated proof
% (c.f. van Glabbek's paper) in
% which ordinal theory is required.  Actually even the relationship
% between its name and statement is not well explained in many CCS
% textbooks. But van Glabbek's paper has given the so far clearest
% explainion, here we briefly repeat his arguments:

Bisimilarity ($\approx$) is not  congruence, since not preserved by 
the sum operator. For this reason rooted bisimilarity has been
introduced (Definition~\ref{d:...}). 
We discuss in this subsection two proofs of the result stating that
rooted bisimilarity is the coarsest congruence contained in
bisimilarity: Milner's proof \cite{Milner_CCS_book}, that requires
that no process can use all available names; and van Glabbek's proof
\cite{...}, that does not require additional assumptions. 

The `coarsest congruence contained in bisimilarity' is 
the context closure, below called 
``weak bisimilarity congruence'' and written $[\approx]$, formalised thus:
\begin{alltt}
\HOLConst{WEAK_CONGR} \HOLSymConst{=} \HOLConst{CC} \HOLConst{WEAK_EQUIV}
\HOLConst{CC} \HOLFreeVar{R} \HOLSymConst{=} (\HOLTokenLambda{}\HOLBoundVar{g} \HOLBoundVar{h}. \HOLSymConst{\HOLTokenForall{}}\HOLBoundVar{c}. \HOLConst{CONTEXT} \HOLBoundVar{c} \HOLSymConst{\HOLTokenImp{}} \HOLFreeVar{R} (\HOLBoundVar{c} \HOLBoundVar{g}) (\HOLBoundVar{c} \HOLBoundVar{h}))
\end{alltt}

Given the central role of the  
 sum operator, we also consider the closure of bisimilarity under such
 operator, 
below called \emph{sum equivalence} and written $\approx^+$:
\begin{alltt}
\HOLConst{SUM_EQUIV} \HOLSymConst{=} (\HOLTokenLambda{}\HOLBoundVar{p} \HOLBoundVar{q}. \HOLSymConst{\HOLTokenForall{}}\HOLBoundVar{r}. \HOLBoundVar{p} \HOLSymConst{+} \HOLBoundVar{r} \HOLSymConst{\HOLTokenWeakEQ} \HOLBoundVar{q} \HOLSymConst{+} \HOLBoundVar{r})
\end{alltt}


% it doesn't satisfy subsitutivity on direct sums (but if the CCS syntax
% is non-standard, i.e. has only prefixed sums, $\approx$ is indeed a
% congruence). The purpose is to find a coarsest congruence contained in
% weak bisimilarity. (``coarsest'' means, any other congruence finer than it must be contained in it)
% There're two ways to build a congruence from weak bisimilarity, one
% way is the standard definition for observational congruence (rooted
% weak bisimilarity) $\approx^c$ in CCS textbooks, but even it's proven to be a
% congruence we don't know if it's coarsest one.  The other way is to
% build a (pre)congruence closure (Def ??) directly upon
% the original weak bisimilarity relation, we call the resulting
% relation ``Weak bisimilarity congruence'' ($[\approx]$):
% \begin{alltt}
% \HOLConst{WEAK_CONGR} \HOLSymConst{=} \HOLConst{CC} \HOLConst{WEAK_EQUIV}
% \HOLConst{CC} \HOLFreeVar{R} \HOLSymConst{=} (\HOLTokenLambda{}\HOLBoundVar{g} \HOLBoundVar{h}. \HOLSymConst{\HOLTokenForall{}}\HOLBoundVar{c}. \HOLConst{CONTEXT} \HOLBoundVar{c} \HOLSymConst{\HOLTokenImp{}} \HOLFreeVar{R} (\HOLBoundVar{c} \HOLBoundVar{g}) (\HOLBoundVar{c} \HOLBoundVar{h}))
% \end{alltt}
% It can be shown that any such (pre)congruence closure is automatically coarset.
%
% Now it remains to prove that, the congrunce relation built by above
% two quite different approaches actually coincide. To achive this goal,
% we first noticed that, all other operators beside sums used in
% semantic context doesn't matter, because they're already substituible
% for weak bisimilarity. The only important operator is the sum
% operator. To focus on this important operator, we can temporily
% introduce another concept called \emph{sum equivalence}:
% \begin{alltt}
% \HOLConst{SUM_EQUIV} \HOLSymConst{=}
% (\HOLTokenLambda{}\HOLBoundVar{p}
% \HOLBoundVar{q}. \HOLSymConst{\HOLTokenForall{}}\HOLBoundVar{r}. \HOLBoundVar{p} \HOLSymConst{+} \HOLBoundVar{r} 
%\HOLSymConst{\HOLTokenWeakEQ} \HOLBoundVar{q} \HOLSymConst{+} \HOLBoundVar{r})
% \end{alltt}
Rooted bisimilarity, being a congruence contained in bisimilarity,  
is contained in   weak bisimilarity congruence that, in turn,
trivially is contained in sum
equivalence. Thus the crux is proving that sum equivalence implies
rooted bisimilarity:
\begin{equation}
\label{equa:pq}
\forall p\; q.\; ( \forall r.\; p+r \approx
q+r \Longrightarrow 
 p \approx^c\! q  
\end{equation}
% \begin{displaymath}
% \xymatrix{
% {\textrm{Weak bisimilarity } (\approx)} & {\textrm{Sum
%     equivalence } (\approx^+)} \ar@/^3ex/[ldd]^{\subseteq ?}\\
% {\textrm{Weak bisim. congruence } ([\approx])}
% \ar[u]^{\subseteq} \ar[ru]^{\subseteq} \\
% {\textrm{Rooted bisimilarity } (\approx^c)} \ar[u]^{\subseteq}
% }
% \end{displaymath}
The standard argument, following Milner \cite{Milner_ccs_book}, requires that $p$
and $q$ do not make use of all possible. 
Formalising such argument requires however 
a detailed
treatment on free and bound names  of CCS
processes (with the restriction operator being a binder).
However, the proof of (\ref{equa:pq}) can be carried out 
% But it's
% not easy to formalize and use such an assumption without a detailed
% treatment on free and bound names (visible actions) of CCS
% processes.\footnote{There're totally four such concepts: 1) free names
% are all visible actions appearing in a CCS term without surrounding
% $\nu$ (restriction) operator on the same action; 2) bound names are
% the set of all actions ever used by $\nu$ (restriction) operator; 3)
% free variables (or equation variables) are those variables without a
% definition given by recursion
% operator; 4) bound variables (process constants) are variables with
% definitions given by recursion operator. All CCS results using these
% concepts are not touched so far, although these four concepts are
% successfully defined using HOL's set and list theories.} However, by
just assuming that the 
initial weak transitions (those directly emanating from 
 $p$ and $q$) do not use all possible visible labels.
% analyzing the proof steps, we found that, what's really required is to
% not use up all available labels in those weak transitions directly
% lead from $p$ and $q$. In another words, even they have used all
% available labels, as long as their first weak transitions didn't, the
% whole proof can still be finished.\footnote{Further more, $p$ and $q$
%   can be considered separately: the proof can be finished as long as
%   \emph{each} of them didn't use up all labels on first weak
%   transition, while the union of these labels are all labels.}
We
have formalized this property and
 called it the `free action' property:
\begin{alltt}
\HOLTokenTurnstile{} \HOLConst{free_action} \HOLFreeVar{p} \HOLSymConst{\HOLTokenEquiv{}} \HOLSymConst{\HOLTokenExists{}}\HOLBoundVar{a}. \HOLSymConst{\HOLTokenForall{}}\HOLBoundVar{p\sp{\prime}}. \HOLSymConst{\HOLTokenNeg{}}(\HOLFreeVar{p} \HOLTokenWeakTransBegin\HOLConst{label} \HOLBoundVar{a}\HOLTokenWeakTransEnd \HOLBoundVar{p\sp{\prime}})
\end{alltt}
With this property, the formalisation of (\ref{equa''pq}) that we have
obtained is:
\begin{alltt}
\HOLTokenTurnstile{} \HOLConst{free_action} \HOLFreeVar{p} \HOLSymConst{\HOLTokenConj{}} \HOLConst{free_action} \HOLFreeVar{q} \HOLSymConst{\HOLTokenImp{}}
   (\HOLFreeVar{p} \HOLSymConst{\HOLTokenObsCongr} \HOLFreeVar{q} \HOLSymConst{\HOLTokenEquiv{}} \HOLSymConst{\HOLTokenForall{}}\HOLBoundVar{r}. \HOLFreeVar{p} \HOLSymConst{+} \HOLBoundVar{r} \HOLSymConst{\HOLTokenWeakEQ} \HOLFreeVar{q} \HOLSymConst{+} \HOLBoundVar{r})
\end{alltt}
\finish{above: make an implication, rather than an ``iff'', as
  (\ref{equa:pq}) says} 

Now we move to van Glabbek's proof, without the 
`free action' property. For this, 
 The most important
intermediate result is  the following (we recall that a process is 
\emph{stable} if it cannot perform $\tau$-transitions):
\begin{alltt}
\HOLTokenTurnstile{} (\HOLSymConst{\HOLTokenExists{}}\HOLBoundVar{k}.
        \HOLConst{STABLE} \HOLBoundVar{k} \HOLSymConst{\HOLTokenConj{}} (\HOLSymConst{\HOLTokenForall{}}\HOLBoundVar{p\sp{\prime}} \HOLBoundVar{u}. \HOLFreeVar{p} \HOLTokenWeakTransBegin\HOLBoundVar{u}\HOLTokenWeakTransEnd \HOLBoundVar{p\sp{\prime}} \HOLSymConst{\HOLTokenImp{}} \HOLSymConst{\HOLTokenNeg{}}(\HOLBoundVar{p\sp{\prime}} \HOLSymConst{\HOLTokenWeakEQ} \HOLBoundVar{k})) \HOLSymConst{\HOLTokenConj{}}
        \HOLSymConst{\HOLTokenForall{}}\HOLBoundVar{q\sp{\prime}} \HOLBoundVar{u}. \HOLFreeVar{q} \HOLTokenWeakTransBegin\HOLBoundVar{u}\HOLTokenWeakTransEnd \HOLBoundVar{q\sp{\prime}} \HOLSymConst{\HOLTokenImp{}} \HOLSymConst{\HOLTokenNeg{}}(\HOLBoundVar{q\sp{\prime}} \HOLSymConst{\HOLTokenWeakEQ} \HOLBoundVar{k})) \HOLSymConst{\HOLTokenImp{}}
   (\HOLSymConst{\HOLTokenForall{}}\HOLBoundVar{r}. \HOLFreeVar{p} \HOLSymConst{+} \HOLBoundVar{r} \HOLSymConst{\HOLTokenWeakEQ} \HOLFreeVar{q} \HOLSymConst{+} \HOLBoundVar{r}) \HOLSymConst{\HOLTokenImp{}}
   \HOLFreeVar{p} \HOLSymConst{\HOLTokenObsCongr} \HOLFreeVar{q}
\end{alltt}
The result  says that, for any two processes $p$ and $q$,
(\ref{equa:pq}) holds 
 if there is
a stable process $k$ that is not bisimilar to any derivative of
$p$ and $q$. 
 The method given by van Glabbek to obtain such a process $k$  requires a
construction of arbitrarily non-bisimilar processes called ``Klop
processes'':
\begin{definition}{(Klop processes)}
For each ordinal $\lambda$, and an arbitrary chosen non-$\tau$ action $a$,
define a CCS process $k_\lambda$ as follows:
\begin{enumerate}
\item $k_0 = 0$,
\item $k_{\lambda+1} = k_\lambda + a.k_\lambda$ and
\item for $\lambda$ a limit ordinal, $k_\lambda = \sum_{\mu < \lambda}
  k_\mu$, meaning that $k_\lambda$ is constructed from all graphs
  $k_\mu$ for $\mu < \lambda$ by identifying their root.
\end{enumerate}
\end{definition}
All Klop processes are, pairwise,   non-bisimilar (even 
 strongly so).
However, the theory of 
 ordinal numbers cannot be expressed in HOL, 
hence we have not been able to formalise the above construction.
%  The idea is, for any two processes, the union of sets of their all
% transitions cannot be arbitrarily large: it has to be limited by an
% ordinal number. But above contruction can be arbitrarily large, so
% there always exists a Klop process which can be used to satisfy above
% lemma (and finish the proof).
%
% However, this goes beyond HOL's expressivity to define above process, mostly
% because there's no way to express infinite ``sums'' in CCS
% datatype.\footnote{There're actually several ways to modify the
%   datatype to support infinite sums, but none of these infinity can be
% arbitrary. The full version of ordinal theory cannot be expressed in HOL.}
We have instead formalised a limited form of the construction, without
the case (3) of limit ordinals, and therefore limiting our analysis to
finite-state processes.
 Klop processes are then defined thus:
\begin{alltt}
\HOLConst{KLOP} \HOLFreeVar{a} \HOLNumLit{0} \HOLSymConst{=} \HOLConst{nil}
\HOLConst{KLOP} \HOLFreeVar{a} (\HOLConst{SUC} \HOLFreeVar{n}) \HOLSymConst{=} \HOLConst{KLOP} \HOLFreeVar{a} \HOLFreeVar{n} \HOLSymConst{+} \HOLConst{label} \HOLFreeVar{a}\HOLSymConst{..}\HOLConst{KLOP} \HOLFreeVar{a} \HOLFreeVar{n}\hfill[KLOP_def]
\end{alltt}
We also make use of the following result, saying that, given an
equivalence $R$, a finite set $A$ and a countably infinite set $B$
whose elements, pairwise, are not related by $R$, then 
there is  an element of $B$ that is not related to any of the elements
of $A$. 
% But this means we much assume the two processes $p$ and $q$ are
% finite-state, i.e. their corresponding LTS graphs have only finite
% states. Choosing an element from a countable infinite set of processes
% to make sure it's not bisimiar with a given set of processes, this is
% essentially a pure set-theoretic problem, as formalized and proved in
% the following lemma:
\begin{alltt}
\HOLTokenTurnstile{} \HOLConst{equivalence} \HOLFreeVar{R} \HOLSymConst{\HOLTokenImp{}}
   \HOLConst{FINITE} \HOLFreeVar{A} \HOLSymConst{\HOLTokenConj{}} \HOLConst{INFINITE} \HOLFreeVar{B} \HOLSymConst{\HOLTokenConj{}}
   (\HOLSymConst{\HOLTokenForall{}}\HOLBoundVar{x} \HOLBoundVar{y}. \HOLBoundVar{x} \HOLSymConst{\HOLTokenIn{}} \HOLFreeVar{B} \HOLSymConst{\HOLTokenConj{}} \HOLBoundVar{y} \HOLSymConst{\HOLTokenIn{}} \HOLFreeVar{B} \HOLSymConst{\HOLTokenConj{}} \HOLBoundVar{x} \HOLSymConst{\HOLTokenNotEqual{}} \HOLBoundVar{y} \HOLSymConst{\HOLTokenImp{}} \HOLSymConst{\HOLTokenNeg{}}\HOLFreeVar{R} \HOLBoundVar{x} \HOLBoundVar{y}) \HOLSymConst{\HOLTokenImp{}}
   \HOLSymConst{\HOLTokenExists{}}\HOLBoundVar{k}. \HOLBoundVar{k} \HOLSymConst{\HOLTokenIn{}} \HOLFreeVar{B} \HOLSymConst{\HOLTokenConj{}} \HOLSymConst{\HOLTokenForall{}}\HOLBoundVar{n}. \HOLBoundVar{n} \HOLSymConst{\HOLTokenIn{}} \HOLFreeVar{A} \HOLSymConst{\HOLTokenImp{}} \HOLSymConst{\HOLTokenNeg{}}\HOLFreeVar{R} \HOLBoundVar{n} \HOLBoundVar{k}
\end{alltt}
We then derive van Glabbek's result
% ``coarsest congruence contained in $\approx$'' theorem 
for finite-state CCS:
\begin{alltt}
\HOLTokenTurnstile{} \HOLConst{finite_state} \HOLFreeVar{p} \HOLSymConst{\HOLTokenConj{}} \HOLConst{finite_state} \HOLFreeVar{q} \HOLSymConst{\HOLTokenImp{}}
   (\HOLFreeVar{p} \HOLSymConst{\HOLTokenObsCongr} \HOLFreeVar{q} \HOLSymConst{\HOLTokenEquiv{}} \HOLSymConst{\HOLTokenForall{}}\HOLBoundVar{r}. \HOLFreeVar{p} \HOLSymConst{+} \HOLBoundVar{r} \HOLSymConst{\HOLTokenWeakEQ} \HOLFreeVar{q} \HOLSymConst{+} \HOLBoundVar{r})
\end{alltt}

For contraction and rooted contraction, the situation and proof
steps are the same.
% \begin{displaymath}
% \xymatrix{
% {\textrm{Contraction } (\succeq_{\mathrm{bis}})} & {\textrm{Sum
%     contraction } (\succeq_{\mathrm{bis}}^+)} \ar@/^3ex/[ldd]^{\subseteq ?}\\
% {\textrm{Contraction precongruence } ([\succeq_{\mathrm{bis}}])}
% \ar[u]^{\subseteq} \ar[ru]^{\subseteq} \\
% {\textrm{Rooted contraction } (\succeq_{\mathrm{bis}}^c)} \ar[u]^{\subseteq}
% }
% \end{displaymath}
Thus  we have two versions of the theorem stating that  rooted contraction is the
coarsest precongruence of the (bisimilarity) contraction, the second
result on finite-state processes:
\begin{alltt}
\HOLTokenTurnstile{} \HOLConst{free_action} \HOLFreeVar{p} \HOLSymConst{\HOLTokenConj{}} \HOLConst{free_action} \HOLFreeVar{q} \HOLSymConst{\HOLTokenImp{}}
   (\HOLFreeVar{p} \HOLSymConst{\HOLTokenObsContracts} \HOLFreeVar{q} \HOLSymConst{\HOLTokenEquiv{}} \HOLSymConst{\HOLTokenForall{}}\HOLBoundVar{r}. \HOLFreeVar{p} \HOLSymConst{+} \HOLBoundVar{r} \HOLSymConst{\HOLTokenContracts{}} \HOLFreeVar{q} \HOLSymConst{+} \HOLBoundVar{r})
\HOLTokenTurnstile{} \HOLConst{finite_state} \HOLFreeVar{p} \HOLSymConst{\HOLTokenConj{}} \HOLConst{finite_state} \HOLFreeVar{q} \HOLSymConst{\HOLTokenImp{}}
   (\HOLFreeVar{p} \HOLSymConst{\HOLTokenObsContracts} \HOLFreeVar{q} \HOLSymConst{\HOLTokenEquiv{}} \HOLSymConst{\HOLTokenForall{}}\HOLBoundVar{r}. \HOLFreeVar{p} \HOLSymConst{+} \HOLBoundVar{r} \HOLSymConst{\HOLTokenContracts{}} \HOLFreeVar{q} \HOLSymConst{+} \HOLBoundVar{r})
\end{alltt}


\finish{note: in a result in concurrency, a constraint on finite-state
  processes is considered a very big constraint. Hence i am afraid
  people might not impressed by the version of van Glabbek's proof
  with such constraint. Is the proof hard? there is nothing above that
makes one thing that the proof is difficult. In other words: why is
the formalisation above  interesting?} 

