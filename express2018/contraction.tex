\section{Contractions}
\label{s:mcontr}

The constraints on the unique-solution Theorem~\ref{t:Mil89} can be 
weakened if  we move from equations to inequations called
  \emph{contractions}.


  Intuitively,
  the bisimilarity contraction 
$\mcontrBIS$ 
 is  a preorder in which 
$P \mcontrBIS Q  $  holds  if $P \wb Q$ and, in addition, 
$Q$ has the \emph{possibility} of being at least as efficient as $P$ (as far as
$\tau$-actions performed). 
Process $Q$, however, may be nondeterministic and may have other ways
of doing the same work, and these could be  slow (i.e., involving
more $\tau$-steps than those performed by $P$).
Thus, in contrast with expansion,  we cannot really say that `$Q$ is more efficient than
$P$'.


\begin{definition}[bisimulation  contraction]
\label{d:BisCon}
A process relation ${\R}$ 
 is a {\em  bisimulation  contraction}  if, whenever
 $P\RR Q$, %for all $\mu$ 


\begin{enumerate}

\item   $P \arr\mu P'$ implies there is $Q'$ such that $Q \arcap \mu
  Q'$
 and $P' \RR Q'$;

\item 
    $Q \arr\mu Q'$   implies there is $P'$ such that $P \Arcap \mu
 P'$ and $P' 
\wb Q'$.
\end{enumerate}
\emph{Bisimilarity  contraction}, written $\mcontrBIS$, is the union
of all bisimulation contractions. 
\end{definition}
% We say that $P$  {\emph{bisimilarly  contracts} $Q$, written
% $Q \mcontrBIS Q$.
% We write $\mexpaBIS$ for the inverse of $\mcontrBIS$.

In the first clause $Q$ is required to match $P$'s challenge
transition with at most one transition.
This makes sure that $Q$ is capable of mimicking $P$'s
work at least as efficiently as $P$. 
In contrast, the second clause of Definition~\ref{d:BisCon}, on the
challenges from $Q$, entirely ignores efficiency: it is the same
clause of  weak bisimulation~--- the final derivatives are even required
to be related  by $\wb$, rather than by $\R$.
 

Bisimilarity  contraction is coarser than 
 the expansion  relation 
$\expa$ \cite{sangiorgi2015equations, arun1992efficiency}
%of Definition~\ref{d:expa}.
Clause (1) is the same in the two
definitions. But in clause (2) expansion uses 
$P \Arr \mu P'$, rather than $P \Arcap \mu P'$; 
 moreover with
contraction the final derivatives are simply required to be bisimilar.
An expansion 
$P \expa Q$
tells us  that $Q$ is always at least as efficient as $P$, whereas  the
 contraction $P \mcontrBIS Q$  just says that $Q$ has the  possibility of
being at least as efficient as $P$. 

% \iflong\newDS
% Note the use of $\wb$, in place of $\R$, in the second clause:
%  we are only interested to know that $Q$ may  mimic $P$'s
% more in an efficient manner (first clause); we do not care about
% efficiency
% on the
% challenges proposed by $Q$,   which are handled using ordinary
% bisimilarity.
% \fi

\begin{example}
\label{exa:contr}
We have %\mcontrBIS a + \tau^n . a $
 $ a \not  \mcontrBIS \tau. a$. However,
$a+ \tau . a \mcontrBIS a$, as well as its converse, 
$  a \mcontrBIS a +
\tau . a $. Indeed, if $P \wb Q$ then 
$  P  \mcontrBIS P +Q$. The last two relations do not hold with 
$\expa$, which explains the strictness of the inclusion
 ${\expa} \subseteq {\mcontrBIS}$. 
% The inclusion is strict: for instance
% $a+ \tau . a \mcontrBIS a$, where $\mcontrBIS$ cannot be replaced by
%  $\contr$. Also the converse of  $a+ \tau . a \mcontrBIS a$ holds, namely
% $  a \mcontrBIS a +
% \tau . a $. However, we have %\mcontrBIS a + \tau^n . a $
%  $ a \not  \mcontrBIS \tau. a$
\end{example} 

% The substitutivity proofs for for expansion and bisimilarity carry
% over to contraction,   

As  bisimilarity  and expansion, contraction is preserved by all operators but sum.
