


\subsection{Equivalences: bisimilarity and rooted bisimilarity}
\label{ss:BiEx}

The equivalences here are 
 \emph{weak}, in that they
abstract from the number of internal steps performed
\begin{definition}[bisimilarity]
\label{d:wb}
A process relation ${\R}$ 
 is a \emph{bisimulation} if, whenever
 $P\RR Q$, %for all $\mu$ 
we have:
\begin{enumerate}
\item 
    $P \arr\mu P'$   implies that there is $Q'$ such that $Q \Arcap
\mu Q'$ and $P' 
\RR Q'$;

\item the converse of (1)
 on the actions from $Q$.
%\item if $Q  \Ar\mu   Q'$, then there is $P'$ s.t.\ $P
%\Arcap\mu  P'$ and $P' 
%\RR Q'$.
\end{enumerate}  
 $P$ and $Q$ are \emph{bisimilar},
written $P \wb
 Q$, if $P \RR Q$ for some  bisimulation $\R$.  
\qed\end{definition} 
\finish{the definition of rooted bisimilarity } 

We sometimes call bisimilarity \emph{weak} bisimilarity, to
distinguish it from \emph{strong bisimilarity}, $\sim$,
obtained by replacing in the above definition   the weak answer $
Q\Arcap\mu Q'$ with the strong  $Q \arr \mu Q'$.

The bisimulation proof method can be enhanced by means of \emph{up-to
  techniques}. One of the most useful auxiliary relations in up-to
techniques  is the \emph{expansion} relation  $\contr$ \cite{SaMi92}. This is an asymmetric version
of $\wb$ where $P \contr Q$  means that 
 $P \wb Q$,
but also that $Q$  achieves  the same as  $P$ 
with  no more work, i.e. with no more $\tau$ actions.
Intuitively, if $ P \contr Q$, we can think of $Q$ as being 
at least as fast as $P$
or, more generally, we can think that $P$  uses at least as many resources as $Q$. 



\subsection{Preorders: expansion and contraction}


 
\begin{definition}[expansion]
\label{d:expa}
A process relation ${\R}$ 
 is an {\em  expansion} if, whenever
we have $P\RR Q$, %for all $\mu$ 


\begin{enumerate}

\item   $P \arr\mu P'$ implies that there is $Q'$ with $Q \arcap \mu
  Q'$
 and $P' \RR Q'$;

\item 
    $Q \arr\mu Q'$   implies that there is $P'$ with $P \Arr \mu
 P'$ and $P' 
\RR Q'$.


\end{enumerate}
 $P$  {\em expands } $Q$, written
$P  \contr Q$, 
if $P \RR Q$,  for some expansion $\R$. 
\qed\end{definition}

\finish{here the definition of contraction}  

As bisimilarity, so expansion is preserved by all operators but sum. 

