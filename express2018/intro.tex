


\section{Introduction}

A prominent proof method for bisimulation, put forward by Milner and widely used in his
 landmark  CCS book \cite{Mil89}
is
\emph{unique solution of equations}, whereby two tuples of processes are
componentwise bisimilar if they are solutions 
of the same system of equations.
This method  is important in verification techniques and tools
  based on 
algebraic      reasoning \cite{theoryAndPractice,RosUnder10,BaeBOOK}. 

In the   \emph{weak} case (when  behavioural equivalences abstract from internal moves,
which practically is the most relevant case), however, 
Milner's proof method has severe syntactical limitations. 
To overcome such limitations, Sangiorgi proposes to replace
equations with  special inequations called
\emph{contractions} \cite{...}. Contraction is a preorder that, roughly, places some efficiency
constraints on processes.  
Uniqueness of the solutions 
 of a system of contractions
 is defined as with systems of
equations:  
any two solutions must be bisimilar.
The difference with equations is in the meaning of solution:
in the case of contractions
the solution is evaluated with respect to
the contraction preorder, rather than bisimilarity. 
With contractions, most syntactic limitations of the unique-solution theorem can be
removed.  One constraint that still remains in \cite{San...} is on occurrences of the sum operator, due to
the failure of substitutivity of contraction wrt such operator.  

The purpose of this paper is twofold. 
On the one hand, we refine the contraction proof technique by moving to the 
\emph{rooted contraction}, that is, the largest congruence included in the contraction
preoder.  Using rooted contraction one obtains a unique-solution theorem that is valid for
\emph{rooted bisimilarity} (hence also for bisimilarity itself).  
On the other hand, we provide a  
 comprehensive formalization  of the core of the theory of CCS 
 in the HOL
theorem prover (HOL4). The formalisation  includes the theory of unique solution of equations and
contractions discussed in the present paper, but is not limited to it (partly because such
theorems rely on a number of more fundamental results):
indeed the formalisation encompasses the basic properties of strong and weak
bisimilarity (e.g. the fixed-point and substitutivity properties), 
their algebraic theory, 
 various
versions of ``bisimulation up to'' techniques (e.g., bisimulation up-to expansion),
the basic properties  of rooted bisimilarity. Considering the relationship between bisimilarity and rooted
bisimilarity, the formalisation includes the proof that the latter is the coarsest
congruence included in the former, for which two proofs are formalised: one as in Milner's
book,  requiring the hypothesis that  no processes can use all labels; the other without
such hypothesis, essentially formalising van Glabbek's paper ...
Similar theorems are proved for rooted contractions wrt the contraction preorder. 



\finish{for Express, insist more on structured operational semantics,
  to fit the scope of the conference} 