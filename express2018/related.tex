\section{Related formalisation work}
\label{s:rel}

\hl{Monica Nesi made the first CCS formalisations for both pure and
value-passing CCS} \cite{Nesi:1992ve,Nesi:2017wo} using early versions of the HOL
theorem prover.\footnote{Part of her old proof scripts is now available at
  \url{https://github.com/binghe/HOL-CCS/tree/master/CCS-Nesi}.}.
Her main focus was on implementing decision procedures (as a ML
program, e.g.~\cite{CLEAVELAND:1993dr}) for
automatically proving bisimilarities of CCS
processes. %, thus she stopped proving deeper theorems in Milner's
           %book, once the congruence and algebraic laws of 
%strong, weak and rooted bisimilarity were proved.
\hl{Her work is
  the working basis of ours. However, the differences are substantial, especially in the way of defining
bisimilarities. We greatly benefited from features and standard
libraries in recent versions of HOL4, and our formalisation has
covered a much larger spectrum of the whole theory.}

Bengtson,  Parrow and Weber
have made a substantial formalisation work 
on CCS, $\pi$-calculi
and $\psi$-calculi 
using Isabelle/HOL and its \texttt{nominal} datatype package, with main focus on the handling of
name binders \cite{bengtson2010formalising,bengtson2007completeness,bengtson2007formalising}.
% Jesper Bengtson's PhD thesis 
%  \cite{bengtson2010formalising} looks  at 
% formalizations of CCS, $\pi$-calculi and $\psi$-calculi
% into Isabelle/HOL and its
% nominal logic. For CCS, 
% he has 
%  formalized
%  basic properties for strong/weak equivalence (congruence, basic
%  algebraic laws), for a  and observational
% congruence, however the CCS syntax doesn't have constants
% (or recursion operator), using replication in place of recursion.
% The formalisation effort has then been continued 
%
% On the other side, Jesper
% Bengtson and Joachim Parrow made great progress on $\pi$-calculus
% formalization and 
% proved that the algebraic axiomatization of bisimulation
% equivalence in the $\pi$-calculi is sound and
% complete. \cite{bengtson2007completeness} 
Other formalisations in this area include the work of Solange
Coupet-Grimal\footnote{\url{https://github.com/coq-contribs/ccs}} in Coq
and Chaudhuri et al.\;\cite{chaudhuri2015lightweight} in Abella. The
latter focuses on `bisimulation up-to' techniques for strong bisimilarity 
for CCS and $\pi$-calculus.
Damien Pous also formalised up-to techniques and some CCS examples in
Coq. \cite{pous2007new}
Formalisations less related to ours
include Kahsai and Miculan \cite{kahsai2008implementing} for the spi
calculus  in Isabelle, and Hirschkoff \cite{hirschkoff1997full} for the $\pi$-calculus in Coq.
