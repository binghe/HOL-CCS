\section{Related formalisation work}

Monica Nesi formalized both pure CCS \cite{Nesi:1992ve} and Value-Passing CCS
\cite{Nesi:2017wo} in old versions of HOL theorem prover, but the
related proof scripts were not publically available.\footnote{Part of
 the formalization of pure CCS can be found at \url{https://github.com/binghe/informatica-public/tree/master/CCS2/CCS-Nesi}.}
She was mainly focusing on writing decision procedures (ML functions) which
automatically prove theorems showning bisimilarity of CCS
processes.%
% \footnote{The work is ported (by the 2nd author) from
% HOL88 to latest HOL4, with small improvements on existing work, then
% added the theory of context and (pre)congruence, coarsest
% (pre)congruence contained in bisimilarity, bisimulation up to,
% expansion, contraction, and unique solution of
% equations/expansions/contractions et al, these additions have doubled
% the size of original work.}

Beside Nesi, there're three other CCS formalisations. One
is done in Isabelle/HOL (based on Nominal datatypes) in 2010, by Jesper Bengtson as part of his PhD
thesis project \cite{bengtson2010formalising}, available in Isabelle's Archive of Formal
Proofs (AFP)\footnote{\url{https://www.isa-afp.org/entries/CCS.html}}. He has formalized
classic properties for strong/weak equivalence and observational
congruence, however the CCS grammar doesn't include process constants
(or recursion operators). And the proved theorems also look quite
different with their forms in textbooks.
The other formalisation is based on Coq, done by Solange
Coupet-Grimal in 1995, which is part of the
``coq-contribs'' archive (Coq's official user contributed code)
\footnote{\url{https://github.com/coq-contribs/ccs}}. This is actually
a formalisation of transition systems without any algebraic operator,
and covers the very basic properties of strong/weak equivalence and observational
congruence. The last one was done by Chaudhuri et
al.\cite{chaudhuri2014formalization} using Abella, this work
particularily focus on bisimulation up to techniques while the CCS
type doesn't have restriction, recursion and relabelling operators.
