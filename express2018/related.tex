\section{Related formalisation work}

Monica Nesi formalized both pure CCS \cite{Nesi:1992ve} and Value-Passing CCS
\cite{Nesi:2017wo} in old versions of HOL theorem prover.\footnote{Part of
 the proof scripts for pure CCS now is available at \url{https://github.com/binghe/informatica-public/tree/master/CCS2/CCS-Nesi}.}.
Nesi's main focus was on
 decision procedures (as ML functions) for
automatically proving bisimilarities between CCS
processes. She stopped once all needed algebraic laws were proved. We
first ported her pure CCS formalisation to HOL4 with many essential
changes and then added new concepts and deep theorems. The total lines of
proof scripts gets doubled for now.

Bengtson,  Parrow and Weber
have made a substantial formalisation work 
on CCS, $\pi$-calculi
\cite{bengtson2010formalising,bengtson2007formalising,bengtson2007completeness}
and $\psi$-calculi 
using Isabelle/HOL and its nominal datatype package, with main focus on the handling of name binders.
% Jesper Bengtson's PhD thesis 
%  \cite{bengtson2010formalising} looks  at 
% formalizations of CCS, $\pi$-calculi and $\psi$-calculi
% into Isabelle/HOL and its
% nominal logic. For CCS, 
% he has 
%  formalized
%  basic properties for strong/weak equivalence (congruence, basic
%  algebraic laws), for a  and observational
% congruence, however the CCS syntax doesn't have constants
% (or recursion operator), using replication in place of recursion.
% The formalisation effort has then been continued 
%
% On the other side, Jesper
% Bengtson and Joachim Parrow made great progress on $\pi$-calculus
% formalization and 
% proved that the algebraic axiomatization of bisimulation
% equivalence in the $\pi$-calculi is sound and
% complete. \cite{bengtson2007completeness} 
Other formalisations in this area include the work of Solange
Coupet-Grimal\footnote{\url{https://github.com/coq-contribs/ccs}} in Coq
and Chaudhuri et al.\;\cite{chaudhuri2015lightweight} in Abella. The
latter focuses on `bisimulation up-to' techniques for strong bisimilarity 
for CCS and $\pi$-calculus.
Formalisations less related to ours
include Kahsai and Miculan \cite{kahsai2008implementing} for the spi
calculus  in Isabelle, and Hirschkoff \cite{hirschkoff1997full} for the $\pi$-calculus in Coq.

% The other formalisation is based on Coq, done by Solange
% Coupet-Grimal in 1995, which is part of the
% ``coq-contribs'' archive (Coq's official user contributed code)
% \footnote{\url{https://github.com/coq-contribs/ccs}}. This is actually
% a formalisation of transition systems without any algebraic operator,
% and covers the very basic properties of strong/weak equivalence and
% observational 
% congruence. 
