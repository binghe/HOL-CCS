\section{Related formalisation work}

Monica Nesi formalized both pure CCS \cite{Nesi:1992ve} and value-passing CCS
\cite{Nesi:2017wo} in old versions of HOL theorem prover.
She was mainly focusing on writing decision procedures (ML functions) which
automatically prove theorems showning bisimilarity of CCS processes,
thus HOL interactive shell could be used in
place of model checking tools like CWB
\cite{cleaveland1993concurrency}. The author ported her work from
HOL88 to latest HOL4, with small improvement, then added more concepts
and theorems, with the total lines of proof scripts at least doubled.

Beside Nesi, there're two other CCS formalisations. One
is done in Isabelle/HOL (based on Nominal datatypes) in 2010, by Jesper Bengtson as part of his PhD
thesis project \cite{bengtson2010formalising}, available in Isabelle's Archive of Formal
Proofs (AFP)\footnote{\url{https://www.isa-afp.org/entries/CCS.html}}. He has formalized
classic properties for strong/weak equivalence and observational
congruence, however the CCS grammar doesn't include process constants
(or recursion operators). And the proved theorems also look quite
different with their forms in textbooks.
The other formalisation is based on Coq, done by Solange
Coupet-Grimal in 1995, which is part of the
``coq-contribs'' archive (Coq's official user contributed code)
\footnote{\url{https://github.com/coq-contribs/ccs}}. This is actually
a formalisation of transition systems without any algebraic operator,
and covers the very basic properties of strong/weak equivalence and observational
congruence.
