\section{Related formalisation work}
\label{s:rel}
The closest to our work is Monica Nesi's.
She made the first formalisation of CCS, using
early versions of the HOL theorem prover (HOL88)\footnote{Part of
 the formalization of pure CCS can be found at \url{https://github.com/binghe/informatica-public/tree/master/CCS2/CCS-Nesi}.}.
Nesi's  main focus has been 
 decision procedures (as ML functions) for
automatically proving bisimilarities between CCS
processes.%
Her work has been a basis for ours. However the differences are
substantial, both because we have used the most recent version of HOL
(e.g., including support for coinduction) and because of the size of
the body of the CCS theory considered. 

\finish{above: is it fair?} 

% formalized both pure CCS \cite{Nesi:1992ve} and Value-Passing CCS
% \cite{Nesi:2017wo} in old versions of HOL theorem prover, but the
% related proof scripts were not publically available.
% \footnote{The work is ported (by the 2nd author) from
% HOL88 to latest HOL4, with small improvements on existing work, then
% added the theory of context and (pre)congruence, coarsest
% (pre)congruence contained in bisimilarity, bisimulation up to,
% expansion, contraction, and unique solution of
% equations/expansions/contractions et al, these additions have doubled
% the size of original work.}

%Beside Nesi, there exist other CCS formalisations. 
Bengtson,  Parrow, and  Weber
have made a substantial formalisation work 
on CCS, $\pi$-calculi and $\psi$-calculi 
into Isabelle/HOL and its
nominal logic
 \cite{bengtson2010formalising,bengtson2007completeness,DBLP:journals/jar/BengtsonPW16}. The main focus has been 
 $\psi$-calculi and the handling  of name binders.
The formalisations also  include
 basic properties for strong and weak bisimilarities  such as
 substitutivity and 
 algebraic laws.
% Jesper Bengtson's PhD thesis 
%  \cite{bengtson2010formalising} looks  at 
% formalizations of CCS, $\pi$-calculi and $\psi$-calculi
% into Isabelle/HOL and its
% nominal logic. For CCS, 
% he has 
%  formalized
%  basic properties for strong/weak equivalence (congruence, basic
%  algebraic laws), for a  and observational
% congruence, however the CCS syntax doesn't have constants
% (or recursion operator), using replication in place of recursion.
% The formalisation effort has then been continued 
%
% On the other side, Jesper
% Bengtson and Joachim Parrow made great progress on $\pi$-calculus
% formalization and 
% proved that the algebraic axiomatization of bisimulation
% equivalence in the $\pi$-calculi is sound and
% complete. \cite{bengtson2007completeness} 
Other formalisation of CCS or related calculi include: 
 Solange
Coupet-Grimal, in 
1995, in Coq (this  is part of the
``coq-contribs'' archive --- Coq's official user contributed
code\footnote{\url{https://github.com/coq-contribs/ccs}})  which is
actually
a formalisation of transition systems (without an algebraic structure),
and covers  basic properties of strong and weak bisimilarities;
and  Chaudhuri et
al.\   \cite{chaudhuri2014formalization}  in Abella, which
 focuses on `bisimulation up-to' techniques for strong bisimilarity 
for  CCS and $\pi$-calculus. Formalisations less related to ours
include
 Kahsai and Miculan  \cite{...} for the spi-calculus    in
Nominal Isabelle; 
  Hirschkoff  \cite{...} for the $\pi$-calculus    in Coq. 


% The other formalisation is based on Coq, done by Solange
% Coupet-Grimal in 1995, which is part of the
% ``coq-contribs'' archive (Coq's official user contributed code)
% \footnote{\url{https://github.com/coq-contribs/ccs}}. This is actually
% a formalisation of transition systems without any algebraic operator,
% and covers the very basic properties of strong/weak equivalence and
% observational 
% congruence. 



\finish{make sure we mention the papers below}

\begin{itemize}
\item
Bengtson, J., Parrow, J.: Formalising the pi-calculus using nominal logic. Logical Methods in Computer
Science
5
(2) (2008)


\item 
  Hirschkoff, D.: A full formalisation of pi-calculus theory in the calculus of constructions. In: E.L. Gunter,
A.P. Felty (eds.) Proceedings TPHOLs 97,
LNCS, vol. 1275,  Springer (1997


\item 
Kahsai, T., Miculan, M.: Implementing spi calculus using nominal techniques. In: 
 Proceedings CiE 2008,
LNCS, vol. 5028, Springer (2008)
\end{itemize}
 

\finish{in the references, the citation below is not complete: 
Kaustuv Chaudhuri, Matteo Cimini, Dale Miller:
A Lightweight Formalization of the Metatheory of Bisimulation-Up-To. CPP 2015: 157-166}