\subsection{Systems of contractions}
\label{ss:SysContr}

A \emph{system of contractions} is defined as a system of equations,
except that the contraction symbol $\mcontr$ is used in the place of
the equality symbol $=$. Thus a system of contractions is a set 
$\{  X_i \mcontr E_i\}_{i\in I}$
where $I$ is an  indexing set and expressions
$E_i$  may contain the  \behavC\  variables 
$\{  X_i\}_{i\in I}$.

\begin{definition}
\label{d:uniContra}
Given a  system of contractions 
$\{  X_i \mcontr E_i\}_{i\in I}$, 
 we say that:
\begin{itemize}
\item
 $\til P$ is a \emph{solution  for $\mcontrBIS$ of the 
 system of contractions} 
 if $\til P \mcontrBIS \til E [\til P]$;
\item 
the system  
has  \emph{a unique 
solution for $\approx$}
if 
whenever 
 $\til P$ and $\til Q$ are both solutions  for  $\mcontrBIS$
 then $\til P \approx \til Q$.
\end{itemize}
\end{definition}
  


\begin{definition}
\label{d:guarded}
A system of contractions $\{  X_i \mcontr E_i\}_{i\in I}$
 is
\emph{weakly guarded}
if,  in each    $E_i$, each occurrence of
a \behavC\ variable is underneath a prefix.

The system use \emph{weakly-guarded sums} if 
each $E_i$ only makes use of guarded sums.
\end{definition}



 
\begin{lemma}
\label{l:uptocon}
Suppose $\til P$ and $\til Q$ are solutions  for $\mcontrBIS$
 of a system of weakly-guarded contractions that uses 
weakly-guarded sums.
For any context $\qct$  that uses 
weakly-guarded sums,
if  $\ct{\til P}\Arr{\mu}  R$,
 then 
there is a context $\qctp$  that uses 
weakly-guarded sums
such that $R \mcontrBIS \ctp{\til P}$ and  $\ct{\til Q} \Arcap{\mu}
 \wb \ctp{\til Q}$.
\end{lemma}

\begin{proof}{[Sketch]}
Let $n$ be the length of the transition $\ct{\til P}\Arr\mu R$  (the
number of `strong steps' of which it is composed), and  
let $\ctpp {\til P}$ and $\ctpp {\til Q}$  be the processes obtained
from  $\ct {\til P}$ and $\ct {\til Q}$ by unfolding the definitions
of the contractions $n$ times. Thus in $\qctpp$ each hole is
underneath at least $n$ prefixes, and cannot contribute to an action
in the first $n$ transitions; moreover all the contexts use
weakly-guarded sums. 


We have $\ct{\til P} \mcontrBIS \ctpp{\til P}$, and 
$\ct{\til Q} \mcontrBIS \ctpp{\til Q}$, 
 by the substitutivity  properties of $\mcontrBIS$ (we exploit here
 the syntactic constraints on sums). Moreover,
 since each hole of the  context $\qctpp$ is underneath at least $n$
 prefixes, applying  
the definition
 of $ \mcontrBIS$ on the transition 
 $\ct{\til P}\Arr{\mu}  R$, we infer the existence
 of $\qctp$ such that 
$
\ctpp{\til P}\Arcap{\mu} \ctp{\til P} \mexpaBIS R
$
and 
$
\ctpp{\til Q}\Arcap{\mu}  \ctp{\til Q} 
. $
Finally, again applying the definition of $\mcontrBIS$ on 
$\ct{\til Q} \mcontrBIS \ctpp{\til Q}$, 
we derive 
$
\ct{\til Q}\Arcap{\mu}  \wb \ctp{\til Q} 
.$
\end{proof}

\begin{theorem}[unique solution of contractions for $\wb$]
\label{t:contraBisimulationU}
A system of weakly-guarded contractions
 that uses  weakly-guarded sums    has 
a unique solution 
 for $\wb$.
\end{theorem} 

\begin{proof}{[Sketch]}
One shows that if $\til P$ and $\til Q$ are solutions for $\wb$ of a system of weakly-guarded
contractions that uses weakly-guarded sums,  then the
 relation 
\[\R \DSdefi \{ 
(R,S) \st R \wb \ct{\til P}, S \wb \ct{\til Q} \mbox{ for some context
$\qct$} \} \, . 
\]
 is a bisimulation, exploiting 
 Lemma~\ref{l:uptocon}.
\end{proof} 


\section{Rooted contraction}
\label{ss:new}

The unique solution theorem of Section~\ref{ss:SysContr} requires a
constrained syntax for sums, due to the congruence and precongruence
problems of bisimilarity and contraction with such operator. 
We show here that the constraints can be
removed by moving to the induced congruence and precongruence, the
latter 
called \emph{rooted contraction}.
\begin{definition}[rooted contraction]
\label{d:rcontra}
Two processes $P$ and $Q$ are in the \emph{rooted contraction}, written 
 $P\rcontr Q$, if
\begin{enumerate}
\item $P \arr\mu P'$ implies that there is $Q'$ with $Q \arr \mu Q'$
 and $P'\mcontrBIS Q'$;
\item $Q \arr\mu Q'$   implies that there is $P'$ with $P \Arr \mu
 P'$ and $P' \wb Q'$.
\end{enumerate}
\end{definition}

The definition of rooted contraction adapts the definition of rooted
bisimilarity on top of that of the  contraction preorder
$\mcontrBIS$.


% The quick finding of rooted contraction was by help of
% theorem provers, and was based on two principles: 1) Its definition must NOT be recursive
% (or coinductive), instead it should mimic the definition of rooted bisimilarity
% ($\approx^c$); 2) It must be built on top of the existing \emph{contracts}
% relation $\mcontrBIS$, which we believe it's the \emph{right} one (as
% it's complete). Multiple candicates were tested, and finally the above
% definition was proven to be a precongruence:


\begin{theorem}
\label{t:rcontrPrecongruence}
Relation $\rcontr$ is a precongruence in CCS.  Indeed it is the
largest precongruence contained in $\contr$.
\end{theorem}  
The proof of this result is along the lines of the analogous result
for rooted bisimilarity with respect to bisimilarity. 
%Similarly one can define and handle the rooted variant of the expansion preorder. 

For a system of contractions, the meaning of 
``solution for $\rcontr$'' and of 
 \emph{a unique 
solution for $\rapprox$}
is the expected one---just replace in Definition~\ref{d:uniContra}  the preorder 
$\contr$ with $\rcontr$, and the equivalence 
$\approx$ with $\rapprox$.

For the rooted relations, the analogous of Lemma~\ref{l:uptocon} and of
Theorem~\ref{t:contraBisimulationU} can now be stated without constraints on the sum
operator.  
The schema of the proofs is also the same; the substitutivity
properties of 
$\rcontr$ and $\rapprox$ allow us to avoid issues with the sum
operator. Some care is needed in Theorem~\ref{t:rcontraBisimulationU}
to make sure that the final result is about  
$\rapprox$, rather than (the weaker) $\approx$.


\begin{lemma}
\label{l:ruptocon}
Suppose $\til P$ and $\til Q$ are solutions  for $\rcontr$ 
 of a system of weakly-guarded
contractions.
For any context $\qct$, 
if  $\ct{\til P}\Arr{\mu}  R$,
 then 
there is a  context $\qctp$
such that $R \mcontrBIS \ctp{\til P}$ and  $\ct{\til Q} \Arr{\mu}
 \wb \ctp{\til Q}$.
\end{lemma}

\begin{theorem}[Unique solution of contractions for $\rapprox$]
\label{t:rcontraBisimulationU}
A system of weakly-guarded contractions
    has 
a unique solution 
 for $\rapprox$.
\end{theorem} 

