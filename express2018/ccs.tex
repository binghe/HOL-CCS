% \section{Background}
% \label{s:back}

\section{CCS}
\label{ss:ccs}

\finish{maybe use recursion in place of constants, so to be
  more faithful wrt the HOL formalization}    

We assume  a possibly infinite set of \emph{input names} $a, b,
\ldots$ and \emph{output names} $\outC a, \outC b, \ldots$,
 and a set of \emph{constant identifiers} (or simply \emph{constants})
 $A, B, \ldots$ for writing recursive processes. 
The special symbol $\tau$ does not occur in the names and in the 
constants. 
The class  of the CCS processes is inductively built from $\nil$ by the operators
of parallel composition,  sum,
 restriction, prefix, and constants:
\begin{equation*}
\begin{array}{ccl}
P  & := &  \nil \; \midd \;  \mu . P \; \midd \;  P_1 |  P_2 \; \midd  \;
P_1 + P_2 \; \midd % \; \mu . P\; \midd  \; 
  (\res a\!)\, P  \;  \midd \;   K   %\\[\tkpS]
\end{array}
\end{equation*}
Each constant $K$ has  a definition 
 $K \Defi P$.
We sometimes omit trailing $\nil$, e.g., writing $a|b$ for $a.\nil |b .\nil $ .
The operational semantics is given by means of an LTS, and is
reported in Figure~\ref{f:LTSCCS} (the symmetric version  of the two rules for
parallel composition and the rule for sum  has been omitted).

 The \emph{immediate derivatives} of a
process $P$ are the elements of  the set $\{P' \st P \arr\mu P' \mbox{ for some $\mu$ }
\}$.   
We use $\ell$ to range over
 visible actions (i.e.,  inputs or outputs, excluding  $\tau$).
\begin{figure*}
\begin{center}
 $\displaystyle{   P \arr\mu   P' \over   P + Q   \arr\mu
P'  } $  $ \hb$   
%{{\sf par1}}:
 $\displaystyle{   P \arr\mu   P' \over   P | Q   \arr\mu
P' | Q } $  $ \hb$   
  $\; \;$  $\displaystyle{ P \arr{ a}P' \hk \hk  Q
\arr{\outC a }Q'  \over     P|  Q \arr{ \tau} P'
|  Q'  }$ 
\\
\vskip .1cm
\hskip .6cm  
 $\displaystyle{  \over  \mu.  P    \arr\mu
P } $  $ \hb$   
\hskip .6cm  
$\displaystyle{ P \arr{\mu}P' \over
 (\res a\!)\, P   \arr{\mu} (\res a\!)\, P'} $ $ \mu \neq a, \outC a$
$ \hb$ %  &
\hskip .5cm 
$\displaystyle{ P \arr{\mu}P' \over
 K   \arr{ \mu} P'  } $  if  $  K \Defi P$
\end{center}
%  \begin{center}
% \begin{tabular}{rlcrlcrl}
% %{{\sf sum}}:
% & $  \Sigma_{i\in I} \mu_i.   P_i 
% \arr{ \mu_i  }P_i$ & $ \hb$ &
% %{{\sf par1}}:
% & $\displaystyle{   P \arr\mu   P' \over   P | Q   \arr\mu
% P' | Q } $ & $ \hb$ &  
% % \multicolumn{5}{c}{
% %  {{\sf com1}}: 
% &  $\; \;$  $\displaystyle{ P \arr{ a}P' \hk \hk  Q
% \arr{\outC a }Q'  \over     P|  Q \arr{ \tau} P'
% |  Q'  }$ 
%  \\[\mysp]
% \multicolumn{8}{c}{ 
% %{{\sf res}}:  %&
% \hskip .4cm  
%  $\displaystyle{ P \arr{\mu}P' \over
%  \res a     P   \arr{\mu} \res a P'} $ $ \mu \neq a, \outC a$
% % & 
% $ \hb$ %  &
% \hskip .4cm 
% %{{\sf const}}: %& 
% \hskip .3cm 
% $\displaystyle{ P \arr{\mu}P' \over
%  K   \arr{ \mu} P'  } $  if  $  K \Defi P$
% }\end{tabular}
%  \end{center} 
\caption{The LTS of CCS}
\label{f:LTSCCS}
\end{figure*}

Some standard notations for transitions:  $\Longrightarrow $ is the 
reflexive and  transitive closure of $\arr\tau $, and 
$\Arr \mu $ is $\Longrightarrow \arr\mu \Longrightarrow $ (the
composition of the three relations).
Moreover,   
$ 
P \arcap \mu P'$ holds if $P \arr\mu P'$ or ($\mu =\tau$ and
$P=P'$); similarly 
$ 
P \Arcap \mu P'$ holds if $P \Arr\mu P'$ or ($\mu =\tau$ and
$P=P'$).
We write $P (\arr\mu)^n P'$ if $P$ can become $P'$ after performing
$n$ $\mu$-transitions. Finally, $P \arr\mu$ holds if there is $P'$
with $P \arr\mu P'$, and similarly for other forms of transitions.




\paragraph{Further notations}
Letters  $\R,\S$ range over relations.
We use infix notation for relations, e.g., 
$P \RR Q$ means that $(P,Q) \in \R$.
We use a tilde to denote a tuple, with countably many elements; thus
the tuple may also be infinite.
 All
notations  are  extended to tuples componentwise;
e.g., $\til P \RR \til Q$ means that $P_i \RR Q_i$, for  each  
component $i$  of the tuples $\til P$ and $\til Q$.
And $\ct{\til P}$ is the process obtained by replacing each hole
$\holei i$ of the  context $\qct$ with $P_i$.  
We write $
\ctx \R$ for the closure of a relation under contexts. Thus $P\: \ctx \R\: Q$
means that there are a context $\qct$ and tuples $\til P,\til Q$ with
$P =  \ct{\til P}, Q =  \ct{\til Q}$ and    $\til P \RR \til Q$.
We use  symbol 
$\DSdefi$ for abbreviations. For instance, $P \DSdefi G $, where
$G$ is some expression, means that  $P$ stands
for the  expression
$G$ (in contrast,  symbol $\Defi$ is used for the definition of
constants, whereas   $=$ is used for syntactic equality and for equations).
If $\leq$ is a preorder, then  $\geq$  is its inverse (and
conversely).    