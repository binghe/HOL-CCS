%%%% -*- Mode: LaTeX -*-
%%
%% This is the draft of the 2nd part of EXPRESS/SOS 2018 paper, co-authored by
%% Prof. Davide Sangiorgi and Chun Tian.

\documentclass{eptcs} % required template (EPTCS) by EXPRESS/SOS 2018

%% Math symbol packages
\usepackage{amsmath}
\usepackage{amssymb}
\usepackage{mathrsfs}

\usepackage{stmaryrd}
\SetSymbolFont{stmry}{bold}{U}{stmry}{m}{n}
% \cupdot (the best one)
\newcommand{\cupdot}{\mathbin{\mathaccent\cdot\cup}}

\usepackage{amsthm}

\newtheorem{definition}{Definition}[section]
\newtheorem{example}[definition]{Example}
\newtheorem{lemma}[definition]{Lemma}
\newtheorem{theorem}[definition]{Theorem}
\newtheorem{corollary}[definition]{Corollary}
\newtheorem{proposition}[definition]{Proposition} 
\newtheorem{remark}[definition]{Remark}

% \theoremstyle{definition}
% \newtheorem{definition}{Definition}[section]
% \theoremstyle{proposition}
% \newtheorem{proposition}{Proposition}[section]


% HOL theorem embedding support
\usepackage{holindex}
\usepackage{alltt}
%% TeX commands needed for generating terms and theorems of our CCS theories:

\newcommand{\HOLTokenStrongEQ}{$\sim{}$}
\newcommand{\HOLTokenWeakEQ}{$\approx{}$}
\newcommand{\HOLTokenObsCongr}{$\approx^{\mathrm{c}}\!$}
\newcommand{\HOLTokenEPS}{$\overset{\epsilon}{\Rightarrow}$}
\newcommand{\HOLTokenTransBegin}{$-$}
\newcommand{\HOLTokenTransEnd}{$\rightarrow$}
\newcommand{\HOLTokenWeakTransBegin}{$=$}
\newcommand{\HOLTokenWeakTransEnd}{$\Rightarrow$}
\newcommand{\HOLTokenExpands}{$\succeq_{\mathrm{e}}\!$}
\newcommand{\HOLTokenContracts}{$\succeq_{\mathrm{bis}}\!$}
\newcommand{\HOLTokenObsContracts}{$\succeq^{\mathrm{c}}_{\mathrm{bis}}\!$}
%\renewcommand{\HOLTokenImp}{\ensuremath{\Longrightarrow}}

% Experimental environments
\usepackage{environ}
\NewEnviron{HOLTrans}{\overset{\BODY}{\longrightarrow}}
\NewEnviron{HOLWeakTrans}{\overset{\BODY}{\Longrightarrow}}


\usepackage[all,cmtip]{xy}

% NAMES and VARIABLES

\def\Names{{\cal N}}            % set of all names
\def\fn#1{\rmsf{fn}(#1)}         % free names
\def\fv#1{\rmsf{fv}(#1)}         % free variables
\def\bv#1{\rmsf{bv}(#1)}         % bound variables
\def\bn#1{\rmsf{bn}(#1)}         % bound names

\newcommand{\dom}[1]{{\rmsf{dom}}(#1)} % the domain of something  

% FOR PROCESSES 

\def\nil{{\boldsymbol 0}} % nil
\def\res#1{{\boldsymbol \nu} #1\:}   % restriction
%% the following definitions allow us to use the symbols ! . and | 
%directly, for the replication, prefix and parallel compoisition
%operators in math mode 
\mathcode`\!="4021 % `!' as prefix operator
\mathcode`\.="602E  % prefix 
\mathcode`\|="326A % `|' as relation operator

\newcommand{\outC}[1]{\overline{#1}}      % CCS  output
\newcommand{\inpC}[1]{#1}                 % CCS  input

\newcommand{\out}[2]{\overline{#1}\langle{#2}\rangle} % output with value    
\newcommand{\inp}[2]{#1(#2)}  % input with value
\newcommand{\inpW}[2]{#1(#2). }  % input with value and dot

\newcommand{\iae}[2]{{#1}\langle{#2}\rangle} % input with value    


\newcommand{\cond}[3]{\myif\ #1\ \mythen\ #2\ \myelse\ #3} % if-then-else  
\newcommand{\myif}{\myspace{\rmtt{if}}\myspace}            % ``if''
\newcommand{\mythen}{\myspace{\rmtt{then}}\myspace}        % ``then''
\newcommand{\myelse}{\myspace{\rmtt{else}}\myspace}        % ``else''

\newcommand{\true}{\rmsf{true}} %boolean true
\newcommand{\false}{\rmsf{false}} %boolean false


% substitutions  (used as a postfixed operator)
\def\sub#1#2{\{\raisebox{.5ex}{\small$#1$}\! / \!\mbox{\small$#2$}\}} 



% TRANSITIONS (arrows)

\newcommand{\racap}{\mathrel{\stackrel{{\;\; {{\wedge}} \;\;}}{\mbox{\rightarrowfill}}}} 

\newcommand{\arr}[1]{\mathrel{\stackrel{{\;\;#1\;\;}}{\mbox{\rightarrowfill}}}}
                                %  strong labelled transition 

\newcommand{\Arr}[1]{\mathrel{\stackrel{{\;\;#1\;\;}}{\mbox{\rightarrowfillWEAK}}}} 
                                    %weak  labelled transitions
                                    
\newcommand{\arcap}[1]{\mathrel{\stackrel{{\;\; {\widehat{#1}} \;\;}}{\mbox{\rightarrowfill}}}} 
                                    %strong labelled transitions with hat
\newcommand{\Arcap}[1]{\mathrel{\stackrel{{\;\;{\widehat{#1}}\;\;}}{\mbox{\rightarrowfillWEAK}}}}
                                    %weak labelled transitions with   hat
                                    
% FOR CONTEXTS

\newcommand{\contexthole}{ [ \cdot  ] }      %hole of a context
\newcommand{\ct}[1]{ C \brac{#1} }   %filled context
\newcommand{\qct}{ C  }              %unfilled context  
\newcommand{\brac}[1]{[#1] }   % brackets for the context hole



%% SOME STYLE COMMANDS
\newcommand{\rmtt}[1]{{\rm\tt{#1}}} % for keywords like ``if'', ``then'' ... 
\newcommand{\rmsf}[1]{{{\rm\sf{#1}}}}


%BEHAVIOURAL  EQUIVALENCES and relations

\def\R{{\cal R}}          % R without spaces around
\def\RR{\mathrel{\cal R}} % R with some space around
\def\S{{\cal S}}          % S without spaces around
\def\SS{\mathrel{\cal S}} % S with some space around


\newcommand{\equival}{=} %\mathrel{\equiv_\alpha} 
                          % equality up to alpha conversion 





%SPECIAL SYMBOLS


\def\midd{\; \; \mbox{\Large{$\mid$}}\;\;}
               %separation symbol in  grammars               

\def\st{\; \mid \;} % ``such that'' in formulas
\def\DSdefi{\stackrel{\mbox{\scriptsize def}}{=}} % definition equal

\def\Defi{\stackrel{\mbox{\scriptsize $\triangle$}}{=}} % definition equal


% \def\qed{}
%  {\unskip\nobreak\hfil\penalty50\hskip1em\null\nobreak\hfil
%   $\Box$\parfillskip=0pt\finalhyphendemerits=0\endgraf}
%                    % end of proofs (or theorems,results without proofs)

\newcommand{\myspace}{\:}  % some spacing abbreviation

% rename tilde to widetilde, to be used for tuples
\renewcommand{\tilde}{\widetilde}


% % environment for proofs (for CUP style file)
% \newenvironment{proof}{\noindent {\bf Proof }}{\qed \bigskip}


\newcommand{\finish}[1]{ \vskip .2cm  {\bf #1} \vskip .2cm   \marginpar{{\bf $DS$}}}

\newcommand{\recu}[2]{{\tt rec}\: #1 . #2}

\newcommand{\rapprox}{\mathrel{\approx^{\rm{c}}}}


% NEW THINGS 

\newcommand{\mylabel}[1]{{\rm (#1).}}
\newcommand{\MYsketch}{[Sketch] }


\newcommand{\behav}{equation}
\newcommand{\behavC}{contraction}


\newcommand{\hb}{\hskip .5cm}

\newcommand{\ArrN}[2]{\mathrel{\stackrel{{\;\;#1\;\;}}{\mbox{\rightarrowfillWEAK}}_{#2}
  }} 
                                    %weak  labelled transitions weighted

\newcommand{\Var}{{\cal X}}

\newcommand{\AL}{{\cal RL}}
\newcommand{\PL}{{\cal L}}

\newcommand{\sign}{\Sigma}
\newcommand{\prsign}{\pr_\sign}


% possible actions in Abnients:
\newcommand{\qina}{{\ccc{in}} } 
\newcommand{\qout}{{\ccc{out}} }
\newcommand{\qopen}{{\ccc{open}} }
\newcommand{\capa}{{\ccc{capa}} }
\newcommand{\amb}[2]{#1 [\, #2 \,] } % ambients

\newcommand{\HOAMB}{\mbox{\rm{HO$\pi$Amb}}}
\newcommand{\SeqCCS}{\mbox{\rm{SeqCCS}}}


\newcommand{\SE}{{E\!S}}
\newcommand{\SL}{{L\!S}}
\newcommand{\SEp}{{E\!S'}}
\newcommand{\SEpU}{{E\!S'_1}}


%\newcommand{\mcontrP}[1]{\mathrel{\stackrel{{\footnotesize{\mbox{$\succ$}}}}{\footnotesize{\mbox{$#1$}}}}}
\newcommand{\mexpaP}[1]{\mathrel{\stackrel{{\footnotesize{\mbox{$\prec$}}}}{\footnotesize{\mbox{$#1$}}}}}

\newcommand{\wc}{\mathrel{\approx^{\rm c}}}
\newcommand{\wb}{\approx}
\newcommand{\contr}{\mathrel{\succeq_{\rm{bis}}}}
\newcommand{\expa}{\mathrel{\succeq_{\rm{e}}}}


\newcommand{\mcontr}{\mathrel{\succeq}}
\newcommand{\mexpa}{\mathrel{\preceq}}

\newcommand{\mcontrmay}{\mathrel{\succeq_{\rm{ctx}}}}
\newcommand{\mexpamay}{\mathrel{\preceq_{\rm{ctx}}}}


\newcommand{\mcontrTE}{\mathrel{\succeq_{\rm{tr}}}}
\newcommand{\TE}{\approx_{\rm tr}}       %trace equivalence


\newcommand{\mcontrBIS}{\mathrel{\succeq_{\rm{bis}}}}
\newcommand{\mexpaBIS}{\mathrel{\preceq_{\rm{bis}}}}


\newcommand{\rcontr}{\mathrel{\succeq^{\rm{c}}_{\rm{bis}}}}
\newcommand{\rexpa}{\mathrel{\preceq^{\rm{c}}_{\rm{bis}}}}


\newcommand{\til}{\tilde}

\newcommand{\ctx}[1]{#1^{\rm{c}}}

\newcommand{\Dwaleq}[1]{\Dwa^{\leq #1}}
\newcommand{\Dwageq}[1]{\Dwa^{\geq #1}}


\newcommand{\holeDS}{ [ \cdot  ] }
\newcommand{\holei}[1]{[\cdot]_{#1}}
\newcommand{\ctp}[1]{ C' \brac{#1} }  %primed context       
\newcommand{\qctp}{ C'  }
\newcommand{\ctpp}[1]{ C'' \brac{#1} }  %primed context       
\newcommand{\ctppp}[1]{ C''' \brac{#1} }  %primed context       

\newcommand{\qctpp}{ C''  }
\newcommand{\qctppp}{ C'''  }


\newcommand{\may}{\approx_{\rm ctx}}       %may equivalence

\newcommand{\mayHA}{\approx^{\rm{HAmb}}_{\rm ctx}}       %may equivalence

\newcommand{\hk}{\hskip .2cm }
\newcommand{\mysp}{10pt}
\newcommand{\tkp}{10pt}
\newcommand{\tkpS}{6pt}
\newcommand{\tkpSS}{3pt}
\newcommand{\tkpP}{15pt}

\newcommand{\smay}{\mathrel{\sim_{\rm ctx}}}       %may equivalence
\newcommand{\smayHA}{\mathrel{\sim^{\rm{HAmb}}_{\rm ctx}}}       %may equivalence

\newcommand{\holE}{\contexthole}  % hole

\newcommand{\murule}{\fortherules\mu} % mu rule of $\lambda$-calculus
\newcommand{\nurule}{\fortherules\nu} % nu rule of $\lambda$-calculus
\newcommand{\nuvrule}{\fortherules\nuv} % nuv rule of $\lambda$-calculus
\newcommand{\xirule}{\fortherules\xi} % xi rule of $\lambda$-calculus
\newcommand{\betarule}{\fortherules\beta} %beta rule of $\lambda$-calculus
\newcommand{\betavrule}{\fortherules\betav} % beta_v rule of $\lambda$-calculus
\newcommand{\etarule}{\fortherules\eta} % eta rule of $\lambda$-calculus
\newcommand{\nuv}{\nu_{\myrm v}} %beta rule of $\lambda$-calculus
\newcommand{\betav}{\mbox{$\beta_{\myrm{v}}$}} %beta rule of $\lambda$-calculus
\newcommand{\alpharule}{\fortherules\alpha} %alpha rule 
\newcommand{\fortherules}[1]{\mbox{$#1$}} %auxiliary def for the rules



\newcommand{\barbedbis}
{\mathrel{\stackrel{\bfcdotB}{\approx}}}

%OLD:
%{\mbox{ $\approx \! \! \! \!\! \!\!        
%\raisebox{1.15ex}[0ex][0ex]{\bfcdot} \; \,$}}



\newcommand{\wbb}{\mathrel{\approx_{\rm{bar}}}}
\newcommand{\wbc}{\mathrel{\approx^{\rm{c}}_{\rm{bar}}}}
% \newcommand{\wbb}{\mathrel{\barbedbis}}
% \newcommand{\wbc}{\cong}


\newcommand{\bfcdotB}{ {\mbox{\boldmath $.$}}  }         

\newcommand{\bcontra}
{\mathrel{\mcontr_{\rm{bar}}}}
%{\mathrel{\stackrel{\bfcdotB}{\mcontr}}}

\newcommand{\cbc}
{\mathrel{\mcontr^{\rm{c}}_{\rm{bar}}}}
%{\mathrel{{\mcontr_{\rm{bc}}}}}


\newcommand{\mypt}{2pt} 

% --------------



\newenvironment{myquote}
               {\list{}{\rightmargin\leftmargin}%
                \item\relax}
               {\endlist}


% \newenvironment{proofEx}{
% \begin{myquote}
% %\trivlist\parindent=0pt
% %      \item[\hskip \labelsep{\bf Answer: }]}
% \noindent %{\bf Answer to}
% }
% {\qed%\endtrivlist
% \end{myquote}}

%SPECIAL SYMBOLS

\newcommand{\EXX}[1]{{\bf Exercise~\ref{#1}}} % for answers to exercises 
\newcommand{\EXXpa}[2]{{\bf Exercise~\ref{#1}(#2)}} % for answers to exercises 


\newcommand{\Mybar}{\hrulefill} % separation rule

% \def\finish#1{\vskip.2cm\noindent{\em #1}%
%   \marginpar{$\longleftarrow$}\vskip.2cm}

\newcommand{\spaceD}{\,}


\newcommand{\bulletD}{\diamond}
%\mathrel{\lozenge} %\bowtie %blacktriangle %\minuso %\blacktriangleup %filleddiamond %\blackdiamond}

\newcommand{\ccc}{\rmtt} % {\rm\tt{#1}}}  %{\mbox{{\tt #1}}}
\newcommand{\cccTT}{\tt} % {\rm\tt{#1}}}  %{\mbox{{\tt #1}}}

\newcommand{\rr}{\RR}  
\newcommand{\Id}{{\cal I}} % identity relation  


\newcommand{\Prop}{{\cal P}} %
\newcommand{\FF}{F} % a function
\newcommand{\FFbis}{F_{\sim}} 
\newcommand{\FFbisW}{F_{\approx}} 
\newcommand{\finLISTS}{\mbox{{\tt FinLists}$_{A}$}} %   
\newcommand{\fininfLISTS}{\mbox{{\tt FinInfLists}$_{A}$}} %   
\newcommand{\finLISTSi}[1]{\mbox{\tt FinLists}_{#1}} %   
\newcommand{\fininfLISTSi}[1]{\mbox{\tt FinInfLists}_{#1}} %   
\newcommand{\nilLISTS}{\mbox{\tt nil}} %   
\newcommand{\mapLISTS}[2]{\mbox{\tt map}\: #1\: #2 } %   
\newcommand{\qmapLISTS}{\mbox{\tt map}} %   
\newcommand{\iterate}[2]{\mbox{\tt iterate}\: #1\: #2 } %   
\newcommand{\qiterate}{\mbox{\tt iterate}} %   
\newcommand{\qnats}{\mbox{\tt nats}} %   
\newcommand{\qfrom}{\mbox{\tt from}\: } %   
\newcommand{\fibs}{\mbox{\tt fibs}} %   
\newcommand{\qplus}{\mbox{\tt plus}\:} %   
\newcommand{\qtail}{\mbox{\tt tail}\,} %   
\newcommand{\plusU}{+_1} %   
\newcommand{\FFlist}{\Phi_{A\tt list}} 

\newcommand{\consLISTS}{\mbox{\tt cons}} %   
\newcommand{\consLISTSnew}[2]{\langle #1\rangle \bullet #2} %   
\newcommand{\consLISTSnewB}[2]{ #1 \bullet #2} %   


\newcommand{\simList}{\sim}  %_{A\tt list}}} % bisimilarity on lists 


\newcommand{\tree}[1]{\mbox{{\tt Tree}$(#1)$}} %   
%\newcommand{\root}[1]{\mbox{{\tt root}$(#1)$}} %   
\newcommand{\T}{{\cal T}}
\newcommand{\Vp}{{\tt V}}
\newcommand{\Rp}{{\tt R}}
\newcommand{\Gin}[2]{{\cal G}^{\tt ind}(#1,#2)} %   
\newcommand{\Gco}[2]{{\cal G}^{\tt coind}(#1,#2)} %   



\newcommand{\pws}[1]{\wp (#1)} % powerset
\newcommand{\lfp}[1]{\qqlfp(#1)} % least fixed point 
\newcommand{\gfp}[1]{\qqgfp(#1)} % greatest fixed point
\newcommand{\qqlfp}{{\tt lfp}} % least fixed point abbrv.
\newcommand{\qqgfp}{{\tt gfp}} % greatest fixed point abbrv.
\newcommand{\qlfp}{least fixed point}
\newcommand{\qgfp}{greatest fixed point} 


\newcommand{\Fcoin}{F_{\tt coind}} % coind. def. set
\newcommand{\Fin}{F_{\tt ind}}     % ind. def. set

\newcommand{\Lao}{\Lambda^0}  %closed $\lambda$-terms



  %convergence

\newcommand{\Dwa}{\Downarrow}           % plain convergence
\newcommand{\DwaP}[2]{#1 \Downarrow #2} % plain convergence, in rules


\newcommand{\EQsin}[2]{#1 = #2} % syn. equality in rules

\newcommand{\dwa}{\downarrow}           % plain convergence
\newcommand{\Up}{\Uparrow}   % divergence
\newcommand{\UpP}{\Uparrow}   % divergence in rules

\newcommand{\Reach}{\Downarrow}           % reachability
\newcommand{\Termi}{\downharpoonright}        % can terminate
\newcommand{\TermiP}{\Termi}        % can terminate, in rules
\newcommand{\Adiv}{\upharpoonright_\mu}           % \mu-divergence
\newcommand{\Adiva}{\upharpoonright_a}           % \mu-divergence

\newcommand{\LL}{{\cal L}}  % set of terms
\newcommand{\States}[1]{{\tt St}^{#1}} %states of an LTS


\newcommand{\myemptyItem}{\mbox{$ $ }} % utile per il xy package
\newcommand{\NONemptyItem}[1]{\mbox{$#1$ }} % utile per il xy package

\newcommand{\LongrightarrowN}[1]{\Longrightarrow_{#1}}


% for imperative programs
\newcommand{\XX}{\spaceD{\ccc{X}}}
\newcommand{\YY}{\spaceD{\ccc{Y}}}

% for LTSs
\newcommand{\Act}{\mbox{\it Act}} 
\newcommand{\pr}{\mbox{\it Pr}} % \mbox{\it Pr}}  %{{\mathbb P}} %{{\cal P}r} 
\newcommand{\power}{\wp} 
\renewcommand{\Pr}{\pr} % \mbox{\it Pr}}  %{{\mathbb P}} %{{\cal P}r} 


% membership stuff among relations
\newcommand{\memb}[3]{ #1 #3 #2 } 
\newcommand{\rmemb}[3]{ {(#1 , #3)} \in { #2} } 
%\newcommand{\Rmemb}[3]{ \tobr{#1 , #3} \in  #2 } 

% symbols
%\newcommand{\vv}{P} 
%\newcommand{\ww}{Q} 
\newcommand{\pp}{P} 
\newcommand{\qq}{Q} 

\newcommand{\rmm}[1]{\mbox{\rm #1}} %labels of transitions in figure 


%% for inference rules


\newcommand{\infrule}[3]{\[
{\trans{#1}\quadrule \displaystyle{#2 \over #3} } %\\[10pt]
\]}
\newcommand{\infruleSIDE}[4]{\[
{\trans{#1}\quadrule\displaystyle{#2 \over #3}\;\; #4 } %\\[10pt]
\]}  % inf rule with a side condition
\newcommand{\shortinfrule}[3]{ {\trans{#1}} \quadrule
     \displaystyle{#2 \over #3}}
\newcommand{\shortinfruleSIDE}[4]{ {\trans{#1}} \quadrule
     \displaystyle{#2 \over #3}\;\; #4}

\newcommand{\shortaxiom}[2]{{\trans{#1}}\quadrule
\displaystyle{ \over #2}}

\newcommand{\myinf}[3]{{\rn{#1}}\quadrule \displaystyle{#2 \over #3} }
    % for  plain  inference rules

\def\trans#1{\rn{#1}}   % for the names of transition rules
\newcommand{\rn}[1]{%
  \ifmmode 
    \mathchoice
      {\mbox{\sc #1}}
      {\mbox{\sc #1}}
      {\mbox{\small\sc #1}}
      {\mbox{\tiny\uppercase{#1}}}%
  \else
    {\sc #1}%
  \fi}

\newcommand{\quadrule}{\hskip .2cm }

\newcommand{\andalso}{\quad\quad}


% references

\def\reff#1{(\ref{#1})}       %references between brackets



\newcommand{\enco}[1]{[\! [ #1 ] \! ]  }


\newcommand{\mydots}{,\ldots,}

% for not\sim_n

\newcommand{\notsimN}[1]{\mathrel{\not\!{\sim_{#1}}} }



\newcommand{\cti}[2]{ C_{#1} \brac{#2} }   %filled context
\newcommand{\ctD}[1]{ D \brac{#1} }   %filled context
\newcommand{\qcti}[1]{ C_{#1}  }   % context

\newcommand{\ctDp}[1]{ D' \brac{#1} }   %filled context
\newcommand{\qctD}{D}   % context
\newcommand{\qctDp}{D'}   % context

%\newcommand{\qctp}{C'}   % context


\newcommand{\beginlongtable}{
 \begin{longtable}{l@{\extracolsep{\fill}}p{76mm}@{\extracolsep{\fill}}r}
%{|l@{\extracolsep{\fill}}p{80mm}@{\extracolsep{\fill}}r|}
%% READ THIS !!!
%% ho cancellato sotto altrimenti mi fa una entry nella list of tables
%\caption{ffff} \\
%\hline
%symbol          & description                & page \\
%\hline
}
\newcommand{\ENDlongtable}{% \hline
\end{longtable}}

\newcommand{\GLSbeg}[1]{\noindent %\underline
{\bf \large #1}}

\newcommand{\GLS}[1]{
\multicolumn{3}{l}{%\noindent 
%\underline
{\bf \large #1}}\\ }
\newcommand{\GLSb}[1]{
\multicolumn{3}{l}{%\noindent 
%\underline
{\it \large #1}}\\ }

\newcommand{\beginlongtableIN}{\\}
\newcommand{\ENDlongtableIN}{\\}

% % to add or remove a line to a page use these 
% \newcommand{\longpage}{\enlargethispage{\baselineskip}} 
% \newcommand{\shortpage}{\enlargethispage{-\baselineskip}} 





 \usepackage{DSarrow} % this is something to use extensible arrows in transitions

% \usepackage{pgf,tikz}
% \usetikzlibrary{arrows}


\begin{document}


\section{Introduction}

A prominent proof method for bisimulation, put forward by Milner and widely used in his
landmark CCS book \cite{Mil89} is the
\emph{unique solution of equations}, whereby two tuples of processes are
componentwise bisimilar if they are solutions 
of the same system of equations.
This method  is important in verification techniques and tools
based on algebraic reasoning \cite{theoryAndPractice,RosUnder10,BaeBOOK}. 

In the   \emph{weak} case (when  behavioural equivalences abstract from internal moves,
which practically is the most relevant case), however, 
Milner's proof method has severe syntactical limitations. 
To overcome such limitations, Sangiorgi proposes to replace
equations with  special inequations called
\emph{contractions} \cite{sangiorgi2015equations}. Contraction is a
preorder that, roughly, places some efficiency
constraints on processes.  Uniqueness of the solutions of a system of contractions
 is defined as with systems of equations:  
any two solutions must be bisimilar.
The difference with equations is in the meaning of solution:
in the case of contractions
the solution is evaluated with respect to
the contraction preorder, rather than bisimilarity. 
With contractions, most syntactic limitations of the unique-solution theorem can be
removed.  One constraint that still remains in
\cite{sangiorgi2015equations} is on occurrences of the sum operator,
due to the failure of substitutivity of contraction w.r.t. such operator.


The main goal  of the work described in this paper is 
a rather  
 comprehensive formalisation  of the core of the theory of CCS 
 in the HOL
theorem prover (HOL4),  with a focus on the theory of unique solutions of contractions.
The formalisation however is not confined to the theory of  unique solutions, but embraces 
most of the 
core of the theory of CCS \cite{Mil89}
(partly because the theory of unique solutions
 relies on a number of more fundamental results):
indeed the formalisation encompasses the basic properties of strong and weak
bisimilarity (e.g. the fixed-point and substitutivity properties), 
their algebraic theory, various versions of ``bisimulation up to''
techniques (e.g., bisimulation up-to bisimilarity),
the main properties  of the rooted bisimilarity (the congruence induced by weak
bisimilarity, also called observation congruence). Concerning rooed bisimilarity, the formalisation
includes Hennessy and Deng lemmas, and two proofs that rooted bisimilarity is the largest
congruence included in bisimilarity:
one as in Milner's
book,  requiring the hypothesis that  no processes can use all labels; the other without
such hypothesis, essentially formalising van Glabbek's paper \cite{vanGlabbeek:2005ur}
(such proof however follows the structure of the ordinal numbers, which cannot be handled
in HOL, and therefore it is restricted to finite-state processes ).
Similar theorems are proved for the rooted contraction preorder.

  
In this respect,  the work is an extensive experiment with the use of the HOL theorem prover and its
most recent developments, including a package  for expressing coinductive definitions.
The
work consists of about 20,000 lines of proof scripts in Standard ML.



From the CCS theory viewpoint, the formalisation has offered us the possibility of 
further refining the theory of unique solutions. 
In particular, the existing theory  placed limitations on the body of the contractions due to the
substitutivity problems of weak bisimilarity and other behavioural relations with respect
to the sum operator.  
We have thus refined the proof  technique based on contractions by moving to the 
\emph{rooted contraction}, that is, the coarsest (pre)congruence contained in the contraction
preorder.  Using rooted contraction one obtains a unique-solution theorem that is valid for
\emph{rooted bisimilarity} (hence also for bisimilarity itself), and without syntactic
constraints on the occurrences of sums.   

\finish{ I had to remove the reference \cite{Tian:2017wrba} here since it would further
  weaken this paper. } 

Another advantage of the formalisation is 
that we can take advantage of results about different 
equivalences or preorders that share a similar  proof structure. 
Examples are: the results that rooted bisimilarity and rooted contraction are,
respectively, the coarsest congruence contained in weak bisimilarity 
and the coarsest precongruence contained in the contraction  preorder; 
the result about unique solution of equations for weak bisimilarity that uses the
contraction preorder as an auxiliary relation, and other unique solution results (e.g., 
the one for rooted in which
the auxiliary relation is rooted contraction); various forms of enhancement of the bisimulation
proof method (the `up-to' techniques).  
In these cases, there are only a few places in which the HOL scripts have to be modified.
Then the succesful termination of the scripts  gives us a guarantee that the proof is
completed,  removing the risk 
of overlooking or missing details as in hand-written proofs. 


% to describe
% The purpose of this paper is twofold. 
% On the one hand, 
% On the other hand, we provide a  
%  comprehensive formalisation  of the core of the theory of CCS 
%  in the HOL
% theorem prover (HOL4). The formalisation  includes the proofs of
% Milner's 3 ``unique solution of equations'' theorems and
% contractions discussed in the present paper, but is not limited to it (partly because such
% theorems rely on a number of more fundamental results):
% indeed the formalisation encompasses the basic properties of strong and weak
% bisimilarity (e.g. the fixed-point and substitutivity properties), 
% their algebraic theory, various versions of ``bisimulation up to''
% techniques (e.g., bisimulation up-to expansion),
% the basic properties  of rooted bisimilarity. 
% Thus the work is an extensive experiment with the use of the HOL theorem provers and its
% most recent developments, including a facility  for expressing coinductive definitions.

% % Considering the relationship between bisimilarity and rooted
% % bisimilarity, the formalisation includes the proof that the latter is the coarsest
% % congruence included in the former, for which two proofs are formalized: one as in Milner's
% % book,  requiring the hypothesis that  no processes can use all labels; the other without
% % such hypothesis, essentially formalising van Glabbek's paper \cite{vanGlabbeek:2005ur}.
% % Similar theorems are proved for rooted contractions wrt the contraction preorder. 


\paragraph{Structure of the paper}....
% \section{Background}
% \label{s:back}

\section{CCS}
\label{ss:ccs}


We assume  a possibly infinite set of \emph{names} $\mathscr{L} = \{a, b,
\ldots\}$ forming input and output actions, plus a special invisible
action $\tau$ not in $\mathscr{L}$, and a set of variables $A,B,
\ldots$ for defining recursive behaviours.
The class  of the CCS processes is inductively defined from $\nil$ by the operators
of prefixing, parallel composition, sum (binary choice), restriction, recursion and \hl{relabeling}:
\begin{equation*}
\begin{array}{ccl}
\mu  & := &  \tau \hspace{.3pt} \; \midd \; a  \; \midd \;  \outC a  \\
P  & := &  \nil \; \midd \;  \mu . P \; \midd \;  P_1 |  P_2 \; \midd  \;
P_1 + P_2 \; \midd % \; \mu . P\; \midd  \; 
  (\res a\!)\, P  \;  \midd \;  A \; \midd \; \recu A  P
\; \midd \; P\; [r\!\!f]  % relabelling
\end{array}
\end{equation*}
%We sometimes omit trailing $\nil$, e.g., writing $a|b$ for $a.\nil |b .\nil $ .
The operational semantics of CCS is given by means of
a Labeled Transition System (LTS), shown in Fig.~\ref{f:LTSCCS} as SOS
rules (the symmetric version of the two rules for
parallel composition and the rule for sum are omitted).
A CCS expression uses only \emph{weakly-guarded sums} if all occurrences of
the sum operator are of the form $\mu_1.P_1 + \mu_2.P_2 + \ldots
+ \mu_n.P_n$, for some $n \geq 2$.
 The \emph{immediate derivatives} of a
process $P$ are the elements of the set $\{P' \st P \arr\mu P' \mbox{
  for some $\mu$}\}$.
% We use $\ell$ to range over
%  visible actions (i.e.~inputs or outputs, excluding  $\tau$).
\begin{figure*}
\begin{center}
\vskip .1cm
 $\displaystyle{  \over  \mu.  P    \arr\mu
P } $  $ \hb$   
\hskip .5cm
 $\displaystyle{   P \arr\mu   P' \over   P + Q   \arr\mu
P'  } $  $ \hb$   
\hskip .5cm
 $\displaystyle{   P \arr\mu   P' \over   P | Q   \arr\mu
P' | Q } $  $ \hb$   
\hskip .3cm
  $\; \;$  $\displaystyle{ P \arr{ a}P' \hk \hk  Q
\arr{\outC a }Q'  \over     P|  Q \arr{ \tau} P'
|  Q'  }$ 
\\
\vspace{.2cm}
$\displaystyle{ P \arr{\mu}P' \over
 (\res a\!)\, P   \arr{\mu} (\res a\!)\, P'} $ $ \mu \neq a, \outC a$
$ \hb$
%
$\displaystyle{ P \sub {\recu A P} A \arr{\mu}P' \over
 \recu A P   \arr{ \mu} P'  } $
\hskip .5cm  
$\displaystyle{ P \arr{\mu} P' \over
 P \;[r\!\!f] \arr{r\!\!f(\mu)} P' \;[r\!\!f]} $ $\forall a.\, r\!\!f(\outC a) = \overline{r\!\!f(a)}$
$ \hb$ %  &
\end{center}
\caption{\hl{Structural Operational Semantics} of CCS}
\label{f:LTSCCS}
\end{figure*}
Some standard notations for transitions:  $\Arr\epsilon$ is the 
reflexive and transitive closure of $\arr\tau$, and 
$\Arr \mu $ is $\Arr\epsilon \arr\mu \Arr\epsilon$ (the
composition of the three relations).
Moreover,   
$ 
P \arcap \mu P'$ holds if $P \arr\mu P'$ or ($\mu =\tau$ and
$P=P'$); similarly 
$ 
P \Arcap \mu P'$ holds if $P \Arr\mu P'$ or ($\mu =\tau$ and
$P=P'$).
We write $P \:(\arr\mu)^n P'$ if $P$ can become $P'$ after performing
$n$ $\mu$-transitions. Finally, $P \arr\mu$ holds if there is $P'$
with $P \arr\mu P'$, and similarly for other forms of transitions.




\paragraph{Further notations}
Letters  $\R$, $\S$ range over relations.
We use infix notation for relations, e.g., 
$P \RR Q$ means that $(P,Q) \in \R$.
We use a tilde to denote a tuple, with countably many elements; thus
the tuple may also be infinite.
 All
notations  are  extended to tuples componentwise;
e.g., $\til P \RR \til Q$ means that $P_i \RR Q_i$, for  each  
component $i$  of the tuples $\til P$ and $\til Q$.
And $\ct{\til P}$ is the process obtained by replacing each hole
$\holei i$ of the  context $\qct$ with $P_i$.  
We write $
\ctx \R$ for the closure of a relation under contexts. Thus $P\: \ctx \R\: Q$
means that there are context $\qct$ and tuples $\til P,\til Q$ with
$P =  \ct{\til P}, Q =  \ct{\til Q}$ and    $\til P \RR \til Q$.
We use  symbol 
$\DSdefi$ for abbreviations. For instance, $P \DSdefi G $, where
$G$ is some expression, means that  $P$ stands
for the  expression
$G$.
If $\leq$ is a preorder, then  $\geq$  is its inverse (and
conversely).


\subsection{Bisimilarity and rooted bisimilarity}
\label{ss:BiEx}

The equivalences we consider here are mainly \emph{weak} ones, in that they
abstract from the number of internal steps being performed:
\begin{definition}%[bisimilarity]
\label{d:wb}
A process relation ${\R}$ is a \textbf{bisimulation} if, whenever
 $P\RR Q$, \hl{for all $\mu\in \mathscr{L}\cup\{\tau\}$} we have:
\begin{enumerate}
\item $P \arr\mu P'$ implies that there is $Q'$ such that $Q \Arcap \mu Q'$ and $P' \RR Q'$;
\item $Q \arr\mu Q'$,implies that there is $P'$ such that $P \Arcap
  \mu P'$ and $P' \RR Q'$\enspace.
\end{enumerate}
 $P$ and $Q$ are \textbf{bisimilar},
written as $P \wb Q$, if $P \RR Q$ for some bisimulation $\R$.
\end{definition}

We sometimes call bisimilarity the \emph{weak} one, to
distinguish it from \emph{strong} bisimilarity ($\sim$),
obtained by replacing in the above definition   the weak answer $
Q\Arcap\mu Q'$ with the strong  $Q \arr \mu Q'$.
Weak bisimilarity is not preserved by the sum operator (except for
guarded sums). For this, Milner introduced \emph{observational congruence}, also called \emph{rooted
  bisimilarity} \cite{Gorrieri:2015jt,Sangiorgi:2011ut}:
\begin{definition}%[rooted bisimilarity]
\label{d:rootedBisimilarity}
Two processes $P$ and $Q$ are \textbf{rooted bisimilar}, written as $P
\rapprox Q$, iff \hl{for all $\mu\in \mathscr{L}\cup\{\tau\}$}
\begin{enumerate}
 \item  $P \arr\mu P'$ implies that there is $Q'$ such that $Q
   \Arr\mu Q'$ and $P' \wb Q'$;
 \item  $Q \arr\mu Q'$ implies that there is $P'$ such that $P
   \Arr\mu P'$ and $P' \wb Q'$\enspace.
\end{enumerate}
\end{definition}
% Besides reducing the rooted bisimiarity of two processes to
% the bisimilarities of their first-step transition ends, this definition also brings a proof technique for proving the
% rooted bisimiarity by constructing a bisimulation:
% \begin{lemma}{(Rooted bisimilarity by constructing a bisimulation)}
% \label{l:obsCongrByWeakBisim}
% Given a (weak) bisimulation $\RR$, if two processes $P$ and $Q$
% satisfies the following properties:
% \begin{enumerate}
%  \item  $P \arr\mu P'$ implies that there is $Q'$ such that $Q
%    \Arr\mu Q'$ and $P' \RR Q'$;
% \item the converse of (1) on the actions from $Q$.
% \end{enumerate}
% then $P$ and $Q$ are rooted bisimilar, i.e.~$P \approx^c Q$.
% \end{lemma}

\begin{theorem}
\label{t:rapproxCongruence}
$\rapprox$ is a congruence in CCS, and it is the
coarsest congruence contained in $\approx$ \cite{van2005characterisation}.
\end{theorem}

\subsection{Expansions}
\label{s:expa}

\hl{The bisimulation proof method can be enhanced by means of \emph{up-to
techniques}. One of the most useful auxiliary relations in up-to
techniques is the \emph{expansion} relation} $\expa$ \cite{arun1992efficiency,sangiorgi2015equations}.
This is an asymmetric version
of $\wb$ where $P \expa Q$ means that $P \wb Q$,
but also that $Q$ achieves the same as $P$
with no more work, i.e.~with no more $\tau$ actions.
Intuitively, if $P \expa Q$, we can think of $Q$ as being
at least as fast as $P$
or, more generally, we can think that $P$ uses at least as many resources as $Q$.
\begin{definition}%[expansion]
\label{d:expa}
A process relation ${\R}$
  is an \textbf{expansion} if, whenever
we have $P\RR Q$, for all $\mu$
 \begin{enumerate}
 \item   $P \arr\mu P'$ implies that there is $Q'$ with $Q \arcap \mu
   Q'$
  and $P' \RR Q'$;
 \item
     $Q \arr\mu Q'$   implies that there is $P'$ with $P \Arr \mu
  P'$ and $P'
 \RR Q'$.
 \end{enumerate}
  $P$  {\em expands} $Q$, written as
 $P  \expa Q$,
 if $P \RR Q$ for some expansion $\R$.
 \end{definition}

Same as bisimilarity, the expansion preorder is preserved by all operators but (direct) sums.

% next file: contraction.tex

\section{Equations and contractions}
\label{s:eq}

In the CCS syntax, 
a recursion $\recu A  P$ acts as a binder for $A$ in the body $P$. 
This gives rise, in the expected manner, to the notions of 
\emph{free} and \emph{bound} recursion variables in a CCS expression. 
For instance,  $X$ is free in $a.X + \recu Y (b.Y)$ while $Y$
is bound; \hl{And} in $a.X + \recu X (b.X)$, $X$ is both free and bound.
A \hl{term} without free variables is \hl{called} a \emph{process}.
% This setting does not cause any ambiguity, but sometimes leads to more
% complex proofs.} (see Section~\ref{sec:multivariate} for more details.)

In this paper (and the formalisation work), we use the agent
variables also as \emph{equation variables}. This eliminates the need of
another type for  equations, and we can reuse the existing
variable substitution operation (cf.~the SOS rule for the Recursion in
Fig.~\ref{f:LTSCCS}) for the substitution of equation variables.
For example, the result of substituting variable $X$ with $\nil$ in $a.X +
\recu X (b.X)$,  written $(a.X + \recu X (b.X)) \sub {\nil} X$, is
$a.\nil + \recu X (b.X)$ (with the part $\recu X (b.X)$
untouched). \Multivariate substitutions are written in the same syntax,
e.g. $E \sub {\til P} {\til X}$. Whenever $\til X$ is clear from the
context, we may also write $E[\til P]$ instead of $E \sub {\til P} {\til
  X}$ (and $E[P]$ for $E \sub P X$ if there is a single equation
variable $X$). 

% In fact, free agent variables have the same transitional behavior
% as the deadlock $\nil$ according to the SOS rules, as there is
% no rule for their transitions at all. In fact, most CCS theorems still hold if
% the involved CCS terms contain free variables. (The most notable
% exceptions are all versions of unique solution of equations/contracts,
% where all solutions must be \emph{pure} processes (i.e. no free variable).)
 
\subsection{Systems of equations}
\label{ss:SysEq}
When discussing equations it is standard to talk about `context'. This is a 
 a CCS expression  possibly containing  free variables that, however, may not occur within
the body of recursive definitions. 
Milner's ``unique solution of equations'' theorems~\cite{Mil89} intuitively
say that, if a context $C$
%  \hl{(i.e., a CCS expression with possibly
% free variables)}\footnote{\hl{Rigorously speaking, under our setting
% (i.e.~reusing free agent variables as equation variables) not all
% valid CCS expressions
% are valid contexts: those where free variables occur inside
% recusion operators must be all excluded. For instace, $\recu X (a.X +
% b.Y)$ is not a valid context with the variable $Y$. This special requirement
% is perfectly aligned with CCS literature using process constants, as any
% equation variable cannot occur inside the definition of any
% constant. In fact, all versions of ``unique solution of
% equations/contractions'' theorems do not hold if equation variables
% are allowed to occur inside any recursion operator.}} 
obeys certain conditions,
then all processes $P$ that satisfy the equation $P \wb \ct P$ are
bisimilar with each other.

\begin{definition}[equations] % Def 3.1
  \label{def:equation}
Assume that, for each $i$ of 
 a countable indexing set $I$, we have variables $X_i$, and expressions
$E_i$ possibly containing such variables $\cup_i \{ X_i\}$. Then 
$\{ X_i = E_i\}_{i\in I}$ is 
  a \emph{system of equations}. (There is one equation $E_i$ for each variable $X_i$.)
\end{definition}

The components of $\til P$ need not be
different from each other, as it must be for the variables $\til X$.
% If the system has infinitely many equations, the  tuples $\til P$
% and $\til X$ are infinite too. 

\begin{definition}[solutions and unique solutions]
  \label{def:solution}
Suppose $\{ X_i = E_i\}_{i\in I}$ is a system of equations: 
\begin{itemize}
\item
 $\til P$ is a \emph{solution of the system of equations (for $\wb$)} 
if for each $i$ it holds that $P_i \wb E_i [\til P]$;
\item The system has \emph{a unique solution for $\wb$}  if whenever 
 $\til P$ and $\til Q$ are both solutions then $\til P \wb \til Q$. 
\end{itemize} 
 \end{definition}
Similarily, the \emph{(unique) solution of a system of equations for $\sim$}
(or for $\rapprox$) can be obtained by replacing all occurrences of $\wb$
in above definition with $\sim$ and $\rapprox$, respectively.

For instance, the solution of the equation $X = a. X$ 
is the process
$R \DSdefi \recu A {\, (a. A)}$, and for any other solution $P$ we have $P \wb R$.
In contrast, the equation 
 $X = a|  X$ has solutions that may be quite different, for instance,
 $K$ and $K | b$, for $K \DSdefi \recu K {\, (a. K)}$. (Actually any process capable of
continuously performing $a$--actions is a solution of $X = a|  X$.)

%
% The unique solution of the system (1), modulo $\wb$,  is the constant $K \Defi a
% . K$:  for any other solution $P$ we have $P \wb K$.
% The unique solution of (2), modulo $\wb$, are the constants $K_1 , K_2$
% with $K_1 \Defi a . K_2$ and $K_2 \Defi b. K_1$; again, for any other
% pair of solutions $P_1,P_2$ we have $K_1 \wb P_1$ and $K_2 \wb P_2$.
%
Examples of systems that do not have unique solutions are: $X = X$, $X
= \tau . X$ and $X = a | X$.

\begin{definition}[guardedness of equations]
\label{def:guardness}
A system of equations $\{ X_i = E_i\}_{i\in I}$ is 
\begin{itemize}
\item \emph{weakly guarded} if, in each $E_i$, each occurrence of
  each $X_i$ is underneath a prefix;

\item \emph{guarded} if, in each $E_i$, each occurrence of
  each $X_i$ is underneath a \emph{visible} prefix;

\item \emph{sequential} if, in each $E_i$, each
  occurrence of each $X_i$ is only underneath prefixes and sums.
\end{itemize}
\end{definition}

In other words, if a system of equations is sequential, then for
each  $E_i$, any subexpression of $E_i$ in which $X_j $
appears, apart from $X_j$ itself, is a sum of prefixed expressions.
For instance,
\begin{itemize}
\item $X = \tau. X + \mu . \nil$ is sequential but not 
 guarded, because the guarding prefix for the variable
is not visible;
\item $X =  \ell . X | P$ is guarded but not sequential;
\item $X =  \ell . X + \tau. \res a (a .\outC b | a.\nil)$, as well as
$X = \tau . (a. X + \tau . b .X + \tau  )$ are both guarded and sequential.
\end{itemize}

Milner has  three versions of  ``unique solution of equations''
theorems, for $\sim$, $\wb$ and $\rapprox$, respectively, though only the
following two versions are explicitly mentioned in~\citep[p.~103, 158]{Mil89}:
\begin{theorem}[unique solution of equations for $\sim$]
\label{t:Mil89s1}
Let $E_i$ be weakly guarded with free variables in $\til X$,
and let ${\til P} \sim {\til E}\{\til P /\til X\}$,
  ${\til Q} \sim {\til E}\{\til Q /\til X\}$. Then ${\til P} \sim {\til Q}$.
\end{theorem}

\begin{theorem}[unique solution of equations for $\rapprox$]
\label{t:Mil89s3}
Let $E_i$ be guarded and sequential with free
variables in $\til X$, and let ${\til P} \rapprox {\til E}\{\til P /\til X\}$,
  ${\til Q} \rapprox {\til E}\{\til Q /\til X\}$. Then ${\til P} \rapprox {\til Q}$.
\end{theorem}

The version of Milner's unique-solution theorem for $\wb$ further requires
that all sums are guarded:
% \footnote{But if the CCS syntax were defined with only guarded
%   sums, i.e., $\sum_{i\in I} \mu_i.P_i$ as
%   in~\cite{sangiorgi2015equations}, this addition
%   requirement disappears automatically, and we can even say $\wb$ is indeed a
%   congruence.}:
\begin{theorem}[unique solution of equations for $\wb$]
\label{t:Mil89}
Let $E_i$ be guarded and sequential with free
variables in $\til X$, and let ${\til P} \wb {\til E}\{\til P /\til X\}$,
  ${\til Q} \wb {\til E}\{\til Q /\til X\}$. Then ${\til P} \wb {\til Q}$.
\end{theorem}

The proof of the theorem above exploits an invariance
property on immediate derivatives
of guarded and sequential expressions, and then extracts a bisimulation
(up to bisimilarity) out
of the solutions of the system.
To see the need of the sequentiality  condition, consider
 the equation (from \cite{Mil89}) $X = \res a (a. X | \outC a)$
where $X$ is guarded but not sequential. Any process that does not use
$a$ is a solution, e.g. $\nil$ and $b.\nil$.

For more details of above three theorems, see Section~\ref{ss:part2}
for the \univariate case and
Section~\ref{sec:multivariate} for the \multivariate case.

%% next file: contraction.tex
 % Systems of  equations
\subsection{Expansions and Contractions}
\label{s:mcontr}

Milner's ``unique solution of equations'' theorem for $\wb$
(Theorem~\ref{t:Mil89})
brings a new proof technique for proving (weak) bisimilarities. However, it has
 limitations: the equations must be guarded and sequential. (\hl{Moreover,}
all sums where equation variables appear must be guarded sums.)
This limits the usefulness of the technique, since
the occurrences of other operators \hl{using equation} variables, such as parallel
composition and restriction,
in general cannot be \hl{eliminated}. % without changing the meaning of equations
The constraints \hl{in} Theorem~\ref{t:Mil89}, however, can be
weakened if we move from equations to a special kind of inequations called
  \emph{contractions}.

Intuitively, the bisimilarity contraction $\mcontrBIS$ is a preorder
\hlD{in which} $P \mcontrBIS Q$ holds if $P \wb Q$ and, in addition, 
\emph{$Q$ has the possibility of being at least as efficient as $P$} (as far as
$\tau$-actions are performed).
The process $Q$, however, may be nondeterministic and may have other ways
\hl{to do} the same work, \hl{ways} which could be \hl{slower} (i.e., involving
more $\tau$-actions than those performed by $P$).
% Thus, in contrast with expansion,  we cannot really say that `$Q$ is more efficient than
% $P$'.

\begin{definition}[contraction]
\label{d:BisCon}
A process relation ${\R}$ 
 is a \emph{(bisimulation) contraction} if, whenever $P\RR Q$,

\begin{enumerate}
\item $P \arr\mu P'$ implies \hl{that} there is $Q'$ \hl{with} $Q \arcap \mu
  Q'$ and $P' \RR Q'$;
\item $Q \arr\mu Q'$ implies \hl{that} there is $P'$ \hl{with} $P \Arcap \mu
 P'$ and $P' \wb Q'$.
\end{enumerate}
Two processes $P$ and $Q$ are in the \emph{bisimilarity
contraction}, written as $P \mcontrBIS Q$,
if $P\ \R\ Q$ for some contraction $\R$.
Sometimes we write $\mexpaBIS$ for the inverse of $\mcontrBIS$.
\end{definition}
In clause (1) \hl{of the above definition}, $Q$ is required to match the challenge
transition \hl{of $P$} with at most one transition.
This makes sure that $Q$ is capable of mimicking % verb: mimic
$P$'s work at least as efficiently as $P$. 
In contrast, \hl{clause (2) entirely ignores efficiency on the challenges from $Q$:}
the final derivatives are required to be related by $\wb$, rather than by $\R$.

Bisimilarity contraction is coarser than bisimilarity expansion
$\expa$~\cite{arun1992efficiency,sangiorgi2015equations}, \hl{one of the
most useful auxiliary relations in up-to techniques}:
\begin{definition}[expansion]
\label{d:expa}
A process relation ${\R}$
  is an \emph{expansion} if, whenever $P\RR Q$,
 \begin{enumerate}
 \item   $P \arr\mu P'$ implies that there is $Q'$ with $Q \arcap \mu  Q'$
  and $P' \RR Q'$;
 \item $Q \arr\mu Q'$ implies that there is $P'$ with $P \Arr \mu P'$ and $P' \RR Q'$.
 \end{enumerate}
Two processes $P$ and $Q$ are in the \emph{bisimilarity
  expansion}, written as $P \expa Q$, if $P \RR Q$ for some expansion $\R$.
 \end{definition}
\hl{Bisimilarity expansion} is widely used in proof techniques for bisimilarity.
\hl{It} intuitively refines bisimilarity by 
formalising the idea of ``efficiency'' between processes.
Clause (1) is the same in the both preorders, while in clause (2) expansion \hl{requires}
$P \Arr \mu P'$, rather than $P \Arcap \mu P'$.
\hl{Moreover,} in clause (2) of Def.~\ref{d:BisCon} the final derivatives
are simply required to be bisimilar ($P' \wb Q'$).
Intuitively, $P \expa Q$ holds if $P\wb Q$ and, in addition, \emph{$Q$
  is always at least as efficient as $P$}.

\begin{example}
\label{exa:contr}
We have %\mcontrBIS a + \tau^n . a $
 $ a \not  \mcontrBIS \tau. a$. However,
$a+ \tau . a \mcontrBIS a$, as well as its converse, 
$  a \mcontrBIS a +
\tau . a $. Indeed, if $P \wb Q$ then 
$  P  \mcontrBIS P +Q$. The last two relations do not hold with 
$\expa$, which explains the strictness of the inclusion
 ${\expa} \subset {\mcontrBIS}$. 
% The inclusion is strict: for instance
% $a+ \tau . a \mcontrBIS a$, where $\mcontrBIS$ cannot be replaced by
%  $\contr$. Also the converse of  $a+ \tau . a \mcontrBIS a$ holds, namely
% $  a \mcontrBIS a +
% \tau . a $. However, we have %\mcontrBIS a + \tau^n . a $
%  $ a \not  \mcontrBIS \tau. a$
\end{example} 

\hlD{Bisimilarity expansion and bisimilarity contraction are both
preorders.}
Similarily with (weak) bisimilarity, both the expansion and the
contraction preorders are preserved by all CCS operators except the
summation. The proofs are similar \hlD{to those for bisimilarity},
see, e.g.~\cite{sangiorgi2017equations} \hl{for details.}

% next file: unique.tex

\subsection{Systems of contractions}
\label{ss:SysContr}

A \emph{system of contractions} is defined as a system of equations,
except that the contraction symbol $\mcontr$ is used in the place of
the equality symbol $=$. Thus a system of contractions is a set 
$\{  X_i \mcontr E_i\}_{i\in I}$
where $I$ is an  indexing set and expressions
$E_i$  may contain the  \behavC\  variables 
$\{  X_i\}_{i\in I}$.

\begin{definition}
\label{d:uniContra}
Given a system of contractions 
$\{  X_i \mcontr E_i\}_{i\in I}$, 
 we say that:
\begin{itemize}
\item $\til P$ is a \emph{solution (for $\mcontrBIS$) of the 
 system of contractions} if $\til P \mcontrBIS \til E [\til P]$;
\item the system has \emph{a unique solution (for $\approx$)}
if $\til P \approx \til Q$ whenever $\til P$ and $\til Q$ are both solutions.
\end{itemize}
\end{definition}

The guardedness of contractions follows Def.~\ref{def:guardness} (for equations).
% \begin{definition}
% \label{d:guarded}
% A system of contractions $\{  X_i \mcontr E_i\}_{i\in I}$
%  is
% \emph{weakly guarded}
% if,  in each    $E_i$, each occurrence of
% a \behavC\ variable is underneath a prefix.

% The system use \emph{weakly-guarded sums} if 
% each $E_i$ only makes use of guarded sums.
% \end{definition}

\begin{lemma}
\label{l:uptocon}
Suppose $\til P$ and $\til Q$ are solutions  for $\mcontrBIS$
 of a system of weakly-guarded contractions that uses 
weakly-guarded sums.
For any context $\qct$  that uses 
weakly-guarded sums,
if  $\ct{\til P}\Arr{\mu}  R$,
 then 
there is a context $\qctp$  that uses 
weakly-guarded sums
such that $R \mcontrBIS \ctp{\til P}$ and $\ct{\til Q} \Arcap{\mu}
 \wb \ctp{\til Q}$.\footnote{There's no typo here: $\ct{\til Q} \Arcap{\mu} \wb \ctp{\til
     Q}$ means $\exists {\til R}.\; \ct{\til Q} \Arcap{\mu} {\til R}
   \wb \ctp{\til Q}$. Same as in Lemma~\ref{l:ruptocon}.}
\end{lemma}

\begin{proof}{(sketch from \cite{sangiorgi2017equations})}
Let $n$ be the length of the transition $\ct{\til P}\Arr\mu R$  (the
number of `strong steps' of which it is composed), and  
let $\ctpp {\til P}$ and $\ctpp {\til Q}$  be the processes obtained
from  $\ct {\til P}$ and $\ct {\til Q}$ by unfolding the definitions
of the contractions $n$ times. Thus in $\qctpp$ each hole is
underneath at least $n$ prefixes, and cannot contribute to an action
in the first $n$ transitions; moreover all the contexts have only
weakly-guarded sums.

We have $\ct{\til P} \mcontrBIS \ctpp{\til P}$, and 
$\ct{\til Q} \mcontrBIS \ctpp{\til Q}$, 
 by the substitutivity  properties of $\mcontrBIS$ (we exploit here
 the syntactic constraints on sums). Moreover,
 since each hole of the  context $\qctpp$ is underneath at least $n$
 prefixes, applying  
the definition
 of $ \mcontrBIS$ on the transition 
 $\ct{\til P}\Arr{\mu}  R$, we infer the existence
 of $\qctp$ such that 
$
\ctpp{\til P}\Arcap{\mu} \ctp{\til P} \mexpaBIS R
$
and 
$
\ctpp{\til Q}\Arcap{\mu}  \ctp{\til Q} 
. $
Finally, again applying the definition of $\mcontrBIS$ on 
$\ct{\til Q} \mcontrBIS \ctpp{\til Q}$, 
we derive 
$
\ct{\til Q}\Arcap{\mu}  \wb \ctp{\til Q} 
.$
\end{proof}

\begin{theorem}[unique solution of contractions for $\wb$]
\label{t:contraBisimulationU}
A system of weakly-guarded contractions
having only weakly-guarded sums, has a unique solution for $\wb$.
\end{theorem}

\begin{proof}{(sketch from \cite{sangiorgi2017equations})}
Suppose $\til P$ and $\til Q$ are two such solutions (for $\wb$) and consider
the relation
\begin{equation}
\label{eq:R}
\R \DSdefi \{ 
(R,S) \st R \wb \ct{\til P}, S \wb \ct{\til Q} \mbox{ for some context
$\qct$ (having only weakly-guarded sums)} \} \enspace.
\end{equation}
We show that $\R$ is a bisimulation. \hl{Suppose $R\ \R\ S$ vis the context
$C$}, and $R \arr{\mu} R'$. We have to find $S'$ with $S \Arcap{\mu}
S'$ and $R'\ \R\ S'$. From $R \wb C[{\til P}]$, we derive $C[{\til P}]
\Arcap{\mu} R'' \wb R'$ for some $R''$. By Lemma~\ref{l:uptocon},
there is $C'$ with $R'' \mcontrBIS C'[{\til P}]$ and $C[{\til Q}]
\Arcap{\mu} \wb C'[{\til Q}]$. Hence, by definition of $\wb$, there is
also $S'$ with $S \Arcap{\mu} S' \wb C'[{\til Q}]$. This closes the
proof, as we have $R' \wb C'[{\til P}]$ and $S' \wb C'[{\til Q}]$.
\end{proof}

\section{Rooted contraction}
\label{ss:new}

The unique solution theorem of Section~\ref{ss:SysContr} requires a
constrained syntax for sums, due to the congruence and precongruence
problems of bisimilarity and contraction with such operator. 
We show here that the constraints can be
removed by moving to the induced congruence and precongruence, the
latter called \emph{rooted contraction}:
\begin{definition}
\label{d:rcontra}
Two processes $P$ and $Q$ are in \emph{rooted contraction}, written as
 $P\rcontr Q$, if
\begin{enumerate}
\item $P \arr\mu P'$ implies that there is $Q'$ with $Q \arr \mu Q'$
 and $P'\mcontrBIS Q'$;
\item $Q \arr\mu Q'$   implies that there is $P'$ with $P \Arr \mu
 P'$ and $P' \wb Q'$.
\end{enumerate}
\end{definition}

%Above definition adapts the definition of rooted
%bisimilarity on top of that of the  contraction preorder
%$\mcontrBIS$.  %% Reviewer said this sentance is unclear. I too think so.

\hl{The discovery of this definition is with help of HOL theorem
  prover and
the following two principles:} (1) Its definition must not be recursive,
instead it should resemble the definition of rooted bisimilarity
$\approx^c$ in Def.~\ref{d:rootedBisimilarity};
(2) It must be built on top of existing \emph{contracts}
relation $\mcontrBIS$, which we believe it's the \emph{right} one
because of its completeness. \hl{Multiple candicates were quickly tested,
finally only above definition is proven to be a precongruence, as the
following theorem states.} (The proof of this result is along the lines of the analogous result
for rooted bisimilarity with respect to bisimilarity.)

\begin{theorem}
\label{t:rcontrPrecongruence}
$\rcontr$ is a precongruence in CCS, and it is the
coarsest precongruence contained in $\contr$.
\end{theorem}  

For a system of rooted contractions, the meaning of 
``solution for $\rcontr$'' and of \emph{a unique solution for $\rapprox$}
is the expected one --- just replace in Definition~\ref{d:uniContra}  the preorder 
$\contr$ with $\rcontr$, and the equivalence 
$\approx$ with $\rapprox$.
%
For this new relation, the analogous of Lemma~\ref{l:uptocon} and of
Theorem~\ref{t:contraBisimulationU} can now be stated without constraints on the sum
operator.
The schema of the proofs is almost the same, because all needed
properties of $\rcontr$ in the proof is its precongruence, which is
now true for unrestricted contexts using direct sums:

\begin{lemma}
\label{l:ruptocon}
Suppose $\til P$ and $\til Q$ are solutions  for $\rcontr$ 
 of a system of weakly-guarded
contractions.
For any context $\qct$, 
if  $\ct{\til P}\Arr{\mu}  R$,
 then 
there is a  context $\qctp$
such that $R \mcontrBIS \ctp{\til P}$ and  $\ct{\til Q} \Arr{\mu}
 \wb \ctp{\til Q}$.
\end{lemma}

\begin{theorem}[unique solution of contractions for $\rapprox$]
\label{t:rcontraBisimulationU}
A system of weakly-guarded contractions has a unique solution 
 for $\rapprox$. (thus also for $\wb$)
\end{theorem} 

\begin{proof}
We first follow the same steps as in the proof of Theorem~\ref{t:contraBisimulationU} to show the relation $\R$ (now
with $\rcontr$ and unrestrict context $C$) in (\ref{eq:R}) is bisimulation,
exploting Lemma~\ref{l:ruptocon}. \hl{Then it remains to show that,} for
any two process $P$ and $Q$ with action $\mu$, if $P \arr{\mu} P'$ then
there is $Q'$ such that $Q \Arr{\mu} Q'$ (not $Q \Arcap{\mu} Q'$!) and
$P'\ \R\ Q'$, and also for the converse direction, exploting Lemma
4.13 of \cite{Mil89} \hl{(unexpected!)}. \hl{By definition of
\emph{bisimulation} (not $\wb$!) and $\approx^c$, we actually proved $P
\approx^c Q$ instead of $P \wb Q$.}
\end{proof}





\section{Formalization}

In the second half of this paper, we present a formalization of CCS
including the new concepts and theorems proposed in the first half of
this paper. The main purpose is to convince the readers that, there's no flaw\footnote{Perheps this is the first
  time a frontier development in Concurrency Theory ships with its
  formalization in the same paper.} in the informal proofs. The
formalization work is based on a formalization of CCS in HOL theorem
prover (HOL4), described in second author's MSc thesis
\cite{Chun:2017} and now lives in the official examples folder of HOL4 source
code\footnote{\url{https://github.com/HOL-Theorem-Prover/HOL}}.
The work is essentially a porting and extension of the early CCS
formalization by Monica Nesi \cite{Nesi} in HOL88.

We will not repeat all definitions and theorems of CCS again by their
formalized versions. Instead, we just focus on several highlights in
this work, i.e.
\begin{enumerate}
\item The use of recursion operators in formalizing process constants
  and variables;
\item The formalization of semantics context in $\lambda$-expressions and the theory of
  (pre)congruence for CCS;
\item The definition and uses of trace in the proof of unique solution of
  contractions theorem;
\item The proof of hard version of ``coarsest congruence
  contained in $\approx$'' theorem.
\end{enumerate}

\subsection{Simplifications and Limitations}

We have only formally verified simplified forms of the unique
solution theorems introduced in the first part. This is to not only
save the formalization efforts, but also to make us focusing on the most
important part of the proof. The most important simplification is
that, we have only proved all unique solution theorems with only single-variable equations
(contractions).

But such a simplification doesn't hurt the rigorousness of our entire
work, as once the single-equation version of the theorem is
proven, the general case is just a routine
adaptation for paper proofs. \footnote{However, it's impossible to directly derive the
  multi-variable versions of unique solution theorems from their
  single-variable versions.} \footnote{Also, I don't know if there
  exists a higher-level general
  duality theorem such that some kind of results on single-variable CCS
  equations must also hold for multi-variable equation groups. --Chun.}
It's not easy to formalize multi-variable CCS equations with
introducing a large number of definitions and intermediate
results. It's easy, however, to directly use the semantics context to
represent single-variable equations without the need of formalizing
the concept of ``equation'' at all.

\subsubsection{Limitations}

There're two major limitations in this work. One is the class of CCS
processes, i.e. we're limited with Finitary CCS which has no infinite
sums or parallel compositions. The only important theorem in which
arbitrary infinite sums of CCS processes must be used, is the following
``coarsest congruence contained in weak bisimilarity'' theorem without
any limitations on the involved processes:
\begin{equation}
\forall p\; q.\; p \approx^c\! q \Longleftrightarrow \forall r.\; p+r \approx
q+r
\end{equation}
Milner's classical proof of above result needs to assume that $p$ and $q$ didn't
use up all available labels. The newer proof by van Glabbek requires a
special choice of $r$ which is not weak bisimilar with any process
transited from $p$ and $q$, and the actual chosen process may contain
arbitrary large sums of CCS processes. This goes beyond the
expressiveness of HOL, as result we followed the same proof ideas but
limited with finite state CCS only.

The other limitation is, even for single-variable equation, the
variable doesn't appear inside any constant.  For example, an equation like
$P \approx E\{P/X\}$ in which $E = A$ and $A \overset{\mathrm{def}}{=}
\tau.A + b.X$ is not supported, because the equation variable $X$
appears inside the definition of constant $A$. On the other side, if
$E = A + X$ with whatever definition of constant $A$ in which there's
no appearence of $X$, such equations are supported. This is not
exactly the same as \emph{pure} equations (no constants anywhere), but
essentially works in the same way. This limitation comes from the fact
that, our congruence definition doesn't consider constants at all.
And to fix the issue, we must prove the Proposition 4.12 in Milner's
book first:
\begin{proposition}
Let $\tilde{E}$ and $\tilde{F}$ contain variables $\tilde{X}$ at
most. Let
$\tilde{A}\overset{\mathrm{def}}{=}\tilde{E}\{\tilde{A}/\tilde{X}\},
\tilde{B}\overset{\mathrm{def}}{=}\tilde{E}\{\tilde{B}/\tilde{X}\}$
and $\tilde{E}\sim\tilde{F}$. Then $\tilde{A}\sim\tilde{B}$.
\end{proposition}
which has one of the longest proof in that book, depends on bisimulation upto
$\sim$ and lemmas on free variables appeared in CCS
processes. This complicated part is planed as the future work.

\subsection{CCS, Transitions and Bisimilarities}

In our formalization of CCS, the set of transition actions are not
defined by the union of all visiable actions (names and conames) and the invisible
action $\tau$. Instead, we define only visible names as a simple
algebraic datatype ``\HOLinline{\ensuremath{\beta} \HOLTyOp{Label}}`` in which the type parameter
$\beta$ can be any other type (thus we don't know anything about the
cardinality of the set of all visiable actions), and then use HOL's
\texttt{optionTheory} to further extend the type by one more
element, $\tau$. Having a dedicated type for all visiable actions is quite
convenient and make some statements shorter. (e.g. the restriction
operator can only take a set of visible actions)

The type of CCS process is defined by an algebraic datatype in HOL,
i.e. an inductive defined finite structure constructed by unary and
binary operators.\footnote{It seems possible to switch to co-algebraic
datatypes for the support of potential infinite sums. Indeed, we never
need to induct on CCS processes in any proof so far.} 
It's not easy to directly extend this datatype with new operators,
without hurting the existing proved results.\footnote{This is usually
  called \emph{deep embedding} when implementing a new logic in theorem
  provers. In case of CCS, \emph{shallow embedding} seems impossible.}
Relabeling operator is not necessary from the theoretical view, as
it's possible to use \emph{syntactic
  substitution} instead\footnote{c.f. Section 4.1.2. (page 168) of Robert Gorrieri's CCS book.}
However, it's quite hard to formalize syntactic substitution for CCS,
as it requires to prove many results on the free and bound names
first.

Without constants (or recursion operator), a CCS process cannot
represent infinite-state LTS and many deep results in CCS cease to
exist then. Therefore any serious CCS formalization
must contain the support of constants (or recursion operator. Nowadays
most CCS textbooks and papers use constants, but from the formal view
it's better (or only possible) to use the original form: the recursion
operator, because any CCS process must be presented as single, finite term in
theorem prover. In Milner's book, this CCS operator is called \texttt{fix},
as it essentially returns the fixpoint of a CCS equation. We use the
symbol \texttt{rec} instead and limit ourself to single-variable case,
as we want to extend to multi-variable recursion operator in the
future with the original name. So instead of saying a CCS process $A$
with constant definitions
$A \overset{\mathrm{def}}{=} \tau.A + b.B$ and $B
\overset{\mathrm{def}}{=} b.B$, we use $\mathbf{rec}\;A.\, (\tau.A +
b.(\mathbf{rec}\;B.\,(b.B)))$ instead.

The extra beauty of above recursion opeator is, if we imagine the
\texttt{rec} symbol as $\lambda$ in $\lambda$-expressions, i.e. $\lambda A.\, (\tau.A +
b.(\lambda B.\,(b.B)))$, then we can see both $A$ and $B$ are bound
variables in this expression. On the other side, it's easy to write
an extra constant $C$ inside previous process without a surrounding $\lambda C$, and it must
be a \emph{free variable} of that $\lambda$-expressions. If we go back to the constant-based CCS
syntax, this would mean a malformed CCS process because not all the
constants used in it are defined. Our point here is:
\begin{enumerate}
\item Under the recursion-based CCS syntax, all possible CCS terms
  constructed from the algebraic datatype are well-formed, and those
  free variables can be seen and used\footnote{However, in our
    formalization of unique solution of equations theorems, we focus
    on single-variable case and directly used $\lambda$-expression
    (typed: $CCS\rightarrow CCS$) as equation. The current discussion
    is mainly for the support of multi-variable equations in the future.} as equation variables.
\item All proved theorems about CCS naturally take CCS process with
  extra free variables, as they by themselves have no transition, just
  like $\mathrm{0}$. (this is a natural consequence derived from SOS rules)
\end{enumerate}

The transition of CCS processes is defined by an inductive
relation, literally the same as SOS rules appearing in CCS
textbooks.\footnote{There's an interesting finding here: we never need to use the induction
theorem for transition relation at all, after having proved almost all
notable results for CCS. In another word, the whole CCS formal theory
builds normally, even if the transition relation
is defined co-inductively (by the same set of SOS rules). A potential
need is the prove the set of ending process of transitions from any
process doesn't use any new labels, but so far this result is not
formalized yet. The question
is: do the largest fixpoint and smallest fixpoint \emph{coincide} for
a relation defined by SOS rules? (if not, what else do we have in the
co-inductive relation?)} The only notable SOS rule which is different
with their usual version in textbooks, is the recursion rule:
\begin{alltt}
\HOLTokenTurnstile{} \HOLConst{CCS_Subst} \HOLFreeVar{E} (\HOLConst{rec} \HOLFreeVar{X} \HOLFreeVar{E}) \HOLFreeVar{X} \HOLTokenTransBegin\HOLFreeVar{u}\HOLTokenTransEnd \HOLFreeVar{E\sb{\mathrm{1}}} \HOLSymConst{\HOLTokenImp{}} \HOLConst{rec} \HOLFreeVar{X} \HOLFreeVar{E} \HOLTokenTransBegin\HOLFreeVar{u}\HOLTokenTransEnd \HOLFreeVar{E\sb{\mathrm{1}}}
\end{alltt}
which says, if we substitute all appearences of constant $X$ in $E$ to
$(\mathbf{rec}\; X E)$ and the resulting process has a transition to $E_1$
with action $u$, then $(\mathbf{rec}\; X E)$ has the same
transition.

\subsection{Semantic context, guardness and (pre)congruence}

The congruence (or substitutivity) of bisimilarity relations are
usually defined by a series of substitutivity theorems in most CCS
textbooks (c.f. Milners' and Gorrieri's). The concept of semantic context wasn't explicitly given in
Milner's book.\footnote{Milner says (p.\,84): ``.. congruence
  relation; that is, it is preserved by all algebraic contexts'', but
  he didn't further give a clear definition of algerbaic contexts.}
Later books made it explicitly, for exmaple, in Sangior's book\footnote{Introduction to Bisimulation and
  Coinduction},  a \emph{(semantic) context} is a process expression with a single occurrence of a hold
$[\cdot]$ in it, as a subexpression (c.f. p.99 of Sangiorgi's book).

In van Glabbek's paper\footnote{A characterisation of weak
  bisimulation congruence}), maybe for the first time the concept of
congruence is clearly defined in association with equivalence
relation:
\begin{definition}
An equivalence relation $\approx$ is a congruence for a set of
operators $L$ iff for every $n$-ary operator $f$ in $L$ one has
$g_1\approx h_1 \wedge \cdots \wedge g_n \approx h_n \Rightarrow
f(g_1,\ldots,g_n) \approx f(h_1,\ldots,h_n)$. This is the case iff for
every semantic context $C[\cdot]$ one has $g\approx h \Rightarrow
C[g]\approx C[h]$.
\end{definition}

van Glabbek further gave a way to construct congruence relations from
any existing equivalence relations:
\begin{definition}
\label{def:precc}
Given an equivalence relation $\sim$, its (pre)congruence closure,
notated as $[\sim]$ is defined by
\begin{equation}
g\,[\sim]\, h \mbox{ iff } C[g] \sim C[h] \mbox{ for every semantic
  context } C[\cdot].
\end{equation}
\end{definition}
And it's simple to prove that, for any equivation relation $\sim$,
$[\sim]$ is the coarsest congruence finer than $\sim$.

Here we have basically followed van Glabbek's approach. But the
restriction of single occurrence of the hole seems unnecessary: all
congruence relations in CCS hold also under semantic context with
multiple holes. Further more, a multi-hole semantic context $C$ can be
naturally used for representing an equation (or contraction), e.g. $P
\sim C[P]$.

It remains to find a suitable formal definition of semantic
context. There're multiple ways.\footnote{A literal approach is to
define another algebraic datatype which copies everything from the
existing CCS datatype, plus a dedicated symbol for that ``hole'', then
an inductively defined \texttt{replace} operator can be used for
replacing that hole with any CCS process, returning a normal CCS
datatype.} Here we have chosen to use $\lambda$-expressions (typed
$CCS\rightarrow CCS$) to represent a (multi-hole) semantic
context. The definition is inductive:\footnote{Due to symbol conflicts
  with $\lambda$-calculus, the prefix operator in CCS is represented
  as double-dot in HOL.}
\begin{alltt}
\HOLConst{CONTEXT} (\HOLTokenLambda{}\HOLBoundVar{t}. \HOLBoundVar{t})
\HOLConst{CONTEXT} (\HOLTokenLambda{}\HOLBoundVar{t}. \HOLFreeVar{p})
\HOLConst{CONTEXT} \HOLFreeVar{e} \HOLSymConst{\HOLTokenImp{}} \HOLConst{CONTEXT} (\HOLTokenLambda{}\HOLBoundVar{t}. \HOLFreeVar{a}\HOLSymConst{..}\HOLFreeVar{e} \HOLBoundVar{t})
\HOLConst{CONTEXT} \HOLFreeVar{e\sb{\mathrm{1}}} \HOLSymConst{\HOLTokenConj{}} \HOLConst{CONTEXT} \HOLFreeVar{e\sb{\mathrm{2}}} \HOLSymConst{\HOLTokenImp{}} \HOLConst{CONTEXT} (\HOLTokenLambda{}\HOLBoundVar{t}. \HOLFreeVar{e\sb{\mathrm{1}}} \HOLBoundVar{t} \HOLSymConst{+} \HOLFreeVar{e\sb{\mathrm{2}}} \HOLBoundVar{t})
\HOLConst{CONTEXT} \HOLFreeVar{e\sb{\mathrm{1}}} \HOLSymConst{\HOLTokenConj{}} \HOLConst{CONTEXT} \HOLFreeVar{e\sb{\mathrm{2}}} \HOLSymConst{\HOLTokenImp{}} \HOLConst{CONTEXT} (\HOLTokenLambda{}\HOLBoundVar{t}. \HOLFreeVar{e\sb{\mathrm{1}}} \HOLBoundVar{t} \HOLSymConst{\ensuremath{\parallel}} \HOLFreeVar{e\sb{\mathrm{2}}} \HOLBoundVar{t})
\HOLConst{CONTEXT} \HOLFreeVar{e} \HOLSymConst{\HOLTokenImp{}} \HOLConst{CONTEXT} (\HOLTokenLambda{}\HOLBoundVar{t}. \HOLSymConst{\ensuremath{\nu}} \HOLFreeVar{L} (\HOLFreeVar{e} \HOLBoundVar{t}))
\HOLConst{CONTEXT} \HOLFreeVar{e} \HOLSymConst{\HOLTokenImp{}} \HOLConst{CONTEXT} (\HOLTokenLambda{}\HOLBoundVar{t}. \HOLConst{relab} (\HOLFreeVar{e} \HOLBoundVar{t}) \HOLFreeVar{rf})
\end{alltt}

Under above definition, we can formally define the concept of
``precongruence'' and ``congruence'' in the following ways:
\begin{alltt}
\HOLConst{precongruence} \HOLFreeVar{R} \HOLSymConst{\HOLTokenEquiv{}}
\HOLSymConst{\HOLTokenForall{}}\HOLBoundVar{x} \HOLBoundVar{y} \HOLBoundVar{ctx}. \HOLConst{CONTEXT} \HOLBoundVar{ctx} \HOLSymConst{\HOLTokenImp{}} \HOLFreeVar{R} \HOLBoundVar{x} \HOLBoundVar{y} \HOLSymConst{\HOLTokenImp{}} \HOLFreeVar{R} (\HOLBoundVar{ctx} \HOLBoundVar{x}) (\HOLBoundVar{ctx} \HOLBoundVar{y})
\HOLConst{congruence} \HOLFreeVar{R} \HOLSymConst{\HOLTokenEquiv{}} \HOLConst{equivalence} \HOLFreeVar{R} \HOLSymConst{\HOLTokenConj{}} \HOLConst{precongruence} \HOLFreeVar{R}
\end{alltt}

A \emph{weak guarded} context is a context in which all holes appears
in the sub-expression of prefixed processes. We can easily define it
following the same idea:
\begin{alltt}
\HOLConst{WG} (\HOLTokenLambda{}\HOLBoundVar{t}. \HOLFreeVar{p})
\HOLConst{CONTEXT} \HOLFreeVar{e} \HOLSymConst{\HOLTokenImp{}} \HOLConst{WG} (\HOLTokenLambda{}\HOLBoundVar{t}. \HOLFreeVar{a}\HOLSymConst{..}\HOLFreeVar{e} \HOLBoundVar{t})
\HOLConst{WG} \HOLFreeVar{e\sb{\mathrm{1}}} \HOLSymConst{\HOLTokenConj{}} \HOLConst{WG} \HOLFreeVar{e\sb{\mathrm{2}}} \HOLSymConst{\HOLTokenImp{}} \HOLConst{WG} (\HOLTokenLambda{}\HOLBoundVar{t}. \HOLFreeVar{e\sb{\mathrm{1}}} \HOLBoundVar{t} \HOLSymConst{+} \HOLFreeVar{e\sb{\mathrm{2}}} \HOLBoundVar{t})
\HOLConst{WG} \HOLFreeVar{e\sb{\mathrm{1}}} \HOLSymConst{\HOLTokenConj{}} \HOLConst{WG} \HOLFreeVar{e\sb{\mathrm{2}}} \HOLSymConst{\HOLTokenImp{}} \HOLConst{WG} (\HOLTokenLambda{}\HOLBoundVar{t}. \HOLFreeVar{e\sb{\mathrm{1}}} \HOLBoundVar{t} \HOLSymConst{\ensuremath{\parallel}} \HOLFreeVar{e\sb{\mathrm{2}}} \HOLBoundVar{t})
\HOLConst{WG} \HOLFreeVar{e} \HOLSymConst{\HOLTokenImp{}} \HOLConst{WG} (\HOLTokenLambda{}\HOLBoundVar{t}. \HOLSymConst{\ensuremath{\nu}} \HOLFreeVar{L} (\HOLFreeVar{e} \HOLBoundVar{t}))
\HOLConst{WG} \HOLFreeVar{e} \HOLSymConst{\HOLTokenImp{}} \HOLConst{WG} (\HOLTokenLambda{}\HOLBoundVar{t}. \HOLConst{relab} (\HOLFreeVar{e} \HOLBoundVar{t}) \HOLFreeVar{rf})
\end{alltt}
(Notice the differences between a weak guarded context and a normal
one: $\lambda t. t$ is not weakly guarded as the variable is directly
exposed without any prefixed action. And $\lambda t. a.e[t]$ is weakly
guarded as long as $e[\cdot]$ is a semantic context, no necessary weakly guaded.)

By induction on the structure of $\lambda$-expressions, we can easily
prove that, a weakly guarded context is also a (normal) context, and the
composition of a (normal) context and a weakly guarded context is
still weakly guarded:
\begin{alltt}
WG_IS_CONTEXT:
\HOLTokenTurnstile{} \HOLConst{WG} \HOLFreeVar{e} \HOLSymConst{\HOLTokenImp{}} \HOLConst{CONTEXT} \HOLFreeVar{e}
CONTEXT_WG_combin:
\HOLTokenTurnstile{} \HOLConst{CONTEXT} \HOLFreeVar{c} \HOLSymConst{\HOLTokenConj{}} \HOLConst{WG} \HOLFreeVar{e} \HOLSymConst{\HOLTokenImp{}} \HOLConst{WG} (\HOLFreeVar{c} \HOLSymConst{\HOLTokenCompose} \HOLFreeVar{e})
\end{alltt}
It's a similar process to define strongly guarded context
(\texttt{SG}), sequential context (\texttt{SEQ}) and their
variants without direct sums (\texttt{GCONTEXT}, \texttt{WGS},
\texttt{GSEQ}). Some lemmas about their relationships have very long
(but trivial) formal proofs due to the combinatorial explosion of
inductions on their structures. For example, one such 
lemma says, for any semantic context $E$ which is both strongly guarded
and sequential (no direct sums) and another sequential context $H$ (no
direct sums), the composition $H \circ E$ is still both strongly
guarded and sequential (no direct sums):
\begin{alltt}
SG_GSEQ_combin:
\HOLTokenTurnstile{} \HOLConst{SG} \HOLFreeVar{E} \HOLSymConst{\HOLTokenConj{}} \HOLConst{GSEQ} \HOLFreeVar{E} \HOLSymConst{\HOLTokenImp{}} \HOLSymConst{\HOLTokenForall{}}\HOLBoundVar{H}. \HOLConst{GSEQ} \HOLBoundVar{H} \HOLSymConst{\HOLTokenImp{}} \HOLConst{SG} (\HOLBoundVar{H} \HOLSymConst{\HOLTokenCompose} \HOLFreeVar{E}) \HOLSymConst{\HOLTokenConj{}} \HOLConst{GSEQ} (\HOLBoundVar{H} \HOLSymConst{\HOLTokenCompose} \HOLFreeVar{E})
\end{alltt}

\subsection{Milner's unique solution of equations theorems}

Once the representation issue of CCS equations is resolved, the actual
proofs of Milner's unique solution of equations theorems is not very
interesting from the view of theorem proving, although it's a 
precise proof engineering work producing quite long proofs.
Since we have chosen to use semantic context to represent
single-variable equations, an equation like $P \sim E\{P/X\}$ now
becomes $P \sim E[P]$, in which $E$ is a (multi-hole) semantic
context, and there's no need to say it ``contains at most the variable
$X$'' any more, as equation variable doesn't appear in $E$ at
all. Under these simplifications, the formal proof of Milner's unique
solution of equations theorem for $\sim$ is a literal mapping for informal
proofs based on bisimulation upto $\sim$, induction and case analysis
of weakly guarded contexts. Below is the formal version of Lemma 4.13
and Proposition 4.14 in Milner's book:
\begin{alltt}
\HOLTokenTurnstile{} \HOLConst{WG} \HOLFreeVar{E} \HOLSymConst{\HOLTokenImp{}}
   \HOLSymConst{\HOLTokenForall{}}\HOLBoundVar{P} \HOLBoundVar{a} \HOLBoundVar{P\sp{\prime}}.
       \HOLFreeVar{E} \HOLBoundVar{P} \HOLTokenTransBegin\HOLBoundVar{a}\HOLTokenTransEnd \HOLBoundVar{P\sp{\prime}} \HOLSymConst{\HOLTokenImp{}}
       \HOLSymConst{\HOLTokenExists{}}\HOLBoundVar{E\sp{\prime}}. \HOLConst{CONTEXT} \HOLBoundVar{E\sp{\prime}} \HOLSymConst{\HOLTokenConj{}} (\HOLBoundVar{P\sp{\prime}} \HOLSymConst{=} \HOLBoundVar{E\sp{\prime}} \HOLBoundVar{P}) \HOLSymConst{\HOLTokenConj{}} \HOLSymConst{\HOLTokenForall{}}\HOLBoundVar{Q}. \HOLFreeVar{E} \HOLBoundVar{Q} \HOLTokenTransBegin\HOLBoundVar{a}\HOLTokenTransEnd \HOLBoundVar{E\sp{\prime}} \HOLBoundVar{Q}
\HOLTokenTurnstile{} \HOLConst{WG} \HOLFreeVar{E} \HOLSymConst{\HOLTokenImp{}} \HOLSymConst{\HOLTokenForall{}}\HOLBoundVar{P} \HOLBoundVar{Q}. \HOLBoundVar{P} \HOLSymConst{\HOLTokenStrongEQ} \HOLFreeVar{E} \HOLBoundVar{P} \HOLSymConst{\HOLTokenConj{}} \HOLBoundVar{Q} \HOLSymConst{\HOLTokenStrongEQ} \HOLFreeVar{E} \HOLBoundVar{Q} \HOLSymConst{\HOLTokenImp{}} \HOLBoundVar{P} \HOLSymConst{\HOLTokenStrongEQ} \HOLBoundVar{Q}
\end{alltt}

Milner's book only gives the unique solution of equations theorem of
observational congruence ($\approx^c$, Proposition 7.13):
\begin{alltt}
OBS_UNIQUE_SOLUTION:
\HOLTokenTurnstile{} \HOLConst{SG} \HOLFreeVar{E} \HOLSymConst{\HOLTokenConj{}} \HOLConst{SEQ} \HOLFreeVar{E} \HOLSymConst{\HOLTokenImp{}} \HOLSymConst{\HOLTokenForall{}}\HOLBoundVar{P} \HOLBoundVar{Q}. \HOLBoundVar{P} \HOLSymConst{\HOLTokenObsCongr} \HOLFreeVar{E} \HOLBoundVar{P} \HOLSymConst{\HOLTokenConj{}} \HOLBoundVar{Q} \HOLSymConst{\HOLTokenObsCongr} \HOLFreeVar{E} \HOLBoundVar{Q} \HOLSymConst{\HOLTokenImp{}} \HOLBoundVar{P} \HOLSymConst{\HOLTokenObsCongr} \HOLBoundVar{Q}
\end{alltt}
But its original proof is wrong and cannot be fixed even using the
alternative version of bisimulation upto $\approx$ (c.f. The problem
of ``Weak Bisimulation up to''). For this proof, we've made a direct
proof without using any bisimulation upto technique, which essentially
depends on the following relationship between observational congruence
and weak bisimulation (not bisimilarity):
\begin{alltt}
OBS_CONGR_BY_WEAK_BISIM:
\HOLTokenTurnstile{} \HOLConst{WEAK_BISIM} \HOLFreeVar{Wbsm} \HOLSymConst{\HOLTokenImp{}}
   \HOLSymConst{\HOLTokenForall{}}\HOLBoundVar{E} \HOLBoundVar{E\sp{\prime}}.
       (\HOLSymConst{\HOLTokenForall{}}\HOLBoundVar{u}.
            (\HOLSymConst{\HOLTokenForall{}}\HOLBoundVar{E\sb{\mathrm{1}}}. \HOLBoundVar{E} \HOLTokenTransBegin\HOLBoundVar{u}\HOLTokenTransEnd \HOLBoundVar{E\sb{\mathrm{1}}} \HOLSymConst{\HOLTokenImp{}} \HOLSymConst{\HOLTokenExists{}}\HOLBoundVar{E\sb{\mathrm{2}}}. \HOLBoundVar{E\sp{\prime}} \HOLTokenWeakTransBegin\HOLBoundVar{u}\HOLTokenWeakTransEnd \HOLBoundVar{E\sb{\mathrm{2}}} \HOLSymConst{\HOLTokenConj{}} \HOLFreeVar{Wbsm} \HOLBoundVar{E\sb{\mathrm{1}}} \HOLBoundVar{E\sb{\mathrm{2}}}) \HOLSymConst{\HOLTokenConj{}}
            \HOLSymConst{\HOLTokenForall{}}\HOLBoundVar{E\sb{\mathrm{2}}}. \HOLBoundVar{E\sp{\prime}} \HOLTokenTransBegin\HOLBoundVar{u}\HOLTokenTransEnd \HOLBoundVar{E\sb{\mathrm{2}}} \HOLSymConst{\HOLTokenImp{}} \HOLSymConst{\HOLTokenExists{}}\HOLBoundVar{E\sb{\mathrm{1}}}. \HOLBoundVar{E} \HOLTokenWeakTransBegin\HOLBoundVar{u}\HOLTokenWeakTransEnd \HOLBoundVar{E\sb{\mathrm{1}}} \HOLSymConst{\HOLTokenConj{}} \HOLFreeVar{Wbsm} \HOLBoundVar{E\sb{\mathrm{1}}} \HOLBoundVar{E\sb{\mathrm{2}}}) \HOLSymConst{\HOLTokenImp{}}
       \HOLBoundVar{E} \HOLSymConst{\HOLTokenObsCongr} \HOLBoundVar{E\sp{\prime}}
\end{alltt}

There's actual a third version for weak
bisimilarity not explicitly mentioned in Milner's book, mostly because
it's even more restrictive: beside being strongly guarded and sequential, the
equation variable $X$ must not appear in any direct
sum:\footnote{Here, \texttt{GSEQ} means a sequential context without
  direct sums.}
\begin{alltt}
WEAK_UNIQUE_SOLUTION:
\HOLTokenTurnstile{} \HOLConst{SG} \HOLFreeVar{E} \HOLSymConst{\HOLTokenConj{}} \HOLConst{GSEQ} \HOLFreeVar{E} \HOLSymConst{\HOLTokenImp{}} \HOLSymConst{\HOLTokenForall{}}\HOLBoundVar{P} \HOLBoundVar{Q}. \HOLBoundVar{P} \HOLSymConst{\HOLTokenWeakEQ} \HOLFreeVar{E} \HOLBoundVar{P} \HOLSymConst{\HOLTokenConj{}} \HOLBoundVar{Q} \HOLSymConst{\HOLTokenWeakEQ} \HOLFreeVar{E} \HOLBoundVar{Q} \HOLSymConst{\HOLTokenImp{}} \HOLBoundVar{P} \HOLSymConst{\HOLTokenWeakEQ} \HOLBoundVar{Q}
\end{alltt}

\subsection{Unique solution of contractions}

The major difficultis in formalizing Sangiorgi's unique solution of
contractions theorem is to prove its main lemma (c.f. Lemma 3.9 of
main paper), in which the length of weak transitions are used for
induction. Introducing a special vesion of weak transition relation
with length would be an easier choice, but we have chosen to use trace
instead, so that in the future it's possible to extend the work with
trace equivalence included.

A trace is represented by the beginning and ending CCS processes, plus
a list of action it passes. Insteading of defining it directly, we
have first defined a new concept called Reflexive Transitive Closure with a
List (LRTC):
\begin{alltt}
\HOLTokenTurnstile{} \HOLConst{LRTC} \HOLFreeVar{R} \HOLFreeVar{a} \HOLFreeVar{l} \HOLFreeVar{b} \HOLSymConst{\HOLTokenEquiv{}}
   \HOLSymConst{\HOLTokenForall{}}\HOLBoundVar{P}.
       (\HOLSymConst{\HOLTokenForall{}}\HOLBoundVar{x}. \HOLBoundVar{P} \HOLBoundVar{x} [] \HOLBoundVar{x}) \HOLSymConst{\HOLTokenConj{}}
       (\HOLSymConst{\HOLTokenForall{}}\HOLBoundVar{x} \HOLBoundVar{h} \HOLBoundVar{y} \HOLBoundVar{t} \HOLBoundVar{z}. \HOLFreeVar{R} \HOLBoundVar{x} \HOLBoundVar{h} \HOLBoundVar{y} \HOLSymConst{\HOLTokenConj{}} \HOLBoundVar{P} \HOLBoundVar{y} \HOLBoundVar{t} \HOLBoundVar{z} \HOLSymConst{\HOLTokenImp{}} \HOLBoundVar{P} \HOLBoundVar{x} (\HOLBoundVar{h}\HOLSymConst{::}\HOLBoundVar{t}) \HOLBoundVar{z}) \HOLSymConst{\HOLTokenImp{}}
       \HOLBoundVar{P} \HOLFreeVar{a} \HOLFreeVar{l} \HOLFreeVar{b}
\end{alltt}
For any labelled translation relation $R$, \HOLinline{\HOLConst{LRTC} \HOLFreeVar{R}} builds a new
relation accumulating all labels in the middle. Then the trace of CCS processes can be
defined by simply combining LRTC with the (strong) CCS transition
relation:
\begin{alltt}
\HOLConst{TRACE} \HOLSymConst{=} \HOLConst{LRTC} \HOLConst{TRANS}
\end{alltt}

If there's at most one visible action (label) in the list of actions of a trace,
then the trace is also a weak transition. We divided this observation
into two cases: no label and unique label. The definition of ``no
label'' in an action list is easy and clear:
\begin{alltt}
\HOLTokenTurnstile{} \HOLConst{NO_LABEL} \HOLFreeVar{L} \HOLSymConst{\HOLTokenEquiv{}} \HOLSymConst{\HOLTokenNeg{}}\HOLSymConst{\HOLTokenExists{}}\HOLBoundVar{l}. \HOLConst{MEM} (\HOLConst{label} \HOLBoundVar{l}) \HOLFreeVar{L}
\end{alltt}
while the definition of ``unique label'' can be done in many ways, in
which we have chosen to use the version learnt from Robert Beers in
a private discussion which prevented counting or filtering in the list:
\begin{alltt}
\HOLTokenTurnstile{} \HOLConst{UNIQUE_LABEL} \HOLFreeVar{u} \HOLFreeVar{L} \HOLSymConst{\HOLTokenEquiv{}}
   \HOLSymConst{\HOLTokenExists{}}\HOLBoundVar{L\sb{\mathrm{1}}} \HOLBoundVar{L\sb{\mathrm{2}}}.
       (\HOLBoundVar{L\sb{\mathrm{1}}} \HOLSymConst{\HOLTokenDoublePlus} [\HOLFreeVar{u}] \HOLSymConst{\HOLTokenDoublePlus} \HOLBoundVar{L\sb{\mathrm{2}}} \HOLSymConst{=} \HOLFreeVar{L}) \HOLSymConst{\HOLTokenConj{}}
       \HOLSymConst{\HOLTokenNeg{}}\HOLSymConst{\HOLTokenExists{}}\HOLBoundVar{l}. \HOLConst{MEM} (\HOLConst{label} \HOLBoundVar{l}) \HOLBoundVar{L\sb{\mathrm{1}}} \HOLSymConst{\HOLTokenDisj{}} \HOLConst{MEM} (\HOLConst{label} \HOLBoundVar{l}) \HOLBoundVar{L\sb{\mathrm{2}}}
\end{alltt}
It says, a label is unique in an action list iff it there's no other
labels in the rest part of the list.

The final relationship between traces and weak transitions is stated
and proved in the following lemma:
\begin{alltt}
WEAK_TRANS_AND_TRACE:
\HOLTokenTurnstile{} \HOLFreeVar{E} \HOLTokenWeakTransBegin\HOLFreeVar{u}\HOLTokenWeakTransEnd \HOLFreeVar{E\sp{\prime}} \HOLSymConst{\HOLTokenEquiv{}}
   \HOLSymConst{\HOLTokenExists{}}\HOLBoundVar{us}.
       \HOLConst{TRACE} \HOLFreeVar{E} \HOLBoundVar{us} \HOLFreeVar{E\sp{\prime}} \HOLSymConst{\HOLTokenConj{}} \HOLSymConst{\HOLTokenNeg{}}\HOLConst{NULL} \HOLBoundVar{us} \HOLSymConst{\HOLTokenConj{}} \HOLKeyword{if} \HOLFreeVar{u} \HOLSymConst{=} \HOLSymConst{\ensuremath{\tau}} \HOLKeyword{then} \HOLConst{NO_LABEL} \HOLBoundVar{us}
       \HOLKeyword{else} \HOLConst{UNIQUE_LABEL} \HOLFreeVar{u} \HOLBoundVar{us}
\end{alltt}
(A weak transition $E\overset{u}{\rightarrow}E'$ is a trace with non
empty action list: 1) without any visible label, if $u = \tau$, or 2)
$u$ is the unique label in the list, if $u \neq \tau$.)

To finish the proof of Lemma 3.9 in Sangiorgi's paper, we used above
lemma to convert weak transitions to traces, which either keep its
length or become shorter after passing weak bisimilations, then we
used above lemma again to convert the traces back to weak transitions.

The proof of the final unique solution of contractions theorem is just
a literal transition of its informal proofs: (noticed that the weakly
guarded context is the version without direct sums)
\begin{alltt}
\HOLTokenTurnstile{} \HOLConst{WGS} \HOLFreeVar{E} \HOLSymConst{\HOLTokenImp{}} \HOLSymConst{\HOLTokenForall{}}\HOLBoundVar{P} \HOLBoundVar{Q}. \HOLBoundVar{P} \HOLSymConst{\HOLTokenContracts{}} \HOLFreeVar{E} \HOLBoundVar{P} \HOLSymConst{\HOLTokenConj{}} \HOLBoundVar{Q} \HOLSymConst{\HOLTokenContracts{}} \HOLFreeVar{E} \HOLBoundVar{Q} \HOLSymConst{\HOLTokenImp{}} \HOLBoundVar{P} \HOLSymConst{\HOLTokenWeakEQ} \HOLBoundVar{Q}
\end{alltt}

\subsection{Unique solution of rooted contractions}

The proof of Unique solution of rooted contractions theorem is the
same proof steps of Unique solution of contractions theorem plus the
application of \texttt{OBS_CONGR_BY_WEAK_BISIM} at the
beginning.\footnote{In the thesis work, the conclusion of this theorem
is weak bisimilarity, and the proof is exactly the same as Unique
solution of contractions theorem. Now we got a stronger conclusion
using observational congruence, and the previous version becomes a
trivial collorary of the current version.} The
two proofs are quite similar, mostly because the only property we need
from (rooted) contraction is its precongruence. Once we have proved
the precongruence of rooted contracion, we can naturally use the
normal version of weakly guarded expressions with direct sums included.

\begin{alltt}
UNIQUE_SOLUTION_OF_ROOTED_CONTRACTIONS:
\HOLTokenTurnstile{} \HOLConst{WG} \HOLFreeVar{E} \HOLSymConst{\HOLTokenImp{}} \HOLSymConst{\HOLTokenForall{}}\HOLBoundVar{P} \HOLBoundVar{Q}. \HOLBoundVar{P} \HOLSymConst{\HOLTokenObsContracts} \HOLFreeVar{E} \HOLBoundVar{P} \HOLSymConst{\HOLTokenConj{}} \HOLBoundVar{Q} \HOLSymConst{\HOLTokenObsContracts} \HOLFreeVar{E} \HOLBoundVar{Q} \HOLSymConst{\HOLTokenImp{}} \HOLBoundVar{P} \HOLSymConst{\HOLTokenObsCongr} \HOLBoundVar{Q}
\end{alltt}

\subsection{Coarsest (pre)congruence contained in $\approx$ (or $\succeq_{\mathrm{bis}}$)}

The ``coarsest congruence contained in weak bisimilarity ($\approx$)''
theorem in CCS is somehow special, as its current known proofs either
rely on quite restricted conditions, or have an extremely complicated proof
(c.f. van Glabbek's paper) in
which ordinal theory is required.  Actually even the relationship
between its name and statement is not well explained in many CCS
textbooks. But van Glabbek's paper has given the so far clearest
explainion, here we briefly repeat his arguments:

As we know weak bisimilarity ($\approx$) is not (real) congruence, as
it doesn't satisfy subsitutivity on direct sums (but if the CCS syntax
is non-standard, i.e. has only prefixed sums, $\approx$ is indeed a
congruence). The purpose is to find a coarsest congruence contained in
weak bisimilarity. (``coarsest'' means, any other congruence finer than it must be contained in it)
There're two ways to build a congruence from weak bisimilarity, one
way is the standard definition for observational congruence (rooted
weak bisimilarity) $\approx^c$ in CCS textbooks, but even it's proven to be a
congruence we don't know if it's coarsest one.  The other way is to
build a (pre)congruence closure (Def.\,\ref{def:precc}) directly upon
the original weak bisimilarity relation, we call the resulting
relation ``Weak bisimilarity congruence'' ($[\approx]$):
\begin{alltt}
\HOLConst{WEAK_CONGR} \HOLSymConst{=} \HOLConst{CC} \HOLConst{WEAK_EQUIV}
\HOLConst{CC} \HOLFreeVar{R} \HOLSymConst{=} (\HOLTokenLambda{}\HOLBoundVar{g} \HOLBoundVar{h}. \HOLSymConst{\HOLTokenForall{}}\HOLBoundVar{c}. \HOLConst{CONTEXT} \HOLBoundVar{c} \HOLSymConst{\HOLTokenImp{}} \HOLFreeVar{R} (\HOLBoundVar{c} \HOLBoundVar{g}) (\HOLBoundVar{c} \HOLBoundVar{h}))
\end{alltt}
It can be shown that any such (pre)congruence closure is automatically coarset.

Now it remains to prove that, the congrunce relation built by above
two quite different approaches actually coincide. To achive this goal,
we first noticed that, all other operators beside sums used in
semantic context doesn't matter, because they're already substituible
for weak bisimilarity. The only important operator is the sum
operator. To focus on this important operator, we can temporily
introduce another concept called \emph{sum equivalence}:
\begin{alltt}
\HOLConst{SUM_EQUIV} \HOLSymConst{=} (\HOLTokenLambda{}\HOLBoundVar{p} \HOLBoundVar{q}. \HOLSymConst{\HOLTokenForall{}}\HOLBoundVar{r}. \HOLBoundVar{p} \HOLSymConst{+} \HOLBoundVar{r} \HOLSymConst{\HOLTokenWeakEQ} \HOLBoundVar{q} \HOLSymConst{+} \HOLBoundVar{r})
\end{alltt}
It can be shown that, weak bisimilarity congruence implies this sum
equivalence. Thus if we can further prove that the sum
equivalence imples observational congrence, all the three relations
must coincide together, as shown in the following diagram:
\begin{displaymath}
\xymatrix{
{\textrm{Weak bisimilarity } (\approx)} & {\textrm{Sum
    equivalence } (\approx^+)} \ar@/^3ex/[ldd]^{\subseteq ?}\\
{\textrm{Weak bisim. congruence } ([\approx])}
\ar[u]^{\subseteq} \ar[ru]^{\subseteq} \\
{\textrm{Rooted bisimilarity } (\approx^c)} \ar[u]^{\subseteq}
}
\end{displaymath}
In another words, observational congruence is the coarsest congruence
contained in weak bisimilary if and only if
\begin{equation}
\forall p\; q.\; p \approx^c\! q \Longleftrightarrow \forall r.\; p+r \approx
q+r
\end{equation}

The classical assumption for proving above theorem requires that $p$
and $q$ do not use up all available visible actions,
i.e. $\mathrm{fn}(p) \cup \mathrm{fn}(q) \neq \mathscr{L}$. But it's
not easy to formalize and use such an assumption without a detailed
treatment on free and bound names (visible actions) of CCS
processes.\footnote{There're totally four such concepts: 1) free names
are all visible actions appearing in a CCS term without surrounding
$\nu$ (restriction) operator on the same action; 2) bound names are
the set of all actions ever used by $\nu$ (restriction) operator; 3)
free variables (or equation variables) are those variables without a
definition given by recursion
operator; 4) bound variables (process constants) are variables with
definitions given by recursion operator. All CCS results using these
concepts are not touched so far, although these four concepts are
successfully defined using HOL's set and list theories.} However, by
analyzing the proof steps, we found that, what's really required is to
not use up all available labels in those weak transitions directly
lead from $p$ and $q$. In another words, even they have used all
available labels, as long as their first weak transitions didn't, the
whole proof can still be finished.\footnote{Further more, $p$ and $q$
  can be considered separately: the proof can be finished as long as
  \emph{each} of them didn't use up all labels on first weak
  transition, while the union of these labels are all labels.}
We have formalized this property of
CCS process and call it ``free action'' property:
\begin{alltt}
\HOLTokenTurnstile{} \HOLConst{free_action} \HOLFreeVar{p} \HOLSymConst{\HOLTokenEquiv{}} \HOLSymConst{\HOLTokenExists{}}\HOLBoundVar{a}. \HOLSymConst{\HOLTokenForall{}}\HOLBoundVar{p\sp{\prime}}. \HOLSymConst{\HOLTokenNeg{}}(\HOLFreeVar{p} \HOLTokenWeakTransBegin\HOLConst{label} \HOLBoundVar{a}\HOLTokenWeakTransEnd \HOLBoundVar{p\sp{\prime}})
\end{alltt}
With this property, the classical form of this theorem that we have
formally proved is:
\begin{alltt}
\HOLTokenTurnstile{} \HOLConst{free_action} \HOLFreeVar{p} \HOLSymConst{\HOLTokenConj{}} \HOLConst{free_action} \HOLFreeVar{q} \HOLSymConst{\HOLTokenImp{}}
   (\HOLFreeVar{p} \HOLSymConst{\HOLTokenObsCongr} \HOLFreeVar{q} \HOLSymConst{\HOLTokenEquiv{}} \HOLSymConst{\HOLTokenForall{}}\HOLBoundVar{r}. \HOLFreeVar{p} \HOLSymConst{+} \HOLBoundVar{r} \HOLSymConst{\HOLTokenWeakEQ} \HOLFreeVar{q} \HOLSymConst{+} \HOLBoundVar{r})
\end{alltt}

If we drop this classical assumption, then the proof becomes much
harder, as given a van Glabbek's paper. The most important
intermediate result we have proved here, is the following lemma:
\begin{alltt}
\HOLTokenTurnstile{} (\HOLSymConst{\HOLTokenExists{}}\HOLBoundVar{k}.
        \HOLConst{STABLE} \HOLBoundVar{k} \HOLSymConst{\HOLTokenConj{}} (\HOLSymConst{\HOLTokenForall{}}\HOLBoundVar{p\sp{\prime}} \HOLBoundVar{u}. \HOLFreeVar{p} \HOLTokenWeakTransBegin\HOLBoundVar{u}\HOLTokenWeakTransEnd \HOLBoundVar{p\sp{\prime}} \HOLSymConst{\HOLTokenImp{}} \HOLSymConst{\HOLTokenNeg{}}(\HOLBoundVar{p\sp{\prime}} \HOLSymConst{\HOLTokenWeakEQ} \HOLBoundVar{k})) \HOLSymConst{\HOLTokenConj{}}
        \HOLSymConst{\HOLTokenForall{}}\HOLBoundVar{q\sp{\prime}} \HOLBoundVar{u}. \HOLFreeVar{q} \HOLTokenWeakTransBegin\HOLBoundVar{u}\HOLTokenWeakTransEnd \HOLBoundVar{q\sp{\prime}} \HOLSymConst{\HOLTokenImp{}} \HOLSymConst{\HOLTokenNeg{}}(\HOLBoundVar{q\sp{\prime}} \HOLSymConst{\HOLTokenWeakEQ} \HOLBoundVar{k})) \HOLSymConst{\HOLTokenImp{}}
   (\HOLSymConst{\HOLTokenForall{}}\HOLBoundVar{r}. \HOLFreeVar{p} \HOLSymConst{+} \HOLBoundVar{r} \HOLSymConst{\HOLTokenWeakEQ} \HOLFreeVar{q} \HOLSymConst{+} \HOLBoundVar{r}) \HOLSymConst{\HOLTokenImp{}}
   \HOLFreeVar{p} \HOLSymConst{\HOLTokenObsCongr} \HOLFreeVar{q}
\end{alltt}
It roughly says, for any two processes $p$ and $q$, if we can find
another stable (no $\tau$-transitions) process $k$ which is not weak bisimilar to any transition of
$p$ and $q$, then the hard part of our main theorem is proved. In
practice, once two processes were given, it's not hard to find
such a process, but the arbitrariness of $p$ and $q$ made this result
extremely hard to prove. The method given by van Glabbek requires a
construction of arbitrarily non-bisimilar processes called ``Klop
processes'':
\begin{definition}{(Klop processes)}
For each ordinal $\lambda$, and an arbitrary chosen non-$\tau$ action $a$,
define a CCS process $k_\lambda$ as follows:
\begin{enumerate}
\item $k_0 = 0$,
\item $k_{\lambda+1} = k_\lambda + a.k_\lambda$ and
\item for $\lambda$ a limit ordinal, $k_\lambda = \sum_{\mu < \lambda}
  k_\mu$, meaning that $k_\lambda$ is constructed from all graphs
  $k_\mu$ for $\mu < \lambda$ by identifying their root.
\end{enumerate}
\end{definition}
It's not hard to prove that, all $k_i$ are non-bisimilar (not only
weakly but also strongly). The idea is, for any two processes, the union of sets of their all
transitions cannot be arbitrarily large: it has to be limited by an
ordinal number. But above contruction can be arbitrarily large, so
there always exists a Klop process which can be used to satisfy above
lemma (and finish the proof).

However, this goes beyond HOL's expressivity to define above process, mostly
because there's no way to express infinite ``sums'' in CCS
datatype.\footnote{There're actually several ways to modify the
  datatype to support infinite sums, but none of these infinity can be
arbitrary. The full version of ordinal theory cannot be expressed in HOL.}
What we can do is to eliminate the last branch with limiting
ordinals, and define Klop processes only on finite cases:
\begin{alltt}
\HOLConst{KLOP} \HOLFreeVar{a} \HOLNumLit{0} \HOLSymConst{=} \HOLConst{nil}
\HOLConst{KLOP} \HOLFreeVar{a} (\HOLConst{SUC} \HOLFreeVar{n}) \HOLSymConst{=} \HOLConst{KLOP} \HOLFreeVar{a} \HOLFreeVar{n} \HOLSymConst{+} \HOLConst{label} \HOLFreeVar{a}\HOLSymConst{..}\HOLConst{KLOP} \HOLFreeVar{a} \HOLFreeVar{n}\hfill[KLOP_def]
\end{alltt}
But this means we much assume the two processes $p$ and $q$ are
finite-state, i.e. their corresponding LTS graphs have only finite
states. Choosing an element from a countable infinite set of processes
to make sure it's not bisimiar with a given set of processes, this is
essentially a pure set-theoretic problem, as formalized and proved in
the following lemma:
\begin{alltt}
\HOLTokenTurnstile{} \HOLConst{equivalence} \HOLFreeVar{R} \HOLSymConst{\HOLTokenImp{}}
   \HOLConst{FINITE} \HOLFreeVar{A} \HOLSymConst{\HOLTokenConj{}} \HOLConst{INFINITE} \HOLFreeVar{B} \HOLSymConst{\HOLTokenConj{}}
   (\HOLSymConst{\HOLTokenForall{}}\HOLBoundVar{x} \HOLBoundVar{y}. \HOLBoundVar{x} \HOLSymConst{\HOLTokenIn{}} \HOLFreeVar{B} \HOLSymConst{\HOLTokenConj{}} \HOLBoundVar{y} \HOLSymConst{\HOLTokenIn{}} \HOLFreeVar{B} \HOLSymConst{\HOLTokenConj{}} \HOLBoundVar{x} \HOLSymConst{\HOLTokenNotEqual{}} \HOLBoundVar{y} \HOLSymConst{\HOLTokenImp{}} \HOLSymConst{\HOLTokenNeg{}}\HOLFreeVar{R} \HOLBoundVar{x} \HOLBoundVar{y}) \HOLSymConst{\HOLTokenImp{}}
   \HOLSymConst{\HOLTokenExists{}}\HOLBoundVar{k}. \HOLBoundVar{k} \HOLSymConst{\HOLTokenIn{}} \HOLFreeVar{B} \HOLSymConst{\HOLTokenConj{}} \HOLSymConst{\HOLTokenForall{}}\HOLBoundVar{n}. \HOLBoundVar{n} \HOLSymConst{\HOLTokenIn{}} \HOLFreeVar{A} \HOLSymConst{\HOLTokenImp{}} \HOLSymConst{\HOLTokenNeg{}}\HOLFreeVar{R} \HOLBoundVar{n} \HOLBoundVar{k}
\end{alltt}

With all these results, finally we can prove the ``coarsest congruence
contained in $\approx$'' theorem for finite-state CCS:
\begin{alltt}
\HOLTokenTurnstile{} \HOLConst{finite_state} \HOLFreeVar{p} \HOLSymConst{\HOLTokenConj{}} \HOLConst{finite_state} \HOLFreeVar{q} \HOLSymConst{\HOLTokenImp{}}
   (\HOLFreeVar{p} \HOLSymConst{\HOLTokenObsCongr} \HOLFreeVar{q} \HOLSymConst{\HOLTokenEquiv{}} \HOLSymConst{\HOLTokenForall{}}\HOLBoundVar{r}. \HOLFreeVar{p} \HOLSymConst{+} \HOLBoundVar{r} \HOLSymConst{\HOLTokenWeakEQ} \HOLFreeVar{q} \HOLSymConst{+} \HOLBoundVar{r})
\end{alltt}

For contraction and rooted contraction, the situation (and proof
steps) is exactly the same:
\begin{displaymath}
\xymatrix{
{\textrm{Contraction } (\succeq_{\mathrm{bis}})} & {\textrm{Sum
    contraction } (\succeq_{\mathrm{bis}}^+)} \ar@/^3ex/[ldd]^{\subseteq ?}\\
{\textrm{Contraction precongruence } ([\succeq_{\mathrm{bis}}])}
\ar[u]^{\subseteq} \ar[ru]^{\subseteq} \\
{\textrm{Rooted contraction } (\succeq_{\mathrm{bis}}^c)} \ar[u]^{\subseteq}
}
\end{displaymath}

And we got two versions of theorem saying rooted contraction is the
coarsest precongruence of (bisimilarity) contraction:
\begin{alltt}
\HOLTokenTurnstile{} \HOLConst{free_action} \HOLFreeVar{p} \HOLSymConst{\HOLTokenConj{}} \HOLConst{free_action} \HOLFreeVar{q} \HOLSymConst{\HOLTokenImp{}}
   (\HOLFreeVar{p} \HOLSymConst{\HOLTokenObsContracts} \HOLFreeVar{q} \HOLSymConst{\HOLTokenEquiv{}} \HOLSymConst{\HOLTokenForall{}}\HOLBoundVar{r}. \HOLFreeVar{p} \HOLSymConst{+} \HOLBoundVar{r} \HOLSymConst{\HOLTokenContracts{}} \HOLFreeVar{q} \HOLSymConst{+} \HOLBoundVar{r})
\HOLTokenTurnstile{} \HOLConst{finite_state} \HOLFreeVar{p} \HOLSymConst{\HOLTokenConj{}} \HOLConst{finite_state} \HOLFreeVar{q} \HOLSymConst{\HOLTokenImp{}}
   (\HOLFreeVar{p} \HOLSymConst{\HOLTokenObsContracts} \HOLFreeVar{q} \HOLSymConst{\HOLTokenEquiv{}} \HOLSymConst{\HOLTokenForall{}}\HOLBoundVar{r}. \HOLFreeVar{p} \HOLSymConst{+} \HOLBoundVar{r} \HOLSymConst{\HOLTokenContracts{}} \HOLFreeVar{q} \HOLSymConst{+} \HOLBoundVar{r})
\end{alltt}

\end{document}
