% \section{To Do}
% \label{s:todo}

% \finish{discuss multiple equations/contractions} 

% The formalisation of multi-variable context (or expressions) and the version of
% ``unique solutions'' theorems with multiple equations/contractions, is
% a relative hard problem. To the best of our knowledge, so far there's no formalization (in any kind of
% process calculi) touching these areas yet. However, we've made a few initial steps
% of this interesting problem.

\section{Conclusions and future work}
\label{s:concl}

We have highlighted a formalisation of the theory of CCS in the 
HOL4 theorem prover (for lack of space we have not discussed 
the formalisation of some basic algebraic theory, of the basic
properties of the expansion preorder,   and of a few
 versions of `bisimulation up to'
techniques such as  bisimulation up-to bisimilarity). 
The formalisation has focused on the theory of
unique solution of equations and contractions. 
The formalisation has also allowed us to further develop the theory,
notably the basic properties of rooted contraction, and the unique
solution theorem for it with respect to rooted bisimilarity. 
The formalisation brings up and exploits similarities between results
and proofs for different equivalences and preorders. 
We think that the statements in the formalisation are easy to read and
understand, as they remain very close  to  the ordinary  statements of
the theory of CCS. 
Moreover, 
the logic foundations of HOL makes the whole work (or individual
parts) easily portable to other theorem provers.


As future work,  it would be worth extending the current work
to multiple equations and contractions; and considering other equivalences
and preorders, notably trace relations.  
On another line, one could  
examine the formalisation of  a different approach to unique
solutions, by Durier et al.\ \cite{DurierHS17}, in which the use of contraction is
replaced by semantic conditions on process transitions such as
divergence.  

% Further plan on the formalisation mainly includes: 1) the extension to
% multi-variable equation/contractions. 2) the support of recursion
% operators in 
% CCS context and expressions.


% We believe that, beside the discovery of the rooted contraction $\rcontr$
% with a more elegant unique solution theorem,
% our CCS formalisation in HOL4 has also
% provided a solid formal framework for future theoretical developments in
% Concurrency Theory, particularily for process algebras like CCS. It's
% easily understandable, with statements extremely close to the original
% textbook. The logic foundations of HOL makes the whole work (or individual
% parts) easily portable to other theorem provers.

\subsection*{Acknowledgements}

We would like to thank many people from the HOL community
for their help in the past two years, in particular 
Michael Norrish, Thomas Tuerk, 
Ramana Kumar, Konrad Slind, Robert Beers, and Jeremy Dawson. 
% The second half of this
% paper was written in memory of  Mike J. Gordon.
