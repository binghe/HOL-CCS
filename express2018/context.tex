%%%% -*- Mode: LaTeX -*-
%%
%% This is the draft of the 2nd part of EXPRESS/SOS 2018 paper, co-authored by
%% Prof. Davide Sangiorgi and Chun Tian.

\subsection{Context, guardness and (pre)congruence}

We need to find a suitable formal definition of 
context. There're multiple ways. Here we have chosen to use $\lambda$-expressions (typed
$CCS\rightarrow CCS$) to represent a (multi-hole) 
context. The definition is inductive:
\begin{alltt}
\HOLConst{CONTEXT} (\HOLTokenLambda{}\HOLBoundVar{t}. \HOLBoundVar{t})
\HOLConst{CONTEXT} (\HOLTokenLambda{}\HOLBoundVar{t}. \HOLFreeVar{p})
\HOLConst{CONTEXT} \HOLFreeVar{e} \HOLSymConst{\HOLTokenImp{}} \HOLConst{CONTEXT} (\HOLTokenLambda{}\HOLBoundVar{t}. \HOLFreeVar{a}\HOLSymConst{..}\HOLFreeVar{e} \HOLBoundVar{t})
\HOLConst{CONTEXT} \HOLFreeVar{e\sb{\mathrm{1}}} \HOLSymConst{\HOLTokenConj{}} \HOLConst{CONTEXT} \HOLFreeVar{e\sb{\mathrm{2}}} \HOLSymConst{\HOLTokenImp{}} \HOLConst{CONTEXT} (\HOLTokenLambda{}\HOLBoundVar{t}. \HOLFreeVar{e\sb{\mathrm{1}}} \HOLBoundVar{t} \HOLSymConst{+} \HOLFreeVar{e\sb{\mathrm{2}}} \HOLBoundVar{t})
\HOLConst{CONTEXT} \HOLFreeVar{e\sb{\mathrm{1}}} \HOLSymConst{\HOLTokenConj{}} \HOLConst{CONTEXT} \HOLFreeVar{e\sb{\mathrm{2}}} \HOLSymConst{\HOLTokenImp{}} \HOLConst{CONTEXT} (\HOLTokenLambda{}\HOLBoundVar{t}. \HOLFreeVar{e\sb{\mathrm{1}}} \HOLBoundVar{t} \HOLSymConst{\ensuremath{\parallel}} \HOLFreeVar{e\sb{\mathrm{2}}} \HOLBoundVar{t})
\HOLConst{CONTEXT} \HOLFreeVar{e} \HOLSymConst{\HOLTokenImp{}} \HOLConst{CONTEXT} (\HOLTokenLambda{}\HOLBoundVar{t}. \HOLSymConst{\ensuremath{\nu}} \HOLFreeVar{L} (\HOLFreeVar{e} \HOLBoundVar{t}))
\HOLConst{CONTEXT} \HOLFreeVar{e} \HOLSymConst{\HOLTokenImp{}} \HOLConst{CONTEXT} (\HOLTokenLambda{}\HOLBoundVar{t}. \HOLConst{relab} (\HOLFreeVar{e} \HOLBoundVar{t}) \HOLFreeVar{rf})\hfill{[CONTEXT_rules]}
\end{alltt}

Under above definition, we can formally define the concept of
``precongruence'' and ``congruence'' in the following ways:
\begin{alltt}
\HOLConst{precongruence} \HOLFreeVar{R} \HOLSymConst{\HOLTokenEquiv{}}
\HOLSymConst{\HOLTokenForall{}}\HOLBoundVar{x} \HOLBoundVar{y} \HOLBoundVar{ctx}. \HOLConst{CONTEXT} \HOLBoundVar{ctx} \HOLSymConst{\HOLTokenImp{}} \HOLFreeVar{R} \HOLBoundVar{x} \HOLBoundVar{y} \HOLSymConst{\HOLTokenImp{}} \HOLFreeVar{R} (\HOLBoundVar{ctx} \HOLBoundVar{x}) (\HOLBoundVar{ctx} \HOLBoundVar{y})\hfill{[precongruence_def]}
\HOLConst{congruence} \HOLFreeVar{R} \HOLSymConst{\HOLTokenEquiv{}} \HOLConst{equivalence} \HOLFreeVar{R} \HOLSymConst{\HOLTokenConj{}} \HOLConst{precongruence} \HOLFreeVar{R}\hfill{[congruence_def]}
\end{alltt}

A \emph{weak guarded} context is a context in which all holes appears
in the sub-expression of prefixed processes. We can easily define it
following the same idea:
\begin{alltt}
\HOLConst{WG} (\HOLTokenLambda{}\HOLBoundVar{t}. \HOLFreeVar{p})
\HOLConst{CONTEXT} \HOLFreeVar{e} \HOLSymConst{\HOLTokenImp{}} \HOLConst{WG} (\HOLTokenLambda{}\HOLBoundVar{t}. \HOLFreeVar{a}\HOLSymConst{..}\HOLFreeVar{e} \HOLBoundVar{t})
\HOLConst{WG} \HOLFreeVar{e\sb{\mathrm{1}}} \HOLSymConst{\HOLTokenConj{}} \HOLConst{WG} \HOLFreeVar{e\sb{\mathrm{2}}} \HOLSymConst{\HOLTokenImp{}} \HOLConst{WG} (\HOLTokenLambda{}\HOLBoundVar{t}. \HOLFreeVar{e\sb{\mathrm{1}}} \HOLBoundVar{t} \HOLSymConst{+} \HOLFreeVar{e\sb{\mathrm{2}}} \HOLBoundVar{t})
\HOLConst{WG} \HOLFreeVar{e\sb{\mathrm{1}}} \HOLSymConst{\HOLTokenConj{}} \HOLConst{WG} \HOLFreeVar{e\sb{\mathrm{2}}} \HOLSymConst{\HOLTokenImp{}} \HOLConst{WG} (\HOLTokenLambda{}\HOLBoundVar{t}. \HOLFreeVar{e\sb{\mathrm{1}}} \HOLBoundVar{t} \HOLSymConst{\ensuremath{\parallel}} \HOLFreeVar{e\sb{\mathrm{2}}} \HOLBoundVar{t})
\HOLConst{WG} \HOLFreeVar{e} \HOLSymConst{\HOLTokenImp{}} \HOLConst{WG} (\HOLTokenLambda{}\HOLBoundVar{t}. \HOLSymConst{\ensuremath{\nu}} \HOLFreeVar{L} (\HOLFreeVar{e} \HOLBoundVar{t}))
\HOLConst{WG} \HOLFreeVar{e} \HOLSymConst{\HOLTokenImp{}} \HOLConst{WG} (\HOLTokenLambda{}\HOLBoundVar{t}. \HOLConst{relab} (\HOLFreeVar{e} \HOLBoundVar{t}) \HOLFreeVar{rf})\hfill{[WG_rules]}
\end{alltt}
(Notice the differences between a weak guarded context and a normal
one: $\lambda t. t$ is not weakly guarded as the variable is directly
exposed without any prefixed action. And $\lambda t. a.e[t]$ is weakly
guarded as long as $e[\cdot]$ is a context, not necessary weakly guaded.)

It's a similar process to define strongly guarded context
(\texttt{SG}), sequential context (\texttt{SEQ}) and their
variants without direct sums (\texttt{GCONTEXT}, \texttt{WGS},
\texttt{GSEQ}). Some lemmas about their relationships have very long
(but trivial) formal proofs due to multiple levels of
inductions on context structures. For example, one such 
lemma says, for any  context $E$ which is both strongly guarded
and sequential (no direct sums) and another sequential context $H$ (no
direct sums), the composition $H \circ E$ is still both strongly
guarded and sequential (no direct sums):
\begin{alltt}
\HOLTokenTurnstile{} \HOLSymConst{\HOLTokenForall{}}\HOLBoundVar{E}. \HOLConst{SG} \HOLBoundVar{E} \HOLSymConst{\HOLTokenConj{}} \HOLConst{GSEQ} \HOLBoundVar{E} \HOLSymConst{\HOLTokenImp{}} \HOLSymConst{\HOLTokenForall{}}\HOLBoundVar{H}. \HOLConst{GSEQ} \HOLBoundVar{H} \HOLSymConst{\HOLTokenImp{}} \HOLConst{SG} (\HOLBoundVar{H} \HOLSymConst{\HOLTokenCompose} \HOLBoundVar{E}) \HOLSymConst{\HOLTokenConj{}} \HOLConst{GSEQ} (\HOLBoundVar{H} \HOLSymConst{\HOLTokenCompose} \HOLBoundVar{E})
\end{alltt}

% \subsection{Milner's ``unique solution of equations'' theorems}

% Once the representation issue of CCS equations is resolved, the actual
% proofs of Milner's unique solution of equations theorems is not very
% interesting from the view of theorem proving, although it's a 
% precise proof engineering work producing quite long proofs.
% Since we have chosen to use  context to represent
% single-variable equations, an equation like $P \sim E\{P/X\}$ now
% becomes $P \sim E[P]$, in which $E$ is a (multi-hole) 
% context, and there's no need to say it ``contains at most the variable
% $X$'' any more, as equation variable doesn't appear in $E$ at
% all. Under these simplifications, the formal proof of Milner's unique
% solution of equations theorem for $\sim$ is a literal mapping for informal
% proofs based on bisimulation upto $\sim$, induction and case analysis
% of weakly guarded contexts. Below is the formal version of Lemma 4.13
% and Proposition 4.14 in Milner's book:

% \begin{alltt}
% STRONG_UNIQUE_SOLUTION:
% \end{alltt}

% \begin{alltt}
% WEAK_UNIQUE_SOLUTION:
% \end{alltt}

% \begin{alltt}
% OBS_UNIQUE_SOLUTION:
% \end{alltt}

