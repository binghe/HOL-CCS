%%%% -*- Mode: LaTeX -*-
%%
%% This is the draft of the 2nd part of EXPRESS/SOS 2018 paper, co-authored by
%% Prof. Davide Sangiorgi and Chun Tian.

\subsection{Context, guardness and (pre)congruence}

% We need to find a suitable formal definition of 
% context. There're multiple ways. Here 

We have chosen to use $\lambda$-expressions (typed
$CCS\rightarrow CCS$) to represent  (multi-hole) 
contexts. The definition is inductive:
\begin{alltt}
\HOLTokenTurnstile{} \HOLConst{CONTEXT} (\HOLTokenLambda{}\HOLBoundVar{t}. \HOLBoundVar{t}) \HOLSymConst{\HOLTokenConj{}} (\HOLSymConst{\HOLTokenForall{}}\HOLBoundVar{p}. \HOLConst{CONTEXT} (\HOLTokenLambda{}\HOLBoundVar{t}. \HOLBoundVar{p})) \HOLSymConst{\HOLTokenConj{}}
   (\HOLSymConst{\HOLTokenForall{}}\HOLBoundVar{a} \HOLBoundVar{e}. \HOLConst{CONTEXT} \HOLBoundVar{e} \HOLSymConst{\HOLTokenImp{}} \HOLConst{CONTEXT} (\HOLTokenLambda{}\HOLBoundVar{t}. \HOLBoundVar{a}\HOLSymConst{..}\HOLBoundVar{e} \HOLBoundVar{t})) \HOLSymConst{\HOLTokenConj{}}
   (\HOLSymConst{\HOLTokenForall{}}\HOLBoundVar{e\sb{\mathrm{1}}} \HOLBoundVar{e\sb{\mathrm{2}}}. \HOLConst{CONTEXT} \HOLBoundVar{e\sb{\mathrm{1}}} \HOLSymConst{\HOLTokenConj{}} \HOLConst{CONTEXT} \HOLBoundVar{e\sb{\mathrm{2}}} \HOLSymConst{\HOLTokenImp{}} \HOLConst{CONTEXT} (\HOLTokenLambda{}\HOLBoundVar{t}. \HOLBoundVar{e\sb{\mathrm{1}}} \HOLBoundVar{t} \HOLSymConst{+} \HOLBoundVar{e\sb{\mathrm{2}}} \HOLBoundVar{t})) \HOLSymConst{\HOLTokenConj{}}
   (\HOLSymConst{\HOLTokenForall{}}\HOLBoundVar{e\sb{\mathrm{1}}} \HOLBoundVar{e\sb{\mathrm{2}}}. \HOLConst{CONTEXT} \HOLBoundVar{e\sb{\mathrm{1}}} \HOLSymConst{\HOLTokenConj{}} \HOLConst{CONTEXT} \HOLBoundVar{e\sb{\mathrm{2}}} \HOLSymConst{\HOLTokenImp{}} \HOLConst{CONTEXT} (\HOLTokenLambda{}\HOLBoundVar{t}. \HOLBoundVar{e\sb{\mathrm{1}}} \HOLBoundVar{t} \HOLSymConst{\ensuremath{\parallel}} \HOLBoundVar{e\sb{\mathrm{2}}} \HOLBoundVar{t})) \HOLSymConst{\HOLTokenConj{}}
   (\HOLSymConst{\HOLTokenForall{}}\HOLBoundVar{L} \HOLBoundVar{e}. \HOLConst{CONTEXT} \HOLBoundVar{e} \HOLSymConst{\HOLTokenImp{}} \HOLConst{CONTEXT} (\HOLTokenLambda{}\HOLBoundVar{t}. \HOLSymConst{\ensuremath{\nu}} \HOLBoundVar{L} (\HOLBoundVar{e} \HOLBoundVar{t}))) \HOLSymConst{\HOLTokenConj{}}
   \HOLSymConst{\HOLTokenForall{}}\HOLBoundVar{rf} \HOLBoundVar{e}. \HOLConst{CONTEXT} \HOLBoundVar{e} \HOLSymConst{\HOLTokenImp{}} \HOLConst{CONTEXT} (\HOLTokenLambda{}\HOLBoundVar{t}. \HOLConst{relab} (\HOLBoundVar{e} \HOLBoundVar{t}) \HOLBoundVar{rf})\hfill{[CONTEXT_rules]}
\end{alltt}

Under above definition, we can formally define the concept of
``precongruence'' and ``congruence'' in the following ways:
\begin{alltt}
\HOLTokenTurnstile{} \HOLConst{precongruence} \HOLFreeVar{R} \HOLSymConst{\HOLTokenEquiv{}}
   \HOLConst{PreOrder} \HOLFreeVar{R} \HOLSymConst{\HOLTokenConj{}}
   \HOLSymConst{\HOLTokenForall{}}\HOLBoundVar{x} \HOLBoundVar{y} \HOLBoundVar{ctx}. \HOLConst{CONTEXT} \HOLBoundVar{ctx} \HOLSymConst{\HOLTokenImp{}} \HOLFreeVar{R} \HOLBoundVar{x} \HOLBoundVar{y} \HOLSymConst{\HOLTokenImp{}} \HOLFreeVar{R} (\HOLBoundVar{ctx} \HOLBoundVar{x}) (\HOLBoundVar{ctx} \HOLBoundVar{y})
\HOLTokenTurnstile{} \HOLConst{congruence} \HOLFreeVar{R} \HOLSymConst{\HOLTokenEquiv{}}
   \HOLConst{equivalence} \HOLFreeVar{R} \HOLSymConst{\HOLTokenConj{}}
   \HOLSymConst{\HOLTokenForall{}}\HOLBoundVar{x} \HOLBoundVar{y} \HOLBoundVar{ctx}. \HOLConst{CONTEXT} \HOLBoundVar{ctx} \HOLSymConst{\HOLTokenImp{}} \HOLFreeVar{R} \HOLBoundVar{x} \HOLBoundVar{y} \HOLSymConst{\HOLTokenImp{}} \HOLFreeVar{R} (\HOLBoundVar{ctx} \HOLBoundVar{x}) (\HOLBoundVar{ctx} \HOLBoundVar{y})
\end{alltt}

A \emph{weakly guarded} context is a context in which each hole is
underneath a  prefix (where \texttt{WG} stands for weakly guarded):
\begin{alltt}
\HOLTokenTurnstile{} (\HOLSymConst{\HOLTokenForall{}}\HOLBoundVar{p}. \HOLConst{WG} (\HOLTokenLambda{}\HOLBoundVar{t}. \HOLBoundVar{p})) \HOLSymConst{\HOLTokenConj{}} (\HOLSymConst{\HOLTokenForall{}}\HOLBoundVar{a} \HOLBoundVar{e}. \HOLConst{CONTEXT} \HOLBoundVar{e} \HOLSymConst{\HOLTokenImp{}} \HOLConst{WG} (\HOLTokenLambda{}\HOLBoundVar{t}. \HOLBoundVar{a}\HOLSymConst{..}\HOLBoundVar{e} \HOLBoundVar{t})) \HOLSymConst{\HOLTokenConj{}}
   (\HOLSymConst{\HOLTokenForall{}}\HOLBoundVar{e\sb{\mathrm{1}}} \HOLBoundVar{e\sb{\mathrm{2}}}. \HOLConst{WG} \HOLBoundVar{e\sb{\mathrm{1}}} \HOLSymConst{\HOLTokenConj{}} \HOLConst{WG} \HOLBoundVar{e\sb{\mathrm{2}}} \HOLSymConst{\HOLTokenImp{}} \HOLConst{WG} (\HOLTokenLambda{}\HOLBoundVar{t}. \HOLBoundVar{e\sb{\mathrm{1}}} \HOLBoundVar{t} \HOLSymConst{+} \HOLBoundVar{e\sb{\mathrm{2}}} \HOLBoundVar{t})) \HOLSymConst{\HOLTokenConj{}}
   (\HOLSymConst{\HOLTokenForall{}}\HOLBoundVar{e\sb{\mathrm{1}}} \HOLBoundVar{e\sb{\mathrm{2}}}. \HOLConst{WG} \HOLBoundVar{e\sb{\mathrm{1}}} \HOLSymConst{\HOLTokenConj{}} \HOLConst{WG} \HOLBoundVar{e\sb{\mathrm{2}}} \HOLSymConst{\HOLTokenImp{}} \HOLConst{WG} (\HOLTokenLambda{}\HOLBoundVar{t}. \HOLBoundVar{e\sb{\mathrm{1}}} \HOLBoundVar{t} \HOLSymConst{\ensuremath{\parallel}} \HOLBoundVar{e\sb{\mathrm{2}}} \HOLBoundVar{t})) \HOLSymConst{\HOLTokenConj{}}
   (\HOLSymConst{\HOLTokenForall{}}\HOLBoundVar{L} \HOLBoundVar{e}. \HOLConst{WG} \HOLBoundVar{e} \HOLSymConst{\HOLTokenImp{}} \HOLConst{WG} (\HOLTokenLambda{}\HOLBoundVar{t}. \HOLSymConst{\ensuremath{\nu}} \HOLBoundVar{L} (\HOLBoundVar{e} \HOLBoundVar{t}))) \HOLSymConst{\HOLTokenConj{}}
   \HOLSymConst{\HOLTokenForall{}}\HOLBoundVar{rf} \HOLBoundVar{e}. \HOLConst{WG} \HOLBoundVar{e} \HOLSymConst{\HOLTokenImp{}} \HOLConst{WG} (\HOLTokenLambda{}\HOLBoundVar{t}. \HOLConst{relab} (\HOLBoundVar{e} \HOLBoundVar{t}) \HOLBoundVar{rf})\hfill{[WG_rules]}
\end{alltt}
% (Notice the differences between a weak guarded context and a normal
% one: $\lambda t. t$ is not weakly guarded as the variable is directly
% exposed without any prefixed action. And $\lambda t. a.e[t]$ is weakly
% guarded as long as $e[\cdot]$ is a context, not necessary weakly guarded.)
A \emph{strongly guarded} (\texttt{SG}) context is a context in which
 each hole is underneath a visible prefix:
\begin{alltt}
\HOLTokenTurnstile{} (\HOLSymConst{\HOLTokenForall{}}\HOLBoundVar{p}. \HOLConst{SG} (\HOLTokenLambda{}\HOLBoundVar{t}. \HOLBoundVar{p})) \HOLSymConst{\HOLTokenConj{}}
   (\HOLSymConst{\HOLTokenForall{}}\HOLBoundVar{l} \HOLBoundVar{e}. \HOLConst{CONTEXT} \HOLBoundVar{e} \HOLSymConst{\HOLTokenImp{}} \HOLConst{SG} (\HOLTokenLambda{}\HOLBoundVar{t}. \HOLConst{label} \HOLBoundVar{l}\HOLSymConst{..}\HOLBoundVar{e} \HOLBoundVar{t})) \HOLSymConst{\HOLTokenConj{}}
   (\HOLSymConst{\HOLTokenForall{}}\HOLBoundVar{a} \HOLBoundVar{e}. \HOLConst{SG} \HOLBoundVar{e} \HOLSymConst{\HOLTokenImp{}} \HOLConst{SG} (\HOLTokenLambda{}\HOLBoundVar{t}. \HOLBoundVar{a}\HOLSymConst{..}\HOLBoundVar{e} \HOLBoundVar{t})) \HOLSymConst{\HOLTokenConj{}}
   (\HOLSymConst{\HOLTokenForall{}}\HOLBoundVar{e\sb{\mathrm{1}}} \HOLBoundVar{e\sb{\mathrm{2}}}. \HOLConst{SG} \HOLBoundVar{e\sb{\mathrm{1}}} \HOLSymConst{\HOLTokenConj{}} \HOLConst{SG} \HOLBoundVar{e\sb{\mathrm{2}}} \HOLSymConst{\HOLTokenImp{}} \HOLConst{SG} (\HOLTokenLambda{}\HOLBoundVar{t}. \HOLBoundVar{e\sb{\mathrm{1}}} \HOLBoundVar{t} \HOLSymConst{+} \HOLBoundVar{e\sb{\mathrm{2}}} \HOLBoundVar{t})) \HOLSymConst{\HOLTokenConj{}}
   (\HOLSymConst{\HOLTokenForall{}}\HOLBoundVar{e\sb{\mathrm{1}}} \HOLBoundVar{e\sb{\mathrm{2}}}. \HOLConst{SG} \HOLBoundVar{e\sb{\mathrm{1}}} \HOLSymConst{\HOLTokenConj{}} \HOLConst{SG} \HOLBoundVar{e\sb{\mathrm{2}}} \HOLSymConst{\HOLTokenImp{}} \HOLConst{SG} (\HOLTokenLambda{}\HOLBoundVar{t}. \HOLBoundVar{e\sb{\mathrm{1}}} \HOLBoundVar{t} \HOLSymConst{\ensuremath{\parallel}} \HOLBoundVar{e\sb{\mathrm{2}}} \HOLBoundVar{t})) \HOLSymConst{\HOLTokenConj{}}
   (\HOLSymConst{\HOLTokenForall{}}\HOLBoundVar{L} \HOLBoundVar{e}. \HOLConst{SG} \HOLBoundVar{e} \HOLSymConst{\HOLTokenImp{}} \HOLConst{SG} (\HOLTokenLambda{}\HOLBoundVar{t}. \HOLSymConst{\ensuremath{\nu}} \HOLBoundVar{L} (\HOLBoundVar{e} \HOLBoundVar{t}))) \HOLSymConst{\HOLTokenConj{}}
   \HOLSymConst{\HOLTokenForall{}}\HOLBoundVar{rf} \HOLBoundVar{e}. \HOLConst{SG} \HOLBoundVar{e} \HOLSymConst{\HOLTokenImp{}} \HOLConst{SG} (\HOLTokenLambda{}\HOLBoundVar{t}. \HOLConst{relab} (\HOLBoundVar{e} \HOLBoundVar{t}) \HOLBoundVar{rf})\hfill{[SG_rules]}
\end{alltt}
And a \emph{sequential} (\texttt{SEQ}) context is a context in which each hole is under prefix or sums:
 \begin{alltt}
\HOLTokenTurnstile{} \HOLConst{SEQ} (\HOLTokenLambda{}\HOLBoundVar{t}. \HOLBoundVar{t}) \HOLSymConst{\HOLTokenConj{}} (\HOLSymConst{\HOLTokenForall{}}\HOLBoundVar{p}. \HOLConst{SEQ} (\HOLTokenLambda{}\HOLBoundVar{t}. \HOLBoundVar{p})) \HOLSymConst{\HOLTokenConj{}}
   (\HOLSymConst{\HOLTokenForall{}}\HOLBoundVar{a} \HOLBoundVar{e}. \HOLConst{SEQ} \HOLBoundVar{e} \HOLSymConst{\HOLTokenImp{}} \HOLConst{SEQ} (\HOLTokenLambda{}\HOLBoundVar{t}. \HOLBoundVar{a}\HOLSymConst{..}\HOLBoundVar{e} \HOLBoundVar{t})) \HOLSymConst{\HOLTokenConj{}}
   \HOLSymConst{\HOLTokenForall{}}\HOLBoundVar{e\sb{\mathrm{1}}} \HOLBoundVar{e\sb{\mathrm{2}}}. \HOLConst{SEQ} \HOLBoundVar{e\sb{\mathrm{1}}} \HOLSymConst{\HOLTokenConj{}} \HOLConst{SEQ} \HOLBoundVar{e\sb{\mathrm{2}}} \HOLSymConst{\HOLTokenImp{}} \HOLConst{SEQ} (\HOLTokenLambda{}\HOLBoundVar{t}. \HOLBoundVar{e\sb{\mathrm{1}}} \HOLBoundVar{t} \HOLSymConst{+} \HOLBoundVar{e\sb{\mathrm{2}}} \HOLBoundVar{t})\hfill{[SEQ_rules]}
\end{alltt}
In the same manner, we can also define all
variants of above concepts without \emph{direct} sums (\texttt{GCONTEXT}, \texttt{WGS} and
\texttt{GSEQ}). A large amount of lemmas must be proved for their relationships and properties of compositions. These proofs all have long (and trivial) steps due to multiple levels of
inductions on their structures.%  For example, one such 

% lemma says, for any  context $E$ which is both strongly guarded
% and sequential (no direct sums) and another sequential context $H$ (no
% direct sums), the composition $H \circ E$ is still both strongly
% guarded and sequential (no direct sums):
% \begin{alltt}

% \end{alltt}

% \subsection{Milner's ``unique solution of equations'' theorems}

% Once the representation issue of CCS equations is resolved, the actual
% proofs of Milner's unique solution of equations theorems is not very
% interesting from the view of theorem proving, although it's a 
% precise proof engineering work producing quite long proofs.
% Since we have chosen to use  context to represent
% single-variable equations, an equation like $P \sim E\{P/X\}$ now
% becomes $P \sim E[P]$, in which $E$ is a (multi-hole) 
% context, and there's no need to say it ``contains at most the variable
% $X$'' any more, as equation variable doesn't appear in $E$ at
% all. Under these simplifications, the formal proof of Milner's unique
% solution of equations theorem for $\sim$ is a literal mapping for informal
% proofs based on bisimulation upto $\sim$, induction and case analysis
% of weakly guarded contexts. Below is the formal version of Lemma 4.13
% and Proposition 4.14 in Milner's book:

% \begin{alltt}
% STRONG_UNIQUE_SOLUTION:
% \end{alltt}

% \begin{alltt}
% WEAK_UNIQUE_SOLUTION:
% \end{alltt}

% \begin{alltt}
% OBS_UNIQUE_SOLUTION:
% \end{alltt}
