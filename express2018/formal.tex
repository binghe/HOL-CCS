\section{Formalisation}
\label{s:for}
We highlight here a formalisation of CCS
in the HOL theorem
prover (HOL4) \cite{slind2008brief},
including the new concepts and theorems proposed in the first half of
this paper.
%  The main purpose is to convince the readers that, there's no flaw
%  in the informal proofs. 
The whole formalisation can be found 
in \cite{Tian:2017wrba}  or in 
 in the official example folder of HOL4 source
code\footnote{\url{https://github.com/HOL-Theorem-Prover/HOL}}. The
work consists of about 20,000 lines of proof scripts in Standard ML
(and that follows the pioneering work by Nesi \cite{Nesi:1992ve}, in
1992-1995). 

Higher Order Logic (or ``HOL Logic'') \cite{hollogic} stands for simple-typed $\lambda$-calculus plus Hibert
choice operator, axiom of infinity, and rank-1 polymorphism (type
variables). HOL4 implements the original HOL logics, 
in contrast with 
 other theorem provers of the HOL family (e.g. Isabelle/HOL) that have
made extensions.
%  (they made the formal language more powerful,
% but they also bring the possibilities that the entire logic becomes
% inconsistent). 
Indeed the HOL Logic has considerable simpler logic
foundations than most other theorem provers. %, e.g. Coq. 
As a result,
formal theories in HOL can be easily migrated (sometimes even
automatically) into other theorem provers.

HOL4 is written in Standard ML (ML for abbreviation), which plays three roles here:
\begin{enumerate}
\item It serves as the underlying implementation language for the core HOL engine;
\item it is used to implement tactics (and tacticals) for writing proofs;
\item it is used as the command language of the HOL interactive shell.
\end{enumerate}
Besides, HOL4 users usually write complex automatic
verification tools (in the same language) by calling HOL's interactive theorem proving facilities.
\iflong
The HOL logic language, on the other
 side, is more close to Classic ML, in which the early HOL systems were built.
\DS{removed the sentence because  it did not
  seem important} 
\fi

% \finish{below: to be reformulated (by Davide) later} 
% We will not repeat all definitions and theorems of CCS again by their
% formalized versions. Instead, we just focus on several highlights in
% this work, i.e.
% \begin{enumerate}
% \item The use of HOL's new coinductive relation package for defining bisimilarity;
% \item The formalisation of context by $\lambda$-expression and the theory of
%   (pre)congruence for CCS;
% \item The definition and uses of trace in the proof of unique solution of
%   contractions theorem;
% \end{enumerate}\finish{TODO: need updates}

In this
formalisation we consider only single equations/contractions. 
This setup considerably simplified the required proof steps while still aligned with
their informal proofs.
%  (on 
% the extension  to multiple equations/contractions
% see also  Section~\ref{s:concl}).

% \subsection{Simplifications and Limitations}

% We have only formally verified simplified forms of the unique
% solution theorems introduced in the first part. This is to not only
% save the formalisation efforts, but also to make us focusing on the most
% important part of the proof. The most important simplification is
% that, we have only proved all unique solution theorems with only single-variable equations
% (contractions).

% But such a simplification doesn't hurt the rigorousness of our entire
% work, as once the single-equation version of the theorem is
% proven, the general case is just a routine
% adaptation for paper proofs.
% It's not easy to formalize multi-variable CCS equations with
% introducing a large number of definitions and intermediate
% results. It's easy, however, to directly use the semantics context to
% represent single-variable equations without the need of formalizing
% the concept of ``equation'' at all.

% There're two major limitations in this work. One is due to HOL, we only support Finitary CCS, i.e.
% no infinite sum nor infinite parallel composition.
% The other limitation is, equation variables (or holes in semantics contexts)
% cannot appear inside recursion operator closure. However, these limitations do not
% cause any difference in the proof of main results represented in this paper.
