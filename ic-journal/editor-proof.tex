%BeginFileInfo
%%Publisher=ESCH
%%Project=YINCO
%%Manuscript=YINCO104606
%%MS position=
%%Stage=204
%%TID=lgiriuniene
%%Pages=4,32
%%Format=2019
%%Distribution=vtex
%%Destination=DVI
%%PDF type=
%%DVI.Maker=luadvi
%%PS.Maker=luaps
%%Spelled=Dictionary: British, Computer: 1GSRED593, 2020.07.30 10:52
%%History1=2020.07.24 12:25
%EndFileInfo
%
%Spelling_date
% Opcijos: [secthm,seceqn,secfloat,xxtheorem,noappeqn]
%%
\RequirePackage{styledata}
\documentclass[GCNS]{yincog}
\usepackage[left]{vmkcol}\usepackage{multirow}
%\usepackage[hidebg]{tmultirow}
\usepackage{tabcols} %%% [debug]
\usepackage[v5.6.0]{jadtd}
\usepackage{mathrsfs}
\usepackage{stmaryrd}
%\usepackage{holindex}
\usepackage{listings}
%\usepackage{soul}
\usepackage{xyfix}
\usepackage[all,dvips]{xy}%,dvips
\usepackage{alltt}
\usepackage{proof}
\usepackage{holtenbasic}
%\usepackage{DSarrow}
%%\usepackage[show]{elscon}
\usepackage{xspace}
\PROOF
%\preCRC
%\psdraft
%\psprintphoto
\articlenumber{104606}% Updated by PTS2LaTeX.exe, 24.07.2020 12:25
\aid{104606}% Updated by PTS2LaTeX.exe, 24.07.2020 12:25
\PII{S0890-5401(20)30094-8}% Updated by PTS2LaTeX.exe, 24.07.2020 12:28
\doi{10.1016/j.ic.2020.104606}% Updated by PTS2LaTeX.exe, 24.07.2020 12:28
\docsubty{FLA}
\volume{00}
%\issue{00}
%\supplement{}
\pubyear{2020}
%\pagenumbering{roman}%Roman,alph,Alph
\firstpage{1}
\lastpage{32}
\TID{Loreta}
%
\crossmark{0}% Updated by PTS2LaTeX.exe, 24.07.2020 12:25
%
% APIBREZIMAI:
%
\startlocaldefs
\begin{nonSGML}
\DeclareRobustCommand\setData[4][]{%
  \expandafter\gdef\csname#2:\detokenize{#3}\endcsname{#4}%
  \@for\curr@option:={#1}\do{%
    \expandafter\split@@eq\curr@option\split@@end
    \expandafter\xdef\csname#2:\detokenize{#3}.\split@@eq@one\endcsname{\split@@eq@two}%
    }%
  }
\end{nonSGML}
\renewcommand{\ttdefault}{cmtt}
\begin{nonSGML}
\renewcommand{\HOLinline}[1]{\mbox{\textup{\texttt{#1}}}}
\renewcommand{\HOLTokenDoublePlus}{\ensuremath{\mathbin{+\mkern -10mu+}}}
\end{nonSGML}
\renewcommand{\HOLConst}[1]{\texttt{#1}}
\renewcommand{\HOLNumLit}[1]{\ensuremath{#1}}
\renewcommand{\HOLStringLit}[1]{\textrm{``#1''}}
\renewcommand{\HOLTyOp}[1]{\texttt{#1}}
\renewcommand{\HOLBoundVar}[1]{\ensuremath{\mathit{#1}}}
\renewcommand{\HOLFreeVar}[1]{\ensuremath{\mathit{#1}}}
\renewcommand{\HOLKeyword}[1]{{\textbf{\textsf{#1}}}}
\renewcommand{\HOLSymConst}[1]{#1}
\renewcommand{\HOLTokenConj}{\ensuremath{\wedge}}
\renewcommand{\HOLTokenEmpty}{\ensuremath{\emptyset}}

\renewcommand{\HOLTokenIn}{\ensuremath{\in}}
\renewcommand{\HOLTokenLeftbrace}{\ensuremath{\left \{\right .}}
\renewcommand{\HOLTokenRightbrace}{\ensuremath{\left .\right \}}}
\renewcommand{\HOLTokenLeq}{\ensuremath{\leq}}
\renewcommand{\HOLTokenLt}{\ensuremath{<}}
\renewcommand{\HOLTokenMapto}{\ensuremath{\mapsto}}
\renewcommand{\HOLTokenNeg}{\ensuremath{\neg}}
\renewcommand{\HOLTokenNotEqual}{\ensuremath{\neq}}
\renewcommand{\HOLTokenRCompose}{\ensuremath{\circ _r}}
\renewcommand{\HOLTokenRSubset}{\ensuremath{\subseteq _r}}
\renewcommand{\HOLTokenSubset}{\ensuremath{\subseteq}}
\renewcommand{\HOLTokenUnion}{\ensuremath{\cup}}


\begin{nonSGML}
\lstset{tabsize=8,language=ML,basicstyle=\small \ttfamily \bfseries,
keywordstyle=\ttfamily,
stringstyle=\color{red}\ttfamily,
commentstyle=\color{green}\ttfamily,
morecomment=[l][\color{magenta}]{\#}}

\def \rightarrowfill{$\m@th\mathord{\smash-}\mkern-6mu%
  \cleaders\hbox{$\mkern-2mu\mathord{\smash-}\mkern-2mu$}\hfill
  \mkern-6mu\mathord\rightarrow$}

\def \rightarrowfillWEAK{$\m@th\mathord{\smash=}\mkern-6mu%
  \cleaders\hbox{$\mkern-2mu\mathord{\smash=}\mkern-2mu$}\hfill
  \mkern-6mu\mathord\Rightarrow$}
%\renewcommand{\HOLTokenDefEquality}{\ensuremath{\mathrel{\overset{\makebox [0pt]{\mbox{\normalfont \tiny \textsf{def}}}}{=}}}}
\renewcommand{\HOLTokenDefEquality}{\ensuremath{\mathrel{\overset{\makebox [0pt]{\text{\tiny \textsf{def}}}}{=}}}}


%\renewcommand{\HOLTokenEquiv}{\ensuremath{\iff}}

\renewcommand{\HOLTokenEquiv}{\ensuremath{\Longleftrightarrow}}

\end{nonSGML}
\renewcommand{\HOLTokenExists}{\ensuremath{\exists \,}}
\renewcommand{\HOLTokenForall}{\ensuremath{\forall \,}}
\renewcommand{\HOLTokenLambda}{\ensuremath{\lambda \,}}
\renewcommand{\HOLTokenNotIn}{\ensuremath{\notin}}
\renewcommand{\HOLTokenSupStar}{\ensuremath{{}^{*}}}
\renewcommand{\HOLTokenTurnstile}{\ensuremath{\:\:\vdash}}

\renewcommand{\HOLTokenDisj}{\ensuremath{\vee}}
\renewcommand{\HOLTokenBar}{\texttt{|}}
\renewcommand{\HOLTokenBigUnion}{\ensuremath{\bigcup}}
\renewcommand{\HOLTokenCompose}{\ensuremath{\circ}}

\theoremstyle{remark}
\newtheorem{definition}{Definition}[section]
\theoremstyle{theorem}

\newtheorem{lemma}[definition]{Lemma}
\newtheorem{theorem}[definition]{Theorem}
\newtheorem{corollary}[definition]{Corollary}
\newtheorem{proposition}[definition]{Proposition}
\theoremstyle{remark}
\newtheorem{remark}[definition]{Remark}
\newtheorem{example}[definition]{Example}
\newcommand{\HOLTokenStrongEQ}{$\sim$}
\newcommand{\HOLTokenWeakEQ}{$\approx$}
\newcommand{\HOLTokenObsCongr}{$\approx^{\mathrm{c}}\!$}
\newcommand{\HOLTokenEPS}{$\overset{\epsilon}{\Longrightarrow}$}
\newcommand{\HOLTokenTransBegin}{$-$}
\newcommand{\HOLTokenTransEnd}{$\rightarrow$\xspace}
\newcommand{\HOLTokenWeakTransBegin}{$=$}
\newcommand{\HOLTokenWeakTransEnd}{$\Rightarrow$\xspace}
\newcommand{\HOLTokenExpands}{$\succeq_{\mathrm{e}}\!$}
\newcommand{\HOLTokenContracts}{$\succeq_{\mathrm{bis}}\!$}
\newcommand{\HOLTokenObsContracts}{$\succeq^{\mathrm{c}}_{\mathrm{bis}}\!$}
\newcommand{\HOLTokenInputAct}{$\mathrm{in}$}
\newcommand{\HOLTokenOutputAct}{$\overline{\mathrm{out}}$}

\renewcommand{\HOLTokenImp}{\ensuremath{\Longrightarrow}}

\newcommand{\univariate}{univariable\xspace}
\newcommand{\multivariate}{multivariable\xspace}
\newcommand{\Multivariate}{Multivariable\xspace}

\def\fvvtex#1{\rmsf{fv}(#1)}
\def\bvvtex#1{\rmsf{bv}(#1)}

\def\nil{{\boldsymbol 0}}
\def\res#1{{\boldsymbol \nu} #1\:}
\mathcode `\!="4021 \mathcode `\.="602E \mathcode `\|="326A

\newcommand{\outC}[1]{\overline{#1}}


\begin{nonSGML}
\def\sub#1#2{\{\raisebox{.5ex}{\small$#1$}\! / \!\mbox{\small$#2$}\}}
\end{nonSGML}
\newcommand{\arr}[1]{\mathrel{\stackrel{{\;\;#1\;\;}}{\mbox{\rightarrowfill}}}}

\newcommand{\Arr}[1]{\mathrel{\stackrel{{\;\;#1\;\;}}{\mbox{\rightarrowfillWEAK}}}}

\newcommand{\arcap}[1]{\mathrel{\stackrel{{\;\; {\widehat{#1}} \;\;}}{\mbox{\rightarrowfill}}}}
\newcommand{\Arcap}[1]{\mathrel{\stackrel{{\;\;{\widehat{#1}}\;\;}}{\mbox{\rightarrowfillWEAK}}}}

\newcommand{\ctvtex}[1]{ C \brac{#1} }
\newcommand{\qct}{ C }
\newcommand{\brac}[1]{[#1] }

\newcommand{\rmsf}[1]{{\mathsf{{#1}}}}

\def\Rvtex{\mathcal{R}}
\def\RRvtex{\mathrel{\mathcal{R}}}
\def\Svtex{\mathcal{S}}

\def\midd{\; \; \mbox{\Large{$\mid$}}\;\;}

\def\stvtex{\; \mid \;}
\def\DSdefi{\stackrel{\text{def}}{=}}

\renewcommand{\tilde}{\widetilde}

\newcommand{\recu}[2]{\mathtt{rec}\: #1 . #2}

\newcommand{\rapprox}{\mathrel{\approx^{\mathrm{c}}}}

\newcommand{\hbvtex}{\hspace{0.5cm}}

\newcommand{\wbvtex}{\approx}
\newcommand{\contr}{\mathrel{\succeq_{\mathrm{bis}}}}
\newcommand{\expa}{\mathrel{\succeq_{\mathrm{e}}}}

\newcommand{\mcontrBIS}{\mathrel{\succeq_{\mathrm{bis}}}}
\newcommand{\mexpaBIS}{\mathrel{\preceq_{\mathrm{bis}}}}

\newcommand{\rcontr}{\mathrel{\succeq^{\mathrm{c}}_{\mathrm{bis}}}}

\newcommand{\til}{\tilde}

\newcommand{\ctp}[1]{ C' \brac{#1} }
\newcommand{\qctp}{ C' }
\newcommand{\ctpp}[1]{ C'' \brac{#1} }

\newcommand{\qctpp}{ C'' }

\newcommand{\hkvtex}{\hspace{0.2cm}}

\newcommand{\Tvtex}{\mathcal{T}}
\newcommand\fun{{\to}}
\newcommand{\tyvtex}[1]{\textsl{#1}}
\newcommand\conj{\ \wedge\ }
\newcommand\imp{ \Rightarrow }
\newcommand\turn{\ \vdash\ }
\newcommand\hilbert{\varepsilon}
\newcommand{\uquant}[1]{\forall #1.\ }
\newcommand{\equant}[1]{\exists #1.\ }
\newcommand{\lquant}[1]{\lambda #1.\ }
\newcommand{\con}[1]{\mathrm{#1}}

\newcommand\ind{\tyvtex{ind}}

\renewcommand{\Tvtex}{\con{T}}
\newcommand\Fvtex{\con{F}}
\newcommand\OneOne{\con{One\_One}}
\newcommand\Onto{\con{Onto}}
\newcommand\TyDef{\con{Type\_Definition}}

\newlength{\ttX}
\settowidth{\ttX}{\texttt{X}}
\newcommand{\tyvar}{\setlength{\unitlength}{\ttX}\begin{picture}(1,6)
\put(.5,0){\makebox(0,0)[b]{\footnotesize type variables}}
\put(0,1.5){\vector(0,1){4.5}}
\end{picture}}
\newcommand{\tyatom}{\setlength{\unitlength}{\ttX}\begin{picture}(1,6)
\put(.5,2.3){\makebox(0,0)[b]{\footnotesize atomic types}}
\put(.5,3.3){\vector(0,1){2.6}}
\end{picture}}
\newcommand{\funty}{\setlength{\unitlength}{\ttX}\begin{picture}(1,6)
\put(.5,1.5){\makebox(0,0)[b]{\footnotesize function types}}
\put(.5,0){\makebox(0,0)[b]{\footnotesize (domain $\sigma_1$, codomain $\sigma_2$)}}
\put(1,2.5){\vector(0,1){3.5}}
\end{picture}}
\newcommand{\cmpty}{\setlength{\unitlength}{\ttX}\begin{picture}(1,6)
\put(2,3.3){\makebox(0,0)[b]{\footnotesize compound types}}
\put(1.9,4.5){\vector(0,1){1.5}}
\end{picture}}

\settowidth{\ttX}{\texttt{X}}
\newcommand{\var}{\setlength{\unitlength}{\ttX}\begin{picture}(1,6)
\put(.5,0){\makebox(0,0)[b]{\footnotesize variables}}
\put(0,1.5){\vector(0,1){4.5}}
\end{picture}}
\newcommand{\const}{\setlength{\unitlength}{\ttX}\begin{picture}(1,6)
\put(.5,2.3){\makebox(0,0)[b]{\footnotesize constants}}
\put(.5,3.5){\vector(0,1){2.4}}
\end{picture}}
\newcommand{\app}{\setlength{\unitlength}{\ttX}\begin{picture}(1,6)
\put(.5,1.5){\makebox(0,0)[b]{\footnotesize function applications}}
\put(.5,0){\makebox(0,0)[b]{\footnotesize (function $t$, argument $t'$)}}
\put(0.5,2.5){\vector(0,1){3.5}}
\end{picture}}
\newcommand{\abs}{\setlength{\unitlength}{\ttX}\begin{picture}(1,6)
\put(1,3.3){\makebox(0,0)[b]{\footnotesize $\lambda$-abstractions}}
\put(0.7,4.5){\vector(0,1){1.5}}
\end{picture}}

\endlocaldefs
%
\begin{copyrightinfo}[type=standard]% Updated by PTS2LaTeX.exe, 24.07.2020 12:25
\cpcopyrightNotice{\textCopyright\ 2020 Elsevier Inc. All rights reserved.}% Updated by PTS2LaTeX.exe, 24.07.2020 12:28
%\cplicenseLine{}
%\oauserLicense{}
\cpcopyrightYear{2020}% Updated by PTS2LaTeX.exe, 24.07.2020 12:25
\cpcopyrightHolderDisplayName{Elsevier Inc.}% Updated by PTS2LaTeX.exe, 24.07.2020 12:28
\cpxmlCopyrightType{full-transfer}% Updated by PTS2LaTeX.exe, 24.07.2020 12:28
\copyrt{2020}
\CopyrightStatus{001}
\end{copyrightinfo}
%
\begin{document}
\begin{frontmatter}
%% REFERSTO
%\dochead{}
\title{Unique solutions of contractions, CCS, and their HOL
formalisation\thanksref{ATL1}}
% \title{}
%\artthanks[]{}
\artthanks[ATL1]{This is an extended and refined
version of the paper with the same title published in EXPRESS/SOS 2018~\cite{EPTCS276.10}.}
%\begin{aug}
%\runauthor{}
%\author[A]{\inits{}\fnm{} \snm{}}
%\corthanks[cor]{Corresponding author.}
%\address[A]{}
%\end{aug}
\begin{aug}
\author[a,b]{\inits{C.}\fnm{Chun}~\snm{Tian}\ead{chun.tian@unitn.it}\thanksref{THNKS2}},
\thanks[THNKS2]{Part of this work was carried out when the first
author was studying at the University of Bologna, Italy.}
\author[c,d]{\inits{D.}\fnm{Davide}~\snm{Sangiorgi}\ead{davide.sangiorgi@unibo.it}}
\address[a]{\orgname{University of Trento}, \cny{Italy}%
   \adrsource{University of Trento, Italy}}
\address[b]{\orgname{Fondazione Bruno Kessler},
   \cny{Italy}%
   \adrsource{Fondazione Bruno Kessler, Italy}}
\address[c]{\orgname{University of Bologna},
   \cny{Italy}%
   \adrsource{University of Bologna, Italy}}
\address[d]{\orgname{INRIA},
   \cny{France}%
   \adrsource{INRIA, France}}
\end{aug}
%
% HISTORY:
\received{\sday{22} \smonth{6} \syear{2020}}% Updated by PTS2LaTeX.exe, 24.07.2020 12:28
%\revised{\sday{} \smonth{} \syear{}}% Updated by PTS2LaTeX.exe, 24.07.2020 12:28
\accepted{\sday{29} \smonth{6} \syear{2020}}% Updated by PTS2LaTeX.exe, 24.07.2020 12:28
%\pubonline{\sday{} \smonth{} \syear{}}
%
%\dataset[]{}
%
%\dedicated{}
%
%\begin{abstract}    %%% Visada FLA tipui (<=150 zodziu)
%\end{abstract}
%
%\begin{abstract}[class=graphical]  %%% Jei yra
%\begin{figure} \includegraphics{<aid>fab} \end{figure} \abstext{}
%\end{abstract}
%
%\begin{abstract}[class=author-highlights,title=Highlights] %%% Jei yra
%\begin{itemize} %\item
%\end{itemize}
%\end{abstract}
\begin{abstract}
The unique solution of contractions is a proof technique for (weak) bisimilarity
that overcomes certain syntactic limitations of Milner's ``unique solution
of equations'' theorem. This paper presents an overview of a comprehensive
formalisation of Milner's Calculus of Communicating Systems (CCS) in the
HOL theorem prover (HOL4), with a focus towards the theory of unique solutions
of equations and contractions. The formalisation consists of about 24,000
lines (1MB) of code in total. Some refinements of the ``unique solution
of contractions'' theory itself are obtained. In particular we remove the
constraints on summation, which must be guarded, by moving from contraction
to \emph{rooted contraction}. We prove the ``unique solution of rooted contractions''
theorem and show that rooted contraction is the coarsest precongruence
contained in the contraction preorder.
\end{abstract}
%
% Opcijos: [class=KWD]
%\begin{keyword}             %%% Jei yra FLA tipui
%\kwd{}\kwd{}
%\end{keyword}
\begin{keyword}
\kwd{Process calculi}
\kwd{Theorem proving}
\kwd{Coinduction}
\kwd{Unique solution of equations}
\kwd{Congruence}
\end{keyword}
\end{frontmatter}
%spell_from    *************** Text entry area ******************%


%s1 #&#
\section{Introduction}
 \label{sec1}

A prominent proof method for bisimulation, put forward by Robin Milner
and widely used in his landmark CCS book~\cite{Mil89}, is the
\emph{unique solution of equations}, whereby two tuples of processes are
componentwise bisimilar if they are solutions of the same system of equations.
This method is important in verification techniques and tools based on
algebraic reasoning~\cite{BaeBOOK,theoryAndPractice,RosUnder10}.

Milner's unique-solution theorem for weak bisimilarity, however, has severe
syntactic limitations: the equations must be both \emph{guarded} and
\emph{sequential}. That is, the variables of the equations can only occur
underneath visible prefixes and summation. One way to overcome these limitations
is to replace the equations with special inequations called
\emph{contractions}~\cite{sangiorgi2015equations,sangiorgi2017equations}.
The contraction relation is a preorder that, roughly, places some efficiency
constraints on processes. The unique solution of contractions is defined
as with equations: any two solutions must be componentwise bisimilar. The
difference from equations is in the meaning of a solution: in the case
of contractions the solution is evaluated with respect to the contraction
preorder rather than bisimilarity. With contractions, most syntactic limitations
of the unique-solution theorem are eliminated. One constraint that still
remains in~\cite{sangiorgi2017equations} (where the issue is bypassed by
using a more restrictive CCS syntax) is the occurrences of arbitrary sums
(e.g.~$P + Q$) due to the non-substitutivity of the contraction preorder
in this case.

This paper presents a comprehensive formalisation of Milner's Calculus
of Communicating Systems (CCS) in the HOL theorem prover (HOL4), with a
focus towards the theory of unique solutions of equations and contractions.
Many results in Milner's CCS book~\cite{Mil89} are covered, since the unique-solution
theorems rely on a large number of fundamental results. Indeed the formalisation
encompasses all basic properties of strong, weak and rooted bisimilarities
(e.g.~the fixed-point and substitutivity properties), and their algebraic
laws. Further extensions include several versions of ``bisimulation up
to'' techniques, and properties of the expansion and contraction preorders.
Concerning rooted bisimilarity, the formalisation includes Hennessy's Lemma,
Deng's Lemma, and two theorems saying that rooted bisimilarity is the coarsest
(largest) congruence contained in weak bisimilarity ($\wbvtex $): one is
classical with the hypothesis that no process uses up all labels; the other
one is without such hypothesis, essentially formalising van Glabbeek's
proof~\cite{van2005characterisation}. With this respect, the work is also
an extensive experiment using the HOL theorem prover with its recent developments,
including its coinduction package.

This formalisation has offered us the possibility of further refining the
theory of unique solutions of contractions. In particular, existing results~\cite{sangiorgi2017equations}
have limitations on the body of the contractions due to the substitutivity
problems of weak bisimilarity and other behavioural relations with respect
to the sum operator. In this paper, the contraction-based proof technique
is further refined by moving from the contraction preorder to
\emph{rooted contraction}, which is shown to be the coarsest precongruence
contained in the contraction preorder. The resulting unique-solution theorem
is now valid for \emph{rooted bisimilarity} (hence also for bisimilarity
itself), and places no constraints on summation.

Another benefit of the formalisation is that we can take advantage of results
about different equivalences and preorders that share similar proof structures.
Such structural \xch{similarities}{simililarities} can be found, for instance, in the following
cases: the proofs that rooted bisimilarity and rooted contraction are,
respectively, the coarsest congruence contained in weak bisimilarity and
the coarsest precongruence contained in the contraction preorder; the proofs
about unique solution(s) of equations for weak bisimilarity that use the
contraction preorder as an auxiliary relation, and other unique-solution
theorems (e.g. the one for rooted bisimilarity in which the auxiliary relation
is rooted contraction); the proofs about various forms of enhancements
of the bisimulation proof method (the ``up-to'' techniques). In these cases,
when moving between proofs there are only a few places in the HOL proof
scripts that have to be modified. Then the successful termination of a
proof gives us the guarantee that the proof is correct, eliminating the
risks of overlooking or missing details as in \emph{paper-and-pencil} proofs.

Concerning the formalisation of unique-solution theorems, we first consider
the case of a single equation (or contraction), the
\emph{\univariate case}. Then we turn to the \emph{\multivariate case} where
more equations (or contractions) with multiple equation variables are involved.
The \univariate versions of the unique-solution theorems can directly use
$\lambda $-functions to represent CCS equations, while the
\multivariate versions require more careful and delicate treatments of
CCS expressions, with multiple variables and substitutions acting on them.
In contrast to literature such as~\cite{Gorrieri:2015jt}, we have followed
Milner's original approach~\cite{milner1990operational} and adopted the
same type for both CCS equations and processes: those undefined constants
in CCS terms are treated as (free) equation variables, and a CCS process
is a CCS expression without equation variables. This allows us a smoother
move from the \univariate case to the \multivariate one. (See Section~\ref{sec:multivariate}
for more details.)

%p1 #&#
\bigskip\noindent\textit{Structure of the paper}.
In Section~\ref{ss:ccs} we recall the core theory of CCS, including its
syntax, operational semantics, bisimilarity and rooted bisimilarity. Then,
Section~\ref{s:eq} discusses equations and contractions, and in particular,
Section~\ref{ss:new} presents rooted contraction and the related unique-solution
result for rooted bisimilarity. In Section~\ref{s:for} we highlight the
CCS formalisation in HOL4, with the unique-solution theorems in the
\univariate case. Section~\ref{sec:multivariate} describes the extension
to the \multivariate case. Finally, in \xch{Sections}{Section}~\ref{s:rel} and \ref{s:concl} we discuss the related work, conclusions, and a few directions
for future work.

%s2 #&#
\section{Calculus of communicating systems (CCS)}
%%LEAP%%%\label{sec2}
 \label{ss:ccs}

We assume a possibly infinite set of \emph{names}
$\mathscr{L} = \{a, b, \ldots \}$ yielding \emph{input} and
\emph{output labels} (or \emph{actions}), a special \emph{invisible} action
$\tau \notin \mathscr{L}$, and another possibly infinite set of
\emph{agent variables} $\mathscr{X} = \{X, Y, \ldots \}$. The class of CCS
terms is then inductively defined from $\nil $ (the terminated process)
and agent variables by the operators of \emph{prefixing} (.),
\emph{parallel composition} ($|$), \emph{summation} (or
\emph{binary choice}, $+$), \emph{restriction} ($\nu $),
\emph{relabeling} ($[\cdot ]$) and \emph{recursion} ($\texttt{rec}$):
%
\begin{equation*}
%
\begin{array}{cccl}
\mu & := & \tau \!\!\!\! & \midd \; a \; \midd \; \outC a
\\
P & := & \nil \!\!\!\! & \midd \; \mu . P \; \midd \; P_1 | P_2 \;
\midd \; P_1 + P_2 \; \midd (\res L)\, P \; \midd \; P\; [\mathit{rf}]
\; \midd \; X \; \midd \; \recu X P
\end{array}
%
\end{equation*}
%
In our presentation of CCS, the restriction operator takes a set of labels
$L \subseteq \mathscr{L}$ rather than a single one. The relabeling operator
takes a relabeling function
$\mathit{rf} \colon \mathscr{L} \cup \overline{\mathscr{L}} \cup \{
\tau \} \rightarrow \mathscr{L} \cup \overline{\mathscr{L}} \cup \{
\tau \}$, that can handle multiple actions including $\tau $. A valid relabeling
function $\mathit{rf}$ must however satisfy
$\mathit{rf}(\tau ) = \tau $ and
$\forall l\in \mathscr{L} \cup \overline{\mathscr{L}}.\, \mathit{rf}(
\overline{l}) = \overline{\mathit{rf}(l)}$ (with $\bar{\bar l} = l$ for
all $l \in \mathscr{L}$). We sometimes omit a trailing $\nil $, e.g., writing
$a|b$ for $a.\nil |b .\nil $. A CCS process $P$ may evolve to another one
(i.e. having a \emph{transition}), say $P'$, under an action $\mu $, written
by $P \arr{\mu} P'$. The transition semantics of CCS processes is given
by means of a Labeled Transition System (LTS) expressed in Structural Operational
Semantics (SOS) rules shown in \reftext{Fig.~\ref{f:LTSCCS}}. A CCS process has or
uses \emph{guarded sums} if all occurrences of summation are in the form
$a.P + b.Q$. The \emph{immediate derivatives} of a process $P$ are the elements
of $\{P' \stvtex P \arr\mu P' \mbox{ for some $\mu $}\}$. We use
$\ell $ to range over visible actions (i.e.~inputs or outputs, excluding
$\tau $) and $\mu $ to range over all actions.
%
%f1 #&#
\begin{figure*}%[t]
%
\begin{sgmlfig}
\begin{center}\normalsize
\vskip .1cm $\displaystyle{ \over \mu . P \arr\mu P } $ $ \hbvtex $
\hskip .5cm $\displaystyle{ P \arr\mu P' \over P + Q \arr\mu P' } $
$ \hbvtex $
\hskip .5cm
$\displaystyle{ P \arr\mu P' \over P | Q \arr\mu P' | Q } $
$ \hbvtex $
\hskip .3cm $\; \;$
$\displaystyle{ P \arr{ a}P' \hkvtex \hkvtex Q \arr{\outC a }Q'
\over P| Q \arr{ \tau} P' | Q' }$
\\
\vspace{.2cm}
$\displaystyle{ P \arr{\mu}P' \over (\res L\!)\, P \arr{\mu} (\res L
\!)\, P'} $ $ \mu , \outC\mu \notin L$ $ \hbvtex $
$\displaystyle{ P \sub {\recu X P} X \arr{\mu}P' \over \recu X P
\arr{\mu} P'} $
\hskip .5cm
$\displaystyle{ P \arr{\mu} P' \over P \;[r\!f] \arr{r\!f(\mu )} P'
\;[r\!f]} $ $ \hbvtex $
\end{center}
\end{sgmlfig}
%
\caption{Structural Operational Semantics of CCS. (The symmetric rules for $+$
and $|$ are omitted.)}
 \label{f:LTSCCS}
\end{figure*}

Some standard notations for transitions: $\Arr\epsilon $ is the reflexive
and transitive closure of $\arr\tau $, and $\Arr \mu $ is
$\Arr\epsilon \arr\mu \Arr\epsilon $ (the composition of the three relations).
Moreover, $P \arcap \mu P'$ holds if $P \arr\mu P'$ or
$\mu =\tau \wedge P = P'$; similarly $P \Arcap \mu P'$ holds if ($P
\Arr\mu P'$ or $\mu =\tau \wedge P = P'$). We write
$P \:(\arr\mu )^{n} P'$ if $P$ can become $P'$ after performing $n$
$\mu $-transitions. Finally, $P \arr\mu $ holds if there is $P'$ with
$P \arr\mu P'$, and similarly for other forms of transitions.

%p2 #&#
\bigskip\noindent\textit{Further notations}. We let $\Rvtex $, $\Svtex $ range over binary
relations, sometimes using infix notation for them, e.g.
$P \,\Rvtex \, Q$ means $(P,Q) \in \Rvtex $. We use a tilde, as in
$\til P$, to denote (finite) tuples of elements. All relation notations
can be extended to tuples componentwise, e.g.,
$\til P \,\Rvtex \, \til Q$ means $P_i \,\Rvtex \, Q_i$ for each index
$i$ of the tuples $\til P$ and $\til Q$. We use $\DSdefi $ for abbreviations.
For instance, $P \DSdefi G $, where $G$ is some expression, means that
$P$ stands for the expression $G$. If $\leq $ is a preorder, then
$\geq $ is its inverse (and conversely).

%s2.1 #&#
\subsection{Bisimilarity and rooted bisimilarity}
%%LEAP%%%\label{sec2.1}
 \label{ss:BiEx}

The equivalences we consider here are mainly \emph{weak} ones, in that they
abstract from the number of internal steps being performed:
%
%d2.1 #&#
\begin{definition}%
 \label{d:wb}
A process relation ${\Rvtex}$ is a (weak) \emph{bisimulation} if, whenever
$P\RRvtex Q$,
%
\begin{enumerate}
%
\item $P \arr\mu P'$ implies that there is $Q'$ such that
$Q \Arcap \mu Q'$ and $P' \RRvtex Q'$;
%
\item $Q \arr\mu Q'$, implies that there is $P'$ such that
$P \Arcap \mu P'$ and $P' \RRvtex Q'$;
%
\end{enumerate}
%
$P$ and $Q$ are (weakly) \emph{bisimilar}, written as $P \wbvtex Q$, if
$P \RRvtex Q$ for some bisimulation $\Rvtex $.
\end{definition}

\emph{Strong} bisimulation and strong bisimilarity ($\sim $) can be obtained
by replacing the weak transition $Q\Arcap\mu Q'$ in clause (1) with the
transition $Q \arr \mu Q'$ (and similarly for the other clause). Weak bisimilarity
is not preserved by the sum operator (except for guarded sums), i.e.~$P_1
\wbvtex Q_1$ and $P_2 \wbvtex Q_2$ do not imply
$P_1 + P_2 \wbvtex Q_1 + Q_2$. For this reason, Milner introduced the concept
of \emph{observational congruence}, also called
\emph{rooted bisimilarity}~\cite{Gorrieri:2015jt,Sangiorgi:2011ut}:
%
%d2.2 #&#
\begin{definition}%
 \label{d:rootedBisimilarity}
Two processes $P$ and $Q$ are \emph{rooted bisimilar}, written as
$P \rapprox Q$, if
%
\begin{enumerate}
%
\item $P \arr\mu P'$ implies that there is $Q'$ such that
$Q \Arr\mu Q'$ and $P' \wbvtex Q'$;
%
\item $Q \arr\mu Q'$ implies that there is $P'$ such that
$P \Arr\mu P'$ and $P' \wbvtex Q'$.
%
\end{enumerate}
%
\end{definition}
%
Relation $\rapprox $ is indeed preserved by the sum operator. The above
definition also brings up a proof technique for proving rooted bisimilarity
from a given bisimulation. The next lemma plays an important role in proving
some key results in this paper:
%
%l2.3 #&#
\begin{lemma}[rooted bisimilarity by bisimulation]
 \label{lobsCongrByWeakBisim}
Given a (weak) bisimulation $\RRvtex $, suppose two processes $P$ and
$Q$ satisfy the following properties:
%
\begin{enumerate}
%
\item $P \arr\mu P'$ implies that there is $Q'$ such that
$Q \Arr\mu Q'$ and $P' \RRvtex Q'$;
%
\item $Q \arr\mu Q'$ implies that there is $P'$ such that
$P \Arr\mu P'$ and $P' \RRvtex Q'$.
%
\end{enumerate}
%
Then $P$ and $Q$ are rooted bisimilar, i.e.~$P \approx ^{c} Q$.
\end{lemma}

One basic property of rooted \xch{bisimilarity}{bisimiarity} is congruence, indeed it is the
coarsest congruence contained in bisimilarity (see Section~\ref{s:coarsest}
and Section~\ref{sscontext} for more details):
%
%t2.4 #&#
\begin{theorem}
 \label{t:rapproxCongruence}
$\rapprox $ is a congruence in CCS, and it is the coarsest congruence contained
in $\wbvtex $.
\end{theorem}

%s3 #&#
\section{Equations and contractions}
%%LEAP%%%\label{sec3}
 \label{s:eq}

In the CCS syntax, a recursion $\recu A P$ acts as a binder for $A$ in
the body $P$. This gives rise, in the expected manner, to the notions of
\emph{free} and \emph{bound} recursion variables in a CCS expression. For
instance, $X$ is free in $a.X + \recu Y (b.Y)$ while $Y$ is bound; whereas
$X$ is both free and bound in $a.X + \recu X (b.X)$. A term without free
variables is a \emph{process}.

In this paper (and the formalisation work), we use the agent variables
also as \emph{equation variables}. This eliminates the need of an additional
type for CCS equations, and we can reuse the existing variable substitution
operation (cf.~the SOS rule for the Recursion in \reftext{Fig.~\ref{f:LTSCCS}}) for
substitutions of equation variables. For example, the result of substituting
variable $X$ with $\nil $ in $a.X + \recu X (b.X)$, written as
$(a.X + \recu X (b.X)) \sub {\nil} X$, is $a.\nil + \recu X (b.X)$ with
$\recu X (b.X)$ untouched. \Multivariate substitutions are written in the
same syntax, e.g. $E \sub {\til P} {\til X}$. Whenever $\til X$ is clear
from the context, we may also write $E[\til P]$ instead of
$E \sub {\til P} {\til X}$ (and $E[P]$ for $E \sub {P} X$ if there is a single
equation variable $X$).

%s3.1 #&#
\subsection{Systems of equations}
%%LEAP%%%\label{sec3.1}
 \label{ss:SysEq}

When discussing equations it is standard to talk about ``contexts''. This
is a CCS expression possibly containing free variables that, however, may
not occur within the body of recursive definitions. Milner's ``unique solution
of equations'' theorems~\cite{Mil89} intuitively say that, if a context
$C$ obeys certain conditions, then all processes $P$ that satisfy the equation
$P \wbvtex \ctvtex P$ are bisimilar with each other.

%d3.1 #&#
\begin{definition}[equations]
 \label{def:equation}
Assume that, for each $i$ of a countable indexing set $I$, we have variables
$X_i$, and expressions $E_i$ possibly containing such variables
$\bigcup _i \{ X_i\}$. Then $\{ X_i = E_i\}_{i\in I}$ is a
\emph{system of equations}. (There is one equation $E_i$ for each variable
$X_i$.)
\end{definition}

The components of $\til P$ need not be different from each other, as it
must be for the variables $\til X$.

%d3.2 #&#
\begin{definition}[solutions and the unique solution]
 \label{def:solution}
Suppose $\{ X_i = E_i\}_{i\in I}$ is a system of equations:
%
\begin{itemize}
%
\item $\til P$ is a
\emph{solution of the system of equations (for $\wbvtex $)} if for each
$i$ it holds that $P_i \wbvtex E_i [\til P]$;
%
\item The system has \emph{a unique solution for $\wbvtex $} if whenever
$\til P$ and $\til Q$ are both solutions then
$\til P \wbvtex \til Q$.
%
\end{itemize}
%
\end{definition}
%
\xch{Similarly}{Similarily}, the
\emph{(unique) solution of a system of equations for $\sim $} (or for
$\rapprox $) can be obtained by replacing all occurrences of
$\wbvtex $ in the above definition with $\sim $ and $\rapprox $, respectively.

For instance, the solution of the equation $X \wbvtex a. X$ is the process
$R \DSdefi \recu A {\, (a. A)}$, and for any other solution $P$ we have
$P \wbvtex R$. In contrast, the equation $X \wbvtex a| X$ has solutions
that may be quite different, for instance, $K$ and $K | b$, for
$K \DSdefi \recu K {\, (a. K)}$. (Actually any process capable of continuously
performing $a$--actions is a solution of $X \wbvtex a| X$.) Examples of
systems that do not have unique solutions are: $X = X$,
$X = \tau . X$ and $X = a | X$.

%d3.3 #&#
\begin{definition}[guardedness of equations]
 \label{def:guardness}
A system of equations $\{ X_i = E_i\}_{i\in I}$ is
%
\begin{itemize}
%
\item \emph{weakly guarded} if, in each $E_i$, each occurrence of each
$X_i$ is underneath a prefix;
%
\item \emph{guarded} if, in each $E_i$, each occurrence of each $X_i$ is
underneath a \emph{visible} prefix;
%
\item \emph{sequential} if, in each $E_i$, each occurrence of each
$X_i$ is only underneath prefixes and sums.
%
\end{itemize}
%
\end{definition}

In other words, if a system of equations is sequential, then for each
$E_i$, any subexpression of $E_i$ in which $X_j $ appears, apart from
$X_j$ itself, is a sum of prefixed expressions. For instance,
%
\begin{itemize}
%
\item $X = \tau . X + \mu . \nil $ is sequential but not guarded, because
the guarding prefix for the variable is not visible;
%
\item $X = \ell . X | P$ is guarded but not sequential;
%
\item $X = \ell . X + \tau . \res a (a .\outC b | a.\nil )$, as well as
$X = \tau . (a. X + \tau . b .X + \tau )$ are both guarded and sequential.
%
\end{itemize}

Milner has three versions of ``unique solution of equations'' theorems,
for $\sim $, $\wbvtex $ and $\rapprox $, respectively, though only the
following two versions are explicitly mentioned in~\citep[p.~103, 158]{Mil89}:
%
%t3.4 #&#
\begin{theorem}[unique solution of equations for $\sim $]
 \label{t:Mil89s1}
Let $E_i$ be weakly guarded with free variables in $\til X$, and let
${\til P} \sim {\til E}\{\til P /\til X\}$,
${\til Q} \sim {\til E}\{\til Q /\til X\}$. Then
${\til P} \sim {\til Q}$.
\end{theorem}

%t3.5 #&#
\begin{theorem}[unique solution of equations for $\rapprox $]
 \label{t:Mil89s3}
Let $E_i$ be guarded and sequential with free variables in $\til X$, and
let ${\til P} \rapprox {\til E}\{\til P /\til X\}$,
${\til Q} \rapprox {\til E}\{\til Q /\til X\}$. Then
${\til P} \rapprox {\til Q}$.
\end{theorem}

The version of Milner's unique-solution theorem for $\wbvtex $ further
requires that all sums are guarded:
%
%t3.6 #&#
\begin{theorem}[unique solution of equations for $\wbvtex $]
 \label{t:Mil89}
Let $E_i$ (with only guarded sums) be guarded and sequential, and with
free variables in $\til X$. Let
${\til P} \wbvtex {\til E}\{\til P /\til X\}$,
${\til Q} \wbvtex {\til E}\{\til Q /\til X\}$, then
${\til P} \wbvtex {\til Q}$.
\end{theorem}

The proof of the last two theorems above exploits an invariance property
on immediate derivatives of guarded and sequential expressions, and then
extracts a bisimulation (up to bisimilarity) out of the solutions of the
system. To see the need of the sequentiality condition, consider the equation
(from~\cite{Mil89}) $X \wbvtex \res a (a. X | \outC a)$ where $X$ is guarded
but not sequential. Any process that does not perform $a$--action is a
solution, e.g. $\nil $ and $b.\nil $.

For more details on the formalisation of the above three theorems, see
Section~\ref{ss:part2} for the \univariate case and Section~\ref{sec:multivariate}
for the \multivariate case.

%s3.2 #&#
\subsection{Expansions and contractions}
%%LEAP%%%\label{sec3.2}
 \label{s:mcontr}

Milner's ``unique solution of equations'' theorem for $\wbvtex $ (\reftext{Theorem~\ref{t:Mil89}})
brings a new proof technique for proving (weak) bisimilarities. However,
it has limitations: the equations must be guarded and sequential. (Moreover,
all sums where equation variables appear must be guarded sums.) This limits
the usefulness of the technique, since the occurrences of other operators
using equation variables, such as parallel composition and restriction,
in general cannot be eliminated. The constraints in \reftext{Theorem~\ref{t:Mil89}},
however, can be weakened if we move from equations to a special kind of
inequations called \emph{contractions}.

Intuitively, the bisimilarity contraction $\mcontrBIS $ is a preorder in
which $P \mcontrBIS Q$ holds if $P \wbvtex Q$ and, in addition,
\emph{$Q$ has the possibility of being at least as efficient as $P$} (as
far as $\tau $-actions are performed). The process $Q$, however, may be
nondeterministic and may have other ways to do the same work, ways which
could be slower (i.e., involving more $\tau $-actions than those performed
by $P$).

%d3.7 #&#
\begin{definition}[contraction]
 \label{d:BisCon}
A process relation ${\Rvtex}$ is a \emph{(bisimulation) contraction} if,
whenever $P\RRvtex Q$,
%
\begin{enumerate}
%
\item $P \arr\mu P'$ implies that there is $Q'$ with
$Q \arcap \mu Q'$ and $P' \RRvtex Q'$;
%
\item $Q \arr\mu Q'$ implies that there is $P'$ with
$P \Arcap \mu P'$ and $P' \wbvtex Q'$.
%
\end{enumerate}
%
Two processes $P$ and $Q$ are in the \emph{bisimilarity contraction}, written
as $P \mcontrBIS Q$, if $P\ \Rvtex \ Q$ for some contraction
$\Rvtex $. Sometimes we write $\mexpaBIS $ for the inverse of
$\mcontrBIS $.
\end{definition}
%
In clause (1) of the above definition, $Q$ is required to match the challenge
transition of $P$ with at most one transition. This makes sure that
$Q$ is capable of mimicking $P$'s work at least as efficiently as
$P$. In contrast, clause (2) entirely ignores efficiency on the challenges
from $Q$: the final derivatives are required to be related by
$\wbvtex $, rather than by $\Rvtex $.

Bisimilarity contraction is coarser than bisimilarity expansion
$\expa $~\cite{arun1992efficiency,sangiorgi2015equations}, one of the most
useful auxiliary relations in up-to techniques:
%
%d3.8 #&#
\begin{definition}[expansion]
 \label{d:expa}
A process relation ${\Rvtex}$ is an \emph{expansion} if, whenever
$P\RRvtex Q$,
%
\begin{enumerate}
%
\item $P \arr\mu P'$ implies that there is $Q'$ with
$Q \arcap \mu Q'$ and $P' \RRvtex Q'$;
%
\item $Q \arr\mu Q'$ implies that there is $P'$ with $P \Arr \mu P'$ and
$P' \RRvtex Q'$.
%
\end{enumerate}
%
Two processes $P$ and $Q$ are in the \emph{bisimilarity expansion}, written
as $P \expa Q$, if $P \RRvtex Q$ for some expansion $\Rvtex $.
\end{definition}
%
Bisimilarity expansion is widely used in proof techniques for bisimilarity.
It intuitively refines bisimilarity by formalising the idea of ``efficiency''
between processes. Clause (1) is the same as for contraction, while in
clause (2) expansion requires $P \Arr \mu P'$, rather than
$P \Arcap \mu P'$. Moreover, in clause (2) of \reftext{Definition~\ref{d:BisCon}} the final
derivatives are simply required to be bisimilar ($P' \wbvtex Q'$). Intuitively,
$P \expa Q$ holds if $P\wbvtex Q$ and, in addition,
\emph{$Q$ is always at least as efficient as $P$}.

%e3.9 #&#
\begin{example}
 \label{exa:contr}
We have $ a \not \mcontrBIS \tau . a$. However,
$a+ \tau . a \mcontrBIS a$, as well as its converse,
$ a \mcontrBIS a + \tau . a $. Indeed, if $P \wbvtex Q$ then
$ P \mcontrBIS P +Q$. The last two relations do not hold with
$\expa $, which explains the strictness of the inclusion
${\expa} \subset {\mcontrBIS}$.
\end{example}

Bisimilarity expansion and bisimilarity contraction are both preorders.
Similarily with (weak) bisimilarity, both the expansion and the contraction
preorders are preserved by all CCS operators except the summation. The
proofs are similar to those for bisimilarity, see, e.g.~\cite{sangiorgi2017equations}
for details.

%s3.3 #&#
\subsection{Systems of contractions}
%%LEAP%%%\label{sec3.3}
 \label{ss:SysContr}

A \emph{system of contractions} is defined as a system of equations, except
that the contraction symbol $\mcontrBIS $ is used in the place of
$=$ in \reftext{Definition~\ref{def:equation}}. Thus a system of contractions is a set
$\{ X_i \mcontrBIS E_i\}_{i\in I}$ where $I$ is an indexing set and each
$E_i$ contains variables in $\til X$.

Now we recall the ``unique solution of contractions'' theorem~\cite{sangiorgi2017equations},
which weakens the requirements of Milner's result (\reftext{Theorem~\ref{t:Mil89}}).

%l3.10 #&#
\begin{lemma}
 \label{l:uptocon}
Suppose $\til P$ and $\til Q$ are solutions (for $\mcontrBIS $) of a system
of weakly guarded contractions that uses guarded sums. For any context
$\qct $ that uses guarded sums, if $\ctvtex{\til P}\Arr{\mu} R$, then there
is a context $\qctp $ that uses guarded sums such that
$R \mcontrBIS \ctp{\til P}$ and
$\ctvtex{\til Q} \Arcap{\mu} R' \wbvtex \ctp{\til Q}$ for some $R'$.
\end{lemma}

\begin{proof}%
{(sketch from~\cite{sangiorgi2017equations})} Let $n$ be the length (i.e.,
the number of one-step transitions) of a transition
$\ctvtex{\til P}\Arr\mu R$, and let $\ctpp {\til P}$ and
$\ctpp {\til Q}$ be the processes obtained from $\ctvtex {\til P}$ and
$\ctvtex {\til Q}$ by unfolding the definition of $\mcontrBIS $ for
$n$ times. Thus in $\qctpp $ each hole is underneath at least $n$ prefixes,
and therefore cannot contribute to an action in the first $n$ transitions.
Moreover, all involved contexts use guarded sums.

We have $\ctvtex{\til P} \mcontrBIS \ctpp{\til P}$ and
$\ctvtex{\til Q} \mcontrBIS \ctpp{\til Q}$ from the precongruence property
of $\mcontrBIS $ (we exploit here the syntactic constraints on sums). Moreover,
since each hole of the context $\qctpp $ is underneath at least $n$ prefixes,
applying the definition of $\mcontrBIS $ on the transition
$\ctvtex{\til P}\Arr{\mu} R$, we infer the existence of $\qctp $ such that
$\ctpp{\til P}\Arcap{\mu} \ctp{\til P} \mexpaBIS R$ and
$\ctpp{\til Q}\Arcap{\mu} \ctp{\til Q}$. Finally, again applying the definition
of $\mcontrBIS $ on $\ctvtex{\til Q} \mcontrBIS \ctpp{\til Q}$, we derive
$\ctvtex{\til Q}\Arcap{\mu} R' \wbvtex \ctp{\til Q}$ for some $R'$.
\end{proof}

%t3.11 #&#
\begin{theorem}[unique solution of contractions~\cite{sangiorgi2017equations}]
 \label{t:contraBisimulationU}
Let $E_i$ (and with guarded sums only) be weakly guarded, with free variables
in $\til X$, and let $\til P \mcontrBIS \til E\{\til P/\til X\}$,
$\til Q \mcontrBIS \til E\{\til Q/\til X\}$. Then
$\til P \wbvtex \til Q$.
\end{theorem}

\begin{proof}
We prove $P_i \wbvtex Q_i$ (for each $i \in I$) by considering the following
relation
%
\begin{equation*}
\Rvtex \DSdefi \{(R,S) \stvtex R \wbvtex \ctvtex{\til P}, S \wbvtex
\ctvtex{\til Q}
\mbox{~for some context $\qct $ (with only guarded sums)} \} .
\end{equation*}

Obviously we have $(P_i,Q_i) \in \Rvtex $ (by taking $C = E_i$, and the
fact that $\mcontrBIS $ implies $\wbvtex $). It remains to show that
$\Rvtex $ is a bisimulation. Suppose $(R, S) \in \Rvtex $ via a context
$C$. For any $R'$ such that $R \arr{\mu} R'$, we have to find an
$S'$ with $S \Arcap{\mu} S'$ and $(R', S') \in \Rvtex $. From
$R \wbvtex C[{\til P}]$, we derive
$C[{\til P}] \Arcap{\mu} R'' \wbvtex R'$ for some $R''$. Then by \reftext{Lemma~\ref{l:uptocon}},
there exists $C'$ with $R'' \mcontrBIS C'[{\til P}]$ and
$C[{\til Q}] \Arcap{\mu} S'' \wbvtex C'[{\til Q}]$ for some $S''$. From
$S \wbvtex C[{\til Q}]$ and $C[{\til Q}] \Arcap{\mu} S''$, by induction
and definition of $\wbvtex $, we find $S'$ with $S \Arcap{\mu} S'$. This
completes the proof, as we have $R' \wbvtex C'[{\til P}]$ and
$S' \wbvtex C'[{\til Q}]$ by transitivity of $\wbvtex $ and the fact that
$\mcontrBIS $ implies $\wbvtex $. (The other side from $S$ follows in the
same manner.) See \reftext{Fig.~\ref{fig:310}} for a visual illustration.
\end{proof}

%f2 #&#
\begin{figure}%[ht]
\vspace{-12pt}
%d3.3 #&#
\begin{sgmlfig}\normalsize
%
$$
\xymatrix{
{C[{\til P}]} \ar@{.}[r]^{\wbvtex} \ar@{=>}[d]^{\widehat{\mu}} & {R} \ar@{-}[r]
\ar@{->}[d]^{\mu} & {\Rvtex} \ar@{-}[r] & {S} \ar@{.}[r]^{\wbvtex}
\ar@{=>}[d]^{\widehat{\mu}} & {C[{\til Q}]} \ar@{=>}[d]^{\widehat{\mu}} \\
{R''} \ar@{.}[d]_{\mcontrBIS} \ar@{.}[r]^{\wbvtex} & {R'} \ar@{-}[r]
\ar@{.}[ld]^{\wbvtex} & {\Rvtex} \ar@{-}[r] & {S'} \ar@{.}[r]^{\wbvtex} \ar@{.}[rd]_{\wbvtex}
& {S''} \ar@{.}[d]^{\wbvtex} \\
{C'[{\til P}]} & {} & {} & {} & {C'[{\til Q}]}
}
%
$$\vspace{-12pt}
\end{sgmlfig}
%
\caption{Proof illustration of \reftext{Theorem~\ref{t:contraBisimulationU}} (showing
$\Rvtex $ is a bisimulation).}
 \label{fig:310}
\end{figure}

%s3.4 #&#
\subsection{Rooted contraction}
%%LEAP%%%\label{sec3.4}
 \label{ss:new}

\reftext{Theorem~\ref{t:contraBisimulationU}} brings a new proof technique for bisimilarity,
which is less restrictive than Milner's \reftext{Theorem~\ref{t:Mil89}} (but with
the additional costs of checking $\mcontrBIS $ in addition to
$\wbvtex $). However, compared with Milner's \reftext{Theorem~\ref{t:Mil89s1}}, there
remains the limitation on the need of guarded sums. This is mainly due
to the fact that $\wbvtex $ is not a congruence and also
$\mcontrBIS $ is not a precongruence. Inspired by rooted bisimilarity,
to eliminate the restriction on guarded sums we refine the idea of contractions
by moving to \emph{rooted contractions}:

%d3.12 #&#
\begin{definition}
 \label{d:rcontra}
Two processes $P$ and $Q$ are in \emph{rooted contraction}, written as
$P\rcontr Q$, if
%
\begin{enumerate}
%
\item $P \arr\mu P'$ implies that there is $Q'$ with $Q \arr \mu Q'$ and
$P'\mcontrBIS Q'$;
%
\item $Q \arr\mu Q'$ implies that there is $P'$ with $P \Arr \mu P'$ and
$P' \wbvtex Q'$.
%
\end{enumerate}
%
\end{definition}

The above definition was found with the help of interactive theorem proving.
The following two principles were adopted when manually searching for a
possible definition: (1) the definition should not be coinductive, along
the lines of rooted bisimilarity $\rapprox $ (\reftext{Definition~\ref{d:rootedBisimilarity}});
(2) the definition should be built on top of the existing
\emph{contraction} relation $\mcontrBIS $. Furthermore, we needed to prove
that the definition being found indeed yields the coarsest precongruence
contained in $\mcontrBIS $. The proof is similar with the analogous result
for $\rapprox $. See Section~\ref{s:coarsest} for more details.

%t3.13 #&#
\begin{theorem}
 \label{t:rcontrPrecongruence}
$\rcontr $ is a precongruence in CCS, and it is the coarsest precongruence
contained in $\mcontrBIS $.
\end{theorem}

For this new relation, the analogous of \reftext{Lemma~\ref{l:uptocon}} and of \reftext{Theorem~\ref{t:contraBisimulationU}}
can now be stated without constraints on summation. The schema of the proofs
is almost identical, because \xch{the only property}{all properties} of $\rcontr $ needed in this
proof is its precongruence, which is indeed true for all weakly guarded
contexts:
%
%l3.14 #&#
\begin{lemma}
 \label{l:ruptocon}
Let $\til P$ and $\til Q$ be solutions (for $\rcontr $) of a system of
weakly guarded contractions. For any context $\qct $, if
$\ctvtex{\til P}\Arr{\mu} R$, then there is a context $\qctp $ such that
$R \mcontrBIS \ctp{\til P}$ and
$\ctvtex{\til Q} \Arr{\mu} R' \wbvtex \ctp{\til Q}$ for some $R'$.
\end{lemma}

The next lemma is actually Lemma 3.13 of~\citep[p.~102]{Mil89}, and is
needed to prove Milner's ``unique solution of equations for $\sim $'' (\reftext{Theorem~\ref{t:Mil89s1}}):
%
%l3.15 #&#
\begin{lemma}
 \label{lem:milner313}
If the variables ${\til X}$ are weakly guarded in $E$, and
$E\{\til P/\til X\} \arr{\alpha} P'$, then $P'$ takes the form
$E'\{\til P/\til X\}$ (for some expression $E'$), and moreover, for any
${\til Q}$, $E\{\til Q/\til X\} \arr{\alpha} E'\{\til Q/\til X\}$.
\end{lemma}

%t3.16 #&#
\begin{theorem}[unique solution of rooted contractions]
 \label{t:rcontraBisimulationU}
Let $E_i$ be weakly guarded with free variables in $\til X$, and let
$\til P \rcontr \til E\{\til P/\til X\}$,
$\til Q \rcontr \til E\{\til Q/\til X\}$. Then
$\til P \rapprox \til Q$.
\end{theorem}

\begin{proof}
We prove $P_i \rapprox Q_i$ (for each $i \in I$) by considering the following
relation
%
\begin{equation*}
\Rvtex \DSdefi \{(R,S) \stvtex R \wbvtex \ctvtex{\til P}, S \wbvtex
\ctvtex{\til Q} \mbox{~for some context $\qct $} \} .
\end{equation*}

Following the same steps in the proof of \reftext{Theorem~\ref{t:contraBisimulationU}}
(using \reftext{Lemma~\ref{l:ruptocon}} in place of \reftext{Lemma~\ref{l:uptocon}}), we can
prove that $(P_i, Q_i) \in \Rvtex $ and $\Rvtex $ is indeed a bisimulation.
But this only shows $P_i \wbvtex Q_i$. To further show
$P_i \rapprox Q_i$, we appeal to \reftext{Lemma~\ref{lobsCongrByWeakBisim}}: for
any $P'$ such that $P_i \arr{\mu} P'$, we have to find a $Q'$ with
$Q_i \Arr{\mu} Q'$ and $(R',S') \in \Rvtex $. From
$P_i \rcontr E_i[{\til P}]$, by definition of $\rcontr $ we derive
$E_i[{\til P}] \arr{\mu} P''$ for some $P''$. Then by \reftext{Lemma~\ref{lem:milner313}}
there exists a context $E'$ with $P'' = E'[{\til P}]$ and
$E_i[{\til Q}] \arr{\mu} E'[{\til Q}]$. From
$Q_i \rcontr E_i[{\til Q}]$ and
$E_i[{\til Q}] \arr{\mu} E'[{\til Q}]$, by definition of $\rcontr $ we
have $Q_i \Arr{\mu} Q'$ for some $Q'$ and $Q' \wbvtex E'[{\til Q}]$. This
completes the proof, as we have $P' \wbvtex C[{\til P}]$ (by the fact that
$\mcontrBIS $ implies $\wbvtex $) and $Q' \wbvtex C[{\til Q}]$. (The other
side from $Q_i$ follows in the same manner.) See \reftext{Fig.~\ref{fig:314}} for
a visual illustration.
\end{proof}

%f3 #&#
\begin{figure}%[ht]
\vspace{-12pt}%
%d3.4 #&#
\begin{sgmlfig}\normalsize
%
$$
\xymatrix{
{E_i[{\til P}]} \ar@{.}[r]^{\preceq ^{\mathrm{c}}_{\mathrm{bis}}} \ar@{->}[d]^{\mu} & {P_i} \ar@{-}[r]
\ar@{->}[d]^{\mu} & {\Rvtex} \ar@{-}[r] & {Q_i} \ar@{.}[r]^{\rcontr}
\ar@{=>}[d]^{\mu} & {E_i[{\til Q}]} \ar@{->}[d]^{\mu} \\
{P'' = E'[{\til P}]} \ar@{.}[r]^{\qquad \preceq _{\mathrm{bis}}} & {P'} \ar@{-}[r] & {\Rvtex} \ar@{-}[r] & {Q'} \ar@{.}[r]^{\wbvtex}
& {E'[{\til Q}]} \\
}
%
$$\vspace{-12pt}
\end{sgmlfig}
%
\caption{Proof illustration of \reftext{Theorem~\ref{t:rcontraBisimulationU}} (showing
$P_i \rapprox Q_i$).}
 \label{fig:314}
\end{figure}

%s4 #&#
\section{The formalisation}
%%LEAP%%%\label{sec4}
 \label{s:for}

We highlight here a comprehensive formalisation of CCS in the HOL theorem
prover (HOL4)~\cite{Melham:1993vl,slind2008brief}, with a focus towards
the theory (and formal proofs) of the unique solution of equations/contractions
theorems mentioned in Section~\ref{s:eq} and \ref{s:mcontr}. All proof
scripts are available as part of HOL's official examples.\footnote{\url{https://github.com/HOL-Theorem-Prover/HOL/tree/master/examples/CCS}.}
The work so far consists of about 24,000 lines (1MB) of code in total,
in which about 5,000 lines were derived from the early work of Monica Nesi~\cite{Nesi:1992ve}
on HOL88, with major modifications.

Higher Order Logic (HOL)~\cite{hollogic} traces its roots back to the
\emph{Logic of Computable Functions (LCF)}~\cite{gordon1979edinburgh,milner1972logic}
by Robin Milner and others since 1972. It is a variant of Church's Simple
Theory of Types (STT)~\cite{church1940formulation}, plus a higher order
version of Hilbert's choice operator $\varepsilon $, Axiom of Infinity,
and Rank-1 (prenex) polymorphism. HOL4 has implemented the original HOL,
while some other theorem provers in the HOL family (e.g.~Isabelle/HOL)
have certain extensions. Indeed, HOL has considerably simpler logical foundations
than most other theorem provers. As a consequence, theories and proofs
verified in HOL are easier to understand for people who are not \xch{familiar}{familar}
with more advanced dependent type theories, e.g. the Calculus of Constructions
implemented in Coq.

HOL4 is implemented in Standard ML, and the same programming language plays
three different roles:
%
\begin{itemize}
%
\item The underlying implementation language for the core HOL engine;
%
\item The language in which proof tactics are implemented;
%
\item The interface language of the HOL proof scripts and interactive shell.
%
\end{itemize}
%
Moreover, using the same language HOL4 users can write complex automatic
verification tools by calling HOL's theorem proving facilities. The formal
proofs of the CCS theorems that we have carried out are mostly done in
a manner that closely follows their paper proofs, with minimal automatic
proof searching.

%s4.1 #&#
\subsection{Higher order logic (HOL)}
 \label{sec4.1}

HOL is a formal system of typed logical terms. The types are expressions
that denote sets (in the universe $\mathcal{U}$). HOL type system is much
simpler than those based on dependent types and other type theories. There
are four kinds of types in the HOL logic, as illustrated in \reftext{Fig.~\ref{fig:hol-types}} for its BNF grammar. In HOL, the standard atomic types
\emph{bool} and \emph{ind} denote, respectively, the distinguished two-element
set $\mathbf{2}$ and the distinguished infinite set $\mathbf{I}$.


%f4 #&#
\begin{figure}%[h]
\vspace{-12pt}%
\begin{sgmlfig}\normalsize
%
$$
\sigma \quad ::=\quad {\mathord{\mathop{\alpha}\limits _{\tyvar}}}
\quad \mid \quad{\mathord{\mathop{c}\limits _{\tyatom}}} \quad \mid
\quad \underbrace{(\sigma _1, \ldots , \sigma _n){op}}_{\cmpty}
\quad \mid \quad \underbrace{\sigma _1\fun \sigma _2}_{\funty}
%
$$\vspace{-12pt}
\end{sgmlfig}
%
\caption{HOL's type grammar.}
 \label{fig:hol-types}
\end{figure}

HOL terms represent elements of the sets denoted by their types. There
are four kinds of HOL terms, which can be described (in simplified forms)
by the BNF grammar in \reftext{Fig.~\ref{fig:hol-terms}}. (See~\cite{hollogic} for
a complete description of HOL, including the primitive derivative rules
to be mentioned below.)


%f5 #&#
\begin{figure}%[h]
\vspace{-12pt}%
\begin{sgmlfig}\normalsize
%
$$
t \quad ::=\quad {\mathord{\mathop{x}\limits _{\var}}} \quad \mid
\quad{\mathord{\mathop{c}\limits _{\const}}} \quad \mid \quad
\underbrace{t\ t'}_{\app} \quad \mid \quad
\underbrace{\lambda x .\ t}_{\abs}
%
$$\vspace{-12pt}
\end{sgmlfig}
%
\caption{HOL's term grammar.}
 \label{fig:hol-terms}
\end{figure}

The deductive system of HOL is specified by eight primitive derivative
rules:
%
\begin{enumerate}
%
\item Assumption introduction (\texttt{ASSUME});
%
\item Reflexivity (\texttt{REFL});
%
\item $\beta $-conversion (\texttt{BETA\_CONV});
%
\item Substitution (\texttt{SUBST});
%
\item Abstraction (\texttt{ABS});
%
\item Type instantiation (\texttt{INST\_TYPE});
%
\item Discharging an assumption (\texttt{DISCH});
%
\item Modus Ponens (\texttt{MP}).
%
\end{enumerate}
%
All proofs are eventually reduced to applications of the above rules, which
also give the semantics of two \xch{fundamental}{foundamental} logical connectives, equality
($=$) and implication ($\Rightarrow $). The remaining logical connectives
and first-order quantifiers, including the logical true (\HOLinline{\HOLConst{T}})
and false (\HOLinline{\HOLConst{F}}), are further defined as
$\lambda $-functions:
%
\begin{equation*}
%
\begin{array}{l}
\turn \Tvtex \HOLTokenDefEquality ((\lquant{x_{\tyvtex{bool}}}x) = (
\lquant{x_{\tyvtex{bool}}}x))
\\
\turn \forall \HOLTokenDefEquality
\lquant{P_{\alpha \fun \tyvtex{bool}}} P = (\lquant{x}\Tvtex )
\\
\turn \exists \HOLTokenDefEquality
\lquant{P_{\alpha \fun \tyvtex{bool}}} P({\hilbert}\, P)
\\
\turn \Fvtex \HOLTokenDefEquality \uquant{b_{\tyvtex{bool}}} b
\\
\turn \neg \HOLTokenDefEquality \lquant{b} b \imp \Fvtex
\\
\turn {\wedge} \HOLTokenDefEquality \lquant{b_1\ b_2}\uquant{b} (b_1
\imp (b_2 \imp b)) \imp b
\\
\turn {\vee} \HOLTokenDefEquality \lquant{b_1\ b_2}\uquant{b} (b_1
\imp b)\imp ((b_2 \imp b) \imp b)
\\
\turn \OneOne \HOLTokenDefEquality \lquant{f_{\alpha \fun \beta}}
\uquant{x_1\ x_2} (f\ x_1 = f\ x_2) \imp (x_1 = x_2)
\\
\turn \Onto \HOLTokenDefEquality \lquant{f_{\alpha \fun \beta}}
\uquant{y}\equant{x} y = f\ x
\\
\turn \TyDef \HOLTokenDefEquality \lambda P_{\alpha \fun
\tyvtex{bool}}\
 rep_{\beta \fun \alpha}.\; \OneOne \ rep \ \wedge{}\  (\uquant{x}P x =
(\equant{y} x = rep\ y))
\end{array}
%
\end{equation*}
%
The last logical constant, $\TyDef $, can be used to define new HOL types
as bijections of subsets of existing types~\cite{Melham:1989dk}. HOL
\texttt{Datatype} package~\cite{Melham:1991,holdesc} automates this tedious
process, and can be used for defining the types needed for CCS. Finally,
the whole HOL \emph{standard} theory is based on the following four axioms\footnote{HOL
is strictly weaker than ZFC (the Zermelo-Frankel set theory with the Axiom
of Choice), thus not all theorems valid in ZFC can be formalised in HOL.
(See~\cite{hollogic} for more details.)}:
%
\begin{equation*}
%
\begin{array}{@{}l@{\qquad}l}
\mbox{\texttt{BOOL\_CASES\_AX}} &\vdash \uquant{b} (b = \Tvtex )\
\vee \ (b = \Fvtex )
\\
\mbox{\texttt{ETA\_AX}} & \vdash \uquant{f_{\alpha \fun \beta}}(
\lquant{x}f\ x) = f
\\
\mbox{\texttt{SELECT\_AX}} & \vdash
\uquant{P_{\alpha \fun \tyvtex{bool}}\ x} P\ x \imp P({\hilbert}\ P)
\\
\mbox{\texttt{INFINITY\_AX}}& \vdash \equant{f_{\ind \fun \ind}}
\OneOne \ f \conj \neg (\Onto \ f)
\\
\end{array}
%
\end{equation*}

Usually the above four axioms are the only axioms allowed in conventional
formalisation projects in HOL4: adding new axioms manually may break logical
consistency.

%s4.2 #&#
\subsection{The CCS formalisation}
%%LEAP%%%\label{sec4.2}
 \label{ss:formalCCS}

The CCS formalisation starts with type definitions for action, relabeling
and then the processes. We use the type ``\HOLinline{\ensuremath{\beta} \HOLTyOp{Label}}''
for all labels (i.e. visible actions), where the type variable
$\beta $ corresponds to $\mathscr{L}$ (the set of names for labels) mentioned
at the beginning of Section~\ref{ss:ccs}. (Thus the cardinality of ``\HOLinline{\ensuremath{\beta} \HOLTyOp{Label}}''
depends on its type variable: when $\beta $ is finite or countable, ``\HOLinline{\ensuremath{\beta} \HOLTyOp{Label}}''
is countable.) All labels are divided into input and output ones. The type
``\HOLinline{\ensuremath{\beta} \HOLTyOp{Label}}'' is defined by HOL's
\texttt{Datatype} package in the following syntax:
%
\begin{lstlisting}
Datatype: Label = name 'b | coname 'b
End
\end{lstlisting} Intuitively, ``\HOLinline{\ensuremath{\beta} \HOLTyOp{Label}}''
turns names in $\mathscr{L}$ into \emph{input and output labels}. For instance,
if the type $\beta $ is instantiated to \texttt{string} (the type of all
ASCII strings), then the HOL terms
\HOLinline{\HOLConst{name} \HOLStringLit{a}} and
\HOLinline{\HOLConst{coname} \HOLStringLit{b}} denote the input label
$a$ and output label $\overline{b}$, respectively. The type ``\HOLinline{\ensuremath{\beta} \HOLTyOp{Action}}''
is the union of all visible actions (input and output labels) and the invisible
action $\tau $ (\texttt{tau}). For \xch{instance}{instace}, the input \emph{action}
$a$ and output \emph{action} $\overline{b}$ of type ``\HOLinline{\HOLTyOp{string} \HOLTyOp{Action}}''
are denoted by
\HOLinline{\HOLConst{label} \ensuremath{(}\HOLConst{name} \HOLStringLit{a}\ensuremath{)}}
and
\HOLinline{\HOLConst{label} \ensuremath{(}\HOLConst{coname} \HOLStringLit{b}\ensuremath{)}},
respectively. On the other hand,
\HOLinline{\HOLConst{label} \ensuremath{(}\HOLConst{name} \HOLNumLit{1}\ensuremath{)}}
and
\HOLinline{\HOLConst{label} \ensuremath{(}\HOLConst{coname} \HOLNumLit{2}\ensuremath{)}}
could be actions of type ``\HOLinline{\HOLTyOp{num} \HOLTyOp{Action}}'',
where \HOLinline{\HOLTyOp{num}} is the type of natural numbers in HOL.

The type ``\HOLinline{\ensuremath{(}\ensuremath{\alpha}, \ensuremath{\beta}\ensuremath{)} \HOLTyOp{CCS}}'',
accounting for all CCS terms, has two type variables $\alpha $ and
$\beta $ corresponding to the set of agent variables $\mathscr{X}$ and
the set of names $\mathscr{L}$, respectively. (Indeed the CCS syntax in
Section~\ref{ss:ccs} is parametric with respect to the choice of these
two sets.) The type ``\HOLinline{\ensuremath{(}\ensuremath{\alpha}, \ensuremath{\beta}\ensuremath{)} \HOLTyOp{CCS}}''
is defined inductively by the \texttt{Datatype} package (here ``\HOLinline{\ensuremath{\beta} \HOLTyOp{Relabeling}}''
is the type of all relabeling functions; we have also formalised relabeling,
though it is not discussed below):\looseness=-1
%
\begin{lstlisting}
Datatype: CCS = nil
              | var 'a
              | prefix ('b Action) CCS
              | sum CCS CCS
              | par CCS CCS
              | restr (('b Label) set) CCS
              | relab CCS ('b Relabeling)
              | rec 'a CCS
End
\end{lstlisting} The above definition allows us to write terms like
\texttt{nil} and \texttt{sum P Q} in HOL4. Their correspondences with conventional
CCS notations in the literature are given in \reftext{Table~\ref{tab:ccsoperator}},
where most CCS operators have also more readable abbreviated forms, either
for end users or for \TeX{} outputs. (All formal theorems and definitions
in this paper are generated from HOL4. Also, by default, all theorems are
fully specialised with outermost universal quantifiers removed.)

%t1 #&#
\begin{table}
%\tablewidth=
\caption{Syntax of some CCS concepts in HOL.}
 \label{tab:ccsoperator}
%
%
\begin{tabular*}{\tablewidth}{lllll}
\hline
CCS concept & Notation & HOL term &
                                             HOL abbrev. & \TeX{} outputs \\
\hline
nil & $\textbf{0}$ & \texttt{nil} & \texttt{nil} & \HOLinline{\HOLConst{\ensuremath{\mathbf{0}}}} \\
prefix & $\mu .P$ & \texttt{prefix u P} & \texttt{u..P}
& \HOLinline{\HOLFreeVar{u}\HOLSymConst{\ensuremath{\ldotp}}\HOLFreeVar{P}} \\
summation & $P + Q$ & \texttt{sum P Q} & \texttt{P + Q}
& \HOLinline{\HOLFreeVar{P} \HOLSymConst{\ensuremath{+}} \HOLFreeVar{Q}} \\
parallel composition & $P \,\mid \, Q$ & \texttt{par P Q} & \texttt{P || Q} & \HOLinline{\HOLFreeVar{P} \HOLSymConst{\ensuremath{\mid}} \HOLFreeVar{Q}} \\
restriction & $(\nu{}L)\;P$ & \texttt{restr L P} & \texttt{(nu L) P} & \HOLinline{\ensuremath{(\nu}\HOLFreeVar{L}\ensuremath{)} \HOLFreeVar{P}}  \\
recursion & $\recu A P$ & \texttt{rec A P} & \texttt{rec A P} & \HOLinline{\HOLConst{rec} \HOLFreeVar{A} \HOLFreeVar{P}}  \\
relabeling & $P\;[r\!f]$ & \texttt{relab P rf} &\texttt{relab P rf} & \HOLinline{\HOLConst{relab} \HOLFreeVar{P} \HOLFreeVar{rf}}  \\
constant/variable & $A$ & \texttt{var A} &\texttt{var A} & \HOLinline{\HOLConst{var} \HOLFreeVar{A}} \\
[\tvspace{6pt}]
invisible action & $\tau $ & \texttt{tau} & \texttt{tau} & \HOLinline{\HOLSymConst{\ensuremath{\tau}}} \\
input action & $a$ & \texttt{label (name a)} &\texttt{In(a)} & \HOLinline{\HOLConst{\HOLTokenInputAct} \HOLFreeVar{a}} \\
output action & $\outC a$ & \texttt{label (coname a)} & \texttt{Out(a)} & \HOLinline{\HOLConst{\HOLTokenOutputAct} \HOLFreeVar{a}} \\
[\tvspace{6pt}]
variable substitution & $E\{E'/X\}$ & \texttt{CCS\_Subst E E' X} &
                                           \texttt{[E'/X]  E} & \HOLinline{\ensuremath{[}\ensuremath{\HOLFreeVar{E}\sp{\prime}}\ensuremath{/}\HOLFreeVar{X}\ensuremath{]} \HOLFreeVar{E}} \\
transition & $P\overset{\mu}{\longrightarrow}Q$
                       & \texttt{TRANS P u Q} & \texttt{P ---u-> Q} & \HOLinline{\HOLFreeVar{P} \HOLTokenTransBegin \HOLFreeVar{u}\HOLTokenTransEnd \HOLFreeVar{Q}} \\
weak transition & $P\overset{\mu}{\Longrightarrow}Q$
                       & \texttt{WEAK\_TRANS P u Q} & \texttt{P ==u=> Q} & \HOLinline{\HOLFreeVar{P} \HOLTokenWeakTransBegin \HOLFreeVar{u}\HOLTokenWeakTransEnd \HOLFreeVar{Q}} \\
$\epsilon $--transition & $P\overset{\epsilon}{\Longrightarrow}Q$
                       & \texttt{EPS P Q} & \texttt{EPS P Q} & \HOLinline{\HOLFreeVar{P} \HOLSymConst{\HOLTokenEPS} \HOLFreeVar{Q}} \\
\hline
\end{tabular*}
%
%
\end{table}

The transition semantics of CCS processes strictly follows the SOS rules
given in \reftext{Fig.~\ref{f:LTSCCS}}. However, they are not axioms but consequences
of an \emph{inductive relation} definition of
\HOLinline{\HOLConst{TRANS}} by the \texttt{HOL\_reln} function of HOL4
(see~\citep[p.~219]{holdesc} for more details). The successful invocation
of the definitional principle returns three important theorems (\texttt{TRANS\_rules},
\texttt{TRANS\_ind} and \texttt{TRANS\_cases}):
%
\begin{itemize}
%
\item \texttt{TRANS\_rules} is a conjunction of implications (each corresponding
to a SOS rule) that will be the same as the input term. In fact, the following
formal versions of SOS rules are extracted from \texttt{TRANS\_rules}:
%
%a4.2 #&#
\begin{alltt}
\HOLTokenTurnstile{} \HOLFreeVar{u}\HOLSymConst{\ensuremath{\ldotp}}\HOLFreeVar{E} \HOLTokenTransBegin\HOLFreeVar{u}\HOLTokenTransEnd \HOLFreeVar{E}\hfill\texttt{[PREFIX]}
\HOLTokenTurnstile{} \HOLFreeVar{E} \HOLTokenTransBegin\HOLFreeVar{u}\HOLTokenTransEnd \ensuremath{\HOLFreeVar{E}\sb{\mathrm{1}}} \HOLSymConst{\HOLTokenImp{}} \HOLFreeVar{E} \HOLSymConst{\ensuremath{+}} \ensuremath{\HOLFreeVar{E}\sp{\prime}} \HOLTokenTransBegin\HOLFreeVar{u}\HOLTokenTransEnd \ensuremath{\HOLFreeVar{E}\sb{\mathrm{1}}}\hfill\texttt{[SUM1]}
\HOLTokenTurnstile{} \HOLFreeVar{E} \HOLTokenTransBegin\HOLFreeVar{u}\HOLTokenTransEnd \ensuremath{\HOLFreeVar{E}\sb{\mathrm{1}}} \HOLSymConst{\HOLTokenImp{}} \ensuremath{\HOLFreeVar{E}\sp{\prime}} \HOLSymConst{\ensuremath{+}} \HOLFreeVar{E} \HOLTokenTransBegin\HOLFreeVar{u}\HOLTokenTransEnd \ensuremath{\HOLFreeVar{E}\sb{\mathrm{1}}}\hfill\texttt{[SUM2]}
\HOLTokenTurnstile{} \HOLFreeVar{E} \HOLTokenTransBegin\HOLFreeVar{u}\HOLTokenTransEnd \ensuremath{\HOLFreeVar{E}\sb{\mathrm{1}}} \HOLSymConst{\HOLTokenImp{}} \HOLFreeVar{E} \HOLSymConst{\ensuremath{\mid}} \ensuremath{\HOLFreeVar{E}\sp{\prime}} \HOLTokenTransBegin\HOLFreeVar{u}\HOLTokenTransEnd \ensuremath{\HOLFreeVar{E}\sb{\mathrm{1}}} \HOLSymConst{\ensuremath{\mid}} \ensuremath{\HOLFreeVar{E}\sp{\prime}}\hfill\texttt{[PAR1]}
\HOLTokenTurnstile{} \HOLFreeVar{E} \HOLTokenTransBegin\HOLFreeVar{u}\HOLTokenTransEnd \ensuremath{\HOLFreeVar{E}\sb{\mathrm{1}}} \HOLSymConst{\HOLTokenImp{}} \ensuremath{\HOLFreeVar{E}\sp{\prime}} \HOLSymConst{\ensuremath{\mid}} \HOLFreeVar{E} \HOLTokenTransBegin\HOLFreeVar{u}\HOLTokenTransEnd \ensuremath{\HOLFreeVar{E}\sp{\prime}} \HOLSymConst{\ensuremath{\mid}} \ensuremath{\HOLFreeVar{E}\sb{\mathrm{1}}}\hfill\texttt{[PAR2]}
\end{alltt}
%
\begin{equation*}
\infer[\HOLRuleName{PAR3}]{\HOLinline{\HOLFreeVar{E} \HOLSymConst{\ensuremath{\mid}} \ensuremath{\HOLFreeVar{E}\sp{\prime}} \HOLTokenTransBegin \HOLSymConst{\ensuremath{\tau}}\HOLTokenTransEnd \ensuremath{\HOLFreeVar{E}\sb{\mathrm{1}}} \HOLSymConst{\ensuremath{\mid}} \ensuremath{\HOLFreeVar{E}\sb{\mathrm{2}}}}}{\HOLinline{\HOLFreeVar{E} \HOLTokenTransBegin \HOLConst{label} \HOLFreeVar{l}\HOLTokenTransEnd \ensuremath{\HOLFreeVar{E}\sb{\mathrm{1}}}}&\HOLinline{\ensuremath{\HOLFreeVar{E}\sp{\prime}} \HOLTokenTransBegin \HOLConst{label} \ensuremath{(}\HOLConst{COMPL} \HOLFreeVar{l}\ensuremath{)}\HOLTokenTransEnd \ensuremath{\HOLFreeVar{E}\sb{\mathrm{2}}}}}
\end{equation*}
%
\begin{equation*}
\infer[\HOLRuleName{RESTR}]{\HOLinline{\ensuremath{(\nu}\HOLFreeVar{L}\ensuremath{)} \HOLFreeVar{E} \HOLTokenTransBegin \HOLFreeVar{u}\HOLTokenTransEnd \ensuremath{(\nu}\HOLFreeVar{L}\ensuremath{)} \ensuremath{\HOLFreeVar{E}\sp{\prime}}}}{\HOLinline{\HOLFreeVar{E} \HOLTokenTransBegin \HOLFreeVar{u}\HOLTokenTransEnd \ensuremath{\HOLFreeVar{E}\sp{\prime}}}&\HOLinline{\HOLFreeVar{u} \HOLSymConst{\ensuremath{=}} \HOLSymConst{\ensuremath{\tau}} \HOLSymConst{\HOLTokenDisj{}} \HOLFreeVar{u} \HOLSymConst{\ensuremath{=}} \HOLConst{label} \HOLFreeVar{l} \HOLSymConst{\HOLTokenConj{}} \HOLFreeVar{l} \HOLSymConst{\HOLTokenNotIn{}} \HOLFreeVar{L} \HOLSymConst{\HOLTokenConj{}} \HOLConst{COMPL} \HOLFreeVar{l} \HOLSymConst{\HOLTokenNotIn{}} \HOLFreeVar{L}}}
\end{equation*}
%
\begin{equation*}
\infer[\HOLRuleName{RELABELING}]{\HOLinline{\HOLConst{relab} \HOLFreeVar{E} \HOLFreeVar{rf} \HOLTokenTransBegin \HOLConst{relabel} \HOLFreeVar{rf} \HOLFreeVar{u}\HOLTokenTransEnd \HOLConst{relab} \ensuremath{\HOLFreeVar{E}\sp{\prime}} \HOLFreeVar{rf}}}{\HOLinline{\HOLFreeVar{E} \HOLTokenTransBegin \HOLFreeVar{u}\HOLTokenTransEnd \ensuremath{\HOLFreeVar{E}\sp{\prime}}}}
\end{equation*}
%
\begin{equation*}
\infer[\HOLRuleName{REC}]{\HOLinline{\HOLConst{rec} \HOLFreeVar{X} \HOLFreeVar{E} \HOLTokenTransBegin \HOLFreeVar{u}\HOLTokenTransEnd \ensuremath{\HOLFreeVar{E}\sb{\mathrm{1}}}}}{\HOLinline{\ensuremath{[}\HOLConst{rec} \HOLFreeVar{X} \HOLFreeVar{E}\ensuremath{/}\HOLFreeVar{X}\ensuremath{]} \HOLFreeVar{E} \HOLTokenTransBegin \HOLFreeVar{u}\HOLTokenTransEnd \ensuremath{\HOLFreeVar{E}\sb{\mathrm{1}}}}}
\end{equation*}
%
\item \texttt{TRANS\_ind} is the induction principle for the relation (see
Section~\ref{sec:multivariate} for its exact statement and an application
in the proof of \reftext{Proposition~\ref{prop:transFV}}).
%
\item \texttt{TRANS\_cases} is the so-called `cases' or `inversion' theorem
for the relation, and is used to decompose an element in the relation into
the possible ways of obtaining it by the rules:
%
\begin{alltt}
\HOLTokenTurnstile{} \HOLSymConst{\HOLTokenForall{}}\ensuremath{\HOLBoundVar{a}\sb{\mathrm{0}}} \ensuremath{\HOLBoundVar{a}\sb{\mathrm{1}}} \ensuremath{\HOLBoundVar{a}\sb{\mathrm{2}}}.
       \ensuremath{\HOLBoundVar{a}\sb{\mathrm{0}}} \HOLTokenTransBegin\ensuremath{\HOLBoundVar{a}\sb{\mathrm{1}}}\HOLTokenTransEnd \ensuremath{\HOLBoundVar{a}\sb{\mathrm{2}}} \HOLSymConst{\HOLTokenEquiv{}}
       \ensuremath{\HOLBoundVar{a}\sb{\mathrm{0}}} \HOLSymConst{\ensuremath{=}} \ensuremath{\HOLBoundVar{a}\sb{\mathrm{1}}}\HOLSymConst{\ensuremath{\ldotp}}\ensuremath{\HOLBoundVar{a}\sb{\mathrm{2}}} \HOLSymConst{\HOLTokenDisj{}} \ensuremath{(}\HOLSymConst{\HOLTokenExists{}}\HOLBoundVar{E} \ensuremath{\HOLBoundVar{E}\sp{\prime}}. \ensuremath{\HOLBoundVar{a}\sb{\mathrm{0}}} \HOLSymConst{\ensuremath{=}} \HOLBoundVar{E} \HOLSymConst{\ensuremath{+}} \ensuremath{\HOLBoundVar{E}\sp{\prime}} \HOLSymConst{\HOLTokenConj{}} \HOLBoundVar{E} \HOLTokenTransBegin\ensuremath{\HOLBoundVar{a}\sb{\mathrm{1}}}\HOLTokenTransEnd \ensuremath{\HOLBoundVar{a}\sb{\mathrm{2}}}\ensuremath{)} \HOLSymConst{\HOLTokenDisj{}}
       \ensuremath{(}\HOLSymConst{\HOLTokenExists{}}\HOLBoundVar{E} \ensuremath{\HOLBoundVar{E}\sp{\prime}}. \ensuremath{\HOLBoundVar{a}\sb{\mathrm{0}}} \HOLSymConst{\ensuremath{=}} \ensuremath{\HOLBoundVar{E}\sp{\prime}} \HOLSymConst{\ensuremath{+}} \HOLBoundVar{E} \HOLSymConst{\HOLTokenConj{}} \HOLBoundVar{E} \HOLTokenTransBegin\ensuremath{\HOLBoundVar{a}\sb{\mathrm{1}}}\HOLTokenTransEnd \ensuremath{\HOLBoundVar{a}\sb{\mathrm{2}}}\ensuremath{)} \HOLSymConst{\HOLTokenDisj{}}
       \ensuremath{(}\HOLSymConst{\HOLTokenExists{}}\HOLBoundVar{E} \ensuremath{\HOLBoundVar{E}\sb{\mathrm{1}}} \ensuremath{\HOLBoundVar{E}\sp{\prime}}. \ensuremath{\HOLBoundVar{a}\sb{\mathrm{0}}} \HOLSymConst{\ensuremath{=}} \HOLBoundVar{E} \HOLSymConst{\ensuremath{\mid}} \ensuremath{\HOLBoundVar{E}\sp{\prime}} \HOLSymConst{\HOLTokenConj{}} \ensuremath{\HOLBoundVar{a}\sb{\mathrm{2}}} \HOLSymConst{\ensuremath{=}} \ensuremath{\HOLBoundVar{E}\sb{\mathrm{1}}} \HOLSymConst{\ensuremath{\mid}} \ensuremath{\HOLBoundVar{E}\sp{\prime}} \HOLSymConst{\HOLTokenConj{}} \HOLBoundVar{E} \HOLTokenTransBegin\ensuremath{\HOLBoundVar{a}\sb{\mathrm{1}}}\HOLTokenTransEnd \ensuremath{\HOLBoundVar{E}\sb{\mathrm{1}}}\ensuremath{)} \HOLSymConst{\HOLTokenDisj{}}
       \ensuremath{(}\HOLSymConst{\HOLTokenExists{}}\HOLBoundVar{E} \ensuremath{\HOLBoundVar{E}\sb{\mathrm{1}}} \ensuremath{\HOLBoundVar{E}\sp{\prime}}. \ensuremath{\HOLBoundVar{a}\sb{\mathrm{0}}} \HOLSymConst{\ensuremath{=}} \ensuremath{\HOLBoundVar{E}\sp{\prime}} \HOLSymConst{\ensuremath{\mid}} \HOLBoundVar{E} \HOLSymConst{\HOLTokenConj{}} \ensuremath{\HOLBoundVar{a}\sb{\mathrm{2}}} \HOLSymConst{\ensuremath{=}} \ensuremath{\HOLBoundVar{E}\sp{\prime}} \HOLSymConst{\ensuremath{\mid}} \ensuremath{\HOLBoundVar{E}\sb{\mathrm{1}}} \HOLSymConst{\HOLTokenConj{}} \HOLBoundVar{E} \HOLTokenTransBegin\ensuremath{\HOLBoundVar{a}\sb{\mathrm{1}}}\HOLTokenTransEnd \ensuremath{\HOLBoundVar{E}\sb{\mathrm{1}}}\ensuremath{)} \HOLSymConst{\HOLTokenDisj{}}
       \ensuremath{(}\HOLSymConst{\HOLTokenExists{}}\HOLBoundVar{E} \HOLBoundVar{l} \ensuremath{\HOLBoundVar{E}\sb{\mathrm{1}}} \ensuremath{\HOLBoundVar{E}\sp{\prime}} \ensuremath{\HOLBoundVar{E}\sb{\mathrm{2}}}.
            \ensuremath{\HOLBoundVar{a}\sb{\mathrm{0}}} \HOLSymConst{\ensuremath{=}} \HOLBoundVar{E} \HOLSymConst{\ensuremath{\mid}} \ensuremath{\HOLBoundVar{E}\sp{\prime}} \HOLSymConst{\HOLTokenConj{}} \ensuremath{\HOLBoundVar{a}\sb{\mathrm{1}}} \HOLSymConst{\ensuremath{=}} \HOLSymConst{\ensuremath{\tau}} \HOLSymConst{\HOLTokenConj{}} \ensuremath{\HOLBoundVar{a}\sb{\mathrm{2}}} \HOLSymConst{\ensuremath{=}} \ensuremath{\HOLBoundVar{E}\sb{\mathrm{1}}} \HOLSymConst{\ensuremath{\mid}} \ensuremath{\HOLBoundVar{E}\sb{\mathrm{2}}} \HOLSymConst{\HOLTokenConj{}} \HOLBoundVar{E} \HOLTokenTransBegin\HOLConst{label} \HOLBoundVar{l}\HOLTokenTransEnd \ensuremath{\HOLBoundVar{E}\sb{\mathrm{1}}} \HOLSymConst{\HOLTokenConj{}}
            \ensuremath{\HOLBoundVar{E}\sp{\prime}} \HOLTokenTransBegin\HOLConst{label} \ensuremath{(}\HOLConst{COMPL} \HOLBoundVar{l}\ensuremath{)}\HOLTokenTransEnd \ensuremath{\HOLBoundVar{E}\sb{\mathrm{2}}}\ensuremath{)} \HOLSymConst{\HOLTokenDisj{}}
       \ensuremath{(}\HOLSymConst{\HOLTokenExists{}}\HOLBoundVar{E} \ensuremath{\HOLBoundVar{E}\sp{\prime}} \HOLBoundVar{l} \HOLBoundVar{L}.
            \ensuremath{\HOLBoundVar{a}\sb{\mathrm{0}}} \HOLSymConst{\ensuremath{=}} \ensuremath{(\nu}\HOLBoundVar{L}\ensuremath{)} \HOLBoundVar{E} \HOLSymConst{\HOLTokenConj{}} \ensuremath{\HOLBoundVar{a}\sb{\mathrm{2}}} \HOLSymConst{\ensuremath{=}} \ensuremath{(\nu}\HOLBoundVar{L}\ensuremath{)} \ensuremath{\HOLBoundVar{E}\sp{\prime}} \HOLSymConst{\HOLTokenConj{}} \HOLBoundVar{E} \HOLTokenTransBegin\ensuremath{\HOLBoundVar{a}\sb{\mathrm{1}}}\HOLTokenTransEnd \ensuremath{\HOLBoundVar{E}\sp{\prime}} \HOLSymConst{\HOLTokenConj{}}
            \ensuremath{(}\ensuremath{\HOLBoundVar{a}\sb{\mathrm{1}}} \HOLSymConst{\ensuremath{=}} \HOLSymConst{\ensuremath{\tau}} \HOLSymConst{\HOLTokenDisj{}} \ensuremath{\HOLBoundVar{a}\sb{\mathrm{1}}} \HOLSymConst{\ensuremath{=}} \HOLConst{label} \HOLBoundVar{l} \HOLSymConst{\HOLTokenConj{}} \HOLBoundVar{l} \HOLSymConst{\HOLTokenNotIn{}} \HOLBoundVar{L} \HOLSymConst{\HOLTokenConj{}} \HOLConst{COMPL} \HOLBoundVar{l} \HOLSymConst{\HOLTokenNotIn{}} \HOLBoundVar{L}\ensuremath{)}\ensuremath{)} \HOLSymConst{\HOLTokenDisj{}}
       \ensuremath{(}\HOLSymConst{\HOLTokenExists{}}\HOLBoundVar{E} \HOLBoundVar{u} \ensuremath{\HOLBoundVar{E}\sp{\prime}} \HOLBoundVar{rf}.
            \ensuremath{\HOLBoundVar{a}\sb{\mathrm{0}}} \HOLSymConst{\ensuremath{=}} \HOLConst{relab} \HOLBoundVar{E} \HOLBoundVar{rf} \HOLSymConst{\HOLTokenConj{}} \ensuremath{\HOLBoundVar{a}\sb{\mathrm{1}}} \HOLSymConst{\ensuremath{=}} \HOLConst{relabel} \HOLBoundVar{rf} \HOLBoundVar{u} \HOLSymConst{\HOLTokenConj{}} \ensuremath{\HOLBoundVar{a}\sb{\mathrm{2}}} \HOLSymConst{\ensuremath{=}} \HOLConst{relab} \ensuremath{\HOLBoundVar{E}\sp{\prime}} \HOLBoundVar{rf} \HOLSymConst{\HOLTokenConj{}}
            \HOLBoundVar{E} \HOLTokenTransBegin\HOLBoundVar{u}\HOLTokenTransEnd \ensuremath{\HOLBoundVar{E}\sp{\prime}}\ensuremath{)} \HOLSymConst{\HOLTokenDisj{}} \HOLSymConst{\HOLTokenExists{}}\HOLBoundVar{E} \HOLBoundVar{X}. \ensuremath{\HOLBoundVar{a}\sb{\mathrm{0}}} \HOLSymConst{\ensuremath{=}} \HOLConst{rec} \HOLBoundVar{X} \HOLBoundVar{E} \HOLSymConst{\HOLTokenConj{}} \ensuremath{[}\HOLConst{rec} \HOLBoundVar{X} \HOLBoundVar{E}\ensuremath{/}\HOLBoundVar{X}\ensuremath{]} \HOLBoundVar{E} \HOLTokenTransBegin\ensuremath{\HOLBoundVar{a}\sb{\mathrm{1}}}\HOLTokenTransEnd \ensuremath{\HOLBoundVar{a}\sb{\mathrm{2}}}
\end{alltt}
%
For instance, the following proposition requires \texttt{TRANS\_cases}:
%
%a4.2 #&#
\begin{alltt}
\HOLTokenTurnstile{} \HOLFreeVar{E} \HOLSymConst{\ensuremath{+}} \ensuremath{\HOLFreeVar{E}\sp{\prime}} \HOLTokenTransBegin\HOLFreeVar{u}\HOLTokenTransEnd \ensuremath{\HOLFreeVar{E}\sp{\Prime}} \HOLSymConst{\HOLTokenEquiv{}} \HOLFreeVar{E} \HOLTokenTransBegin\HOLFreeVar{u}\HOLTokenTransEnd \ensuremath{\HOLFreeVar{E}\sp{\Prime}} \HOLSymConst{\HOLTokenDisj{}} \ensuremath{\HOLFreeVar{E}\sp{\prime}} \HOLTokenTransBegin\HOLFreeVar{u}\HOLTokenTransEnd \ensuremath{\HOLFreeVar{E}\sp{\Prime}}\hfill{[TRANS\_SUM\_EQ]}
\end{alltt}
%
\end{itemize}

In particular, the SOS rule \texttt{REC} (Recursion) says that if we substitute
all occurrences of the variable $A$ in $P$ to $(\recu A P)$ and the resulting
process has a transition to $P'$ with an action $u$, then
$(\recu A P)$ has the same transition. Here ``\HOLinline{\ensuremath{[}\HOLConst{rec}\;\;\HOLFreeVar{X}\;\;\HOLFreeVar{E}\ensuremath{/}\HOLFreeVar{X}\ensuremath{]}\;\;\HOLFreeVar{E}}''
is an abbreviation for ``\HOLinline{\HOLConst{CCS\_Subst}\;\;\HOLFreeVar{E}\;\;\ensuremath{(}\HOLConst{rec}\;\;\HOLFreeVar{X}\;\;\HOLFreeVar{E}\ensuremath{)}\;\;\HOLFreeVar{X}}'',
where \HOLinline{\HOLConst{CCS\_Subst}} is a recursive function substituting
all occurrences of a free variable with a CCS term. For most CCS operators
\HOLinline{\HOLConst{CCS\_Subst}} just recursively goes into a deeper level
without changing anything, e.g.:
%
%a4.2 #&#
\begin{alltt}
\HOLTokenTurnstile{} \ensuremath{[}\ensuremath{\HOLFreeVar{E}\sp{\prime}}\ensuremath{/}\HOLFreeVar{X}\ensuremath{]} \ensuremath{(}\ensuremath{\HOLFreeVar{E}\sb{\mathrm{1}}} \HOLSymConst{\ensuremath{+}} \ensuremath{\HOLFreeVar{E}\sb{\mathrm{2}}}\ensuremath{)} \HOLSymConst{\ensuremath{=}} \ensuremath{[}\ensuremath{\HOLFreeVar{E}\sp{\prime}}\ensuremath{/}\HOLFreeVar{X}\ensuremath{]} \ensuremath{\HOLFreeVar{E}\sb{\mathrm{1}}} \HOLSymConst{\ensuremath{+}} \ensuremath{[}\ensuremath{\HOLFreeVar{E}\sp{\prime}}\ensuremath{/}\HOLFreeVar{X}\ensuremath{]} \ensuremath{\HOLFreeVar{E}\sb{\mathrm{2}}}\hfill{[CCS\_Subst\_sum]}
\end{alltt}
%
The only two \xch{interesting}{insteresting} cases are those for agent variables and recursion:
%
%a4.2 #&#
\begin{alltt}
\HOLTokenTurnstile{} \ensuremath{[}\ensuremath{\HOLFreeVar{E}\sp{\prime}}\ensuremath{/}\HOLFreeVar{X}\ensuremath{]} \ensuremath{(}\HOLConst{var} \HOLFreeVar{Y}\ensuremath{)} \HOLSymConst{\ensuremath{=}} \HOLKeyword{if} \HOLFreeVar{Y} \HOLSymConst{\ensuremath{=}} \HOLFreeVar{X} \HOLKeyword{then} \ensuremath{\HOLFreeVar{E}\sp{\prime}} \HOLKeyword{else} \HOLConst{var} \HOLFreeVar{Y}\hfill{[CCS\_Subst\_var]}
\HOLTokenTurnstile{} \ensuremath{[}\ensuremath{\HOLFreeVar{E}\sp{\prime}}\ensuremath{/}\HOLFreeVar{X}\ensuremath{]} \ensuremath{(}\HOLConst{rec} \HOLFreeVar{Y} \HOLFreeVar{E}\ensuremath{)} \HOLSymConst{\ensuremath{=}} \HOLKeyword{if} \HOLFreeVar{Y} \HOLSymConst{\ensuremath{=}} \HOLFreeVar{X} \HOLKeyword{then} \HOLConst{rec} \HOLFreeVar{Y} \HOLFreeVar{E} \HOLKeyword{else} \HOLConst{rec} \HOLFreeVar{Y} \ensuremath{(}\ensuremath{[}\ensuremath{\HOLFreeVar{E}\sp{\prime}}\ensuremath{/}\HOLFreeVar{X}\ensuremath{]} \HOLFreeVar{E}\ensuremath{)}\hfill{[CCS\_Subst\_rec]}
\end{alltt}
%
Notice that variable substitutions only affect free variables. For instance,
the variable $Y$ in ``\HOLinline{\HOLConst{rec}\!\;\HOLFreeVar{Y}\!\;\HOLFreeVar{E}}''
is bound and therefore the substitution ignores it.

A useful facility exploiting the interplay between HOL4 and Standard ML
(which follows an idea of Nesi~\cite{Nesi:1992ve}) is a recursive ML function
that takes a CCS process and returns a theorem indicating all direct transitions
of the process. (If the input process is infinitely branching, the function
will not terminate, however.) For instance, we know that the process
$(a.\nil | \bar{a}.\nil )$ has three immediate derivatives given by the
following transitions:
$(a.\nil | \bar{a}.\nil ) \overset{a}{\longrightarrow} (\nil |
\bar{a}.\nil )$,
$(a.\nil | \bar{a}.\nil ) \overset{\bar{a}}{\longrightarrow} (a.\nil |
\nil )$ and
$(a.\nil | \bar{a}.\nil ) \overset{\tau}{\longrightarrow} (\nil |
\nil )$. To completely describe all possible transitions of the process,
the following two facts have to be proved: (1) there are indeed the three
transitions mentioned above; (2) there is no other transition. For large
CCS processes it is surprisingly tedious to manually derive all the possible
transitions and prove the non-existence of other transitions. This shows
the usefulness of appealing to an ML function
\texttt{CCS\_TRANS\_CONV} that is designed to automate the whole process.
For instance, taking the input $(a.\nil | \bar{a}.\nil )$ the function
returns the following theorem which describes all its one-step transitions:
%
%a4.2 #&#
\begin{alltt}
\HOLTokenTurnstile{} \HOLConst{\HOLTokenInputAct} \HOLStringLit{a}\HOLSymConst{\ensuremath{\ldotp}}\HOLConst{\ensuremath{\mathbf{0}}} \HOLSymConst{\ensuremath{\mid}} \HOLConst{\HOLTokenOutputAct} \HOLStringLit{a}\HOLSymConst{\ensuremath{\ldotp}}\HOLConst{\ensuremath{\mathbf{0}}} \HOLTokenTransBegin\HOLFreeVar{u}\HOLTokenTransEnd \HOLFreeVar{E} \HOLSymConst{\HOLTokenEquiv{}}
   \ensuremath{(}\HOLFreeVar{u} \HOLSymConst{\ensuremath{=}} \HOLConst{\HOLTokenInputAct} \HOLStringLit{a} \HOLSymConst{\HOLTokenConj{}} \HOLFreeVar{E} \HOLSymConst{\ensuremath{=}} \HOLConst{\ensuremath{\mathbf{0}}} \HOLSymConst{\ensuremath{\mid}} \HOLConst{\HOLTokenOutputAct} \HOLStringLit{a}\HOLSymConst{\ensuremath{\ldotp}}\HOLConst{\ensuremath{\mathbf{0}}} \HOLSymConst{\HOLTokenDisj{}} \HOLFreeVar{u} \HOLSymConst{\ensuremath{=}} \HOLConst{\HOLTokenOutputAct} \HOLStringLit{a} \HOLSymConst{\HOLTokenConj{}} \HOLFreeVar{E} \HOLSymConst{\ensuremath{=}} \HOLConst{\HOLTokenInputAct} \HOLStringLit{a}\HOLSymConst{\ensuremath{\ldotp}}\HOLConst{\ensuremath{\mathbf{0}}} \HOLSymConst{\ensuremath{\mid}} \HOLConst{\ensuremath{\mathbf{0}}}\ensuremath{)} \HOLSymConst{\HOLTokenDisj{}}
   \HOLFreeVar{u} \HOLSymConst{\ensuremath{=}} \HOLSymConst{\ensuremath{\tau}} \HOLSymConst{\HOLTokenConj{}} \HOLFreeVar{E} \HOLSymConst{\ensuremath{=}} \HOLConst{\ensuremath{\mathbf{0}}} \HOLSymConst{\ensuremath{\mid}} \HOLConst{\ensuremath{\mathbf{0}}}
\end{alltt}

%s4.3 #&#
\subsection{Bisimulation and bisimilarity}
%%LEAP%%%\label{sec4.3}
 \label{ss:bb}

A highlight of this CCS formalisation is the simplified definitions of
bisimilarities using the new coinduction package of HOL4. Without this
package, \xch{bisimilarities}{bisimilaries} can still be defined in HOL, but proving their properties
would be more tedious and complicated~\citep[p.~91]{Mil89}. Below we briefly
describe how weak bisimulation and weak bisimilarity are defined in HOL.
Strong bisimulation and strong bisimilarity, as well as other concepts
such as expansion and contraction, can be defined in the same manner.

To define (weak) bisimilarity, first we need to define weak transitions
of CCS processes. A (possibly empty) sequence of $\tau $-transitions between
two processes is defined as a new binary relation
\HOLinline{\HOLConst{EPS}} ($\overset{\epsilon}{\Longrightarrow}$), which
is the reflexive and transitive closure (RTC, denoted by a superscript
$^{*}$) of ordinary $\tau $-transitions of CCS processes:
%
%a4.3 #&#
\begin{alltt}
   \HOLConst{EPS} \HOLTokenDefEquality{} \ensuremath{(}\HOLTokenLambda{}\HOLBoundVar{E} \ensuremath{\HOLBoundVar{E}\sp{\prime}}. \HOLBoundVar{E} \HOLTokenTransBegin\HOLSymConst{\ensuremath{\tau}}\HOLTokenTransEnd \ensuremath{\HOLBoundVar{E}\sp{\prime}}\ensuremath{)}\HOLSymConst{\HOLTokenSupStar{}}\hfill{[EPS\_def]}
\end{alltt}
%
Then we can define a weak transition as an ordinary transition wrapped
by two $\epsilon $-transitions:
%
%a4.3 #&#
\begin{alltt}
   \HOLFreeVar{E} \HOLTokenWeakTransBegin\HOLFreeVar{u}\HOLTokenWeakTransEnd \ensuremath{\HOLFreeVar{E}\sp{\prime}} \HOLTokenDefEquality{} \HOLSymConst{\HOLTokenExists{}}\ensuremath{\HOLBoundVar{E}\sb{\mathrm{1}}} \ensuremath{\HOLBoundVar{E}\sb{\mathrm{2}}}. \HOLFreeVar{E} \HOLSymConst{\HOLTokenEPS} \ensuremath{\HOLBoundVar{E}\sb{\mathrm{1}}} \HOLSymConst{\HOLTokenConj{}} \ensuremath{\HOLBoundVar{E}\sb{\mathrm{1}}} \HOLTokenTransBegin\HOLFreeVar{u}\HOLTokenTransEnd \ensuremath{\HOLBoundVar{E}\sb{\mathrm{2}}} \HOLSymConst{\HOLTokenConj{}} \ensuremath{\HOLBoundVar{E}\sb{\mathrm{2}}} \HOLSymConst{\HOLTokenEPS} \ensuremath{\HOLFreeVar{E}\sp{\prime}}\hfill{[WEAK\_TRANS]}
\end{alltt}
%
The definition of weak bisimulation is then based on weak and
$\epsilon $--transitions:
%
%a4.3 #&#
\begin{alltt}
   \HOLConst{WEAK\_BISIM} \HOLFreeVar{Wbsm} \HOLTokenDefEquality{}
     \HOLSymConst{\HOLTokenForall{}}\HOLBoundVar{E} \ensuremath{\HOLBoundVar{E}\sp{\prime}}.
         \HOLFreeVar{Wbsm} \HOLBoundVar{E} \ensuremath{\HOLBoundVar{E}\sp{\prime}} \HOLSymConst{\HOLTokenImp{}}
         \ensuremath{(}\HOLSymConst{\HOLTokenForall{}}\HOLBoundVar{l}.
              \ensuremath{(}\HOLSymConst{\HOLTokenForall{}}\ensuremath{\HOLBoundVar{E}\sb{\mathrm{1}}}. \HOLBoundVar{E} \HOLTokenTransBegin\HOLConst{label} \HOLBoundVar{l}\HOLTokenTransEnd \ensuremath{\HOLBoundVar{E}\sb{\mathrm{1}}} \HOLSymConst{\HOLTokenImp{}} \HOLSymConst{\HOLTokenExists{}}\ensuremath{\HOLBoundVar{E}\sb{\mathrm{2}}}. \ensuremath{\HOLBoundVar{E}\sp{\prime}} \HOLTokenWeakTransBegin\HOLConst{label} \HOLBoundVar{l}\HOLTokenWeakTransEnd \ensuremath{\HOLBoundVar{E}\sb{\mathrm{2}}} \HOLSymConst{\HOLTokenConj{}} \HOLFreeVar{Wbsm} \ensuremath{\HOLBoundVar{E}\sb{\mathrm{1}}} \ensuremath{\HOLBoundVar{E}\sb{\mathrm{2}}}\ensuremath{)} \HOLSymConst{\HOLTokenConj{}}
              \HOLSymConst{\HOLTokenForall{}}\ensuremath{\HOLBoundVar{E}\sb{\mathrm{2}}}. \ensuremath{\HOLBoundVar{E}\sp{\prime}} \HOLTokenTransBegin\HOLConst{label} \HOLBoundVar{l}\HOLTokenTransEnd \ensuremath{\HOLBoundVar{E}\sb{\mathrm{2}}} \HOLSymConst{\HOLTokenImp{}} \HOLSymConst{\HOLTokenExists{}}\ensuremath{\HOLBoundVar{E}\sb{\mathrm{1}}}. \HOLBoundVar{E} \HOLTokenWeakTransBegin\HOLConst{label} \HOLBoundVar{l}\HOLTokenWeakTransEnd \ensuremath{\HOLBoundVar{E}\sb{\mathrm{1}}} \HOLSymConst{\HOLTokenConj{}} \HOLFreeVar{Wbsm} \ensuremath{\HOLBoundVar{E}\sb{\mathrm{1}}} \ensuremath{\HOLBoundVar{E}\sb{\mathrm{2}}}\ensuremath{)} \HOLSymConst{\HOLTokenConj{}}
         \ensuremath{(}\HOLSymConst{\HOLTokenForall{}}\ensuremath{\HOLBoundVar{E}\sb{\mathrm{1}}}. \HOLBoundVar{E} \HOLTokenTransBegin\HOLSymConst{\ensuremath{\tau}}\HOLTokenTransEnd \ensuremath{\HOLBoundVar{E}\sb{\mathrm{1}}} \HOLSymConst{\HOLTokenImp{}} \HOLSymConst{\HOLTokenExists{}}\ensuremath{\HOLBoundVar{E}\sb{\mathrm{2}}}. \ensuremath{\HOLBoundVar{E}\sp{\prime}} \HOLSymConst{\HOLTokenEPS} \ensuremath{\HOLBoundVar{E}\sb{\mathrm{2}}} \HOLSymConst{\HOLTokenConj{}} \HOLFreeVar{Wbsm} \ensuremath{\HOLBoundVar{E}\sb{\mathrm{1}}} \ensuremath{\HOLBoundVar{E}\sb{\mathrm{2}}}\ensuremath{)} \HOLSymConst{\HOLTokenConj{}}
         \HOLSymConst{\HOLTokenForall{}}\ensuremath{\HOLBoundVar{E}\sb{\mathrm{2}}}. \ensuremath{\HOLBoundVar{E}\sp{\prime}} \HOLTokenTransBegin\HOLSymConst{\ensuremath{\tau}}\HOLTokenTransEnd \ensuremath{\HOLBoundVar{E}\sb{\mathrm{2}}} \HOLSymConst{\HOLTokenImp{}} \HOLSymConst{\HOLTokenExists{}}\ensuremath{\HOLBoundVar{E}\sb{\mathrm{1}}}. \HOLBoundVar{E} \HOLSymConst{\HOLTokenEPS} \ensuremath{\HOLBoundVar{E}\sb{\mathrm{1}}} \HOLSymConst{\HOLTokenConj{}} \HOLFreeVar{Wbsm} \ensuremath{\HOLBoundVar{E}\sb{\mathrm{1}}} \ensuremath{\HOLBoundVar{E}\sb{\mathrm{2}}}\hfill{[WEAK\_BISIM]}
\end{alltt}

We can prove, for example, that the identity relation (\HOLinline{\HOLTokenLambda{}\HOLBoundVar{x}\;\!\HOLBoundVar{y}.\;\!\HOLBoundVar{x}\;\!\HOLSymConst{\ensuremath{=}}\;\!\HOLBoundVar{y}})
is indeed a bisimulation, and that bisimulation is preserved by inversion,
composition, and union operations. Weak bisimilarity can be then defined
in HOL4 by the following code:
%
\begin{lstlisting}
CoInductive WEAK_EQUIV :
    !(E :('a, 'b) CCS) (E' :('a, 'b) CCS).
       (!l.
         (!E1. TRANS E  (label l) E1 ==>
               (?E2. WEAK_TRANS E' (label l) E2 /\ WEAK_EQUIV E1 E2)) /\
         (!E2. TRANS E' (label l) E2 ==>
               (?E1. WEAK_TRANS E  (label l) E1 /\ WEAK_EQUIV E1 E2))) /\
       (!E1. TRANS E  tau E1 ==> (?E2. EPS E' E2 /\ WEAK_EQUIV E1 E2)) /\
       (!E2. TRANS E' tau E2 ==> (?E1. EPS E  E1 /\ WEAK_EQUIV E1 E2))
      ==> WEAK_EQUIV E E'
End
\end{lstlisting} Like the case of \HOLinline{\HOLConst{TRANS}}, the successful
invocation of the coinductive definitional principle returns three theorems
(\texttt{WEAK\_EQUIV\_rules}, \texttt{WEAK\_EQUIV\_coind} and
\texttt{WEAK\_EQUIV\_cases}):
%
\begin{itemize}
%
\item \texttt{WEAK\_EQUIV\_rules} is the same as the input term, which now
becomes a theorem:
%
%a4.2 #&#
\begin{alltt}
\HOLTokenTurnstile{} \ensuremath{(}\HOLSymConst{\HOLTokenForall{}}\HOLBoundVar{l}.
        \ensuremath{(}\HOLSymConst{\HOLTokenForall{}}\ensuremath{\HOLBoundVar{E}\sb{\mathrm{1}}}. \HOLFreeVar{E} \HOLTokenTransBegin\HOLConst{label} \HOLBoundVar{l}\HOLTokenTransEnd \ensuremath{\HOLBoundVar{E}\sb{\mathrm{1}}} \HOLSymConst{\HOLTokenImp{}} \HOLSymConst{\HOLTokenExists{}}\ensuremath{\HOLBoundVar{E}\sb{\mathrm{2}}}. \ensuremath{\HOLFreeVar{E}\sp{\prime}} \HOLTokenWeakTransBegin\HOLConst{label} \HOLBoundVar{l}\HOLTokenWeakTransEnd \ensuremath{\HOLBoundVar{E}\sb{\mathrm{2}}} \HOLSymConst{\HOLTokenConj{}} \ensuremath{\HOLBoundVar{E}\sb{\mathrm{1}}} \HOLSymConst{\HOLTokenWeakEQ} \ensuremath{\HOLBoundVar{E}\sb{\mathrm{2}}}\ensuremath{)} \HOLSymConst{\HOLTokenConj{}}
        \HOLSymConst{\HOLTokenForall{}}\ensuremath{\HOLBoundVar{E}\sb{\mathrm{2}}}. \ensuremath{\HOLFreeVar{E}\sp{\prime}} \HOLTokenTransBegin\HOLConst{label} \HOLBoundVar{l}\HOLTokenTransEnd \ensuremath{\HOLBoundVar{E}\sb{\mathrm{2}}} \HOLSymConst{\HOLTokenImp{}} \HOLSymConst{\HOLTokenExists{}}\ensuremath{\HOLBoundVar{E}\sb{\mathrm{1}}}. \HOLFreeVar{E} \HOLTokenWeakTransBegin\HOLConst{label} \HOLBoundVar{l}\HOLTokenWeakTransEnd \ensuremath{\HOLBoundVar{E}\sb{\mathrm{1}}} \HOLSymConst{\HOLTokenConj{}} \ensuremath{\HOLBoundVar{E}\sb{\mathrm{1}}} \HOLSymConst{\HOLTokenWeakEQ} \ensuremath{\HOLBoundVar{E}\sb{\mathrm{2}}}\ensuremath{)} \HOLSymConst{\HOLTokenConj{}}
   \ensuremath{(}\HOLSymConst{\HOLTokenForall{}}\ensuremath{\HOLBoundVar{E}\sb{\mathrm{1}}}. \HOLFreeVar{E} \HOLTokenTransBegin\HOLSymConst{\ensuremath{\tau}}\HOLTokenTransEnd \ensuremath{\HOLBoundVar{E}\sb{\mathrm{1}}} \HOLSymConst{\HOLTokenImp{}} \HOLSymConst{\HOLTokenExists{}}\ensuremath{\HOLBoundVar{E}\sb{\mathrm{2}}}. \ensuremath{\HOLFreeVar{E}\sp{\prime}} \HOLSymConst{\HOLTokenEPS} \ensuremath{\HOLBoundVar{E}\sb{\mathrm{2}}} \HOLSymConst{\HOLTokenConj{}} \ensuremath{\HOLBoundVar{E}\sb{\mathrm{1}}} \HOLSymConst{\HOLTokenWeakEQ} \ensuremath{\HOLBoundVar{E}\sb{\mathrm{2}}}\ensuremath{)} \HOLSymConst{\HOLTokenConj{}}
   \ensuremath{(}\HOLSymConst{\HOLTokenForall{}}\ensuremath{\HOLBoundVar{E}\sb{\mathrm{2}}}. \ensuremath{\HOLFreeVar{E}\sp{\prime}} \HOLTokenTransBegin\HOLSymConst{\ensuremath{\tau}}\HOLTokenTransEnd \ensuremath{\HOLBoundVar{E}\sb{\mathrm{2}}} \HOLSymConst{\HOLTokenImp{}} \HOLSymConst{\HOLTokenExists{}}\ensuremath{\HOLBoundVar{E}\sb{\mathrm{1}}}. \HOLFreeVar{E} \HOLSymConst{\HOLTokenEPS} \ensuremath{\HOLBoundVar{E}\sb{\mathrm{1}}} \HOLSymConst{\HOLTokenConj{}} \ensuremath{\HOLBoundVar{E}\sb{\mathrm{1}}} \HOLSymConst{\HOLTokenWeakEQ} \ensuremath{\HOLBoundVar{E}\sb{\mathrm{2}}}\ensuremath{)} \HOLSymConst{\HOLTokenImp{}}
   \HOLFreeVar{E} \HOLSymConst{\HOLTokenWeakEQ} \ensuremath{\HOLFreeVar{E}\sp{\prime}}
\end{alltt}
%
\item \texttt{WEAK\_EQUIV\_coind} is the coinduction principle for
\HOLinline{\HOLConst{WEAK\_EQUIV}} ($\wbvtex $):
%
%a4.2 #&#
\begin{alltt}
\HOLTokenTurnstile{} \ensuremath{(}\HOLSymConst{\HOLTokenForall{}}\ensuremath{\HOLBoundVar{a}\sb{\mathrm{0}}} \ensuremath{\HOLBoundVar{a}\sb{\mathrm{1}}}.
        \ensuremath{\HOLFreeVar{WEAK\HOLTokenUnderscore{}EQUIV}\sp{\prime}} \ensuremath{\HOLBoundVar{a}\sb{\mathrm{0}}} \ensuremath{\HOLBoundVar{a}\sb{\mathrm{1}}} \HOLSymConst{\HOLTokenImp{}}
        \ensuremath{(}\HOLSymConst{\HOLTokenForall{}}\HOLBoundVar{l}.
             \ensuremath{(}\HOLSymConst{\HOLTokenForall{}}\ensuremath{\HOLBoundVar{E}\sb{\mathrm{1}}}.
                  \ensuremath{\HOLBoundVar{a}\sb{\mathrm{0}}} \HOLTokenTransBegin\HOLConst{label} \HOLBoundVar{l}\HOLTokenTransEnd \ensuremath{\HOLBoundVar{E}\sb{\mathrm{1}}} \HOLSymConst{\HOLTokenImp{}}
                  \HOLSymConst{\HOLTokenExists{}}\ensuremath{\HOLBoundVar{E}\sb{\mathrm{2}}}. \ensuremath{\HOLBoundVar{a}\sb{\mathrm{1}}} \HOLTokenWeakTransBegin\HOLConst{label} \HOLBoundVar{l}\HOLTokenWeakTransEnd \ensuremath{\HOLBoundVar{E}\sb{\mathrm{2}}} \HOLSymConst{\HOLTokenConj{}} \ensuremath{\HOLFreeVar{WEAK\HOLTokenUnderscore{}EQUIV}\sp{\prime}} \ensuremath{\HOLBoundVar{E}\sb{\mathrm{1}}} \ensuremath{\HOLBoundVar{E}\sb{\mathrm{2}}}\ensuremath{)} \HOLSymConst{\HOLTokenConj{}}
             \HOLSymConst{\HOLTokenForall{}}\ensuremath{\HOLBoundVar{E}\sb{\mathrm{2}}}.
                 \ensuremath{\HOLBoundVar{a}\sb{\mathrm{1}}} \HOLTokenTransBegin\HOLConst{label} \HOLBoundVar{l}\HOLTokenTransEnd \ensuremath{\HOLBoundVar{E}\sb{\mathrm{2}}} \HOLSymConst{\HOLTokenImp{}} \HOLSymConst{\HOLTokenExists{}}\ensuremath{\HOLBoundVar{E}\sb{\mathrm{1}}}. \ensuremath{\HOLBoundVar{a}\sb{\mathrm{0}}} \HOLTokenWeakTransBegin\HOLConst{label} \HOLBoundVar{l}\HOLTokenWeakTransEnd \ensuremath{\HOLBoundVar{E}\sb{\mathrm{1}}} \HOLSymConst{\HOLTokenConj{}} \ensuremath{\HOLFreeVar{WEAK\HOLTokenUnderscore{}EQUIV}\sp{\prime}} \ensuremath{\HOLBoundVar{E}\sb{\mathrm{1}}} \ensuremath{\HOLBoundVar{E}\sb{\mathrm{2}}}\ensuremath{)} \HOLSymConst{\HOLTokenConj{}}
        \ensuremath{(}\HOLSymConst{\HOLTokenForall{}}\ensuremath{\HOLBoundVar{E}\sb{\mathrm{1}}}. \ensuremath{\HOLBoundVar{a}\sb{\mathrm{0}}} \HOLTokenTransBegin\HOLSymConst{\ensuremath{\tau}}\HOLTokenTransEnd \ensuremath{\HOLBoundVar{E}\sb{\mathrm{1}}} \HOLSymConst{\HOLTokenImp{}} \HOLSymConst{\HOLTokenExists{}}\ensuremath{\HOLBoundVar{E}\sb{\mathrm{2}}}. \ensuremath{\HOLBoundVar{a}\sb{\mathrm{1}}} \HOLSymConst{\HOLTokenEPS} \ensuremath{\HOLBoundVar{E}\sb{\mathrm{2}}} \HOLSymConst{\HOLTokenConj{}} \ensuremath{\HOLFreeVar{WEAK\HOLTokenUnderscore{}EQUIV}\sp{\prime}} \ensuremath{\HOLBoundVar{E}\sb{\mathrm{1}}} \ensuremath{\HOLBoundVar{E}\sb{\mathrm{2}}}\ensuremath{)} \HOLSymConst{\HOLTokenConj{}}
        \HOLSymConst{\HOLTokenForall{}}\ensuremath{\HOLBoundVar{E}\sb{\mathrm{2}}}. \ensuremath{\HOLBoundVar{a}\sb{\mathrm{1}}} \HOLTokenTransBegin\HOLSymConst{\ensuremath{\tau}}\HOLTokenTransEnd \ensuremath{\HOLBoundVar{E}\sb{\mathrm{2}}} \HOLSymConst{\HOLTokenImp{}} \HOLSymConst{\HOLTokenExists{}}\ensuremath{\HOLBoundVar{E}\sb{\mathrm{1}}}. \ensuremath{\HOLBoundVar{a}\sb{\mathrm{0}}} \HOLSymConst{\HOLTokenEPS} \ensuremath{\HOLBoundVar{E}\sb{\mathrm{1}}} \HOLSymConst{\HOLTokenConj{}} \ensuremath{\HOLFreeVar{WEAK\HOLTokenUnderscore{}EQUIV}\sp{\prime}} \ensuremath{\HOLBoundVar{E}\sb{\mathrm{1}}} \ensuremath{\HOLBoundVar{E}\sb{\mathrm{2}}}\ensuremath{)} \HOLSymConst{\HOLTokenImp{}}
   \HOLSymConst{\HOLTokenForall{}}\ensuremath{\HOLBoundVar{a}\sb{\mathrm{0}}} \ensuremath{\HOLBoundVar{a}\sb{\mathrm{1}}}. \ensuremath{\HOLFreeVar{WEAK\HOLTokenUnderscore{}EQUIV}\sp{\prime}} \ensuremath{\HOLBoundVar{a}\sb{\mathrm{0}}} \ensuremath{\HOLBoundVar{a}\sb{\mathrm{1}}} \HOLSymConst{\HOLTokenImp{}} \ensuremath{\HOLBoundVar{a}\sb{\mathrm{0}}} \HOLSymConst{\HOLTokenWeakEQ} \ensuremath{\HOLBoundVar{a}\sb{\mathrm{1}}}
\end{alltt}
%
\item \texttt{WEAK\_EQUIV\_cases} is the so-called `cases' or `inversion'
theorem for the relations, and is used to decompose an element in the relation
into the possible ways of obtaining it by the rules:
%
%a4.2 #&#
\begin{alltt}
\HOLTokenTurnstile{} \ensuremath{\HOLFreeVar{a}\sb{\mathrm{0}}} \HOLSymConst{\HOLTokenWeakEQ} \ensuremath{\HOLFreeVar{a}\sb{\mathrm{1}}} \HOLSymConst{\HOLTokenEquiv{}}
   \ensuremath{(}\HOLSymConst{\HOLTokenForall{}}\HOLBoundVar{l}.
        \ensuremath{(}\HOLSymConst{\HOLTokenForall{}}\ensuremath{\HOLBoundVar{E}\sb{\mathrm{1}}}. \ensuremath{\HOLFreeVar{a}\sb{\mathrm{0}}} \HOLTokenTransBegin\HOLConst{label} \HOLBoundVar{l}\HOLTokenTransEnd \ensuremath{\HOLBoundVar{E}\sb{\mathrm{1}}} \HOLSymConst{\HOLTokenImp{}} \HOLSymConst{\HOLTokenExists{}}\ensuremath{\HOLBoundVar{E}\sb{\mathrm{2}}}. \ensuremath{\HOLFreeVar{a}\sb{\mathrm{1}}} \HOLTokenWeakTransBegin\HOLConst{label} \HOLBoundVar{l}\HOLTokenWeakTransEnd \ensuremath{\HOLBoundVar{E}\sb{\mathrm{2}}} \HOLSymConst{\HOLTokenConj{}} \ensuremath{\HOLBoundVar{E}\sb{\mathrm{1}}} \HOLSymConst{\HOLTokenWeakEQ} \ensuremath{\HOLBoundVar{E}\sb{\mathrm{2}}}\ensuremath{)} \HOLSymConst{\HOLTokenConj{}}
        \HOLSymConst{\HOLTokenForall{}}\ensuremath{\HOLBoundVar{E}\sb{\mathrm{2}}}. \ensuremath{\HOLFreeVar{a}\sb{\mathrm{1}}} \HOLTokenTransBegin\HOLConst{label} \HOLBoundVar{l}\HOLTokenTransEnd \ensuremath{\HOLBoundVar{E}\sb{\mathrm{2}}} \HOLSymConst{\HOLTokenImp{}} \HOLSymConst{\HOLTokenExists{}}\ensuremath{\HOLBoundVar{E}\sb{\mathrm{1}}}. \ensuremath{\HOLFreeVar{a}\sb{\mathrm{0}}} \HOLTokenWeakTransBegin\HOLConst{label} \HOLBoundVar{l}\HOLTokenWeakTransEnd \ensuremath{\HOLBoundVar{E}\sb{\mathrm{1}}} \HOLSymConst{\HOLTokenConj{}} \ensuremath{\HOLBoundVar{E}\sb{\mathrm{1}}} \HOLSymConst{\HOLTokenWeakEQ} \ensuremath{\HOLBoundVar{E}\sb{\mathrm{2}}}\ensuremath{)} \HOLSymConst{\HOLTokenConj{}}
   \ensuremath{(}\HOLSymConst{\HOLTokenForall{}}\ensuremath{\HOLBoundVar{E}\sb{\mathrm{1}}}. \ensuremath{\HOLFreeVar{a}\sb{\mathrm{0}}} \HOLTokenTransBegin\HOLSymConst{\ensuremath{\tau}}\HOLTokenTransEnd \ensuremath{\HOLBoundVar{E}\sb{\mathrm{1}}} \HOLSymConst{\HOLTokenImp{}} \HOLSymConst{\HOLTokenExists{}}\ensuremath{\HOLBoundVar{E}\sb{\mathrm{2}}}. \ensuremath{\HOLFreeVar{a}\sb{\mathrm{1}}} \HOLSymConst{\HOLTokenEPS} \ensuremath{\HOLBoundVar{E}\sb{\mathrm{2}}} \HOLSymConst{\HOLTokenConj{}} \ensuremath{\HOLBoundVar{E}\sb{\mathrm{1}}} \HOLSymConst{\HOLTokenWeakEQ} \ensuremath{\HOLBoundVar{E}\sb{\mathrm{2}}}\ensuremath{)} \HOLSymConst{\HOLTokenConj{}}
   \HOLSymConst{\HOLTokenForall{}}\ensuremath{\HOLBoundVar{E}\sb{\mathrm{2}}}. \ensuremath{\HOLFreeVar{a}\sb{\mathrm{1}}} \HOLTokenTransBegin\HOLSymConst{\ensuremath{\tau}}\HOLTokenTransEnd \ensuremath{\HOLBoundVar{E}\sb{\mathrm{2}}} \HOLSymConst{\HOLTokenImp{}} \HOLSymConst{\HOLTokenExists{}}\ensuremath{\HOLBoundVar{E}\sb{\mathrm{1}}}. \ensuremath{\HOLFreeVar{a}\sb{\mathrm{0}}} \HOLSymConst{\HOLTokenEPS} \ensuremath{\HOLBoundVar{E}\sb{\mathrm{1}}} \HOLSymConst{\HOLTokenConj{}} \ensuremath{\HOLBoundVar{E}\sb{\mathrm{1}}} \HOLSymConst{\HOLTokenWeakEQ} \ensuremath{\HOLBoundVar{E}\sb{\mathrm{2}}}
\end{alltt}
%
\end{itemize}

The coinduction principle \texttt{WEAK\_EQUIV\_coind} says that any bisimulation
is contained in the resulting relation. The purpose of
\texttt{WEAK\_EQUIV\_cases} is to further assert that such resulting relation
is indeed a fixed point. Thus \texttt{WEAK\_EQUIV\_coind} and
\texttt{WEAK\_EQUIV\_cases} together make sure that bisimilarity is the greatest
fixed point.

The original definition of \texttt{WEAK\_EQUIV} now becomes a theorem:
%
%a4.2 #&#
\begin{alltt}
\HOLTokenTurnstile{} \HOLFreeVar{E} \HOLSymConst{\HOLTokenWeakEQ} \ensuremath{\HOLFreeVar{E}\sp{\prime}} \HOLSymConst{\HOLTokenEquiv{}} \HOLSymConst{\HOLTokenExists{}}\HOLBoundVar{Wbsm}. \HOLBoundVar{Wbsm} \HOLFreeVar{E} \ensuremath{\HOLFreeVar{E}\sp{\prime}} \HOLSymConst{\HOLTokenConj{}} \HOLConst{WEAK\_BISIM} \HOLBoundVar{Wbsm}\hfill{[WEAK\_EQUIV]}
\end{alltt}

The formal definition of rooted bisimilarity ($\rapprox $,
\texttt{OBS\_CONGR}) is not recursive and follows \reftext{Definition~\ref{d:rootedBisimilarity}}:
%
%a4.2 #&#
\begin{alltt}
   \HOLFreeVar{E} \HOLSymConst{\HOLTokenObsCongr} \ensuremath{\HOLFreeVar{E}\sp{\prime}} \HOLTokenDefEquality{}
     \HOLSymConst{\HOLTokenForall{}}\HOLBoundVar{u}.
         \ensuremath{(}\HOLSymConst{\HOLTokenForall{}}\ensuremath{\HOLBoundVar{E}\sb{\mathrm{1}}}. \HOLFreeVar{E} \HOLTokenTransBegin\HOLBoundVar{u}\HOLTokenTransEnd \ensuremath{\HOLBoundVar{E}\sb{\mathrm{1}}} \HOLSymConst{\HOLTokenImp{}} \HOLSymConst{\HOLTokenExists{}}\ensuremath{\HOLBoundVar{E}\sb{\mathrm{2}}}. \ensuremath{\HOLFreeVar{E}\sp{\prime}} \HOLTokenWeakTransBegin\HOLBoundVar{u}\HOLTokenWeakTransEnd \ensuremath{\HOLBoundVar{E}\sb{\mathrm{2}}} \HOLSymConst{\HOLTokenConj{}} \ensuremath{\HOLBoundVar{E}\sb{\mathrm{1}}} \HOLSymConst{\HOLTokenWeakEQ} \ensuremath{\HOLBoundVar{E}\sb{\mathrm{2}}}\ensuremath{)} \HOLSymConst{\HOLTokenConj{}}
         \HOLSymConst{\HOLTokenForall{}}\ensuremath{\HOLBoundVar{E}\sb{\mathrm{2}}}. \ensuremath{\HOLFreeVar{E}\sp{\prime}} \HOLTokenTransBegin\HOLBoundVar{u}\HOLTokenTransEnd \ensuremath{\HOLBoundVar{E}\sb{\mathrm{2}}} \HOLSymConst{\HOLTokenImp{}} \HOLSymConst{\HOLTokenExists{}}\ensuremath{\HOLBoundVar{E}\sb{\mathrm{1}}}. \HOLFreeVar{E} \HOLTokenWeakTransBegin\HOLBoundVar{u}\HOLTokenWeakTransEnd \ensuremath{\HOLBoundVar{E}\sb{\mathrm{1}}} \HOLSymConst{\HOLTokenConj{}} \ensuremath{\HOLBoundVar{E}\sb{\mathrm{1}}} \HOLSymConst{\HOLTokenWeakEQ} \ensuremath{\HOLBoundVar{E}\sb{\mathrm{2}}}\hfill{[OBS\_CONGR]}
\end{alltt}
%
Below is the formal version of \reftext{Lemma~\ref{lobsCongrByWeakBisim}} (\texttt{OBS\_CONGR\_BY\_WEAK\_BISIM}),
which is needed later, in the proof of \reftext{Theorem~\ref{t:rcontraBisimulationU}}:
%
%a4.2 #&#
\begin{alltt}
\HOLTokenTurnstile{} \HOLConst{WEAK\_BISIM} \HOLFreeVar{Wbsm} \HOLSymConst{\HOLTokenImp{}}
   \HOLSymConst{\HOLTokenForall{}}\HOLBoundVar{E} \ensuremath{\HOLBoundVar{E}\sp{\prime}}.
       \ensuremath{(}\HOLSymConst{\HOLTokenForall{}}\HOLBoundVar{u}.
            \ensuremath{(}\HOLSymConst{\HOLTokenForall{}}\ensuremath{\HOLBoundVar{E}\sb{\mathrm{1}}}. \HOLBoundVar{E} \HOLTokenTransBegin\HOLBoundVar{u}\HOLTokenTransEnd \ensuremath{\HOLBoundVar{E}\sb{\mathrm{1}}} \HOLSymConst{\HOLTokenImp{}} \HOLSymConst{\HOLTokenExists{}}\ensuremath{\HOLBoundVar{E}\sb{\mathrm{2}}}. \ensuremath{\HOLBoundVar{E}\sp{\prime}} \HOLTokenWeakTransBegin\HOLBoundVar{u}\HOLTokenWeakTransEnd \ensuremath{\HOLBoundVar{E}\sb{\mathrm{2}}} \HOLSymConst{\HOLTokenConj{}} \HOLFreeVar{Wbsm} \ensuremath{\HOLBoundVar{E}\sb{\mathrm{1}}} \ensuremath{\HOLBoundVar{E}\sb{\mathrm{2}}}\ensuremath{)} \HOLSymConst{\HOLTokenConj{}}
            \HOLSymConst{\HOLTokenForall{}}\ensuremath{\HOLBoundVar{E}\sb{\mathrm{2}}}. \ensuremath{\HOLBoundVar{E}\sp{\prime}} \HOLTokenTransBegin\HOLBoundVar{u}\HOLTokenTransEnd \ensuremath{\HOLBoundVar{E}\sb{\mathrm{2}}} \HOLSymConst{\HOLTokenImp{}} \HOLSymConst{\HOLTokenExists{}}\ensuremath{\HOLBoundVar{E}\sb{\mathrm{1}}}. \HOLBoundVar{E} \HOLTokenWeakTransBegin\HOLBoundVar{u}\HOLTokenWeakTransEnd \ensuremath{\HOLBoundVar{E}\sb{\mathrm{1}}} \HOLSymConst{\HOLTokenConj{}} \HOLFreeVar{Wbsm} \ensuremath{\HOLBoundVar{E}\sb{\mathrm{1}}} \ensuremath{\HOLBoundVar{E}\sb{\mathrm{2}}}\ensuremath{)} \HOLSymConst{\HOLTokenImp{}}
       \HOLBoundVar{E} \HOLSymConst{\HOLTokenObsCongr} \ensuremath{\HOLBoundVar{E}\sp{\prime}}
\end{alltt}

Finally, on the relationship between (weak) bisimilarity and rooted bisimilarity,
we have proved Deng's Lemma and Hennessy's Lemma (Lemma 4.1 and 4.2 of~\citep[p.~176,~178]{Gorrieri:2015jt}):
%
%a4.2 #&#
\begin{alltt}
\HOLTokenTurnstile{} \HOLSymConst{\HOLTokenForall{}}\HOLBoundVar{p} \HOLBoundVar{q}.
       \HOLBoundVar{p} \HOLSymConst{\HOLTokenWeakEQ} \HOLBoundVar{q} \HOLSymConst{\HOLTokenImp{}}
       \ensuremath{(}\HOLSymConst{\HOLTokenExists{}}\ensuremath{\HOLBoundVar{p}\sp{\prime}}. \HOLBoundVar{p} \HOLTokenTransBegin\HOLSymConst{\ensuremath{\tau}}\HOLTokenTransEnd \ensuremath{\HOLBoundVar{p}\sp{\prime}} \HOLSymConst{\HOLTokenConj{}} \ensuremath{\HOLBoundVar{p}\sp{\prime}} \HOLSymConst{\HOLTokenWeakEQ} \HOLBoundVar{q}\ensuremath{)} \HOLSymConst{\HOLTokenDisj{}} \ensuremath{(}\HOLSymConst{\HOLTokenExists{}}\ensuremath{\HOLBoundVar{q}\sp{\prime}}. \HOLBoundVar{q} \HOLTokenTransBegin\HOLSymConst{\ensuremath{\tau}}\HOLTokenTransEnd \ensuremath{\HOLBoundVar{q}\sp{\prime}} \HOLSymConst{\HOLTokenConj{}} \HOLBoundVar{p} \HOLSymConst{\HOLTokenWeakEQ} \ensuremath{\HOLBoundVar{q}\sp{\prime}}\ensuremath{)} \HOLSymConst{\HOLTokenDisj{}} \HOLBoundVar{p} \HOLSymConst{\HOLTokenObsCongr} \HOLBoundVar{q}\hfill{[DENG\_LEMMA]}
\end{alltt}

\begin{alltt}
\HOLTokenTurnstile{} \HOLSymConst{\HOLTokenForall{}}\HOLBoundVar{p} \HOLBoundVar{q}. \HOLBoundVar{p} \HOLSymConst{\HOLTokenWeakEQ} \HOLBoundVar{q} \HOLSymConst{\HOLTokenEquiv{}} \HOLBoundVar{p} \HOLSymConst{\HOLTokenObsCongr} \HOLBoundVar{q} \HOLSymConst{\HOLTokenDisj{}} \HOLBoundVar{p} \HOLSymConst{\HOLTokenObsCongr} \HOLSymConst{\ensuremath{\tau}}\HOLSymConst{\ensuremath{\ldotp}}\HOLBoundVar{q} \HOLSymConst{\HOLTokenDisj{}} \HOLSymConst{\ensuremath{\tau}}\HOLSymConst{\ensuremath{\ldotp}}\HOLBoundVar{p} \HOLSymConst{\HOLTokenObsCongr} \HOLBoundVar{q}\hfill{[HENNESSY\_LEMMA]}
\end{alltt}
%
These are useful results in the theory of CCS (though we will not need
them in the remainder of the paper).

%s4.4 #&#
\subsection{Algebraic laws}
%%LEAP%%%\label{sec4.4}
 \label{ss:alaws}

Having formalised the definitions of strong bisimulation and strong bisimilarity,
we can derive \emph{algebraic laws} for the bisimilarities. We only report
a few algebraic laws on summation:
%
%a4.4 #&#
\begin{alltt}
STRONG\_SUM\_IDEMP:          \HOLTokenTurnstile{} \HOLFreeVar{E} \HOLSymConst{\ensuremath{+}} \HOLFreeVar{E} \HOLSymConst{\HOLTokenStrongEQ} \HOLFreeVar{E}
STRONG\_SUM\_COMM:           \HOLTokenTurnstile{} \HOLFreeVar{E} \HOLSymConst{\ensuremath{+}} \ensuremath{\HOLFreeVar{E}\sp{\prime}} \HOLSymConst{\HOLTokenStrongEQ} \ensuremath{\HOLFreeVar{E}\sp{\prime}} \HOLSymConst{\ensuremath{+}} \HOLFreeVar{E}
STRONG\_SUM\_IDENT\_L:        \HOLTokenTurnstile{} \HOLConst{\ensuremath{\mathbf{0}}} \HOLSymConst{\ensuremath{+}} \HOLFreeVar{E} \HOLSymConst{\HOLTokenStrongEQ} \HOLFreeVar{E}
STRONG\_SUM\_IDENT\_R:        \HOLTokenTurnstile{} \HOLFreeVar{E} \HOLSymConst{\ensuremath{+}} \HOLConst{\ensuremath{\mathbf{0}}} \HOLSymConst{\HOLTokenStrongEQ} \HOLFreeVar{E}
STRONG\_SUM\_ASSOC\_R:        \HOLTokenTurnstile{} \HOLFreeVar{E} \HOLSymConst{\ensuremath{+}} \ensuremath{\HOLFreeVar{E}\sp{\prime}} \HOLSymConst{\ensuremath{+}} \ensuremath{\HOLFreeVar{E}\sp{\Prime}} \HOLSymConst{\HOLTokenStrongEQ} \HOLFreeVar{E} \HOLSymConst{\ensuremath{+}} \ensuremath{(}\ensuremath{\HOLFreeVar{E}\sp{\prime}} \HOLSymConst{\ensuremath{+}} \ensuremath{\HOLFreeVar{E}\sp{\Prime}}\ensuremath{)}
STRONG\_SUM\_ASSOC\_L:        \HOLTokenTurnstile{} \HOLFreeVar{E} \HOLSymConst{\ensuremath{+}} \ensuremath{(}\ensuremath{\HOLFreeVar{E}\sp{\prime}} \HOLSymConst{\ensuremath{+}} \ensuremath{\HOLFreeVar{E}\sp{\Prime}}\ensuremath{)} \HOLSymConst{\HOLTokenStrongEQ} \HOLFreeVar{E} \HOLSymConst{\ensuremath{+}} \ensuremath{\HOLFreeVar{E}\sp{\prime}} \HOLSymConst{\ensuremath{+}} \ensuremath{\HOLFreeVar{E}\sp{\Prime}}
STRONG\_SUM\_MID\_IDEMP:      \HOLTokenTurnstile{} \HOLFreeVar{E} \HOLSymConst{\ensuremath{+}} \ensuremath{\HOLFreeVar{E}\sp{\prime}} \HOLSymConst{\ensuremath{+}} \HOLFreeVar{E} \HOLSymConst{\HOLTokenStrongEQ} \ensuremath{\HOLFreeVar{E}\sp{\prime}} \HOLSymConst{\ensuremath{+}} \HOLFreeVar{E}
STRONG\_LEFT\_SUM\_MID\_IDEMP: \HOLTokenTurnstile{} \HOLFreeVar{E} \HOLSymConst{\ensuremath{+}} \ensuremath{\HOLFreeVar{E}\sp{\prime}} \HOLSymConst{\ensuremath{+}} \ensuremath{\HOLFreeVar{E}\sp{\Prime}} \HOLSymConst{\ensuremath{+}} \ensuremath{\HOLFreeVar{E}\sp{\prime}} \HOLSymConst{\HOLTokenStrongEQ} \HOLFreeVar{E} \HOLSymConst{\ensuremath{+}} \ensuremath{\HOLFreeVar{E}\sp{\Prime}} \HOLSymConst{\ensuremath{+}} \ensuremath{\HOLFreeVar{E}\sp{\prime}}
\end{alltt}

The first five of them are proven by constructing appropriate bisimulations,
and their formal proofs are written in a goal-directed manner~\citep[Chapter 4]{holdesc}.
In \xch{contrast}{constrast}, the last three algebraic laws are derived in a forward manner
by applications of previous proven laws (without directly using the SOS
inference rules and the definition of bisimulation). These algebraic laws
also hold for weak bisimilarity and rooted bisimilarity, as these are coarser
than strong bisimilarity. For weak bisimilarity and rooted bisimilarity,
the following algebraic laws, called $\tau $-laws, hold:
%
%a4.4 #&#
\begin{alltt}
TAU1:      \HOLTokenTurnstile{} \HOLFreeVar{u}\HOLSymConst{\ensuremath{\ldotp}}\HOLSymConst{\ensuremath{\tau}}\HOLSymConst{\ensuremath{\ldotp}}\HOLFreeVar{E} \HOLSymConst{\HOLTokenObsCongr} \HOLFreeVar{u}\HOLSymConst{\ensuremath{\ldotp}}\HOLFreeVar{E}
TAU2:      \HOLTokenTurnstile{} \HOLFreeVar{E} \HOLSymConst{\ensuremath{+}} \HOLSymConst{\ensuremath{\tau}}\HOLSymConst{\ensuremath{\ldotp}}\HOLFreeVar{E} \HOLSymConst{\HOLTokenObsCongr} \HOLSymConst{\ensuremath{\tau}}\HOLSymConst{\ensuremath{\ldotp}}\HOLFreeVar{E}
TAU3:      \HOLTokenTurnstile{} \HOLFreeVar{u}\HOLSymConst{\ensuremath{\ldotp}}\ensuremath{(}\HOLFreeVar{E} \HOLSymConst{\ensuremath{+}} \HOLSymConst{\ensuremath{\tau}}\HOLSymConst{\ensuremath{\ldotp}}\ensuremath{\HOLFreeVar{E}\sp{\prime}}\ensuremath{)} \HOLSymConst{\ensuremath{+}} \HOLFreeVar{u}\HOLSymConst{\ensuremath{\ldotp}}\ensuremath{\HOLFreeVar{E}\sp{\prime}} \HOLSymConst{\HOLTokenObsCongr} \HOLFreeVar{u}\HOLSymConst{\ensuremath{\ldotp}}\ensuremath{(}\HOLFreeVar{E} \HOLSymConst{\ensuremath{+}} \HOLSymConst{\ensuremath{\tau}}\HOLSymConst{\ensuremath{\ldotp}}\ensuremath{\HOLFreeVar{E}\sp{\prime}}\ensuremath{)}
TAU\_STRAT: \HOLTokenTurnstile{} \HOLFreeVar{E} \HOLSymConst{\ensuremath{+}} \HOLSymConst{\ensuremath{\tau}}\HOLSymConst{\ensuremath{\ldotp}}\ensuremath{(}\ensuremath{\HOLFreeVar{E}\sp{\prime}} \HOLSymConst{\ensuremath{+}} \HOLFreeVar{E}\ensuremath{)} \HOLSymConst{\HOLTokenObsCongr} \HOLSymConst{\ensuremath{\tau}}\HOLSymConst{\ensuremath{\ldotp}}\ensuremath{(}\ensuremath{\HOLFreeVar{E}\sp{\prime}} \HOLSymConst{\ensuremath{+}} \HOLFreeVar{E}\ensuremath{)}
TAU\_WEAK:  \HOLTokenTurnstile{} \HOLSymConst{\ensuremath{\tau}}\HOLSymConst{\ensuremath{\ldotp}}\HOLFreeVar{E} \HOLSymConst{\HOLTokenWeakEQ} \HOLFreeVar{E}
\end{alltt}

%s4.5 #&#
\subsection{Expansion, contraction and rooted contraction}
 \label{sec4.5}

We formalise and contraction along the lines of strong and weak bisimilarity:
%
%a4.5 #&#
\begin{alltt}
   \HOLConst{EXPANSION} \HOLFreeVar{Exp} \HOLTokenDefEquality{}
     \HOLSymConst{\HOLTokenForall{}}\HOLBoundVar{E} \ensuremath{\HOLBoundVar{E}\sp{\prime}}.
         \HOLFreeVar{Exp} \HOLBoundVar{E} \ensuremath{\HOLBoundVar{E}\sp{\prime}} \HOLSymConst{\HOLTokenImp{}}
         \ensuremath{(}\HOLSymConst{\HOLTokenForall{}}\HOLBoundVar{l}.
              \ensuremath{(}\HOLSymConst{\HOLTokenForall{}}\ensuremath{\HOLBoundVar{E}\sb{\mathrm{1}}}. \HOLBoundVar{E} \HOLTokenTransBegin\HOLConst{label} \HOLBoundVar{l}\HOLTokenTransEnd \ensuremath{\HOLBoundVar{E}\sb{\mathrm{1}}} \HOLSymConst{\HOLTokenImp{}} \HOLSymConst{\HOLTokenExists{}}\ensuremath{\HOLBoundVar{E}\sb{\mathrm{2}}}. \ensuremath{\HOLBoundVar{E}\sp{\prime}} \HOLTokenTransBegin\HOLConst{label} \HOLBoundVar{l}\HOLTokenTransEnd \ensuremath{\HOLBoundVar{E}\sb{\mathrm{2}}} \HOLSymConst{\HOLTokenConj{}} \HOLFreeVar{Exp} \ensuremath{\HOLBoundVar{E}\sb{\mathrm{1}}} \ensuremath{\HOLBoundVar{E}\sb{\mathrm{2}}}\ensuremath{)} \HOLSymConst{\HOLTokenConj{}}
              \HOLSymConst{\HOLTokenForall{}}\ensuremath{\HOLBoundVar{E}\sb{\mathrm{2}}}. \ensuremath{\HOLBoundVar{E}\sp{\prime}} \HOLTokenTransBegin\HOLConst{label} \HOLBoundVar{l}\HOLTokenTransEnd \ensuremath{\HOLBoundVar{E}\sb{\mathrm{2}}} \HOLSymConst{\HOLTokenImp{}} \HOLSymConst{\HOLTokenExists{}}\ensuremath{\HOLBoundVar{E}\sb{\mathrm{1}}}. \HOLBoundVar{E} \HOLTokenWeakTransBegin\HOLConst{label} \HOLBoundVar{l}\HOLTokenWeakTransEnd \ensuremath{\HOLBoundVar{E}\sb{\mathrm{1}}} \HOLSymConst{\HOLTokenConj{}} \HOLFreeVar{Exp} \ensuremath{\HOLBoundVar{E}\sb{\mathrm{1}}} \ensuremath{\HOLBoundVar{E}\sb{\mathrm{2}}}\ensuremath{)} \HOLSymConst{\HOLTokenConj{}}
         \ensuremath{(}\HOLSymConst{\HOLTokenForall{}}\ensuremath{\HOLBoundVar{E}\sb{\mathrm{1}}}. \HOLBoundVar{E} \HOLTokenTransBegin\HOLSymConst{\ensuremath{\tau}}\HOLTokenTransEnd \ensuremath{\HOLBoundVar{E}\sb{\mathrm{1}}} \HOLSymConst{\HOLTokenImp{}} \HOLFreeVar{Exp} \ensuremath{\HOLBoundVar{E}\sb{\mathrm{1}}} \ensuremath{\HOLBoundVar{E}\sp{\prime}} \HOLSymConst{\HOLTokenDisj{}} \HOLSymConst{\HOLTokenExists{}}\ensuremath{\HOLBoundVar{E}\sb{\mathrm{2}}}. \ensuremath{\HOLBoundVar{E}\sp{\prime}} \HOLTokenTransBegin\HOLSymConst{\ensuremath{\tau}}\HOLTokenTransEnd \ensuremath{\HOLBoundVar{E}\sb{\mathrm{2}}} \HOLSymConst{\HOLTokenConj{}} \HOLFreeVar{Exp} \ensuremath{\HOLBoundVar{E}\sb{\mathrm{1}}} \ensuremath{\HOLBoundVar{E}\sb{\mathrm{2}}}\ensuremath{)} \HOLSymConst{\HOLTokenConj{}}
         \HOLSymConst{\HOLTokenForall{}}\ensuremath{\HOLBoundVar{E}\sb{\mathrm{2}}}. \ensuremath{\HOLBoundVar{E}\sp{\prime}} \HOLTokenTransBegin\HOLSymConst{\ensuremath{\tau}}\HOLTokenTransEnd \ensuremath{\HOLBoundVar{E}\sb{\mathrm{2}}} \HOLSymConst{\HOLTokenImp{}} \HOLSymConst{\HOLTokenExists{}}\ensuremath{\HOLBoundVar{E}\sb{\mathrm{1}}}. \HOLBoundVar{E} \HOLTokenWeakTransBegin\HOLSymConst{\ensuremath{\tau}}\HOLTokenWeakTransEnd \ensuremath{\HOLBoundVar{E}\sb{\mathrm{1}}} \HOLSymConst{\HOLTokenConj{}} \HOLFreeVar{Exp} \ensuremath{\HOLBoundVar{E}\sb{\mathrm{1}}} \ensuremath{\HOLBoundVar{E}\sb{\mathrm{2}}}\hfill{[EXPANSION]}
\end{alltt}

\begin{alltt}
\HOLTokenTurnstile{} \HOLFreeVar{P} \HOLSymConst{\HOLTokenExpands{}} \HOLFreeVar{Q} \HOLSymConst{\HOLTokenEquiv{}} \HOLSymConst{\HOLTokenExists{}}\HOLBoundVar{Exp}. \HOLBoundVar{Exp} \HOLFreeVar{P} \HOLFreeVar{Q} \HOLSymConst{\HOLTokenConj{}} \HOLConst{EXPANSION} \HOLBoundVar{Exp}\hfill{[expands\_thm]}
\end{alltt}

\begin{alltt}
   \HOLConst{CONTRACTION} \HOLFreeVar{Con} \HOLTokenDefEquality{}
     \HOLSymConst{\HOLTokenForall{}}\HOLBoundVar{E} \ensuremath{\HOLBoundVar{E}\sp{\prime}}.
         \HOLFreeVar{Con} \HOLBoundVar{E} \ensuremath{\HOLBoundVar{E}\sp{\prime}} \HOLSymConst{\HOLTokenImp{}}
         \ensuremath{(}\HOLSymConst{\HOLTokenForall{}}\HOLBoundVar{l}.
              \ensuremath{(}\HOLSymConst{\HOLTokenForall{}}\ensuremath{\HOLBoundVar{E}\sb{\mathrm{1}}}. \HOLBoundVar{E} \HOLTokenTransBegin\HOLConst{label} \HOLBoundVar{l}\HOLTokenTransEnd \ensuremath{\HOLBoundVar{E}\sb{\mathrm{1}}} \HOLSymConst{\HOLTokenImp{}} \HOLSymConst{\HOLTokenExists{}}\ensuremath{\HOLBoundVar{E}\sb{\mathrm{2}}}. \ensuremath{\HOLBoundVar{E}\sp{\prime}} \HOLTokenTransBegin\HOLConst{label} \HOLBoundVar{l}\HOLTokenTransEnd \ensuremath{\HOLBoundVar{E}\sb{\mathrm{2}}} \HOLSymConst{\HOLTokenConj{}} \HOLFreeVar{Con} \ensuremath{\HOLBoundVar{E}\sb{\mathrm{1}}} \ensuremath{\HOLBoundVar{E}\sb{\mathrm{2}}}\ensuremath{)} \HOLSymConst{\HOLTokenConj{}}
              \HOLSymConst{\HOLTokenForall{}}\ensuremath{\HOLBoundVar{E}\sb{\mathrm{2}}}. \ensuremath{\HOLBoundVar{E}\sp{\prime}} \HOLTokenTransBegin\HOLConst{label} \HOLBoundVar{l}\HOLTokenTransEnd \ensuremath{\HOLBoundVar{E}\sb{\mathrm{2}}} \HOLSymConst{\HOLTokenImp{}} \HOLSymConst{\HOLTokenExists{}}\ensuremath{\HOLBoundVar{E}\sb{\mathrm{1}}}. \HOLBoundVar{E} \HOLTokenWeakTransBegin\HOLConst{label} \HOLBoundVar{l}\HOLTokenWeakTransEnd \ensuremath{\HOLBoundVar{E}\sb{\mathrm{1}}} \HOLSymConst{\HOLTokenConj{}} \ensuremath{\HOLBoundVar{E}\sb{\mathrm{1}}} \HOLSymConst{\HOLTokenWeakEQ} \ensuremath{\HOLBoundVar{E}\sb{\mathrm{2}}}\ensuremath{)} \HOLSymConst{\HOLTokenConj{}}
         \ensuremath{(}\HOLSymConst{\HOLTokenForall{}}\ensuremath{\HOLBoundVar{E}\sb{\mathrm{1}}}. \HOLBoundVar{E} \HOLTokenTransBegin\HOLSymConst{\ensuremath{\tau}}\HOLTokenTransEnd \ensuremath{\HOLBoundVar{E}\sb{\mathrm{1}}} \HOLSymConst{\HOLTokenImp{}} \HOLFreeVar{Con} \ensuremath{\HOLBoundVar{E}\sb{\mathrm{1}}} \ensuremath{\HOLBoundVar{E}\sp{\prime}} \HOLSymConst{\HOLTokenDisj{}} \HOLSymConst{\HOLTokenExists{}}\ensuremath{\HOLBoundVar{E}\sb{\mathrm{2}}}. \ensuremath{\HOLBoundVar{E}\sp{\prime}} \HOLTokenTransBegin\HOLSymConst{\ensuremath{\tau}}\HOLTokenTransEnd \ensuremath{\HOLBoundVar{E}\sb{\mathrm{2}}} \HOLSymConst{\HOLTokenConj{}} \HOLFreeVar{Con} \ensuremath{\HOLBoundVar{E}\sb{\mathrm{1}}} \ensuremath{\HOLBoundVar{E}\sb{\mathrm{2}}}\ensuremath{)} \HOLSymConst{\HOLTokenConj{}}
         \HOLSymConst{\HOLTokenForall{}}\ensuremath{\HOLBoundVar{E}\sb{\mathrm{2}}}. \ensuremath{\HOLBoundVar{E}\sp{\prime}} \HOLTokenTransBegin\HOLSymConst{\ensuremath{\tau}}\HOLTokenTransEnd \ensuremath{\HOLBoundVar{E}\sb{\mathrm{2}}} \HOLSymConst{\HOLTokenImp{}} \HOLSymConst{\HOLTokenExists{}}\ensuremath{\HOLBoundVar{E}\sb{\mathrm{1}}}. \HOLBoundVar{E} \HOLSymConst{\HOLTokenEPS} \ensuremath{\HOLBoundVar{E}\sb{\mathrm{1}}} \HOLSymConst{\HOLTokenConj{}} \ensuremath{\HOLBoundVar{E}\sb{\mathrm{1}}} \HOLSymConst{\HOLTokenWeakEQ} \ensuremath{\HOLBoundVar{E}\sb{\mathrm{2}}}\hfill{[CONTRACTION]}
\end{alltt}

\begin{alltt}
\HOLTokenTurnstile{} \HOLFreeVar{P} \HOLSymConst{\HOLTokenContracts{}} \HOLFreeVar{Q} \HOLSymConst{\HOLTokenEquiv{}} \HOLSymConst{\HOLTokenExists{}}\HOLBoundVar{Con}. \HOLBoundVar{Con} \HOLFreeVar{P} \HOLFreeVar{Q} \HOLSymConst{\HOLTokenConj{}} \HOLConst{CONTRACTION} \HOLBoundVar{Con}\hfill{[contracts\_thm]}
\end{alltt}

The contraction preorder ($\mcontrBIS $) contains the expansion preorder
($\expa $), and they are both contained in weak bisimilarity ($
\wbvtex $):
%
%p4.1 #&#
\begin{proposition}%
{(Relationships between contraction preorder, expansion preorder and weak
bisimilarity)}
%
\begin{enumerate}
%
\item (Expansion preorder implies contraction preorder)
%
%a1 #&#
\begin{alltt}
\HOLTokenTurnstile{} \HOLSymConst{\HOLTokenForall{}}\HOLBoundVar{P} \HOLBoundVar{Q}. \HOLBoundVar{P} \HOLSymConst{\HOLTokenExpands{}} \HOLBoundVar{Q} \HOLSymConst{\HOLTokenImp{}} \HOLBoundVar{P} \HOLSymConst{\HOLTokenContracts{}} \HOLBoundVar{Q}\hfill[expands\_IMP\_contracts]
\end{alltt}
%
\item (Contraction preorder implies weak bisimilarity)
%
%a2 #&#
\begin{alltt}
\HOLTokenTurnstile{} \HOLSymConst{\HOLTokenForall{}}\HOLBoundVar{P} \HOLBoundVar{Q}. \HOLBoundVar{P} \HOLSymConst{\HOLTokenContracts{}} \HOLBoundVar{Q} \HOLSymConst{\HOLTokenImp{}} \HOLBoundVar{P} \HOLSymConst{\HOLTokenWeakEQ} \HOLBoundVar{Q}\hfill[contracts\_IMP\_WEAK\_EQUIV]
\end{alltt}
%
\end{enumerate}
%
\end{proposition}

The proofs of properties for contraction are generally harder than those
for expansion. This is mostly due to the fact that, although the contraction
preorder ($\mcontrBIS $) is contained in bisimilarity ($\wbvtex $), a contraction
need not be a bisimulation. In another words, the following proposition
does not hold:
%
%a4.5 #&#
\begin{alltt}
   \HOLinline{\HOLSymConst{\HOLTokenForall{}}\HOLBoundVar{Con}. \HOLConst{CONTRACTION} \HOLBoundVar{Con} \HOLSymConst{\HOLTokenImp{}} \HOLConst{WEAK\_BISIM} \HOLBoundVar{Con}}
\end{alltt}
%
However it does hold that, if $\Rvtex $ is a contraction, then
$\Rvtex \;\cup \wbvtex $ is a bisimulation. For instance, we can prove
\texttt{contracts\_IMP\_WEAK\_EQUIV} by constructing a bisimulation
\HOLinline{\HOLFreeVar{Wbsm}} containing two processes $P$ and $Q$, given
that they are in $Con$ (a contraction):
%
%a4.5 #&#
\begin{alltt}
        \HOLinline{\HOLSymConst{\HOLTokenExists{}}\HOLBoundVar{Wbsm}. \HOLBoundVar{Wbsm} \HOLFreeVar{P} \HOLFreeVar{Q} \HOLSymConst{\HOLTokenConj{}} \HOLConst{WEAK\_BISIM} \HOLBoundVar{Wbsm}}
   ------------------------------------
    0.  \HOLinline{\HOLFreeVar{Con} \HOLFreeVar{P} \HOLFreeVar{Q}}
    1.  \HOLinline{\HOLConst{CONTRACTION} \HOLFreeVar{Con}}
\end{alltt}
%
To complete the proof, one cannot choose $Con$ for $Wbsm$ and show that
$Con$ itself is a bisimulation, but rather that
$Con\;\cup \!\wbvtex $ is a bisimulation. In contrast, in the corresponding
lemma for expansion one can just take $Con$. Finally, the rooted contraction
($\rcontr $) is formalised as follows:
%
%a4.5 #&#
\begin{alltt}
   \HOLFreeVar{E} \HOLSymConst{\HOLTokenObsContracts} \ensuremath{\HOLFreeVar{E}\sp{\prime}} \HOLTokenDefEquality{}
     \HOLSymConst{\HOLTokenForall{}}\HOLBoundVar{u}.
         \ensuremath{(}\HOLSymConst{\HOLTokenForall{}}\ensuremath{\HOLBoundVar{E}\sb{\mathrm{1}}}. \HOLFreeVar{E} \HOLTokenTransBegin\HOLBoundVar{u}\HOLTokenTransEnd \ensuremath{\HOLBoundVar{E}\sb{\mathrm{1}}} \HOLSymConst{\HOLTokenImp{}} \HOLSymConst{\HOLTokenExists{}}\ensuremath{\HOLBoundVar{E}\sb{\mathrm{2}}}. \ensuremath{\HOLFreeVar{E}\sp{\prime}} \HOLTokenTransBegin\HOLBoundVar{u}\HOLTokenTransEnd \ensuremath{\HOLBoundVar{E}\sb{\mathrm{2}}} \HOLSymConst{\HOLTokenConj{}} \ensuremath{\HOLBoundVar{E}\sb{\mathrm{1}}} \HOLSymConst{\HOLTokenContracts{}} \ensuremath{\HOLBoundVar{E}\sb{\mathrm{2}}}\ensuremath{)} \HOLSymConst{\HOLTokenConj{}}
         \HOLSymConst{\HOLTokenForall{}}\ensuremath{\HOLBoundVar{E}\sb{\mathrm{2}}}. \ensuremath{\HOLFreeVar{E}\sp{\prime}} \HOLTokenTransBegin\HOLBoundVar{u}\HOLTokenTransEnd \ensuremath{\HOLBoundVar{E}\sb{\mathrm{2}}} \HOLSymConst{\HOLTokenImp{}} \HOLSymConst{\HOLTokenExists{}}\ensuremath{\HOLBoundVar{E}\sb{\mathrm{1}}}. \HOLFreeVar{E} \HOLTokenWeakTransBegin\HOLBoundVar{u}\HOLTokenWeakTransEnd \ensuremath{\HOLBoundVar{E}\sb{\mathrm{1}}} \HOLSymConst{\HOLTokenConj{}} \ensuremath{\HOLBoundVar{E}\sb{\mathrm{1}}} \HOLSymConst{\HOLTokenWeakEQ} \ensuremath{\HOLBoundVar{E}\sb{\mathrm{2}}}\hfill{[OBS\_contracts]}
\end{alltt}

%s4.6 #&#
\subsection{The formalisation of ``bisimulation up to''}
 \label{sec4.6}

``Bisimulation up to'' is a family of powerful proof techniques for reducing
the sizes of relations needed for defining bisimulations~\cite{PousS19}.
By definition, two processes are bisimilar iff there exists a bisimulation
relation containing them. However, in practice this definition is sometimes
hard to apply. Instead, to reduce the sizes of the exhibited relations,
one prefers to define relations which are bisimulations only when closed
up under some specific and privileged relation, so to relieve the needed
proof work. These techniques are usually called
\emph{``up-to'' techniques}.

Recall that we often write $P \;\Rvtex \; Q$ to denote
$(P, Q) \in \Rvtex $ for any binary relation $\Rvtex $. Moreover,
$\sim \mathcal{S} \sim $ is the composition of three binary relations:
$\sim $, $\Svtex $ and $\sim $. Hence $P \sim \Svtex \sim Q$ means that
there exist $P'$ and $Q'$ such that $P \sim P'$, $P' \;\Svtex \; Q'$ and
$Q' \sim Q$.
%
%d4.2 #&#
\begin{definition}%
 \label{def:bisimUptoSim}
$\Svtex $ is a ``bisimulation up to $\sim $'' if $P\ \Svtex \ Q$ implies,
for all $\mu $,
%
\begin{enumerate}
%
\item Whenever $P \overset{\mu}{\rightarrow} P'$ then, for some $Q'$,
$Q \overset{\mu}{\rightarrow} Q'$ and $P' \sim \Svtex \sim Q'$,
%
\item Whenever $Q \overset{\mu}{\rightarrow} Q'$ then, for some $P'$,
$P \overset{\mu}{\rightarrow} P'$ and $P' \sim \Svtex \sim Q'$.
%
\end{enumerate}
%
\end{definition}

%t4.3 #&#
\begin{theorem}
If $\mathcal{S}$ is a ``bisimulation up to $\sim $'', then
$\mathcal{S} \subseteq \;\sim $:
%
%a4.3 #&#
\begin{alltt}
\HOLTokenTurnstile{} \HOLConst{STRONG\_BISIM\_UPTO} \HOLFreeVar{Bsm} \HOLSymConst{\HOLTokenConj{}} \HOLFreeVar{Bsm} \HOLFreeVar{P} \HOLFreeVar{Q} \HOLSymConst{\HOLTokenImp{}} \HOLFreeVar{P} \HOLSymConst{\HOLTokenStrongEQ} \HOLFreeVar{Q}\hfill{[STRONG\_EQUIV\_BY\_BISIM\_UPTO]}
\end{alltt}
%
\end{theorem}
%
Hence, to prove $P \sim Q$, one only needs to find a bisimulation up to
$\sim $ that contains $(P, Q)$. For weak bisimilarity, the naive weak bisimulation
up to weak bisimilarity is unsound: if one simply replaces all occurrences
of $\sim $ in \reftext{Definition~\ref{def:bisimUptoSim}} with $\wbvtex $, the resulting
``weak bisimulation up to'' relation need not be contained in weak bisimilarity
($\wbvtex $)~\cite{PoSa2019}. There are a few ways to fix this problem,
and one is the following:
%
%d4.4 #&#
\begin{definition}%
 \label{def:singleweak}
$\Svtex $ is a ``bisimulation up to $\approx $'' if $P\ \Svtex \ Q$ implies,
for all $\mu $,
%
\begin{enumerate}
%
\item Whenever $P \arr{\mu} P'$ then, for some $Q'$,
$Q \Arcap{\mu} Q'$ and $P' \sim \Svtex \approx Q'$,
%
\item Whenever $Q \arr{\mu} Q'$ then, for some $P'$,
$P \Arcap{\mu} P'$ and $P' \approx \Svtex \sim Q'$.
%
\end{enumerate}
%
Formally:
%
%a4.4 #&#
\begin{alltt}
   \HOLConst{WEAK\_BISIM\_UPTO} \HOLFreeVar{Wbsm} \HOLTokenDefEquality{}
     \HOLSymConst{\HOLTokenForall{}}\HOLBoundVar{E} \ensuremath{\HOLBoundVar{E}\sp{\prime}}.
         \HOLFreeVar{Wbsm} \HOLBoundVar{E} \ensuremath{\HOLBoundVar{E}\sp{\prime}} \HOLSymConst{\HOLTokenImp{}}
         \ensuremath{(}\HOLSymConst{\HOLTokenForall{}}\HOLBoundVar{l}.
              \ensuremath{(}\HOLSymConst{\HOLTokenForall{}}\ensuremath{\HOLBoundVar{E}\sb{\mathrm{1}}}.
                   \HOLBoundVar{E} \HOLTokenTransBegin\HOLConst{label} \HOLBoundVar{l}\HOLTokenTransEnd \ensuremath{\HOLBoundVar{E}\sb{\mathrm{1}}} \HOLSymConst{\HOLTokenImp{}}
                   \HOLSymConst{\HOLTokenExists{}}\ensuremath{\HOLBoundVar{E}\sb{\mathrm{2}}}.
                       \ensuremath{\HOLBoundVar{E}\sp{\prime}} \HOLTokenWeakTransBegin\HOLConst{label} \HOLBoundVar{l}\HOLTokenWeakTransEnd \ensuremath{\HOLBoundVar{E}\sb{\mathrm{2}}} \HOLSymConst{\HOLTokenConj{}}
                       \ensuremath{(}\HOLConst{WEAK\_EQUIV} \HOLSymConst{\HOLTokenRCompose{}} \HOLFreeVar{Wbsm} \HOLSymConst{\HOLTokenRCompose{}} \HOLConst{STRONG\_EQUIV}\ensuremath{)} \ensuremath{\HOLBoundVar{E}\sb{\mathrm{1}}} \ensuremath{\HOLBoundVar{E}\sb{\mathrm{2}}}\ensuremath{)} \HOLSymConst{\HOLTokenConj{}}
              \HOLSymConst{\HOLTokenForall{}}\ensuremath{\HOLBoundVar{E}\sb{\mathrm{2}}}.
                  \ensuremath{\HOLBoundVar{E}\sp{\prime}} \HOLTokenTransBegin\HOLConst{label} \HOLBoundVar{l}\HOLTokenTransEnd \ensuremath{\HOLBoundVar{E}\sb{\mathrm{2}}} \HOLSymConst{\HOLTokenImp{}}
                  \HOLSymConst{\HOLTokenExists{}}\ensuremath{\HOLBoundVar{E}\sb{\mathrm{1}}}.
                      \HOLBoundVar{E} \HOLTokenWeakTransBegin\HOLConst{label} \HOLBoundVar{l}\HOLTokenWeakTransEnd \ensuremath{\HOLBoundVar{E}\sb{\mathrm{1}}} \HOLSymConst{\HOLTokenConj{}}
                      \ensuremath{(}\HOLConst{STRONG\_EQUIV} \HOLSymConst{\HOLTokenRCompose{}} \HOLFreeVar{Wbsm} \HOLSymConst{\HOLTokenRCompose{}} \HOLConst{WEAK\_EQUIV}\ensuremath{)} \ensuremath{\HOLBoundVar{E}\sb{\mathrm{1}}} \ensuremath{\HOLBoundVar{E}\sb{\mathrm{2}}}\ensuremath{)} \HOLSymConst{\HOLTokenConj{}}
         \ensuremath{(}\HOLSymConst{\HOLTokenForall{}}\ensuremath{\HOLBoundVar{E}\sb{\mathrm{1}}}.
              \HOLBoundVar{E} \HOLTokenTransBegin\HOLSymConst{\ensuremath{\tau}}\HOLTokenTransEnd \ensuremath{\HOLBoundVar{E}\sb{\mathrm{1}}} \HOLSymConst{\HOLTokenImp{}}
              \HOLSymConst{\HOLTokenExists{}}\ensuremath{\HOLBoundVar{E}\sb{\mathrm{2}}}. \ensuremath{\HOLBoundVar{E}\sp{\prime}} \HOLSymConst{\HOLTokenEPS} \ensuremath{\HOLBoundVar{E}\sb{\mathrm{2}}} \HOLSymConst{\HOLTokenConj{}} \ensuremath{(}\HOLConst{WEAK\_EQUIV} \HOLSymConst{\HOLTokenRCompose{}} \HOLFreeVar{Wbsm} \HOLSymConst{\HOLTokenRCompose{}} \HOLConst{STRONG\_EQUIV}\ensuremath{)} \ensuremath{\HOLBoundVar{E}\sb{\mathrm{1}}} \ensuremath{\HOLBoundVar{E}\sb{\mathrm{2}}}\ensuremath{)} \HOLSymConst{\HOLTokenConj{}}
         \HOLSymConst{\HOLTokenForall{}}\ensuremath{\HOLBoundVar{E}\sb{\mathrm{2}}}.
             \ensuremath{\HOLBoundVar{E}\sp{\prime}} \HOLTokenTransBegin\HOLSymConst{\ensuremath{\tau}}\HOLTokenTransEnd \ensuremath{\HOLBoundVar{E}\sb{\mathrm{2}}} \HOLSymConst{\HOLTokenImp{}}
             \HOLSymConst{\HOLTokenExists{}}\ensuremath{\HOLBoundVar{E}\sb{\mathrm{1}}}. \HOLBoundVar{E} \HOLSymConst{\HOLTokenEPS} \ensuremath{\HOLBoundVar{E}\sb{\mathrm{1}}} \HOLSymConst{\HOLTokenConj{}} \ensuremath{(}\HOLConst{STRONG\_EQUIV} \HOLSymConst{\HOLTokenRCompose{}} \HOLFreeVar{Wbsm} \HOLSymConst{\HOLTokenRCompose{}} \HOLConst{WEAK\_EQUIV}\ensuremath{)} \ensuremath{\HOLBoundVar{E}\sb{\mathrm{1}}} \ensuremath{\HOLBoundVar{E}\sb{\mathrm{2}}}
\end{alltt}
%
\end{definition}
%
Note that the HOL term corresponding to $\sim \Rvtex \wbvtex $ is ``\HOLinline{\HOLConst{WEAK\_EQUIV} \HOLSymConst{\HOLTokenRCompose{}} \HOLFreeVar{R} \HOLSymConst{\HOLTokenRCompose{}} \HOLConst{STRONG\_EQUIV}}''
where the order of $\sim $ and $\approx $ seems reverted. This is because,
in HOL notation, the rightmost relation (\HOLinline{\HOLConst{STRONG\_EQUIV}}
or $\sim $) in the relational composition is applied first.

%t4.5 #&#
\begin{theorem}
If $\mathcal{S}$ is a bisimulation up to $\approx $, then
$\mathcal{S} \subseteq \;\approx $:
%
%a4.5 #&#
\begin{alltt}
\HOLTokenTurnstile{} \HOLConst{WEAK\_BISIM\_UPTO} \HOLFreeVar{Bsm} \HOLSymConst{\HOLTokenConj{}} \HOLFreeVar{Bsm} \HOLFreeVar{P} \HOLFreeVar{Q} \HOLSymConst{\HOLTokenImp{}} \HOLFreeVar{P} \HOLSymConst{\HOLTokenWeakEQ} \HOLFreeVar{Q}\hfill{[WEAK\_EQUIV\_BY\_BISIM\_UPTO]}
\end{alltt}
%
\end{theorem}

The above version of ``bisimulation up to $\wbvtex $'' is not powerful
enough for Milner's ``unique solution of equations'' theorem for
$\wbvtex $ (\reftext{Theorem~\ref{t:Mil89}}, see~\cite{sangiorgi1992problem} for
more details). The following version, with weak arrows, is used in the
proof of \reftext{Theorem~\ref{t:Mil89}}:
%
%d4.6 #&#
\begin{definition}%
 \label{def:doubleweak}
$\mathcal{S}$ is a ``bisimulation up to $\approx $ with weak arrows'' if
$P \; \mathcal{S} \; Q$ implies, for all $\mu $,
%
\begin{enumerate}
%
\item Whenever $P \Arr{\mu} P'$ then, for some $Q'$,
$Q \Arcap{\mu} Q'$ and $P' \wbvtex \Svtex \wbvtex Q'$,
%
\item Whenever $Q \Arr{\mu} Q'$ then, for some $P'$,
$P \Arcap{\mu} P'$ and $P' \wbvtex \Svtex \wbvtex Q'$.
%
\end{enumerate}
%
Formally:
%
%a4.6 #&#
\begin{alltt}
   \HOLConst{WEAK\_BISIM\_UPTO\_ALT} \HOLFreeVar{Wbsm} \HOLTokenDefEquality{}
     \HOLSymConst{\HOLTokenForall{}}\HOLBoundVar{E} \ensuremath{\HOLBoundVar{E}\sp{\prime}}.
         \HOLFreeVar{Wbsm} \HOLBoundVar{E} \ensuremath{\HOLBoundVar{E}\sp{\prime}} \HOLSymConst{\HOLTokenImp{}}
         \ensuremath{(}\HOLSymConst{\HOLTokenForall{}}\HOLBoundVar{l}.
              \ensuremath{(}\HOLSymConst{\HOLTokenForall{}}\ensuremath{\HOLBoundVar{E}\sb{\mathrm{1}}}.
                   \HOLBoundVar{E} \HOLTokenWeakTransBegin\HOLConst{label} \HOLBoundVar{l}\HOLTokenWeakTransEnd \ensuremath{\HOLBoundVar{E}\sb{\mathrm{1}}} \HOLSymConst{\HOLTokenImp{}}
                   \HOLSymConst{\HOLTokenExists{}}\ensuremath{\HOLBoundVar{E}\sb{\mathrm{2}}}.
                       \ensuremath{\HOLBoundVar{E}\sp{\prime}} \HOLTokenWeakTransBegin\HOLConst{label} \HOLBoundVar{l}\HOLTokenWeakTransEnd \ensuremath{\HOLBoundVar{E}\sb{\mathrm{2}}} \HOLSymConst{\HOLTokenConj{}}
                       \ensuremath{(}\HOLConst{WEAK\_EQUIV} \HOLSymConst{\HOLTokenRCompose{}} \HOLFreeVar{Wbsm} \HOLSymConst{\HOLTokenRCompose{}} \HOLConst{WEAK\_EQUIV}\ensuremath{)} \ensuremath{\HOLBoundVar{E}\sb{\mathrm{1}}} \ensuremath{\HOLBoundVar{E}\sb{\mathrm{2}}}\ensuremath{)} \HOLSymConst{\HOLTokenConj{}}
              \HOLSymConst{\HOLTokenForall{}}\ensuremath{\HOLBoundVar{E}\sb{\mathrm{2}}}.
                  \ensuremath{\HOLBoundVar{E}\sp{\prime}} \HOLTokenWeakTransBegin\HOLConst{label} \HOLBoundVar{l}\HOLTokenWeakTransEnd \ensuremath{\HOLBoundVar{E}\sb{\mathrm{2}}} \HOLSymConst{\HOLTokenImp{}}
                  \HOLSymConst{\HOLTokenExists{}}\ensuremath{\HOLBoundVar{E}\sb{\mathrm{1}}}.
                      \HOLBoundVar{E} \HOLTokenWeakTransBegin\HOLConst{label} \HOLBoundVar{l}\HOLTokenWeakTransEnd \ensuremath{\HOLBoundVar{E}\sb{\mathrm{1}}} \HOLSymConst{\HOLTokenConj{}}
                      \ensuremath{(}\HOLConst{WEAK\_EQUIV} \HOLSymConst{\HOLTokenRCompose{}} \HOLFreeVar{Wbsm} \HOLSymConst{\HOLTokenRCompose{}} \HOLConst{WEAK\_EQUIV}\ensuremath{)} \ensuremath{\HOLBoundVar{E}\sb{\mathrm{1}}} \ensuremath{\HOLBoundVar{E}\sb{\mathrm{2}}}\ensuremath{)} \HOLSymConst{\HOLTokenConj{}}
         \ensuremath{(}\HOLSymConst{\HOLTokenForall{}}\ensuremath{\HOLBoundVar{E}\sb{\mathrm{1}}}.
              \HOLBoundVar{E} \HOLTokenWeakTransBegin\HOLSymConst{\ensuremath{\tau}}\HOLTokenWeakTransEnd \ensuremath{\HOLBoundVar{E}\sb{\mathrm{1}}} \HOLSymConst{\HOLTokenImp{}}
              \HOLSymConst{\HOLTokenExists{}}\ensuremath{\HOLBoundVar{E}\sb{\mathrm{2}}}. \ensuremath{\HOLBoundVar{E}\sp{\prime}} \HOLSymConst{\HOLTokenEPS} \ensuremath{\HOLBoundVar{E}\sb{\mathrm{2}}} \HOLSymConst{\HOLTokenConj{}} \ensuremath{(}\HOLConst{WEAK\_EQUIV} \HOLSymConst{\HOLTokenRCompose{}} \HOLFreeVar{Wbsm} \HOLSymConst{\HOLTokenRCompose{}} \HOLConst{WEAK\_EQUIV}\ensuremath{)} \ensuremath{\HOLBoundVar{E}\sb{\mathrm{1}}} \ensuremath{\HOLBoundVar{E}\sb{\mathrm{2}}}\ensuremath{)} \HOLSymConst{\HOLTokenConj{}}
         \HOLSymConst{\HOLTokenForall{}}\ensuremath{\HOLBoundVar{E}\sb{\mathrm{2}}}.
             \ensuremath{\HOLBoundVar{E}\sp{\prime}} \HOLTokenWeakTransBegin\HOLSymConst{\ensuremath{\tau}}\HOLTokenWeakTransEnd \ensuremath{\HOLBoundVar{E}\sb{\mathrm{2}}} \HOLSymConst{\HOLTokenImp{}}
             \HOLSymConst{\HOLTokenExists{}}\ensuremath{\HOLBoundVar{E}\sb{\mathrm{1}}}. \HOLBoundVar{E} \HOLSymConst{\HOLTokenEPS} \ensuremath{\HOLBoundVar{E}\sb{\mathrm{1}}} \HOLSymConst{\HOLTokenConj{}} \ensuremath{(}\HOLConst{WEAK\_EQUIV} \HOLSymConst{\HOLTokenRCompose{}} \HOLFreeVar{Wbsm} \HOLSymConst{\HOLTokenRCompose{}} \HOLConst{WEAK\_EQUIV}\ensuremath{)} \ensuremath{\HOLBoundVar{E}\sb{\mathrm{1}}} \ensuremath{\HOLBoundVar{E}\sb{\mathrm{2}}}
\end{alltt}
%
\end{definition}

%t4.7 #&#
\begin{theorem}
If $\mathcal{S}$ is a bisimulation up to $\approx $ with weak arrows, then
$\mathcal{S} \subseteq \;\approx $:
%
%a4.7 #&#
\begin{alltt}
\HOLTokenTurnstile{} \HOLConst{WEAK\_BISIM\_UPTO\_ALT} \HOLFreeVar{Bsm} \HOLSymConst{\HOLTokenConj{}} \HOLFreeVar{Bsm} \HOLFreeVar{P} \HOLFreeVar{Q} \HOLSymConst{\HOLTokenImp{}} \HOLFreeVar{P} \HOLSymConst{\HOLTokenWeakEQ} \HOLFreeVar{Q}\hfill{[WEAK\_EQUIV\_BY\_BISIM\_UPTO\_ALT]}
\end{alltt}
%
\end{theorem}

%s4.7 #&#
\subsection{Context, guardedness and (pre)congruence}
%%LEAP%%%\label{sec4.7}
 \label{sscontext}

CCS contexts are needed in defining (pre)congruence. To prevent doing variable
\xch{substitutions}{substititions}, one can take \univariate $\lambda $-expressions (of type
``\HOLinline{\ensuremath{(}\ensuremath{\alpha}, \ensuremath{\beta}\ensuremath{)} \HOLTyOp{CCS} \HOLTokenTransEnd \ensuremath{(}\ensuremath{\alpha}, \ensuremath{\beta}\ensuremath{)} \HOLTyOp{CCS}}'')
as \emph{multi-hole CCS contexts}. This choice has a significant advantage
over \emph{one-hole contexts}, as each hole corresponds to one occurrence
of the (same) variable in \univariate CCS expressions or equations. Thus
\emph{contexts} can be used both in (pre)congruence definitions and in formulating
the unique solution of equations theorems in the \univariate case. The
precise definition of CCS contexts is inductive:
%
%a4.7 #&#
\begin{alltt}
\HOLTokenTurnstile{} \HOLConst{CONTEXT} \ensuremath{(}\HOLTokenLambda{}\HOLBoundVar{t}. \HOLBoundVar{t}\ensuremath{)} \HOLSymConst{\HOLTokenConj{}} \ensuremath{(}\HOLSymConst{\HOLTokenForall{}}\HOLBoundVar{p}. \HOLConst{CONTEXT} \ensuremath{(}\HOLTokenLambda{}\HOLBoundVar{t}. \HOLBoundVar{p}\ensuremath{)}\ensuremath{)} \HOLSymConst{\HOLTokenConj{}}
   \ensuremath{(}\HOLSymConst{\HOLTokenForall{}}\HOLBoundVar{a} \HOLBoundVar{e}. \HOLConst{CONTEXT} \HOLBoundVar{e} \HOLSymConst{\HOLTokenImp{}} \HOLConst{CONTEXT} \ensuremath{(}\HOLTokenLambda{}\HOLBoundVar{t}. \HOLBoundVar{a}\HOLSymConst{\ensuremath{\ldotp}}\HOLBoundVar{e} \HOLBoundVar{t}\ensuremath{)}\ensuremath{)} \HOLSymConst{\HOLTokenConj{}}
   \ensuremath{(}\HOLSymConst{\HOLTokenForall{}}\ensuremath{\HOLBoundVar{e}\sb{\mathrm{1}}} \ensuremath{\HOLBoundVar{e}\sb{\mathrm{2}}}. \HOLConst{CONTEXT} \ensuremath{\HOLBoundVar{e}\sb{\mathrm{1}}} \HOLSymConst{\HOLTokenConj{}} \HOLConst{CONTEXT} \ensuremath{\HOLBoundVar{e}\sb{\mathrm{2}}} \HOLSymConst{\HOLTokenImp{}} \HOLConst{CONTEXT} \ensuremath{(}\HOLTokenLambda{}\HOLBoundVar{t}. \ensuremath{\HOLBoundVar{e}\sb{\mathrm{1}}} \HOLBoundVar{t} \HOLSymConst{\ensuremath{+}} \ensuremath{\HOLBoundVar{e}\sb{\mathrm{2}}} \HOLBoundVar{t}\ensuremath{)}\ensuremath{)} \HOLSymConst{\HOLTokenConj{}}
   \ensuremath{(}\HOLSymConst{\HOLTokenForall{}}\ensuremath{\HOLBoundVar{e}\sb{\mathrm{1}}} \ensuremath{\HOLBoundVar{e}\sb{\mathrm{2}}}. \HOLConst{CONTEXT} \ensuremath{\HOLBoundVar{e}\sb{\mathrm{1}}} \HOLSymConst{\HOLTokenConj{}} \HOLConst{CONTEXT} \ensuremath{\HOLBoundVar{e}\sb{\mathrm{2}}} \HOLSymConst{\HOLTokenImp{}} \HOLConst{CONTEXT} \ensuremath{(}\HOLTokenLambda{}\HOLBoundVar{t}. \ensuremath{\HOLBoundVar{e}\sb{\mathrm{1}}} \HOLBoundVar{t} \HOLSymConst{\ensuremath{\mid}} \ensuremath{\HOLBoundVar{e}\sb{\mathrm{2}}} \HOLBoundVar{t}\ensuremath{)}\ensuremath{)} \HOLSymConst{\HOLTokenConj{}}
   \ensuremath{(}\HOLSymConst{\HOLTokenForall{}}\HOLBoundVar{L} \HOLBoundVar{e}. \HOLConst{CONTEXT} \HOLBoundVar{e} \HOLSymConst{\HOLTokenImp{}} \HOLConst{CONTEXT} \ensuremath{(}\HOLTokenLambda{}\HOLBoundVar{t}. \ensuremath{(\nu}\HOLBoundVar{L}\ensuremath{)} \ensuremath{(}\HOLBoundVar{e} \HOLBoundVar{t}\ensuremath{)}\ensuremath{)}\ensuremath{)} \HOLSymConst{\HOLTokenConj{}}
   \HOLSymConst{\HOLTokenForall{}}\HOLBoundVar{rf} \HOLBoundVar{e}. \HOLConst{CONTEXT} \HOLBoundVar{e} \HOLSymConst{\HOLTokenImp{}} \HOLConst{CONTEXT} \ensuremath{(}\HOLTokenLambda{}\HOLBoundVar{t}. \HOLConst{relab} \ensuremath{(}\HOLBoundVar{e} \HOLBoundVar{t}\ensuremath{)} \HOLBoundVar{rf}\ensuremath{)}\hfill{[CONTEXT\_rules]}
\end{alltt}
%
In the above definition (actually generated by \texttt{HOL\_reln}), for
any process $p$, (\HOLinline{\HOLTokenLambda{}\HOLBoundVar{t}.\\\;\HOLFreeVar{p}})
is a valid context with no hole (similarly to an equation without variables).

Below is the formalisation of \reftext{Definition~\ref{def:guardness}}. A context is
\emph{weakly guarded} (\HOLinline{\HOLConst{WG}}) if each hole is underneath
a prefix:
%
%a4.7 #&#
\begin{alltt}
\HOLTokenTurnstile{} \ensuremath{(}\HOLSymConst{\HOLTokenForall{}}\HOLBoundVar{p}. \HOLConst{WG} \ensuremath{(}\HOLTokenLambda{}\HOLBoundVar{t}. \HOLBoundVar{p}\ensuremath{)}\ensuremath{)} \HOLSymConst{\HOLTokenConj{}} \ensuremath{(}\HOLSymConst{\HOLTokenForall{}}\HOLBoundVar{a} \HOLBoundVar{e}. \HOLConst{CONTEXT} \HOLBoundVar{e} \HOLSymConst{\HOLTokenImp{}} \HOLConst{WG} \ensuremath{(}\HOLTokenLambda{}\HOLBoundVar{t}. \HOLBoundVar{a}\HOLSymConst{\ensuremath{\ldotp}}\HOLBoundVar{e} \HOLBoundVar{t}\ensuremath{)}\ensuremath{)} \HOLSymConst{\HOLTokenConj{}}
   \ensuremath{(}\HOLSymConst{\HOLTokenForall{}}\ensuremath{\HOLBoundVar{e}\sb{\mathrm{1}}} \ensuremath{\HOLBoundVar{e}\sb{\mathrm{2}}}. \HOLConst{WG} \ensuremath{\HOLBoundVar{e}\sb{\mathrm{1}}} \HOLSymConst{\HOLTokenConj{}} \HOLConst{WG} \ensuremath{\HOLBoundVar{e}\sb{\mathrm{2}}} \HOLSymConst{\HOLTokenImp{}} \HOLConst{WG} \ensuremath{(}\HOLTokenLambda{}\HOLBoundVar{t}. \ensuremath{\HOLBoundVar{e}\sb{\mathrm{1}}} \HOLBoundVar{t} \HOLSymConst{\ensuremath{+}} \ensuremath{\HOLBoundVar{e}\sb{\mathrm{2}}} \HOLBoundVar{t}\ensuremath{)}\ensuremath{)} \HOLSymConst{\HOLTokenConj{}}
   \ensuremath{(}\HOLSymConst{\HOLTokenForall{}}\ensuremath{\HOLBoundVar{e}\sb{\mathrm{1}}} \ensuremath{\HOLBoundVar{e}\sb{\mathrm{2}}}. \HOLConst{WG} \ensuremath{\HOLBoundVar{e}\sb{\mathrm{1}}} \HOLSymConst{\HOLTokenConj{}} \HOLConst{WG} \ensuremath{\HOLBoundVar{e}\sb{\mathrm{2}}} \HOLSymConst{\HOLTokenImp{}} \HOLConst{WG} \ensuremath{(}\HOLTokenLambda{}\HOLBoundVar{t}. \ensuremath{\HOLBoundVar{e}\sb{\mathrm{1}}} \HOLBoundVar{t} \HOLSymConst{\ensuremath{\mid}} \ensuremath{\HOLBoundVar{e}\sb{\mathrm{2}}} \HOLBoundVar{t}\ensuremath{)}\ensuremath{)} \HOLSymConst{\HOLTokenConj{}}
   \ensuremath{(}\HOLSymConst{\HOLTokenForall{}}\HOLBoundVar{L} \HOLBoundVar{e}. \HOLConst{WG} \HOLBoundVar{e} \HOLSymConst{\HOLTokenImp{}} \HOLConst{WG} \ensuremath{(}\HOLTokenLambda{}\HOLBoundVar{t}. \ensuremath{(\nu}\HOLBoundVar{L}\ensuremath{)} \ensuremath{(}\HOLBoundVar{e} \HOLBoundVar{t}\ensuremath{)}\ensuremath{)}\ensuremath{)} \HOLSymConst{\HOLTokenConj{}}
   \HOLSymConst{\HOLTokenForall{}}\HOLBoundVar{rf} \HOLBoundVar{e}. \HOLConst{WG} \HOLBoundVar{e} \HOLSymConst{\HOLTokenImp{}} \HOLConst{WG} \ensuremath{(}\HOLTokenLambda{}\HOLBoundVar{t}. \HOLConst{relab} \ensuremath{(}\HOLBoundVar{e} \HOLBoundVar{t}\ensuremath{)} \HOLBoundVar{rf}\ensuremath{)}\hfill{[WG\_rules]}
\end{alltt}
%
Notice the differences between a weakly guarded context and an ordinary
context: $(\lambda t. t)$ is not weakly guarded as the variable $t$ is
directly exposed without any prefixed action, while
$(\lambda t. a.e[t])$ is weakly guarded as long as $e[\cdot ]$ is a context,
which is not necessary weakly guarded.

A context is \emph{(strongly)} guarded (\HOLinline{\HOLConst{SG}}) if each
hole is underneath a \emph{visible} prefix:
%
%a4.7 #&#
\begin{alltt}
\HOLTokenTurnstile{} \ensuremath{(}\HOLSymConst{\HOLTokenForall{}}\HOLBoundVar{p}. \HOLConst{SG} \ensuremath{(}\HOLTokenLambda{}\HOLBoundVar{t}. \HOLBoundVar{p}\ensuremath{)}\ensuremath{)} \HOLSymConst{\HOLTokenConj{}} \ensuremath{(}\HOLSymConst{\HOLTokenForall{}}\HOLBoundVar{l} \HOLBoundVar{e}. \HOLConst{CONTEXT} \HOLBoundVar{e} \HOLSymConst{\HOLTokenImp{}} \HOLConst{SG} \ensuremath{(}\HOLTokenLambda{}\HOLBoundVar{t}. \HOLConst{label} \HOLBoundVar{l}\HOLSymConst{\ensuremath{\ldotp}}\HOLBoundVar{e} \HOLBoundVar{t}\ensuremath{)}\ensuremath{)} \HOLSymConst{\HOLTokenConj{}}
   \ensuremath{(}\HOLSymConst{\HOLTokenForall{}}\HOLBoundVar{a} \HOLBoundVar{e}. \HOLConst{SG} \HOLBoundVar{e} \HOLSymConst{\HOLTokenImp{}} \HOLConst{SG} \ensuremath{(}\HOLTokenLambda{}\HOLBoundVar{t}. \HOLBoundVar{a}\HOLSymConst{\ensuremath{\ldotp}}\HOLBoundVar{e} \HOLBoundVar{t}\ensuremath{)}\ensuremath{)} \HOLSymConst{\HOLTokenConj{}}
   \ensuremath{(}\HOLSymConst{\HOLTokenForall{}}\ensuremath{\HOLBoundVar{e}\sb{\mathrm{1}}} \ensuremath{\HOLBoundVar{e}\sb{\mathrm{2}}}. \HOLConst{SG} \ensuremath{\HOLBoundVar{e}\sb{\mathrm{1}}} \HOLSymConst{\HOLTokenConj{}} \HOLConst{SG} \ensuremath{\HOLBoundVar{e}\sb{\mathrm{2}}} \HOLSymConst{\HOLTokenImp{}} \HOLConst{SG} \ensuremath{(}\HOLTokenLambda{}\HOLBoundVar{t}. \ensuremath{\HOLBoundVar{e}\sb{\mathrm{1}}} \HOLBoundVar{t} \HOLSymConst{\ensuremath{+}} \ensuremath{\HOLBoundVar{e}\sb{\mathrm{2}}} \HOLBoundVar{t}\ensuremath{)}\ensuremath{)} \HOLSymConst{\HOLTokenConj{}}
   \ensuremath{(}\HOLSymConst{\HOLTokenForall{}}\ensuremath{\HOLBoundVar{e}\sb{\mathrm{1}}} \ensuremath{\HOLBoundVar{e}\sb{\mathrm{2}}}. \HOLConst{SG} \ensuremath{\HOLBoundVar{e}\sb{\mathrm{1}}} \HOLSymConst{\HOLTokenConj{}} \HOLConst{SG} \ensuremath{\HOLBoundVar{e}\sb{\mathrm{2}}} \HOLSymConst{\HOLTokenImp{}} \HOLConst{SG} \ensuremath{(}\HOLTokenLambda{}\HOLBoundVar{t}. \ensuremath{\HOLBoundVar{e}\sb{\mathrm{1}}} \HOLBoundVar{t} \HOLSymConst{\ensuremath{\mid}} \ensuremath{\HOLBoundVar{e}\sb{\mathrm{2}}} \HOLBoundVar{t}\ensuremath{)}\ensuremath{)} \HOLSymConst{\HOLTokenConj{}}
   \ensuremath{(}\HOLSymConst{\HOLTokenForall{}}\HOLBoundVar{L} \HOLBoundVar{e}. \HOLConst{SG} \HOLBoundVar{e} \HOLSymConst{\HOLTokenImp{}} \HOLConst{SG} \ensuremath{(}\HOLTokenLambda{}\HOLBoundVar{t}. \ensuremath{(\nu}\HOLBoundVar{L}\ensuremath{)} \ensuremath{(}\HOLBoundVar{e} \HOLBoundVar{t}\ensuremath{)}\ensuremath{)}\ensuremath{)} \HOLSymConst{\HOLTokenConj{}}
   \HOLSymConst{\HOLTokenForall{}}\HOLBoundVar{rf} \HOLBoundVar{e}. \HOLConst{SG} \HOLBoundVar{e} \HOLSymConst{\HOLTokenImp{}} \HOLConst{SG} \ensuremath{(}\HOLTokenLambda{}\HOLBoundVar{t}. \HOLConst{relab} \ensuremath{(}\HOLBoundVar{e} \HOLBoundVar{t}\ensuremath{)} \HOLBoundVar{rf}\ensuremath{)}\hfill{[SG\_rules]}
\end{alltt}

A context is \emph{sequential} (\HOLinline{\HOLConst{SEQ}}) if each of its
\emph{subcontexts} with a hole, apart from the hole itself, is in forms
of prefixes or sums:
%
%a4.7 #&#
\begin{alltt}
\HOLTokenTurnstile{} \HOLConst{SEQ} \ensuremath{(}\HOLTokenLambda{}\HOLBoundVar{t}. \HOLBoundVar{t}\ensuremath{)} \HOLSymConst{\HOLTokenConj{}} \ensuremath{(}\HOLSymConst{\HOLTokenForall{}}\HOLBoundVar{p}. \HOLConst{SEQ} \ensuremath{(}\HOLTokenLambda{}\HOLBoundVar{t}. \HOLBoundVar{p}\ensuremath{)}\ensuremath{)} \HOLSymConst{\HOLTokenConj{}} \ensuremath{(}\HOLSymConst{\HOLTokenForall{}}\HOLBoundVar{a} \HOLBoundVar{e}. \HOLConst{SEQ} \HOLBoundVar{e} \HOLSymConst{\HOLTokenImp{}} \HOLConst{SEQ} \ensuremath{(}\HOLTokenLambda{}\HOLBoundVar{t}. \HOLBoundVar{a}\HOLSymConst{\ensuremath{\ldotp}}\HOLBoundVar{e} \HOLBoundVar{t}\ensuremath{)}\ensuremath{)} \HOLSymConst{\HOLTokenConj{}}
   \HOLSymConst{\HOLTokenForall{}}\ensuremath{\HOLBoundVar{e}\sb{\mathrm{1}}} \ensuremath{\HOLBoundVar{e}\sb{\mathrm{2}}}. \HOLConst{SEQ} \ensuremath{\HOLBoundVar{e}\sb{\mathrm{1}}} \HOLSymConst{\HOLTokenConj{}} \HOLConst{SEQ} \ensuremath{\HOLBoundVar{e}\sb{\mathrm{2}}} \HOLSymConst{\HOLTokenImp{}} \HOLConst{SEQ} \ensuremath{(}\HOLTokenLambda{}\HOLBoundVar{t}. \ensuremath{\HOLBoundVar{e}\sb{\mathrm{1}}} \HOLBoundVar{t} \HOLSymConst{\ensuremath{+}} \ensuremath{\HOLBoundVar{e}\sb{\mathrm{2}}} \HOLBoundVar{t}\ensuremath{)}\hfill{[SEQ\_rules]}
\end{alltt}

In the same manner, we can also define variants of contexts (\HOLinline{\HOLConst{GCONTEXT}})
and weakly guarded contexts (\HOLinline{\HOLConst{WGS}}) in which only
guarded sums are allowed:
%
%a4.7 #&#
\begin{alltt}
\HOLTokenTurnstile{} \HOLConst{GCONTEXT} \ensuremath{(}\HOLTokenLambda{}\HOLBoundVar{t}. \HOLBoundVar{t}\ensuremath{)} \HOLSymConst{\HOLTokenConj{}} \ensuremath{(}\HOLSymConst{\HOLTokenForall{}}\HOLBoundVar{p}. \HOLConst{GCONTEXT} \ensuremath{(}\HOLTokenLambda{}\HOLBoundVar{t}. \HOLBoundVar{p}\ensuremath{)}\ensuremath{)} \HOLSymConst{\HOLTokenConj{}}
   \ensuremath{(}\HOLSymConst{\HOLTokenForall{}}\HOLBoundVar{a} \HOLBoundVar{e}. \HOLConst{GCONTEXT} \HOLBoundVar{e} \HOLSymConst{\HOLTokenImp{}} \HOLConst{GCONTEXT} \ensuremath{(}\HOLTokenLambda{}\HOLBoundVar{t}. \HOLBoundVar{a}\HOLSymConst{\ensuremath{\ldotp}}\HOLBoundVar{e} \HOLBoundVar{t}\ensuremath{)}\ensuremath{)} \HOLSymConst{\HOLTokenConj{}}
   \ensuremath{(}\HOLSymConst{\HOLTokenForall{}}\ensuremath{\HOLBoundVar{a}\sb{\mathrm{1}}} \ensuremath{\HOLBoundVar{a}\sb{\mathrm{2}}} \ensuremath{\HOLBoundVar{e}\sb{\mathrm{1}}} \ensuremath{\HOLBoundVar{e}\sb{\mathrm{2}}}.
        \HOLConst{GCONTEXT} \ensuremath{\HOLBoundVar{e}\sb{\mathrm{1}}} \HOLSymConst{\HOLTokenConj{}} \HOLConst{GCONTEXT} \ensuremath{\HOLBoundVar{e}\sb{\mathrm{2}}} \HOLSymConst{\HOLTokenImp{}} \HOLConst{GCONTEXT} \ensuremath{(}\HOLTokenLambda{}\HOLBoundVar{t}. \ensuremath{\HOLBoundVar{a}\sb{\mathrm{1}}}\HOLSymConst{\ensuremath{\ldotp}}\ensuremath{\HOLBoundVar{e}\sb{\mathrm{1}}} \HOLBoundVar{t} \HOLSymConst{\ensuremath{+}} \ensuremath{\HOLBoundVar{a}\sb{\mathrm{2}}}\HOLSymConst{\ensuremath{\ldotp}}\ensuremath{\HOLBoundVar{e}\sb{\mathrm{2}}} \HOLBoundVar{t}\ensuremath{)}\ensuremath{)} \HOLSymConst{\HOLTokenConj{}}
   \ensuremath{(}\HOLSymConst{\HOLTokenForall{}}\ensuremath{\HOLBoundVar{e}\sb{\mathrm{1}}} \ensuremath{\HOLBoundVar{e}\sb{\mathrm{2}}}. \HOLConst{GCONTEXT} \ensuremath{\HOLBoundVar{e}\sb{\mathrm{1}}} \HOLSymConst{\HOLTokenConj{}} \HOLConst{GCONTEXT} \ensuremath{\HOLBoundVar{e}\sb{\mathrm{2}}} \HOLSymConst{\HOLTokenImp{}} \HOLConst{GCONTEXT} \ensuremath{(}\HOLTokenLambda{}\HOLBoundVar{t}. \ensuremath{\HOLBoundVar{e}\sb{\mathrm{1}}} \HOLBoundVar{t} \HOLSymConst{\ensuremath{\mid}} \ensuremath{\HOLBoundVar{e}\sb{\mathrm{2}}} \HOLBoundVar{t}\ensuremath{)}\ensuremath{)} \HOLSymConst{\HOLTokenConj{}}
   \ensuremath{(}\HOLSymConst{\HOLTokenForall{}}\HOLBoundVar{L} \HOLBoundVar{e}. \HOLConst{GCONTEXT} \HOLBoundVar{e} \HOLSymConst{\HOLTokenImp{}} \HOLConst{GCONTEXT} \ensuremath{(}\HOLTokenLambda{}\HOLBoundVar{t}. \ensuremath{(\nu}\HOLBoundVar{L}\ensuremath{)} \ensuremath{(}\HOLBoundVar{e} \HOLBoundVar{t}\ensuremath{)}\ensuremath{)}\ensuremath{)} \HOLSymConst{\HOLTokenConj{}}
   \HOLSymConst{\HOLTokenForall{}}\HOLBoundVar{rf} \HOLBoundVar{e}. \HOLConst{GCONTEXT} \HOLBoundVar{e} \HOLSymConst{\HOLTokenImp{}} \HOLConst{GCONTEXT} \ensuremath{(}\HOLTokenLambda{}\HOLBoundVar{t}. \HOLConst{relab} \ensuremath{(}\HOLBoundVar{e} \HOLBoundVar{t}\ensuremath{)} \HOLBoundVar{rf}\ensuremath{)}\hfill{[GCONTEXT\_rules]}
\end{alltt}
%
%a4.7 #&#
\begin{alltt}
\HOLTokenTurnstile{} \ensuremath{(}\HOLSymConst{\HOLTokenForall{}}\HOLBoundVar{p}. \HOLConst{WGS} \ensuremath{(}\HOLTokenLambda{}\HOLBoundVar{t}. \HOLBoundVar{p}\ensuremath{)}\ensuremath{)} \HOLSymConst{\HOLTokenConj{}} \ensuremath{(}\HOLSymConst{\HOLTokenForall{}}\HOLBoundVar{a} \HOLBoundVar{e}. \HOLConst{GCONTEXT} \HOLBoundVar{e} \HOLSymConst{\HOLTokenImp{}} \HOLConst{WGS} \ensuremath{(}\HOLTokenLambda{}\HOLBoundVar{t}. \HOLBoundVar{a}\HOLSymConst{\ensuremath{\ldotp}}\HOLBoundVar{e} \HOLBoundVar{t}\ensuremath{)}\ensuremath{)} \HOLSymConst{\HOLTokenConj{}}
   \ensuremath{(}\HOLSymConst{\HOLTokenForall{}}\ensuremath{\HOLBoundVar{a}\sb{\mathrm{1}}} \ensuremath{\HOLBoundVar{a}\sb{\mathrm{2}}} \ensuremath{\HOLBoundVar{e}\sb{\mathrm{1}}} \ensuremath{\HOLBoundVar{e}\sb{\mathrm{2}}}. \HOLConst{GCONTEXT} \ensuremath{\HOLBoundVar{e}\sb{\mathrm{1}}} \HOLSymConst{\HOLTokenConj{}} \HOLConst{GCONTEXT} \ensuremath{\HOLBoundVar{e}\sb{\mathrm{2}}} \HOLSymConst{\HOLTokenImp{}} \HOLConst{WGS} \ensuremath{(}\HOLTokenLambda{}\HOLBoundVar{t}. \ensuremath{\HOLBoundVar{a}\sb{\mathrm{1}}}\HOLSymConst{\ensuremath{\ldotp}}\ensuremath{\HOLBoundVar{e}\sb{\mathrm{1}}} \HOLBoundVar{t} \HOLSymConst{\ensuremath{+}} \ensuremath{\HOLBoundVar{a}\sb{\mathrm{2}}}\HOLSymConst{\ensuremath{\ldotp}}\ensuremath{\HOLBoundVar{e}\sb{\mathrm{2}}} \HOLBoundVar{t}\ensuremath{)}\ensuremath{)} \HOLSymConst{\HOLTokenConj{}}
   \ensuremath{(}\HOLSymConst{\HOLTokenForall{}}\ensuremath{\HOLBoundVar{e}\sb{\mathrm{1}}} \ensuremath{\HOLBoundVar{e}\sb{\mathrm{2}}}. \HOLConst{WGS} \ensuremath{\HOLBoundVar{e}\sb{\mathrm{1}}} \HOLSymConst{\HOLTokenConj{}} \HOLConst{WGS} \ensuremath{\HOLBoundVar{e}\sb{\mathrm{2}}} \HOLSymConst{\HOLTokenImp{}} \HOLConst{WGS} \ensuremath{(}\HOLTokenLambda{}\HOLBoundVar{t}. \ensuremath{\HOLBoundVar{e}\sb{\mathrm{1}}} \HOLBoundVar{t} \HOLSymConst{\ensuremath{\mid}} \ensuremath{\HOLBoundVar{e}\sb{\mathrm{2}}} \HOLBoundVar{t}\ensuremath{)}\ensuremath{)} \HOLSymConst{\HOLTokenConj{}}
   \ensuremath{(}\HOLSymConst{\HOLTokenForall{}}\HOLBoundVar{L} \HOLBoundVar{e}. \HOLConst{WGS} \HOLBoundVar{e} \HOLSymConst{\HOLTokenImp{}} \HOLConst{WGS} \ensuremath{(}\HOLTokenLambda{}\HOLBoundVar{t}. \ensuremath{(\nu}\HOLBoundVar{L}\ensuremath{)} \ensuremath{(}\HOLBoundVar{e} \HOLBoundVar{t}\ensuremath{)}\ensuremath{)}\ensuremath{)} \HOLSymConst{\HOLTokenConj{}}
   \HOLSymConst{\HOLTokenForall{}}\HOLBoundVar{rf} \HOLBoundVar{e}. \HOLConst{WGS} \HOLBoundVar{e} \HOLSymConst{\HOLTokenImp{}} \HOLConst{WGS} \ensuremath{(}\HOLTokenLambda{}\HOLBoundVar{t}. \HOLConst{relab} \ensuremath{(}\HOLBoundVar{e} \HOLBoundVar{t}\ensuremath{)} \HOLBoundVar{rf}\ensuremath{)}\hfill{[WGS\_rules]}
\end{alltt}

Several lemmas about the above concepts (\HOLinline{\HOLConst{CONTEXT}},
\HOLinline{\HOLConst{WG}}, \HOLinline{\HOLConst{SEQ}}, etc.) are needed
for capturing properties about the relationships among these kinds of contexts
and about their compositions. These proofs are usually tedious and long,
due to multiple levels of inductions on the structure of the contexts.

A (pre)congruence is a relation on CCS processes defined on top of
\HOLinline{\HOLConst{CONTEXT}}. The only difference between congruence
and precongruence is that the former is an equivalence, while the latter
can just be a preorder:
%
%a4.7 #&#
\begin{alltt}
   \HOLConst{congruence} \HOLFreeVar{R} \HOLTokenDefEquality{}
     \HOLConst{equivalence} \HOLFreeVar{R} \HOLSymConst{\HOLTokenConj{}} \HOLSymConst{\HOLTokenForall{}}\HOLBoundVar{x} \HOLBoundVar{y} \HOLBoundVar{ctx}. \HOLConst{CONTEXT} \HOLBoundVar{ctx} \HOLSymConst{\HOLTokenImp{}} \HOLFreeVar{R} \HOLBoundVar{x} \HOLBoundVar{y} \HOLSymConst{\HOLTokenImp{}} \HOLFreeVar{R} \ensuremath{(}\HOLBoundVar{ctx} \HOLBoundVar{x}\ensuremath{)} \ensuremath{(}\HOLBoundVar{ctx} \HOLBoundVar{y}\ensuremath{)}\hfill{[congruence]}
\end{alltt}

\begin{alltt}
   \HOLConst{precongruence} \HOLFreeVar{R} \HOLTokenDefEquality{}
     \HOLConst{PreOrder} \HOLFreeVar{R} \HOLSymConst{\HOLTokenConj{}} \HOLSymConst{\HOLTokenForall{}}\HOLBoundVar{x} \HOLBoundVar{y} \HOLBoundVar{ctx}. \HOLConst{CONTEXT} \HOLBoundVar{ctx} \HOLSymConst{\HOLTokenImp{}} \HOLFreeVar{R} \HOLBoundVar{x} \HOLBoundVar{y} \HOLSymConst{\HOLTokenImp{}} \HOLFreeVar{R} \ensuremath{(}\HOLBoundVar{ctx} \HOLBoundVar{x}\ensuremath{)} \ensuremath{(}\HOLBoundVar{ctx} \HOLBoundVar{y}\ensuremath{)}\hfill{[precongruence]}
\end{alltt}

Both strong bisimilarity ($\sim $) and rooted bisimilarity ($\approx ^{c}$)
are congruence relations:
%
%a4.7 #&#
\begin{alltt}
\HOLTokenTurnstile{} \HOLConst{congruence} \HOLConst{STRONG\_EQUIV}\hfill{[STRONG\_EQUIV\_congruence]}
\HOLTokenTurnstile{} \HOLConst{congruence} \HOLConst{OBS\_CONGR}\hfill{[OBS\_CONGR\_congruence]}
\end{alltt}

Although weak bisimilarity ($\approx $) is \emph{not} a congruence with
respect to~\HOLinline{\HOLConst{CONTEXT}}, it is substitutive with respect
to~\HOLinline{\HOLConst{GCONTEXT}} as $\approx $ is preserved by guarded
sums. Similarily, contraction ($\mcontrBIS $) is substitutive with respect
to \HOLinline{\HOLConst{GCONTEXT}}. Rooted contraction ($\rcontr $, or
\HOLinline{\HOLConst{OBS\_contracts}} in HOL), on the other hand, is indeed
a precongruence:
%
%a4.7 #&#
\begin{alltt}
\HOLTokenTurnstile{} \HOLConst{precongruence} \HOLConst{OBS\_contracts}\hfill{[OBS\_contracts\_precongruence]}
\end{alltt}

%s4.8 #&#
\subsection{Coarsest (pre)congruence contained in $\approx $ ($\succeq _{\mathrm{bis}}$)}
%%LEAP%%%\label{sec4.8}
 \label{s:coarsest}

In this section we give a proof of the second part of \reftext{Theorem~\ref{t:rapproxCongruence}},
i.e. $\rapprox $ is the coarsest congruence contained in $\wbvtex $. The
general form of this theorem is the following one~\cite{van2005characterisation,Gorrieri:2015jt,Mil89}:
%
%p4.8 #&#
\begin{proposition}
 \label{prop:coarsest}
Rooted bisimilarity ($\rapprox $) is the coarsest congruence contained
in weak bisimilarity ($\wbvtex $):
%
%e1 #&#
\begin{equation}
 \label{eq:coarsest}
\forall p\ \ q.\ p\ \rapprox \ \! q\ \Longleftrightarrow \ ( \forall r.
\ p\ +\
r\ \approx \ q\ +\ r ).
\end{equation}
%
\end{proposition}
%
From left to right \reftext{(\ref{eq:coarsest})} trivially holds, due to the substitutivity
of $\rapprox $ for summation and the fact that $\rapprox $ implies
$\wbvtex $: (Thus we are only interested in \reftext{(\ref{eq:coarsest})} from right
to left.)
%
%a4.8 #&#
\begin{alltt}
\HOLTokenTurnstile{} \HOLSymConst{\HOLTokenForall{}}\HOLBoundVar{p} \HOLBoundVar{q}. \HOLBoundVar{p} \HOLSymConst{\HOLTokenObsCongr} \HOLBoundVar{q} \HOLSymConst{\HOLTokenImp{}} \HOLSymConst{\HOLTokenForall{}}\HOLBoundVar{r}. \HOLBoundVar{p} \HOLSymConst{\ensuremath{+}} \HOLBoundVar{r} \HOLSymConst{\HOLTokenWeakEQ} \HOLBoundVar{q} \HOLSymConst{\ensuremath{+}} \HOLBoundVar{r}\hfill{[COARSEST\_CONGR\_LR]}
\end{alltt}

The formalisation of this theorem presents some delicate aspects. For instance,
within our CCS syntax which supports only binary sums, one way to prove
\reftext{Proposition~\ref{prop:coarsest}} is to add an hypothesis that the involved
processes do not use all available labels. Indeed, this is the standard
argument by Milner~\citep[p.~153]{Mil89}. Formalising this result, however,
requires a detailed treatment of free and bound names (of labels) of CCS
processes, with the restriction operator acting as a binder. In our actual
formalisation, instead, we assume the weaker hypothesis that all
\emph{immediate weak} derivatives of $p$ and $q$ do not use all available
labels. We call this the \emph{free action} property:
%
%a4.8 #&#
\begin{alltt}
   \HOLConst{free\_action} \HOLFreeVar{p} \HOLTokenDefEquality{} \HOLSymConst{\HOLTokenExists{}}\HOLBoundVar{a}. \HOLSymConst{\HOLTokenForall{}}\ensuremath{\HOLBoundVar{p}\sp{\prime}}. \HOLSymConst{\HOLTokenNeg{}}\ensuremath{(}\HOLFreeVar{p} \HOLTokenWeakTransBegin\HOLConst{label} \HOLBoundVar{a}\HOLTokenWeakTransEnd \ensuremath{\HOLBoundVar{p}\sp{\prime}}\ensuremath{)}\hfill{[free\_action\_def]}
\end{alltt}

Now we show how \reftext{(\ref{eq:coarsest})} is connected with the statement of
\reftext{Proposition~\ref{prop:coarsest}}, and prove it under the free action assumptions.
The coarsest congruence contained in (weak) bisimilarity, namely
\emph{bisimilarity congruence} (\texttt{WEAK\_CONGR} in HOL), is the
\emph{composition closure} (\texttt{CC}) of (weak) bisimilarity:
%
%a4.8 #&#
\begin{alltt}
   \HOLConst{CC} \HOLFreeVar{R} \HOLTokenDefEquality{} \ensuremath{(}\HOLTokenLambda{}\HOLBoundVar{g} \HOLBoundVar{h}. \HOLSymConst{\HOLTokenForall{}}\HOLBoundVar{c}. \HOLConst{CONTEXT} \HOLBoundVar{c} \HOLSymConst{\HOLTokenImp{}} \HOLFreeVar{R} \ensuremath{(}\HOLBoundVar{c} \HOLBoundVar{g}\ensuremath{)} \ensuremath{(}\HOLBoundVar{c} \HOLBoundVar{h}\ensuremath{)}\ensuremath{)}\hfill{[CC\_def]}
   \HOLConst{WEAK\_CONGR} \HOLTokenDefEquality{} \HOLConst{CC} \HOLConst{WEAK\_EQUIV}\hfill{[WEAK\_CONGR]}
\end{alltt}
%
We do not need to put $R\ g\ h$ into the antecedents of
\texttt{CC\_def}, as this is anyhow obtained from the trivial context
$(\lambda x.\,x)$. The next result shows that, for any binary relation
$R$ on CCS processes, the composition closure of $R$ is always at least
as fine as $R$ (here $\subseteq _r$ stands for
\emph{relational subset}):
%
%a4.8 #&#
\begin{alltt}
\HOLTokenTurnstile{} \HOLSymConst{\HOLTokenForall{}}\HOLBoundVar{R}. \HOLConst{CC} \HOLBoundVar{R} \HOLSymConst{\HOLTokenRSubset{}} \HOLBoundVar{R}\hfill{[CC\_is\_finer]}
\end{alltt}
%
Furthermore, we prove that any (pre)congruence contained in $R$, that itself
needs not to be a (pre)congruence, is contained in the composition closure
of $R$ (hence the composition closure is indeed the coarsest one):
%
%a4.8 #&#
\begin{alltt}
\HOLTokenTurnstile{} \HOLSymConst{\HOLTokenForall{}}\HOLBoundVar{R} \ensuremath{\HOLBoundVar{R}\sp{\prime}}. \HOLConst{congruence} \ensuremath{\HOLBoundVar{R}\sp{\prime}} \HOLSymConst{\HOLTokenConj{}} \ensuremath{\HOLBoundVar{R}\sp{\prime}} \HOLSymConst{\HOLTokenRSubset{}} \HOLBoundVar{R} \HOLSymConst{\HOLTokenImp{}} \ensuremath{\HOLBoundVar{R}\sp{\prime}} \HOLSymConst{\HOLTokenRSubset{}} \HOLConst{CC} \HOLBoundVar{R}\hfill{[CC\_is\_coarsest]}
\HOLTokenTurnstile{} \HOLSymConst{\HOLTokenForall{}}\HOLBoundVar{R} \ensuremath{\HOLBoundVar{R}\sp{\prime}}. \HOLConst{precongruence} \ensuremath{\HOLBoundVar{R}\sp{\prime}} \HOLSymConst{\HOLTokenConj{}} \ensuremath{\HOLBoundVar{R}\sp{\prime}} \HOLSymConst{\HOLTokenRSubset{}} \HOLBoundVar{R} \HOLSymConst{\HOLTokenImp{}} \ensuremath{\HOLBoundVar{R}\sp{\prime}} \HOLSymConst{\HOLTokenRSubset{}} \HOLConst{CC} \HOLBoundVar{R}\hfill{[PCC\_is\_coarsest]}
\end{alltt}

Given the central role of summation, we also consider the relation closure
of bisimilarity with respect to summation, called
\emph{equivalence compatible with summation} (\texttt{SUM\_EQUIV}):
%
%a4.8 #&#
\begin{alltt}
   \HOLConst{SUM\_EQUIV} \HOLTokenDefEquality{} \ensuremath{(}\HOLTokenLambda{}\HOLBoundVar{p} \HOLBoundVar{q}. \HOLSymConst{\HOLTokenForall{}}\HOLBoundVar{r}. \HOLBoundVar{p} \HOLSymConst{\ensuremath{+}} \HOLBoundVar{r} \HOLSymConst{\HOLTokenWeakEQ} \HOLBoundVar{q} \HOLSymConst{\ensuremath{+}} \HOLBoundVar{r}\ensuremath{)}\hfill{[SUM\_EQUIV]}
\end{alltt}

Rooted bisimilarity $\rapprox $ (as a congruence contained in
$\wbvtex $) is now contained in \texttt{WEAK\_CONGR}, which in turn is trivially
contained in \texttt{SUM\_EQUIV}, as shown in \reftext{Fig.~\ref{fig:relationship}}.
Thus, to prove \reftext{Proposition~\ref{prop:coarsest}}, the crux is to prove that
\texttt{SUM\_EQUIV} is contained in $\rapprox $, making all three relations
($\rapprox $, \texttt{WEAK\_CONGR} and \texttt{SUM\_EQUIV}) coincide:
%
%f6 #&#
\begin{figure}%[ht]
%
%d4.8 #&#
\begin{sgmlfig}\normalsize
%
$$\xymatrix{
{\textrm{Weak bisimilarity } (\approx )} & {\textrm{Equiv.
compatible with summation (\texttt{SUM\_EQUIV})}}
\ar@/^3ex/[ldd]^{\supseteq \; ?}\\
{\textrm{Bisimilarity congruence (\texttt{WEAK\_CONGR})}}
\ar[u]^{\subseteq} \ar[ru]^{\subseteq} \\
{\textrm{Rooted bisimilarity } (\rapprox )} \ar[u]^{\subseteq}
}$$
\end{sgmlfig}
%
\caption{Relationships between several equivalences and $\wbvtex $.}
 \label{fig:relationship}
\end{figure}\vspace{-3pt}
%e2 #&#
\begin{equation}
 \label{equa:pq}
\forall p\ \ q.\ ( \forall r.\ p\ +\ r \;\approx \; q\ +\ r ) \
\Longrightarrow \ p\ \rapprox \ \! q.
\end{equation}


Here is the formalisation of \reftext{(\ref{equa:pq})} under free action hypothesis:\vspace{-3pt}
%
%t4.9 #&#
\begin{theorem}[\texttt{COARSEST\_CONGR\_RL}]
 \label{thm:coarsestR}
Under the free action hypothesis, $\rapprox $ is coarsest congruence contained
in $\wbvtex $.\vspace{-3pt}
%
%a4.9 #&#
\begin{alltt}
\HOLTokenTurnstile{} \HOLSymConst{\HOLTokenForall{}}\HOLBoundVar{p} \HOLBoundVar{q}. \HOLConst{free\_action} \HOLBoundVar{p} \HOLSymConst{\HOLTokenConj{}} \HOLConst{free\_action} \HOLBoundVar{q} \HOLSymConst{\HOLTokenImp{}} \ensuremath{(}\HOLSymConst{\HOLTokenForall{}}\HOLBoundVar{r}. \HOLBoundVar{p} \HOLSymConst{\ensuremath{+}} \HOLBoundVar{r} \HOLSymConst{\HOLTokenWeakEQ} \HOLBoundVar{q} \HOLSymConst{\ensuremath{+}} \HOLBoundVar{r}\ensuremath{)} \HOLSymConst{\HOLTokenImp{}} \HOLBoundVar{p} \HOLSymConst{\HOLTokenObsCongr} \HOLBoundVar{q}
\end{alltt}
%
\end{theorem}

With an almost identical proof, rooted contraction ($\rcontr $) is also
the coarsest precongruence contained in the bisimilarity contraction ($
\mcontrBIS $):
%
%t4.10 #&#
\begin{theorem}[\texttt{COARSEST\_PRECONGR\_RL}]
 \label{thm:coarsestPre}
Under the free action hypothesis, $\mcontrBIS $ is the coarsest precongruence
contained in $\contr $.
%
%a4.10 #&#
\begin{alltt}
\HOLTokenTurnstile{} \HOLSymConst{\HOLTokenForall{}}\HOLBoundVar{p} \HOLBoundVar{q}. \HOLConst{free\_action} \HOLBoundVar{p} \HOLSymConst{\HOLTokenConj{}} \HOLConst{free\_action} \HOLBoundVar{q} \HOLSymConst{\HOLTokenImp{}} \ensuremath{(}\HOLSymConst{\HOLTokenForall{}}\HOLBoundVar{r}. \HOLBoundVar{p} \HOLSymConst{\ensuremath{+}} \HOLBoundVar{r} \HOLSymConst{\HOLTokenContracts{}} \HOLBoundVar{q} \HOLSymConst{\ensuremath{+}} \HOLBoundVar{r}\ensuremath{)} \HOLSymConst{\HOLTokenImp{}} \HOLBoundVar{p} \HOLSymConst{\HOLTokenObsContracts} \HOLBoundVar{q}
\end{alltt}
%
\end{theorem}

The formal proofs of \reftext{\xch{Theorems}{Theorem}~\ref{thm:coarsestR} and \ref{thm:coarsestPre}} precisely follow Milner~\citep[p.~153--154]{Mil89}.
Although Milner requires a stronger hypothesis:
$\mathrm{fn}(p) \cup \mathrm{fn}(q) \neq \mathscr{L}$ (here
$\mathrm{fn}$ stands for \emph{free names}), the actual proof essentially
requires only the above free action property. Indeed, in the proof one
only looks at the immediate weak derivatives of $p$ and $q$, and only requires
that there is an input or output label that never occurs as a label of
the involved transitions.

%s4.9 #&#
\subsection{Arbitrarily many non-bisimilar processes}
%%LEAP%%%\label{sec4.9}
 \label{ss:arbitrarily}

As the type ``\HOLinline{\ensuremath{(}\ensuremath{\alpha}, \ensuremath{\beta}\ensuremath{)} \HOLTyOp{CCS}}''
is parameterized with two type variables, if the type of all label names
$\beta $ has a small cardinality (a singleton in the extreme case), it
is possible that the processes of \reftext{Proposition~\ref{prop:coarsest}} use all
available labels, and thus the free action hypothesis does not hold. In
this case, it is still possible to prove \reftext{Proposition~\ref{prop:coarsest}};
however, due to some limitations of HOL itself we have to assume that the
processes are \emph{finite-state}, i.e. the set of all their derivatives
is finite. The original proof, due to van Glabbeek~\cite{van2005characterisation},
does not require finite-state CCS. Here is the main theorem:
%
%t4.11 #&#
\begin{theorem}[\texttt{COARSEST\_CONGR\_FINITE}]
 \label{thm:coarsestfiniteState}
For finite-state CCS, $\rapprox $ is the coarsest congruence contained
in $\wbvtex $:
%
%a4.11 #&#
\begin{alltt}
\HOLTokenTurnstile{} \HOLSymConst{\HOLTokenForall{}}\HOLBoundVar{p} \HOLBoundVar{q}. \HOLConst{finite\_state} \HOLBoundVar{p} \HOLSymConst{\HOLTokenConj{}} \HOLConst{finite\_state} \HOLBoundVar{q} \HOLSymConst{\HOLTokenImp{}} \ensuremath{(}\HOLBoundVar{p} \HOLSymConst{\HOLTokenObsCongr} \HOLBoundVar{q} \HOLSymConst{\HOLTokenEquiv{}} \HOLSymConst{\HOLTokenForall{}}\HOLBoundVar{r}. \HOLBoundVar{p} \HOLSymConst{\ensuremath{+}} \HOLBoundVar{r} \HOLSymConst{\HOLTokenWeakEQ} \HOLBoundVar{q} \HOLSymConst{\ensuremath{+}} \HOLBoundVar{r}\ensuremath{)}
\end{alltt}
%
\end{theorem}

The precise definition of \texttt{finite\_state} used in above theorem will
be given later. We start with a core lemma (\texttt{PROP3\_COMMON}) saying
that, for any two processes $p$ and $q$, if there exists a
\emph{stable} (i.e.~without $\tau $ transitions) process which is not bisimilar
with any weak derivative of $p$ and $q$, then
\HOLinline{\HOLConst{SUM\_EQUIV}} indeed implies rooted bisimilarity ($
\rapprox $)~\cite{van2005characterisation,Tian:2017wrba}:
%
%a4.9 #&#
\begin{alltt}
\HOLTokenTurnstile{} \HOLSymConst{\HOLTokenForall{}}\HOLBoundVar{p} \HOLBoundVar{q}.
       \ensuremath{(}\HOLSymConst{\HOLTokenExists{}}\HOLBoundVar{k}.
            \HOLConst{STABLE} \HOLBoundVar{k} \HOLSymConst{\HOLTokenConj{}} \ensuremath{(}\HOLSymConst{\HOLTokenForall{}}\ensuremath{\HOLBoundVar{p}\sp{\prime}} \HOLBoundVar{u}. \HOLBoundVar{p} \HOLTokenWeakTransBegin\HOLBoundVar{u}\HOLTokenWeakTransEnd \ensuremath{\HOLBoundVar{p}\sp{\prime}} \HOLSymConst{\HOLTokenImp{}} \HOLSymConst{\HOLTokenNeg{}}\ensuremath{(}\ensuremath{\HOLBoundVar{p}\sp{\prime}} \HOLSymConst{\HOLTokenWeakEQ} \HOLBoundVar{k}\ensuremath{)}\ensuremath{)} \HOLSymConst{\HOLTokenConj{}}
            \HOLSymConst{\HOLTokenForall{}}\ensuremath{\HOLBoundVar{q}\sp{\prime}} \HOLBoundVar{u}. \HOLBoundVar{q} \HOLTokenWeakTransBegin\HOLBoundVar{u}\HOLTokenWeakTransEnd \ensuremath{\HOLBoundVar{q}\sp{\prime}} \HOLSymConst{\HOLTokenImp{}} \HOLSymConst{\HOLTokenNeg{}}\ensuremath{(}\ensuremath{\HOLBoundVar{q}\sp{\prime}} \HOLSymConst{\HOLTokenWeakEQ} \HOLBoundVar{k}\ensuremath{)}\ensuremath{)} \HOLSymConst{\HOLTokenImp{}}
       \ensuremath{(}\HOLSymConst{\HOLTokenForall{}}\HOLBoundVar{r}. \HOLBoundVar{p} \HOLSymConst{\ensuremath{+}} \HOLBoundVar{r} \HOLSymConst{\HOLTokenWeakEQ} \HOLBoundVar{q} \HOLSymConst{\ensuremath{+}} \HOLBoundVar{r}\ensuremath{)} \HOLSymConst{\HOLTokenImp{}}
       \HOLBoundVar{p} \HOLSymConst{\HOLTokenObsCongr} \HOLBoundVar{q}\hfill{[PROP3\_COMMON]}
\end{alltt}
%
%a4.9 #&#
\begin{alltt}
   \HOLConst{STABLE} \HOLFreeVar{p} \HOLTokenDefEquality{} \HOLSymConst{\HOLTokenForall{}}\HOLBoundVar{u} \ensuremath{\HOLBoundVar{p}\sp{\prime}}. \HOLFreeVar{p} \HOLTokenTransBegin\HOLBoundVar{u}\HOLTokenTransEnd \ensuremath{\HOLBoundVar{p}\sp{\prime}} \HOLSymConst{\HOLTokenImp{}} \HOLBoundVar{u} \HOLSymConst{\HOLTokenNotEqual{}} \HOLSymConst{\ensuremath{\tau}}\hfill{[STABLE]}
\end{alltt}

To prove \reftext{Theorem~\ref{thm:coarsestfiniteState}}, it only remains to construct
such stable process $k$ for any two finite-state processes $p$ and
$q$. For arbitrary CCS processes, this construction relies on arbitrary
infinite sums of processes (not within our CCS syntax) and transfinite
induction to obtain an arbitrary large sequence of processes that are all
pairwise non-bisimilar, which was firstly introduced by Jan Willem Klop~(see
\cite{van2005characterisation} for some historical notes). We have only
partially formalised van Glabbeek's proof, mostly because our CCS syntax
does not allow infinite summation (and it is not easy to extend it with
this support). Another more important reason is that the typed logic implemented
in various HOL systems (including Isabelle/HOL) is not strong enough to
define a type for all possible ordinals~\cite{norrish2013ordinals} which
is required in van Glabbeek's proof. As the consequence, the formalisation
(\reftext{Theorem~\ref{thm:coarsestfiniteState}}) can only apply to finite-state
CCS.

The above core lemma (\texttt{PROP3\_COMMON}) requires the existence of
a special CCS process, which is not weakly bisimilar to any weak derivative
of the two root processes. There could be infinitely many such subprocesses,
even on finitely branching processes. We can, however, consider the equivalence
classes of CCS processes modulo weak bisimilarity. If there are infinitely
many such classes, then it will be possible to choose one that is distinct
from all the (finitely many) states in the transition graphs of the two
given processes. This can be done by following Klop's \xch{construction}{contruction}. We call
the processes in this construction the ``Klop processes'':
%
%d4.12 #&#
\begin{definition}[Klop processes]
For each ordinal $\lambda $, and an arbitrary chosen action
$a \neq \tau $, define a CCS process $k_\lambda $ as follows:
%
\begin{itemize}
%
\item $k_0 = 0$,
%
\item $k_{\lambda +1} = k_\lambda + a.k_\lambda $ and
%
\item for $\lambda $ a limit ordinal,
$k_\lambda = \sum _{\mu < \lambda} k_\mu $ (meaning that
$k_\lambda $ is constructed from all graphs $k_\mu $ for
$\mu < \lambda $ by identifying their root).
%
\end{itemize}
%
\end{definition}
%
When processes are finite-state, that is, the number of states in which
a process may evolve by performing transitions is finite, we can use the
following subset of Klop processes, defined as a recursive function (on
natural numbers) in HOL4:
%
%d4.13 #&#
\begin{definition}%
{(Klop processes as recursive function on natural numbers)}
%
%a4.13 #&#
\begin{alltt}
   \HOLConst{KLOP} \HOLFreeVar{a} \HOLNumLit{0} \HOLTokenDefEquality{} \HOLConst{\ensuremath{\mathbf{0}}}
   \HOLConst{KLOP} \HOLFreeVar{a} \ensuremath{(}\HOLConst{SUC} \HOLFreeVar{n}\ensuremath{)} \HOLTokenDefEquality{} \HOLConst{KLOP} \HOLFreeVar{a} \HOLFreeVar{n} \HOLSymConst{\ensuremath{+}} \HOLConst{label} \HOLFreeVar{a}\HOLSymConst{\ensuremath{\ldotp}}\HOLConst{KLOP} \HOLFreeVar{a} \HOLFreeVar{n}\hfill{[KLOP\_def]}
\end{alltt}
%
\end{definition}

Following the inductive structure of the above definition, and using the
SOS rules ($\mathrm{Sum}_1$) and ($\mathrm{Sum}_2$), we can prove the following
properties of Klop functions:
%
%p4.14 #&#
\begin{proposition}%
{(Properties of Klop functions and processes)}
%
\begin{enumerate}
%
\item (All Klop processes are stable)
%
%a1 #&#
\begin{alltt}
\HOLTokenTurnstile{} \HOLConst{STABLE} \ensuremath{(}\HOLConst{KLOP} \HOLFreeVar{a} \HOLFreeVar{n}\ensuremath{)}\hfill[KLOP\_PROP0]
\end{alltt}

%
\item (Any transition from a Klop process leads to a smaller Klop process,
and conversely)
%
%a2 #&#
\begin{alltt}
\HOLTokenTurnstile{} \HOLConst{KLOP} \HOLFreeVar{a} \HOLFreeVar{n} \HOLTokenTransBegin\HOLConst{label} \HOLFreeVar{a}\HOLTokenTransEnd \HOLFreeVar{E} \HOLSymConst{\HOLTokenEquiv{}} \HOLSymConst{\HOLTokenExists{}}\HOLBoundVar{m}. \HOLBoundVar{m} \HOLSymConst{\HOLTokenLt{}} \HOLFreeVar{n} \HOLSymConst{\HOLTokenConj{}} \HOLFreeVar{E} \HOLSymConst{\ensuremath{=}} \HOLConst{KLOP} \HOLFreeVar{a} \HOLBoundVar{m}\hfill{[KLOP\_PROP1]}
\end{alltt}
%
\item (The weak version of the previous property)
%
%a3 #&#
\begin{alltt}
\HOLTokenTurnstile{} \HOLConst{KLOP} \HOLFreeVar{a} \HOLFreeVar{n} \HOLTokenWeakTransBegin\HOLConst{label} \HOLFreeVar{a}\HOLTokenWeakTransEnd \HOLFreeVar{E} \HOLSymConst{\HOLTokenEquiv{}} \HOLSymConst{\HOLTokenExists{}}\HOLBoundVar{m}. \HOLBoundVar{m} \HOLSymConst{\HOLTokenLt{}} \HOLFreeVar{n} \HOLSymConst{\HOLTokenConj{}} \HOLFreeVar{E} \HOLSymConst{\ensuremath{=}} \HOLConst{KLOP} \HOLFreeVar{a} \HOLBoundVar{m}\hfill{[KLOP\_PROP1']}
\end{alltt}
%
\item (All Klop processes are distinct according to strong bisimilarity)
%
%a4 #&#
\begin{alltt}
\HOLTokenTurnstile{} \HOLFreeVar{m} \HOLSymConst{\HOLTokenLt{}} \HOLFreeVar{n} \HOLSymConst{\HOLTokenImp{}} \HOLSymConst{\HOLTokenNeg{}}\ensuremath{(}\HOLConst{KLOP} \HOLFreeVar{a} \HOLFreeVar{m} \HOLSymConst{\HOLTokenStrongEQ} \HOLConst{KLOP} \HOLFreeVar{a} \HOLFreeVar{n}\ensuremath{)}\hfill{[KLOP\_PROP2]}
\end{alltt}
%
\item (All Klop processes are distinct according to weak bisimilarity)
%
%a5 #&#
\begin{alltt}
\HOLTokenTurnstile{} \HOLFreeVar{m} \HOLSymConst{\HOLTokenLt{}} \HOLFreeVar{n} \HOLSymConst{\HOLTokenImp{}} \HOLSymConst{\HOLTokenNeg{}}\ensuremath{(}\HOLConst{KLOP} \HOLFreeVar{a} \HOLFreeVar{m} \HOLSymConst{\HOLTokenWeakEQ} \HOLConst{KLOP} \HOLFreeVar{a} \HOLFreeVar{n}\ensuremath{)}\hfill{[KLOP\_PROP2']}
\end{alltt}
%
\item (Klop functions are one-one)
%
%a6 #&#
\begin{alltt}
\HOLTokenTurnstile{} \HOLConst{ONE\_ONE} \ensuremath{(}\HOLConst{KLOP} \HOLFreeVar{a}\ensuremath{)}\hfill{[KLOP\_ONE\_ONE]}
\end{alltt}
%
\end{enumerate}
%
\end{proposition}

For any \HOLinline{\HOLConst{label}\;\HOLFreeVar{a}}, having the function
``\HOLinline{\HOLConst{KLOP}\;\HOLFreeVar{a}}'' (of type ``\HOLinline{\HOLTyOp{num} \HOLTokenTransEnd \ensuremath{(}\ensuremath{\alpha}, \ensuremath{\beta}\ensuremath{)} \HOLTyOp{CCS}}'')
defined on the natural numbers, we obtain a countable set of Klop processes
built from the same label. As the number of all Klop processes in this
set is (countably) infinite, and they are all pairwise non-\xch{bisimilar}{bisimiar}, we
can always choose a number, corresponding to a Klop process, that is non-bisimilar
with any derivative of two given (finite-state) processes $p$ and
$q$, even when $a$ is the only element of type $\beta $, i.e. the only
label name in $\mathscr{L}$. This property is captured by appealing to
the following set-\xch{theoretic}{theoreric} \xch{lemma}{lamma} (see~\cite{Tian:2017wrba} for its proof):
%
%l4.15 #&#
\begin{lemma}
Given an equivalence relation $R$ defined on a type, and two sets
$A, B$ of elements in this type, if $A$ is finite, $B$ is infinite, and
all elements in $B$ belong to distinct equivalence classes, then there
exists an element $k$ in $B$ which is not equivalent to any element in
$A$:
%
%a4.15 #&#
\begin{alltt}
\HOLTokenTurnstile{} \HOLConst{equivalence} \HOLFreeVar{R} \HOLSymConst{\HOLTokenImp{}}
   \HOLConst{FINITE} \HOLFreeVar{A} \HOLSymConst{\HOLTokenConj{}} \HOLConst{INFINITE} \HOLFreeVar{B} \HOLSymConst{\HOLTokenConj{}} \ensuremath{(}\HOLSymConst{\HOLTokenForall{}}\HOLBoundVar{x} \HOLBoundVar{y}. \HOLBoundVar{x} \HOLSymConst{\HOLTokenIn{}} \HOLFreeVar{B} \HOLSymConst{\HOLTokenConj{}} \HOLBoundVar{y} \HOLSymConst{\HOLTokenIn{}} \HOLFreeVar{B} \HOLSymConst{\HOLTokenConj{}} \HOLBoundVar{x} \HOLSymConst{\HOLTokenNotEqual{}} \HOLBoundVar{y} \HOLSymConst{\HOLTokenImp{}} \HOLSymConst{\HOLTokenNeg{}}\HOLFreeVar{R} \HOLBoundVar{x} \HOLBoundVar{y}\ensuremath{)} \HOLSymConst{\HOLTokenImp{}}
   \HOLSymConst{\HOLTokenExists{}}\HOLBoundVar{k}. \HOLBoundVar{k} \HOLSymConst{\HOLTokenIn{}} \HOLFreeVar{B} \HOLSymConst{\HOLTokenConj{}} \HOLSymConst{\HOLTokenForall{}}\HOLBoundVar{n}. \HOLBoundVar{n} \HOLSymConst{\HOLTokenIn{}} \HOLFreeVar{A} \HOLSymConst{\HOLTokenImp{}} \HOLSymConst{\HOLTokenNeg{}}\HOLFreeVar{R} \HOLBoundVar{n} \HOLBoundVar{k}\hfill[INFINITE\_EXISTS\_LEMMA]
\end{alltt}
%
\end{lemma}

To reason about finite-state CCS, we also need to define the concept of
``finite-state CCS'' as a predicate on CCS processes:
%
%d4.16 #&#
\begin{definition}[finite-state CCS]
\mbox{}
%
\begin{enumerate}
%
\item A binary relation \texttt{Reach} is the RTC (reflexive and transitive
closure) of a relation indicating the existence of a transition between
two processes:
%
%a1 #&#
\begin{alltt}
\HOLConst{Reach} \HOLTokenDefEquality{} \ensuremath{(}\HOLTokenLambda{}\HOLBoundVar{E} \ensuremath{\HOLBoundVar{E}\sp{\prime}}. \HOLSymConst{\HOLTokenExists{}}\HOLBoundVar{u}. \HOLBoundVar{E} \HOLTokenTransBegin\HOLBoundVar{u}\HOLTokenTransEnd \ensuremath{\HOLBoundVar{E}\sp{\prime}}\ensuremath{)}\HOLSymConst{\HOLTokenSupStar{}}\hfill[Reach\_def]
\end{alltt}
%
\item The set of all derivatives (\texttt{NODES}) of a process is the set
of all processes reachable from it:
%
%a2 #&#
\begin{alltt}
\HOLConst{NODES} \HOLFreeVar{p} \HOLTokenDefEquality{} \HOLTokenLeftbrace{}\HOLBoundVar{q} \HOLTokenBar{} \HOLConst{Reach} \HOLFreeVar{p} \HOLBoundVar{q}\HOLTokenRightbrace{}\hfill[NODES\_def]
\end{alltt}
%
\item A process is \texttt{finite-state} if the set of all derivatives
is finite:
%
%a3 #&#
\begin{alltt}
\HOLConst{finite\_state} \HOLFreeVar{p} \HOLTokenDefEquality{} \HOLConst{FINITE} \ensuremath{(}\HOLConst{NODES} \HOLFreeVar{p}\ensuremath{)}\hfill[finite\_state\_def]
\end{alltt}
%
\end{enumerate}
%
\end{definition}
%
We rely on various properties of the above definitions, such as the following
one:
%
%p4.17 #&#
\begin{proposition}
If $p$ has a weak transition to $q$, then $q$ is among the derivatives
of $p$:
%
%a4.17 #&#
\begin{alltt}
\HOLTokenTurnstile{} \HOLFreeVar{p} \HOLTokenWeakTransBegin\HOLFreeVar{u}\HOLTokenWeakTransEnd \HOLFreeVar{q} \HOLSymConst{\HOLTokenImp{}} \HOLFreeVar{q} \HOLSymConst{\HOLTokenIn{}} \HOLConst{NODES} \HOLFreeVar{p}\hfill[WEAK\_TRANS\_IN\_NODES]
\end{alltt}
%
\end{proposition}

Using all the above results, now we can prove the following finite-state
version of ``Klop lemma'':
%
%l4.18 #&#
\begin{lemma}[Klop lemma for finite-state CCS]
 \label{lem:klop-lemma-finite}
For any two finite-state CCS $p$ and $q$, there is another process
$k$, which is not weakly bisimilar with any weak derivative of $p$ and
$q$ (i.e., any process reachable from $p$ or $q$ by means of transitions):
%
%a4.18 #&#
\begin{alltt}
\HOLTokenTurnstile{} \HOLSymConst{\HOLTokenForall{}}\HOLBoundVar{p} \HOLBoundVar{q}.
       \HOLConst{finite\_state} \HOLBoundVar{p} \HOLSymConst{\HOLTokenConj{}} \HOLConst{finite\_state} \HOLBoundVar{q} \HOLSymConst{\HOLTokenImp{}}
       \HOLSymConst{\HOLTokenExists{}}\HOLBoundVar{k}.
           \HOLConst{STABLE} \HOLBoundVar{k} \HOLSymConst{\HOLTokenConj{}} \ensuremath{(}\HOLSymConst{\HOLTokenForall{}}\ensuremath{\HOLBoundVar{p}\sp{\prime}} \HOLBoundVar{u}. \HOLBoundVar{p} \HOLTokenWeakTransBegin\HOLBoundVar{u}\HOLTokenWeakTransEnd \ensuremath{\HOLBoundVar{p}\sp{\prime}} \HOLSymConst{\HOLTokenImp{}} \HOLSymConst{\HOLTokenNeg{}}\ensuremath{(}\ensuremath{\HOLBoundVar{p}\sp{\prime}} \HOLSymConst{\HOLTokenWeakEQ} \HOLBoundVar{k}\ensuremath{)}\ensuremath{)} \HOLSymConst{\HOLTokenConj{}}
           \HOLSymConst{\HOLTokenForall{}}\ensuremath{\HOLBoundVar{q}\sp{\prime}} \HOLBoundVar{u}. \HOLBoundVar{q} \HOLTokenWeakTransBegin\HOLBoundVar{u}\HOLTokenWeakTransEnd \ensuremath{\HOLBoundVar{q}\sp{\prime}} \HOLSymConst{\HOLTokenImp{}} \HOLSymConst{\HOLTokenNeg{}}\ensuremath{(}\ensuremath{\HOLBoundVar{q}\sp{\prime}} \HOLSymConst{\HOLTokenWeakEQ} \HOLBoundVar{k}\ensuremath{)}\hfill{[KLOP\_LEMMA\_FINITE]}
\end{alltt}
%
\end{lemma}
%
Combining the above lemma with the core lemma (\texttt{PROP3\_COMMON}) and
\reftext{Theorem~\ref{thm:coarsestR}} (\texttt{COARSEST\_CONGR\_RL}), yields the proof
of \reftext{Theorem~\ref{thm:coarsestfiniteState}} (\texttt{COARSEST\_CONGR\_FINITE}).
The same proof idea can also be used with contraction and rooted \xch{contraction}{contration}.

%s4.10 #&#
\subsection{Unique solution of equations}
%%LEAP%%%\label{sec4.10}
 \label{ss:part2}

In this section we describe the formalisation of Milner's ``unique solution
of equations'' theorems, limited to \univariate equations. (The
\multivariate extension is described in Section~\ref{sec:multivariate}.)

%s4.10.1 #&#
\subsubsection{The version for strong bisimilarity}
 \label{sec4.10.1}

Using ``bisimulation up to $\sim $'' technique, we obtain the following
key lemma, which states that, if $X$ is weakly guarded in $E$, then the
``first move'' of $E$ is independent of the agent substituted for
$X$:
%
%l4.19 #&#
\begin{lemma}[\texttt{STRONG\_UNIQUE\_SOLUTION\_LEMMA}, Lemma 3.13
of~\cite{Mil89}, \univariate version]
 \label{lem:313}
If the variable $X$ is weakly guarded in $E$, and
$E\{P/X\}\overset{\alpha}{\rightarrow} P'$, then $P'$ takes the form
$E'\{P/X\}$ (for some expression $E'$), and moreover, for any $Q$,
$E\{Q/X\}\overset{\alpha}{\rightarrow} E'\{Q/X\}$:
%
%a4.19 #&#
\begin{alltt}
\HOLTokenTurnstile{} \HOLConst{WG} \HOLFreeVar{E} \HOLSymConst{\HOLTokenImp{}}
   \HOLSymConst{\HOLTokenForall{}}\HOLBoundVar{P} \HOLBoundVar{a} \ensuremath{\HOLBoundVar{P}\sp{\prime}}. \HOLFreeVar{E} \HOLBoundVar{P} \HOLTokenTransBegin\HOLBoundVar{a}\HOLTokenTransEnd \ensuremath{\HOLBoundVar{P}\sp{\prime}} \HOLSymConst{\HOLTokenImp{}} \HOLSymConst{\HOLTokenExists{}}\ensuremath{\HOLBoundVar{E}\sp{\prime}}. \HOLConst{CONTEXT} \ensuremath{\HOLBoundVar{E}\sp{\prime}} \HOLSymConst{\HOLTokenConj{}} \ensuremath{\HOLBoundVar{P}\sp{\prime}} \HOLSymConst{\ensuremath{=}} \ensuremath{\HOLBoundVar{E}\sp{\prime}} \HOLBoundVar{P} \HOLSymConst{\HOLTokenConj{}} \HOLSymConst{\HOLTokenForall{}}\HOLBoundVar{Q}. \HOLFreeVar{E} \HOLBoundVar{Q} \HOLTokenTransBegin\HOLBoundVar{a}\HOLTokenTransEnd \ensuremath{\HOLBoundVar{E}\sp{\prime}} \HOLBoundVar{Q}
\end{alltt}
%
\end{lemma}

Then, by structural induction on weakly guarded contexts (\texttt{WG}),
we prove \reftext{Theorem~\ref{t:Mil89s1}} in the \univariate case. The formal proof
basically follows the outline of its informal version~\citep[p.~102--103]{Mil89},
which is tedious due to large amounts of case analyses on each CCS operator.
%
%t4.20 #&#
\begin{theorem}[\texttt{STRONG\_UNIQUE\_SOLUTION}, \univariate version
of \reftext{Theorem~\ref{t:Mil89s1}}]
 \label{thm:Mil89s1f}
Suppose the expression $E$ contains at most the variable $X$, and let
$X$ be weakly guarded in $E$.
%
%e3 #&#
\begin{equation}
\text{If } P \sim E\{P/X\} \text{ and } Q \sim E\{Q/X\} \text{ then } P
\sim Q.
\end{equation}
%
%a4.20 #&#
\begin{alltt}
\HOLTokenTurnstile{} \HOLConst{WG} \HOLFreeVar{E} \HOLSymConst{\HOLTokenConj{}} \HOLFreeVar{P} \HOLSymConst{\HOLTokenStrongEQ} \HOLFreeVar{E} \HOLFreeVar{P} \HOLSymConst{\HOLTokenConj{}} \HOLFreeVar{Q} \HOLSymConst{\HOLTokenStrongEQ} \HOLFreeVar{E} \HOLFreeVar{Q} \HOLSymConst{\HOLTokenImp{}} \HOLFreeVar{P} \HOLSymConst{\HOLTokenStrongEQ} \HOLFreeVar{Q}
\end{alltt}
%
\end{theorem}

%s4.10.2 #&#
\subsubsection{The version for rooted bisimilarity}
 \label{sec4.10.2}

For the proof of \reftext{Theorem~\ref{t:Mil89s3}}, we did not use any ``bisimulation
up-to'' technique. Instead, we have used a different technique based on
\reftext{Lemma~\ref{lobsCongrByWeakBisim}}, whose formal version is the following
one (\texttt{OBS\_CONGR\_BY\_WEAK\_BISIM}):
%
%a4.10.2 #&#
\begin{alltt}
\HOLTokenTurnstile{} \HOLConst{WEAK\_BISIM} \HOLFreeVar{Wbsm} \HOLSymConst{\HOLTokenImp{}}
   \HOLSymConst{\HOLTokenForall{}}\HOLBoundVar{E} \ensuremath{\HOLBoundVar{E}\sp{\prime}}.
       \ensuremath{(}\HOLSymConst{\HOLTokenForall{}}\HOLBoundVar{u}.
            \ensuremath{(}\HOLSymConst{\HOLTokenForall{}}\ensuremath{\HOLBoundVar{E}\sb{\mathrm{1}}}. \HOLBoundVar{E} \HOLTokenTransBegin\HOLBoundVar{u}\HOLTokenTransEnd \ensuremath{\HOLBoundVar{E}\sb{\mathrm{1}}} \HOLSymConst{\HOLTokenImp{}} \HOLSymConst{\HOLTokenExists{}}\ensuremath{\HOLBoundVar{E}\sb{\mathrm{2}}}. \ensuremath{\HOLBoundVar{E}\sp{\prime}} \HOLTokenWeakTransBegin\HOLBoundVar{u}\HOLTokenWeakTransEnd \ensuremath{\HOLBoundVar{E}\sb{\mathrm{2}}} \HOLSymConst{\HOLTokenConj{}} \HOLFreeVar{Wbsm} \ensuremath{\HOLBoundVar{E}\sb{\mathrm{1}}} \ensuremath{\HOLBoundVar{E}\sb{\mathrm{2}}}\ensuremath{)} \HOLSymConst{\HOLTokenConj{}}
            \HOLSymConst{\HOLTokenForall{}}\ensuremath{\HOLBoundVar{E}\sb{\mathrm{2}}}. \ensuremath{\HOLBoundVar{E}\sp{\prime}} \HOLTokenTransBegin\HOLBoundVar{u}\HOLTokenTransEnd \ensuremath{\HOLBoundVar{E}\sb{\mathrm{2}}} \HOLSymConst{\HOLTokenImp{}} \HOLSymConst{\HOLTokenExists{}}\ensuremath{\HOLBoundVar{E}\sb{\mathrm{1}}}. \HOLBoundVar{E} \HOLTokenWeakTransBegin\HOLBoundVar{u}\HOLTokenWeakTransEnd \ensuremath{\HOLBoundVar{E}\sb{\mathrm{1}}} \HOLSymConst{\HOLTokenConj{}} \HOLFreeVar{Wbsm} \ensuremath{\HOLBoundVar{E}\sb{\mathrm{1}}} \ensuremath{\HOLBoundVar{E}\sb{\mathrm{2}}}\ensuremath{)} \HOLSymConst{\HOLTokenImp{}}
       \HOLBoundVar{E} \HOLSymConst{\HOLTokenObsCongr} \ensuremath{\HOLBoundVar{E}\sp{\prime}}
\end{alltt}
%
Using \reftext{Lemma~\ref{lobsCongrByWeakBisim}}, the next two results are proved
by directly constructing the required bisimulation: (see~\cite{Tian:2017wrba}
for more details)
%
%l4.21 #&#
\begin{lemma}[\texttt{OBS\_UNIQUE\_SOLUTION\_LEMMA}]
If the variable $X$ is guarded and sequential in $G$, and
$G\{P/X\}\overset{\alpha}{\rightarrow} P'$, then $P'$ takes the form
$H\{P/X\}$, for some $H$, and for any $Q$, we also have
$G\{Q/X\}\overset{\alpha}{\rightarrow} H\{Q/X\}$. Moreover $H$ is sequential,
and if $\alpha = \tau $, then $H$ is also guarded.
%
%a4.21 #&#
\begin{alltt}
\HOLTokenTurnstile{} \HOLConst{SG} \HOLFreeVar{G} \HOLSymConst{\HOLTokenConj{}} \HOLConst{SEQ} \HOLFreeVar{G} \HOLSymConst{\HOLTokenImp{}}
   \HOLSymConst{\HOLTokenForall{}}\HOLBoundVar{P} \HOLBoundVar{a} \ensuremath{\HOLBoundVar{P}\sp{\prime}}.
       \HOLFreeVar{G} \HOLBoundVar{P} \HOLTokenTransBegin\HOLBoundVar{a}\HOLTokenTransEnd \ensuremath{\HOLBoundVar{P}\sp{\prime}} \HOLSymConst{\HOLTokenImp{}}
       \HOLSymConst{\HOLTokenExists{}}\HOLBoundVar{H}. \HOLConst{SEQ} \HOLBoundVar{H} \HOLSymConst{\HOLTokenConj{}} \ensuremath{(}\HOLBoundVar{a} \HOLSymConst{\ensuremath{=}} \HOLSymConst{\ensuremath{\tau}} \HOLSymConst{\HOLTokenImp{}} \HOLConst{SG} \HOLBoundVar{H}\ensuremath{)} \HOLSymConst{\HOLTokenConj{}} \ensuremath{\HOLBoundVar{P}\sp{\prime}} \HOLSymConst{\ensuremath{=}} \HOLBoundVar{H} \HOLBoundVar{P} \HOLSymConst{\HOLTokenConj{}} \HOLSymConst{\HOLTokenForall{}}\HOLBoundVar{Q}. \HOLFreeVar{G} \HOLBoundVar{Q} \HOLTokenTransBegin\HOLBoundVar{a}\HOLTokenTransEnd \HOLBoundVar{H} \HOLBoundVar{Q}
\end{alltt}
%
\end{lemma}

%t4.22 #&#
\begin{theorem}[\texttt{OBS\_UNIQUE\_SOLUTION}, \univariate version of
\reftext{Theorem~\ref{t:Mil89s3}}]
 \label{thm:Mil89s3f}
Let $E$ be guarded and sequential expressions, and let
$P \rapprox E\{P/X\}$, $Q \rapprox E\{Q/X\}$. Then $P \rapprox Q$.
%
%a4.22 #&#
\begin{alltt}
\HOLTokenTurnstile{} \HOLConst{SG} \HOLFreeVar{E} \HOLSymConst{\HOLTokenConj{}} \HOLConst{SEQ} \HOLFreeVar{E} \HOLSymConst{\HOLTokenConj{}} \HOLFreeVar{P} \HOLSymConst{\HOLTokenObsCongr} \HOLFreeVar{E} \HOLFreeVar{P} \HOLSymConst{\HOLTokenConj{}} \HOLFreeVar{Q} \HOLSymConst{\HOLTokenObsCongr} \HOLFreeVar{E} \HOLFreeVar{Q} \HOLSymConst{\HOLTokenImp{}} \HOLFreeVar{P} \HOLSymConst{\HOLTokenObsCongr} \HOLFreeVar{Q}
\end{alltt}
%
\end{theorem}

%s4.10.3 #&#
\subsubsection{The version for weak bisimilarity}
 \label{sec4.10.3}

Milner~\cite{Mil89} only mentioned two ``unique solution of equations''
theorems, one for strong equivalence, the other for rooted bisimilarity.
There is, however, another version for weak bisimilarity (\reftext{Theorem~\ref{t:Mil89}})
that shares a large portion of proof steps with the proof for rooted bisimilarity.
As weak bisimilarity is not a congruence, we have to be more restrictive
on the syntax of the equation, using only guarded sums. The related formal
proofs are tedious but closely follow the informal proof~\citep[p.~158--159]{Mil89},
with the exception of the use of ``bisimulation up to with weak arrows''
(\reftext{Definition~\ref{def:doubleweak}}, instead of \reftext{Definition~\ref{def:singleweak}}).

%l4.23 #&#
\begin{lemma}[\texttt{WEAK\_UNIQUE\_SOLUTION\_LEMMA}]
If the variable $X$ is guarded and sequential (with only guarded sums)
in $G$, and $G\{P/X\}\overset{\alpha}{\rightarrow} P'$, then $P'$ takes
the form $H\{P/X\}$, for some expression $H$, and for any $Q$, we have
$G\{Q/X\}\overset{\alpha}{\rightarrow} H\{Q/X\}$. Moreover $H$ is sequential,
and if $\alpha = \tau $ then $H$ is also guarded.
%
%a4.23 #&#
\begin{alltt}
\HOLTokenTurnstile{} \HOLConst{SG} \HOLFreeVar{G} \HOLSymConst{\HOLTokenConj{}} \HOLConst{GSEQ} \HOLFreeVar{G} \HOLSymConst{\HOLTokenImp{}}
   \HOLSymConst{\HOLTokenForall{}}\HOLBoundVar{P} \HOLBoundVar{a} \ensuremath{\HOLBoundVar{P}\sp{\prime}}.
       \HOLFreeVar{G} \HOLBoundVar{P} \HOLTokenTransBegin\HOLBoundVar{a}\HOLTokenTransEnd \ensuremath{\HOLBoundVar{P}\sp{\prime}} \HOLSymConst{\HOLTokenImp{}}
       \HOLSymConst{\HOLTokenExists{}}\HOLBoundVar{H}. \HOLConst{GSEQ} \HOLBoundVar{H} \HOLSymConst{\HOLTokenConj{}} \ensuremath{(}\HOLBoundVar{a} \HOLSymConst{\ensuremath{=}} \HOLSymConst{\ensuremath{\tau}} \HOLSymConst{\HOLTokenImp{}} \HOLConst{SG} \HOLBoundVar{H}\ensuremath{)} \HOLSymConst{\HOLTokenConj{}} \ensuremath{\HOLBoundVar{P}\sp{\prime}} \HOLSymConst{\ensuremath{=}} \HOLBoundVar{H} \HOLBoundVar{P} \HOLSymConst{\HOLTokenConj{}} \HOLSymConst{\HOLTokenForall{}}\HOLBoundVar{Q}. \HOLFreeVar{G} \HOLBoundVar{Q} \HOLTokenTransBegin\HOLBoundVar{a}\HOLTokenTransEnd \HOLBoundVar{H} \HOLBoundVar{Q}
\end{alltt}
%
\end{lemma}

%t4.24 #&#
\begin{theorem}[\texttt{WEAK\_UNIQUE\_SOLUTION}, \univariate version of
\reftext{Theorem~\ref{t:Mil89}}]
 \label{thm:Mil89f}
Let $E$ be guarded and sequential expressions, and let
$P \wbvtex E\{P/X\}$, $Q \wbvtex E\{Q/X\}$. Then $P \wbvtex Q$.
%
%a4.24 #&#
\begin{alltt}
\HOLTokenTurnstile{} \HOLConst{SG} \HOLFreeVar{E} \HOLSymConst{\HOLTokenConj{}} \HOLConst{GSEQ} \HOLFreeVar{E} \HOLSymConst{\HOLTokenConj{}} \HOLFreeVar{P} \HOLSymConst{\HOLTokenWeakEQ} \HOLFreeVar{E} \HOLFreeVar{P} \HOLSymConst{\HOLTokenConj{}} \HOLFreeVar{Q} \HOLSymConst{\HOLTokenWeakEQ} \HOLFreeVar{E} \HOLFreeVar{Q} \HOLSymConst{\HOLTokenImp{}} \HOLFreeVar{P} \HOLSymConst{\HOLTokenWeakEQ} \HOLFreeVar{Q}
\end{alltt}
%
\end{theorem}

%s4.11 #&#
\subsection{Unique solution of contractions}
 \label{sec4.11}

A delicate point in the formalisation of the results about unique solution
of contractions is the proof of \reftext{Lemma~\ref{l:ruptocon}} and lemmas alike.
In particular, we use induction on the length of weak transitions. For
this, rather than introducing a refined form of weak transition relation
enriched with its length, we found it more elegant to work with
\emph{traces}, i.e., sequences of transitions. (A further motivation is
to set the ground for future extensions of this formalisation work to
\emph{trace equivalence} in place of bisimilarity.)

A trace is formally represented by the initial and final processes, plus
a list of actions so performed. For this, we first define the concept of
\emph{label-accumulated reflexive transitive closure} (\HOLinline{\HOLConst{LRTC}}).
Then, given any labeled transition relation
\HOLinline{\HOLFreeVar{R}} (of type ``\HOLinline{\ensuremath{\alpha} \HOLTokenTransEnd \ensuremath{\beta} \HOLTokenTransEnd \ensuremath{\alpha} \HOLTokenTransEnd \HOLTyOp{bool}}''),
\HOLinline{\HOLConst{LRTC} \HOLFreeVar{R}} is a relation representing traces
over \HOLinline{\HOLFreeVar{R}} (of type ``\HOLinline{\ensuremath{\alpha} \HOLTokenTransEnd \ensuremath{\beta} \HOLTyOp{list} \HOLTokenTransEnd \ensuremath{\alpha} \HOLTokenTransEnd \HOLTyOp{bool}}''):
%
%a4.11 #&#
\begin{alltt}
   \HOLConst{LRTC} \HOLFreeVar{R} \HOLFreeVar{a} \HOLFreeVar{l} \HOLFreeVar{b} \HOLTokenDefEquality{}
     \HOLSymConst{\HOLTokenForall{}}\HOLBoundVar{P}.
         \ensuremath{(}\HOLSymConst{\HOLTokenForall{}}\HOLBoundVar{x}. \HOLBoundVar{P} \HOLBoundVar{x} \ensuremath{[}\ensuremath{]} \HOLBoundVar{x}\ensuremath{)} \HOLSymConst{\HOLTokenConj{}}
         \ensuremath{(}\HOLSymConst{\HOLTokenForall{}}\HOLBoundVar{x} \HOLBoundVar{h} \HOLBoundVar{y} \HOLBoundVar{t} \HOLBoundVar{z}. \HOLFreeVar{R} \HOLBoundVar{x} \HOLBoundVar{h} \HOLBoundVar{y} \HOLSymConst{\HOLTokenConj{}} \HOLBoundVar{P} \HOLBoundVar{y} \HOLBoundVar{t} \HOLBoundVar{z} \HOLSymConst{\HOLTokenImp{}} \HOLBoundVar{P} \HOLBoundVar{x} \ensuremath{(}\HOLBoundVar{h}\HOLSymConst{::}\HOLBoundVar{t}\ensuremath{)} \HOLBoundVar{z}\ensuremath{)} \HOLSymConst{\HOLTokenImp{}}
         \HOLBoundVar{P} \HOLFreeVar{a} \HOLFreeVar{l} \HOLFreeVar{b}\hfill{[LRTC\_DEF]}
\end{alltt}
%
The trace relation for CCS, \HOLinline{\HOLConst{TRACE}} (of the type ``\HOLinline{\ensuremath{(}\ensuremath{\alpha}, \ensuremath{\beta}\ensuremath{)} \HOLTyOp{CCS} \HOLTokenTransEnd \ensuremath{\beta} \HOLTyOp{Action} \HOLTyOp{list} \HOLTokenTransEnd \ensuremath{(}\ensuremath{\alpha}, \ensuremath{\beta}\ensuremath{)} \HOLTyOp{CCS} \HOLTokenTransEnd \HOLTyOp{bool}}''),
is then obtained by combining \texttt{LRTC} and the \texttt{TRANS} ($
\overset{\mu}{\rightarrow}$) relation defining the SOS rules:
%
%a4.11 #&#
\begin{alltt}
   \HOLConst{TRACE} \HOLTokenDefEquality{} \HOLConst{LRTC} \HOLConst{TRANS}\hfill{[TRACE\_def]}
\end{alltt}

In a trace $P\overset{acts}{\longrightarrow}Q$, the list of actions
$acts$ may be empty, in which case there is no transition, and the initial
and final processes are the same, i.e. $P = Q$. If there is at most one
visible action $\mu $ in $acts$, and all other actions are $\tau $'s, then
the trace represents a weak transition $P \Arr{\mu} Q$. For this, we have
to distinguish two cases in the list of actions: no label and unique label.
The definition of ``no label'' for a list of actions is as follows, where
\texttt{MEM} tests if an element is a member of a list:
%
%a4.11 #&#
\begin{alltt}
   \HOLConst{NO\_LABEL} \HOLFreeVar{L} \HOLTokenDefEquality{} \HOLSymConst{\HOLTokenNeg{}}\HOLSymConst{\HOLTokenExists{}}\HOLBoundVar{l}. \HOLConst{MEM} \ensuremath{(}\HOLConst{label} \HOLBoundVar{l}\ensuremath{)} \HOLFreeVar{L}\hfill{[NO\_LABEL\_def]}
\end{alltt}

The definition of ``unique label'' can be done in many ways. The following
definition (due to a suggestion from Robert Beers) avoids any counting
or filtering in the list. It says that a label is unique in a list of actions
if there is no label in the rest of list (here
$\HOLTokenDoublePlus $ is the concatenation of lists):
%
%a4.11 #&#
\begin{alltt}
   \HOLConst{UNIQUE\_LABEL} \HOLFreeVar{u} \HOLFreeVar{L} \HOLTokenDefEquality{}
     \HOLSymConst{\HOLTokenExists{}}\ensuremath{\HOLBoundVar{L}\sb{\mathrm{1}}} \ensuremath{\HOLBoundVar{L}\sb{\mathrm{2}}}. \ensuremath{\HOLBoundVar{L}\sb{\mathrm{1}}} \HOLSymConst{\HOLTokenDoublePlus} \ensuremath{[}\HOLFreeVar{u}\ensuremath{]} \HOLSymConst{\HOLTokenDoublePlus} \ensuremath{\HOLBoundVar{L}\sb{\mathrm{2}}} \HOLSymConst{\ensuremath{=}} \HOLFreeVar{L} \HOLSymConst{\HOLTokenConj{}} \HOLConst{NO\_LABEL} \ensuremath{\HOLBoundVar{L}\sb{\mathrm{1}}} \HOLSymConst{\HOLTokenConj{}} \HOLConst{NO\_LABEL} \ensuremath{\HOLBoundVar{L}\sb{\mathrm{2}}}\hfill{[UNIQUE\_LABEL\_def]}
\end{alltt}

Finally, the relationship between traces and weak transitions is stated
and proved in the following theorem. It says that a weak transition
$P \Arr{u} P'$ is also a trace $P\overset{acts}{\longrightarrow}P'$ with
a non-empty action list $acts$, in which either there is no label (for
$u = \tau $), or $u$ is the unique label (for $u \neq \tau $):
%
%a4.11 #&#
\begin{alltt}
\HOLTokenTurnstile{} \HOLFreeVar{P} \HOLTokenWeakTransBegin\HOLFreeVar{u}\HOLTokenWeakTransEnd \ensuremath{\HOLFreeVar{P}\sp{\prime}} \HOLSymConst{\HOLTokenEquiv{}}
   \HOLSymConst{\HOLTokenExists{}}\HOLBoundVar{acts}.
       \HOLConst{TRACE} \HOLFreeVar{P} \HOLBoundVar{acts} \ensuremath{\HOLFreeVar{P}\sp{\prime}} \HOLSymConst{\HOLTokenConj{}} \HOLSymConst{\HOLTokenNeg{}}\HOLConst{NULL} \HOLBoundVar{acts} \HOLSymConst{\HOLTokenConj{}}
       \HOLKeyword{if} \HOLFreeVar{u} \HOLSymConst{\ensuremath{=}} \HOLSymConst{\ensuremath{\tau}} \HOLKeyword{then} \HOLConst{NO\_LABEL} \HOLBoundVar{acts} \HOLKeyword{else} \HOLConst{UNIQUE\_LABEL} \HOLFreeVar{u} \HOLBoundVar{acts}\hfill{[WEAK\_TRANS\_AND\_TRACE]}
\end{alltt}

Now the formal version of \reftext{Lemma~\ref{l:uptocon}} (\texttt{UNIQUE\_SOLUTION\_OF\_CONTRACTIONS\_LEMMA}):
%
%a4.11 #&#
\begin{alltt}
\HOLTokenTurnstile{} \ensuremath{(}\HOLSymConst{\HOLTokenExists{}}\HOLBoundVar{E}. \HOLConst{WGS} \HOLBoundVar{E} \HOLSymConst{\HOLTokenConj{}} \HOLFreeVar{P} \HOLSymConst{\HOLTokenContracts{}} \HOLBoundVar{E} \HOLFreeVar{P} \HOLSymConst{\HOLTokenConj{}} \HOLFreeVar{Q} \HOLSymConst{\HOLTokenContracts{}} \HOLBoundVar{E} \HOLFreeVar{Q}\ensuremath{)} \HOLSymConst{\HOLTokenImp{}}
   \HOLSymConst{\HOLTokenForall{}}\HOLBoundVar{C}.
       \HOLConst{GCONTEXT} \HOLBoundVar{C} \HOLSymConst{\HOLTokenImp{}}
       \ensuremath{(}\HOLSymConst{\HOLTokenForall{}}\HOLBoundVar{l} \HOLBoundVar{R}.
            \HOLBoundVar{C} \HOLFreeVar{P} \HOLTokenWeakTransBegin\HOLConst{label} \HOLBoundVar{l}\HOLTokenWeakTransEnd \HOLBoundVar{R} \HOLSymConst{\HOLTokenImp{}}
            \HOLSymConst{\HOLTokenExists{}}\ensuremath{\HOLBoundVar{C}\sp{\prime}}.
                \HOLConst{GCONTEXT} \ensuremath{\HOLBoundVar{C}\sp{\prime}} \HOLSymConst{\HOLTokenConj{}} \HOLBoundVar{R} \HOLSymConst{\HOLTokenContracts{}} \ensuremath{\HOLBoundVar{C}\sp{\prime}} \HOLFreeVar{P} \HOLSymConst{\HOLTokenConj{}}
                \ensuremath{(}\HOLConst{WEAK\_EQUIV} \HOLSymConst{\HOLTokenRCompose{}} \ensuremath{(}\HOLTokenLambda{}\HOLBoundVar{x} \HOLBoundVar{y}. \HOLBoundVar{x} \HOLTokenWeakTransBegin\HOLConst{label} \HOLBoundVar{l}\HOLTokenWeakTransEnd \HOLBoundVar{y}\ensuremath{)}\ensuremath{)} \ensuremath{(}\HOLBoundVar{C} \HOLFreeVar{Q}\ensuremath{)} \ensuremath{(}\ensuremath{\HOLBoundVar{C}\sp{\prime}} \HOLFreeVar{Q}\ensuremath{)}\ensuremath{)} \HOLSymConst{\HOLTokenConj{}}
       \HOLSymConst{\HOLTokenForall{}}\HOLBoundVar{R}.
           \HOLBoundVar{C} \HOLFreeVar{P} \HOLTokenWeakTransBegin\HOLSymConst{\ensuremath{\tau}}\HOLTokenWeakTransEnd \HOLBoundVar{R} \HOLSymConst{\HOLTokenImp{}}
           \HOLSymConst{\HOLTokenExists{}}\ensuremath{\HOLBoundVar{C}\sp{\prime}}. \HOLConst{GCONTEXT} \ensuremath{\HOLBoundVar{C}\sp{\prime}} \HOLSymConst{\HOLTokenConj{}} \HOLBoundVar{R} \HOLSymConst{\HOLTokenContracts{}} \ensuremath{\HOLBoundVar{C}\sp{\prime}} \HOLFreeVar{P} \HOLSymConst{\HOLTokenConj{}} \ensuremath{(}\HOLConst{WEAK\_EQUIV} \HOLSymConst{\HOLTokenRCompose{}} \HOLConst{EPS}\ensuremath{)} \ensuremath{(}\HOLBoundVar{C} \HOLFreeVar{Q}\ensuremath{)} \ensuremath{(}\ensuremath{\HOLBoundVar{C}\sp{\prime}} \HOLFreeVar{Q}\ensuremath{)}
\end{alltt}

Traces are actually used in the proof of above lemma via the following
lemma (\texttt{unfolding\_lemma4}):
%
%a4.11 #&#
\begin{alltt}
\HOLTokenTurnstile{} \HOLConst{GCONTEXT} \HOLFreeVar{C} \HOLSymConst{\HOLTokenConj{}} \HOLConst{WGS} \HOLFreeVar{E} \HOLSymConst{\HOLTokenConj{}} \HOLConst{TRACE} \ensuremath{(}\ensuremath{(}\HOLFreeVar{C} \HOLSymConst{\HOLTokenCompose} \HOLConst{FUNPOW} \HOLFreeVar{E} \HOLFreeVar{n}\ensuremath{)} \HOLFreeVar{P}\ensuremath{)} \HOLFreeVar{xs} \ensuremath{\HOLFreeVar{P}\sp{\prime}} \HOLSymConst{\HOLTokenConj{}} \HOLConst{LENGTH} \HOLFreeVar{xs} \HOLSymConst{\HOLTokenLeq{}} \HOLFreeVar{n} \HOLSymConst{\HOLTokenImp{}}
   \HOLSymConst{\HOLTokenExists{}}\ensuremath{\HOLBoundVar{C}\sp{\prime}}. \HOLConst{GCONTEXT} \ensuremath{\HOLBoundVar{C}\sp{\prime}} \HOLSymConst{\HOLTokenConj{}} \ensuremath{\HOLFreeVar{P}\sp{\prime}} \HOLSymConst{\ensuremath{=}} \ensuremath{\HOLBoundVar{C}\sp{\prime}} \HOLFreeVar{P} \HOLSymConst{\HOLTokenConj{}} \HOLSymConst{\HOLTokenForall{}}\HOLBoundVar{Q}. \HOLConst{TRACE} \ensuremath{(}\ensuremath{(}\HOLFreeVar{C} \HOLSymConst{\HOLTokenCompose} \HOLConst{FUNPOW} \HOLFreeVar{E} \HOLFreeVar{n}\ensuremath{)} \HOLBoundVar{Q}\ensuremath{)} \HOLFreeVar{xs} \ensuremath{(}\ensuremath{\HOLBoundVar{C}\sp{\prime}} \HOLBoundVar{Q}\ensuremath{)}
\end{alltt}
%
which roughly says that, for any context $C$ and weakly guarded context
$E$, if $C [\, E^{n}[P]\,] \overset{xs}{\longrightarrow} P'$ and the length
of actions $xs \leqslant n$, then $P'$ has the form $C'[P]$. Traces are
used in reasoning about the number of intermediate actions in weak transitions.
For instance, from \reftext{Definition~\ref{d:BisCon}}, it is easy to see that a weak transition
either becomes shorter or remains the same when moving between
$\mcontrBIS $-related processes. This property is essential in the proof
of \reftext{Lemma~\ref{l:uptocon}}. We show only one such lemma, for the case of
visible weak transitions:
%
%a4.11 #&#
\begin{alltt}
\HOLTokenTurnstile{} \HOLFreeVar{P} \HOLSymConst{\HOLTokenContracts{}} \HOLFreeVar{Q} \HOLSymConst{\HOLTokenImp{}}
   \HOLSymConst{\HOLTokenForall{}}\HOLBoundVar{xs} \HOLBoundVar{l} \ensuremath{\HOLBoundVar{P}\sp{\prime}}.
       \HOLConst{TRACE} \HOLFreeVar{P} \HOLBoundVar{xs} \ensuremath{\HOLBoundVar{P}\sp{\prime}} \HOLSymConst{\HOLTokenConj{}} \HOLConst{UNIQUE\_LABEL} \ensuremath{(}\HOLConst{label} \HOLBoundVar{l}\ensuremath{)} \HOLBoundVar{xs} \HOLSymConst{\HOLTokenImp{}}
       \HOLSymConst{\HOLTokenExists{}}\ensuremath{\HOLBoundVar{xs}\sp{\prime}} \ensuremath{\HOLBoundVar{Q}\sp{\prime}}.
           \HOLConst{TRACE} \HOLFreeVar{Q} \ensuremath{\HOLBoundVar{xs}\sp{\prime}} \ensuremath{\HOLBoundVar{Q}\sp{\prime}} \HOLSymConst{\HOLTokenConj{}} \HOLFreeVar{P} \HOLSymConst{\HOLTokenContracts{}} \HOLFreeVar{Q} \HOLSymConst{\HOLTokenConj{}} \HOLConst{LENGTH} \ensuremath{\HOLBoundVar{xs}\sp{\prime}} \HOLSymConst{\HOLTokenLeq{}} \HOLConst{LENGTH} \HOLBoundVar{xs} \HOLSymConst{\HOLTokenConj{}}
           \HOLConst{UNIQUE\_LABEL} \ensuremath{(}\HOLConst{label} \HOLBoundVar{l}\ensuremath{)} \ensuremath{\HOLBoundVar{xs}\sp{\prime}}\hfill{[contracts\_AND\_TRACE\_label]}
\end{alltt}


With all above lemmas, we can thus finally prove \reftext{Theorem~\ref{t:contraBisimulationU}}:
%
%a4.11 #&#
\begin{alltt}
\HOLTokenTurnstile{} \HOLConst{WGS} \HOLFreeVar{E} \HOLSymConst{\HOLTokenConj{}} \HOLFreeVar{P} \HOLSymConst{\HOLTokenContracts{}} \HOLFreeVar{E} \HOLFreeVar{P} \HOLSymConst{\HOLTokenConj{}} \HOLFreeVar{Q} \HOLSymConst{\HOLTokenContracts{}} \HOLFreeVar{E} \HOLFreeVar{Q} \HOLSymConst{\HOLTokenImp{}} \HOLFreeVar{P} \HOLSymConst{\HOLTokenWeakEQ} \HOLFreeVar{Q}\hfill{[UNIQUE\_SOLUTION\_OF\_CONTRACTIONS]}
\end{alltt}

%s4.12 #&#
\subsection{Unique solution of rooted contractions}
 \label{sec4.12}

The formal proof of ``unique solution of rooted contractions theorem''
(\reftext{Theorem~\ref{t:rcontraBisimulationU}}, \univariate version) shares the
same initial proof steps as \reftext{Theorem~\ref{t:contraBisimulationU}}. It then
requires a few more steps to handle rooted bisimilarity in the conclusion.
The two proofs are similar, mostly because the main property needed from
contraction and rooted contraction is precongruence. Below is the formal
version of \reftext{Theorem~\ref{t:rcontraBisimulationU}}:
%
%a4.12 #&#
\begin{alltt}
\HOLTokenTurnstile{} \HOLConst{WG} \HOLFreeVar{E} \HOLSymConst{\HOLTokenConj{}} \HOLFreeVar{P} \HOLSymConst{\HOLTokenObsContracts} \HOLFreeVar{E} \HOLFreeVar{P} \HOLSymConst{\HOLTokenConj{}} \HOLFreeVar{Q} \HOLSymConst{\HOLTokenObsContracts} \HOLFreeVar{E} \HOLFreeVar{Q} \HOLSymConst{\HOLTokenImp{}} \HOLFreeVar{P} \HOLSymConst{\HOLTokenObsCongr} \HOLFreeVar{Q}\hfill{[UNIQUE\_SOLUTION\_OF\_ROOTED\_CONTRACTIONS]}
\end{alltt}

Having proved the precongruence property of rooted contraction ($
\rcontr $), now we can use weakly guarded expressions in the above theorem,
which has no more constraints on summation (as shown by the appearance
of \HOLinline{\HOLConst{WG}} rather than \HOLinline{\HOLConst{WGS}}).

Having removed the constraints on summation, this result is close to Milner's
original `unique solution of equations' theorem for \emph{strong} bisimilarity
(\reftext{Theorem~\ref{t:Mil89s1}})~--- the same weakly guarded context (\HOLinline{\HOLConst{WG}})
is required. In contrast, Milner's ``unique solution of equations'' theorem
for rooted bisimilarity ($\rapprox $) (\reftext{Theorem~\ref{t:Mil89s3}}), has more
rigid constraints, as the \emph{equations} must be both guarded and sequential.

%s5 #&#
\section{The \multivariate formalisation}
%%LEAP%%%\label{sec5}
 \label{sec:multivariate}

In this section we attack the case of multiple equations (the `\multivariate '
case), for two major theorems earlier discussed: Milner's ``unique solution
of equations (for $\sim $)'' (\reftext{Theorem~\ref{t:Mil89s1}}) and the ``unique
solution of rooted contractions'' (\reftext{Theorem~\ref{t:rcontraBisimulationU}}).
The formalisation supports finitely many equations/contractions and equation
variables.\footnote{The original theorems hold for even infinitely many
equations and equation variables.} We chose these two theorems because
of their relevance, and because they well illustrate the work needed in
the \multivariate case.

The central problem of the \multivariate formalisation is the representation
of \multivariate CCS equations (expressions and contexts). In the
\univariate case, $\lambda $-functions are used for representing
\univariate CCS equations, and variable substitutions are simply applications
of $\lambda $-functions to CCS terms. This idea, however, cannot be extended
to the \multivariate case, as we do not have a fixed number of variables
to deal with.

In the literature (\citep[p.~102]{Gorrieri:2015jt}, e.g.), the variables
$X_i$ of a system of equations $\{X_i = E_i\}_{i\in I}$ are usually considered
as \emph{process variables} outside the CCS syntax. Then, an equation body
$E_i$ is an \emph{open} expression built with CCS operators plus these equation
variables. In a formalisation, this means either defining a whole new datatype
(with the new equation variables), in which each CCS operator must be duplicated,
or adding equation variables as a new primitive to the existing CCS type.
In either cases the \emph{disjointness} between equation variables and agent
variables is syntactically guaranteed. Both solutions are rather combersome,
and require a non-trivial modification of the existing \univariate formalisation.

In our work, we have followed Milner's original approach~\cite{milner1990operational},
using the \emph{same} alphabet for both agent variables and equation variables.
This approach allows a large reuse of the \univariate formalisation. For
instance, the variable substitution in the SOS rule \texttt{REC} can be
reused for substituting equation variables. However, care is needed in
the proofs of many fundamental lemmas, mostly due to variable capture issues
between free equation variables and bound agent variables.

%s5.1 #&#
\subsection{Free and bound variables}
 \label{sec5.1}

As mentioned at the beginning of Section~\ref{s:eq}, within our CCS syntax
an agent variable may occur outside the recursion in which it is bound.
Thus the same variable can appear both \emph{free} and \emph{bound} in a
CCS term. We denote the set of bound variables of a given CCS term
$E$ (the variables that are bound in a recursion subexpression of
$E$) as $\bvvtex{E}$ (or \HOLinline{\HOLConst{BV}\\\;\HOLFreeVar{E}} in
HOL), and the set of free variables as $\fvvtex{E}$ (or
\HOLinline{\HOLConst{FV}\\\;\HOLFreeVar{E}} in HOL). Both
\HOLinline{\HOLConst{BV}} and \HOLinline{\HOLConst{FV}} have the type ``\HOLinline{\ensuremath{(}\ensuremath{\alpha}, \ensuremath{\beta}\ensuremath{)} \HOLTyOp{CCS} \HOLTokenTransEnd \ensuremath{\alpha} \HOLTokenTransEnd \HOLTyOp{bool}}'',
i.e. functions taking CCS terms and returning (finite) sets of variables
(of type \HOLinline{\ensuremath{\alpha}}). For their definition the interesting
cases are those of recursion and agent variables, shown below (here
\texttt{DELETE} and \texttt{INSERT} are set-theoretic operators of HOL's
\texttt{pred\_set} theory):
%
\begin{center}
%
\begin{tabular}{ll}
\hline
\HOLConst{FV} \ensuremath{(}\HOLConst{var} \HOLFreeVar{X}\ensuremath{)} \HOLTokenDefEquality{} \HOLTokenLeftbrace{}\HOLFreeVar{X}\HOLTokenRightbrace{} & \HOLConst{FV} \ensuremath{(}\HOLConst{rec} \HOLFreeVar{X} \HOLFreeVar{p}\ensuremath{)} \HOLTokenDefEquality{} \HOLConst{FV} \HOLFreeVar{p} \HOLConst{DELETE} \HOLFreeVar{X} \\
\HOLConst{BV} \ensuremath{(}\HOLConst{var} \HOLFreeVar{X}\ensuremath{)} \HOLTokenDefEquality{} \HOLSymConst{\HOLTokenEmpty{}} & \HOLConst{BV} \ensuremath{(}\HOLConst{rec} \HOLFreeVar{X} \HOLFreeVar{p}\ensuremath{)} \HOLTokenDefEquality{} \HOLFreeVar{X} \HOLConst{INSERT} \HOLConst{BV} \HOLFreeVar{p} \\
\hline
\end{tabular}
%
\end{center}
%
Furthermore, $E$ is a process, written
\HOLinline{\HOLConst{IS\_PROC} \HOLFreeVar{E}}, if it does not contain any
free variable, i.e. $\fvvtex{E} = \emptyset $:
%
%a5.1 #&#
\begin{alltt}
   \HOLConst{IS\_PROC} \HOLFreeVar{E} \HOLTokenDefEquality{} \HOLConst{FV} \HOLFreeVar{E} \HOLSymConst{\ensuremath{=}} \HOLSymConst{\HOLTokenEmpty{}}\hfill{[IS\_PROC\_def]}
\end{alltt}
%
And a list of CCS processes can be asserted by
\HOLinline{\HOLConst{ALL\_PROC}} defined upon
\HOLinline{\HOLConst{IS\_PROC}}:
%
%a5.1 #&#
\begin{alltt}
   \HOLConst{ALL\_PROC} \HOLFreeVar{Es} \HOLTokenDefEquality{} \HOLConst{EVERY} \HOLConst{IS\_PROC} \HOLFreeVar{Es}\hfill{[ALL\_PROC\_def]}
\end{alltt}

The set of free and bound variables of a term need not be disjoint, e.g.~in
$X + \recu X E$. More importantly, when going from a CCS expression to
its sub-expressions, the set of free variables may increase, while the
set of bound variables either remains the same or decreases. For instance,
$\fvvtex{\recu X (\mu .X)} = \emptyset $, while
$\fvvtex{\mu .X} = \{X\}$. This property of $\fvvtex{\cdot}$ brings some
\xch{difficulties}{difficulities} in transition inductions. As an evidence, we prove the following
fundamental property of $\fvvtex{\cdot}$~\citep[p.~1209]{milner1990operational}:
%
%p5.1 #&#
\begin{proposition}
 \label{prop:transFV}
The derivatives of a process are themselves processes, i.e. if
$E \arr{\mu} E'$ and $\fvvtex{E} = \emptyset $, then
$\fvvtex{E'} = \emptyset $. Formally:
%
%a5.1 #&#
\begin{alltt}
\HOLTokenTurnstile{} \HOLSymConst{\HOLTokenForall{}}\HOLBoundVar{E} \HOLBoundVar{u} \ensuremath{\HOLBoundVar{E}\sp{\prime}}. \HOLBoundVar{E} \HOLTokenTransBegin\HOLBoundVar{u}\HOLTokenTransEnd \ensuremath{\HOLBoundVar{E}\sp{\prime}} \HOLSymConst{\HOLTokenConj{}} \HOLConst{IS\_PROC} \HOLBoundVar{E} \HOLSymConst{\HOLTokenImp{}} \HOLConst{IS\_PROC} \ensuremath{\HOLBoundVar{E}\sp{\prime}}\hfill{[TRANS\_PROC]}
\end{alltt}
%
\end{proposition}

\begin{proof}
We prove a stronger result, asserting that the set of free variables in
a CCS process does not increase in its derivatives:
%
%a5.1 #&#
\begin{alltt}
\HOLTokenTurnstile{} \HOLSymConst{\HOLTokenForall{}}\HOLBoundVar{E} \HOLBoundVar{u} \ensuremath{\HOLBoundVar{E}\sp{\prime}}. \HOLBoundVar{E} \HOLTokenTransBegin\HOLBoundVar{u}\HOLTokenTransEnd \ensuremath{\HOLBoundVar{E}\sp{\prime}} \HOLSymConst{\HOLTokenImp{}} \HOLConst{FV} \ensuremath{\HOLBoundVar{E}\sp{\prime}} \HOLSymConst{\HOLTokenSubset{}} \HOLConst{FV} \HOLBoundVar{E}\hfill{[TRANS\_FV]}
\end{alltt}

In~\cite{milner1990operational}, the proof of the above property is commented
as ``an easy action induction''. As a matter of fact, in the HOL proof
``action induction'' (also known as \emph{transition induction}) becomes
a form of higher-order application of the following
\emph{induction principle} (generated together with the SOS rules), which
essentially says that \HOLinline{\HOLConst{TRANS}} is the smallest relation
satisfying the SOS rules:
%
%a5.1 #&#
\begin{alltt}
\HOLTokenTurnstile{} \HOLSymConst{\HOLTokenForall{}}\HOLBoundVar{P}.
       \ensuremath{(}\HOLSymConst{\HOLTokenForall{}}\HOLBoundVar{E} \HOLBoundVar{u}. \HOLBoundVar{P} \ensuremath{(}\HOLBoundVar{u}\HOLSymConst{\ensuremath{\ldotp}}\HOLBoundVar{E}\ensuremath{)} \HOLBoundVar{u} \HOLBoundVar{E}\ensuremath{)} \HOLSymConst{\HOLTokenConj{}} \ensuremath{(}\HOLSymConst{\HOLTokenForall{}}\HOLBoundVar{E} \HOLBoundVar{u} \ensuremath{\HOLBoundVar{E}\sb{\mathrm{1}}} \ensuremath{\HOLBoundVar{E}\sp{\prime}}. \HOLBoundVar{P} \HOLBoundVar{E} \HOLBoundVar{u} \ensuremath{\HOLBoundVar{E}\sb{\mathrm{1}}} \HOLSymConst{\HOLTokenImp{}} \HOLBoundVar{P} \ensuremath{(}\HOLBoundVar{E} \HOLSymConst{\ensuremath{+}} \ensuremath{\HOLBoundVar{E}\sp{\prime}}\ensuremath{)} \HOLBoundVar{u} \ensuremath{\HOLBoundVar{E}\sb{\mathrm{1}}}\ensuremath{)} \HOLSymConst{\HOLTokenConj{}}
       \ensuremath{(}\HOLSymConst{\HOLTokenForall{}}\HOLBoundVar{E} \HOLBoundVar{u} \ensuremath{\HOLBoundVar{E}\sb{\mathrm{1}}} \ensuremath{\HOLBoundVar{E}\sp{\prime}}. \HOLBoundVar{P} \HOLBoundVar{E} \HOLBoundVar{u} \ensuremath{\HOLBoundVar{E}\sb{\mathrm{1}}} \HOLSymConst{\HOLTokenImp{}} \HOLBoundVar{P} \ensuremath{(}\ensuremath{\HOLBoundVar{E}\sp{\prime}} \HOLSymConst{\ensuremath{+}} \HOLBoundVar{E}\ensuremath{)} \HOLBoundVar{u} \ensuremath{\HOLBoundVar{E}\sb{\mathrm{1}}}\ensuremath{)} \HOLSymConst{\HOLTokenConj{}}
       \ensuremath{(}\HOLSymConst{\HOLTokenForall{}}\HOLBoundVar{E} \HOLBoundVar{u} \ensuremath{\HOLBoundVar{E}\sb{\mathrm{1}}} \ensuremath{\HOLBoundVar{E}\sp{\prime}}. \HOLBoundVar{P} \HOLBoundVar{E} \HOLBoundVar{u} \ensuremath{\HOLBoundVar{E}\sb{\mathrm{1}}} \HOLSymConst{\HOLTokenImp{}} \HOLBoundVar{P} \ensuremath{(}\HOLBoundVar{E} \HOLSymConst{\ensuremath{\mid}} \ensuremath{\HOLBoundVar{E}\sp{\prime}}\ensuremath{)} \HOLBoundVar{u} \ensuremath{(}\ensuremath{\HOLBoundVar{E}\sb{\mathrm{1}}} \HOLSymConst{\ensuremath{\mid}} \ensuremath{\HOLBoundVar{E}\sp{\prime}}\ensuremath{)}\ensuremath{)} \HOLSymConst{\HOLTokenConj{}}
       \ensuremath{(}\HOLSymConst{\HOLTokenForall{}}\HOLBoundVar{E} \HOLBoundVar{u} \ensuremath{\HOLBoundVar{E}\sb{\mathrm{1}}} \ensuremath{\HOLBoundVar{E}\sp{\prime}}. \HOLBoundVar{P} \HOLBoundVar{E} \HOLBoundVar{u} \ensuremath{\HOLBoundVar{E}\sb{\mathrm{1}}} \HOLSymConst{\HOLTokenImp{}} \HOLBoundVar{P} \ensuremath{(}\ensuremath{\HOLBoundVar{E}\sp{\prime}} \HOLSymConst{\ensuremath{\mid}} \HOLBoundVar{E}\ensuremath{)} \HOLBoundVar{u} \ensuremath{(}\ensuremath{\HOLBoundVar{E}\sp{\prime}} \HOLSymConst{\ensuremath{\mid}} \ensuremath{\HOLBoundVar{E}\sb{\mathrm{1}}}\ensuremath{)}\ensuremath{)} \HOLSymConst{\HOLTokenConj{}}
       \ensuremath{(}\HOLSymConst{\HOLTokenForall{}}\HOLBoundVar{E} \HOLBoundVar{l} \ensuremath{\HOLBoundVar{E}\sb{\mathrm{1}}} \ensuremath{\HOLBoundVar{E}\sp{\prime}} \ensuremath{\HOLBoundVar{E}\sb{\mathrm{2}}}.
            \HOLBoundVar{P} \HOLBoundVar{E} \ensuremath{(}\HOLConst{label} \HOLBoundVar{l}\ensuremath{)} \ensuremath{\HOLBoundVar{E}\sb{\mathrm{1}}} \HOLSymConst{\HOLTokenConj{}} \HOLBoundVar{P} \ensuremath{\HOLBoundVar{E}\sp{\prime}} \ensuremath{(}\HOLConst{label} \ensuremath{(}\HOLConst{COMPL} \HOLBoundVar{l}\ensuremath{)}\ensuremath{)} \ensuremath{\HOLBoundVar{E}\sb{\mathrm{2}}} \HOLSymConst{\HOLTokenImp{}}
            \HOLBoundVar{P} \ensuremath{(}\HOLBoundVar{E} \HOLSymConst{\ensuremath{\mid}} \ensuremath{\HOLBoundVar{E}\sp{\prime}}\ensuremath{)} \HOLSymConst{\ensuremath{\tau}} \ensuremath{(}\ensuremath{\HOLBoundVar{E}\sb{\mathrm{1}}} \HOLSymConst{\ensuremath{\mid}} \ensuremath{\HOLBoundVar{E}\sb{\mathrm{2}}}\ensuremath{)}\ensuremath{)} \HOLSymConst{\HOLTokenConj{}}
       \ensuremath{(}\HOLSymConst{\HOLTokenForall{}}\HOLBoundVar{E} \HOLBoundVar{u} \ensuremath{\HOLBoundVar{E}\sp{\prime}} \HOLBoundVar{l} \HOLBoundVar{L}.
            \HOLBoundVar{P} \HOLBoundVar{E} \HOLBoundVar{u} \ensuremath{\HOLBoundVar{E}\sp{\prime}} \HOLSymConst{\HOLTokenConj{}} \ensuremath{(}\HOLBoundVar{u} \HOLSymConst{\ensuremath{=}} \HOLSymConst{\ensuremath{\tau}} \HOLSymConst{\HOLTokenDisj{}} \HOLBoundVar{u} \HOLSymConst{\ensuremath{=}} \HOLConst{label} \HOLBoundVar{l} \HOLSymConst{\HOLTokenConj{}} \HOLBoundVar{l} \HOLSymConst{\HOLTokenNotIn{}} \HOLBoundVar{L} \HOLSymConst{\HOLTokenConj{}} \HOLConst{COMPL} \HOLBoundVar{l} \HOLSymConst{\HOLTokenNotIn{}} \HOLBoundVar{L}\ensuremath{)} \HOLSymConst{\HOLTokenImp{}}
            \HOLBoundVar{P} \ensuremath{(}\ensuremath{(\nu}\HOLBoundVar{L}\ensuremath{)} \HOLBoundVar{E}\ensuremath{)} \HOLBoundVar{u} \ensuremath{(}\ensuremath{(\nu}\HOLBoundVar{L}\ensuremath{)} \ensuremath{\HOLBoundVar{E}\sp{\prime}}\ensuremath{)}\ensuremath{)} \HOLSymConst{\HOLTokenConj{}}
       \ensuremath{(}\HOLSymConst{\HOLTokenForall{}}\HOLBoundVar{E} \HOLBoundVar{u} \ensuremath{\HOLBoundVar{E}\sp{\prime}} \HOLBoundVar{rf}. \HOLBoundVar{P} \HOLBoundVar{E} \HOLBoundVar{u} \ensuremath{\HOLBoundVar{E}\sp{\prime}} \HOLSymConst{\HOLTokenImp{}} \HOLBoundVar{P} \ensuremath{(}\HOLConst{relab} \HOLBoundVar{E} \HOLBoundVar{rf}\ensuremath{)} \ensuremath{(}\HOLConst{relabel} \HOLBoundVar{rf} \HOLBoundVar{u}\ensuremath{)} \ensuremath{(}\HOLConst{relab} \ensuremath{\HOLBoundVar{E}\sp{\prime}} \HOLBoundVar{rf}\ensuremath{)}\ensuremath{)} \HOLSymConst{\HOLTokenConj{}}
       \ensuremath{(}\HOLSymConst{\HOLTokenForall{}}\HOLBoundVar{E} \HOLBoundVar{u} \HOLBoundVar{X} \ensuremath{\HOLBoundVar{E}\sb{\mathrm{1}}}. \HOLBoundVar{P} \ensuremath{(}\ensuremath{[}\HOLConst{rec} \HOLBoundVar{X} \HOLBoundVar{E}\ensuremath{/}\HOLBoundVar{X}\ensuremath{]} \HOLBoundVar{E}\ensuremath{)} \HOLBoundVar{u} \ensuremath{\HOLBoundVar{E}\sb{\mathrm{1}}} \HOLSymConst{\HOLTokenImp{}} \HOLBoundVar{P} \ensuremath{(}\HOLConst{rec} \HOLBoundVar{X} \HOLBoundVar{E}\ensuremath{)} \HOLBoundVar{u} \ensuremath{\HOLBoundVar{E}\sb{\mathrm{1}}}\ensuremath{)} \HOLSymConst{\HOLTokenImp{}}
       \HOLSymConst{\HOLTokenForall{}}\ensuremath{\HOLBoundVar{a}\sb{\mathrm{0}}} \ensuremath{\HOLBoundVar{a}\sb{\mathrm{1}}} \ensuremath{\HOLBoundVar{a}\sb{\mathrm{2}}}. \ensuremath{\HOLBoundVar{a}\sb{\mathrm{0}}} \HOLTokenTransBegin\ensuremath{\HOLBoundVar{a}\sb{\mathrm{1}}}\HOLTokenTransEnd \ensuremath{\HOLBoundVar{a}\sb{\mathrm{2}}} \HOLSymConst{\HOLTokenImp{}} \HOLBoundVar{P} \ensuremath{\HOLBoundVar{a}\sb{\mathrm{0}}} \ensuremath{\HOLBoundVar{a}\sb{\mathrm{1}}} \ensuremath{\HOLBoundVar{a}\sb{\mathrm{2}}}\hfill{[TRANS\_ind]}
\end{alltt}
%
The above long theorem is of the form
\HOLinline{\HOLSymConst{\HOLTokenForall{}}\HOLBoundVar{P}.\\\;\HOLFreeVar{X}\\\;\HOLSymConst{\HOLTokenImp{}}\\\;\HOLSymConst{\HOLTokenForall{}}\HOLBoundVar{E}\\\;\HOLBoundVar{u}\\\;\ensuremath{\HOLBoundVar{E}\sp{\prime}}.\\\;\HOLBoundVar{E}\\\;\HOLTokenTransBegin \HOLBoundVar{u}\HOLTokenTransEnd \\\;\ensuremath{\HOLBoundVar{E}\sp{\prime}}\\\;\HOLSymConst{\HOLTokenImp{}}\\\;\HOLBoundVar{P}\\\;\HOLBoundVar{E}\\\;\HOLBoundVar{u}\\\;\ensuremath{\HOLBoundVar{E}\sp{\prime}}}
(with $a_0, a_1, a_2$ renamed), where the outermost universal quantifier
$P$ is a higher-order proposition (taking $E$, $\mu $ and $E'$), and
$X$ is another higher-order proposition. The proof goal is actually in
the form of
\HOLinline{\HOLSymConst{\HOLTokenForall{}}\HOLBoundVar{E}\\\;\HOLBoundVar{u}\\\;\ensuremath{\HOLBoundVar{E}\sp{\prime}}.\\\;\HOLBoundVar{E}\\\;\HOLTokenTransBegin \HOLBoundVar{u}\HOLTokenTransEnd \\\;\ensuremath{\HOLBoundVar{E}\sp{\prime}}\\\;\HOLSymConst{\HOLTokenImp{}}\\\;\HOLFreeVar{P}\\\;\HOLBoundVar{E}\\\;\HOLBoundVar{u}\\\;\ensuremath{\HOLBoundVar{E}\sp{\prime}}},
where
\HOLinline{\HOLFreeVar{P}\\\;\HOLSymConst{\ensuremath{=}}\\\;\ensuremath{(}\HOLTokenLambda{}\HOLBoundVar{E}\\\;\HOLBoundVar{u}\\\;\ensuremath{\HOLBoundVar{E}\sp{\prime}}.\\\;\HOLConst{FV}\\\;\ensuremath{\HOLBoundVar{E}\sp{\prime}}\\\;\HOLSymConst{\HOLTokenSubset{}}\\\;\HOLConst{FV}\\\;\HOLBoundVar{E}\ensuremath{)}}.
Thus, if we can prove $X$ under this specific $P$, by Modus Ponens (MP)
the original proof is completed. Now the goal can be reduced to several
conjunct subgoals, each corresponding to one SOS rule. For instance, in
the subgoal for \texttt{SUM1} we need to prove
$\fvvtex{E_1} \subseteq \fvvtex{E} \Longrightarrow \fvvtex{E_1}
\subseteq \fvvtex{E + E'}$, which holds as
$\fvvtex{E} \subseteq \fvvtex{E + E'} = \fvvtex{E} \cup \fvvtex{E'}$. Eventually
we have only the following goal left (the term above the dash line is the
goal, those below the line are assumptions):
%
%a5.1 #&#
\begin{alltt}
        \HOLinline{\HOLConst{FV} \ensuremath{\HOLFreeVar{E}\sp{\prime}} \HOLSymConst{\HOLTokenSubset{}} \HOLConst{FV} \HOLFreeVar{E} \HOLConst{DELETE} \HOLFreeVar{X}}
   ------------------------------------
    0.  \HOLinline{\HOLConst{FV} \ensuremath{\HOLFreeVar{E}\sp{\prime}} \HOLSymConst{\HOLTokenSubset{}} \HOLConst{FV} \ensuremath{(}\ensuremath{[}\HOLConst{rec} \HOLFreeVar{X} \HOLFreeVar{E}\ensuremath{/}\HOLFreeVar{X}\ensuremath{]} \HOLFreeVar{E}\ensuremath{)}}
\end{alltt}
%
Note that
\HOLinline{\HOLConst{FV}\\\;\HOLFreeVar{E}\\\;\HOLConst{DELETE}\\\;\HOLFreeVar{X}\\\;\HOLSymConst{\ensuremath{=}}\\\;\HOLConst{FV}\\\;\ensuremath{(}\HOLConst{rec}\\\;\HOLFreeVar{X}\\\;\HOLFreeVar{E}\ensuremath{)}},
thus the current proof goal is indeed the consequence of action induction
on the recursion operator. Here the problem is that we know nothing about
\HOLinline{\HOLConst{FV}\\\;\ensuremath{(}\ensuremath{[}\HOLConst{rec}\\\;\HOLFreeVar{X}\\\;\HOLFreeVar{E}\ensuremath{/}\HOLFreeVar{X}\ensuremath{]}\\\;\HOLFreeVar{E}\ensuremath{)}}.
To further proceed, we need to prove some other (easier) lemmas. First,
the next lemma can be proven by induction on $E$ and a few basic set-theoretic
facts:
%
%a5.1 #&#
\begin{alltt}
\HOLTokenTurnstile{} \HOLSymConst{\HOLTokenForall{}}\HOLBoundVar{X} \HOLBoundVar{E} \ensuremath{\HOLBoundVar{E}\sp{\prime}}. \HOLConst{FV} \ensuremath{(}\ensuremath{[}\ensuremath{\HOLBoundVar{E}\sp{\prime}}\ensuremath{/}\HOLBoundVar{X}\ensuremath{]} \HOLBoundVar{E}\ensuremath{)} \HOLSymConst{\HOLTokenSubset{}} \HOLConst{FV} \HOLBoundVar{E} \HOLSymConst{\HOLTokenUnion{}} \HOLConst{FV} \ensuremath{\HOLBoundVar{E}\sp{\prime}}\hfill{[FV\_SUBSET]}
\end{alltt}
%
Now, if we take $E' = \recu X E$ in the above lemma, we get
\HOLinline{\HOLConst{FV}\\\;\ensuremath{(}\ensuremath{[}\HOLConst{rec}\\\;\HOLFreeVar{X}\\\;\HOLFreeVar{E}\ensuremath{/}\HOLFreeVar{X}\ensuremath{]}\\\;\HOLFreeVar{E}\ensuremath{)}\\\;\HOLSymConst{\HOLTokenSubset{}}\\\;\HOLConst{FV}\\\;\HOLFreeVar{E}\\\;\HOLSymConst{\HOLTokenUnion{}}\\\;\HOLConst{FV}\\\;\ensuremath{(}\HOLConst{rec}\\\;\HOLFreeVar{X}\\\;\HOLFreeVar{E}\ensuremath{)}}
$=$
\HOLinline{\HOLConst{FV}\\\;\HOLFreeVar{E}\\\;\HOLSymConst{\HOLTokenUnion{}}}\break
\HOLinline{\ensuremath{(}\HOLConst{FV}\\\;\HOLFreeVar{E}\\\;\HOLConst{DELETE}\\\;\HOLFreeVar{X}\ensuremath{)}}
$=$ \HOLinline{\HOLConst{FV}\\\;\HOLFreeVar{E}}, i.e. the following lemma:
%
%a5.1 #&#
\begin{alltt}
\HOLTokenTurnstile{} \HOLSymConst{\HOLTokenForall{}}\HOLBoundVar{X} \HOLBoundVar{E}. \HOLConst{FV} \ensuremath{(}\ensuremath{[}\HOLConst{rec} \HOLBoundVar{X} \HOLBoundVar{E}\ensuremath{/}\HOLBoundVar{X}\ensuremath{]} \HOLBoundVar{E}\ensuremath{)} \HOLSymConst{\HOLTokenSubset{}} \HOLConst{FV} \HOLBoundVar{E}\hfill{[FV\_SUBSET\_REC]}
\end{alltt}
%
Thus we can enrich the assumptions of the current proof goal with above
lemma, and obtain
\HOLinline{\HOLConst{FV} \ensuremath{\HOLFreeVar{E}\sp{\prime}} \HOLSymConst{\HOLTokenSubset{}} \HOLConst{FV} \HOLFreeVar{E}}
by the transitivity of $\subseteq $:
%
%a5.1 #&#
\begin{alltt}
        \HOLinline{\HOLConst{FV} \ensuremath{\HOLFreeVar{E}\sp{\prime}} \HOLSymConst{\HOLTokenSubset{}} \HOLConst{FV} \HOLFreeVar{E} \HOLConst{DELETE} \HOLFreeVar{X}}
   ------------------------------------
    0.  \HOLinline{\HOLConst{FV} \ensuremath{\HOLFreeVar{E}\sp{\prime}} \HOLSymConst{\HOLTokenSubset{}} \HOLConst{FV} \ensuremath{(}\ensuremath{[}\HOLConst{rec} \HOLFreeVar{X} \HOLFreeVar{E}\ensuremath{/}\HOLFreeVar{X}\ensuremath{]} \HOLFreeVar{E}\ensuremath{)}}
    1.  \HOLinline{\HOLConst{FV} \ensuremath{(}\ensuremath{[}\HOLConst{rec} \HOLFreeVar{X} \HOLFreeVar{E}\ensuremath{/}\HOLFreeVar{X}\ensuremath{]} \HOLFreeVar{E}\ensuremath{)} \HOLSymConst{\HOLTokenSubset{}} \HOLConst{FV} \HOLFreeVar{E}}
    2.  \HOLinline{\HOLConst{FV} \ensuremath{\HOLFreeVar{E}\sp{\prime}} \HOLSymConst{\HOLTokenSubset{}} \HOLConst{FV} \HOLFreeVar{E}}
\end{alltt}
%
Knowing
\HOLinline{\HOLConst{FV} \ensuremath{\HOLFreeVar{E}\sp{\prime}} \HOLSymConst{\HOLTokenSubset{}} \HOLConst{FV} \HOLFreeVar{E}}
we cannot prove
\HOLinline{\HOLConst{FV} \ensuremath{\HOLFreeVar{E}\sp{\prime}} \HOLSymConst{\HOLTokenSubset{}} \HOLConst{FV} \HOLFreeVar{E} \HOLConst{DELETE} \HOLFreeVar{X}}.
However, if we knew
\HOLinline{\HOLFreeVar{X} \HOLSymConst{\HOLTokenNotIn{}} \HOLConst{FV} \ensuremath{\HOLFreeVar{E}\sp{\prime}}},
then
\HOLinline{\HOLConst{FV} \ensuremath{\HOLFreeVar{E}\sp{\prime}} \HOLConst{DELETE}}\break \HOLinline{\HOLFreeVar{X} \HOLSymConst{\ensuremath{=}} \HOLConst{FV} \ensuremath{\HOLFreeVar{E}\sp{\prime}}},
and then
\HOLinline{\HOLConst{FV} \ensuremath{\HOLFreeVar{E}\sp{\prime}} \HOLSymConst{\HOLTokenSubset{}} \HOLConst{FV} \HOLFreeVar{E} \HOLSymConst{\HOLTokenImp{}} \HOLConst{FV} \ensuremath{\HOLFreeVar{E}\sp{\prime}} \HOLConst{DELETE} \HOLFreeVar{X} \HOLSymConst{\HOLTokenSubset{}} \HOLConst{FV} \HOLFreeVar{E} \HOLConst{DELETE} \HOLFreeVar{X}},
no matter if
\HOLinline{\HOLFreeVar{X} \HOLSymConst{\HOLTokenIn{}} \HOLConst{FV} \HOLFreeVar{E}}
or not, and the proof would complete. Thus it remains to show that
%
%a5.1 #&#
\begin{alltt}
        \HOLinline{\HOLFreeVar{X} \HOLSymConst{\HOLTokenNotIn{}} \HOLConst{FV} \ensuremath{\HOLFreeVar{E}\sp{\prime}}}
   ------------------------------------
    0.  \HOLinline{\HOLConst{FV} \ensuremath{\HOLFreeVar{E}\sp{\prime}} \HOLSymConst{\HOLTokenSubset{}} \HOLConst{FV} \ensuremath{(}\ensuremath{[}\HOLConst{rec} \HOLFreeVar{X} \HOLFreeVar{E}\ensuremath{/}\HOLFreeVar{X}\ensuremath{]} \HOLFreeVar{E}\ensuremath{)}}
    1.  \HOLinline{\HOLConst{FV} \ensuremath{(}\ensuremath{[}\HOLConst{rec} \HOLFreeVar{X} \HOLFreeVar{E}\ensuremath{/}\HOLFreeVar{X}\ensuremath{]} \HOLFreeVar{E}\ensuremath{)} \HOLSymConst{\HOLTokenSubset{}} \HOLConst{FV} \HOLFreeVar{E}}
    2.  \HOLinline{\HOLConst{FV} \ensuremath{\HOLFreeVar{E}\sp{\prime}} \HOLSymConst{\HOLTokenSubset{}} \HOLConst{FV} \HOLFreeVar{E}}
\end{alltt}
%
Now we try the proof by contradiction (\emph{reductio ad absurdum}): if
the goal does not hold, i.e.
\HOLinline{\HOLFreeVar{X} \HOLSymConst{\HOLTokenIn{}} \HOLConst{FV} \ensuremath{\HOLFreeVar{E}\sp{\prime}}},
then by assumption 0 we have
\HOLinline{\HOLFreeVar{X} \HOLSymConst{\HOLTokenIn{}} \HOLConst{FV} \ensuremath{(}\ensuremath{[}\HOLConst{rec} \HOLFreeVar{X} \HOLFreeVar{E}\ensuremath{/}\HOLFreeVar{X}\ensuremath{]} \HOLFreeVar{E}\ensuremath{)}}:
%
%a5.1 #&#
\begin{alltt}
        F
   ------------------------------------
    0.  \HOLinline{\HOLConst{FV} \ensuremath{\HOLFreeVar{E}\sp{\prime}} \HOLSymConst{\HOLTokenSubset{}} \HOLConst{FV} \ensuremath{(}\ensuremath{[}\HOLConst{rec} \HOLFreeVar{X} \HOLFreeVar{E}\ensuremath{/}\HOLFreeVar{X}\ensuremath{]} \HOLFreeVar{E}\ensuremath{)}}
    1.  \HOLinline{\HOLConst{FV} \ensuremath{(}\ensuremath{[}\HOLConst{rec} \HOLFreeVar{X} \HOLFreeVar{E}\ensuremath{/}\HOLFreeVar{X}\ensuremath{]} \HOLFreeVar{E}\ensuremath{)} \HOLSymConst{\HOLTokenSubset{}} \HOLConst{FV} \HOLFreeVar{E}}
    2.  \HOLinline{\HOLConst{FV} \ensuremath{\HOLFreeVar{E}\sp{\prime}} \HOLSymConst{\HOLTokenSubset{}} \HOLConst{FV} \HOLFreeVar{E}}
    3.  \HOLinline{\HOLFreeVar{X} \HOLSymConst{\HOLTokenIn{}} \HOLConst{FV} \ensuremath{\HOLFreeVar{E}\sp{\prime}}}
    4.  \HOLinline{\HOLFreeVar{X} \HOLSymConst{\HOLTokenIn{}} \HOLConst{FV} \ensuremath{(}\ensuremath{[}\HOLConst{rec} \HOLFreeVar{X} \HOLFreeVar{E}\ensuremath{/}\HOLFreeVar{X}\ensuremath{]} \HOLFreeVar{E}\ensuremath{)}}
\end{alltt}
%
But this is impossible, because all free occurrences of $X$ in $E$ now
become bound in the form of $\recu X E$. In fact, the following lemma can
be proven by induction on $E$:
%
%a5.1 #&#
\begin{alltt}
\HOLTokenTurnstile{} \HOLSymConst{\HOLTokenForall{}}\HOLBoundVar{X} \HOLBoundVar{E} \ensuremath{\HOLBoundVar{E}\sp{\prime}}. \HOLBoundVar{X} \HOLSymConst{\HOLTokenNotIn{}} \HOLConst{FV} \ensuremath{(}\ensuremath{[}\HOLConst{rec} \HOLBoundVar{X} \ensuremath{\HOLBoundVar{E}\sp{\prime}}\ensuremath{/}\HOLBoundVar{X}\ensuremath{]} \HOLBoundVar{E}\ensuremath{)}\hfill{[NOTIN\_FV\_lemma]}
\end{alltt}
%
Adding the above lemma (taking $E' = E$) into the assumption list immediately
causes a contradiction with assumption 4, and the proof finally completes.
\end{proof}

\reftext{Proposition~\ref{prop:transFV}} is essential in the proofs of the
\multivariate versions of all unique-solution theorems. On the other hand,
the analogous result for bound variables is indeed just a simple transition
induction. (The easy proof is omitted.)
%
%p5.2 #&#
\begin{proposition}
 \label{prop:transBV}
if $E \arr{\mu} E'$ then $\bvvtex{E'} \subseteq \bvvtex{E}$, or formally:
%
%a5.2 #&#
\begin{alltt}
\HOLTokenTurnstile{} \HOLSymConst{\HOLTokenForall{}}\HOLBoundVar{E} \HOLBoundVar{u} \ensuremath{\HOLBoundVar{E}\sp{\prime}}. \HOLBoundVar{E} \HOLTokenTransBegin\HOLBoundVar{u}\HOLTokenTransEnd \ensuremath{\HOLBoundVar{E}\sp{\prime}} \HOLSymConst{\HOLTokenImp{}} \HOLConst{BV} \ensuremath{\HOLBoundVar{E}\sp{\prime}} \HOLSymConst{\HOLTokenSubset{}} \HOLConst{BV} \HOLBoundVar{E}\hfill{[TRANS\_BV]}
\end{alltt}
%
\end{proposition}

%s5.2 #&#
\subsection{\Multivariate substitutions}
 \label{sec5.2}

There are two natural ways to implement the \multivariate substitution
$E\{\til P/\til X\}$, which replaces each free occurrence of the variables
$X_i$ in $E$ with the corresponding $P_i$: (1) iterately applying the existing
\univariate \HOLinline{\HOLConst{CCS\_Subst}}; (2) defining a new
\multivariate substitution function which substitutes all variables
$\til X$ in parallel. Note that there is the possibility that
$\til P$ syntactically again contains variables from $\til X$, thus different
orders of iterated substitutions may lead to different results. We thus
prefer to define a \multivariate version of
\HOLinline{\HOLConst{CCS\_Subst}} called
\HOLinline{\HOLConst{CCS\_SUBST}}, based on HOL \texttt{finite\_map} theory~\cite{holdesc}.

A finite map (of type
\HOLinline{\ensuremath{\alpha} \HOLTokenMapto{} \ensuremath{\beta}}) is
like a function (of type
\HOLinline{\ensuremath{\alpha} \HOLTokenTransEnd \ensuremath{\beta}}) having
only finitely many elements in its domain. In HOL, an empty finite map
is denoted as \HOLinline{\HOLConst{FEMPTY}}. If
\HOLinline{\HOLFreeVar{fm}} is a finite map, its domain (as a set of keys)
is denoted as \HOLinline{\HOLConst{FDOM} \HOLFreeVar{fm}}. Applying
\HOLinline{\HOLFreeVar{fm}} on a certain key, say $k$, is denoted by
\HOLinline{\HOLFreeVar{fm} \HOLConst{'} \HOLFreeVar{k}}.

The function \HOLinline{\HOLConst{CCS\_SUBST}} takes a finite map
\HOLinline{\HOLFreeVar{fm}} (of type
\HOLinline{\ensuremath{\alpha} \HOLTokenMapto{} \ensuremath{(}\ensuremath{\alpha}, \ensuremath{\beta}\ensuremath{)} \HOLTyOp{CCS}})
and a CCS expression, and returns another CCS expression in which all occurrences
of variables in the finite domain of \HOLinline{\HOLFreeVar{fm}} are substituted
with the corresponding value in \HOLinline{\HOLFreeVar{fm}}. Initially,
such a finite map can be built from a list of variables
\HOLinline{\HOLFreeVar{Xs}} and the corresponding substituted terms
\HOLinline{\HOLFreeVar{Ps}} by a helper function
\HOLinline{\HOLConst{fromList}} (whose details are omitted here).
\HOLinline{\HOLConst{CCS\_SUBST}\\\;\ensuremath{(}\HOLConst{fromList}\\\;\HOLFreeVar{Xs}\\\;\HOLFreeVar{Ps}\ensuremath{)}\\\;\HOLFreeVar{E}}
is usually abbreviated as
\HOLinline{\ensuremath{[}\HOLFreeVar{Ps}\ensuremath{/}\HOLFreeVar{Xs}\ensuremath{]}\\\;\HOLFreeVar{E}}.
Using finite maps, the substitution mechanism is order-independent. For
most CCS operators, \HOLinline{\HOLConst{CCS\_SUBST}} recursively calls
itself on subterms. The most interesting cases are at agent variables and
recursion:
%
%a5.2 #&#
\begin{alltt}
   \HOLConst{CCS\_SUBST} \HOLFreeVar{fm} \ensuremath{(}\HOLConst{var} \HOLFreeVar{X}\ensuremath{)} \HOLTokenDefEquality{} \HOLKeyword{if} \HOLFreeVar{X} \HOLSymConst{\HOLTokenIn{}} \HOLConst{FDOM} \HOLFreeVar{fm} \HOLKeyword{then} \HOLFreeVar{fm} \HOLConst{'} \HOLFreeVar{X} \HOLKeyword{else} \HOLConst{var} \HOLFreeVar{X}\hfill{[CCS\_SUBST\_var]}
\end{alltt}

\begin{alltt}
   \HOLConst{CCS\_SUBST} \HOLFreeVar{fm} \ensuremath{(}\HOLConst{rec} \HOLFreeVar{X} \HOLFreeVar{E}\ensuremath{)} \HOLTokenDefEquality{}
     \HOLKeyword{if} \HOLFreeVar{X} \HOLSymConst{\HOLTokenIn{}} \HOLConst{FDOM} \HOLFreeVar{fm} \HOLKeyword{then} \HOLConst{rec} \HOLFreeVar{X} \ensuremath{(}\HOLConst{CCS\_SUBST} \ensuremath{(}\HOLFreeVar{fm} \HOLSymConst{\ensuremath{\setminus}} \HOLFreeVar{X}\ensuremath{)} \HOLFreeVar{E}\ensuremath{)}
     \HOLKeyword{else} \HOLConst{rec} \HOLFreeVar{X} \ensuremath{(}\HOLConst{CCS\_SUBST} \HOLFreeVar{fm} \HOLFreeVar{E}\ensuremath{)}\hfill{[CCS\_SUBST\_rec]}
\end{alltt}
%
Variable substitution only occurs on free variables. When the term
$E$ of ``\HOLinline{\HOLConst{CCS\_SUBST}\\\;\HOLFreeVar{fm}\\\;\HOLFreeVar{E}}''
is in the form of $\recu X E'$, if $X$ (a bound variable) is in the domain
of \HOLinline{\HOLFreeVar{fm}}, \HOLinline{\HOLConst{CCS\_SUBST}} must continue
the substitution on $E'$ using a reduced map~--- without $X$~--- so that
all \xch{occurrences}{occurences} of $X$ in $E'$ are correctly bypassed. Since
\HOLinline{\HOLConst{CCS\_SUBST}} only runs once on each subterm, the possible
free variables in \HOLinline{\HOLFreeVar{Ps}} are never substituted.

Below we present some key lemmas about
\HOLinline{\HOLConst{CCS\_SUBST}}, omitting the proofs\footnote{Hereafter,
some new logical constants from HOL's \texttt{pred\_set} and
\texttt{list} theories are used (see~\cite{holdesc} for more details.):
\HOLinline{\HOLConst{DISJOINT}} denotes set disjointness; ``\HOLinline{\HOLConst{set}\\\;\HOLFreeVar{Xs}}''
is the set converted from the list \HOLinline{\HOLFreeVar{Xs}};
\HOLinline{\HOLConst{ALL\_DISTINCT}} says that the elements of a list are
distinct; \HOLinline{\HOLConst{MAP}} is the mapping function from one list
to another; \HOLinline{\HOLConst{EVERY}} means each element of a list satisfies
a predicate; \HOLinline{\HOLConst{ZIP}} creates a new list from two lists
of the same length, and each element of the new list is a pair of elements
from the given lists.}:

%l5.3 #&#
\begin{lemma}[\texttt{CCS\_SUBST\_elim}]
If the free variables of $E$ \xch{are}{is} disjoint with
\HOLinline{\HOLFreeVar{Xs}}, a substitution of
\HOLinline{\HOLFreeVar{Xs}} in $E$ does not change $E$:
%
%a5.3 #&#
\begin{alltt}
\HOLTokenTurnstile{} \HOLConst{DISJOINT} \ensuremath{(}\HOLConst{FV} \HOLFreeVar{E}\ensuremath{)} \ensuremath{(}\HOLConst{set} \HOLFreeVar{Xs}\ensuremath{)} \HOLSymConst{\HOLTokenConj{}} \HOLConst{LENGTH} \HOLFreeVar{Ps} \HOLSymConst{\ensuremath{=}} \HOLConst{LENGTH} \HOLFreeVar{Xs} \HOLSymConst{\HOLTokenImp{}} \ensuremath{[}\HOLFreeVar{Ps}\ensuremath{/}\HOLFreeVar{Xs}\ensuremath{]} \HOLFreeVar{E} \HOLSymConst{\ensuremath{=}} \HOLFreeVar{E}
\end{alltt}
%
\end{lemma}

%l5.4 #&#
\begin{lemma}[\texttt{CCS\_SUBST\_self}]
Substituting each free variable $X$ in $E$ with
\HOLinline{\HOLConst{var}\\\;\HOLFreeVar{X}} (itself) does not change
$E$:
%
%a5.4 #&#
\begin{alltt}
\HOLTokenTurnstile{} \HOLConst{ALL\_DISTINCT} \HOLFreeVar{Xs} \HOLSymConst{\HOLTokenImp{}} \ensuremath{[}\HOLConst{MAP} \HOLConst{var} \HOLFreeVar{Xs}\ensuremath{/}\HOLFreeVar{Xs}\ensuremath{]} \HOLFreeVar{E} \HOLSymConst{\ensuremath{=}} \HOLFreeVar{E}
\end{alltt}
%
\end{lemma}

The next lemma plays an important role. It essentially swaps the order
of two substitutions:
%
%l5.5 #&#
\begin{lemma}[\texttt{CCS\_SUBST\_nested}]
Under certain conditions (to get rid of substitution orders), two nested
substitutions can be converted into a single substitution where the targets
are substituted first:
%
%a5.5 #&#
\begin{alltt}
\HOLTokenTurnstile{} \HOLConst{ALL\_DISTINCT} \HOLFreeVar{Xs} \HOLSymConst{\HOLTokenConj{}} \HOLConst{LENGTH} \HOLFreeVar{Ps} \HOLSymConst{\ensuremath{=}} \HOLConst{LENGTH} \HOLFreeVar{Xs} \HOLSymConst{\HOLTokenConj{}} \HOLConst{LENGTH} \HOLFreeVar{Es} \HOLSymConst{\ensuremath{=}} \HOLConst{LENGTH} \HOLFreeVar{Xs} \HOLSymConst{\HOLTokenConj{}}
   \HOLConst{DISJOINT} \ensuremath{(}\HOLConst{BV} \HOLFreeVar{C}\ensuremath{)} \ensuremath{(}\HOLConst{set} \HOLFreeVar{Xs}\ensuremath{)} \HOLSymConst{\HOLTokenImp{}}
   \ensuremath{[}\HOLFreeVar{Ps}\ensuremath{/}\HOLFreeVar{Xs}\ensuremath{]} \ensuremath{(}\ensuremath{[}\HOLFreeVar{Es}\ensuremath{/}\HOLFreeVar{Xs}\ensuremath{]} \HOLFreeVar{C}\ensuremath{)} \HOLSymConst{\ensuremath{=}} \ensuremath{[}\HOLConst{MAP} \ensuremath{[}\HOLFreeVar{Ps}\ensuremath{/}\HOLFreeVar{Xs}\ensuremath{]} \HOLFreeVar{Es}\ensuremath{/}\HOLFreeVar{Xs}\ensuremath{]} \HOLFreeVar{C}
\end{alltt}
%
\end{lemma}

The next three lemmas show the relative correctness of
\texttt{CCS\_SUBST} with respect to~\texttt{CCS\_Subst}:
%
%l5.6 #&#
\begin{lemma}[\texttt{CCS\_SUBST\_sing}]
If there is only one single variable $X$ in the map (with the target expression
$E'$), then \HOLinline{\HOLConst{CCS\_SUBST}} behaves exactly the same as
(the \univariate ) \HOLinline{\HOLConst{CCS\_Subst}}:
%
%a5.6 #&#
\begin{alltt}
\HOLTokenTurnstile{} \ensuremath{[}\ensuremath{[}\ensuremath{\HOLFreeVar{E}\sp{\prime}}\ensuremath{]}\ensuremath{/}\ensuremath{[}\HOLFreeVar{X}\ensuremath{]}\ensuremath{]} \HOLFreeVar{E} \HOLSymConst{\ensuremath{=}} \ensuremath{[}\ensuremath{\HOLFreeVar{E}\sp{\prime}}\ensuremath{/}\HOLFreeVar{X}\ensuremath{]} \HOLFreeVar{E}
\end{alltt}
%
\end{lemma}

%l5.7 #&#
\begin{lemma}[\texttt{CCS\_SUBST\_reduce}]
Under certain conditions (to get rid of substitution orders), a
\multivariate substitution of variables
\HOLinline{\HOLFreeVar{X}\HOLSymConst{::}\HOLFreeVar{Xs}} can be reduced
to a \multivariate substitution of variables
\HOLinline{\HOLFreeVar{Xs}} and an \univariate substitution of $X$:
%
%a5.7 #&#
\begin{alltt}
\HOLTokenTurnstile{} \HOLSymConst{\HOLTokenNeg{}}\HOLConst{MEM} \HOLFreeVar{X} \HOLFreeVar{Xs} \HOLSymConst{\HOLTokenConj{}} \HOLConst{ALL\_DISTINCT} \HOLFreeVar{Xs} \HOLSymConst{\HOLTokenConj{}} \HOLConst{LENGTH} \HOLFreeVar{Ps} \HOLSymConst{\ensuremath{=}} \HOLConst{LENGTH} \HOLFreeVar{Xs} \HOLSymConst{\HOLTokenConj{}}
   \HOLConst{EVERY} \ensuremath{(}\HOLTokenLambda{}\HOLBoundVar{e}. \HOLFreeVar{X} \HOLSymConst{\HOLTokenNotIn{}} \HOLConst{FV} \HOLBoundVar{e}\ensuremath{)} \HOLFreeVar{Ps} \HOLSymConst{\HOLTokenImp{}}
   \HOLSymConst{\HOLTokenForall{}}\HOLBoundVar{E}. \ensuremath{[}\HOLFreeVar{P}\HOLSymConst{::}\HOLFreeVar{Ps}\ensuremath{/}\HOLFreeVar{X}\HOLSymConst{::}\HOLFreeVar{Xs}\ensuremath{]} \HOLBoundVar{E} \HOLSymConst{\ensuremath{=}} \ensuremath{[}\HOLFreeVar{P}\ensuremath{/}\HOLFreeVar{X}\ensuremath{]} \ensuremath{(}\ensuremath{[}\HOLFreeVar{Ps}\ensuremath{/}\HOLFreeVar{Xs}\ensuremath{]} \HOLBoundVar{E}\ensuremath{)}
\end{alltt}
%
\end{lemma}

%l5.8 #&#
\begin{lemma}[\texttt{CCS\_SUBST\_FOLDR}]
Under certain conditions (to get rid of substitution orders), a
\multivariate substitution can be reduced to repeated applications of
\univariate substitutions of each variable:
%
%a5.8 #&#
\begin{alltt}
\HOLTokenTurnstile{} \HOLConst{ALL\_DISTINCT} \HOLFreeVar{Xs} \HOLSymConst{\HOLTokenConj{}} \HOLConst{LENGTH} \HOLFreeVar{Ps} \HOLSymConst{\ensuremath{=}} \HOLConst{LENGTH} \HOLFreeVar{Xs} \HOLSymConst{\HOLTokenConj{}}
   \HOLConst{EVERY} \ensuremath{(}\HOLTokenLambda{}\ensuremath{(}\HOLBoundVar{x}\HOLSymConst{,}\HOLBoundVar{p}\ensuremath{)}. \HOLConst{FV} \HOLBoundVar{p} \HOLSymConst{\HOLTokenSubset{}} \HOLTokenLeftbrace{}\HOLBoundVar{x}\HOLTokenRightbrace{}\ensuremath{)} \ensuremath{(}\HOLConst{ZIP} \ensuremath{(}\HOLFreeVar{Xs}\HOLSymConst{,}\HOLFreeVar{Ps}\ensuremath{)}\ensuremath{)} \HOLSymConst{\HOLTokenImp{}}
   \ensuremath{[}\HOLFreeVar{Ps}\ensuremath{/}\HOLFreeVar{Xs}\ensuremath{]} \HOLFreeVar{E} \HOLSymConst{\ensuremath{=}} \HOLConst{FOLDR} \ensuremath{(}\HOLTokenLambda{}\ensuremath{(}\HOLBoundVar{x}\HOLSymConst{,}\HOLBoundVar{y}\ensuremath{)} \HOLBoundVar{e}. \ensuremath{[}\HOLBoundVar{y}\ensuremath{/}\HOLBoundVar{x}\ensuremath{]} \HOLBoundVar{e}\ensuremath{)} \HOLFreeVar{E} \ensuremath{(}\HOLConst{ZIP} \ensuremath{(}\HOLFreeVar{Xs}\HOLSymConst{,}\HOLFreeVar{Ps}\ensuremath{)}\ensuremath{)}
\end{alltt}
%
\end{lemma}

Finally, the following two lemmas precisely estimate the free and bound
variables of a substituted term:
%
%l5.9 #&#
\begin{lemma}[\texttt{BV\_SUBSET\_BIGUNION}]
The bound variables of
\HOLinline{\ensuremath{[}\HOLFreeVar{Ps}\ensuremath{/}\HOLFreeVar{Xs}\ensuremath{]}\\\;\HOLFreeVar{E}}
are contained in the union of the bound variables in $E$ and
\HOLinline{\HOLFreeVar{Ps}}.
%
%a5.9 #&#
\begin{alltt}
\HOLTokenTurnstile{} \HOLConst{ALL\_DISTINCT} \HOLFreeVar{Xs} \HOLSymConst{\HOLTokenConj{}} \HOLConst{LENGTH} \HOLFreeVar{Ps} \HOLSymConst{\ensuremath{=}} \HOLConst{LENGTH} \HOLFreeVar{Xs} \HOLSymConst{\HOLTokenConj{}} \HOLConst{DISJOINT} \ensuremath{(}\HOLConst{BV} \HOLFreeVar{E}\ensuremath{)} \ensuremath{(}\HOLConst{set} \HOLFreeVar{Xs}\ensuremath{)} \HOLSymConst{\HOLTokenImp{}}
   \HOLConst{BV} \ensuremath{(}\ensuremath{[}\HOLFreeVar{Ps}\ensuremath{/}\HOLFreeVar{Xs}\ensuremath{]} \HOLFreeVar{E}\ensuremath{)} \HOLSymConst{\HOLTokenSubset{}} \HOLConst{BV} \HOLFreeVar{E} \HOLSymConst{\HOLTokenUnion{}} \HOLConst{\HOLTokenBigUnion{}} \ensuremath{(}\HOLConst{IMAGE} \HOLConst{BV} \ensuremath{(}\HOLConst{set} \HOLFreeVar{Ps}\ensuremath{)}\ensuremath{)}
\end{alltt}
%
\end{lemma}

%l5.10 #&#
\begin{lemma}[\texttt{FV\_SUBSET\_BIGUNION\_PRO}]
The free variables of
\HOLinline{\ensuremath{[}\HOLFreeVar{Ps}\ensuremath{/}\HOLFreeVar{Xs}\ensuremath{]}\\\;\HOLFreeVar{E}}
are contained in the union of the free variables in $E$ and
\HOLinline{\HOLFreeVar{Ps}}, excluding \HOLinline{\HOLFreeVar{Xs}}.
%
%a5.10 #&#
\begin{alltt}
\HOLTokenTurnstile{} \HOLConst{ALL\_DISTINCT} \HOLFreeVar{Xs} \HOLSymConst{\HOLTokenConj{}} \HOLConst{LENGTH} \HOLFreeVar{Ps} \HOLSymConst{\ensuremath{=}} \HOLConst{LENGTH} \HOLFreeVar{Xs} \HOLSymConst{\HOLTokenConj{}} \HOLConst{DISJOINT} \ensuremath{(}\HOLConst{BV} \HOLFreeVar{E}\ensuremath{)} \ensuremath{(}\HOLConst{set} \HOLFreeVar{Xs}\ensuremath{)} \HOLSymConst{\HOLTokenImp{}}
   \HOLConst{FV} \ensuremath{(}\ensuremath{[}\HOLFreeVar{Ps}\ensuremath{/}\HOLFreeVar{Xs}\ensuremath{]} \HOLFreeVar{E}\ensuremath{)} \HOLSymConst{\HOLTokenSubset{}} \HOLConst{FV} \HOLFreeVar{E} \HOLConst{DIFF} \HOLConst{set} \HOLFreeVar{Xs} \HOLSymConst{\HOLTokenUnion{}} \HOLConst{\HOLTokenBigUnion{}} \ensuremath{(}\HOLConst{IMAGE} \HOLConst{FV} \ensuremath{(}\HOLConst{set} \HOLFreeVar{Ps}\ensuremath{)}\ensuremath{)}
\end{alltt}
%
\end{lemma}

In some of the above lemmas, the condition
\HOLinline{\HOLConst{DISJOINT}\\\;\ensuremath{(}\HOLConst{BV}\\\;\HOLFreeVar{E}\ensuremath{)}\\\;\ensuremath{(}\HOLConst{set}\\\;\HOLFreeVar{Xs}\ensuremath{)}}
(the bound variables of $E$ and the free substitution variables are disjoint)
is added to ease the proofs. The condition is not necessary and could be
eliminated, at the price of some extra effort.

%s5.3 #&#
\subsection{\Multivariate (weakly guarded) contexts}
 \label{sec5.3}

As mentioned in Section~\ref{sscontext}, \univariate contexts and their
guarded companions are defined as predicates over $\lambda $-expressions
of type ``\HOLinline{\ensuremath{(}\ensuremath{\alpha}, \ensuremath{\beta}\ensuremath{)} \HOLTyOp{CCS} \HOLTokenTransEnd \ensuremath{(}\ensuremath{\alpha}, \ensuremath{\beta}\ensuremath{)} \HOLTyOp{CCS}}''.
These predicates, such as \HOLinline{\HOLConst{CONTEXT}} (multi-hole contexts),
\HOLinline{\HOLConst{WG}} (weak guarded contexts),
\HOLinline{\HOLConst{SG}} (guarded contexts) and
\HOLinline{\HOLConst{SEQ}} (sequential contexts), are all defined inductively,
i.e.~they are built from some ``holes'' in a bottom-up manner. For instance,
\HOLinline{\HOLConst{WG}\\\;\ensuremath{(}\HOLTokenLambda{}\HOLBoundVar{t}.\\\;\HOLFreeVar{a}\HOLSymConst{\ensuremath{\ldotp}}\HOLBoundVar{t}\\\;\HOLSymConst{\ensuremath{\mid}}\\\;\HOLFreeVar{P}\ensuremath{)}}
holds because
\HOLinline{\HOLConst{WG}\\\;\ensuremath{(}\HOLTokenLambda{}\HOLBoundVar{t}.\\\;\HOLFreeVar{a}\HOLSymConst{\ensuremath{\ldotp}}\HOLBoundVar{t}\ensuremath{)}}
holds as a base case of the inductive definition of
\HOLinline{\HOLConst{WG}}:
%
%a5.3 #&#
\begin{alltt}
\HOLTokenTurnstile{} \HOLSymConst{\HOLTokenForall{}}\HOLBoundVar{a}. \HOLConst{WG} \ensuremath{(}\HOLTokenLambda{}\HOLBoundVar{t}. \HOLBoundVar{a}\HOLSymConst{\ensuremath{\ldotp}}\HOLBoundVar{t}\ensuremath{)}\hfill{[WG1]}
\end{alltt}
%
For any agent variable $X$, we have by $\beta $-reduction:
\HOLinline{\ensuremath{(}\HOLTokenLambda{}\HOLBoundVar{t}.\\\;\HOLFreeVar{l}\HOLSymConst{\ensuremath{\ldotp}}\HOLBoundVar{t}\\\;\HOLSymConst{\ensuremath{+}}\\\;\HOLFreeVar{P}\ensuremath{)}\\\;\ensuremath{(}\HOLConst{var}\\\;\HOLFreeVar{X}\ensuremath{)}\\\;\HOLSymConst{\ensuremath{=}}\\\;\HOLFreeVar{l}\HOLSymConst{\ensuremath{\ldotp}}\HOLConst{var}\\\;\HOLFreeVar{X}\\\;\HOLSymConst{\ensuremath{+}}\\\;\HOLFreeVar{P}}
(or ``$l.X + P$'' in textbook notation), and the expression is weakly guarded
(\reftext{Definition~\ref{def:guardness}}).

It could be possible to inductively define \multivariate contexts and their
guarded variants, but care is needed for variables that occur within a
\xch{recursion}{recusion}.
We \xch{prefer}{have preferred} a different solution, in which the definitions
of \multivariate contexts \xch{are}{is} based on the existing \univariate definitions.
For this, intuitively, we replace a single variable with a hole (viewing
the other variables as constants), so to obtain a single-variable context
on which predicates like \HOLinline{\HOLConst{CONTEXT}} can be used. For
example, to see the weak guardedness of $a.X + b.X + c.Y$, we substitute
each occurrence of a variable with a ``hole'' and consider the resulting
term as a multi-hole context (i.e., a $\lambda $-function); then all such
multi-hole contexts should satisfy \HOLinline{\HOLConst{WG}}, i.e.~the
following results hold:
%
%a5.3 #&#
\begin{alltt}
\HOLTokenTurnstile{} \HOLConst{WG} \ensuremath{(}\HOLTokenLambda{}\HOLBoundVar{t}. \HOLFreeVar{a}\HOLSymConst{\ensuremath{\ldotp}}\HOLBoundVar{t} \HOLSymConst{\ensuremath{+}} \HOLFreeVar{b}\HOLSymConst{\ensuremath{\ldotp}}\HOLBoundVar{t} \HOLSymConst{\ensuremath{+}} \HOLFreeVar{c}\HOLSymConst{\ensuremath{\ldotp}}\HOLConst{var} \HOLFreeVar{Y}\ensuremath{)}
\HOLTokenTurnstile{} \HOLConst{WG} \ensuremath{(}\HOLTokenLambda{}\HOLBoundVar{t}. \HOLFreeVar{a}\HOLSymConst{\ensuremath{\ldotp}}\HOLConst{var} \HOLFreeVar{X} \HOLSymConst{\ensuremath{+}} \HOLFreeVar{b}\HOLSymConst{\ensuremath{\ldotp}}\HOLConst{var} \HOLFreeVar{X} \HOLSymConst{\ensuremath{+}} \HOLFreeVar{c}\HOLSymConst{\ensuremath{\ldotp}}\HOLBoundVar{t}\ensuremath{)}
\end{alltt}
%
Note that
\HOLinline{\HOLFreeVar{a}\HOLSymConst{\ensuremath{\ldotp}}\HOLFreeVar{t}\\\;\HOLSymConst{\ensuremath{+}}\\\;\HOLFreeVar{b}\HOLSymConst{\ensuremath{\ldotp}}\HOLFreeVar{t}\\\;\HOLSymConst{\ensuremath{+}}\\\;\HOLFreeVar{c}\HOLSymConst{\ensuremath{\ldotp}}\HOLConst{var}\\\;\HOLFreeVar{Y}\\\;\HOLSymConst{\ensuremath{=}}\\\;\ensuremath{[}\HOLFreeVar{t}\ensuremath{/}\HOLFreeVar{X}\ensuremath{]}\\\;\ensuremath{(}\HOLFreeVar{a}\HOLSymConst{\ensuremath{\ldotp}}\HOLConst{var}\\\;\HOLFreeVar{X}\\\;\HOLSymConst{\ensuremath{+}}\\\;\HOLFreeVar{b}\HOLSymConst{\ensuremath{\ldotp}}\HOLConst{var}\\\;\HOLFreeVar{X}\\\;\HOLSymConst{\ensuremath{+}}\\\;\HOLFreeVar{c}\HOLSymConst{\ensuremath{\ldotp}}\HOLConst{var}\\\;\HOLFreeVar{Y}\ensuremath{)}}
can be expressed as a \univariate substitution of the original term. This
idea leads to the following definitions:
%
%a5.3 #&#
\begin{alltt}
   \HOLConst{context} \HOLFreeVar{Xs} \HOLFreeVar{E} \HOLTokenDefEquality{} \HOLConst{EVERY} \ensuremath{(}\HOLTokenLambda{}\HOLBoundVar{X}. \HOLConst{CONTEXT} \ensuremath{(}\HOLTokenLambda{}\HOLBoundVar{t}. \ensuremath{[}\HOLBoundVar{t}\ensuremath{/}\HOLBoundVar{X}\ensuremath{]} \HOLFreeVar{E}\ensuremath{)}\ensuremath{)} \HOLFreeVar{Xs}\hfill{[context\_def]}
\end{alltt}

\begin{alltt}
   \HOLConst{weakly\_guarded} \HOLFreeVar{Xs} \HOLFreeVar{E} \HOLTokenDefEquality{} \HOLConst{EVERY} \ensuremath{(}\HOLTokenLambda{}\HOLBoundVar{X}. \HOLConst{WG} \ensuremath{(}\HOLTokenLambda{}\HOLBoundVar{t}. \ensuremath{[}\HOLBoundVar{t}\ensuremath{/}\HOLBoundVar{X}\ensuremath{]} \HOLFreeVar{E}\ensuremath{)}\ensuremath{)} \HOLFreeVar{Xs}\hfill{[weakly\_guarded\_def]}
\end{alltt}
%
The above definitions take an extra list of variables
\HOLinline{\HOLFreeVar{Xs}} and assert the CCS expression
\HOLinline{\HOLFreeVar{E}} with respect to this list. This allows us to
formalise the concepts of contexts and guardedness independently of the
free variables of \HOLinline{\HOLFreeVar{E}}. Then a logical term ``\HOLinline{\HOLConst{weakly\_guarded}\\\;\ensuremath{(}\HOLConst{SET\_TO\_LIST}\\\;\ensuremath{(}\HOLConst{FV}\\\;\HOLFreeVar{E}\ensuremath{)}\ensuremath{)}\\\;\HOLFreeVar{E}}''
can be used to assert the weak guardedness of $E$ with respect to all its
free variables.\footnote{Here \HOLinline{\HOLConst{SET\_TO\_LIST}} converts
a finite set to a list of the same elements. It turns out that we never
need this, because in all unique-solution theorems a list of variables
\HOLinline{\HOLFreeVar{Xs}} is fixed and then all equations are required
to contain free variables in \HOLinline{\HOLFreeVar{Xs}}.} (The
\multivariate guardedness and sequentiality can also be defined \xch{similarly}{similarily},
using their \univariate companions \HOLinline{\HOLConst{SG}} and
\HOLinline{\HOLConst{SEQ}}.)

The most important property of \multivariate contexts is that strong bisimilarity
and other (pre)congruence relations are preserved by them, for instance:
%
%l5.11 #&#
\begin{lemma}[\texttt{STRONG\_EQUIV\_subst\_context}]
If two tuples of processes $\til P$ and $\til Q$ are strongly bisimilar,\footnote{A
HOL term like
\HOLinline{\HOLFreeVar{Ps}\\\;\HOLSymConst{\HOLTokenStrongEQ}\\\;\HOLFreeVar{Qs}}
means that two lists of CCS processes are componentwise bisimilar ($P_i
\sim Q_i$).} then for any context $E$, $E\{\til P/\til X\}$ and
$E\{\til Q/\til X\}$ are strongly bisimilar:
%
%a5.11 #&#
\begin{alltt}
\HOLTokenTurnstile{} \HOLConst{ALL\_DISTINCT} \HOLFreeVar{Xs} \HOLSymConst{\HOLTokenConj{}} \HOLConst{LENGTH} \HOLFreeVar{Ps} \HOLSymConst{\ensuremath{=}} \HOLConst{LENGTH} \HOLFreeVar{Xs} \HOLSymConst{\HOLTokenConj{}} \HOLFreeVar{Ps} \HOLSymConst{\HOLTokenStrongEQ} \HOLFreeVar{Qs} \HOLSymConst{\HOLTokenImp{}}
   \HOLSymConst{\HOLTokenForall{}}\HOLBoundVar{E}. \HOLConst{context} \HOLFreeVar{Xs} \HOLBoundVar{E} \HOLSymConst{\HOLTokenImp{}} \ensuremath{[}\HOLFreeVar{Ps}\ensuremath{/}\HOLFreeVar{Xs}\ensuremath{]} \HOLBoundVar{E} \HOLSymConst{\HOLTokenStrongEQ} \ensuremath{[}\HOLFreeVar{Qs}\ensuremath{/}\HOLFreeVar{Xs}\ensuremath{]} \HOLBoundVar{E}
\end{alltt}
%
\end{lemma}
%
Similar properties also hold if $\sim $ is replaced with rooted bisimilarity
($\rapprox $) and rooted contraction ($\rcontr $). It does not hold for
weak bisimilarity ($\wbvtex $) and the contraction preorder ($
\mcontrBIS $), as they are not (pre)congruence relations.

Another important property of contexts is their composability: if we substitute
some free variables in a context with some other contexts, the resulting
term is still a valid context (with respect to the same set of variables):
%
%a5.3 #&#
\begin{alltt}
\HOLTokenTurnstile{} \HOLConst{ALL\_DISTINCT} \HOLFreeVar{Xs} \HOLSymConst{\HOLTokenConj{}} \HOLConst{context} \HOLFreeVar{Xs} \HOLFreeVar{C} \HOLSymConst{\HOLTokenConj{}} \HOLConst{EVERY} \ensuremath{(}\HOLConst{context} \HOLFreeVar{Xs}\ensuremath{)} \HOLFreeVar{Es} \HOLSymConst{\HOLTokenConj{}}
   \HOLConst{LENGTH} \HOLFreeVar{Es} \HOLSymConst{\ensuremath{=}} \HOLConst{LENGTH} \HOLFreeVar{Xs} \HOLSymConst{\HOLTokenImp{}}
   \HOLConst{context} \HOLFreeVar{Xs} \ensuremath{(}\ensuremath{[}\HOLFreeVar{Es}\ensuremath{/}\HOLFreeVar{Xs}\ensuremath{]} \HOLFreeVar{C}\ensuremath{)}\hfill{[context\_combin]}
\end{alltt}

Not every term within the CCS syntax is a context, as context (or equation)
variables are not allowed to occur within recursion. The converse holds
however: if the set of free variables of a CCS term is disjoint from a
list of variables, then the term is indeed a context with respect to such
list of variables:
%
%a5.3 #&#
\begin{alltt}
\HOLTokenTurnstile{} \HOLConst{DISJOINT} \ensuremath{(}\HOLConst{FV} \HOLFreeVar{E}\ensuremath{)} \ensuremath{(}\HOLConst{set} \HOLFreeVar{Xs}\ensuremath{)} \HOLSymConst{\HOLTokenImp{}} \HOLConst{context} \HOLFreeVar{Xs} \HOLFreeVar{E}\hfill{[disjoint\_imp\_context]}
\end{alltt}
%
On the other hand, for any context (with respect to $\til X$) of the form
$\recu Y E$, the set of free variables of $E$ excluding $Y$ is disjoint
with $\til X$:
%
%a5.3 #&#
\begin{alltt}
\HOLTokenTurnstile{} \HOLConst{context} \HOLFreeVar{Xs} \ensuremath{(}\HOLConst{rec} \HOLFreeVar{Y} \HOLFreeVar{E}\ensuremath{)} \HOLSymConst{\HOLTokenImp{}} \HOLConst{DISJOINT} \ensuremath{(}\HOLConst{FV} \HOLFreeVar{E} \HOLConst{DELETE} \HOLFreeVar{Y}\ensuremath{)} \ensuremath{(}\HOLConst{set} \HOLFreeVar{Xs}\ensuremath{)}\hfill{[context\_rec]}
\end{alltt}
%
In the above lemma we cannot conclude
\HOLinline{\HOLConst{DISJOINT}\\\;\ensuremath{(}\HOLConst{FV}\\\;\HOLFreeVar{E}\ensuremath{)}\\\;\ensuremath{(}\HOLConst{set}\\\;\HOLFreeVar{Xs}\ensuremath{)}},
because $Y$ may appear in both sets. We also cannot conclude
\HOLinline{\HOLConst{context}\\\;\HOLFreeVar{Xs}\\\;\HOLFreeVar{E}}, because
in $\recu Y E$ the bound variable $Y$ may occur freely within another nested
recursion. For instance, variable $Y$ is free in $\recu Y E$, for
$E = \recu Z (a.Y + b.Z) + c.Y$.

For weakly guarded contexts, besides their usual properties as in the
\univariate case (i.e., weak guardedness, \HOLinline{\HOLConst{WG}}), we
also need their composability with respect to~\multivariate contexts: if
we substitute some free variables in a context $C$ with some weakly guarded
contexts, the resulting context is weakly guarded (with respect to the
same set of variables):
%
%a5.3 #&#
\begin{alltt}
\HOLTokenTurnstile{} \HOLConst{ALL\_DISTINCT} \HOLFreeVar{Xs} \HOLSymConst{\HOLTokenConj{}} \HOLConst{context} \HOLFreeVar{Xs} \HOLFreeVar{C} \HOLSymConst{\HOLTokenConj{}} \HOLConst{weakly\_guarded} \HOLFreeVar{Xs} \HOLFreeVar{Es} \HOLSymConst{\HOLTokenConj{}}
   \HOLConst{LENGTH} \HOLFreeVar{Es} \HOLSymConst{\ensuremath{=}} \HOLConst{LENGTH} \HOLFreeVar{Xs} \HOLSymConst{\HOLTokenImp{}}
   \HOLConst{weakly\_guarded} \HOLFreeVar{Xs} \ensuremath{(}\ensuremath{[}\HOLFreeVar{Es}\ensuremath{/}\HOLFreeVar{Xs}\ensuremath{]} \HOLFreeVar{C}\ensuremath{)}\hfill{[weakly\_guarded\_combin]}
\end{alltt}

%s5.4 #&#
\subsection{\Multivariate equations and solutions}
 \label{sec5.4}

With the formal definitions of \multivariate substitution and
\multivariate (weakly guarded) contexts, now we are ready to formally define
\multivariate CCS equations/contractions and their (unique) solutions.
Consider a system of equation $\{X_i = E_i\}_{i\in I}$ (\reftext{Definition~\ref{def:equation}}),
or its expanded form (here we suppose $I = [1,n] \in \mathbb{N}$, i.e.~a
finite list):
%
\begin{equation*}
%
\begin{cases}
&X_1 = E_1[\til X]
\\
&X_2 = E_2[\til X]
\\
& \cdots
\\
&X_n = E_n[\til X]
\end{cases}
%
\end{equation*}
%
Consider its two essential ingredients:
%
\begin{itemize}
%
\item $\til X = (X_1, X_2, \ldots , X_n)$: a list of equation variables;
%
\item $\til E = (E_1, E_2, \ldots , E_n)$: a list of CCS contexts with
possible occurrences of free variables in $\til X$.
%
\end{itemize}
%
The two lists $\til X$ and $\til E$ should have the same length, and the
variables in $\til X$ should be distinct. Furthermore, $\til E$ does not
contain free variables that are not in $\til X$. Finally, for each
$E_i$, the set of its bound variables should be \emph{disjoint} from
$\til X$. These requirements yield the \xch{formal}{fomal} definition of
\multivariate CCS equation:
%
%a5.4 #&#
\begin{alltt}
   \HOLConst{CCS\_equation} \HOLFreeVar{Xs} \HOLFreeVar{Es} \HOLTokenDefEquality{}
     \HOLConst{ALL\_DISTINCT} \HOLFreeVar{Xs} \HOLSymConst{\HOLTokenConj{}} \HOLConst{LENGTH} \HOLFreeVar{Es} \HOLSymConst{\ensuremath{=}} \HOLConst{LENGTH} \HOLFreeVar{Xs} \HOLSymConst{\HOLTokenConj{}}
     \HOLConst{EVERY} \ensuremath{(}\HOLTokenLambda{}\HOLBoundVar{e}. \HOLConst{FV} \HOLBoundVar{e} \HOLSymConst{\HOLTokenSubset{}} \HOLConst{set} \HOLFreeVar{Xs}\ensuremath{)} \HOLFreeVar{Es} \HOLSymConst{\HOLTokenConj{}} \HOLConst{EVERY} \ensuremath{(}\HOLTokenLambda{}\HOLBoundVar{e}. \HOLConst{DISJOINT} \ensuremath{(}\HOLConst{BV} \HOLBoundVar{e}\ensuremath{)} \ensuremath{(}\HOLConst{set} \HOLFreeVar{Xs}\ensuremath{)}\ensuremath{)} \HOLFreeVar{Es}
\end{alltt}

Now consider (formally) what is a solution $\til P$ of CCS equations. First
of all, the definition should be
\xch{parameterized}{parametrized} on a binary CCS relation
$\Rvtex $ such as $\sim $ and $\rcontr $, so that a single definition can
be used to represent solutions of all kinds of CCS equations. Then,
${\til P} \ \Rvtex \  {\til E}\{\til P/\til X\}$ should hold:
%
\begin{equation*}
%
\begin{cases}
&P_1 \ \Rvtex \ E_1\{\til P/\til X\}
\\
&P_2 \ \Rvtex \ E_2\{\til P/\til X\}
\\
& \cdots
\\
&P_n \ \Rvtex \ E_n\{\til P/\til X\}
\end{cases}
%
\end{equation*}
%
Furthermore, each $P_i$ should be a pure process, i.e.~having no free variable.
Finally, the set of bound variables is disjoint with $\til X$. (This disjointness
requirement is optional but makes many proofs easier.) Putting all together,
below is the formal definition of a solution of \multivariate CCS equations\footnote{If
$R$ is a binary relation of CCS processes,
\HOLinline{\HOLConst{LIST\_REL}\\\;\HOLFreeVar{R}} is the same binary relation
but for lists of CCS processes with the same length. Implicitly implies
that the two lists $A$ and $B$ have the same length.}:
%
%a5.4 #&#
\begin{alltt}
   \HOLConst{CCS\_solution} \HOLFreeVar{R} \HOLFreeVar{Xs} \HOLFreeVar{Es} \HOLFreeVar{Ps} \HOLTokenDefEquality{}
     \HOLConst{ALL\_PROC} \HOLFreeVar{Ps} \HOLSymConst{\HOLTokenConj{}} \HOLConst{EVERY} \ensuremath{(}\HOLTokenLambda{}\HOLBoundVar{e}. \HOLConst{DISJOINT} \ensuremath{(}\HOLConst{BV} \HOLBoundVar{e}\ensuremath{)} \ensuremath{(}\HOLConst{set} \HOLFreeVar{Xs}\ensuremath{)}\ensuremath{)} \HOLFreeVar{Ps} \HOLSymConst{\HOLTokenConj{}}
     \HOLConst{LIST\_REL} \HOLFreeVar{R} \HOLFreeVar{Ps} \ensuremath{(}\HOLConst{MAP} \ensuremath{[}\HOLFreeVar{Ps}\ensuremath{/}\HOLFreeVar{Xs}\ensuremath{]} \HOLFreeVar{Es}\ensuremath{)}
\end{alltt}
%
Note that the two logical terms ``\HOLinline{\HOLConst{CCS\_solution}\\\;\HOLFreeVar{R}\\\;\HOLFreeVar{Xs}\\\;\HOLFreeVar{Es}\\\;\HOLFreeVar{Ps}}''
and ``\HOLinline{\HOLFreeVar{Ps}\\\;\HOLSymConst{\HOLTokenIn{}}\\\;\HOLConst{CCS\_solution}\\\;\HOLFreeVar{R}\\\;\HOLFreeVar{Xs}\\\;\HOLFreeVar{Es}}''
in HOL have the same meaning (they are equivalent). The latter form suggests
that ``\HOLinline{\HOLConst{CCS\_solution}\\\;\HOLFreeVar{R}\\\;\HOLFreeVar{Xs}\\\;\HOLFreeVar{Es}}''
is actually a set containing all solutions of ``\HOLinline{\HOLConst{CCS\_equation}\\\;\HOLFreeVar{Xs}\\\;\HOLFreeVar{Es}}''.
Then the unique-solution theorems can be understood thus: any two elements
in the solution set are bisimilar.

%s5.5 #&#
\subsection{Unique solution of equations/contractions (the \multivariate version)}
 \label{sec5.5}

With all above machineries of \multivariate contexts and substitutions,
the \multivariate case of \reftext{Lemma~\ref{lem:milner313}} is formalised below
(\texttt{strong\_unique\_solution\_lemma}):
%
%a5.5 #&#
\begin{alltt}
\HOLTokenTurnstile{} \HOLConst{weakly\_guarded} \HOLFreeVar{Xs} \HOLFreeVar{E} \HOLSymConst{\HOLTokenConj{}} \HOLConst{FV} \HOLFreeVar{E} \HOLSymConst{\HOLTokenSubset{}} \HOLConst{set} \HOLFreeVar{Xs} \HOLSymConst{\HOLTokenConj{}} \HOLConst{DISJOINT} \ensuremath{(}\HOLConst{BV} \HOLFreeVar{E}\ensuremath{)} \ensuremath{(}\HOLConst{set} \HOLFreeVar{Xs}\ensuremath{)} \HOLSymConst{\HOLTokenImp{}}
   \HOLSymConst{\HOLTokenForall{}}\HOLBoundVar{Ps}.
       \HOLConst{LENGTH} \HOLBoundVar{Ps} \HOLSymConst{\ensuremath{=}} \HOLConst{LENGTH} \HOLFreeVar{Xs} \HOLSymConst{\HOLTokenImp{}}
       \HOLSymConst{\HOLTokenForall{}}\HOLBoundVar{u} \ensuremath{\HOLBoundVar{P}\sp{\prime}}.
           \ensuremath{[}\HOLBoundVar{Ps}\ensuremath{/}\HOLFreeVar{Xs}\ensuremath{]} \HOLFreeVar{E} \HOLTokenTransBegin\HOLBoundVar{u}\HOLTokenTransEnd \ensuremath{\HOLBoundVar{P}\sp{\prime}} \HOLSymConst{\HOLTokenImp{}}
           \HOLSymConst{\HOLTokenExists{}}\ensuremath{\HOLBoundVar{E}\sp{\prime}}.
               \HOLConst{context} \HOLFreeVar{Xs} \ensuremath{\HOLBoundVar{E}\sp{\prime}} \HOLSymConst{\HOLTokenConj{}} \HOLConst{FV} \ensuremath{\HOLBoundVar{E}\sp{\prime}} \HOLSymConst{\HOLTokenSubset{}} \HOLConst{set} \HOLFreeVar{Xs} \HOLSymConst{\HOLTokenConj{}} \HOLConst{DISJOINT} \ensuremath{(}\HOLConst{BV} \ensuremath{\HOLBoundVar{E}\sp{\prime}}\ensuremath{)} \ensuremath{(}\HOLConst{set} \HOLFreeVar{Xs}\ensuremath{)} \HOLSymConst{\HOLTokenConj{}}
               \ensuremath{\HOLBoundVar{P}\sp{\prime}} \HOLSymConst{\ensuremath{=}} \ensuremath{[}\HOLBoundVar{Ps}\ensuremath{/}\HOLFreeVar{Xs}\ensuremath{]} \ensuremath{\HOLBoundVar{E}\sp{\prime}} \HOLSymConst{\HOLTokenConj{}}
               \HOLSymConst{\HOLTokenForall{}}\HOLBoundVar{Qs}. \HOLConst{LENGTH} \HOLBoundVar{Qs} \HOLSymConst{\ensuremath{=}} \HOLConst{LENGTH} \HOLFreeVar{Xs} \HOLSymConst{\HOLTokenImp{}} \ensuremath{[}\HOLBoundVar{Qs}\ensuremath{/}\HOLFreeVar{Xs}\ensuremath{]} \HOLFreeVar{E} \HOLTokenTransBegin\HOLBoundVar{u}\HOLTokenTransEnd \ensuremath{[}\HOLBoundVar{Qs}\ensuremath{/}\HOLFreeVar{Xs}\ensuremath{]} \ensuremath{\HOLBoundVar{E}\sp{\prime}}
\end{alltt}

The \multivariate version of \reftext{Theorem~\ref{t:Mil89s1}} is formalised thus:
%
%a5.5 #&#
\begin{alltt}
\HOLTokenTurnstile{} \HOLConst{CCS\_equation} \HOLFreeVar{Xs} \HOLFreeVar{Es} \HOLSymConst{\HOLTokenConj{}} \HOLConst{weakly\_guarded} \HOLFreeVar{Xs} \HOLFreeVar{Es} \HOLSymConst{\HOLTokenConj{}}
   \HOLFreeVar{Ps} \HOLSymConst{\HOLTokenIn{}} \HOLConst{CCS\_solution} \HOLConst{STRONG\_EQUIV} \HOLFreeVar{Xs} \HOLFreeVar{Es} \HOLSymConst{\HOLTokenConj{}}
   \HOLFreeVar{Qs} \HOLSymConst{\HOLTokenIn{}} \HOLConst{CCS\_solution} \HOLConst{STRONG\_EQUIV} \HOLFreeVar{Xs} \HOLFreeVar{Es} \HOLSymConst{\HOLTokenImp{}}
   \HOLFreeVar{Ps} \HOLSymConst{\HOLTokenStrongEQ} \HOLFreeVar{Qs}\hfill{[strong\_unique\_solution\_thm]}
\end{alltt}

Now the \multivariate version of \reftext{Theorem~\ref{t:rcontraBisimulationU}}:
%
%a5.5 #&#
\begin{alltt}
\HOLTokenTurnstile{} \HOLConst{CCS\_equation} \HOLFreeVar{Xs} \HOLFreeVar{Es} \HOLSymConst{\HOLTokenConj{}} \HOLConst{weakly\_guarded} \HOLFreeVar{Xs} \HOLFreeVar{Es} \HOLSymConst{\HOLTokenConj{}}
   \HOLFreeVar{Ps} \HOLSymConst{\HOLTokenIn{}} \HOLConst{CCS\_solution} \HOLConst{OBS\_contracts} \HOLFreeVar{Xs} \HOLFreeVar{Es} \HOLSymConst{\HOLTokenConj{}}
   \HOLFreeVar{Qs} \HOLSymConst{\HOLTokenIn{}} \HOLConst{CCS\_solution} \HOLConst{OBS\_contracts} \HOLFreeVar{Xs} \HOLFreeVar{Es} \HOLSymConst{\HOLTokenImp{}}
   \HOLFreeVar{Ps} \HOLSymConst{\HOLTokenObsCongr} \HOLFreeVar{Qs}\hfill{[unique\_solution\_of\_rooted\_contractions]}
\end{alltt}
%
In summary, while each step related to \multivariate substitutions is more
difficult than that for the \univariate case, the structure of the overall
proofs is the same; various proof steps can even be copied from the
\univariate case.

%s6 #&#
\section{Related work on formalisation}
%%LEAP%%%\label{sec6}
 \label{s:rel}

Monica Nesi did the first CCS formalisations for both pure and value-passing
CCS \cite{Nesi:1992ve,Nesi:2017wo} using early versions of the HOL theorem
prover.\footnote{Part of this work can now be found at
\url{https://github.com/binghe/HOL-CCS/tree/master/CCS-Nesi}.} Her main
focus was on implementing decision procedures (as a ML program, e.g.~\cite{cleaveland1993concurrency})
for automatically proving bisimilarities of CCS processes. Her work has
been a basis for ours~\cite{Tian:2017wrba}. However, the differences are
substantial, especially in the way of defining bisimilarities. We greatly
benefited from new features and standard libraries in recent versions of
HOL4, and our formalisation has covered a larger spectrum of the (pure)
CCS theory.

Bengtson, Parrow and Weber did a substantial formalisation work on CCS,
$\pi $-calculi and $\psi $-calculi using Isabelle/HOL and its nominal logic,
with the main focus on the handling of name binders
\cite{bengtson2007completeness,parrow2009formalising}. More details can
be found in Bengtson's PhD thesis~\cite{bengtson2010formalising}. For CCS,
he has \xch{formalised}{formalized} basic properties for strong/weak equivalence (congruence,
basic algebraic laws); the CCS syntax does not have constants or recursion,
using instead replication. Other formalisations in this area include the
early work of T.F.~Melham \cite{melham1994mechanized} and O.A.~Mohamed
\cite{mohamed1995mechanizing} in HOL, Compton
\cite{compton2005embedding} in Isabelle/HOL, Solange\footnote{\url{https://github.com/coq-contribs/ccs}.}
in Coq and Chaudhuri et al.\;\cite{chaudhuri2015lightweight} in Abella,
the latter focuses on ``bisimulation up-to'' techniques (for strong bisimilarity)
for CCS and $\pi $-calculus. Damien Pous \cite{pous2007new} also formalised
up-to techniques and some CCS examples in Coq. Formalisations less related
to ours include Kahsai and Miculan \cite{kahsai2008implementing} for the
spi calculus in Isabelle/HOL, and Hirschkoff
\cite{hirschkoff1997full} for the $\pi $-calculus in Coq.

%s7 #&#
\section{Conclusions and future work}
%%LEAP%%%\label{sec7}
 \label{s:concl}

In this paper, besides introducing rooted contraction and its unique solution
theorems, we have carried out a comprehensive formalisation of the theory
of CCS in the HOL4 theorem prover. In particular, the formalisation supports
four methods for establishing (strong and weak) bisimilarities:
%
\begin{enumerate}
%
\item constructing a bisimulation (the standard bisimulation proof method);
%
\item constructing a ``bisimulation up-to'';
%
\item employing algebraic laws;
%
\item constructing a system of equations or contractions (i.e., the `unique-solution'
method)
%
\end{enumerate}

The formalisation has focused on the theory of unique solution of equations
and contractions, both in the \univariate and in the \multivariate cases.
It has allowed us to further develop the theory, notably the basic properties
of rooted contraction, and the unique solution theorem for it with respect
to rooted bisimilarity. The formalisation brings up and exploits similarities
between results and proofs for different equivalences and preorders. Indeed
we have considered several ``unique-solution'' results (for various equivalences
and preorders); they share many parts of the proofs, but present a few
delicate and subtle differences in a few points. In a paper-and-pencil
proof, checking all details would be long and error-prone, especially in
cases where the proofs are similar to each other or when there are long
case analyses to be carried out. We believe that \xch{our}{the} formalisation of all
definitions and theorems \xch{is}{are} easy to read and to understand, as they are
close to their original statement in textbooks.

For some future work, formalising other equivalences and preorders could
also be considered, notably trace-based equivalences, as well as more refined
process calculi such as value-passing CCS. A different and possibly challenging
research line is the formalisation of a different approach~\cite{DurierHS17,DurierHS18}
to unique solutions, in which the use of contraction is replaced by semantic
conditions on process transitions such as divergence.


%%% PASTABA: Pagal zurnalo ypatumus, reikalinga
%%%          zemiau esanti aplinka {conflict}.
%%%          S5 etapui koduoti NEREIKIA.

\begin{conflict}
The authors declare that they have no known competing financial interests or personal
relationships that could have appeared to influence the work reported in this paper.
\end{conflict}

\begin{acks}
We have benefitted from suggestions and comments from the anonymous reviewers
and several people from the HOL community, including (in alphabetical order)
Robert Beers, Jeremy Dawson, Ramana Kumar, Michael Norrish, Konrad Slind,
and Thomas T\"{u}rk. Sangiorgi has been supported by \gsponsor[id=GS4]{MIUR-PRIN} project
`Analysis of Program Analyses' (ASPRA, ID: \texttt{\gnumber[refid=GS4]{201784YSZ5\_004}}), and
by the \gsponsor[id=GS2,sponsor-id=501100000781,country=European Union]{European Research Council} (ERC) Grant \gnumber[refid=GS2]{DLV-818616 DIAPASoN}. The paper
was written in memory of Michael J.~C.~Gordon (1948--2017), the creator
of the HOL theorem prover.
\end{acks}

%\begin{appm}
%\def\the...{}
%\reset{}{}
%\appendix{}
%\appendix*{}
%\end{appm}%
%spell_to        ********** End of text entry *****************
%
%
\begin{backmatter}%
% structpyb loaded by lgiriuniene, 2020-07-27 15:55:25
\begin{thebibliography}{}
% pybtex-2.60. Style name=elsv1, version=1.53, label_style=original, sorting_style=none, cfg=None, language=None.


%b1 ###
\bibitem{EPTCS276.10}
\begin{bsubitem}
\begin{bcontribution}%[language=fr]%de,it,pl,ru
\bauthor{\fnm{C.} \snm{Tian}}
\bauthor{\fnm{D.} \snm{Sangiorgi}}
\btitle{Unique solutions of contractions, CCS, and their HOL formalisation}
\end{bcontribution}
\begin{bhost}
\begin{beditedbook}
\beditors{%
\beditor{\fnm{J.A.} \snm{P\'{e}rez}}
\beditor{\fnm{S.} \snm{Tini}}}
\btitle{Proceedings Combined 25th International Workshop on Expressiveness
in Concurrency {and 15th Workshop on} Structural Operational Semantics}
\bconference{Beijing, China, September 3, 2018}
\bbookseries{
\bseries{\btitle{Electronic Proceedings in Theoretical Computer Science}\bvolumeno{276}}}
\bdate{2018}
\bpublisher{\bname{Open Publishing Association}}
\end{beditedbook}
\bpages{\bfirstpage{122}\blastpage{139}}
\bdoi{10.4204/EPTCS.276.10}
\end{bhost}
\end{bsubitem}
%
\OrigBibText
C.~Tian, D.~Sangiorgi, {Unique Solutions of Contractions, CCS, and their
HOL Formalisation}, in: J.~A. P\'erez, S.~Tini (Eds.), \textrm{Proceedings
Combined 25th International Workshop on} Expressiveness in Concurrency
\textrm{and 15th Workshop on} Structural Operational Semantics, \textrm{Beijing,
China, September 3, 2018}, Vol. 276 of Electronic Proceedings in Theoretical
Computer Science, Open Publishing Association, 2018, pp. 122--139.
10.4204/EPTCS.276.10.
\endOrigBibText
\bptok{structpyb}%
\endbibitem

%b2 ###
\bibitem{Mil89}
\begin{bsubitem}
\begin{bcontribution}%[language=fr]%de,it,pl,ru
\bauthor{\fnm{R.} \snm{Milner}}
\btitle{Communication and Concurrency}
\end{bcontribution}
\begin{bhost}
\begin{bbook}
\bbookseries{
\bseries{\btitle{{PHI} Series in Computer Science}}}
\bdate{1989}
\bpublisher{\bname{Prentice-Hall}}
\end{bbook}
\end{bhost}
\end{bsubitem}
%
\OrigBibText
R.~Milner, Communication and concurrency, {PHI} Series in computer science,
Prentice-Hall, 1989.
\endOrigBibText
\bptok{structpyb}%
\endbibitem

%b3 ###
\bibitem{BaeBOOK}
\begin{bsubitem}
\begin{bcontribution}%[language=fr]%de,it,pl,ru
\bauthor{\fnm{J.C.M.} \snm{Baeten}}
\bauthor{\fnm{T.} \snm{Basten}}
\bauthor{\fnm{M.A.} \snm{Reniers}}
\btitle{Process Algebra: Equational Theories of Communicating Processes}
\end{bcontribution}
\begin{bhost}
\begin{bbook}
\bdate{2010}
\bpublisher{\bname{Cambridge University Press}}
\end{bbook}
\bdoi{10.1017/CBO9781139195003}
\end{bhost}
\end{bsubitem}
%
\OrigBibText
J.~C.~M. Baeten, T.~Basten, M.~A. Reniers, Process Algebra: Equational
Theories of Communicating Processes, Cambridge University Press, 2010.
10.1017/CBO9781139195003.
\endOrigBibText
\bptok{structpyb}%
\endbibitem

%b4 ###
\bibitem{theoryAndPractice}
\begin{bsubitem}
\begin{bcontribution}%[language=fr]%de,it,pl,ru
\bauthor{\fnm{A.W.} \snm{Roscoe}}
\btitle{The Theory and Practice of Concurrency}
\end{bcontribution}
\begin{bhost}
\begin{bbook}%%[class=report]
\bdate{1998}
\bpublisher{\bname{Prentice Hall}}
\end{bbook}
\end{bhost}
\begin{bhost}
\begin{behost}
\binterref[locator-type=url]{http://www.cs.ox.ac.uk/people/bill.roscoe/publications/68b.pdf}
\end{behost}
\end{bhost}
\end{bsubitem}
%
\OrigBibText
A.~W. Roscoe,
{The
theory and practice of concurrency}, {Prentice Hall}, 1998.
\url{http://www.cs.ox.ac.uk/people/bill.roscoe/publications/68b.pdf}
\endOrigBibText
\bptok{structpyb}%
\endbibitem

%b5 ###
\bibitem{RosUnder10}
\begin{bsubitem}
\begin{bcontribution}%[language=fr]%de,it,pl,ru
\bauthor{\fnm{A.W.} \snm{Roscoe}}
\btitle{Understanding Concurrent Systems}
\end{bcontribution}
\begin{bhost}
\begin{bbook}
\bdate{2010}
\bpublisher{\bname{Springer}}
\end{bbook}
\bdoi{10.1007/978-1-84882-258-0}
\end{bhost}
\end{bsubitem}
%
\OrigBibText
A.~W. Roscoe, Understanding Concurrent Systems, Springer, 2010.
10.1007/978-1-84882-258-0.
\endOrigBibText
\bptok{structpyb}%
\endbibitem

%b6 ###
\bibitem{sangiorgi2015equations}
\begin{bsubitem}
\begin{bcontribution}%[language=fr]%de,it,pl,ru
\bauthor{\fnm{D.} \snm{Sangiorgi}}
\btitle{Equations, contractions, and unique solutions}
\end{bcontribution}
\begin{bhost}
\begin{bissue}
\bseries{\btitle{SIGPLAN Not.} \bvolumeno{50}}
\bissueno{1}
\bdate{2015}
\end{bissue}
\bpages{\bfirstpage{421}\blastpage{432}}
\bdoi{10.1145/2775051.2676965}
\end{bhost}
\end{bsubitem}
%
\OrigBibText
D.~Sangiorgi, Equations, contractions, and unique solutions, SIGPLAN Not.
50~(1) (2015) 421--432. 10.1145/2775051.2676965.
\endOrigBibText
\bptok{structpyb}%
\endbibitem

%b7 ###
\bibitem{sangiorgi2017equations}
\begin{bsubitem}
\begin{bcontribution}%[language=fr]%de,it,pl,ru
\bauthor{\fnm{D.} \snm{Sangiorgi}}
\btitle{Equations, contractions, and unique solutions}
\end{bcontribution}
\begin{bhost}
\begin{bissue}
\bseries{\btitle{ACM Trans. Comput. Log.} \bvolumeno{18}}
\bissueno{1}
\bdate{2017}
\end{bissue}
\bpages{\bfirstpage{4:1}\blastpage{4:30}}
\bdoi{10.1145/2971339}
\end{bhost}
\end{bsubitem}
%
\OrigBibText
D.~Sangiorgi, Equations, contractions, and unique solutions, ACM Trans.
Comput. Logic 18~(1) (2017) 4:1--4:30. 10.1145/2971339.
\endOrigBibText
\bptok{structpyb}%
\endbibitem

%b8 ###
\bibitem{van2005characterisation}
\begin{bsubitem}
\begin{bcontribution}%[language=fr]%de,it,pl,ru
\bauthor{\fnm{R.J.} \snm{van Glabbeek}}
\btitle{A characterisation of weak bisimulation congruence}
\end{bcontribution}
\begin{bhost}
\begin{beditedbook}
\btitle{Processes, Terms and Cycles: Steps on the Road to Infinity}
\bdate{2005}
\bpublisher{\bname{Springer}}
\end{beditedbook}
\bpages{\bfirstpage{26}\blastpage{39}}
\bdoi{10.1007/11601548\_4}
\end{bhost}
\end{bsubitem}
%
\OrigBibText
R.~J. van Glabbeek, A characterisation of weak bisimulation congruence,
in: Processes, Terms and Cycles: Steps on the Road to Infinity, Springer,
2005, pp. 26--39. 10.1007/11601548\_4.
\endOrigBibText
\bptok{structpyb}%
\endbibitem

%b9 ###
\bibitem{Gorrieri:2015jt}
\begin{bsubitem}
\begin{bcontribution}%[language=fr]%de,it,pl,ru
\bauthor{\fnm{R.} \snm{Gorrieri}}
\bauthor{\fnm{C.} \snm{Versari}}
\btitle{Introduction to Concurrency Theory}
\end{bcontribution}
\begin{bhost}
\begin{bbook}
\bbookseries{\bseries{\btitle{Transition Systems and CCS}}}
\bdate{2015}
\bpublisher{\bname{Springer}\blocation{Cham}}
\end{bbook}
\bdoi{10.1007/978-3-319-21491-7}
\end{bhost}
\end{bsubitem}
%
\OrigBibText
R.~Gorrieri, C.~Versari, {Introduction to Concurrency Theory}, Transition
Systems and CCS, Springer, Cham, 2015. 10.1007/978-3-319-21491-7.
\endOrigBibText
\bptok{structpyb}%
\endbibitem

%b10 ###
\bibitem{milner1990operational}
\begin{bsubitem}
\begin{bcontribution}%[language=fr]%de,it,pl,ru
\bauthor{\fnm{R.} \snm{Milner}}
\btitle{Operational and algebraic semantics of concurrent processes}
\end{bcontribution}
\begin{bhost}
\begin{beditedbook}
\btitle{Handbook of Theoretical Comput. Sci.}
\bdate{1990}
\end{beditedbook}
\end{bhost}
\end{bsubitem}
%
\OrigBibText
R.~Milner, Operational and algebraic semantics of concurrent processes,
Handbook of Theoretical Comput. Sci. (1990).
\endOrigBibText
\bptok{structpyb}%
\endbibitem

%b11 ###
\bibitem{Sangiorgi:2011ut}
\begin{bsubitem}
\begin{bcontribution}%[language=fr]%de,it,pl,ru
\bauthor{\fnm{D.} \snm{Sangiorgi}}
\btitle{Introduction to Bisimulation and Coinduction}
\end{bcontribution}
\begin{bhost}
\begin{bbook}
\bdate{2011}
\bpublisher{\bname{Cambridge University Press}}
\end{bbook}
\bdoi{10.1017/CBO9780511777110}
\end{bhost}
\end{bsubitem}
%
\OrigBibText
D.~Sangiorgi, Introduction to Bisimulation and Coinduction, Cambridge University
Press, 2011. 10.1017/CBO9780511777110.
\endOrigBibText
\bptok{structpyb}%
\endbibitem

%b12 ###
\bibitem{arun1992efficiency}
\begin{bsubitem}
\begin{bcontribution}%[language=fr]%de,it,pl,ru
\bauthor{\fnm{S.} \snm{Arun-Kumar}}
\bauthor{\fnm{M.} \snm{Hennessy}}
\btitle{An efficiency preorder for processes}
\end{bcontribution}
\begin{bhost}
\begin{bissue}
\bseries{\btitle{Acta Inform.} \bvolumeno{29}}
\bissueno{8}
\bdate{1992}
\end{bissue}
\bpages{\bfirstpage{737}\blastpage{760}}
\bdoi{10.1007/BF01191894}
\end{bhost}
\end{bsubitem}
%
\OrigBibText
S.~Arun-Kumar, M.~Hennessy, An efficiency preorder for processes, Acta
Informatica 29~(8) (1992) 737--760. 10.1007/BF01191894.
\endOrigBibText
\bptok{structpyb}%
\endbibitem

%b13 ###
\bibitem{Melham:1993vl}
\begin{bsubitem}
\begin{bcontribution}%[language=fr]%de,it,pl,ru
\bauthor{\fnm{M.J.C.} \snm{Gordon}}
\bauthor{\fnm{T.F.} \snm{Melham}}
\btitle{Introduction to {HOL}: A Theorem Proving Environment for Higher Order Logic}
\end{bcontribution}
\begin{bhost}
\begin{bbook}
\bdate{1993}
\bpublisher{\bname{Cambrige University Press}\blocation{New York, NY}}
\end{bbook}
\end{bhost}
\end{bsubitem}
%
\OrigBibText
M.~J.~C. Gordon, T.~F. Melham, Introduction to {HOL}: A Theorem Proving
Environment for Higher Order Logic, Cambrige University Press, New York,
NY, 1993.
\endOrigBibText
\bptok{structpyb}%
\endbibitem

%b14 ###
\bibitem{slind2008brief}
\begin{bsubitem}
\begin{bcontribution}%[language=fr]%de,it,pl,ru
\bauthor{\fnm{K.} \snm{Slind}}
\bauthor{\fnm{M.} \snm{Norrish}}
\btitle{A brief overview of {HOL4}}
\end{bcontribution}
\begin{bhost}
\begin{beditedbook}
\btitle{International Conference on Theorem Proving in Higher Order Logics}
\bdate{2008}
\bpublisher{\bname{Springer}}
\end{beditedbook}
\bpages{\bfirstpage{28}\blastpage{32}}
\bdoi{10.1007/978-3-540-71067-7\_6}
\end{bhost}
\begin{bhost}
\begin{behost}
\binterref[locator-type=url]{http://ts.data61.csiro.au/publications/nicta\_full\_text/1482.pdf}
\end{behost}
\end{bhost}
\end{bsubitem}
%
\OrigBibText
K.~Slind, M.~Norrish,
{A
brief overview of {HOL4}}, in: International Conference on Theorem Proving
in Higher Order Logics, Springer, 2008, pp. 28--32. 10.1007/978-3-540-71067-7\_6.
\url{http://ts.data61.csiro.au/publications/nicta\_full\_text/1482.pdf}
\endOrigBibText
\bptok{structpyb}%
\endbibitem

%b15 ###
\bibitem{Nesi:1992ve}
\begin{bsubitem}
\begin{bcontribution}%[language=fr]%de,it,pl,ru
\bauthor{\fnm{M.} \snm{Nesi}}
\btitle{A formalization of the process algebra CCS in high order logic}
\end{bcontribution}
\bcomment{Tech. Rep. UCAM-CL-TR-278}\prnsep{,\ }
\begin{bhost}
\begin{bbook}[class=report]
\bdate{1992}
\bpublisher{\bname{University of Cambridge, Computer Laboratory}}
\end{bbook}
\end{bhost}
\begin{bhost}
\begin{behost}
\binterref[locator-type=url]{http://www.cl.cam.ac.uk/techreports/UCAM-CL-TR-278.pdf}
\end{behost}
\end{bhost}
\end{bsubitem}
%
\OrigBibText
M.~Nesi,
{{A
formalization of the process algebra CCS in high order logic}}, Tech. Rep.
UCAM-CL-TR-278, University of Cambridge, Computer Laboratory (1992).
\url{http://www.cl.cam.ac.uk/techreports/UCAM-CL-TR-278.pdf}
\endOrigBibText
\bptok{structpyb}%
\endbibitem

%b16 ###
\bibitem{hollogic}
\begin{bsubitem}
\begin{bcontribution}%[language=fr]%de,it,pl,ru
\bauthor{\snm{HOL4 contributors}}
\btitle{The HOL System LOGIC (Kananaskis-13 Release)}
\end{bcontribution}
\begin{bhost}
\begin{bbook}%%[class=report]
\bdate{Aug. 2019}
\end{bbook}
\end{bhost}
\begin{bhost}
\begin{behost}
\binterref[locator-type=url]{http://sourceforge.net/projects/hol/files/hol/kananaskis-13/kananaskis-13-logic.pdf}
\end{behost}
\end{bhost}
\end{bsubitem}
%
\OrigBibText
{HOL4 contributors},
{{The
HOL System LOGIC (Kananaskis-13 release)}} (Aug. 2019).
\url{http://sourceforge.net/projects/hol/files/hol/kananaskis-13/kananaskis-13-logic.pdf}
\endOrigBibText
\bptok{structpyb}%
\endbibitem

%b17 ###
\bibitem{gordon1979edinburgh}
\begin{bsubitem}
\begin{bcontribution}%[language=fr]%de,it,pl,ru
\bauthor{\fnm{M.J.C.} \snm{Gordon}}
\bauthor{\fnm{A.J.} \snm{Milner}}
\bauthor{\fnm{C.P.} \snm{Wadsworth}}
\btitle{Edinburgh {LCF}: A Mechanised Logic of Computation}
\end{bcontribution}
\begin{bhost}
\begin{bbook}
\bbookseries{
\bseries{\btitle{Lecture Notes in Computer Science}\bvolumeno{78}}}
\bdate{1979}
\bpublisher{\bname{Springer}\blocation{Berlin, Heidelberg}}
\end{bbook}
\bdoi{10.1007/3-540-09724-4}
\end{bhost}
\end{bsubitem}
%
\OrigBibText
M.~J.~C. Gordon, A.~J. Milner, C.~P. Wadsworth, Edinburgh {LCF}: A Mechanised
Logic of Computation, Vol.~78 of Lecture Notes in Computer Science, Springer,
Berlin Heidelberg, 1979. 10.1007/3-540-09724-4.
\endOrigBibText
\bptok{structpyb}%
\endbibitem

%b18 ###
\bibitem{milner1972logic}
\begin{bsubitem}
\begin{bcontribution}%[language=fr]%de,it,pl,ru
\bauthor{\fnm{R.} \snm{Milner}}
\btitle{Logic for computable functions: description of a machine implementation}
\end{bcontribution}
\bcomment{Tech. Rep}\prnsep{,\ }
\begin{bhost}
\begin{bbook}[class=report]
\bdate{1972}
\bpublisher{\bname{Stanford Univ., Dept. of Computer Science}}
\end{bbook}
\end{bhost}
\begin{bhost}
\begin{behost}
\binterref[locator-type=url]{http://www.dtic.mil/dtic/tr/fulltext/u2/785072.pdf}
\end{behost}
\end{bhost}
\end{bsubitem}
%
\OrigBibText
R.~Milner,
{Logic for
computable functions: description of a machine implementation}, Tech. rep.,
Stanford Univ., Dept. of Computer Science (1972).
 \url{http://www.dtic.mil/dtic/tr/fulltext/u2/785072.pdf}
\endOrigBibText
\bptok{structpyb}%
\endbibitem

%b19 ###
\bibitem{church1940formulation}
\begin{bsubitem}
\begin{bcontribution}%[language=fr]%de,it,pl,ru
\bauthor{\fnm{A.} \snm{Church}}
\btitle{A formulation of the simple theory of types}
\end{bcontribution}
\begin{bhost}
\begin{bissue}
\bseries{\btitle{J. Symb. Log.} \bvolumeno{5}}
\bissueno{2}
\bdate{1940}
\end{bissue}
\bpages{\bfirstpage{56}\blastpage{68}}
\bdoi{10.2307/2266170}
\end{bhost}
\end{bsubitem}
%
\OrigBibText
A.~Church, A formulation of the simple theory of types, The journal of
symbolic logic 5~(2) (1940) 56--68. 10.2307/2266170.
\endOrigBibText
\bptok{structpyb}%
\endbibitem

%b20 ###
\bibitem{Melham:1989dk}
\begin{bsubitem}
\begin{bcontribution}%[language=fr]%de,it,pl,ru
\bauthor{\fnm{T.F.} \snm{Melham}}
\btitle{Automating recursive type definitions in higher order logic}
\end{bcontribution}
\begin{bhost}
\begin{beditedbook}
\btitle{Current Trends in Hardware Verification and Automated Theorem Proving}
\bdate{1989}
\bpublisher{\bname{Springer}\blocation{New York, NY}}
\end{beditedbook}
\bpages{\bfirstpage{341}\blastpage{386}}
\bdoi{10.1007/978-1-4612-3658-0\_9}
\end{bhost}
\end{bsubitem}
%
\OrigBibText
T.~F. Melham, {Automating Recursive Type Definitions in Higher Order Logic},
in: Current Trends in Hardware Verification and Automated Theorem Proving,
Springer, New York, NY, 1989, pp. 341--386. 10.1007/978-1-4612-3658-0\_9.
\endOrigBibText
\bptok{structpyb}%
\endbibitem

%b21 ###
\bibitem{Melham:1991}
\begin{bsubitem}
\begin{bcontribution}%[language=fr]%de,it,pl,ru
\bauthor{\fnm{T.F.} \snm{Melham}}
\btitle{A package for inductive relation definitions in HOL}
\end{bcontribution}
\begin{bhost}
\begin{beditedbook}
\btitle{1991, International Workshop on the HOL Theorem Proving System and Its Applications}
\bdate{1991}
\bpublisher{\bname{IEEE Computer Society Press}\blocation{Davis, CA, USA}}
\end{beditedbook}
\bpages{\bfirstpage{350}\blastpage{357}}
\bdoi{10.1109/HOL.1991.596299}
\end{bhost}
\end{bsubitem}
%
\OrigBibText
T.~F. Melham, {A Package For Inductive Relation Definitions In HOL}, in:
1991., International Workshop on the HOL Theorem Proving System and Its
Applications, IEEE Computer Society Press, Davis, CA, USA, 1991, pp. 350--357.
10.1109/HOL.1991.596299.
\endOrigBibText
\bptok{structpyb}%
\endbibitem

%b22 ###
\bibitem{holdesc}
\begin{bsubitem}
\begin{bcontribution}%[language=fr]%de,it,pl,ru
\bauthor{\snm{HOL4 contributors}}
\btitle{The HOL System DESCRIPTION (Kananaskis-13 Release)}
\end{bcontribution}
\begin{bhost}
\begin{bbook}%%[class=report]
\bdate{Aug. 2019}
\end{bbook}
\end{bhost}
\begin{bhost}
\begin{behost}
\binterref[locator-type=url]{http://sourceforge.net/projects/hol/files/hol/kananaskis-13/kananaskis-13-description.pdf}
\end{behost}
\end{bhost}
\end{bsubitem}
%
\OrigBibText
{HOL4 contributors},
{{The
HOL System DESCRIPTION (Kananaskis-13 release)}} (Aug. 2019).
\url{http://sourceforge.net/projects/hol/files/hol/kananaskis-13/kananaskis-13-description.pdf}
\endOrigBibText
\bptok{structpyb}%
\endbibitem

%b23 ###
\bibitem{PousS19}
\begin{bsubitem}
\begin{bcontribution}%[language=fr]%de,it,pl,ru
\bauthor{\fnm{D.} \snm{Pous}}
\bauthor{\fnm{D.} \snm{Sangiorgi}}
\btitle{Bisimulation and coinduction enhancements: a historical perspective}
\end{bcontribution}
\begin{bhost}
\begin{bissue}
\bseries{\btitle{Form. Asp. Comput.} \bvolumeno{31}}
\bissueno{6}
\bdate{2019}
\end{bissue}
\bpages{\bfirstpage{733}\blastpage{749}}
\bdoi{10.1007/s00165-019-00497-w}
\end{bhost}
\end{bsubitem}
%
\OrigBibText
D.~Pous, D.~Sangiorgi, Bisimulation and coinduction enhancements: {A} historical
perspective, Formal Asp. Comput. 31~(6) (2019) 733--749. 10.1007/s00165-019-00497-w.
\endOrigBibText
\bptok{structpyb}%
\endbibitem

%b24 ###
\bibitem{PoSa2019}
\begin{bsubitem}
\begin{bcontribution}%[language=fr]%de,it,pl,ru
\bauthor{\fnm{D.} \snm{Pous}}
\bauthor{\fnm{D.} \snm{Sangiorgi}}
\btitle{Bisimulation and coinduction enhancements: a historical perspective}
\end{bcontribution}
\begin{bhost}
\begin{bissue}
\bseries{\btitle{Form. Asp. Comput.}}
\bdate{2019}
\end{bissue}
\bdoi{10.1007/s00165-019-00497-w}
\end{bhost}
\end{bsubitem}
%
\OrigBibText
D.~Pous, D.~Sangiorgi, Bisimulation and coinduction enhancements: A historical
perspective, Formal Aspects of Computing (2019). 10.1007/s00165-019-00497-w.
\endOrigBibText
\bptok{structpyb}%
\endbibitem

%b25 ###
\bibitem{sangiorgi1992problem}
\begin{bsubitem}
\begin{bcontribution}%[language=fr]%de,it,pl,ru
\bauthor{\fnm{D.} \snm{Sangiorgi}}
\bauthor{\fnm{R.} \snm{Milner}}
\btitle{The problem of ``weak bisimulation up to''}
\end{bcontribution}
\begin{bhost}
\begin{beditedbook}
\btitle{LNCS 630 - CONCUR '92 - Concurrency Theory}
\bdate{1992}
\bpublisher{\bname{Springer}}
\end{beditedbook}
\bpages{\bfirstpage{32}\blastpage{46}}
\bdoi{10.1007/BFb0084781}
\end{bhost}
\end{bsubitem}
%
\OrigBibText
D.~Sangiorgi, R.~Milner, {The problem of ``Weak Bisimulation up to''},
in: LNCS 630 - CONCUR '92 - Concurrency Theory, Springer, 1992, pp. 32--46.
10.1007/BFb0084781.
\endOrigBibText
\bptok{structpyb}%
\endbibitem

%b26 ###
\bibitem{Tian:2017wrba}
\begin{bsubitem}
\begin{bcontribution}%[language=fr]%de,it,pl,ru
\bauthor{\fnm{C.} \snm{Tian}}
\btitle{A Formalization of Unique Solutions of Equations in Process
Algebra}
\end{bcontribution}
\bcomment{Master's thesis}\prnsep{,\ }
\begin{bhost}
\begin{bbook}%%[class=report]
\bdate{Dec. 2017}
\bpublisher{\bname{AlmaDigital} \blocation{Bologna}}
\end{bbook}
\end{bhost}
\begin{bhost}
\begin{behost}
\binterref[locator-type=url]{http://amslaurea.unibo.it/14798/}
\end{behost}
\end{bhost}
\end{bsubitem}
%
\OrigBibText
C.~Tian,
{{A Formalization of Unique
Solutions of Equations in Process Algebra}}, Master's thesis, AlmaDigital,
Bologna (Dec. 2017).
 \url{http://amslaurea.unibo.it/14798/}
\endOrigBibText
\bptok{structpyb}%
\endbibitem

%b27 ###
\bibitem{norrish2013ordinals}
\begin{bsubitem}
\begin{bcontribution}%[language=fr]%de,it,pl,ru
\bauthor{\fnm{M.} \snm{Norrish}}
\bauthor{\fnm{B.} \snm{Huffman}}
\btitle{Ordinals in HOL: transfinite arithmetic up to (and beyond) $\omega _1$}
\end{bcontribution}
\begin{bhost}
\begin{beditedbook}
\btitle{International Conference on Interactive Theorem Proving}
\bdate{2013}
\bpublisher{\bname{Springer}}
\end{beditedbook}
\bpages{\bfirstpage{133}\blastpage{146}}
\bdoi{10.1007/978-3-642-39634-2\_12}
\end{bhost}
\end{bsubitem}
%
\OrigBibText
M.~Norrish, B.~Huffman, {Ordinals in HOL: Transfinite arithmetic up to
(and beyond) $\omega _1$}, in: International Conference on Interactive
Theorem Proving, Springer, 2013, pp. 133--146. 10.1007/978-3-642-39634-2\_12.
\endOrigBibText
\bptok{structpyb}%
\endbibitem

%b28 ###
\bibitem{Nesi:2017wo}
\begin{bsubitem}
\begin{bcontribution}%[language=fr]%de,it,pl,ru
\bauthor{\fnm{M.} \snm{Nesi}}
\btitle{Formalising a value-passing calculus in HOL}
\end{bcontribution}
\begin{bhost}
\begin{bissue}
\bseries{\btitle{Form. Asp. Comput.} \bvolumeno{11}}
\bissueno{2}
\bdate{1999}
\end{bissue}
\bpages{\bfirstpage{160}\blastpage{199}}
\bdoi{10.1007/s001650050046}
\end{bhost}
\end{bsubitem}
%
\OrigBibText
M.~Nesi, {Formalising a Value-Passing Calculus in HOL}, Formal Aspects
of Computing 11~(2) (1999) 160--199. 10.1007/s001650050046.
\endOrigBibText
\bptok{structpyb}%
\endbibitem

%b29 ###
\bibitem{cleaveland1993concurrency}
\begin{bsubitem}
\begin{bcontribution}%[language=fr]%de,it,pl,ru
\bauthor{\fnm{R.} \snm{Cleaveland}}
\bauthor{\fnm{J.} \snm{Parrow}}
\bauthor{\fnm{B.} \snm{Steffen}}
\btitle{The concurrency workbench: a semantics-based tool for the verification
of concurrent systems}
\end{bcontribution}
\begin{bhost}
\begin{bissue}
\bseries{\btitle{ACM Trans. Program. Lang. Syst.} \bvolumeno{15}}
\bissueno{1}
\bdate{1993}
\end{bissue}
\bpages{\bfirstpage{36}\blastpage{72}}
\bdoi{10.1145/151646.151648}
\end{bhost}
\end{bsubitem}
%
\OrigBibText
R.~Cleaveland, J.~Parrow, B.~Steffen, The concurrency workbench: A semantics-based
tool for the verification of concurrent systems, ACM Transactions on Programming
Languages and Systems (TOPLAS) 15~(1) (1993) 36--72. 10.1145/151646.151648.
\endOrigBibText
\bptok{structpyb}%
\endbibitem

%b30 ###
\bibitem{bengtson2007completeness}
\begin{bsubitem}
\begin{bcontribution}%[language=fr]%de,it,pl,ru
\bauthor{\fnm{J.} \snm{Bengtson}}
\bauthor{\fnm{J.} \snm{Parrow}}
\btitle{A completeness proof for bisimulation in the pi-calculus using isabelle}
\end{bcontribution}
\begin{bhost}
\begin{bissue}
\bseries{\btitle{Electron. Notes Theor. Comput. Sci.} \bvolumeno{192}}
\bissueno{1}
\bdate{2007}
\end{bissue}
\bpages{\bfirstpage{61}\blastpage{75}}
\bdoi{10.1016/j.entcs.2007.08.017}
\end{bhost}
\end{bsubitem}
%
\OrigBibText
J.~Bengtson, J.~Parrow, A completeness proof for bisimulation in the pi-calculus
using isabelle, Electronic Notes in Theoretical Computer Science 192~(1)
(2007) 61--75. 10.1016/j.entcs.2007.08.017.
\endOrigBibText
\bptok{structpyb}%
\endbibitem

%b31 ###
\bibitem{parrow2009formalising}
\begin{bsubitem}
\begin{bcontribution}%[language=fr]%de,it,pl,ru
\bauthor{\fnm{J.} \snm{Parrow}}
\bauthor{\fnm{J.} \snm{Bengtson}}
\btitle{Formalising the pi-calculus using nominal logic}
\end{bcontribution}
\begin{bhost}
\begin{bissue}
\bseries{\btitle{Log. Methods Comput. Sci.} \bvolumeno{5}}
\bdate{2009}
\end{bissue}
\bdoi{10.2168/LMCS-5(2:16)2009}
\end{bhost}
\end{bsubitem}
%
\OrigBibText
J.~Parrow, J.~Bengtson, Formalising the pi-calculus using nominal logic,
Logical Methods in Computer Science 5 (2009). 10.2168/LMCS-5(2:16)2009.
\endOrigBibText
\bptok{structpyb}%
\endbibitem

%b32 ###
\bibitem{bengtson2010formalising}
\begin{bsubitem}
\begin{bcontribution}%[language=fr]%de,it,pl,ru
\bauthor{\fnm{J.} \snm{Bengtson}}
\btitle{Formalising process calculi}
\end{bcontribution}
\bcomment{Ph.D. thesis}\prnsep{,\ }
\begin{bhost}
\begin{bbook}[class=report]
\bdate{2010}
\bpublisher{\bname{Acta Universitatis Upsaliensis}}
\end{bbook}
\end{bhost}
\begin{bhost}
\begin{behost}
\binterref[locator-type=url]{http://uu.diva-portal.org/smash/record.jsf?pid=diva2\%3A311032}
\end{behost}
\end{bhost}
\end{bsubitem}
%
\OrigBibText
J.~Bengtson,
{Formalising
process calculi}, Ph.D. thesis, Acta Universitatis Upsaliensis (2010).
\url{http://uu.diva-portal.org/smash/record.jsf?pid=diva2\%3A311032}
\endOrigBibText
\bptok{structpyb}%
\endbibitem

%b33 ###
\bibitem{melham1994mechanized}
\begin{bsubitem}
\begin{bcontribution}%[language=fr]%de,it,pl,ru
\bauthor{\fnm{T.F.} \snm{Melham}}
\btitle{A mechanized theory of the pi-calculus in HOL}
\end{bcontribution}
\begin{bhost}
\begin{bissue}
\bseries{\btitle{Nord. J. Comput.} \bvolumeno{1}}
\bissueno{1}
\bdate{1994}
\end{bissue}
\bpages{\bfirstpage{50}\blastpage{76}}
\end{bhost}
\begin{bhost}
\begin{behost}
\binterref[locator-type=url]{http://core.ac.uk/download/pdf/22878407.pdf}
\end{behost}
\end{bhost}
\end{bsubitem}
%
\OrigBibText
T.~F. Melham,
{{A Mechanized
Theory of the Pi-Calculus in HOL}}, Nord. J. Comput. 1~(1) (1994) 50--76.
 \url{http://core.ac.uk/download/pdf/22878407.pdf}
\endOrigBibText
\bptok{structpyb}%
\endbibitem

%b34 ###
\bibitem{mohamed1995mechanizing}
\begin{bsubitem}
\begin{bcontribution}%[language=fr]%de,it,pl,ru
\bauthor{\fnm{O.} \snm{A\"{\i}t Mohamed}}
\btitle{Mechanizing a $\pi $-calculus equivalence in HOL}
\end{bcontribution}
\begin{bhost}
\begin{beditedbook}
\beditors{%
\beditor{\fnm{E.} \snm{Thomas Schubert}}
\beditor{\fnm{P.J.} \snm{Windley}}
\beditor{\fnm{J.} \snm{Alves-Foss}}}
\btitle{Higher Order Logic Theorem Proving and Its Applications}
\bdate{1995}
\bpublisher{\bname{Springer}\blocation{Berlin, Heidelberg}}
\end{beditedbook}
\bpages{\bfirstpage{1}\blastpage{16}}
\bdoi{10.1007/3-540-60275-5\_53}
\end{bhost}
\end{bsubitem}
%
\OrigBibText
O.~A{\"i}t~Mohamed, {{Mechanizing a $\pi $-calculus equivalence in HOL}},
in: E.~Thomas~Schubert, P.~J. Windley, J.~Alves-Foss (Eds.), Higher Order
Logic Theorem Proving and Its Applications, Springer Berlin Heidelberg,
Berlin, Heidelberg, 1995, pp. 1--16. 10.1007/3-540-60275-5\_53.
\endOrigBibText
\bptok{structpyb}%
\endbibitem

%b35 ###
\bibitem{compton2005embedding}
\begin{bsubitem}
\begin{bcontribution}%[language=fr]%de,it,pl,ru
\bauthor{\fnm{M.} \snm{Compton}}
\btitle{Embedding a fair CCS in Isabelle/HOL}
\end{bcontribution}
\begin{bhost}
\begin{beditedbook}
\btitle{Theorem Proving in Higher Order Logics: Emerging Trends Proceedings}
\bdate{2005}
\end{beditedbook}
\bpages{\bfirstpage{30}}
%\bdoi{10.1.1.105.834}
\end{bhost}
\begin{bhost}
\begin{behost}
\binterref[locator-type=url]{https://web.comlab.ox.ac.uk/techreports/oucl/RR-05-02.pdf\#page=36}
\end{behost}
\end{bhost}
\end{bsubitem}
%
\OrigBibText
M.~Compton,
{{Embedding
a fair CCS in Isabelle/HOL}}, in: Theorem Proving in Higher Order Logics:
Emerging Trends Proceedings, 2005, p.~30. \href {https://doi.org/10.1.1.105.834} 10.1.1.105.834.
\url{https://web.comlab.ox.ac.uk/techreports/oucl/RR-05-02.pdf\#page=36}
\endOrigBibText
\bptok{structpyb}%
\endbibitem

%b36 ###
\bibitem{chaudhuri2015lightweight}
\begin{bsubitem}
\begin{bcontribution}%[language=fr]%de,it,pl,ru
\bauthor{\fnm{K.} \snm{Chaudhuri}}
\bauthor{\fnm{M.} \snm{Cimini}}
\bauthor{\fnm{D.} \snm{Miller}}
\btitle{A lightweight formalization of the metatheory of bisimulation-up-to}
\end{bcontribution}
\begin{bhost}
\begin{beditedbook}
\btitle{Proceedings of the 2015 Conference on Certified Programs and Proofs}
\bdate{2015}
\bpublisher{\bname{ACM}}
\end{beditedbook}
\bpages{\bfirstpage{157}\blastpage{166}}
\bdoi{10.1145/2676724.2693170}
\end{bhost}
\end{bsubitem}
%
\OrigBibText
K.~Chaudhuri, M.~Cimini, D.~Miller, A lightweight formalization of the
metatheory of bisimulation-up-to, in: Proceedings of the 2015 Conference
on Certified Programs and Proofs, ACM, 2015, pp. 157--166. 10.1145/2676724.2693170.
\endOrigBibText
\bptok{structpyb}%
\endbibitem

%b37 ###
\bibitem{pous2007new}
\begin{bsubitem}
\begin{bcontribution}%[language=fr]%de,it,pl,ru
\bauthor{\fnm{D.} \snm{Pous}}
\btitle{New up-to techniques for weak bisimulation}
\end{bcontribution}
\begin{bhost}
\begin{bissue}
\bseries{\btitle{Theor. Comput. Sci.} \bvolumeno{380}}
\bdate{2007}
\end{bissue}
\bpages{\bfirstpage{164}\blastpage{180}}
\bdoi{10.1016/j.tcs.2007.02.060}
\end{bhost}
\end{bsubitem}
%
\OrigBibText
D.~Pous, New up-to techniques for weak bisimulation, Theoretical Computer
Science 380 (2007) 164--180. 10.1016/j.tcs.2007.02.060.
\endOrigBibText
\bptok{structpyb}%
\endbibitem

%b38 ###
\bibitem{kahsai2008implementing}
\begin{bsubitem}
\begin{bcontribution}%[language=fr]%de,it,pl,ru
\bauthor{\fnm{T.} \snm{Kahsai}}
\bauthor{\fnm{M.} \snm{Miculan}}
\btitle{Implementing spi calculus using nominal techniques}
\end{bcontribution}
\begin{bhost}
\begin{beditedbook}
\btitle{Conference on Computability in Europe}
\bdate{2008}
\bpublisher{\bname{Springer}}
\end{beditedbook}
\bpages{\bfirstpage{294}\blastpage{305}}
\bdoi{10.1007/978-3-540-69407-6\_33}
\end{bhost}
\end{bsubitem}
%
\OrigBibText
T.~Kahsai, M.~Miculan, Implementing spi calculus using nominal techniques,
in: Conference on Computability in Europe, Springer, 2008, pp. 294--305.
10.1007/978-3-540-69407-6\_33.
\endOrigBibText
\bptok{structpyb}%
\endbibitem

%b39 ###
\bibitem{hirschkoff1997full}
\begin{bsubitem}
\begin{bcontribution}%[language=fr]%de,it,pl,ru
\bauthor{\fnm{D.} \snm{Hirschkoff}}
\btitle{A full formalisation of $\pi $-calculus theory in the calculus of constructions}
\end{bcontribution}
\begin{bhost}
\begin{beditedbook}
\btitle{International Conference on Theorem Proving in Higher Order Logics}
\bdate{1997}
\bpublisher{\bname{Springer}}
\end{beditedbook}
\bpages{\bfirstpage{153}\blastpage{169}}
\bdoi{10.1007/BFb0028392}
\end{bhost}
\end{bsubitem}
%
\OrigBibText
D.~Hirschkoff, A full formalisation of $\pi $-calculus theory in the calculus
of constructions, in: International Conference on Theorem Proving in Higher
Order Logics, Springer, 1997, pp. 153--169. 10.1007/BFb0028392.
\endOrigBibText
\bptok{structpyb}%
\endbibitem

%b40 ###
\bibitem{DurierHS17}
\begin{bsubitem}
\begin{bcontribution}%[language=fr]%de,it,pl,ru
\bauthor{\fnm{A.} \snm{Durier}}
\bauthor{\fnm{D.} \snm{Hirschkoff}}
\bauthor{\fnm{D.} \snm{Sangiorgi}}
\btitle{Divergence and unique solution of equations}
\end{bcontribution}
\begin{bhost}
\begin{beditedbook}
\beditors{%
\beditor{\fnm{R.} \snm{Meyer}}
\beditor{\fnm{U.} \snm{Nestmann}}}
\btitle{28th International Conference on Concurrency Theory (CONCUR 2017)}
\bbookseries{
\bseries{\btitle{Leibniz International Proceedings in Informatics (LIPIcs)}\bvolumeno{85}}}
\bdate{2017}
\bpublisher{\bname{Schloss
Dagstuhl--Leibniz-Zentrum fuer Informatik}\blocation{Dagstuhl, Germany}}
\end{beditedbook}
\bpages{\bfirstpage{11:1}\blastpage{11:16}}
\bdoi{10.4230/LIPIcs.CONCUR.2017.11}
\end{bhost}
\end{bsubitem}
%
\OrigBibText
A.~Durier, D.~Hirschkoff, D.~Sangiorgi, Divergence and unique solution
of equations, in: R.~Meyer, U.~Nestmann (Eds.), 28th International Conference
on Concurrency Theory (CONCUR 2017), Vol.~85 of Leibniz International Proceedings
in Informatics (LIPIcs), Schloss Dagstuhl--Leibniz-Zentrum fuer Informatik,
Dagstuhl, Germany, 2017, pp. 11:1--11:16. 10.4230/LIPIcs.CONCUR.2017.11.
\endOrigBibText
\bptok{structpyb}%
\endbibitem

%b41 ###
\bibitem{DurierHS18}
\begin{bsubitem}
\begin{bcontribution}%[language=fr]%de,it,pl,ru
\bauthor{\fnm{A.} \snm{Durier}}
\bauthor{\fnm{D.} \snm{Hirschkoff}}
\bauthor{\fnm{D.} \snm{Sangiorgi}}
\btitle{Eager functions as processes}
\end{bcontribution}
\begin{bhost}
\begin{beditedbook}
\btitle{\xch{33rd Annual}{33nd Annual} {ACM/IEEE} Symposium on Logic in Computer Science, {LICS} 2018}
\bdate{2018}
\bpublisher{\bname{{IEEE} Computer Society}}
\end{beditedbook}
\bpages{\bfirstpage{364}\blastpage{373}}
\bdoi{10.1145/3209108.3209152}
\end{bhost}
\end{bsubitem}
%
\OrigBibText
A.~Durier, D.~Hirschkoff, D.~Sangiorgi, Eager functions as processes, in:
33nd Annual {ACM/IEEE} Symposium on Logic in Computer Science, {LICS} 2018,
{IEEE} Computer Society, 2018, pp. 364--373. 10.1145/3209108.3209152.
\endOrigBibText
\bptok{structpyb}%
\endbibitem

\end{thebibliography}

\end{backmatter}
\end{document}

