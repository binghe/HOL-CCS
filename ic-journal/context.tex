%%%% -*- Mode: LaTeX -*-
%%
%% This is the draft of the 2nd part of EXPRESS/SOS 2018 paper, co-authored by
%% Prof. Davide Sangiorgi and Chun Tian.

\subsection{Context, guardedness and (pre)congruence}
\label{ss:context}

% We need to find a suitable formal definition of 
% context. There're multiple ways. Here 

We have chosen to use $\lambda$-expressions (with the type
``\HOLinline{\ensuremath{(}\ensuremath{\alpha}, \ensuremath{\beta}\ensuremath{)} \HOLTyOp{CCS} \HOLTokenTransEnd \ensuremath{(}\ensuremath{\alpha}, \ensuremath{\beta}\ensuremath{)} \HOLTyOp{CCS}}'')
to represent \emph{multi-hole contexts}.
%
This choice has a significant advantage over \emph{one-hole
contexts}, as each hole corresponds to one
appearance of the \emph{same} variable in single-variable
expressions (or equations). Thus \emph{contexts} can be directly used in
formulating the unique solution of equations theorems in
single-variable cases. The precise definition is given inductively:
\begin{alltt}
\HOLTokenTurnstile{} \HOLConst{CONTEXT} \ensuremath{(}\HOLTokenLambda{}\HOLBoundVar{t}. \HOLBoundVar{t}\ensuremath{)} \HOLSymConst{\HOLTokenConj{}} \ensuremath{(}\HOLSymConst{\HOLTokenForall{}}\HOLBoundVar{p}. \HOLConst{CONTEXT} \ensuremath{(}\HOLTokenLambda{}\HOLBoundVar{t}. \HOLBoundVar{p}\ensuremath{)}\ensuremath{)} \HOLSymConst{\HOLTokenConj{}}
   \ensuremath{(}\HOLSymConst{\HOLTokenForall{}}\HOLBoundVar{a} \HOLBoundVar{e}. \HOLConst{CONTEXT} \HOLBoundVar{e} \HOLSymConst{\HOLTokenImp{}} \HOLConst{CONTEXT} \ensuremath{(}\HOLTokenLambda{}\HOLBoundVar{t}. \HOLBoundVar{a}\HOLSymConst{\ensuremath{\ldotp}}\HOLBoundVar{e} \HOLBoundVar{t}\ensuremath{)}\ensuremath{)} \HOLSymConst{\HOLTokenConj{}}
   \ensuremath{(}\HOLSymConst{\HOLTokenForall{}}\ensuremath{\HOLBoundVar{e}\sb{\mathrm{1}}} \ensuremath{\HOLBoundVar{e}\sb{\mathrm{2}}}. \HOLConst{CONTEXT} \ensuremath{\HOLBoundVar{e}\sb{\mathrm{1}}} \HOLSymConst{\HOLTokenConj{}} \HOLConst{CONTEXT} \ensuremath{\HOLBoundVar{e}\sb{\mathrm{2}}} \HOLSymConst{\HOLTokenImp{}} \HOLConst{CONTEXT} \ensuremath{(}\HOLTokenLambda{}\HOLBoundVar{t}. \ensuremath{\HOLBoundVar{e}\sb{\mathrm{1}}} \HOLBoundVar{t} \HOLSymConst{\ensuremath{+}} \ensuremath{\HOLBoundVar{e}\sb{\mathrm{2}}} \HOLBoundVar{t}\ensuremath{)}\ensuremath{)} \HOLSymConst{\HOLTokenConj{}}
   \ensuremath{(}\HOLSymConst{\HOLTokenForall{}}\ensuremath{\HOLBoundVar{e}\sb{\mathrm{1}}} \ensuremath{\HOLBoundVar{e}\sb{\mathrm{2}}}. \HOLConst{CONTEXT} \ensuremath{\HOLBoundVar{e}\sb{\mathrm{1}}} \HOLSymConst{\HOLTokenConj{}} \HOLConst{CONTEXT} \ensuremath{\HOLBoundVar{e}\sb{\mathrm{2}}} \HOLSymConst{\HOLTokenImp{}} \HOLConst{CONTEXT} \ensuremath{(}\HOLTokenLambda{}\HOLBoundVar{t}. \ensuremath{\HOLBoundVar{e}\sb{\mathrm{1}}} \HOLBoundVar{t} \HOLSymConst{\ensuremath{\mid}} \ensuremath{\HOLBoundVar{e}\sb{\mathrm{2}}} \HOLBoundVar{t}\ensuremath{)}\ensuremath{)} \HOLSymConst{\HOLTokenConj{}}
   \ensuremath{(}\HOLSymConst{\HOLTokenForall{}}\HOLBoundVar{L} \HOLBoundVar{e}. \HOLConst{CONTEXT} \HOLBoundVar{e} \HOLSymConst{\HOLTokenImp{}} \HOLConst{CONTEXT} \ensuremath{(}\HOLTokenLambda{}\HOLBoundVar{t}. \ensuremath{(\nu}\HOLBoundVar{L}\ensuremath{)} \ensuremath{(}\HOLBoundVar{e} \HOLBoundVar{t}\ensuremath{)}\ensuremath{)}\ensuremath{)} \HOLSymConst{\HOLTokenConj{}}
   \HOLSymConst{\HOLTokenForall{}}\HOLBoundVar{rf} \HOLBoundVar{e}. \HOLConst{CONTEXT} \HOLBoundVar{e} \HOLSymConst{\HOLTokenImp{}} \HOLConst{CONTEXT} \ensuremath{(}\HOLTokenLambda{}\HOLBoundVar{t}. \HOLConst{relab} \ensuremath{(}\HOLBoundVar{e} \HOLBoundVar{t}\ensuremath{)} \HOLBoundVar{rf}\ensuremath{)}\hfill{[CONTEXT_rules]}
\end{alltt}
Notice in above inductive definitions, for any CCS term $p$,
\HOLinline{\HOLTokenLambda{}\HOLBoundVar{t}.\\\;\HOLFreeVar{p}} is a valid context. In this case there is no
hole in the context. On the other hand, if any CCS term with recursion
could be a context, the recursion must be a subterm of this $p$.

A context is \emph{weakly guarded} (\texttt{WG}) if each hole is
underneath a prefix:
\begin{alltt}
\HOLTokenTurnstile{} \ensuremath{(}\HOLSymConst{\HOLTokenForall{}}\HOLBoundVar{p}. \HOLConst{WG} \ensuremath{(}\HOLTokenLambda{}\HOLBoundVar{t}. \HOLBoundVar{p}\ensuremath{)}\ensuremath{)} \HOLSymConst{\HOLTokenConj{}} \ensuremath{(}\HOLSymConst{\HOLTokenForall{}}\HOLBoundVar{a} \HOLBoundVar{e}. \HOLConst{CONTEXT} \HOLBoundVar{e} \HOLSymConst{\HOLTokenImp{}} \HOLConst{WG} \ensuremath{(}\HOLTokenLambda{}\HOLBoundVar{t}. \HOLBoundVar{a}\HOLSymConst{\ensuremath{\ldotp}}\HOLBoundVar{e} \HOLBoundVar{t}\ensuremath{)}\ensuremath{)} \HOLSymConst{\HOLTokenConj{}}
   \ensuremath{(}\HOLSymConst{\HOLTokenForall{}}\ensuremath{\HOLBoundVar{e}\sb{\mathrm{1}}} \ensuremath{\HOLBoundVar{e}\sb{\mathrm{2}}}. \HOLConst{WG} \ensuremath{\HOLBoundVar{e}\sb{\mathrm{1}}} \HOLSymConst{\HOLTokenConj{}} \HOLConst{WG} \ensuremath{\HOLBoundVar{e}\sb{\mathrm{2}}} \HOLSymConst{\HOLTokenImp{}} \HOLConst{WG} \ensuremath{(}\HOLTokenLambda{}\HOLBoundVar{t}. \ensuremath{\HOLBoundVar{e}\sb{\mathrm{1}}} \HOLBoundVar{t} \HOLSymConst{\ensuremath{+}} \ensuremath{\HOLBoundVar{e}\sb{\mathrm{2}}} \HOLBoundVar{t}\ensuremath{)}\ensuremath{)} \HOLSymConst{\HOLTokenConj{}}
   \ensuremath{(}\HOLSymConst{\HOLTokenForall{}}\ensuremath{\HOLBoundVar{e}\sb{\mathrm{1}}} \ensuremath{\HOLBoundVar{e}\sb{\mathrm{2}}}. \HOLConst{WG} \ensuremath{\HOLBoundVar{e}\sb{\mathrm{1}}} \HOLSymConst{\HOLTokenConj{}} \HOLConst{WG} \ensuremath{\HOLBoundVar{e}\sb{\mathrm{2}}} \HOLSymConst{\HOLTokenImp{}} \HOLConst{WG} \ensuremath{(}\HOLTokenLambda{}\HOLBoundVar{t}. \ensuremath{\HOLBoundVar{e}\sb{\mathrm{1}}} \HOLBoundVar{t} \HOLSymConst{\ensuremath{\mid}} \ensuremath{\HOLBoundVar{e}\sb{\mathrm{2}}} \HOLBoundVar{t}\ensuremath{)}\ensuremath{)} \HOLSymConst{\HOLTokenConj{}}
   \ensuremath{(}\HOLSymConst{\HOLTokenForall{}}\HOLBoundVar{L} \HOLBoundVar{e}. \HOLConst{WG} \HOLBoundVar{e} \HOLSymConst{\HOLTokenImp{}} \HOLConst{WG} \ensuremath{(}\HOLTokenLambda{}\HOLBoundVar{t}. \ensuremath{(\nu}\HOLBoundVar{L}\ensuremath{)} \ensuremath{(}\HOLBoundVar{e} \HOLBoundVar{t}\ensuremath{)}\ensuremath{)}\ensuremath{)} \HOLSymConst{\HOLTokenConj{}}
   \HOLSymConst{\HOLTokenForall{}}\HOLBoundVar{rf} \HOLBoundVar{e}. \HOLConst{WG} \HOLBoundVar{e} \HOLSymConst{\HOLTokenImp{}} \HOLConst{WG} \ensuremath{(}\HOLTokenLambda{}\HOLBoundVar{t}. \HOLConst{relab} \ensuremath{(}\HOLBoundVar{e} \HOLBoundVar{t}\ensuremath{)} \HOLBoundVar{rf}\ensuremath{)}\hfill{[WG_rules]}
\end{alltt}
(Notice the differences between a weak guarded context and a normal
one: $\lambda t. t$ is not weakly guarded as the variable is directly
exposed without any prefixed action. And $\lambda t. a.e[t]$ is weakly
guarded as long as $e[\cdot]$ is a context, not necessary weakly guarded.)

A context is \emph{guarded} (\texttt{SG}) if each hole is underneath a \emph{visible} prefix:
\begin{alltt}
\HOLTokenTurnstile{} \ensuremath{(}\HOLSymConst{\HOLTokenForall{}}\HOLBoundVar{p}. \HOLConst{SG} \ensuremath{(}\HOLTokenLambda{}\HOLBoundVar{t}. \HOLBoundVar{p}\ensuremath{)}\ensuremath{)} \HOLSymConst{\HOLTokenConj{}} \ensuremath{(}\HOLSymConst{\HOLTokenForall{}}\HOLBoundVar{l} \HOLBoundVar{e}. \HOLConst{CONTEXT} \HOLBoundVar{e} \HOLSymConst{\HOLTokenImp{}} \HOLConst{SG} \ensuremath{(}\HOLTokenLambda{}\HOLBoundVar{t}. \HOLConst{label} \HOLBoundVar{l}\HOLSymConst{\ensuremath{\ldotp}}\HOLBoundVar{e} \HOLBoundVar{t}\ensuremath{)}\ensuremath{)} \HOLSymConst{\HOLTokenConj{}}
   \ensuremath{(}\HOLSymConst{\HOLTokenForall{}}\HOLBoundVar{a} \HOLBoundVar{e}. \HOLConst{SG} \HOLBoundVar{e} \HOLSymConst{\HOLTokenImp{}} \HOLConst{SG} \ensuremath{(}\HOLTokenLambda{}\HOLBoundVar{t}. \HOLBoundVar{a}\HOLSymConst{\ensuremath{\ldotp}}\HOLBoundVar{e} \HOLBoundVar{t}\ensuremath{)}\ensuremath{)} \HOLSymConst{\HOLTokenConj{}}
   \ensuremath{(}\HOLSymConst{\HOLTokenForall{}}\ensuremath{\HOLBoundVar{e}\sb{\mathrm{1}}} \ensuremath{\HOLBoundVar{e}\sb{\mathrm{2}}}. \HOLConst{SG} \ensuremath{\HOLBoundVar{e}\sb{\mathrm{1}}} \HOLSymConst{\HOLTokenConj{}} \HOLConst{SG} \ensuremath{\HOLBoundVar{e}\sb{\mathrm{2}}} \HOLSymConst{\HOLTokenImp{}} \HOLConst{SG} \ensuremath{(}\HOLTokenLambda{}\HOLBoundVar{t}. \ensuremath{\HOLBoundVar{e}\sb{\mathrm{1}}} \HOLBoundVar{t} \HOLSymConst{\ensuremath{+}} \ensuremath{\HOLBoundVar{e}\sb{\mathrm{2}}} \HOLBoundVar{t}\ensuremath{)}\ensuremath{)} \HOLSymConst{\HOLTokenConj{}}
   \ensuremath{(}\HOLSymConst{\HOLTokenForall{}}\ensuremath{\HOLBoundVar{e}\sb{\mathrm{1}}} \ensuremath{\HOLBoundVar{e}\sb{\mathrm{2}}}. \HOLConst{SG} \ensuremath{\HOLBoundVar{e}\sb{\mathrm{1}}} \HOLSymConst{\HOLTokenConj{}} \HOLConst{SG} \ensuremath{\HOLBoundVar{e}\sb{\mathrm{2}}} \HOLSymConst{\HOLTokenImp{}} \HOLConst{SG} \ensuremath{(}\HOLTokenLambda{}\HOLBoundVar{t}. \ensuremath{\HOLBoundVar{e}\sb{\mathrm{1}}} \HOLBoundVar{t} \HOLSymConst{\ensuremath{\mid}} \ensuremath{\HOLBoundVar{e}\sb{\mathrm{2}}} \HOLBoundVar{t}\ensuremath{)}\ensuremath{)} \HOLSymConst{\HOLTokenConj{}}
   \ensuremath{(}\HOLSymConst{\HOLTokenForall{}}\HOLBoundVar{L} \HOLBoundVar{e}. \HOLConst{SG} \HOLBoundVar{e} \HOLSymConst{\HOLTokenImp{}} \HOLConst{SG} \ensuremath{(}\HOLTokenLambda{}\HOLBoundVar{t}. \ensuremath{(\nu}\HOLBoundVar{L}\ensuremath{)} \ensuremath{(}\HOLBoundVar{e} \HOLBoundVar{t}\ensuremath{)}\ensuremath{)}\ensuremath{)} \HOLSymConst{\HOLTokenConj{}}
   \HOLSymConst{\HOLTokenForall{}}\HOLBoundVar{rf} \HOLBoundVar{e}. \HOLConst{SG} \HOLBoundVar{e} \HOLSymConst{\HOLTokenImp{}} \HOLConst{SG} \ensuremath{(}\HOLTokenLambda{}\HOLBoundVar{t}. \HOLConst{relab} \ensuremath{(}\HOLBoundVar{e} \HOLBoundVar{t}\ensuremath{)} \HOLBoundVar{rf}\ensuremath{)}\hfill{[SG_rules]}
\end{alltt}

A context is \emph{sequential} (\texttt{SEQ}) if each of its \emph{subcontexts} with
a hole, apart from the hole itself, is in forms of prefixes or sums:
(c.f. Def.~\ref{def:guardness} and p.101,157 of \cite{Mil89} for
the informal definitions.)
 \begin{alltt}
\HOLTokenTurnstile{} \HOLConst{SEQ} \ensuremath{(}\HOLTokenLambda{}\HOLBoundVar{t}. \HOLBoundVar{t}\ensuremath{)} \HOLSymConst{\HOLTokenConj{}} \ensuremath{(}\HOLSymConst{\HOLTokenForall{}}\HOLBoundVar{p}. \HOLConst{SEQ} \ensuremath{(}\HOLTokenLambda{}\HOLBoundVar{t}. \HOLBoundVar{p}\ensuremath{)}\ensuremath{)} \HOLSymConst{\HOLTokenConj{}} \ensuremath{(}\HOLSymConst{\HOLTokenForall{}}\HOLBoundVar{a} \HOLBoundVar{e}. \HOLConst{SEQ} \HOLBoundVar{e} \HOLSymConst{\HOLTokenImp{}} \HOLConst{SEQ} \ensuremath{(}\HOLTokenLambda{}\HOLBoundVar{t}. \HOLBoundVar{a}\HOLSymConst{\ensuremath{\ldotp}}\HOLBoundVar{e} \HOLBoundVar{t}\ensuremath{)}\ensuremath{)} \HOLSymConst{\HOLTokenConj{}}
   \HOLSymConst{\HOLTokenForall{}}\ensuremath{\HOLBoundVar{e}\sb{\mathrm{1}}} \ensuremath{\HOLBoundVar{e}\sb{\mathrm{2}}}. \HOLConst{SEQ} \ensuremath{\HOLBoundVar{e}\sb{\mathrm{1}}} \HOLSymConst{\HOLTokenConj{}} \HOLConst{SEQ} \ensuremath{\HOLBoundVar{e}\sb{\mathrm{2}}} \HOLSymConst{\HOLTokenImp{}} \HOLConst{SEQ} \ensuremath{(}\HOLTokenLambda{}\HOLBoundVar{t}. \ensuremath{\HOLBoundVar{e}\sb{\mathrm{1}}} \HOLBoundVar{t} \HOLSymConst{\ensuremath{+}} \ensuremath{\HOLBoundVar{e}\sb{\mathrm{2}}} \HOLBoundVar{t}\ensuremath{)}\hfill{[SEQ_rules]}
\end{alltt}

In the same manner, we can also define variants of contexts (\texttt{GCONTEXT}) and weakly guarded
contexts (\texttt{WGS}) in which only guarded sums are allowed (i.e.~arbitrary sums are forbidden):
\begin{alltt}
\HOLTokenTurnstile{} \HOLConst{GCONTEXT} \ensuremath{(}\HOLTokenLambda{}\HOLBoundVar{t}. \HOLBoundVar{t}\ensuremath{)} \HOLSymConst{\HOLTokenConj{}} \ensuremath{(}\HOLSymConst{\HOLTokenForall{}}\HOLBoundVar{p}. \HOLConst{GCONTEXT} \ensuremath{(}\HOLTokenLambda{}\HOLBoundVar{t}. \HOLBoundVar{p}\ensuremath{)}\ensuremath{)} \HOLSymConst{\HOLTokenConj{}}
   \ensuremath{(}\HOLSymConst{\HOLTokenForall{}}\HOLBoundVar{a} \HOLBoundVar{e}. \HOLConst{GCONTEXT} \HOLBoundVar{e} \HOLSymConst{\HOLTokenImp{}} \HOLConst{GCONTEXT} \ensuremath{(}\HOLTokenLambda{}\HOLBoundVar{t}. \HOLBoundVar{a}\HOLSymConst{\ensuremath{\ldotp}}\HOLBoundVar{e} \HOLBoundVar{t}\ensuremath{)}\ensuremath{)} \HOLSymConst{\HOLTokenConj{}}
   \ensuremath{(}\HOLSymConst{\HOLTokenForall{}}\ensuremath{\HOLBoundVar{a}\sb{\mathrm{1}}} \ensuremath{\HOLBoundVar{a}\sb{\mathrm{2}}} \ensuremath{\HOLBoundVar{e}\sb{\mathrm{1}}} \ensuremath{\HOLBoundVar{e}\sb{\mathrm{2}}}.
        \HOLConst{GCONTEXT} \ensuremath{\HOLBoundVar{e}\sb{\mathrm{1}}} \HOLSymConst{\HOLTokenConj{}} \HOLConst{GCONTEXT} \ensuremath{\HOLBoundVar{e}\sb{\mathrm{2}}} \HOLSymConst{\HOLTokenImp{}} \HOLConst{GCONTEXT} \ensuremath{(}\HOLTokenLambda{}\HOLBoundVar{t}. \ensuremath{\HOLBoundVar{a}\sb{\mathrm{1}}}\HOLSymConst{\ensuremath{\ldotp}}\ensuremath{\HOLBoundVar{e}\sb{\mathrm{1}}} \HOLBoundVar{t} \HOLSymConst{\ensuremath{+}} \ensuremath{\HOLBoundVar{a}\sb{\mathrm{2}}}\HOLSymConst{\ensuremath{\ldotp}}\ensuremath{\HOLBoundVar{e}\sb{\mathrm{2}}} \HOLBoundVar{t}\ensuremath{)}\ensuremath{)} \HOLSymConst{\HOLTokenConj{}}
   \ensuremath{(}\HOLSymConst{\HOLTokenForall{}}\ensuremath{\HOLBoundVar{e}\sb{\mathrm{1}}} \ensuremath{\HOLBoundVar{e}\sb{\mathrm{2}}}. \HOLConst{GCONTEXT} \ensuremath{\HOLBoundVar{e}\sb{\mathrm{1}}} \HOLSymConst{\HOLTokenConj{}} \HOLConst{GCONTEXT} \ensuremath{\HOLBoundVar{e}\sb{\mathrm{2}}} \HOLSymConst{\HOLTokenImp{}} \HOLConst{GCONTEXT} \ensuremath{(}\HOLTokenLambda{}\HOLBoundVar{t}. \ensuremath{\HOLBoundVar{e}\sb{\mathrm{1}}} \HOLBoundVar{t} \HOLSymConst{\ensuremath{\mid}} \ensuremath{\HOLBoundVar{e}\sb{\mathrm{2}}} \HOLBoundVar{t}\ensuremath{)}\ensuremath{)} \HOLSymConst{\HOLTokenConj{}}
   \ensuremath{(}\HOLSymConst{\HOLTokenForall{}}\HOLBoundVar{L} \HOLBoundVar{e}. \HOLConst{GCONTEXT} \HOLBoundVar{e} \HOLSymConst{\HOLTokenImp{}} \HOLConst{GCONTEXT} \ensuremath{(}\HOLTokenLambda{}\HOLBoundVar{t}. \ensuremath{(\nu}\HOLBoundVar{L}\ensuremath{)} \ensuremath{(}\HOLBoundVar{e} \HOLBoundVar{t}\ensuremath{)}\ensuremath{)}\ensuremath{)} \HOLSymConst{\HOLTokenConj{}}
   \HOLSymConst{\HOLTokenForall{}}\HOLBoundVar{rf} \HOLBoundVar{e}. \HOLConst{GCONTEXT} \HOLBoundVar{e} \HOLSymConst{\HOLTokenImp{}} \HOLConst{GCONTEXT} \ensuremath{(}\HOLTokenLambda{}\HOLBoundVar{t}. \HOLConst{relab} \ensuremath{(}\HOLBoundVar{e} \HOLBoundVar{t}\ensuremath{)} \HOLBoundVar{rf}\ensuremath{)}\hfill{[GCONTEXT_rules]}
\end{alltt}
\begin{alltt}
\HOLTokenTurnstile{} \ensuremath{(}\HOLSymConst{\HOLTokenForall{}}\HOLBoundVar{p}. \HOLConst{WGS} \ensuremath{(}\HOLTokenLambda{}\HOLBoundVar{t}. \HOLBoundVar{p}\ensuremath{)}\ensuremath{)} \HOLSymConst{\HOLTokenConj{}} \ensuremath{(}\HOLSymConst{\HOLTokenForall{}}\HOLBoundVar{a} \HOLBoundVar{e}. \HOLConst{GCONTEXT} \HOLBoundVar{e} \HOLSymConst{\HOLTokenImp{}} \HOLConst{WGS} \ensuremath{(}\HOLTokenLambda{}\HOLBoundVar{t}. \HOLBoundVar{a}\HOLSymConst{\ensuremath{\ldotp}}\HOLBoundVar{e} \HOLBoundVar{t}\ensuremath{)}\ensuremath{)} \HOLSymConst{\HOLTokenConj{}}
   \ensuremath{(}\HOLSymConst{\HOLTokenForall{}}\ensuremath{\HOLBoundVar{a}\sb{\mathrm{1}}} \ensuremath{\HOLBoundVar{a}\sb{\mathrm{2}}} \ensuremath{\HOLBoundVar{e}\sb{\mathrm{1}}} \ensuremath{\HOLBoundVar{e}\sb{\mathrm{2}}}. \HOLConst{GCONTEXT} \ensuremath{\HOLBoundVar{e}\sb{\mathrm{1}}} \HOLSymConst{\HOLTokenConj{}} \HOLConst{GCONTEXT} \ensuremath{\HOLBoundVar{e}\sb{\mathrm{2}}} \HOLSymConst{\HOLTokenImp{}} \HOLConst{WGS} \ensuremath{(}\HOLTokenLambda{}\HOLBoundVar{t}. \ensuremath{\HOLBoundVar{a}\sb{\mathrm{1}}}\HOLSymConst{\ensuremath{\ldotp}}\ensuremath{\HOLBoundVar{e}\sb{\mathrm{1}}} \HOLBoundVar{t} \HOLSymConst{\ensuremath{+}} \ensuremath{\HOLBoundVar{a}\sb{\mathrm{2}}}\HOLSymConst{\ensuremath{\ldotp}}\ensuremath{\HOLBoundVar{e}\sb{\mathrm{2}}} \HOLBoundVar{t}\ensuremath{)}\ensuremath{)} \HOLSymConst{\HOLTokenConj{}}
   \ensuremath{(}\HOLSymConst{\HOLTokenForall{}}\ensuremath{\HOLBoundVar{e}\sb{\mathrm{1}}} \ensuremath{\HOLBoundVar{e}\sb{\mathrm{2}}}. \HOLConst{WGS} \ensuremath{\HOLBoundVar{e}\sb{\mathrm{1}}} \HOLSymConst{\HOLTokenConj{}} \HOLConst{WGS} \ensuremath{\HOLBoundVar{e}\sb{\mathrm{2}}} \HOLSymConst{\HOLTokenImp{}} \HOLConst{WGS} \ensuremath{(}\HOLTokenLambda{}\HOLBoundVar{t}. \ensuremath{\HOLBoundVar{e}\sb{\mathrm{1}}} \HOLBoundVar{t} \HOLSymConst{\ensuremath{\mid}} \ensuremath{\HOLBoundVar{e}\sb{\mathrm{2}}} \HOLBoundVar{t}\ensuremath{)}\ensuremath{)} \HOLSymConst{\HOLTokenConj{}}
   \ensuremath{(}\HOLSymConst{\HOLTokenForall{}}\HOLBoundVar{L} \HOLBoundVar{e}. \HOLConst{WGS} \HOLBoundVar{e} \HOLSymConst{\HOLTokenImp{}} \HOLConst{WGS} \ensuremath{(}\HOLTokenLambda{}\HOLBoundVar{t}. \ensuremath{(\nu}\HOLBoundVar{L}\ensuremath{)} \ensuremath{(}\HOLBoundVar{e} \HOLBoundVar{t}\ensuremath{)}\ensuremath{)}\ensuremath{)} \HOLSymConst{\HOLTokenConj{}}
   \HOLSymConst{\HOLTokenForall{}}\HOLBoundVar{rf} \HOLBoundVar{e}. \HOLConst{WGS} \HOLBoundVar{e} \HOLSymConst{\HOLTokenImp{}} \HOLConst{WGS} \ensuremath{(}\HOLTokenLambda{}\HOLBoundVar{t}. \HOLConst{relab} \ensuremath{(}\HOLBoundVar{e} \HOLBoundVar{t}\ensuremath{)} \HOLBoundVar{rf}\ensuremath{)}\hfill{[WGS_rules]}
\end{alltt}

Many lemmas about the above concepts (\HOLinline{\HOLConst{CONTEXT}}, \HOLinline{\HOLConst{WG}},
\HOLinline{\HOLConst{SEQ}}, etc.) have to be proved for the
 relationships among these kinds of contexts and properties about their
 compositions. These proofs are usually tedious and long,  due to multiple levels of
 inductions on the structure of the contexts.

A (pre)congruence is a relation on CCS processes defined on top of
\texttt{CONTEXT}. The only difference between congruence and
precongruence is that the former  is an equivalence (reflexive,
symmetric, transitive), while the latter can  just be  a preorder (reflexive, transitive):
\begin{alltt}
   \HOLConst{congruence} \HOLFreeVar{R} \HOLTokenDefEquality{}
     \HOLConst{equivalence} \HOLFreeVar{R} \HOLSymConst{\HOLTokenConj{}} \HOLSymConst{\HOLTokenForall{}}\HOLBoundVar{x} \HOLBoundVar{y} \HOLBoundVar{ctx}. \HOLConst{CONTEXT} \HOLBoundVar{ctx} \HOLSymConst{\HOLTokenImp{}} \HOLFreeVar{R} \HOLBoundVar{x} \HOLBoundVar{y} \HOLSymConst{\HOLTokenImp{}} \HOLFreeVar{R} \ensuremath{(}\HOLBoundVar{ctx} \HOLBoundVar{x}\ensuremath{)} \ensuremath{(}\HOLBoundVar{ctx} \HOLBoundVar{y}\ensuremath{)}\hfill{[congruence]}

   \HOLConst{precongruence} \HOLFreeVar{R} \HOLTokenDefEquality{}
     \HOLConst{PreOrder} \HOLFreeVar{R} \HOLSymConst{\HOLTokenConj{}} \HOLSymConst{\HOLTokenForall{}}\HOLBoundVar{x} \HOLBoundVar{y} \HOLBoundVar{ctx}. \HOLConst{CONTEXT} \HOLBoundVar{ctx} \HOLSymConst{\HOLTokenImp{}} \HOLFreeVar{R} \HOLBoundVar{x} \HOLBoundVar{y} \HOLSymConst{\HOLTokenImp{}} \HOLFreeVar{R} \ensuremath{(}\HOLBoundVar{ctx} \HOLBoundVar{x}\ensuremath{)} \ensuremath{(}\HOLBoundVar{ctx} \HOLBoundVar{y}\ensuremath{)}\hfill{[precongruence]}
\end{alltt}

Both strong bisimilarity ($\sim$) and
rooted bisimilarity ($\approx^c$) are congruence:
\begin{alltt}
\HOLTokenTurnstile{} \HOLConst{congruence} \HOLConst{STRONG_EQUIV}\hfill{[STRONG_EQUIV_congruence]}
\HOLTokenTurnstile{} \HOLConst{congruence} \HOLConst{OBS_CONGR}\hfill{[OBS_CONGR_congruence]}
\end{alltt}

Although weak bisimilarity ($\approx$) is \emph{not} a congruence
  with respect to~\texttt{CONTEXT}, it is indeed substitutive
  with respect to~\texttt{GCONTEXT} 
% (or if the CCS syntax were defined with
%   only guarded sum operator \cite{sangiorgi2015equations}) 
as $\approx$ is indeed preserved by guarded sums.

Rooted contraction, on the other hand, is a precongruence:
\begin{alltt}
\HOLTokenTurnstile{} \HOLConst{precongruence} \HOLConst{OBS_contracts}\hfill{[OBS_contracts_precongruence]}
\end{alltt}

% next file: coarsest.htex
  