%%%% -*- Mode: LaTeX -*-

\section{Conclusions and future work}
\label{s:concl}

In this paper, we have highlighted a formalisation of the theory of CCS in the 
HOL4 theorem prover (for lack of space we have not discussed 
the formalisation of some basic algebraic theory, of the basic
properties of the expansion preorder,   and of a few
 versions of `bisimulation up to'
techniques). % such as  bisimulation up-to bisimilarity). 
The formalisation has focused on the theory of
unique solution of equations and contractions. 
It has also allowed us to further develop the theory,
notably the basic properties of rooted contraction, and the unique
solution theorem for it with respect to rooted bisimilarity. 
The formalisation brings up and exploits similarities between results
and proofs for different equivalences and preorders. 
We think that the statements in the formalisation are easy to read and
understand, as they are very close to the original statements found in
standard CCS textbooks \cite{Gorrieri:2015jt,Mil89}.

For the future work, it would be worth extending
to multi-variable equations/contractions. A key aspect could be using unguarded constants as free variables
(\texttt{FV}) and defining guardedness directly on expressions of type CCS (instead of
CCS $\rightarrow$ CCS), then linking to contexts. For instance, an expression is weakly-guarded when each
of its free variables, replaced by a hole, results in a weakly-guarded context:
\begin{alltt}
\HOLTokenTurnstile{} \HOLConst{weakly_guarded1} \HOLFreeVar{E} \HOLSymConst{\HOLTokenEquiv{}}
   \HOLSymConst{\HOLTokenForall{}}\HOLBoundVar{X}. \HOLBoundVar{X} \HOLSymConst{\HOLTokenIn{}} \HOLConst{FV} \HOLFreeVar{E} \HOLSymConst{\HOLTokenImp{}} \HOLSymConst{\HOLTokenForall{}}\HOLBoundVar{e}. \HOLConst{CONTEXT} \HOLBoundVar{e} \HOLSymConst{\HOLTokenConj{}} (\HOLBoundVar{e} (\HOLConst{var} \HOLBoundVar{X}) \HOLSymConst{=} \HOLFreeVar{E}) \HOLSymConst{\HOLTokenImp{}} \HOLConst{WG} \HOLBoundVar{e}
\end{alltt}

Formalising other equivalences and preorders could also be considered,
notably the trace equivalences, as well as more refined process
calculi such as value-passing CCS.
% (e.g.~exploiting the type variable
%of actions).}
%
On another research line, one could examine the formalisation of a different
approach \cite{DurierHS17} to unique
solutions, in which the use of contraction is
replaced by semantic conditions on process transitions such as
divergence. 
%We hope this work also inspires new formalisations on other process calculi.

% Further plan on the formalisation mainly includes: 1) the extension to
% multi-variable equation/contractions. 2) the support of recursion
% operators in 
% CCS context and expressions.

% We believe that, beside the discovery of the rooted contraction $\rcontr$
% with a more elegant unique solution theorem,
% our CCS formalisation in HOL4 has also
% provided a solid formal framework for future theoretical developments in
% Concurrency Theory, particularly for process algebras like CCS. It's
% easily understandable, with statements extremely close to the original
% textbook. The logic foundations of HOL makes the whole work (or individual
% parts) easily portable to other theorem provers.

\paragraph{Acknowledgements}

We have benefitted from suggestions and comments 
from several people from the HOL
community, including (in alphabet order) Robert Beers, Jeremy Dawson,
Ramana Kumar,
Michael Norrish, 
Konrad Slind, and
Thomas T\"{u}rk.
%
The second half of this
paper was written in memory of Michael J.~C.~Gordon, the creator of HOL theorem prover.
