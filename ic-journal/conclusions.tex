%%%% -*- Mode: LaTeX -*-

\section{Conclusions and future work}
\label{s:concl}

In this paper, beside the introduction of rooted contraction and its
unique solution theorems as a theoretical extension
of~\cite{sangiorgi2017equations},
we highlighted a comprehensive formalisation of the theory of CCS in the 
HOL4 theorem prover. In particular, the formalisation supports four
methods for establishing (strong and weak) bisimilarities:
\begin{enumerate}
\item constructing a bisimulation (the standard bisimulation proof
method);
\item constructing a `bisimulation up-to'; 
\item employing algebraic laws;
\item constructing a system of equations or contractions
(i.e., the `unique-solution' method)
\end{enumerate}

The formalisation has actually focused on the theory of
unique solution of equations and contractions, both \univariate and
\multivariate cases. It    
 has allowed us to further develop the theory,
notably the basic properties of rooted contraction, and the unique
solution theorem for it with respect to rooted bisimilarity. 
The formalisation brings up and exploits similarities between results
and proofs for different equivalences and preorders. Indeed we have
considered several `unique-solution' results (for various equivalences
and preorders); they share many parts of the proofs, but present a few
delicate and subtle differences in a few points. In a paper-pencil
proof, checking all details would be long and error-prone,
especially in cases where the proofs are similar to each other or
  when there are long case analyses to be carried out.
Some of the textbook proofs were even wrong and the correct version is
known but not available in the literature. This is the case of Milner's
proof of the unique-solution theorem for $wb$
(Theorem~\ref{t:Mil89s3}). Now we have fully verified formal proofs for
people who can read them.
For some other textbook proofs, even they are correct, sometimes they actually
do not need all information from the antecedents, which can be further
weakened. This kind of improvements can be easily found during the
formalisation work.
For our CCS formalisation, we believe that all the definitions and
theorem statements are easy to read and
understand, as they are very close to their original statements in
textbooks~\cite{Gorrieri:2015jt,Mil89}.

For the future work, formalising other equivalences and preorders could also be considered,
notably the trace-based equivalences, as well as more refined process
calculi such as value-passing CCS.
%
On another research line, one could examine the formalisation of a different
approach~\cite{DurierHS17} to unique
solutions, in which the use of contraction is
replaced by semantic conditions on process transitions such as
divergence.
%
Finally, the disjointness between free and bound variables in the current
\multivariate formalisation is certainly a limitation that can be
removed in the future. If needed, extending the current work to
infinite number of equations and variables is also feasible.

% We believe that, beside the discovery of the rooted contraction $\rcontr$
% with a more elegant unique solution theorem,
% our CCS formalisation in HOL4 has also
% provided a solid formal framework for future theoretical developments in
% Concurrency Theory, particularly for process algebras like CCS. It's
% easily understandable, with statements extremely close to the original
% textbook. The logic foundations of HOL makes the whole work (or individual
% parts) easily portable to other theorem provers.

\paragraph{Acknowledgements}

We have  benefitted from suggestions and comments 
from  the \hl{anonymous reviewers} and several people from the HOL
community, including (in \hl{alphabetic} order) Robert Beers, Jeremy Dawson,
Ramana Kumar,
Michael Norrish, 
Konrad Slind, and
Thomas T\"{u}rk.
%
The paper was written in memory of Michael J.~C.~Gordon, the creator of the HOL theorem prover.
