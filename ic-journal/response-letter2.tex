\newcommand{\Mark}{{\underline{\tt R:} }}


\begin{center}
\textbf{
{\small Changes for Submission 1 to SI:EXPRESS/SOS'18 (2nd Round)}\\[5mm]
{\large Unique Solutions of Contractions, CCS, and their HOL
  Formalisation
}}
\end{center}



\vskip 10pt
\noindent

(TODO)
% We would like to thank the reviewers for many useful
% comments and suggestions. We believe we have taken all of them into consideration, as best
% as we could. 

% The main addition, here again following the recommendations of the reviewers 
% has been the extension of the our central theorems 
% to the multivariable case (ie., moving from ``unique solution theorems'' for a single
% equations, or contractions,  to multi-equations  or multi-contractions). 
%  This has been a significant effort (also because a few alternative ways to make the
%  extension had to be examined). At the end, this work 
% consists of another 4,000 lines of proof scripts and a whole new
% Section 5 in the paper.

In the following, we discuss the changes to the paper and provide a
point-by-point reply to the issues that have been raised.

Our answers are marked  with a `` \Mark''  at the beginning.

\ \\

Best regards,
\begin{flushright}
  Chun Tian and Davide Sangiorgi \\
  (TODO) December 22, 2019
\end{flushright}

\vspace{1cm}

\section*{Responses to Review 1}

\subsection*{Meta comments}

\begin{enumerate}

\item \begin{quote}
    The multivariable case has been added. This is great: it makes the result stronger, and adds a real improvement over the conference version. 
  \end{quote}
  \Mark
  Thanks.

\item \begin{quote}
Overall the authors did address the minor comments. But the paper
still contains quite a few grammatical and other minor mistakes, also
in the newly added text. Below are a couple that I found just browsing
again through the paper; but this is certainly inexhaustive and the
authors should do some more careful proofreading.
\end{quote}
\Mark
(TODO) We have done a more careful proofreading and fixed all
grammatically wrong sentences in the paper.

\item \begin{quote}
    In fact, one of the corrections appears to have introduced a new
    bug: Theorem 4.7 in the new version is wrong; I guess Def. 4.6
    should be about weak transitions. In the original submission,
    there was verbatim copied proof code, which was however
    correct. But as it stands now, it suffers from the famous problem
    with weak bisimulation up-to found by the second author (one can
    simply use the same counterexample $\{(\tau. a , 0)\}$).
  \end{quote}
  \Mark
  This is an embarassing typo. The two $\arr{\mu}$ in
  Definition 4.6 should be $\Arr{\mu}$ instead. Our actual formalization is
  correct. Below is the fixed part of Definition 4.6:
\begin{enumerate}
\item Whenever $P \Arr{\mu} P'$ then, for some $Q'$, $Q \Arcap{\mu} Q'$ and $P' \wb \S \wb Q'$,
\item Whenever $Q \Arr{\mu} Q'$ then, for some $P'$, $P \Arcap{\mu} P'$ and $P' \wb \S \wb Q'$.
\end{enumerate}

\end{enumerate}

\subsection*{Detailed comments}

