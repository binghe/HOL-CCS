\newcommand{\Mark}{{\underline{\tt R:} }}


\begin{center}
\textbf{
{\small Changes for Submission 1 to SI:EXPRESS/SOS'18 (2nd Round)}\\[5mm]
{\large Unique Solutions of Contractions, CCS, and their HOL
  Formalisation
}}
\end{center}



\vskip 10pt
\noindent

(TODO)
% We would like to thank the reviewers for many useful
% comments and suggestions. We believe we have taken all of them into consideration, as best
% as we could. 

% The main addition, here again following the recommendations of the reviewers 
% has been the extension of the our central theorems 
% to the multivariable case (ie., moving from ``unique solution theorems'' for a single
% equations, or contractions,  to multi-equations  or multi-contractions). 
%  This has been a significant effort (also because a few alternative ways to make the
%  extension had to be examined). At the end, this work 
% consists of another 4,000 lines of proof scripts and a whole new
% Section 5 in the paper.

In the following, we discuss the changes to the paper and provide a
point-by-point reply to the issues that have been raised.

Our answers are marked  with a `` \Mark''  at the beginning.

\ \\

Best regards,
\begin{flushright}
  Chun Tian and Davide Sangiorgi \\
  (TODO) December 22, 2019
\end{flushright}

\vspace{1cm}

\section*{Responses to Review 1}

\subsection*{Meta comments}

\begin{enumerate}

\item \begin{quote}
    The multivariable case has been added. This is great: it makes the result stronger, and adds a real improvement over the conference version. 
  \end{quote}
  \Mark
  Thanks.

\item \begin{quote}
Overall the authors did address the minor comments. But the paper
still contains quite a few grammatical and other minor mistakes, also
in the newly added text. Below are a couple that I found just browsing
again through the paper; but this is certainly inexhaustive and the
authors should do some more careful proofreading.
\end{quote}
\Mark
(TODO) We have done a more careful proofreading and fixed all
grammatically wrong sentences in the paper.

\item \begin{quote}
    In fact, one of the corrections appears to have introduced a new
    bug: Theorem 4.7 in the new version is wrong; I guess Def. 4.6
    should be about weak transitions. In the original submission,
    there was verbatim copied proof code, which was however
    correct. But as it stands now, it suffers from the famous problem
    with weak bisimulation up-to found by the second author (one can
    simply use the same counterexample $\{(\tau. a , 0)\}$).
  \end{quote}
  \Mark
  This is an embarassing typo. The two $\arr{\mu}$ in
  Definition 4.6 should be $\Arr{\mu}$ instead. Our actual formalization is
  correct. Below is the fixed part of Definition 4.6:
\begin{enumerate}
\item Whenever $P \Arr{\mu} P'$ then, for some $Q'$, $Q \Arcap{\mu} Q'$ and $P' \wb \S \wb Q'$,
\item Whenever $Q \Arr{\mu} Q'$ then, for some $P'$, $P \Arcap{\mu} P'$ and $P' \wb \S \wb Q'$.
\end{enumerate}

\end{enumerate}

\subsection*{Detailed comments}

\begin{enumerate}
  \item \begin{quote}
25 "essentially formalised" => "essentially formalising" ? something is grammatically off with this sentence
\end{quote}
  \Mark We have changed ``formalised'' (a wrong use of the past tense) to ``formalising''. 

  \item \begin{quote}
47 ", then consider" => ". Then we consider" or something like that
\end{quote}
  \Mark Fixed.

  \item \begin{quote}
311 "break the logic consistencies" break the logical consistency?
\end{quote}
  \Mark

  \item \begin{quote}
337 "rec" is blue
\end{quote}
  \Mark Fixed. The ``rec'' was wrongly identified as the keyword of ML
  languages in the \LaTeX{} verbatim package that we used.

  \item \begin{quote}
323 and around: why is "a" in italics and $\bar{b}$ not?
\end{quote}
  \Mark

  \item \begin{quote}
sec 4.2: I'm now totally lost on the meaning of $(\alpha, \beta) CCS$. First $\beta$ is a label, then an action, and then there is some mystifying text between brackets (l328) about how this is not explained. *Please* just explain clearly what $\alpha$ and $\beta$ mean in $(\alpha, \beta) CCS$. 
\end{quote}
  \Mark

  \item \begin{quote}
370 spurious \texttt{CCS_Subst} on line 370
\end{quote}
  \Mark Removed.

  \item \begin{quote}
530 "for the contraction" the contraction what? the game? functional? or remove "the"
\end{quote}
  \Mark

  \item \begin{quote}
555 "an “up-to” techniques." => "“up-to” techniques."
\end{quote}
  \Mark

  \item \begin{quote}
566 "the resulting ``weak bisimulation up``" correct; also the `` at the right should be ''
\end{quote}
  \Mark

  \item \begin{quote}
597 "not powerful *enough* to prove" 
\end{quote}
  \Mark

  \item \begin{quote}
600 perhaps point out that this is (almost) 
\end{quote}
  \Mark

  \item \begin{quote}
608 "This theorem is of special interests to us, because within our framework it seems impossible to prove it without any limitation" interests => interest; also, I do not really understand this sentence (which limitation?)
\end{quote}
  \Mark

  \item \begin{quote}
965 "Recall *from* the beginning of Section 3 *that*"
\end{quote}
  \Mark

  \item \begin{quote}
985 Is there anything fundamentally interesting going on in the proof? Otherwise perhaps remove it or replace by a high-level description?
\end{quote}
  \Mark

  \item \begin{quote}
1066 "the proof cannot complete" ?
\end{quote}
  \Mark

  \item \begin{quote}
1075 "substitutes each possibly occurrences" "substitues each possible occurence"
\end{quote}
  \Mark

  \item \begin{quote}
page 30: some the text in these lemma's is essentially (informal) explanation, with the HOL code giving the formal statements. Perhaps move the text out of the lemma environment? This will also make it look better, as it's now a full page of lemma's. 
\end{quote}
  \Mark

  \item \begin{quote}
1149: "Recall *that* in"
\end{quote}
  \Mark

  \item \begin{quote}
1160: So the idea is that, whenever we replace a variable by a hole and view the other variables as constants, this is a single variable context - right? Perhaps this (expressed in line 960) can be emphasised again around here (but not entirely sure). 
\end{quote}
  \Mark

  \item \begin{quote}
1165: "used *to* assert"
\end{quote}
  \Mark

  \item \begin{quote}
1169 add colon after e.g.
\end{quote}
  \Mark

  \item \begin{quote}
1171 add "then" after the comma 
\end{quote}
  \Mark

  \item \begin{quote}
1175 "The similar properties" => "Similar properties"
\end{quote}
  \Mark

  \item \begin{quote}
1182 "Not every expression fit with CCS syntax " ?
\end{quote}
  \Mark

  \item \begin{quote}
1210 "optional restriction" ?
\end{quote}
  \Mark

  \item \begin{quote}
1219 "that, the set" remove comma; also "must be disjoint" => "is disjoint"
\end{quote}
  \Mark

  \item \begin{quote}
1241 "while Milner did not pointed out explicitly," grammar
\end{quote}
  \Mark

  \item \begin{quote}
1293 " for people who can read them" sounds a bit strange; also the last sentence of the paragraph suggest most people could
\end{quote}
  \Mark

  \item \begin{quote}
1293-1294 "For some other textbook proofs (..) formalisation work." perhaps give a pointer to an instance where this was the case?
\end{quote}
  \Mark

\end{enumerate}
