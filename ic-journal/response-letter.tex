\begin{center}
\textbf{
{\small Changes for Submission 1 to SI:EXPRESS/SOS'18}\\[5mm]
{\large Unique Solutions of Contractions, CCS, and their HOL
  Formalisation
}}
\end{center}

Dear Chairs and Reviewers,

\vskip 10pt
\noindent

thank you for conditionally accepting our paper and for the fruitful
comments. We took all of them into consideration, and thoroughly
addressed each of them in the revised version. We believe that after
this revision, the quality of the paper has been significantly
improved.

In the following, we discuss the changes to the paper and provide a
point-by-point reply to the issues that have been raised.
\ \\

Best regards,
\begin{flushright}
  Chun Tian and Davide Sangiorgi \\
  September 13, 2019
\end{flushright}

\vspace{1cm}

\section*{Responses to Review 1 (accept - minor revision)}

\subsection*{Meta comments}

\begin{enumerate}
% \item \begin{quote}
%     This paper describes a HOL formalisation of several notions in
%     process theory: CCS and various kinds of bisimilarity, and in
%     particular the unique-solutions technique. The latter was
%     introduced by Milner, and recently revisited and refined by the
%     second author.
%   \end{quote}
  
% \item \begin{quote}
%     In general, these kinds of results are error-prone, especially
%     those related to weak bisimilarity: there are multiple known
%     mistakes in the literature. Therefore, formalisation is a
%     worthwile effort. Moreover, the formalisation itself is quite
%     readable. The paper itself documents this formalisation and is,
%     for the most part, clear.
%   \end{quote}

% \item \begin{quote}
%     The paper is an extended version of an EXPRESS/SOS paper. The
%     extension consists of a more detailed report of the proofs -
%     perhaps this is enough.
%   \end{quote}

\item \begin{quote}
    I was disappointed though that multi-variable contexts are still
    not treated (I reviewed the EXPRESS/SOS paper as well and
    complained about that then). This means the current formalisation
    does *not* fully formalise the unique-solution method. The journal
    version would have been ideal to solve this issue. It is briefly
    discussed in the conclusion, but in fact, the first author's
    master thesis argues that it is possible: "With these new devices,
    it’s possible to formalize all “unique solution of equations”
    theorems with multi-variable equations, without touching existing
    formalization framework.". The current discussion should at the
    very least explain this (is the master thesis wrong? is this too
    hard, or too much work?), but hopefully it can just be done
    properly so that the paper fully covers the correctness of the
    unique-solution method. 
  \end{quote}

  Now we have additionally formalised the multivariate (i.e. multi-variable)
  version of the \emph{unique solution of rooted contractions theorem}
  (Theorem~\ref{t:rcontraBisimulationU}) and \emph{Milner's strong
  unique-solution theorem} (Prop.~4.14 (2)
  of~\citep[p.~103]{Mil89}), together with a detailed discussion on
  multivariate contexts, weakly-guarded equations and multivariate
  variable substitutions. This work consists of 4,000 lines of new
  proof scripts, and it is indeed based on the existing framework in
  the sense that, we do not need to change the CCS datatype definition
  and any SOS rule. (c.f. Section~\ref{sec:multivariate} for more details.)

\item \begin{quote}
    overall the paper can use a bit of spell-checking and proofreading (e.g. there are some initials in the margin)
  \end{quote}


\end{enumerate}

\subsection*{Detailed comments}

\begin{enumerate}
\item \begin{quote}
    L45: "about unique solution*s*"
  \end{quote}

  \hl{TODO}: I don't understand this comment. Is the referee indicating
  that our uses of ``solutions'' wasn't accurate because we didn't
  formalize their multivariate case?
  
\item \begin{quote}
    L84 "the weak one" rephrase
  \end{quote}

  Rephrased: ``We sometimes call the bisimulation and bisimilarity defined above
the \emph{weak} bisimulation (and weak bisimilarity), ...''
  
\item \begin{quote}
    L91 "bisimiarity"
  \end{quote}

  We have fixed this typo (totally two instances in the paper, also in L92).
  
\item \begin{quote}
    proof of thm 3.10: could there be some kind of picture to help? also, the "Hence," in the one-to-last line refers only to the property right before it, right? (about $C[\tilde{Q}]$)
  \end{quote}

  We have rephrased this proof (also the theorem statements) with a
  picture (Fig.~\ref{fig:310} illustrating the proof. The "Hence" part
  actually used a property of $\wb$ that can be easily proven by induction
  on each $\tau$-transition in a weak transition.
  
\item \begin{quote}
    L209: "their precongruences" precongruence closure?
  \end{quote}

  Rephrased: ``... because all we need from $\mcontrBIS$ (or
$\rcontr$) is their substitutivity properties (as precongruences)
w.r.t the corresponding definitions of contexts.''

\item \begin{quote}
    L216 "using Lemma 3.13 ... Section 4.9" perhaps just repeat the lemma before the theorem, as it's a bit hard to read Lemma 4.14 at this stage 
  \end{quote}

  We have recalled Lemma 3.13 of \cite{Mil89} as
  Lemma~\ref{lem:milner313} right before Theorem 3.13, which now becomes Theorem~\ref{t:rcontraBisimulationU}).

\item \begin{quote}
    L224: full stop and new sentence after "1972"
  \end{quote}
  Done.
  
\item \begin{quote}
    L227 "considerable" -> "considerably"
  \end{quote}
  Done.
  
\item \begin{quote}
    L228 "formal theories built in HOL is easily convincible" questionable grammar
  \end{quote}

  Now we used the following sentence: ``As a consequence, theories and proofs verified in HOL are relatively more
  convincible to non-experts.''
  
\item \begin{quote}
    L238 how is this without loss of generality?
  \end{quote}

  Sentence eliminated as now we have formalized the multivariate
  version of unique-solution theorems. The ``without loss of
  generality'' part is actually true, in the sense that, all steps of
  informal proofs do appears in the uni-variate formal proofs. The
  additional difficulities of the formal multivariate proof is discussed in Section~\ref{sec:multivariate}.
  
\item \begin{quote}
    L245 maybe more clear to just give the definition of $\beta$ Action 
  \end{quote}

  The original HOL definition of the type Action is now given.
  
\item \begin{quote}
    L250,300: perhaps explain what these (double) backquotes signify or just remove the sentence
  \end{quote}
  Sentences removed. In HOL, single and double backquotes are
  delimitator between ``inner'' (logic) and ``outer'' (proof)
  languages but this is not quite relevant to the present paper.
  After new HOL releases (k13) most uses of backquotes in  definitions can be eliminated.
  
\item \begin{quote}
    L277 "substutiion"
  \end{quote}
  Fixed.
  
\item \begin{quote}
    L280 (around): put some of these equalities on a single line? enough space
  \end{quote}
  Now we have removed most less-interesting cases and explained in
  details the two interesting cases (recursion operator and (free) variables).
  
\item \begin{quote}
    L316 and later, e.g. 323: is "p" a closed ccs term?
  \end{quote}

\item \begin{quote}
    L361 "Many lemmas ... contexts" this is a bit vague
  \end{quote}

  Rephrased: ``Many lemmas about above concepts (\texttt{CONTEXT}, \texttt{WG},
  \texttt{SEQ}, etc.)''
  
\item \begin{quote}
    L374 "An highlight"
  \end{quote}
  Fixed. Now we say ``One highlight...''.
  
\item \begin{quote}
    L378 "is a" => "are"
  \end{quote}

\item \begin{quote}
    L385 "With *the* above"
  \end{quote}

\item \begin{quote}
    L386 "the definition" of what? (bisimilarity, but say that)
  \end{quote}

\item \begin{quote}
    L394 "since *the* Kananaskis-11 release"
  \end{quote}

\item \begin{quote}
    L396 "we call *the* Hol_coreln"
  \end{quote}

\item \begin{quote}
    L410 $E$ and $E'$ missing
  \end{quote}

\item \begin{quote}
    L436 "is defined upon weak transition" grammar
  \end{quote}

\item \begin{quote}
    L505 this is good: it is missing for strong bisimilarity, should
    be added there too (the two definitions are now not explicitly
    related)
  \end{quote}

\item \begin{quote}
    L580 "a contraction" ?
  \end{quote}

\item \begin{quote}
    section 4.6 header: "the formalisation of bisimulation up to
    bisimilarity", i would say; also since you're not using
    compatible/respectful functions
  \end{quote}

\item \begin{quote}
    L599 PRQ not so pretty spacing. perhaps mention around here that
    naive weak bisimulation up to weak bisimilarity is unsound?
  \end{quote}

\item \begin{quote}
    L655 (around here): i'd say in the text above it that this is a
    different bisimulation game, not a different up-to technique
  \end{quote}

\item \begin{quote}
    L708 " closure of bisimilarity under such operator" not sure whether this is gramatically correct
  \end{quote}

\item \begin{quote}
    figure 2: I'm a bit confused since the other implication is missing. 
  \end{quote}

\item \begin{quote}
    Overall, in section 4.7, it seems the claim in the beginning is
    only proved under certain hypotheses: the "free-action" property,
    or in the Glabbeek proof, finite processes. This is all very much
    hidden: and it should be said up-front what we are going to prove
    exactly under which assumptions.
  \end{quote}

\item \begin{quote}
    L740 "this proof" which? 
  \end{quote}

\item \begin{quote}
    The beginning of section 4.8 is really confusing: what are "the
    two root processes"? What are we going to prove in this section,
    and why, what is the aim here? Again, under which assumptions are
    we going to work?
  \end{quote}

\item \begin{quote}
    L820 "bisimularity"
  \end{quote}

\item \begin{quote}
    Lemma 4.14 is nicely introduced, Thm 4.15 not. 
  \end{quote}

\item \begin{quote}
    L898 full stop \& new sentence after "many ways"
  \end{quote}

\item \begin{quote}
    L990-995 aren't some of the textbook proofs even wrong? (e.g. [2]
    in the proof for unique solutions makes use of weak bisimulation
    up to weak bisimilarity, if i remember correctly) - one could
    perhaps say this, it makes the point even stronger
  \end{quote}

\end{enumerate}

\section*{Responses to Review 2 (accept - minor revision)}

\subsection*{Meta comments}

\begin{enumerate}
\item \begin{quote}
    This paper presents a formalisation in HOL of the core classical
    theory of CCS, with a particular focus on unique (modulo a proper
    equivalence) solutions of equations, and of the more recent theory
    of contractions developed by one of the authors.
  \end{quote}

\item \begin{quote}
    The focus is on weak semantics, which is well-known to be quite
    delicate, which makes the results of the paper relevant and
    interesting.
  \end{quote}

\item \begin{quote}
    Notably, the HOL formalisation has allowed the authors to refine
    the theory of unique solutions of contractions. In detail, the
    authors rely on rooted contractions and prove that a system of
    weakly guarded contractions has a unique solution modulo rooted
    weak bisimilarity. Interestingly, no constraint on occurrences of
    summations is required.

    I found this work elegant and interesting, I propose acceptance with minor review.
  \end{quote}

\end{enumerate}

\subsection*{Detailed comments}

\begin{enumerate}

\item \begin{quote}
    There are some typos. E.g., several times I read “bisimiarity” (the “l” is missed).
  \end{quote}

\item \begin{quote}
    There are side remarks that should be removed.
  \end{quote}

\item \begin{quote}
    The two sentences immediately before and after Example 3.7 are equivalent.
  \end{quote}

\item \begin{quote}
    The notion of “substitutivity property” should be introduced, now it’s used but not explained.
  \end{quote}

\item \begin{quote}
    I strongly believe that Section 4 is accessible only to people
    already familiar with HOL syntax. I suggest the authors to give
    some more details. Some examples.
  \end{quote}

\item \begin{quote}
    You assume the reader knows what Hol_reln is, but this not always the case. 
Then, you define the congruence in terms of the equivalence, but is
the equivalence already defined somewhere?
  \end{quote}

\item \begin{quote}
    The argument at lines 387-389 is clear only after the lines 394-427 have been read.
  \end{quote}

\item \begin{quote}
    At lines 577 you write that proving properties for contraction is
    in general harder than proving properties for expansion. Could you
    explain why?
  \end{quote}
\end{enumerate}

\section*{Responses to Review 3 (accept - major revision)}

\subsection*{Meta comments}

\begin{enumerate}
\item \begin{quote}
    This paper presents a HOL formalisation of part of the theory of
    CCS. It formalises the syntax and operational semantics of CCS,
    strong and weak bisimilarity, and several well-known auxiliary
    techniques for establishing that CCS processes are bisimilar.
  \end{quote}

\item \begin{quote}
    The most notable contribution seems to be the formalisation (and
    modest extension) of results already obtained by the second author
    in an ACM TOCL paper (ref. [7]) pertaining regarding the
    uniqueness of solutions of contractions. The main addition with
    respect to the EXPRESS/SOS 2018 publication of the authors with
    the same title (ref. [1]), seems to be a HOL formalisation of the
    proof that rooted weak bisimilarity is the coarsest congruence
    contained in weak bisimilarity, and that rooted contraction is the
    coarsest precongruence contained in contraction. The HOL proof
    scripts are publicly available through a GitHub repository.
  \end{quote}

\item \begin{quote}
    The scientific contribution of the paper is, in my opinion,
    modest, at least in its current form. With the exception of the
    extension of the results in [7] to rooted contractions, the paper
    seems to merely consist of reformulations in HOL of well-known
    definitions and theorems with, here and there, some superficial
    comments about the proof scripts needed to prove the results
    (e.g., the coinduction package available for HOL simplifies the
    formalisation).
  \end{quote}

\item \begin{quote}
    Putting it bluntly, the current message of the paper seems to be
    that there are HOL proof scripts available in a GitHub
    repository. Perhaps there is something interesting and general to
    learn from this particular HOL formalisation, either about HOL or
    about CCS, but then this should be explained in much more detail.
  \end{quote}

\item \begin{quote}
    It does not help that the paper has many typos and ungrammatical
    sentences. Also, the article explains too little about HOL for a
    reader without prior exposure to HOL to fully appreciate the
    contents of Section 4 and, in particular, understand the
    meta-remarks about the formalisation.
  \end{quote}

\item \begin{quote}
    I do find it very useful that the theory of CCS is formalised in
    HOL, so that others can benefit from it when doing
    computer-assisted proofs of results pertaining to CCS, or about
    CCS processes. I think this is important work, and researchers
    should get due credit for this. Currently, the value is really in
    the availability of the HOL proof scripts, and not so much in the
    article they wrote about it.
  \end{quote}

\item \begin{quote}
    In conclusion, on the positive side, I think the authors did some
    important work formalising part of the theory of CCS in HOL, and
    on the negative side I have doubts about the scientific value of
    the article they wrote about it. I am therefore reluctant to
    support publication in Information and Computation at this stage.
  \end{quote}

\item \begin{quote}
    I am therefore reluctant to support publication in Information and
    Computation at this stage. I suggest to ask the authors for a
    major revision in which they make the paper accessible for readers
    not familiar with HOL, comment in more detail on the HOL proof
    scripts and what can be learned from them, and take into account
    the detailed comments below.
  \end{quote}
\end{enumerate}

\subsection*{Line-based comments}

\begin{enumerate}
\item \begin{quote}
    Abstract:
    There should be a remark that this is about weak bisimilarity. In
    general 'bisimilarity' is strong bisimilarity.
  \end{quote}

  We have added "(weak)" before "bisimilarity", in the first line of abstract.

\item \begin{quote}
    L18: Remove the word 'rather'
  \end{quote}

\item \begin{quote}
    L28: Hennessy Lemma -> Hennessy's Lemma
  \end{quote}

\item \begin{quote}
    L28: Deng Lemma -> Deng's lemma
  \end{quote}

\item \begin{quote}
    L28: Remove the word 'long'
  \end{quote}

\item \begin{quote}
    L28: proofs -> theorems
  \end{quote}

\item \begin{quote}
    L32: Add the word 'also' after 'the work is'
  \end{quote}

\item \begin{quote}
    L34: From the view of -> From the point of view of
  \end{quote}

\item \begin{quote}
    L35: Regarding the sentence "as formally proving a previously known result gives us the chance to see
    what is really needed for establishing that result."
    Either omit this sentence or explain what technical insights one
    gains by doing this work
  \end{quote}

\item \begin{quote}
    L43-48: This sentence is difficult to read and could be split into multiple sentences.
  \end{quote}

  We have rephrased it into multiple sentences.
  
\item \begin{quote}
    L55-60: Remove "Given a deadlock 0"
  \end{quote}

\item \begin{quote}
    L58: the trailing 0 -> a trailing 0
  \end{quote}

\item \begin{quote}
    L59: "shown in Fig. 1" is placed after "a Labeled Transition
    System", causing the belief that Figure 1 contains an LTS. 
    But there is no LTS. Figure 1 contains the SOS of CCS.
  \end{quote}

  We have rephrased it, saying Fig.1 represents an LTS expressed in SOS rules.

\item \begin{quote}
    Semantics: The relabeling rule is not explained anywhere.
  \end{quote}

\item \begin{quote}
    Semantics: The constraint on the relabeling functions does not belong to the inference rule, but should be formulated as a restriction on the syntax.
  \end{quote}

\item \begin{quote}
    L74: Add the word 'a' between 'are' and 'context'.
  \end{quote}

\item \begin{quote}
  Contexts: Does an $i$-holed context have exactly one hole $[]_i$ for
  each $i$? This question is not answered anywhere in the text.

  In the formalisation you only use contexts with one kind of hole $[]$. This definition can be changed to only
  use one hole as well, with a clear indication that this hole can occur multiple times.
\end{quote}

\item \begin{quote}
    L82: Remove the comma after $Q'$
  \end{quote}

\item \begin{quote}
    Definition 2.2: Superscript $c$ is already used for contextual closure.
    Does $\approx^c$ stand for the contextual closure of $\approx$, or is it something else?
    Since contextual closure is not used anywhere in the paper, it is best to remove it.
    Otherwise you need different notation for rooted bisimilarity.
  \end{quote}

\item \begin{quote}
    L91: What does 'reducing' mean?
  \end{quote}

\item \begin{quote}
    L91: bisimiarity -> bisimilarity
  \end{quote}

\item \begin{quote}
    L92: bisimiarity -> bisimilarity
  \end{quote}

\item \begin{quote}
    L92: Add a colon after 'results'
  \end{quote}

\item \begin{quote}
    L118: Why is the word 'actions' in bold font? It says 'ds' in the margin, what does that mean?
  \end{quote}

\item \begin{quote}
Definition 3.3: The definition of ‘sequential’ is awkward.  Moreover
this definition calls for some examples since the definitions can be
very confusing.
  \end{quote}

\item \begin{quote}
Theorem 3.4: What are direct sums?
  \end{quote}

\item \begin{quote}
    L.127: I think it would be clearer to give two concrete solutions
    rather than saying that any process not using a is a solution.
  \end{quote}

\item \begin{quote}
    L138: "Bisimilarity contraction ... some contraction R". This is not a grammatically correct sentence.
  \end{quote}

\item \begin{quote}
    L142: ".. are even required". The word 'even' suggests that this
    is a stronger requirement, but it is in fact weaker.
  \end{quote}

\item \begin{quote}
    L143: Add the words "that is" after 'preorder'.
  \end{quote}

\item \begin{quote}
    L144: remove ‘that’
  \end{quote}

\item \begin{quote}
    L153: "As bisimilarity .. but sums." This is not a grammatically correct sentence.
  \end{quote}

\item \begin{quote}
    Example 3.7: this example shows strictness of the inclusion. But
    the sentence before the example suggests that this is going to
    show that the preorder is not preserved by sums.
  \end{quote}

\item \begin{quote}
    L157: Doesn’t this statement deserve a proof?
  \end{quote}

\item \begin{quote}
    L163: $P >= E[P]$ is spelled out earlier in Definition 3.2. Maybe spell out this one as well.
  \end{quote}

\item \begin{quote}
    L165: If you define the notion of guardedness in Definition 3.3 on
    expressions first, and only thereafter extend it to systems of
    equations, then you could here simply reuse the definition on
    expressions to get a definition of the notion for systems of
    contractions. The advantage is that you are precise rather than
    suggestive.
  \end{quote}

\item \begin{quote}
    L169: "The number of strong steps of which is is composed". Note
    that there is not a unique sequence; in fact, there may be many
    ways to reach R from $C[\tilde{P}]$ with a weak $\mu$-trace.
  \end{quote}

\item \begin{quote}
    L176: What does "applying the definition of $>=_{bis}$" mean?
    Footnote: Why not just spell out this definition instead of making a remark about it?  It would make the conclusion of the proof somewhat easier to digest. If you decide to keep the footnote, change $\tilde{R}$ to $R'$. The notation $\tilde{R}$ is used for tuples.
  \end{quote}

\item \begin{quote}
    L181: vis ?
  \end{quote}

\item \begin{quote}
    L184: closes -> completes
  \end{quote}

\item \begin{quote}
    L194: ‘the definition should not be recursive’?
  \end{quote}

\item \begin{quote}
    Definition 3.11: Similar to the remark about Definition 2.2.
  \end{quote}

\item \begin{quote}
    Proof of Theorem 3.14: Why do you use context $C$ instead of tuples of expressions $\tilde{E}$?
    If this is the case because you only use contexts with one kind of hole, then you should use the letter
    E instead of C.
    If $\tilde{P}$ is a solution then $P_i >=^c E_i[\tilde{P}]$ for all i, but this quantification over i is missing.
  \end{quote}

\item \begin{quote}
    L215: The prove -> The proof
  \end{quote}

\item \begin{quote}
    General: the word 'Logic' in "HOL Logic" is redundant.
  \end{quote}

\item \begin{quote}
    L223-225: Perhaps you want to reformulate this long sentence. It is hard to read.
  \end{quote}

\item \begin{quote}
    L227: considerable -> considerably
  \end{quote}

\item \begin{quote}
    L228: is -> are
  \end{quote}

\item \begin{quote}
    L228: what does convincible mean here?
  \end{quote}

\item \begin{quote}
    L230: single?
  \end{quote}

\item \begin{quote}
    L235: The sentence between parentheses is an important remark and
    should not be between parentheses.
  \end{quote}

\item \begin{quote}
    L238: without loss of generality?
  \end{quote}

\item \begin{quote}
    L239: similar to the comment about line 118, theorem-prover is
    bold and 'ds' is in the margin.
  \end{quote}

\item \begin{quote}
    L245: now based on HOL’s option theory?
  \end{quote}

\item \begin{quote}
    L249: What are ‘backquotes’?
  \end{quote}

\item \begin{quote}
    Footnote 3: What are CCS literals?
  \end{quote}

\item \begin{quote}
    L265: What does the turnstile mean? I would expect that structural
    operational rules are axioms of the theory, but the turnstile
    suggests to me that they are theorems.
  \end{quote}

\item \begin{quote}
    L275: we substitute rec A. P for all occurrences of the variable A in P.
  \end{quote}

\item \begin{quote}
    L279: substutiion -> substitution
  \end{quote}

\item \begin{quote}
    L289: these -> the, on the tenary -> of the ternary
  \end{quote}

\item \begin{quote}
    General: in the lines 293-300 the trailing deadlock 0 is given,
    while in the preliminaries you stated that it would be omitted. If
    you decide to keep the deadlock, it should also be in bold font.
  \end{quote}

\item \begin{quote}
    L296: I don’t understand fact (1).
  \end{quote}

\item \begin{quote}
    L299: ‘Hence ... CCS_TRANS_CONV’ is not a grammatically correct sentence; there is a verb missing.
  \end{quote}

\item \begin{quote}
    L316: Explain what is p.
  \end{quote}

\item \begin{quote}
    Footnote 4: Do you mean ‘unguarded recursion’? How can relabelling operators cause infinite branching?
  \end{quote}

\item \begin{quote}
    L362: tedious but long -> tedious and long
  \end{quote}

\item \begin{quote}
    L374: An highlight -> A highlight
  \end{quote}

\item \begin{quote}
    L378: relations on CCS process -> relation on CCS processes
  \end{quote}

\item \begin{quote}
    Footnote 5: Add the word 'respectively' between 'quantifiers' and 'in'.
  \end{quote}

\item \begin{quote}
    Footnote 6: Regarding the last sentence between parentheses. What do you mean by this?
  \end{quote}

\item \begin{quote}
    General: The same trick of applying the coinduction package to the
    definition of bisimulation is repeated twice.
  \end{quote}

\item \begin{quote}
    General: In the sections about the HOL formalisation you refer to
    weak bisimulation instead of bisimulation.    
  \end{quote}

\item \begin{quote}
    L461: I think there is no need to include those 3
    theorems. Wouldn’t it suffice to write that similar theorems are
    returned as for \texttt{STRONG\_EQUIV}?
  \end{quote}

\item \begin{quote}
    L496-503: A similar remark can be made for \texttt{STRONG\_EQUIV}. Why do you put it here?
  \end{quote}

\item \begin{quote}
    L523,524: Similar to the comment about line L28
  \end{quote}

\item \begin{quote}
    L545: What does this statement mean?
  \end{quote}

\item \begin{quote}
    Up-to techniques: You mention four up-to techniques for weak
    bisimilarity and each of them are not powerful enough to prove the
    unique solution theorems. Why are they not powerful enough?
  \end{quote}

\item \begin{quote}
    L580: ‘Surprisingly’ To me this is not really surprising. Why do you find it surprising?
  \end{quote}

\item \begin{quote}
    L606-619: I really see no added value in repeating Def. 4.2 in HOL syntax.
  \end{quote}

\item \begin{quote}
    Def. 4.4: Shouldn’t there be double arrows? (twice)
  \end{quote}

\item \begin{quote}
    L700: This paragraph starts with a sentence in parentheses, which is not very desirable.
    Moreover this sentence does not seem to be connected to the next sentence.
  \end{quote}

\item \begin{quote}
    L701: Sentence does not end with a period.
  \end{quote}

\item \begin{quote}
    L704: What do you mean by "which itself may not be"?
  \end{quote}

\item \begin{quote}
    Figure 2: The $\subseteq$ from \texttt{SUM\_EQUIV} to Rooted
    bisimilarity makes the figure confusing; put a $\supseteq$
  \end{quote}

\item \begin{quote}
    L723: is -> in
  \end{quote}

\item \begin{quote}
    L723-L727: This sentence should be split into multiple sentences.
  \end{quote}

\item \begin{quote}
    L744-745: pure mathematics?
  \end{quote}

\item \begin{quote}
    L746: Arbitrary -> Arbitrarily
  \end{quote}

\item \begin{quote}
    L749: "infinite such sub-processes" this construction does not seem right and could be avoided.
  \end{quote}

\item \begin{quote}
    L754: non-$\tau$ action $a$ -> $a\neq\tau$
  \end{quote}

\item \begin{quote}
    L766: "By induction on the definition", don't you mean by induction on natural numbers?
  \end{quote}

\item \begin{quote}
    L823: are -> is
  \end{quote}

\item \begin{quote}
    L831: `In above proof’ Which proof are your referring too? I only see a theorem.
  \end{quote}

\item \begin{quote}
    L837: Add the word 'a' between 'not' and 'congruence'
  \end{quote}

\item \begin{quote}
    L837: why do you also need the condition of sequentiality?
  \end{quote}

\item \begin{quote}
    L991: error-prune -> error-prone, ’since one is tempted to overlook details’?
  \end{quote}

\item \begin{quote}
    L1001: the trace equivalences -> trace-based equivalences
  \end{quote}

\item \begin{quote}
    L1006: in alphabetic order
  \end{quote}

\item \begin{quote}
    L1040: hol4 -> HOL4
  \end{quote}

\end{enumerate}

\section*{Other changes}

In this section, we listed other changes. All line numbers are based
on the original review version of this paper.

\begin{enumerate}
\item TODO
\end{enumerate}
