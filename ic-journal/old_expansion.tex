
\subsection{Expansions}
\label{s:expa}

\hl{The bisimulation proof method can be enhanced by means of \emph{up-to
techniques}. One of the most useful auxiliary relations in up-to
techniques is the \emph{expansion} relation} $\expa$ \cite{arun1992efficiency,sangiorgi2015equations}.
This is an asymmetric version
of $\wb$ where $P \expa Q$ means that $P \wb Q$,
but also that $Q$ achieves the same as $P$
with no more work, i.e.~with no more $\tau$ actions.
Intuitively, if $P \expa Q$, we can think of $Q$ as being
at least as fast as $P$
or, more generally, we can think that $P$ uses at least as many resources as $Q$.
\begin{definition}%[expansion]
\label{d:expa}
A process relation ${\R}$
  is an \textbf{expansion} if, whenever
we have $P\RR Q$, for all $\mu$
 \begin{enumerate}
 \item   $P \arr\mu P'$ implies that there is $Q'$ with $Q \arcap \mu
   Q'$
  and $P' \RR Q'$;
 \item
     $Q \arr\mu Q'$   implies that there is $P'$ with $P \Arr \mu
  P'$ and $P'
 \RR Q'$.
 \end{enumerate}
  $P$  {\em expands} $Q$, written as
 $P  \expa Q$,
 if $P \RR Q$ for some expansion $\R$.
 \end{definition}

Same as bisimilarity, the expansion preorder is preserved by all operators but (direct) sums.

% next file: contraction.tex
