\section{Equations and contractions}
\label{s:eq}

\hl{In the CCS syntax given in previous section, there is the syntactic
possibility that an agent variable $A$ occurs
outside the recursion operator of the same variable (or, equivalently, a
process constant $C$ is used but undefined). We say such a variable is
\emph{free}, while those variables wrapped by their recursions are
\emph{bound}.}
\hl{A CCS \emph{tem} (or \emph{expression}, i.e. those with valid syntax)
without any free variable is called a CCS \emph{process}.}
\hl{For instance, the variable $X$ is free in $a.X + \recu Y (b.Y)$, while $Y$
is bound. There is also the syntactic possibility that the same
variable $X$ is
both free and bound, e.g. in $a.X + \recu X (b.X)$.
This does not cause any ambiguity, but sometimes leads to more
complex proofs.} (c.f. Section~\ref{sec:multivariate} for more details.)

\hl{In this paper (and the formalisation work), we reuse the free agent
variables as \emph{equation variables}. This eliminates the need of
another type for CCS equations, and we can also reuse the existing
variable substitution operation} (c.f. SOS rule for the Recursion in
Fig.~\ref{f:LTSCCS}) \hl{for the substitution of equation variables.
For example, the result of substituting equation variable $X$ to $\nil$ in $a.X +
\recu X (b.X)$, denoted by $(a.X + \recu X (b.X)) \sub {\nil} X$, is
$a.\nil + \recu X (b.X)$ with the part $\recu X (b.X)$
untouched. Multivariate substitutions are written in the same syntax,
e.g. $E \sub {\til P} {\til X}$. Whenever $\til X$ is clear from the
context, we may also write $E[\til P]$ instead of $E \sub {\til P} {\til
  X}$. ($E[P]$ for $E \sub P X$ in the case of a single variable $X$.)}

\hl{In fact, free agent variables have the same transitional behavior
as the deadlock $\nil$ if we strictly follow the existing SOS rules, as there is
no SOS rule for them at all. In fact, most CCS theorems still hold if
the involved CCS terms contain free variables. (The most notable
exceptions are all versions of unique solution of equations/contracts,
where all solutions must be \emph{pure} processes (without free variables).)}

\subsection{Systems of equations}
\label{ss:SysEq}

Milner's ``unique solution of equations'' theorems~\cite{Mil89} intuitively
say that, if a context $C$ \hl{(i.e., a CCS expression with possibly
free variables)} obeys certain conditions,
then all processes $P$ that satisfy the equation $P \wb \ct P$ are
bisimilar with each other.

% TODO

\begin{definition} % Def 3.1
Assume that, for each $i$ of 
 a countable indexing set $I$, we have variables $X_i$, and expressions
$E_i$ possibly containing  such variables. 
Then 
$\{  X_i = E_i\}_{i\in I}$
is 
  a \emph{system of equations}. (There is one equation for each variable $X_i$.)
\end{definition}

We write $E[\til P]$ for the expression resulting from $E$ by
replacing each variable $X_i$   with the process $P_i$, assuming
$\til P$ and $\til X$ have the same length. (This is syntactic
replacement.) 
% The components of $\til P$ need not be
%  different from each other, as it must be for the variables $\til X$.
% If the system has infinitely many equations,
% the  tuples $\til P$ and $\til X$
%  are infinite too.
\begin{definition}
Suppose  $\{  X_i = E_i\}_{i\in I}$ is a system of equations: 
\begin{itemize}
\item
 $\til P$ is a \emph{solution of the 
system of equations  for $\wb$} 
if for each $i$ it holds
that $P_i \wb E_i [\til P]$;

\item it %the  system
 has \emph{a unique solution for $\wb$}  if whenever 
 $\til P$ and $\til Q$ are both solutions for $\wb$, then $\til P \wb
 \til Q$. 
\end{itemize} 
 \end{definition} 

% Examples of systems with a  unique solution for $\wb$ are: 
% \begin{enumerate}
% \item
% $ X = a. X$ 

% \item 
% $ X_1 = a.  X_2$, $ X_2 = b.  X_1$  

% \end{enumerate}

For instance, the solution of the equation 
$ X = a. X$ 
is  the process
$R \DSdefi \recu A {\, (a. A)}$, and for any other solution $P$ we have $P \wb R$.
In contrast, the equation 
 $X = a|  X$ has solutions that may be quite different, for instance,
 $K$ and $K | b$, for $K \DSdefi \recu K {\, (a. K)}$. (Actually any process capable of
continuously performing \hl{$a$--actions} is a solution \hl{of} $X = a  |  X$.)

% The unique solution of the system (1), modulo $\wb$,  is the constant $K \Defi a
% . K$:  for any other solution $P$ we have $P \wb K$.
% The unique solution of (2), modulo $\wb$, are the constants $K_1 , K_2$
% with $K_1 \Defi a . K_2$ and $K_2 \Defi b. K_1$; again, for any other
% pair of solutions $P_1,P_2$ we have $K_1 \wb P_1$ and $K_2 \wb P_2$.
% Examples of systems that do not have a unique solution are: 
% \begin{enumerate}
% \item $X = X $ 

% \item $X = \tau . X$
% \item $X = a | X$

% \end{enumerate} 
% All processes are solutions of (1) and (2); examples of solutions for
% (3) are $K$ and $K | b$, for $K \Defi a
% . K$.

\begin{definition}[guardedness of equations]
\label{def:guardness}
A system of equations 
$\{  X_i = E_i\}_{i\in I}$
 is 
\begin{itemize}
\item \emph{weakly guarded} if, in each $E_i$, each occurrence of
  an \behav\  variable is underneath a prefix;

\item \emph{(strongly) guarded} if, in each $E_i$, each occurrence of
  an equation variable is underneath a \emph{visible} prefix;

\item \emph{sequential} if, in each $E_i$, each \hl{occurrence of an
    equation variable only appears underneath prefixes and sums.}
\end{itemize}
\end{definition}

\hl{In other words, if the system is sequential, then for
every expression $E_i$, any subexpression of $E_i$ in which $X_j $
appears, apart from $X_j$ itself, is a sum (of prefixed terms).
For instance,}
\begin{itemize}
\item $X = \tau. X + \mu . \nil$ is sequential but not 
 guarded, because the guarding prefix for the variable
is not visible.

\item $X =  \ell . X | P$ is guarded but not sequential.

\item $X =  \ell . X + \tau. \res a (a .\outC b | a.\nil)$, as well as
$X = \tau . (a. X + \tau . b .X + \tau  )$ are both guarded and sequential.
\end{itemize}

\begin{theorem}[unique solution of equations, \cite{Mil89}]
\label{t:Mil89}
A system of guarded and sequential equations (\hl{with guarded sums only})
$\{  X_i = E_i\}_{i\in I}$ has an unique solution for $\wb$.
\end{theorem}

\hl{The proof exploits an invariance property on immediate transitions for
guarded and sequential expressions, and then extracts a bisimulation
(up to bisimilarity) out
of the solutions of the system.  }
To see the need of the sequentiality  condition, consider
 the equation (from \cite{Mil89}) $X = \res a (a. X | \outC a)$
where $X$ is guarded but not sequential. Any process that does not use
$a$ is a solution\hl{, e.g. $\nil$ and $b.\nil$.}

%% next file: contraction.tex
