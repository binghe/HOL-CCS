%%%% -*- Mode: LaTeX -*-
%%
%% This version is for Elsevier's Information and Computation Journal submission.
%% by Davide Sangiorgi and Chun Tian.

\documentclass[3p,preprint]{elsarticle}

\usepackage{lineno,hyperref}
\modulolinenumbers[5]

\usepackage{hyperref}

\journal{Journal of Information and Computation}

% use \iflong to extend this paper from the previous EXPRESS/SOS paper.
\newif\iflong\longfalse

\usepackage{breakurl}

\usepackage[T1]{fontenc}
\usepackage[utf8]{inputenc} % no need for XeTeX, which always uses UTF-8

%% Math symbol packages
\usepackage{amsmath}
\usepackage{amssymb}
\usepackage{mathrsfs}

\usepackage{stmaryrd}
\SetSymbolFont{stmry}{bold}{U}{stmry}{m}{n}
% \cupdot (the best one)
\newcommand{\cupdot}{\mathbin{\mathaccent\cdot\cup}}

\usepackage{amsthm}
%\usepackage{newtxmath} % must be after amsthm (but can't use in this paper)

\newtheorem{definition}{Definition}[section]
\newtheorem{example}[definition]{Example}
\newtheorem{lemma}[definition]{Lemma}
\newtheorem{theorem}[definition]{Theorem}
\newtheorem{corollary}[definition]{Corollary}
\newtheorem{proposition}[definition]{Proposition} 
\newtheorem{remark}[definition]{Remark}

% HOL theorem embedding support
\usepackage{holindex}
\usepackage{alltt}
\usepackage{proof}
%% TeX commands needed for generating terms and theorems of our CCS theories:

\newcommand{\HOLTokenStrongEQ}{$\sim{}$}
\newcommand{\HOLTokenWeakEQ}{$\approx{}$}
\newcommand{\HOLTokenObsCongr}{$\approx^{\mathrm{c}}\!$}
\newcommand{\HOLTokenEPS}{$\overset{\epsilon}{\Rightarrow}$}
\newcommand{\HOLTokenTransBegin}{$-$}
\newcommand{\HOLTokenTransEnd}{$\rightarrow$}
\newcommand{\HOLTokenWeakTransBegin}{$=$}
\newcommand{\HOLTokenWeakTransEnd}{$\Rightarrow$}
\newcommand{\HOLTokenExpands}{$\succeq_{\mathrm{e}}\!$}
\newcommand{\HOLTokenContracts}{$\succeq_{\mathrm{bis}}\!$}
\newcommand{\HOLTokenObsContracts}{$\succeq^{\mathrm{c}}_{\mathrm{bis}}\!$}
%\renewcommand{\HOLTokenImp}{\ensuremath{\Longrightarrow}}

% Experimental environments
\usepackage{environ}
\NewEnviron{HOLTrans}{\overset{\BODY}{\longrightarrow}}
\NewEnviron{HOLWeakTrans}{\overset{\BODY}{\Longrightarrow}}


\usepackage{listings}
\renewcommand{\ttdefault}{cmtt} % {pcr}
\lstset{tabsize=8,language=ML,basicstyle=\small\ttfamily\bfseries,
                keywordstyle=\color{blue}\ttfamily,
                stringstyle=\color{red}\ttfamily,
                commentstyle=\color{green}\ttfamily,
                morecomment=[l][\color{magenta}]{\#}}

%
% LaTeX math formula spacing:
%
% \enskip	leave a horizontal space of respectively half an em
% \quad		space equal to the current font size (= 18 mu), one em
% \qquad	twice of \quad (= 36 mu), i.e. two ems
% \enspace	it's inherited from Plain TeX and is almost the same
%                      as \enskip, but technically it is a kern, rather than a skip.
% \,	3/18 of \quad (= 3 mu)
% \:	4/18 of \quad (= 4 mu)
% \;	5/18 of \quad (= 5 mu)
% \!	-3/18 of \quad (= -3 mu)
% \ (space after backslash!) equivalent of space in normal text

\usepackage{graphicx}
\usepackage[all,cmtip]{xy}
\usepackage{color}

% hightlighting changed texts with \hl{}
\usepackage{soul}
% or disable the hightlighting on final submission
% \newcommand*\hl{}




% NAMES and VARIABLES

\def\Names{{\cal N}}            % set of all names
\def\fn#1{\rmsf{fn}(#1)}         % free names
\def\fv#1{\rmsf{fv}(#1)}         % free variables
\def\bv#1{\rmsf{bv}(#1)}         % bound variables
\def\bn#1{\rmsf{bn}(#1)}         % bound names

\newcommand{\dom}[1]{{\rmsf{dom}}(#1)} % the domain of something  

% FOR PROCESSES 

\def\nil{{\boldsymbol 0}} % nil
\def\res#1{{\boldsymbol \nu} #1\:}   % restriction
%% the following definitions allow us to use the symbols ! . and | 
%directly, for the replication, prefix and parallel compoisition
%operators in math mode 
\mathcode`\!="4021 % `!' as prefix operator
\mathcode`\.="602E  % prefix 
\mathcode`\|="326A % `|' as relation operator

\newcommand{\outC}[1]{\overline{#1}}      % CCS  output
\newcommand{\inpC}[1]{#1}                 % CCS  input

\newcommand{\out}[2]{\overline{#1}\langle{#2}\rangle} % output with value    
\newcommand{\inp}[2]{#1(#2)}  % input with value
\newcommand{\inpW}[2]{#1(#2). }  % input with value and dot

\newcommand{\iae}[2]{{#1}\langle{#2}\rangle} % input with value    


\newcommand{\cond}[3]{\myif\ #1\ \mythen\ #2\ \myelse\ #3} % if-then-else  
\newcommand{\myif}{\myspace{\rmtt{if}}\myspace}            % ``if''
\newcommand{\mythen}{\myspace{\rmtt{then}}\myspace}        % ``then''
\newcommand{\myelse}{\myspace{\rmtt{else}}\myspace}        % ``else''

\newcommand{\true}{\rmsf{true}} %boolean true
\newcommand{\false}{\rmsf{false}} %boolean false


% substitutions  (used as a postfixed operator)
\def\sub#1#2{\{\raisebox{.5ex}{\small$#1$}\! / \!\mbox{\small$#2$}\}} 



% TRANSITIONS (arrows)

\newcommand{\racap}{\mathrel{\stackrel{{\;\; {{\wedge}} \;\;}}{\mbox{\rightarrowfill}}}} 

\newcommand{\arr}[1]{\mathrel{\stackrel{{\;\;#1\;\;}}{\mbox{\rightarrowfill}}}}
                                %  strong labelled transition 

\newcommand{\Arr}[1]{\mathrel{\stackrel{{\;\;#1\;\;}}{\mbox{\rightarrowfillWEAK}}}} 
                                    %weak  labelled transitions
                                    
\newcommand{\arcap}[1]{\mathrel{\stackrel{{\;\; {\widehat{#1}} \;\;}}{\mbox{\rightarrowfill}}}} 
                                    %strong labelled transitions with hat
\newcommand{\Arcap}[1]{\mathrel{\stackrel{{\;\;{\widehat{#1}}\;\;}}{\mbox{\rightarrowfillWEAK}}}}
                                    %weak labelled transitions with   hat
                                    
% FOR CONTEXTS

\newcommand{\contexthole}{ [ \cdot  ] }      %hole of a context
\newcommand{\ct}[1]{ C \brac{#1} }   %filled context
\newcommand{\qct}{ C  }              %unfilled context  
\newcommand{\brac}[1]{[#1] }   % brackets for the context hole



%% SOME STYLE COMMANDS
\newcommand{\rmtt}[1]{{\rm\tt{#1}}} % for keywords like ``if'', ``then'' ... 
\newcommand{\rmsf}[1]{{{\rm\sf{#1}}}}


%BEHAVIOURAL  EQUIVALENCES and relations

\def\R{{\cal R}}          % R without spaces around
\def\RR{\mathrel{\cal R}} % R with some space around
\def\S{{\cal S}}          % S without spaces around
\def\SS{\mathrel{\cal S}} % S with some space around


\newcommand{\equival}{=} %\mathrel{\equiv_\alpha} 
                          % equality up to alpha conversion 





%SPECIAL SYMBOLS


\def\midd{\; \; \mbox{\Large{$\mid$}}\;\;}
               %separation symbol in  grammars               

\def\st{\; \mid \;} % ``such that'' in formulas
\def\DSdefi{\stackrel{\mbox{\scriptsize def}}{=}} % definition equal

\def\Defi{\stackrel{\mbox{\scriptsize $\triangle$}}{=}} % definition equal


% \def\qed{}
%  {\unskip\nobreak\hfil\penalty50\hskip1em\null\nobreak\hfil
%   $\Box$\parfillskip=0pt\finalhyphendemerits=0\endgraf}
%                    % end of proofs (or theorems,results without proofs)

\newcommand{\myspace}{\:}  % some spacing abbreviation

% rename tilde to widetilde, to be used for tuples
\renewcommand{\tilde}{\widetilde}


% % environment for proofs (for CUP style file)
% \newenvironment{proof}{\noindent {\bf Proof }}{\qed \bigskip}


\newcommand{\finish}[1]{ \vskip .2cm  {\bf #1} \vskip .2cm   \marginpar{{\bf $DS$}}}

\newcommand{\recu}[2]{{\tt rec}\: #1 . #2}

\newcommand{\rapprox}{\mathrel{\approx^{\rm{c}}}}


% NEW THINGS 

\newcommand{\mylabel}[1]{{\rm (#1).}}
\newcommand{\MYsketch}{[Sketch] }


\newcommand{\behav}{equation}
\newcommand{\behavC}{contraction}


\newcommand{\hb}{\hskip .5cm}

\newcommand{\ArrN}[2]{\mathrel{\stackrel{{\;\;#1\;\;}}{\mbox{\rightarrowfillWEAK}}_{#2}
  }} 
                                    %weak  labelled transitions weighted

\newcommand{\Var}{{\cal X}}

\newcommand{\AL}{{\cal RL}}
\newcommand{\PL}{{\cal L}}

\newcommand{\sign}{\Sigma}
\newcommand{\prsign}{\pr_\sign}


% possible actions in Abnients:
\newcommand{\qina}{{\ccc{in}} } 
\newcommand{\qout}{{\ccc{out}} }
\newcommand{\qopen}{{\ccc{open}} }
\newcommand{\capa}{{\ccc{capa}} }
\newcommand{\amb}[2]{#1 [\, #2 \,] } % ambients

\newcommand{\HOAMB}{\mbox{\rm{HO$\pi$Amb}}}
\newcommand{\SeqCCS}{\mbox{\rm{SeqCCS}}}


\newcommand{\SE}{{E\!S}}
\newcommand{\SL}{{L\!S}}
\newcommand{\SEp}{{E\!S'}}
\newcommand{\SEpU}{{E\!S'_1}}


%\newcommand{\mcontrP}[1]{\mathrel{\stackrel{{\footnotesize{\mbox{$\succ$}}}}{\footnotesize{\mbox{$#1$}}}}}
\newcommand{\mexpaP}[1]{\mathrel{\stackrel{{\footnotesize{\mbox{$\prec$}}}}{\footnotesize{\mbox{$#1$}}}}}

\newcommand{\wc}{\mathrel{\approx^{\rm c}}}
\newcommand{\wb}{\approx}
\newcommand{\contr}{\mathrel{\succeq_{\rm{bis}}}}
\newcommand{\expa}{\mathrel{\succeq_{\rm{e}}}}


\newcommand{\mcontr}{\mathrel{\succeq}}
\newcommand{\mexpa}{\mathrel{\preceq}}

\newcommand{\mcontrmay}{\mathrel{\succeq_{\rm{ctx}}}}
\newcommand{\mexpamay}{\mathrel{\preceq_{\rm{ctx}}}}


\newcommand{\mcontrTE}{\mathrel{\succeq_{\rm{tr}}}}
\newcommand{\TE}{\approx_{\rm tr}}       %trace equivalence


\newcommand{\mcontrBIS}{\mathrel{\succeq_{\rm{bis}}}}
\newcommand{\mexpaBIS}{\mathrel{\preceq_{\rm{bis}}}}


\newcommand{\rcontr}{\mathrel{\succeq^{\rm{c}}_{\rm{bis}}}}
\newcommand{\rexpa}{\mathrel{\preceq^{\rm{c}}_{\rm{bis}}}}


\newcommand{\til}{\tilde}

\newcommand{\ctx}[1]{#1^{\rm{c}}}

\newcommand{\Dwaleq}[1]{\Dwa^{\leq #1}}
\newcommand{\Dwageq}[1]{\Dwa^{\geq #1}}


\newcommand{\holeDS}{ [ \cdot  ] }
\newcommand{\holei}[1]{[\cdot]_{#1}}
\newcommand{\ctp}[1]{ C' \brac{#1} }  %primed context       
\newcommand{\qctp}{ C'  }
\newcommand{\ctpp}[1]{ C'' \brac{#1} }  %primed context       
\newcommand{\ctppp}[1]{ C''' \brac{#1} }  %primed context       

\newcommand{\qctpp}{ C''  }
\newcommand{\qctppp}{ C'''  }


\newcommand{\may}{\approx_{\rm ctx}}       %may equivalence

\newcommand{\mayHA}{\approx^{\rm{HAmb}}_{\rm ctx}}       %may equivalence

\newcommand{\hk}{\hskip .2cm }
\newcommand{\mysp}{10pt}
\newcommand{\tkp}{10pt}
\newcommand{\tkpS}{6pt}
\newcommand{\tkpSS}{3pt}
\newcommand{\tkpP}{15pt}

\newcommand{\smay}{\mathrel{\sim_{\rm ctx}}}       %may equivalence
\newcommand{\smayHA}{\mathrel{\sim^{\rm{HAmb}}_{\rm ctx}}}       %may equivalence

\newcommand{\holE}{\contexthole}  % hole

\newcommand{\murule}{\fortherules\mu} % mu rule of $\lambda$-calculus
\newcommand{\nurule}{\fortherules\nu} % nu rule of $\lambda$-calculus
\newcommand{\nuvrule}{\fortherules\nuv} % nuv rule of $\lambda$-calculus
\newcommand{\xirule}{\fortherules\xi} % xi rule of $\lambda$-calculus
\newcommand{\betarule}{\fortherules\beta} %beta rule of $\lambda$-calculus
\newcommand{\betavrule}{\fortherules\betav} % beta_v rule of $\lambda$-calculus
\newcommand{\etarule}{\fortherules\eta} % eta rule of $\lambda$-calculus
\newcommand{\nuv}{\nu_{\myrm v}} %beta rule of $\lambda$-calculus
\newcommand{\betav}{\mbox{$\beta_{\myrm{v}}$}} %beta rule of $\lambda$-calculus
\newcommand{\alpharule}{\fortherules\alpha} %alpha rule 
\newcommand{\fortherules}[1]{\mbox{$#1$}} %auxiliary def for the rules



\newcommand{\barbedbis}
{\mathrel{\stackrel{\bfcdotB}{\approx}}}

%OLD:
%{\mbox{ $\approx \! \! \! \!\! \!\!        
%\raisebox{1.15ex}[0ex][0ex]{\bfcdot} \; \,$}}



\newcommand{\wbb}{\mathrel{\approx_{\rm{bar}}}}
\newcommand{\wbc}{\mathrel{\approx^{\rm{c}}_{\rm{bar}}}}
% \newcommand{\wbb}{\mathrel{\barbedbis}}
% \newcommand{\wbc}{\cong}


\newcommand{\bfcdotB}{ {\mbox{\boldmath $.$}}  }         

\newcommand{\bcontra}
{\mathrel{\mcontr_{\rm{bar}}}}
%{\mathrel{\stackrel{\bfcdotB}{\mcontr}}}

\newcommand{\cbc}
{\mathrel{\mcontr^{\rm{c}}_{\rm{bar}}}}
%{\mathrel{{\mcontr_{\rm{bc}}}}}


\newcommand{\mypt}{2pt} 

% --------------



\newenvironment{myquote}
               {\list{}{\rightmargin\leftmargin}%
                \item\relax}
               {\endlist}


% \newenvironment{proofEx}{
% \begin{myquote}
% %\trivlist\parindent=0pt
% %      \item[\hskip \labelsep{\bf Answer: }]}
% \noindent %{\bf Answer to}
% }
% {\qed%\endtrivlist
% \end{myquote}}

%SPECIAL SYMBOLS

\newcommand{\EXX}[1]{{\bf Exercise~\ref{#1}}} % for answers to exercises 
\newcommand{\EXXpa}[2]{{\bf Exercise~\ref{#1}(#2)}} % for answers to exercises 


\newcommand{\Mybar}{\hrulefill} % separation rule

% \def\finish#1{\vskip.2cm\noindent{\em #1}%
%   \marginpar{$\longleftarrow$}\vskip.2cm}

\newcommand{\spaceD}{\,}


\newcommand{\bulletD}{\diamond}
%\mathrel{\lozenge} %\bowtie %blacktriangle %\minuso %\blacktriangleup %filleddiamond %\blackdiamond}

\newcommand{\ccc}{\rmtt} % {\rm\tt{#1}}}  %{\mbox{{\tt #1}}}
\newcommand{\cccTT}{\tt} % {\rm\tt{#1}}}  %{\mbox{{\tt #1}}}

\newcommand{\rr}{\RR}  
\newcommand{\Id}{{\cal I}} % identity relation  


\newcommand{\Prop}{{\cal P}} %
\newcommand{\FF}{F} % a function
\newcommand{\FFbis}{F_{\sim}} 
\newcommand{\FFbisW}{F_{\approx}} 
\newcommand{\finLISTS}{\mbox{{\tt FinLists}$_{A}$}} %   
\newcommand{\fininfLISTS}{\mbox{{\tt FinInfLists}$_{A}$}} %   
\newcommand{\finLISTSi}[1]{\mbox{\tt FinLists}_{#1}} %   
\newcommand{\fininfLISTSi}[1]{\mbox{\tt FinInfLists}_{#1}} %   
\newcommand{\nilLISTS}{\mbox{\tt nil}} %   
\newcommand{\mapLISTS}[2]{\mbox{\tt map}\: #1\: #2 } %   
\newcommand{\qmapLISTS}{\mbox{\tt map}} %   
\newcommand{\iterate}[2]{\mbox{\tt iterate}\: #1\: #2 } %   
\newcommand{\qiterate}{\mbox{\tt iterate}} %   
\newcommand{\qnats}{\mbox{\tt nats}} %   
\newcommand{\qfrom}{\mbox{\tt from}\: } %   
\newcommand{\fibs}{\mbox{\tt fibs}} %   
\newcommand{\qplus}{\mbox{\tt plus}\:} %   
\newcommand{\qtail}{\mbox{\tt tail}\,} %   
\newcommand{\plusU}{+_1} %   
\newcommand{\FFlist}{\Phi_{A\tt list}} 

\newcommand{\consLISTS}{\mbox{\tt cons}} %   
\newcommand{\consLISTSnew}[2]{\langle #1\rangle \bullet #2} %   
\newcommand{\consLISTSnewB}[2]{ #1 \bullet #2} %   


\newcommand{\simList}{\sim}  %_{A\tt list}}} % bisimilarity on lists 


\newcommand{\tree}[1]{\mbox{{\tt Tree}$(#1)$}} %   
%\newcommand{\root}[1]{\mbox{{\tt root}$(#1)$}} %   
\newcommand{\T}{{\cal T}}
\newcommand{\Vp}{{\tt V}}
\newcommand{\Rp}{{\tt R}}
\newcommand{\Gin}[2]{{\cal G}^{\tt ind}(#1,#2)} %   
\newcommand{\Gco}[2]{{\cal G}^{\tt coind}(#1,#2)} %   



\newcommand{\pws}[1]{\wp (#1)} % powerset
\newcommand{\lfp}[1]{\qqlfp(#1)} % least fixed point 
\newcommand{\gfp}[1]{\qqgfp(#1)} % greatest fixed point
\newcommand{\qqlfp}{{\tt lfp}} % least fixed point abbrv.
\newcommand{\qqgfp}{{\tt gfp}} % greatest fixed point abbrv.
\newcommand{\qlfp}{least fixed point}
\newcommand{\qgfp}{greatest fixed point} 


\newcommand{\Fcoin}{F_{\tt coind}} % coind. def. set
\newcommand{\Fin}{F_{\tt ind}}     % ind. def. set

\newcommand{\Lao}{\Lambda^0}  %closed $\lambda$-terms



  %convergence

\newcommand{\Dwa}{\Downarrow}           % plain convergence
\newcommand{\DwaP}[2]{#1 \Downarrow #2} % plain convergence, in rules


\newcommand{\EQsin}[2]{#1 = #2} % syn. equality in rules

\newcommand{\dwa}{\downarrow}           % plain convergence
\newcommand{\Up}{\Uparrow}   % divergence
\newcommand{\UpP}{\Uparrow}   % divergence in rules

\newcommand{\Reach}{\Downarrow}           % reachability
\newcommand{\Termi}{\downharpoonright}        % can terminate
\newcommand{\TermiP}{\Termi}        % can terminate, in rules
\newcommand{\Adiv}{\upharpoonright_\mu}           % \mu-divergence
\newcommand{\Adiva}{\upharpoonright_a}           % \mu-divergence

\newcommand{\LL}{{\cal L}}  % set of terms
\newcommand{\States}[1]{{\tt St}^{#1}} %states of an LTS


\newcommand{\myemptyItem}{\mbox{$ $ }} % utile per il xy package
\newcommand{\NONemptyItem}[1]{\mbox{$#1$ }} % utile per il xy package

\newcommand{\LongrightarrowN}[1]{\Longrightarrow_{#1}}


% for imperative programs
\newcommand{\XX}{\spaceD{\ccc{X}}}
\newcommand{\YY}{\spaceD{\ccc{Y}}}

% for LTSs
\newcommand{\Act}{\mbox{\it Act}} 
\newcommand{\pr}{\mbox{\it Pr}} % \mbox{\it Pr}}  %{{\mathbb P}} %{{\cal P}r} 
\newcommand{\power}{\wp} 
\renewcommand{\Pr}{\pr} % \mbox{\it Pr}}  %{{\mathbb P}} %{{\cal P}r} 


% membership stuff among relations
\newcommand{\memb}[3]{ #1 #3 #2 } 
\newcommand{\rmemb}[3]{ {(#1 , #3)} \in { #2} } 
%\newcommand{\Rmemb}[3]{ \tobr{#1 , #3} \in  #2 } 

% symbols
%\newcommand{\vv}{P} 
%\newcommand{\ww}{Q} 
\newcommand{\pp}{P} 
\newcommand{\qq}{Q} 

\newcommand{\rmm}[1]{\mbox{\rm #1}} %labels of transitions in figure 


%% for inference rules


\newcommand{\infrule}[3]{\[
{\trans{#1}\quadrule \displaystyle{#2 \over #3} } %\\[10pt]
\]}
\newcommand{\infruleSIDE}[4]{\[
{\trans{#1}\quadrule\displaystyle{#2 \over #3}\;\; #4 } %\\[10pt]
\]}  % inf rule with a side condition
\newcommand{\shortinfrule}[3]{ {\trans{#1}} \quadrule
     \displaystyle{#2 \over #3}}
\newcommand{\shortinfruleSIDE}[4]{ {\trans{#1}} \quadrule
     \displaystyle{#2 \over #3}\;\; #4}

\newcommand{\shortaxiom}[2]{{\trans{#1}}\quadrule
\displaystyle{ \over #2}}

\newcommand{\myinf}[3]{{\rn{#1}}\quadrule \displaystyle{#2 \over #3} }
    % for  plain  inference rules

\def\trans#1{\rn{#1}}   % for the names of transition rules
\newcommand{\rn}[1]{%
  \ifmmode 
    \mathchoice
      {\mbox{\sc #1}}
      {\mbox{\sc #1}}
      {\mbox{\small\sc #1}}
      {\mbox{\tiny\uppercase{#1}}}%
  \else
    {\sc #1}%
  \fi}

\newcommand{\quadrule}{\hskip .2cm }

\newcommand{\andalso}{\quad\quad}


% references

\def\reff#1{(\ref{#1})}       %references between brackets



\newcommand{\enco}[1]{[\! [ #1 ] \! ]  }


\newcommand{\mydots}{,\ldots,}

% for not\sim_n

\newcommand{\notsimN}[1]{\mathrel{\not\!{\sim_{#1}}} }



\newcommand{\cti}[2]{ C_{#1} \brac{#2} }   %filled context
\newcommand{\ctD}[1]{ D \brac{#1} }   %filled context
\newcommand{\qcti}[1]{ C_{#1}  }   % context

\newcommand{\ctDp}[1]{ D' \brac{#1} }   %filled context
\newcommand{\qctD}{D}   % context
\newcommand{\qctDp}{D'}   % context

%\newcommand{\qctp}{C'}   % context


\newcommand{\beginlongtable}{
 \begin{longtable}{l@{\extracolsep{\fill}}p{76mm}@{\extracolsep{\fill}}r}
%{|l@{\extracolsep{\fill}}p{80mm}@{\extracolsep{\fill}}r|}
%% READ THIS !!!
%% ho cancellato sotto altrimenti mi fa una entry nella list of tables
%\caption{ffff} \\
%\hline
%symbol          & description                & page \\
%\hline
}
\newcommand{\ENDlongtable}{% \hline
\end{longtable}}

\newcommand{\GLSbeg}[1]{\noindent %\underline
{\bf \large #1}}

\newcommand{\GLS}[1]{
\multicolumn{3}{l}{%\noindent 
%\underline
{\bf \large #1}}\\ }
\newcommand{\GLSb}[1]{
\multicolumn{3}{l}{%\noindent 
%\underline
{\it \large #1}}\\ }

\newcommand{\beginlongtableIN}{\\}
\newcommand{\ENDlongtableIN}{\\}

% % to add or remove a line to a page use these 
% \newcommand{\longpage}{\enlargethispage{\baselineskip}} 
% \newcommand{\shortpage}{\enlargethispage{-\baselineskip}} 





\usepackage{DSarrow} % this is something to use extensible arrows in transitions

% \usepackage{pgf,tikz}
% \usetikzlibrary{arrows}

%% `Elsevier LaTeX' style
\bibliographystyle{elsarticle-num}
%%%%%%%%%%%%%%%%%%%%%%%

% PDF meta data (must be added manually, maybe removed by proceeding editor)
%\hypersetup{pdftitle={Unique Solutions of Contractions, CCS, and their HOL Formalisation},
%	pdfauthor={Chun Tian; Davide Sangiorgi},
%	pdfsubject={Concurrency Theory},
%	pdfkeywords={theorem proving, process calculi, coinduction, unique solution of equations, congruence}}

\begin{document}
\begin{frontmatter}

% title
\title{Unique Solutions of Contractions, CCS, and their HOL
  Formalisation\tnoteref{mytitlenote}}
\tnotetext[mytitlenote]{An extended version of the EXPRESS/SOS 2018
  workshop paper \cite{EPTCS276.10}.}

% authors
\author[mymainaddress]{Chun Tian}%\fnref{myfootnote}
\address[mymainaddress]{Universit\`a di Trento and Fondazione Bruno
  Kessler, Italy}
\ead{chun.tian@unitn.it}
%\fntext[myfootnote]{Part of this work was carried out when the first
%  author was studying at Universit\`a di Bologna.}

\author[mysecondaryaddress]{Davide Sangiorgi}
\ead{davide.sangiorgi@unibo.it}
\address[mysecondaryaddress]{Universit\`a di Bologna and INRIA, Italy}

\begin{abstract}
  The unique solution of contractions is a proof technique for
  bisimilarity that overcomes certain syntactic constraints of
  Milner's ``unique solution of equations'' technique.  The paper
  presents an overview of a rather comprehensive formalisation of the
  core of the theory of CCS in the HOL theorem prover (HOL4), with a
  focus towards the theory of unique solutions of contractions.  (The
  formalisation consists of about 20,000 lines of proof scripts in
  Standard ML.)  Some refinements of the theory itself are obtained.
  In particular we remove the constraints on summation,
  which must be weakly-guarded, by moving to \emph{rooted
  contraction}. We prove that 
 rooted contraction is indeed the coarsest
  precongruence contained in the contraction preorder, in the same way
  as  rooted bisimilarity is  so for weak bisimilarity.
  % The unique solution of contractions is a proof technique for
  % bisimilarity that overcomes certain syntactic constraints of
  % Milner's ``unique solution of equations'' technique, \hl{such that the
  % theorems for weak bisimilarity and its rooted version require that
  % the involved equations must be guarded and sequential. By replacing
  % equations with special inequations called \emph{contraction}, Sangiorgi
  % has proven the ``unique solution of contractions'' theorem for weak
  % bisimialarity, which requires only weakly guarded contractions.}

  % This paper presents a rather comprehensive formalisation of the
  % core theory of CCS in the HOL theorem prover (HOL4), with a
  % focus towards the theory of unique solutions of equations and contractions.  (The
  % formalisation consists of about 20,000 lines of proof scripts.)
  % Some refinements of the theory itself are also obtained.
  % In particular we remove the constraints on summation,
  % \hl{which must be weakly guarded since the contraction
  % preorder is not a precongruence under direct sums,} by moving to \emph{rooted
  % contraction}. \hl{Using rooted contraction one obtains an enhanced
  % unique-solution theorem which is valid for rooted bisimilarity
  % (hence also for bisimilarity itself) while requires (true) weak
  % guardness with arbitrary sums.
  % It is also prove that, the rooted contraction is the coarsest
  % precongruence contained in the contraction preorder (thus is the best possible one),
  % as rooted bisimilarity is the coarsest congruence contained
  % in (weak) bisimilarity.}
\end{abstract}

\begin{keyword}
process calculi \sep theorem provers \sep coinduction \sep unique
solution of equations \sep congruence
\end{keyword}

\end{frontmatter}

\linenumbers

% part 1 (Sangiorgi)
\section{Introduction}

A prominent proof method for bisimulation, put forward by Milner and widely used in his
landmark CCS book \cite{Mil89} is the
\emph{unique solution of equations}, whereby two tuples of processes are
componentwise bisimilar if they are solutions 
of the same system of equations.
This method  is important in verification techniques and tools
based on algebraic reasoning \cite{theoryAndPractice,RosUnder10,BaeBOOK}. 

In the   \emph{weak} case (when  behavioural equivalences abstract from internal moves,
which practically is the most relevant case), however, 
Milner's proof method has severe syntactical limitations. 
To overcome such limitations, Sangiorgi proposes to replace
equations with  special inequations called
\emph{contractions} \cite{sangiorgi2015equations}. Contraction is a
preorder that, roughly, places some efficiency
constraints on processes.  Uniqueness of the solutions of a system of contractions
 is defined as with systems of equations:  
any two solutions must be bisimilar.
The difference with equations is in the meaning of solution:
in the case of contractions
the solution is evaluated with respect to
the contraction preorder, rather than bisimilarity. 
With contractions, most syntactic limitations of the unique-solution theorem can be
removed.  One constraint that still remains in
\cite{sangiorgi2015equations} is on occurrences of the sum operator,
due to the failure of substitutivity of contraction w.r.t. such operator.


The main goal  of the work described in this paper is 
a rather  
 comprehensive formalisation  of the core of the theory of CCS 
 in the HOL
theorem prover (HOL4),  with a focus on the theory of unique solutions of contractions.
The formalisation however is not confined to the theory of  unique solutions, but embraces 
most of the 
core of the theory of CCS \cite{Mil89}
(partly because the theory of unique solutions
 relies on a number of more fundamental results):
indeed the formalisation encompasses the basic properties of strong and weak
bisimilarity (e.g. the fixed-point and substitutivity properties), 
their algebraic theory, various versions of ``bisimulation up to''
techniques (e.g., bisimulation up-to bisimilarity),
the main properties  of the rooted bisimilarity (the congruence induced by weak
bisimilarity, also called observation congruence). Concerning rooed bisimilarity, the formalisation
includes Hennessy and Deng lemmas, and two proofs that rooted bisimilarity is the largest
congruence included in bisimilarity:
one as in Milner's
book,  requiring the hypothesis that  no processes can use all labels; the other without
such hypothesis, essentially formalising van Glabbek's paper \cite{vanGlabbeek:2005ur}
(such proof however follows the structure of the ordinal numbers, which cannot be handled
in HOL, and therefore it is restricted to finite-state processes ).
Similar theorems are proved for the rooted contraction preorder.

  
In this respect,  the work is an extensive experiment with the use of the HOL theorem prover and its
most recent developments, including a package  for expressing coinductive definitions.
The
work consists of about 20,000 lines of proof scripts in Standard ML.



From the CCS theory viewpoint, the formalisation has offered us the possibility of 
further refining the theory of unique solutions. 
In particular, the existing theory  placed limitations on the body of the contractions due to the
substitutivity problems of weak bisimilarity and other behavioural relations with respect
to the sum operator.  
We have thus refined the proof  technique based on contractions by moving to the 
\emph{rooted contraction}, that is, the coarsest (pre)congruence contained in the contraction
preorder.  Using rooted contraction one obtains a unique-solution theorem that is valid for
\emph{rooted bisimilarity} (hence also for bisimilarity itself), and without syntactic
constraints on the occurrences of sums.   

\finish{ I had to remove the reference \cite{Tian:2017wrba} here since it would further
  weaken this paper. } 

Another advantage of the formalisation is 
that we can take advantage of results about different 
equivalences or preorders that share a similar  proof structure. 
Examples are: the results that rooted bisimilarity and rooted contraction are,
respectively, the coarsest congruence contained in weak bisimilarity 
and the coarsest precongruence contained in the contraction  preorder; 
the result about unique solution of equations for weak bisimilarity that uses the
contraction preorder as an auxiliary relation, and other unique solution results (e.g., 
the one for rooted in which
the auxiliary relation is rooted contraction); various forms of enhancement of the bisimulation
proof method (the `up-to' techniques).  
In these cases, there are only a few places in which the HOL scripts have to be modified.
Then the succesful termination of the scripts  gives us a guarantee that the proof is
completed,  removing the risk 
of overlooking or missing details as in hand-written proofs. 


% to describe
% The purpose of this paper is twofold. 
% On the one hand, 
% On the other hand, we provide a  
%  comprehensive formalisation  of the core of the theory of CCS 
%  in the HOL
% theorem prover (HOL4). The formalisation  includes the proofs of
% Milner's 3 ``unique solution of equations'' theorems and
% contractions discussed in the present paper, but is not limited to it (partly because such
% theorems rely on a number of more fundamental results):
% indeed the formalisation encompasses the basic properties of strong and weak
% bisimilarity (e.g. the fixed-point and substitutivity properties), 
% their algebraic theory, various versions of ``bisimulation up to''
% techniques (e.g., bisimulation up-to expansion),
% the basic properties  of rooted bisimilarity. 
% Thus the work is an extensive experiment with the use of the HOL theorem provers and its
% most recent developments, including a facility  for expressing coinductive definitions.

% % Considering the relationship between bisimilarity and rooted
% % bisimilarity, the formalisation includes the proof that the latter is the coarsest
% % congruence included in the former, for which two proofs are formalized: one as in Milner's
% % book,  requiring the hypothesis that  no processes can use all labels; the other without
% % such hypothesis, essentially formalising van Glabbek's paper \cite{vanGlabbeek:2005ur}.
% % Similar theorems are proved for rooted contractions wrt the contraction preorder. 


\paragraph{Structure of the paper}....
% \section{Background}
% \label{s:back}

\section{CCS}
\label{ss:ccs}


We assume  a possibly infinite set of \emph{names} $\mathscr{L} = \{a, b,
\ldots\}$ forming input and output actions, plus a special invisible
action $\tau$ not in $\mathscr{L}$, and a set of variables $A,B,
\ldots$ for defining recursive behaviours.
The class  of the CCS processes is inductively defined from $\nil$ by the operators
of prefixing, parallel composition, sum (binary choice), restriction, recursion and \hl{relabeling}:
\begin{equation*}
\begin{array}{ccl}
\mu  & := &  \tau \hspace{.3pt} \; \midd \; a  \; \midd \;  \outC a  \\
P  & := &  \nil \; \midd \;  \mu . P \; \midd \;  P_1 |  P_2 \; \midd  \;
P_1 + P_2 \; \midd % \; \mu . P\; \midd  \; 
  (\res a\!)\, P  \;  \midd \;  A \; \midd \; \recu A  P
\; \midd \; P\; [r\!\!f]  % relabelling
\end{array}
\end{equation*}
%We sometimes omit trailing $\nil$, e.g., writing $a|b$ for $a.\nil |b .\nil $ .
The operational semantics of CCS is given by means of
a Labeled Transition System (LTS), shown in Fig.~\ref{f:LTSCCS} as SOS
rules (the symmetric version of the two rules for
parallel composition and the rule for sum are omitted).
A CCS expression uses only \emph{weakly-guarded sums} if all occurrences of
the sum operator are of the form $\mu_1.P_1 + \mu_2.P_2 + \ldots
+ \mu_n.P_n$, for some $n \geq 2$.
 The \emph{immediate derivatives} of a
process $P$ are the elements of the set $\{P' \st P \arr\mu P' \mbox{
  for some $\mu$}\}$.
% We use $\ell$ to range over
%  visible actions (i.e.~inputs or outputs, excluding  $\tau$).
\begin{figure*}
\begin{center}
\vskip .1cm
 $\displaystyle{  \over  \mu.  P    \arr\mu
P } $  $ \hb$   
\hskip .5cm
 $\displaystyle{   P \arr\mu   P' \over   P + Q   \arr\mu
P'  } $  $ \hb$   
\hskip .5cm
 $\displaystyle{   P \arr\mu   P' \over   P | Q   \arr\mu
P' | Q } $  $ \hb$   
\hskip .3cm
  $\; \;$  $\displaystyle{ P \arr{ a}P' \hk \hk  Q
\arr{\outC a }Q'  \over     P|  Q \arr{ \tau} P'
|  Q'  }$ 
\\
\vspace{.2cm}
$\displaystyle{ P \arr{\mu}P' \over
 (\res a\!)\, P   \arr{\mu} (\res a\!)\, P'} $ $ \mu \neq a, \outC a$
$ \hb$
%
$\displaystyle{ P \sub {\recu A P} A \arr{\mu}P' \over
 \recu A P   \arr{ \mu} P'  } $
\hskip .5cm  
$\displaystyle{ P \arr{\mu} P' \over
 P \;[r\!\!f] \arr{r\!\!f(\mu)} P' \;[r\!\!f]} $ $\forall a.\, r\!\!f(\outC a) = \overline{r\!\!f(a)}$
$ \hb$ %  &
\end{center}
\caption{\hl{Structural Operational Semantics} of CCS}
\label{f:LTSCCS}
\end{figure*}
Some standard notations for transitions:  $\Arr\epsilon$ is the 
reflexive and transitive closure of $\arr\tau$, and 
$\Arr \mu $ is $\Arr\epsilon \arr\mu \Arr\epsilon$ (the
composition of the three relations).
Moreover,   
$ 
P \arcap \mu P'$ holds if $P \arr\mu P'$ or ($\mu =\tau$ and
$P=P'$); similarly 
$ 
P \Arcap \mu P'$ holds if $P \Arr\mu P'$ or ($\mu =\tau$ and
$P=P'$).
We write $P \:(\arr\mu)^n P'$ if $P$ can become $P'$ after performing
$n$ $\mu$-transitions. Finally, $P \arr\mu$ holds if there is $P'$
with $P \arr\mu P'$, and similarly for other forms of transitions.




\paragraph{Further notations}
Letters  $\R$, $\S$ range over relations.
We use infix notation for relations, e.g., 
$P \RR Q$ means that $(P,Q) \in \R$.
We use a tilde to denote a tuple, with countably many elements; thus
the tuple may also be infinite.
 All
notations  are  extended to tuples componentwise;
e.g., $\til P \RR \til Q$ means that $P_i \RR Q_i$, for  each  
component $i$  of the tuples $\til P$ and $\til Q$.
And $\ct{\til P}$ is the process obtained by replacing each hole
$\holei i$ of the  context $\qct$ with $P_i$.  
We write $
\ctx \R$ for the closure of a relation under contexts. Thus $P\: \ctx \R\: Q$
means that there are context $\qct$ and tuples $\til P,\til Q$ with
$P =  \ct{\til P}, Q =  \ct{\til Q}$ and    $\til P \RR \til Q$.
We use  symbol 
$\DSdefi$ for abbreviations. For instance, $P \DSdefi G $, where
$G$ is some expression, means that  $P$ stands
for the  expression
$G$.
If $\leq$ is a preorder, then  $\geq$  is its inverse (and
conversely).


\subsection{Bisimilarity and rooted bisimilarity}
\label{ss:BiEx}

The equivalences we consider here are mainly \emph{weak} ones, in that they
abstract from the number of internal steps being performed:
\begin{definition}%[bisimilarity]
\label{d:wb}
A process relation ${\R}$ is a \textbf{bisimulation} if, whenever
 $P\RR Q$, \hl{for all $\mu\in \mathscr{L}\cup\{\tau\}$} we have:
\begin{enumerate}
\item $P \arr\mu P'$ implies that there is $Q'$ such that $Q \Arcap \mu Q'$ and $P' \RR Q'$;
\item $Q \arr\mu Q'$,implies that there is $P'$ such that $P \Arcap
  \mu P'$ and $P' \RR Q'$\enspace.
\end{enumerate}
 $P$ and $Q$ are \textbf{bisimilar},
written as $P \wb Q$, if $P \RR Q$ for some bisimulation $\R$.
\end{definition}

We sometimes call bisimilarity the \emph{weak} one, to
distinguish it from \emph{strong} bisimilarity ($\sim$),
obtained by replacing in the above definition   the weak answer $
Q\Arcap\mu Q'$ with the strong  $Q \arr \mu Q'$.
Weak bisimilarity is not preserved by the sum operator (except for
guarded sums). For this, Milner introduced \emph{observational congruence}, also called \emph{rooted
  bisimilarity} \cite{Gorrieri:2015jt,Sangiorgi:2011ut}:
\begin{definition}%[rooted bisimilarity]
\label{d:rootedBisimilarity}
Two processes $P$ and $Q$ are \textbf{rooted bisimilar}, written as $P
\rapprox Q$, iff \hl{for all $\mu\in \mathscr{L}\cup\{\tau\}$}
\begin{enumerate}
 \item  $P \arr\mu P'$ implies that there is $Q'$ such that $Q
   \Arr\mu Q'$ and $P' \wb Q'$;
 \item  $Q \arr\mu Q'$ implies that there is $P'$ such that $P
   \Arr\mu P'$ and $P' \wb Q'$\enspace.
\end{enumerate}
\end{definition}
% Besides reducing the rooted bisimiarity of two processes to
% the bisimilarities of their first-step transition ends, this definition also brings a proof technique for proving the
% rooted bisimiarity by constructing a bisimulation:
% \begin{lemma}{(Rooted bisimilarity by constructing a bisimulation)}
% \label{l:obsCongrByWeakBisim}
% Given a (weak) bisimulation $\RR$, if two processes $P$ and $Q$
% satisfies the following properties:
% \begin{enumerate}
%  \item  $P \arr\mu P'$ implies that there is $Q'$ such that $Q
%    \Arr\mu Q'$ and $P' \RR Q'$;
% \item the converse of (1) on the actions from $Q$.
% \end{enumerate}
% then $P$ and $Q$ are rooted bisimilar, i.e.~$P \approx^c Q$.
% \end{lemma}

\begin{theorem}
\label{t:rapproxCongruence}
$\rapprox$ is a congruence in CCS, and it is the
coarsest congruence contained in $\approx$ \cite{van2005characterisation}.
\end{theorem}

\subsection{Expansions}
\label{s:expa}

\hl{The bisimulation proof method can be enhanced by means of \emph{up-to
techniques}. One of the most useful auxiliary relations in up-to
techniques is the \emph{expansion} relation} $\expa$ \cite{arun1992efficiency,sangiorgi2015equations}.
This is an asymmetric version
of $\wb$ where $P \expa Q$ means that $P \wb Q$,
but also that $Q$ achieves the same as $P$
with no more work, i.e.~with no more $\tau$ actions.
Intuitively, if $P \expa Q$, we can think of $Q$ as being
at least as fast as $P$
or, more generally, we can think that $P$ uses at least as many resources as $Q$.
\begin{definition}%[expansion]
\label{d:expa}
A process relation ${\R}$
  is an \textbf{expansion} if, whenever
we have $P\RR Q$, for all $\mu$
 \begin{enumerate}
 \item   $P \arr\mu P'$ implies that there is $Q'$ with $Q \arcap \mu
   Q'$
  and $P' \RR Q'$;
 \item
     $Q \arr\mu Q'$   implies that there is $P'$ with $P \Arr \mu
  P'$ and $P'
 \RR Q'$.
 \end{enumerate}
  $P$  {\em expands} $Q$, written as
 $P  \expa Q$,
 if $P \RR Q$ for some expansion $\R$.
 \end{definition}

Same as bisimilarity, the expansion preorder is preserved by all operators but (direct) sums.

% next file: contraction.tex

\section{Equations and contractions}
\label{s:eq}

In the CCS syntax, 
a recursion $\recu A  P$ acts as a binder for $A$ in the body $P$. 
This gives rise, in the expected manner, to the notions of 
\emph{free} and \emph{bound} recursion variables in a CCS expression. 
For instance,  $X$ is free in $a.X + \recu Y (b.Y)$ while $Y$
is bound; \hl{And} in $a.X + \recu X (b.X)$, $X$ is both free and bound.
A \hl{term} without free variables is \hl{called} a \emph{process}.
% This setting does not cause any ambiguity, but sometimes leads to more
% complex proofs.} (see Section~\ref{sec:multivariate} for more details.)

In this paper (and the formalisation work), we use the agent
variables also as \emph{equation variables}. This eliminates the need of
another type for  equations, and we can reuse the existing
variable substitution operation (cf.~the SOS rule for the Recursion in
Fig.~\ref{f:LTSCCS}) for the substitution of equation variables.
For example, the result of substituting variable $X$ with $\nil$ in $a.X +
\recu X (b.X)$,  written $(a.X + \recu X (b.X)) \sub {\nil} X$, is
$a.\nil + \recu X (b.X)$ (with the part $\recu X (b.X)$
untouched). \Multivariate substitutions are written in the same syntax,
e.g. $E \sub {\til P} {\til X}$. Whenever $\til X$ is clear from the
context, we may also write $E[\til P]$ instead of $E \sub {\til P} {\til
  X}$ (and $E[P]$ for $E \sub P X$ if there is a single equation
variable $X$). 

% In fact, free agent variables have the same transitional behavior
% as the deadlock $\nil$ according to the SOS rules, as there is
% no rule for their transitions at all. In fact, most CCS theorems still hold if
% the involved CCS terms contain free variables. (The most notable
% exceptions are all versions of unique solution of equations/contracts,
% where all solutions must be \emph{pure} processes (i.e. no free variable).)
 
\subsection{Systems of equations}
\label{ss:SysEq}
When discussing equations it is standard to talk about `context'. This is a 
 a CCS expression  possibly containing  free variables that, however, may not occur within
the body of recursive definitions. 
Milner's ``unique solution of equations'' theorems~\cite{Mil89} intuitively
say that, if a context $C$
%  \hl{(i.e., a CCS expression with possibly
% free variables)}\footnote{\hl{Rigorously speaking, under our setting
% (i.e.~reusing free agent variables as equation variables) not all
% valid CCS expressions
% are valid contexts: those where free variables occur inside
% recusion operators must be all excluded. For instace, $\recu X (a.X +
% b.Y)$ is not a valid context with the variable $Y$. This special requirement
% is perfectly aligned with CCS literature using process constants, as any
% equation variable cannot occur inside the definition of any
% constant. In fact, all versions of ``unique solution of
% equations/contractions'' theorems do not hold if equation variables
% are allowed to occur inside any recursion operator.}} 
obeys certain conditions,
then all processes $P$ that satisfy the equation $P \wb \ct P$ are
bisimilar with each other.

\begin{definition}[equations] % Def 3.1
  \label{def:equation}
Assume that, for each $i$ of 
 a countable indexing set $I$, we have variables $X_i$, and expressions
$E_i$ possibly containing such variables $\cup_i \{ X_i\}$. Then 
$\{ X_i = E_i\}_{i\in I}$ is 
  a \emph{system of equations}. (There is one equation $E_i$ for each variable $X_i$.)
\end{definition}

The components of $\til P$ need not be
different from each other, as it must be for the variables $\til X$.
% If the system has infinitely many equations, the  tuples $\til P$
% and $\til X$ are infinite too. 

\begin{definition}[solutions and unique solutions]
  \label{def:solution}
Suppose $\{ X_i = E_i\}_{i\in I}$ is a system of equations: 
\begin{itemize}
\item
 $\til P$ is a \emph{solution of the system of equations (for $\wb$)} 
if for each $i$ it holds that $P_i \wb E_i [\til P]$;
\item The system has \emph{a unique solution for $\wb$}  if whenever 
 $\til P$ and $\til Q$ are both solutions then $\til P \wb \til Q$. 
\end{itemize} 
 \end{definition}
Similarily, the \emph{(unique) solution of a system of equations for $\sim$}
(or for $\rapprox$) can be obtained by replacing all occurrences of $\wb$
in above definition with $\sim$ and $\rapprox$, respectively.

For instance, the solution of the equation $X = a. X$ 
is the process
$R \DSdefi \recu A {\, (a. A)}$, and for any other solution $P$ we have $P \wb R$.
In contrast, the equation 
 $X = a|  X$ has solutions that may be quite different, for instance,
 $K$ and $K | b$, for $K \DSdefi \recu K {\, (a. K)}$. (Actually any process capable of
continuously performing $a$--actions is a solution of $X = a|  X$.)

%
% The unique solution of the system (1), modulo $\wb$,  is the constant $K \Defi a
% . K$:  for any other solution $P$ we have $P \wb K$.
% The unique solution of (2), modulo $\wb$, are the constants $K_1 , K_2$
% with $K_1 \Defi a . K_2$ and $K_2 \Defi b. K_1$; again, for any other
% pair of solutions $P_1,P_2$ we have $K_1 \wb P_1$ and $K_2 \wb P_2$.
%
Examples of systems that do not have unique solutions are: $X = X$, $X
= \tau . X$ and $X = a | X$.

\begin{definition}[guardedness of equations]
\label{def:guardness}
A system of equations $\{ X_i = E_i\}_{i\in I}$ is 
\begin{itemize}
\item \emph{weakly guarded} if, in each $E_i$, each occurrence of
  each $X_i$ is underneath a prefix;

\item \emph{guarded} if, in each $E_i$, each occurrence of
  each $X_i$ is underneath a \emph{visible} prefix;

\item \emph{sequential} if, in each $E_i$, each
  occurrence of each $X_i$ is only underneath prefixes and sums.
\end{itemize}
\end{definition}

In other words, if a system of equations is sequential, then for
each  $E_i$, any subexpression of $E_i$ in which $X_j $
appears, apart from $X_j$ itself, is a sum of prefixed expressions.
For instance,
\begin{itemize}
\item $X = \tau. X + \mu . \nil$ is sequential but not 
 guarded, because the guarding prefix for the variable
is not visible;
\item $X =  \ell . X | P$ is guarded but not sequential;
\item $X =  \ell . X + \tau. \res a (a .\outC b | a.\nil)$, as well as
$X = \tau . (a. X + \tau . b .X + \tau  )$ are both guarded and sequential.
\end{itemize}

Milner has  three versions of  ``unique solution of equations''
theorems, for $\sim$, $\wb$ and $\rapprox$, respectively, though only the
following two versions are explicitly mentioned in~\citep[p.~103, 158]{Mil89}:
\begin{theorem}[unique solution of equations for $\sim$]
\label{t:Mil89s1}
Let $E_i$ be weakly guarded with free variables in $\til X$,
and let ${\til P} \sim {\til E}\{\til P /\til X\}$,
  ${\til Q} \sim {\til E}\{\til Q /\til X\}$. Then ${\til P} \sim {\til Q}$.
\end{theorem}

\begin{theorem}[unique solution of equations for $\rapprox$]
\label{t:Mil89s3}
Let $E_i$ be guarded and sequential with free
variables in $\til X$, and let ${\til P} \rapprox {\til E}\{\til P /\til X\}$,
  ${\til Q} \rapprox {\til E}\{\til Q /\til X\}$. Then ${\til P} \rapprox {\til Q}$.
\end{theorem}

The version of Milner's unique-solution theorem for $\wb$ further requires
that all sums are guarded:
% \footnote{But if the CCS syntax were defined with only guarded
%   sums, i.e., $\sum_{i\in I} \mu_i.P_i$ as
%   in~\cite{sangiorgi2015equations}, this addition
%   requirement disappears automatically, and we can even say $\wb$ is indeed a
%   congruence.}:
\begin{theorem}[unique solution of equations for $\wb$]
\label{t:Mil89}
Let $E_i$ be guarded and sequential with free
variables in $\til X$, and let ${\til P} \wb {\til E}\{\til P /\til X\}$,
  ${\til Q} \wb {\til E}\{\til Q /\til X\}$. Then ${\til P} \wb {\til Q}$.
\end{theorem}

The proof of the theorem above exploits an invariance
property on immediate derivatives
of guarded and sequential expressions, and then extracts a bisimulation
(up to bisimilarity) out
of the solutions of the system.
To see the need of the sequentiality  condition, consider
 the equation (from \cite{Mil89}) $X = \res a (a. X | \outC a)$
where $X$ is guarded but not sequential. Any process that does not use
$a$ is a solution, e.g. $\nil$ and $b.\nil$.

For more details of above three theorems, see Section~\ref{ss:part2}
for the \univariate case and
Section~\ref{sec:multivariate} for the \multivariate case.

%% next file: contraction.tex

\subsection{Expansions and Contractions}
\label{s:mcontr}

Milner's ``unique solution of equations'' theorem for $\wb$
(Theorem~\ref{t:Mil89})
brings a new proof technique for proving (weak) bisimilarities. However, it has
 limitations: the equations must be guarded and sequential. (\hl{Moreover,}
all sums where equation variables appear must be guarded sums.)
This limits the usefulness of the technique, since
the occurrences of other operators \hl{using equation} variables, such as parallel
composition and restriction,
in general cannot be \hl{eliminated}. % without changing the meaning of equations
The constraints \hl{in} Theorem~\ref{t:Mil89}, however, can be
weakened if we move from equations to a special kind of inequations called
  \emph{contractions}.

Intuitively, the bisimilarity contraction $\mcontrBIS$ is a preorder
\hlD{in which} $P \mcontrBIS Q$ holds if $P \wb Q$ and, in addition, 
\emph{$Q$ has the possibility of being at least as efficient as $P$} (as far as
$\tau$-actions are performed).
The process $Q$, however, may be nondeterministic and may have other ways
\hl{to do} the same work, \hl{ways} which could be \hl{slower} (i.e., involving
more $\tau$-actions than those performed by $P$).
% Thus, in contrast with expansion,  we cannot really say that `$Q$ is more efficient than
% $P$'.

\begin{definition}[contraction]
\label{d:BisCon}
A process relation ${\R}$ 
 is a \emph{(bisimulation) contraction} if, whenever $P\RR Q$,

\begin{enumerate}
\item $P \arr\mu P'$ implies \hl{that} there is $Q'$ \hl{with} $Q \arcap \mu
  Q'$ and $P' \RR Q'$;
\item $Q \arr\mu Q'$ implies \hl{that} there is $P'$ \hl{with} $P \Arcap \mu
 P'$ and $P' \wb Q'$.
\end{enumerate}
Two processes $P$ and $Q$ are in the \emph{bisimilarity
contraction}, written as $P \mcontrBIS Q$,
if $P\ \R\ Q$ for some contraction $\R$.
Sometimes we write $\mexpaBIS$ for the inverse of $\mcontrBIS$.
\end{definition}
In clause (1) \hl{of the above definition}, $Q$ is required to match the challenge
transition \hl{of $P$} with at most one transition.
This makes sure that $Q$ is capable of mimicking % verb: mimic
$P$'s work at least as efficiently as $P$. 
In contrast, \hl{clause (2) entirely ignores efficiency on the challenges from $Q$:}
the final derivatives are required to be related by $\wb$, rather than by $\R$.

Bisimilarity contraction is coarser than bisimilarity expansion
$\expa$~\cite{arun1992efficiency,sangiorgi2015equations}, \hl{one of the
most useful auxiliary relations in up-to techniques}:
\begin{definition}[expansion]
\label{d:expa}
A process relation ${\R}$
  is an \emph{expansion} if, whenever $P\RR Q$,
 \begin{enumerate}
 \item   $P \arr\mu P'$ implies that there is $Q'$ with $Q \arcap \mu  Q'$
  and $P' \RR Q'$;
 \item $Q \arr\mu Q'$ implies that there is $P'$ with $P \Arr \mu P'$ and $P' \RR Q'$.
 \end{enumerate}
Two processes $P$ and $Q$ are in the \emph{bisimilarity
  expansion}, written as $P \expa Q$, if $P \RR Q$ for some expansion $\R$.
 \end{definition}
\hl{Bisimilarity expansion} is widely used in proof techniques for bisimilarity.
\hl{It} intuitively refines bisimilarity by 
formalising the idea of ``efficiency'' between processes.
Clause (1) is the same in the both preorders, while in clause (2) expansion \hl{requires}
$P \Arr \mu P'$, rather than $P \Arcap \mu P'$.
\hl{Moreover,} in clause (2) of Def.~\ref{d:BisCon} the final derivatives
are simply required to be bisimilar ($P' \wb Q'$).
Intuitively, $P \expa Q$ holds if $P\wb Q$ and, in addition, \emph{$Q$
  is always at least as efficient as $P$}.

\begin{example}
\label{exa:contr}
We have %\mcontrBIS a + \tau^n . a $
 $ a \not  \mcontrBIS \tau. a$. However,
$a+ \tau . a \mcontrBIS a$, as well as its converse, 
$  a \mcontrBIS a +
\tau . a $. Indeed, if $P \wb Q$ then 
$  P  \mcontrBIS P +Q$. The last two relations do not hold with 
$\expa$, which explains the strictness of the inclusion
 ${\expa} \subset {\mcontrBIS}$. 
% The inclusion is strict: for instance
% $a+ \tau . a \mcontrBIS a$, where $\mcontrBIS$ cannot be replaced by
%  $\contr$. Also the converse of  $a+ \tau . a \mcontrBIS a$ holds, namely
% $  a \mcontrBIS a +
% \tau . a $. However, we have %\mcontrBIS a + \tau^n . a $
%  $ a \not  \mcontrBIS \tau. a$
\end{example} 

\hlD{Bisimilarity expansion and bisimilarity contraction are both
preorders.}
Similarily with (weak) bisimilarity, both the expansion and the
contraction preorders are preserved by all CCS operators except the
summation. The proofs are similar \hlD{to those for bisimilarity},
see, e.g.~\cite{sangiorgi2017equations} \hl{for details.}

% next file: unique.tex

\subsection{Systems of contractions}
\label{ss:SysContr}

A \emph{system of contractions} is defined as a system of equations,
except that the contraction symbol $\mcontr$ is used in the place of
the equality symbol $=$. Thus a system of contractions is a set 
$\{  X_i \mcontr E_i\}_{i\in I}$
where $I$ is an  indexing set and expressions
$E_i$  may contain the  \behavC\  variables 
$\{  X_i\}_{i\in I}$.

\begin{definition}
\label{d:uniContra}
Given a system of contractions 
$\{  X_i \mcontr E_i\}_{i\in I}$, 
 we say that:
\begin{itemize}
\item $\til P$ is a \emph{solution (for $\mcontrBIS$) of the 
 system of contractions} if $\til P \mcontrBIS \til E [\til P]$;
\item the system has \emph{a unique solution (for $\approx$)}
if $\til P \approx \til Q$ whenever $\til P$ and $\til Q$ are both solutions.
\end{itemize}
\end{definition}

The guardedness of contractions follows Def.~\ref{def:guardness} (for equations).
% \begin{definition}
% \label{d:guarded}
% A system of contractions $\{  X_i \mcontr E_i\}_{i\in I}$
%  is
% \emph{weakly guarded}
% if,  in each    $E_i$, each occurrence of
% a \behavC\ variable is underneath a prefix.

% The system use \emph{weakly-guarded sums} if 
% each $E_i$ only makes use of guarded sums.
% \end{definition}

\begin{lemma}
\label{l:uptocon}
Suppose $\til P$ and $\til Q$ are solutions  for $\mcontrBIS$
 of a system of weakly-guarded contractions that uses 
weakly-guarded sums.
For any context $\qct$  that uses 
weakly-guarded sums,
if  $\ct{\til P}\Arr{\mu}  R$,
 then 
there is a context $\qctp$  that uses 
weakly-guarded sums
such that $R \mcontrBIS \ctp{\til P}$ and $\ct{\til Q} \Arcap{\mu}
 \wb \ctp{\til Q}$.\footnote{There's no typo here: $\ct{\til Q} \Arcap{\mu} \wb \ctp{\til
     Q}$ means $\exists {\til R}.\; \ct{\til Q} \Arcap{\mu} {\til R}
   \wb \ctp{\til Q}$. Same as in Lemma~\ref{l:ruptocon}.}
\end{lemma}

\begin{proof}{(sketch from \cite{sangiorgi2017equations})}
Let $n$ be the length of the transition $\ct{\til P}\Arr\mu R$  (the
number of `strong steps' of which it is composed), and  
let $\ctpp {\til P}$ and $\ctpp {\til Q}$  be the processes obtained
from  $\ct {\til P}$ and $\ct {\til Q}$ by unfolding the definitions
of the contractions $n$ times. Thus in $\qctpp$ each hole is
underneath at least $n$ prefixes, and cannot contribute to an action
in the first $n$ transitions; moreover all the contexts have only
weakly-guarded sums.

We have $\ct{\til P} \mcontrBIS \ctpp{\til P}$, and 
$\ct{\til Q} \mcontrBIS \ctpp{\til Q}$, 
 by the substitutivity  properties of $\mcontrBIS$ (we exploit here
 the syntactic constraints on sums). Moreover,
 since each hole of the  context $\qctpp$ is underneath at least $n$
 prefixes, applying  
the definition
 of $ \mcontrBIS$ on the transition 
 $\ct{\til P}\Arr{\mu}  R$, we infer the existence
 of $\qctp$ such that 
$
\ctpp{\til P}\Arcap{\mu} \ctp{\til P} \mexpaBIS R
$
and 
$
\ctpp{\til Q}\Arcap{\mu}  \ctp{\til Q} 
. $
Finally, again applying the definition of $\mcontrBIS$ on 
$\ct{\til Q} \mcontrBIS \ctpp{\til Q}$, 
we derive 
$
\ct{\til Q}\Arcap{\mu}  \wb \ctp{\til Q} 
.$
\end{proof}

\begin{theorem}[unique solution of contractions for $\wb$]
\label{t:contraBisimulationU}
A system of weakly-guarded contractions
having only weakly-guarded sums, has a unique solution for $\wb$.
\end{theorem}

\begin{proof}{(sketch from \cite{sangiorgi2017equations})}
Suppose $\til P$ and $\til Q$ are two such solutions (for $\wb$) and consider
the relation
\begin{equation}
\label{eq:R}
\R \DSdefi \{ 
(R,S) \st R \wb \ct{\til P}, S \wb \ct{\til Q} \mbox{ for some context
$\qct$ (having only weakly-guarded sums)} \} \enspace.
\end{equation}
We show that $\R$ is a bisimulation. \hl{Suppose $R\ \R\ S$ vis the context
$C$}, and $R \arr{\mu} R'$. We have to find $S'$ with $S \Arcap{\mu}
S'$ and $R'\ \R\ S'$. From $R \wb C[{\til P}]$, we derive $C[{\til P}]
\Arcap{\mu} R'' \wb R'$ for some $R''$. By Lemma~\ref{l:uptocon},
there is $C'$ with $R'' \mcontrBIS C'[{\til P}]$ and $C[{\til Q}]
\Arcap{\mu} \wb C'[{\til Q}]$. Hence, by definition of $\wb$, there is
also $S'$ with $S \Arcap{\mu} S' \wb C'[{\til Q}]$. This closes the
proof, as we have $R' \wb C'[{\til P}]$ and $S' \wb C'[{\til Q}]$.
\end{proof}

\section{Rooted contraction}
\label{ss:new}

The unique solution theorem of Section~\ref{ss:SysContr} requires a
constrained syntax for sums, due to the congruence and precongruence
problems of bisimilarity and contraction with such operator. 
We show here that the constraints can be
removed by moving to the induced congruence and precongruence, the
latter called \emph{rooted contraction}:
\begin{definition}
\label{d:rcontra}
Two processes $P$ and $Q$ are in \emph{rooted contraction}, written as
 $P\rcontr Q$, if
\begin{enumerate}
\item $P \arr\mu P'$ implies that there is $Q'$ with $Q \arr \mu Q'$
 and $P'\mcontrBIS Q'$;
\item $Q \arr\mu Q'$   implies that there is $P'$ with $P \Arr \mu
 P'$ and $P' \wb Q'$.
\end{enumerate}
\end{definition}

%Above definition adapts the definition of rooted
%bisimilarity on top of that of the  contraction preorder
%$\mcontrBIS$.  %% Reviewer said this sentance is unclear. I too think so.

\hl{The discovery of this definition is with help of HOL theorem
  prover and
the following two principles:} (1) Its definition must not be recursive,
instead it should resemble the definition of rooted bisimilarity
$\approx^c$ in Def.~\ref{d:rootedBisimilarity};
(2) It must be built on top of existing \emph{contracts}
relation $\mcontrBIS$, which we believe it's the \emph{right} one
because of its completeness. \hl{Multiple candicates were quickly tested,
finally only above definition is proven to be a precongruence, as the
following theorem states.} (The proof of this result is along the lines of the analogous result
for rooted bisimilarity with respect to bisimilarity.)

\begin{theorem}
\label{t:rcontrPrecongruence}
$\rcontr$ is a precongruence in CCS, and it is the
coarsest precongruence contained in $\contr$.
\end{theorem}  

For a system of rooted contractions, the meaning of 
``solution for $\rcontr$'' and of \emph{a unique solution for $\rapprox$}
is the expected one --- just replace in Definition~\ref{d:uniContra}  the preorder 
$\contr$ with $\rcontr$, and the equivalence 
$\approx$ with $\rapprox$.
%
For this new relation, the analogous of Lemma~\ref{l:uptocon} and of
Theorem~\ref{t:contraBisimulationU} can now be stated without constraints on the sum
operator.
The schema of the proofs is almost the same, because all needed
properties of $\rcontr$ in the proof is its precongruence, which is
now true for unrestricted contexts using direct sums:

\begin{lemma}
\label{l:ruptocon}
Suppose $\til P$ and $\til Q$ are solutions  for $\rcontr$ 
 of a system of weakly-guarded
contractions.
For any context $\qct$, 
if  $\ct{\til P}\Arr{\mu}  R$,
 then 
there is a  context $\qctp$
such that $R \mcontrBIS \ctp{\til P}$ and  $\ct{\til Q} \Arr{\mu}
 \wb \ctp{\til Q}$.
\end{lemma}

\begin{theorem}[unique solution of contractions for $\rapprox$]
\label{t:rcontraBisimulationU}
A system of weakly-guarded contractions has a unique solution 
 for $\rapprox$. (thus also for $\wb$)
\end{theorem} 

\begin{proof}
We first follow the same steps as in the proof of Theorem~\ref{t:contraBisimulationU} to show the relation $\R$ (now
with $\rcontr$ and unrestrict context $C$) in (\ref{eq:R}) is bisimulation,
exploting Lemma~\ref{l:ruptocon}. \hl{Then it remains to show that,} for
any two process $P$ and $Q$ with action $\mu$, if $P \arr{\mu} P'$ then
there is $Q'$ such that $Q \Arr{\mu} Q'$ (not $Q \Arcap{\mu} Q'$!) and
$P'\ \R\ Q'$, and also for the converse direction, exploting Lemma
4.13 of \cite{Mil89} \hl{(unexpected!)}. \hl{By definition of
\emph{bisimulation} (not $\wb$!) and $\approx^c$, we actually proved $P
\approx^c Q$ instead of $P \wb Q$.}
\end{proof}


% part 2 (Tian)
%%%% -*- Mode: LaTeX; -*-

\newcommand\fun{{\to}}
\newcommand\prd{{\times}}
\newcommand{\ty}[1]{\textsl{#1}}
\newcommand\conj{\ \wedge\ }
\newcommand\disj{\ \vee\ }
\newcommand\imp{ \Rightarrow }
\newcommand\eqv{\ \equiv\ }
\newcommand\vbar{\mid}
\newcommand\turn{\ \vdash\ } % FIXME: "\ " resultgs in extra space
\newcommand\hilbert{\varepsilon}
\newcommand{\uquant}[1]{\forall #1.\ }
\newcommand{\equant}[1]{\exists #1.\ }
\newcommand{\hquant}[1]{\hilbert #1.\ }
\newcommand{\iquant}[1]{\exists ! #1.\ }
\newcommand{\lquant}[1]{\lambda #1.\ }
\newcommand{\ml}[1]{\mbox{{\def\_{\char'137}\texttt{#1}}}}
\newcommand{\con}[1]{\mathrm{#1}}

\newcommand\bool{\ty{bool}}
\newcommand\num{\ty{num}}
\newcommand\ind{\ty{ind}}
\newcommand\lst{\ty{list}}

\providecommand{\T}{\con{T}}
\renewcommand{\T}{\con{T}}
\newcommand\F{\con{F}}
\newcommand\OneOne{\con{One\_One}}
\newcommand\OntoSubset{\con{Onto\_Subset}}
\newcommand\Onto{\con{Onto}}
\newcommand\TyDef{\con{Type\_Definition}}

\section{The formalisation}
\label{s:for}

We highlight here a formalisation of CCS in the HOL theorem prover
(HOL4)~\cite{Melham:1993vl,slind2008brief}, with a focus towards the
theory (and formal proofs) of the unique solution of
equations/contractions theorems mentioned in Section~\ref{s:eq} and
\ref{s:mcontr}.
All proof scripts are available as part of HOL's official
examples\footnote{\url{https://github.com/HOL-Theorem-Prover/HOL/tree/master/examples/CCS}}.
The work so far consists of about 24,000 lines (1MB) of code in total,
in which about 5,000 lines were derived from the early work of Monica
Nesi~\cite{Nesi:1992ve} on HOL88, with major modifications.

Higher Order Logic (HOL)~\cite{hollogic} traces
its roots back to LCF
\cite{gordon1979edinburgh,milner1972logic} by Robin Milner and others
since 1972. It is a variant of
Church’s Simple Theory of Types (STT)~\cite{church1940formulation},
plus a higher order version of Hilbert's choice operator $\varepsilon$,
Axiom of Infinity, and Rank-1 (prenex) polymorphism.
HOL4 has implemented the original HOL, 
while some other theorem provers in HOL family (e.g. Isabelle/HOL) have
certain extensions.
%  (they made the formal language more powerful,
% but they also bring the possibilities that the entire logic becomes
% inconsistent). 
Indeed the HOL has considerably simpler logical
foundations than most other theorem provers. %, e.g. Coq. 
As a consequence, theories and proofs verified in HOL are easier to understand
 to people who are not familar with more advanced
dependent type theories.

HOL4 is implemented in Standard ML, the same programming language who
plays three different roles:
\begin{enumerate}
\item \hl{The} underlying implementation language for the core HOL engine;
\item \hl{The} language in which proof tactics are implemented;
\item \hl{The} interface language of the HOL proof scripts and interactive shell.
\end{enumerate}
Moreover, using the same language HOL4 users can write complex automatic
verification tools by calling HOL's theorem proving
facilities. The formal proofs of theorems in CCS theory
are mostly done by an \emph{interactive process} closely following
their informal proofs, with minimal automatic proof searching.

\subsection{Higher Order Logic (HOL)}

HOL is a formal system of typed logical terms. The types are expressions that denote sets (in the
universe $\mathcal{U}$). HOL's type system is much simpler than those
based on dependent types and other type theories. There are four kinds of types in the HOL
logic, as illustrated in Fig. \ref{fig:hol-types} for its BNF
grammar. Noticed that, in HOL the standard atomic types \emph{bool} and \emph{ind}
 denote, respectively, the distinguished two-element set 2 and the
distinguished infinite set $I$.

\newlength{\ttX}
\settowidth{\ttX}{\tt X}
\newcommand{\tyvar}{\setlength{\unitlength}{\ttX}\begin{picture}(1,6)
\put(.5,0){\makebox(0,0)[b]{\footnotesize type variables}}
\put(0,1.5){\vector(0,1){4.5}}
\end{picture}}
\newcommand{\tyatom}{\setlength{\unitlength}{\ttX}\begin{picture}(1,6)
\put(.5,2.3){\makebox(0,0)[b]{\footnotesize atomic types}}
\put(.5,3.3){\vector(0,1){2.6}}
\end{picture}}
\newcommand{\funty}{\setlength{\unitlength}{\ttX}\begin{picture}(1,6)
\put(.5,1.5){\makebox(0,0)[b]{\footnotesize function types}}
\put(.5,0){\makebox(0,0)[b]{\footnotesize (domain $\sigma_1$, codomain $\sigma_2$)}}
\put(1,2.5){\vector(0,1){3.5}}
\end{picture}}
\newcommand{\cmpty}{\setlength{\unitlength}{\ttX}\begin{picture}(1,6)
\put(2,3.3){\makebox(0,0)[b]{\footnotesize compound types}}
\put(1.9,4.5){\vector(0,1){1.5}}
\end{picture}}

\begin{figure}[h]
\begin{equation*}
\sigma\quad ::=\quad {\mathord{\mathop{\alpha}\limits_{\tyvar}}}
        \quad\mid\quad{\mathord{\mathop{c}\limits_{\tyatom}}}
        \quad\mid\quad\underbrace{(\sigma_1, \ldots , \sigma_n){op}}_{\cmpty}
        \quad\mid\quad\underbrace{\sigma_1\fun\sigma_2}_{\funty}
\end{equation*}
   \caption{HOL's type grammar}
   \label{fig:hol-types}
\end{figure}

The terms of the HOL logic are expressions that denote elements of the
sets denoted by types. There're four kinds of terms in the HOL
logic. There can be described approximately by the BNF grammar in
Fig. \ref{fig:hol-terms}.

\settowidth{\ttX}{\tt X}
\newcommand{\var}{\setlength{\unitlength}{\ttX}\begin{picture}(1,6)
\put(.5,0){\makebox(0,0)[b]{\footnotesize variables}}
\put(0,1.5){\vector(0,1){4.5}}
\end{picture}}
\newcommand{\const}{\setlength{\unitlength}{\ttX}\begin{picture}(1,6)
\put(.5,2.3){\makebox(0,0)[b]{\footnotesize constants}}
\put(.5,3.5){\vector(0,1){2.4}}
\end{picture}}
\newcommand{\app}{\setlength{\unitlength}{\ttX}\begin{picture}(1,6)
\put(.5,1.5){\makebox(0,0)[b]{\footnotesize function applications}}
\put(.5,0){\makebox(0,0)[b]{\footnotesize (function $t$, argument $t'$)}}
\put(0.5,2.5){\vector(0,1){3.5}}
\end{picture}}
\newcommand{\abs}{\setlength{\unitlength}{\ttX}\begin{picture}(1,6)
\put(1,3.3){\makebox(0,0)[b]{\footnotesize $\lambda$-abstractions}}
\put(0.7,4.5){\vector(0,1){1.5}}
\end{picture}}

\begin{figure}[h]
\begin{equation*}
t \quad ::=\quad {\mathord{\mathop{x}\limits_{\var}}}
        \quad\mid\quad{\mathord{\mathop{c}\limits_{\const}}}
        \quad\mid\quad\underbrace{t\ t'}_{\app}
        \quad\mid\quad\underbrace{\lambda x .\ t}_{\abs}
\end{equation*}
   \caption{HOL's term grammar}
   \label{fig:hol-terms}
 \end{figure}

The deductive system of the HOL is specified by eight primitive
derivative rules: (c.f.~\cite{hollogic} for more details.)
\begin{enumerate}
\item Assumption introduction (\texttt{ASSUME});
\item Reflexivity (\texttt{REFL});
\item $\beta$-conversion (\texttt{BETA\_CONV});
\item Substitution (\texttt{SUBST});
\item Abstraction (\texttt{ABS});
\item Type instantiation (\texttt{INST\_TYPE});
\item Discharging an assumption (\texttt{DISCH});
\item Modus Ponens (\texttt{MP}).
\end{enumerate}
All proofs are eventually reduced to applications of the above rules,
which also give the semantics of two foundamental
logical connectives, the equality ($=$) and implication
($\Rightarrow$). The rest logical connectives and first-order
quantifiers, including the logical true (\HOLinline{\HOLConst{T}}) and false (\HOLinline{\HOLConst{F}}), are
further defined as $\lambda$-functions: (one rarely needs to care
things at this level, however.)
\begin{equation*}
\begin{array}{l}
\turn \T       =  ((\lquant{x_{\ty{bool}}}x) =
               (\lquant{x_{\ty{bool}}}x))    \\
\turn \forall  =  \lquant{P_{\alpha\fun\ty{bool}}} P =
                    (\lquant{x}\T ) \\
\turn \exists  =  \lquant{P_{\alpha\fun\ty{bool}}} P({\hilbert}\, P) \\
\turn \F  =  \uquant{b_{\ty{bool}}} b  \\
\turn \neg    =  \lquant{b} b \imp \F \\
\turn {\wedge}  =  \lquant{b_1\ b_2}\uquant{b} (b_1\imp (b_2 \imp b)) \imp b \\
\turn {\vee}  =  \lquant{b_1\ b_2}\uquant{b} (b_1 \imp b)\imp ((b_2 \imp b) \imp b) \\
\turn \OneOne  =  \lquant{f_{\alpha \fun\beta}}\uquant{x_1\ x_2}
                    (f\ x_1 = f\ x_2)  \imp (x_1 = x_2) \\
\turn \Onto  =  \lquant{f_{\alpha\fun\beta}}
                  \uquant{y}\equant{x} y = f\ x \\
\turn \TyDef  =  \lambda P_{\alpha\fun\ty{bool}}\
                  rep_{\beta\fun\alpha}.\;
                  \OneOne\ rep \ \wedge{}\  (\uquant{x}P x = (\equant{y} x = rep\ y))
\end{array}
\end{equation*}
It deserves to mention that, the last logical constant above,
$\TyDef$, can be used to define new HOL types as bijections of
subsets of existing types. (c.f.~\cite{Melham:1989dk} for more details.)
HOL's \texttt{Datatype} package~\cite{Melham:1991, holdesc} automates
this tedious process, and can be used to define needed types in CCS.

Finally, the whole HOL \emph{standard} theory is based on the
following four axioms, from which the almost\footnote{HOL is strictly
  weaker than ZFC (the Zermelo-Frankel set theory with the
Axiom of Choice), thus not all theorems valid in ZFC can be formalised
in HOL. (c.f.~\cite{hollogic} for more details.)} entire mathematics
can be formalised:
\begin{equation*}
\begin{array}{@{}l@{\qquad}l}
\mbox{\texttt{BOOL\_CASES\_AX}} &\vdash \uquant{b} (b = \T )\ \vee \ (b = \F )\\
\mbox{\texttt{ETA\_AX}} &
\vdash \uquant{f_{\alpha\fun\beta}}(\lquant{x}f\ x) = f\\
\mbox{\texttt{SELECT\_AX}} &
\vdash \uquant{P_{\alpha\fun\ty{bool}}\ x} P\ x \imp P({\hilbert}\ P)\\
\mbox{\texttt{INFINITY\_AX}}&
\vdash \equant{f_{\ind\fun \ind}} \OneOne \ f \conj \neg(\Onto \ f)\\
\end{array}
\end{equation*}

Notice that, usually the above four axioms are the only axioms allowed
in conventional formalisation projects in HOL: adding new axioms may
either break the logic \hl{consistency} or make the whole
\hl{formalisation} work less
convincible. \hl{In general,} the theorem prover guarantees that,
without adding new axioms, if both the definitions and the statements
of theorems \hl{were} correct, then the \hl{proofs of these theorems} must
be \hl{also} correct.
% This last sentence doesn't have a good link with the previous context,
% disabled. --Chun
%
%% This is also the case here: SOS rules of CCS are
%% \emph{not} axioms but consequences of the inductive relation\hl{, the
%% CCS transition $\rightarrow$}.

% next file: sos.htex

%%%% -*- Mode: LaTeX -*-
%%
%% This is the draft of the 2nd part of EXPRESS/SOS 2018 paper, co-authored by
%% Prof. Davide Sangiorgi and Chun Tian.

\subsection{CCS and its transitions by SOS rules}

In our CCS formalisation, the type ``\HOLinline{\ensuremath{\beta} \HOLTyOp{Label}}'' (\texttt{'b} or
$\beta$ is the type variable for actions) accounts for visible actions, divided into input
and output actions, defined by HOL's Datatype package:
\begin{lstlisting}
val _ = Datatype `Label = name 'b | coname 'b`;
\end{lstlisting}
The type ``\HOLinline{\ensuremath{\beta} \HOLTyOp{Label} \HOLTyOp{option}}'' is the
union of all visible actions, plus invisible action $\tau$ (now based on
HOL's \texttt{option} theory). The cardinality of
``\HOLinline{\ensuremath{\beta} \HOLTyOp{Label} \HOLTyOp{option}}'' (and therefore of all
CCS types built on top of it)
 depends on the choice (or \emph{type-instantiation}) of $\beta$.

The type ``\HOLinline{(\ensuremath{\alpha}, \ensuremath{\beta}) \HOLTyOp{CCS}}'', accounting for the CCS
syntax\footnote{The order of type variables $\alpha$ and $\beta$
    is irrelevant. Our choice is aligned with other CCS literals.
$\mathrm{CCS}(h,k)$ is the CCS subcalculus which can use at most $h$ constants
and $k$ actions. \cite{gorrieri2017ccs} Thus, to formalize theorems on
such a CCS subcalculus, the needed CCS type can be retrieved by instantiating the type
variables $\alpha$ and $\beta$ in ``\HOLinline{(\ensuremath{\alpha}, \ensuremath{\beta}) \HOLTyOp{CCS}}'' with types
having the corresponding cardinalities $h$ and $k$. Monica Nesi goes
too far by adding another type variable $\gamma$ for value-passing CCS \cite{Nesi:2017wo}.}, is then defined inductively:
(\texttt{'a} or $\alpha$ is the type variable for recursion variables,
``\HOLinline{\ensuremath{\beta} \HOLTyOp{Relabeling}}'' is the type of all relabeling functions,
\mbox{\color{blue}{\texttt{`}}} is for backquotes of HOL terms):
\begin{lstlisting}
val _ = Datatype `CCS = nil
		      | var 'a
		      | prefix ('b Action) CCS
		      | sum CCS CCS
		      | par CCS CCS
		      | restr (('b Label) set) CCS
		      | relab CCS ('b Relabeling)
		      | rec 'a CCS`;
\end{lstlisting}

We have added some grammar support,
 using HOL's powerful pretty printer, to represent CCS
processes in more readable forms (c.f. the column \textbf{HOL (abbrev.)}
in Table \ref{tab:ccsoperator}, which summarizes 
the main syntactic notations of CCS). For the restriction
operator, we have chosen to allow a  set of names as a parameter, rather than a
  single name as in the ordinary  CCS syntax; this simplifies 
the manipulation of 
 processes with different orders of
  nested restrictions.
% Also, we do not assume that the uses of \texttt{var} are
%  guarded by \texttt{rec} of the same variable.

%  (Notice the use of
% recursion operator for representing process constants)
\begin{table}[h]
\begin{center}
\begin{tabular}{|c|c|c|c|}
\hline
\textbf{Operator} & \textbf{CCS Notation} & \textbf{HOL term} &
                                                                \textbf{HOL (abbrev.)}\\
\hline
nil & $\textbf{0}$ & \HOLinline{\HOLConst{nil}} & \HOLinline{\HOLConst{nil}} \\
prefix & $u.P$ & \texttt{prefix u P} & \HOLinline{\HOLFreeVar{u}\HOLSymConst{..}\HOLFreeVar{P}} \\
sum & $P + Q$ & \texttt{sum P Q} & \HOLinline{\HOLFreeVar{P} \HOLSymConst{\ensuremath{+}} \HOLFreeVar{Q}} \\
parallel & $P \,\mid\, Q$ & \texttt{par P Q} & \HOLinline{\HOLFreeVar{P} \HOLSymConst{\ensuremath{\parallel}} \HOLFreeVar{Q}} \\
restriction & $(\nu\;L)\;P$ & \texttt{restr L P} & \HOLinline{\HOLSymConst{\ensuremath{\nu}} \HOLFreeVar{L} \HOLFreeVar{P}}  \\
recursion & $\recu A P$ & \texttt{rec A P} & \HOLinline{\HOLConst{rec} \HOLFreeVar{A} \HOLFreeVar{P}}  \\
relabeling & $P\;[r\!f]$ & \texttt{relab P rf} & \HOLinline{\HOLConst{relab} \HOLFreeVar{P} \HOLFreeVar{rf}}  \\
\hline
constant & $A$ & \texttt{var A} & \HOLinline{\HOLConst{var} \HOLFreeVar{A}} \\
invisible action & $\tau$ & \texttt{tau} & \HOLinline{\HOLSymConst{\ensuremath{\tau}}} \\
input action & $a$ & \texttt{label (name a)} & \HOLinline{\HOLConst{In} \HOLFreeVar{a}} \\
output action & $\outC a$ & \texttt{label (coname a)} & \HOLinline{\HOLConst{Out} \HOLFreeVar{a}} \\
\hline
\end{tabular}
\end{center}
%\vspace{-1em}
   \caption{Syntax of CCS operators, constant and actions}
   \label{tab:ccsoperator}
\end{table}

The transition semantics of CCS processes follows Structural
Operational Semantics (SOS) in Fig.~\ref{f:LTSCCS}:
\begin{alltt}
\HOLTokenTurnstile{} \HOLFreeVar{u}\HOLSymConst{..}\HOLFreeVar{P} \HOLTokenTransBegin\HOLFreeVar{u}\HOLTokenTransEnd \HOLFreeVar{P}\hfill\texttt{[PREFIX]}
\HOLTokenTurnstile{} \HOLFreeVar{P} \HOLTokenTransBegin\HOLFreeVar{u}\HOLTokenTransEnd \HOLFreeVar{P\sp{\prime}} \HOLSymConst{\HOLTokenImp{}} \HOLFreeVar{P} \HOLSymConst{\ensuremath{+}} \HOLFreeVar{Q} \HOLTokenTransBegin\HOLFreeVar{u}\HOLTokenTransEnd \HOLFreeVar{P\sp{\prime}}\hfill\texttt{[SUM1]}
\HOLTokenTurnstile{} \HOLFreeVar{P} \HOLTokenTransBegin\HOLFreeVar{u}\HOLTokenTransEnd \HOLFreeVar{P\sp{\prime}} \HOLSymConst{\HOLTokenImp{}} \HOLFreeVar{Q} \HOLSymConst{\ensuremath{+}} \HOLFreeVar{P} \HOLTokenTransBegin\HOLFreeVar{u}\HOLTokenTransEnd \HOLFreeVar{P\sp{\prime}}\hfill\texttt{[SUM2]}
\HOLTokenTurnstile{} \HOLFreeVar{P} \HOLTokenTransBegin\HOLFreeVar{u}\HOLTokenTransEnd \HOLFreeVar{P\sp{\prime}} \HOLSymConst{\HOLTokenImp{}} \HOLFreeVar{P} \HOLSymConst{\ensuremath{\parallel}} \HOLFreeVar{Q} \HOLTokenTransBegin\HOLFreeVar{u}\HOLTokenTransEnd \HOLFreeVar{P\sp{\prime}} \HOLSymConst{\ensuremath{\parallel}} \HOLFreeVar{Q}\hfill\texttt{[PAR1]}
\HOLTokenTurnstile{} \HOLFreeVar{P} \HOLTokenTransBegin\HOLFreeVar{u}\HOLTokenTransEnd \HOLFreeVar{P\sp{\prime}} \HOLSymConst{\HOLTokenImp{}} \HOLFreeVar{Q} \HOLSymConst{\ensuremath{\parallel}} \HOLFreeVar{P} \HOLTokenTransBegin\HOLFreeVar{u}\HOLTokenTransEnd \HOLFreeVar{Q} \HOLSymConst{\ensuremath{\parallel}} \HOLFreeVar{P\sp{\prime}}\hfill\texttt{[PAR2]}
\HOLTokenTurnstile{} \HOLFreeVar{P} \HOLTokenTransBegin\HOLConst{label} \HOLFreeVar{l}\HOLTokenTransEnd \HOLFreeVar{P\sp{\prime}} \HOLSymConst{\HOLTokenConj{}} \HOLFreeVar{Q} \HOLTokenTransBegin\HOLConst{label} (\HOLConst{COMPL} \HOLFreeVar{l})\HOLTokenTransEnd \HOLFreeVar{Q\sp{\prime}} \HOLSymConst{\HOLTokenImp{}} \HOLFreeVar{P} \HOLSymConst{\ensuremath{\parallel}} \HOLFreeVar{Q} \HOLTokenTransBegin\HOLSymConst{\ensuremath{\tau}}\HOLTokenTransEnd \HOLFreeVar{P\sp{\prime}} \HOLSymConst{\ensuremath{\parallel}} \HOLFreeVar{Q\sp{\prime}}\hfill\texttt{[PAR3]}
\HOLTokenTurnstile{} \HOLFreeVar{P} \HOLTokenTransBegin\HOLFreeVar{u}\HOLTokenTransEnd \HOLFreeVar{Q} \HOLSymConst{\HOLTokenConj{}} ((\HOLFreeVar{u} \HOLSymConst{=} \HOLSymConst{\ensuremath{\tau}}) \HOLSymConst{\HOLTokenDisj{}} (\HOLFreeVar{u} \HOLSymConst{=} \HOLConst{label} \HOLFreeVar{l}) \HOLSymConst{\HOLTokenConj{}} \HOLFreeVar{l} \HOLSymConst{\HOLTokenNotIn{}} \HOLFreeVar{L} \HOLSymConst{\HOLTokenConj{}} \HOLConst{COMPL} \HOLFreeVar{l} \HOLSymConst{\HOLTokenNotIn{}} \HOLFreeVar{L}) \HOLSymConst{\HOLTokenImp{}}
   \HOLSymConst{\ensuremath{\nu}} \HOLFreeVar{L} \HOLFreeVar{P} \HOLTokenTransBegin\HOLFreeVar{u}\HOLTokenTransEnd \HOLSymConst{\ensuremath{\nu}} \HOLFreeVar{L} \HOLFreeVar{Q}\hfill\texttt{[RESTR]}
\HOLTokenTurnstile{} \HOLFreeVar{P} \HOLTokenTransBegin\HOLFreeVar{u}\HOLTokenTransEnd \HOLFreeVar{Q} \HOLSymConst{\HOLTokenImp{}} \HOLConst{relab} \HOLFreeVar{P} \HOLFreeVar{rf} \HOLTokenTransBegin\HOLConst{relabel} \HOLFreeVar{rf} \HOLFreeVar{u}\HOLTokenTransEnd \HOLConst{relab} \HOLFreeVar{Q} \HOLFreeVar{rf}\hfill\texttt{[RELABELING]}
\HOLTokenTurnstile{} \HOLConst{CCS_Subst} \HOLFreeVar{P} (\HOLConst{rec} \HOLFreeVar{A} \HOLFreeVar{P}) \HOLFreeVar{A} \HOLTokenTransBegin\HOLFreeVar{u}\HOLTokenTransEnd \HOLFreeVar{P\sp{\prime}} \HOLSymConst{\HOLTokenImp{}} \HOLConst{rec} \HOLFreeVar{A} \HOLFreeVar{P} \HOLTokenTransBegin\HOLFreeVar{u}\HOLTokenTransEnd \HOLFreeVar{P\sp{\prime}}\hfill\texttt{[REC]}
\end{alltt}

The rule \texttt{REC} (Recursion)
 says that if we substitute all appearances of variable $A$ in $P$ to
$(\recu A P)$ and the resulting process has a transition to $P'$
with action $u$, then $(\recu A P)$ has the same
transition. From HOL's viewpoint, these
SOS rules are \emph{inductive 
  definitions} on the tenary relation \HOLinline{\HOLConst{TRANS}} of type ``\HOLinline{(\ensuremath{\alpha}, \ensuremath{\beta}) \HOLTyOp{CCS} \HOLTokenTransEnd \ensuremath{\beta} \HOLTyOp{Label} \HOLTyOp{option} \HOLTokenTransEnd (\ensuremath{\alpha}, \ensuremath{\beta}) \HOLTyOp{CCS} \HOLTokenTransEnd \HOLTyOp{bool}}'', generated by HOL's 
\texttt{Hol_reln} function.

A useful function that we have defined, exploiting the interplay
between HOL4 and Standard ML (and following an idea by Nesi \cite{Nesi:1992ve})
 is a complex Standard ML function
  taking a CCS process and returning a theorem indicating all its
  direct transitions.\footnote{If the input process could yield
    something infinite branching, due to the use of recursion or
    relabeling operators, the program will loop forever without
    outputting a theorem.}
For instance, we know that the process $(a.0 | \bar{a}.0)$ has three
possible transitions: $(a.0 | \bar{a}.0) \overset{a}{\longrightarrow}
(0 | \bar{a}.0)$, $(a.0 | \bar{a}.0)
\overset{\bar{a}}{\longrightarrow} (a.0 | 0)$ and $(a.0 | \bar{a}.0)
\overset{\tau}{\longrightarrow} (0 | 0)$.
To completely describe all possible transitions of a process, if done manually, the
following facts should be proved: (1) there exists transitions from
$(a.0 | \bar{a}.0)$ (optional); (2) the correctness for each of the
transitions; and (3) the non-existence of other transitions.

For large processes it may be surprisingly hard to manually prove the
non-existence of transitions.  Hence the usefulness of appealing to 
the new  function \texttt{CCS\_TRANS\_CONV}. 
For instance this function
is called on the  process $(a.0 | \bar{a}.0)$ thus:
(\mbox{\color{blue}{\texttt{``}}} is for double-backquotes of HOL
  terms, \mbox{\color{blue}{\texttt{>}}} is HOL's prompt)
\begin{lstlisting}
> CCS_TRANS_CONV ``par (prefix (label (name "a")) nil)
                       (prefix (label (coname "a")) nil)``
\end{lstlisting}
This returns the following theorem, indeed describing all immediate
transitions of the process:
\begin{alltt}
\HOLTokenTurnstile{} \HOLConst{In} \HOLStringLit{a}\HOLSymConst{..}\HOLConst{nil} \HOLSymConst{\ensuremath{\parallel}} \HOLConst{Out} \HOLStringLit{a}\HOLSymConst{..}\HOLConst{nil} \HOLTokenTransBegin\HOLFreeVar{u}\HOLTokenTransEnd \HOLFreeVar{E} \HOLSymConst{\HOLTokenEquiv{}}
   ((\HOLFreeVar{u} \HOLSymConst{=} \HOLConst{In} \HOLStringLit{a}) \HOLSymConst{\HOLTokenConj{}} (\HOLFreeVar{E} \HOLSymConst{=} \HOLConst{nil} \HOLSymConst{\ensuremath{\parallel}} \HOLConst{Out} \HOLStringLit{a}\HOLSymConst{..}\HOLConst{nil}) \HOLSymConst{\HOLTokenDisj{}}
    (\HOLFreeVar{u} \HOLSymConst{=} \HOLConst{Out} \HOLStringLit{a}) \HOLSymConst{\HOLTokenConj{}} (\HOLFreeVar{E} \HOLSymConst{=} \HOLConst{In} \HOLStringLit{a}\HOLSymConst{..}\HOLConst{nil} \HOLSymConst{\ensuremath{\parallel}} \HOLConst{nil})) \HOLSymConst{\HOLTokenDisj{}}
   (\HOLFreeVar{u} \HOLSymConst{=} \HOLSymConst{\ensuremath{\tau}}) \HOLSymConst{\HOLTokenConj{}} (\HOLFreeVar{E} \HOLSymConst{=} \HOLConst{nil} \HOLSymConst{\ensuremath{\parallel}} \HOLConst{nil})\hfill{[Example.ex_A]}
\end{alltt}

%%%% -*- Mode: LaTeX -*-
%%
%% This is the draft of the 2nd part of EXPRESS/SOS 2018 paper, co-authored by
%% Prof. Davide Sangiorgi and Chun Tian.

\subsection{Context, guardness and (pre)congruence}

% We need to find a suitable formal definition of 
% context. There're multiple ways. Here 

We have chosen to use $\lambda$-expressions (typed
$CCS\rightarrow CCS$) to represent  (multi-hole) 
contexts. The definition is inductive:
\begin{alltt}
\HOLTokenTurnstile{} \HOLConst{CONTEXT} (\HOLTokenLambda{}\HOLBoundVar{t}. \HOLBoundVar{t}) \HOLSymConst{\HOLTokenConj{}} (\HOLSymConst{\HOLTokenForall{}}\HOLBoundVar{p}. \HOLConst{CONTEXT} (\HOLTokenLambda{}\HOLBoundVar{t}. \HOLBoundVar{p})) \HOLSymConst{\HOLTokenConj{}}
   (\HOLSymConst{\HOLTokenForall{}}\HOLBoundVar{a} \HOLBoundVar{e}. \HOLConst{CONTEXT} \HOLBoundVar{e} \HOLSymConst{\HOLTokenImp{}} \HOLConst{CONTEXT} (\HOLTokenLambda{}\HOLBoundVar{t}. \HOLBoundVar{a}\HOLSymConst{..}\HOLBoundVar{e} \HOLBoundVar{t})) \HOLSymConst{\HOLTokenConj{}}
   (\HOLSymConst{\HOLTokenForall{}}\HOLBoundVar{e\sb{\mathrm{1}}} \HOLBoundVar{e\sb{\mathrm{2}}}. \HOLConst{CONTEXT} \HOLBoundVar{e\sb{\mathrm{1}}} \HOLSymConst{\HOLTokenConj{}} \HOLConst{CONTEXT} \HOLBoundVar{e\sb{\mathrm{2}}} \HOLSymConst{\HOLTokenImp{}} \HOLConst{CONTEXT} (\HOLTokenLambda{}\HOLBoundVar{t}. \HOLBoundVar{e\sb{\mathrm{1}}} \HOLBoundVar{t} \HOLSymConst{+} \HOLBoundVar{e\sb{\mathrm{2}}} \HOLBoundVar{t})) \HOLSymConst{\HOLTokenConj{}}
   (\HOLSymConst{\HOLTokenForall{}}\HOLBoundVar{e\sb{\mathrm{1}}} \HOLBoundVar{e\sb{\mathrm{2}}}. \HOLConst{CONTEXT} \HOLBoundVar{e\sb{\mathrm{1}}} \HOLSymConst{\HOLTokenConj{}} \HOLConst{CONTEXT} \HOLBoundVar{e\sb{\mathrm{2}}} \HOLSymConst{\HOLTokenImp{}} \HOLConst{CONTEXT} (\HOLTokenLambda{}\HOLBoundVar{t}. \HOLBoundVar{e\sb{\mathrm{1}}} \HOLBoundVar{t} \HOLSymConst{\ensuremath{\parallel}} \HOLBoundVar{e\sb{\mathrm{2}}} \HOLBoundVar{t})) \HOLSymConst{\HOLTokenConj{}}
   (\HOLSymConst{\HOLTokenForall{}}\HOLBoundVar{L} \HOLBoundVar{e}. \HOLConst{CONTEXT} \HOLBoundVar{e} \HOLSymConst{\HOLTokenImp{}} \HOLConst{CONTEXT} (\HOLTokenLambda{}\HOLBoundVar{t}. \HOLSymConst{\ensuremath{\nu}} \HOLBoundVar{L} (\HOLBoundVar{e} \HOLBoundVar{t}))) \HOLSymConst{\HOLTokenConj{}}
   \HOLSymConst{\HOLTokenForall{}}\HOLBoundVar{rf} \HOLBoundVar{e}. \HOLConst{CONTEXT} \HOLBoundVar{e} \HOLSymConst{\HOLTokenImp{}} \HOLConst{CONTEXT} (\HOLTokenLambda{}\HOLBoundVar{t}. \HOLConst{relab} (\HOLBoundVar{e} \HOLBoundVar{t}) \HOLBoundVar{rf})\hfill{[CONTEXT_rules]}
\end{alltt}

Under above definition, we can formally define the concept of
``precongruence'' and ``congruence'' in the following ways:
\begin{alltt}
\HOLConst{precongruence} \HOLFreeVar{R} \HOLSymConst{\HOLTokenEquiv{}}
\HOLSymConst{\HOLTokenForall{}}\HOLBoundVar{x} \HOLBoundVar{y} \HOLBoundVar{ctx}. \HOLConst{CONTEXT} \HOLBoundVar{ctx} \HOLSymConst{\HOLTokenImp{}} \HOLFreeVar{R} \HOLBoundVar{x} \HOLBoundVar{y} \HOLSymConst{\HOLTokenImp{}} \HOLFreeVar{R} (\HOLBoundVar{ctx} \HOLBoundVar{x}) (\HOLBoundVar{ctx} \HOLBoundVar{y})\hfill{[precongruence_def]}
\HOLConst{congruence} \HOLFreeVar{R} \HOLSymConst{\HOLTokenEquiv{}} \HOLConst{equivalence} \HOLFreeVar{R} \HOLSymConst{\HOLTokenConj{}} \HOLConst{precongruence} \HOLFreeVar{R}\hfill{[congruence_def]}
\end{alltt}


\finish{
The definintion of congruence  requires the relation to be an equivalence. This is correct.
However the def of precongruence does not require the relation to be a preorder (ie. reflexive and transitive). Can it be added (only in the paper for now) ?
} 

A \emph{weakly guarded} context is a context in which each hole is
underneath a  prefix (where \texttt{WG} stands for weakly guarded):
\begin{alltt}
\HOLTokenTurnstile{} (\HOLSymConst{\HOLTokenForall{}}\HOLBoundVar{p}. \HOLConst{WG} (\HOLTokenLambda{}\HOLBoundVar{t}. \HOLBoundVar{p})) \HOLSymConst{\HOLTokenConj{}} (\HOLSymConst{\HOLTokenForall{}}\HOLBoundVar{a} \HOLBoundVar{e}. \HOLConst{CONTEXT} \HOLBoundVar{e} \HOLSymConst{\HOLTokenImp{}} \HOLConst{WG} (\HOLTokenLambda{}\HOLBoundVar{t}. \HOLBoundVar{a}\HOLSymConst{..}\HOLBoundVar{e} \HOLBoundVar{t})) \HOLSymConst{\HOLTokenConj{}}
   (\HOLSymConst{\HOLTokenForall{}}\HOLBoundVar{e\sb{\mathrm{1}}} \HOLBoundVar{e\sb{\mathrm{2}}}. \HOLConst{WG} \HOLBoundVar{e\sb{\mathrm{1}}} \HOLSymConst{\HOLTokenConj{}} \HOLConst{WG} \HOLBoundVar{e\sb{\mathrm{2}}} \HOLSymConst{\HOLTokenImp{}} \HOLConst{WG} (\HOLTokenLambda{}\HOLBoundVar{t}. \HOLBoundVar{e\sb{\mathrm{1}}} \HOLBoundVar{t} \HOLSymConst{+} \HOLBoundVar{e\sb{\mathrm{2}}} \HOLBoundVar{t})) \HOLSymConst{\HOLTokenConj{}}
   (\HOLSymConst{\HOLTokenForall{}}\HOLBoundVar{e\sb{\mathrm{1}}} \HOLBoundVar{e\sb{\mathrm{2}}}. \HOLConst{WG} \HOLBoundVar{e\sb{\mathrm{1}}} \HOLSymConst{\HOLTokenConj{}} \HOLConst{WG} \HOLBoundVar{e\sb{\mathrm{2}}} \HOLSymConst{\HOLTokenImp{}} \HOLConst{WG} (\HOLTokenLambda{}\HOLBoundVar{t}. \HOLBoundVar{e\sb{\mathrm{1}}} \HOLBoundVar{t} \HOLSymConst{\ensuremath{\parallel}} \HOLBoundVar{e\sb{\mathrm{2}}} \HOLBoundVar{t})) \HOLSymConst{\HOLTokenConj{}}
   (\HOLSymConst{\HOLTokenForall{}}\HOLBoundVar{L} \HOLBoundVar{e}. \HOLConst{WG} \HOLBoundVar{e} \HOLSymConst{\HOLTokenImp{}} \HOLConst{WG} (\HOLTokenLambda{}\HOLBoundVar{t}. \HOLSymConst{\ensuremath{\nu}} \HOLBoundVar{L} (\HOLBoundVar{e} \HOLBoundVar{t}))) \HOLSymConst{\HOLTokenConj{}}
   \HOLSymConst{\HOLTokenForall{}}\HOLBoundVar{rf} \HOLBoundVar{e}. \HOLConst{WG} \HOLBoundVar{e} \HOLSymConst{\HOLTokenImp{}} \HOLConst{WG} (\HOLTokenLambda{}\HOLBoundVar{t}. \HOLConst{relab} (\HOLBoundVar{e} \HOLBoundVar{t}) \HOLBoundVar{rf})\hfill{[WG_rules]}
\end{alltt}
% (Notice the differences between a weak guarded context and a normal
% one: $\lambda t. t$ is not weakly guarded as the variable is directly
% exposed without any prefixed action. And $\lambda t. a.e[t]$ is weakly
% guarded as long as $e[\cdot]$ is a context, not necessary weakly guarded.)

In the same manner we can  define \emph{strongly guarded contexts}
(\texttt{SG}), sequential contexts (\texttt{SEQ}), and their
variants without  sums (\texttt{GCONTEXT}, \texttt{WGS},
\texttt{GSEQ}). Some lemmas about their relationships have very long
(but trivial) formal proofs due to multiple levels of
inductions on context structures.%  For example, one such 

\finish{above: not consistent with the terminology in
  the first part; we should change part 1 to use  `strongly guarded'}
\finish{also: in part 1 the theorems about unique solutions of plain
  contractions say that sums can appear but only in their guarded
  form; is this what is done in the formalisation? (this is a minor
  point, we can adjust it later, since we do not show here the HOL definitions)}  


% lemma says, for any  context $E$ which is both strongly guarded
% and sequential (no direct sums) and another sequential context $H$ (no
% direct sums), the composition $H \circ E$ is still both strongly
% guarded and sequential (no direct sums):
% \begin{alltt}
% \HOLTokenTurnstile{} \HOLSymConst{\HOLTokenForall{}}\HOLBoundVar{E}. \HOLConst{SG} \HOLBoundVar{E} \HOLSymConst{\HOLTokenConj{}} \HOLConst{GSEQ} \HOLBoundVar{E} \HOLSymConst{\HOLTokenImp{}} \HOLSymConst{\HOLTokenForall{}}\HOLBoundVar{H}. \HOLConst{GSEQ} \HOLBoundVar{H} \HOLSymConst{\HOLTokenImp{}} \HOLConst{SG} (\HOLBoundVar{H} \HOLSymConst{\HOLTokenCompose} \HOLBoundVar{E}) \HOLSymConst{\HOLTokenConj{}} \HOLConst{GSEQ} (\HOLBoundVar{H} \HOLSymConst{\HOLTokenCompose} \HOLBoundVar{E})
% \end{alltt}

% \subsection{Milner's ``unique solution of equations'' theorems}

% Once the representation issue of CCS equations is resolved, the actual
% proofs of Milner's unique solution of equations theorems is not very
% interesting from the view of theorem proving, although it's a 
% precise proof engineering work producing quite long proofs.
% Since we have chosen to use  context to represent
% single-variable equations, an equation like $P \sim E\{P/X\}$ now
% becomes $P \sim E[P]$, in which $E$ is a (multi-hole) 
% context, and there's no need to say it ``contains at most the variable
% $X$'' any more, as equation variable doesn't appear in $E$ at
% all. Under these simplifications, the formal proof of Milner's unique
% solution of equations theorem for $\sim$ is a literal mapping for informal
% proofs based on bisimulation upto $\sim$, induction and case analysis
% of weakly guarded contexts. Below is the formal version of Lemma 4.13
% and Proposition 4.14 in Milner's book:

% \begin{alltt}
% STRONG_UNIQUE_SOLUTION:
% \end{alltt}

% \begin{alltt}
% WEAK_UNIQUE_SOLUTION:
% \end{alltt}

% \begin{alltt}
% OBS_UNIQUE_SOLUTION:
% \end{alltt}

%%%% -*- Mode: LaTeX -*-
%%
%% This is the draft of the 2nd part of EXPRESS/SOS 2018 paper, coauthored by
%% Prof. Davide Sangiorgi and Chun Tian.

\subsection{Bisimulation and Bisimilarity}
\label{ss:bb}

A highlight of this formalization project is the simplified definitions of
  bisimilarities using the new coinductive relation package of
  HOL4. Without this package, bisimilaries can still be defined in HOL, but
  proving their properties would be more complicated. Below we explain
  how the weak bisimulation (and bisimilarity) is defined. The way to
  define strong bisimulation (and bisimilarity) and other concepts
  (expansion, contraction, etc.) are in the same manner.

To define (weak) bisimilarity, we need to define weak transitions of CCS processes. 
First of all, a (possibly empty) sequence of $\tau$-transitions between
two processes is defined as a new relation \texttt{EPS}
($\overset{\epsilon}{\Longrightarrow}$), which is the
reflexive transitive closure (RTC, denoted by \mbox{\color{blue}{$^*$}} in
HOL4) of ordinary $\tau$-transitions of CCS processes:
\begin{alltt}
\HOLConst{EPS} \HOLTokenDefEquality{} \ensuremath{(}\HOLTokenLambda{}\HOLBoundVar{E} \ensuremath{\HOLBoundVar{E}\sp{\prime}}. \HOLBoundVar{E} \HOLTokenTransBegin\HOLSymConst{\ensuremath{\tau}}\HOLTokenTransEnd \ensuremath{\HOLBoundVar{E}\sp{\prime}}\ensuremath{)}\HOLSymConst{\HOLTokenSupStar{}}\hfill{[EPS_def]}
\end{alltt}
Then we can define a weak transition as an ordinary transition wrapped by
two $\epsilon$-transitions:
\begin{alltt}
\HOLFreeVar{E} \HOLTokenWeakTransBegin\HOLFreeVar{u}\HOLTokenWeakTransEnd \ensuremath{\HOLFreeVar{E}\sp{\prime}} \HOLTokenDefEquality{} \HOLSymConst{\HOLTokenExists{}}\ensuremath{\HOLBoundVar{E}\sb{\mathrm{1}}} \ensuremath{\HOLBoundVar{E}\sb{\mathrm{2}}}. \HOLFreeVar{E} \HOLSymConst{\HOLTokenEPS} \ensuremath{\HOLBoundVar{E}\sb{\mathrm{1}}} \HOLSymConst{\HOLTokenConj{}} \ensuremath{\HOLBoundVar{E}\sb{\mathrm{1}}} \HOLTokenTransBegin\HOLFreeVar{u}\HOLTokenTransEnd \ensuremath{\HOLBoundVar{E}\sb{\mathrm{2}}} \HOLSymConst{\HOLTokenConj{}} \ensuremath{\HOLBoundVar{E}\sb{\mathrm{2}}} \HOLSymConst{\HOLTokenEPS} \ensuremath{\HOLFreeVar{E}\sp{\prime}}\hfill{[WEAK_TRANS]}
\end{alltt}

The definition of weak bisimulation is based on weak and $\epsilon$--transitions:
\begin{alltt}
\HOLConst{WEAK_BISIM} \HOLFreeVar{Wbsm} \HOLTokenDefEquality{}
  \HOLSymConst{\HOLTokenForall{}}\HOLBoundVar{E} \ensuremath{\HOLBoundVar{E}\sp{\prime}}.
      \HOLFreeVar{Wbsm} \HOLBoundVar{E} \ensuremath{\HOLBoundVar{E}\sp{\prime}} \HOLSymConst{\HOLTokenImp{}}
      \ensuremath{(}\HOLSymConst{\HOLTokenForall{}}\HOLBoundVar{l}.
           \ensuremath{(}\HOLSymConst{\HOLTokenForall{}}\ensuremath{\HOLBoundVar{E}\sb{\mathrm{1}}}. \HOLBoundVar{E} \HOLTokenTransBegin\HOLConst{label} \HOLBoundVar{l}\HOLTokenTransEnd \ensuremath{\HOLBoundVar{E}\sb{\mathrm{1}}} \HOLSymConst{\HOLTokenImp{}} \HOLSymConst{\HOLTokenExists{}}\ensuremath{\HOLBoundVar{E}\sb{\mathrm{2}}}. \ensuremath{\HOLBoundVar{E}\sp{\prime}} \HOLTokenWeakTransBegin\HOLConst{label} \HOLBoundVar{l}\HOLTokenWeakTransEnd \ensuremath{\HOLBoundVar{E}\sb{\mathrm{2}}} \HOLSymConst{\HOLTokenConj{}} \HOLFreeVar{Wbsm} \ensuremath{\HOLBoundVar{E}\sb{\mathrm{1}}} \ensuremath{\HOLBoundVar{E}\sb{\mathrm{2}}}\ensuremath{)} \HOLSymConst{\HOLTokenConj{}}
           \HOLSymConst{\HOLTokenForall{}}\ensuremath{\HOLBoundVar{E}\sb{\mathrm{2}}}. \ensuremath{\HOLBoundVar{E}\sp{\prime}} \HOLTokenTransBegin\HOLConst{label} \HOLBoundVar{l}\HOLTokenTransEnd \ensuremath{\HOLBoundVar{E}\sb{\mathrm{2}}} \HOLSymConst{\HOLTokenImp{}} \HOLSymConst{\HOLTokenExists{}}\ensuremath{\HOLBoundVar{E}\sb{\mathrm{1}}}. \HOLBoundVar{E} \HOLTokenWeakTransBegin\HOLConst{label} \HOLBoundVar{l}\HOLTokenWeakTransEnd \ensuremath{\HOLBoundVar{E}\sb{\mathrm{1}}} \HOLSymConst{\HOLTokenConj{}} \HOLFreeVar{Wbsm} \ensuremath{\HOLBoundVar{E}\sb{\mathrm{1}}} \ensuremath{\HOLBoundVar{E}\sb{\mathrm{2}}}\ensuremath{)} \HOLSymConst{\HOLTokenConj{}}
      \ensuremath{(}\HOLSymConst{\HOLTokenForall{}}\ensuremath{\HOLBoundVar{E}\sb{\mathrm{1}}}. \HOLBoundVar{E} \HOLTokenTransBegin\HOLSymConst{\ensuremath{\tau}}\HOLTokenTransEnd \ensuremath{\HOLBoundVar{E}\sb{\mathrm{1}}} \HOLSymConst{\HOLTokenImp{}} \HOLSymConst{\HOLTokenExists{}}\ensuremath{\HOLBoundVar{E}\sb{\mathrm{2}}}. \ensuremath{\HOLBoundVar{E}\sp{\prime}} \HOLSymConst{\HOLTokenEPS} \ensuremath{\HOLBoundVar{E}\sb{\mathrm{2}}} \HOLSymConst{\HOLTokenConj{}} \HOLFreeVar{Wbsm} \ensuremath{\HOLBoundVar{E}\sb{\mathrm{1}}} \ensuremath{\HOLBoundVar{E}\sb{\mathrm{2}}}\ensuremath{)} \HOLSymConst{\HOLTokenConj{}}
      \HOLSymConst{\HOLTokenForall{}}\ensuremath{\HOLBoundVar{E}\sb{\mathrm{2}}}. \ensuremath{\HOLBoundVar{E}\sp{\prime}} \HOLTokenTransBegin\HOLSymConst{\ensuremath{\tau}}\HOLTokenTransEnd \ensuremath{\HOLBoundVar{E}\sb{\mathrm{2}}} \HOLSymConst{\HOLTokenImp{}} \HOLSymConst{\HOLTokenExists{}}\ensuremath{\HOLBoundVar{E}\sb{\mathrm{1}}}. \HOLBoundVar{E} \HOLSymConst{\HOLTokenEPS} \ensuremath{\HOLBoundVar{E}\sb{\mathrm{1}}} \HOLSymConst{\HOLTokenConj{}} \HOLFreeVar{Wbsm} \ensuremath{\HOLBoundVar{E}\sb{\mathrm{1}}} \ensuremath{\HOLBoundVar{E}\sb{\mathrm{2}}}\hfill{[WEAK_BISIM]}
\end{alltt}

We can prove that the identity relation is a bisimulation, that
bisimulation is preserved by inversion, composition, and union. 
The definition of weak bisimilarity can be generated with the
following scripts:
\begin{lstlisting}
CoInductive WEAK_EQUIV :
    !(E :('a, 'b) CCS) (E' :('a, 'b) CCS).
       (!l.
         (!E1. TRANS E  (label l) E1 ==>
               (?E2. WEAK_TRANS E' (label l) E2 /\ WEAK_EQUIV E1 E2)) /\
         (!E2. TRANS E' (label l) E2 ==>
               (?E1. WEAK_TRANS E  (label l) E1 /\ WEAK_EQUIV E1 E2))) /\
       (!E1. TRANS E  tau E1 ==> (?E2. EPS E' E2 /\ WEAK_EQUIV E1 E2)) /\
       (!E2. TRANS E' tau E2 ==> (?E1. EPS E  E1 /\ WEAK_EQUIV E1 E2))
      ==> WEAK_EQUIV E E'
End
\end{lstlisting}
Like the case of \HOLinline{\HOLConst{TRANS}}, a successful invocation of the coinductive definitional principle returns three
important theorems (\emph{rules}, \emph{coind} and \emph{cases}):
\begin{itemize}
\item \emph{rules} is a conjunction of implications that will be the
    same as the input term:
\begin{alltt}
\HOLTokenTurnstile{} \HOLSymConst{\HOLTokenForall{}}\HOLBoundVar{E} \ensuremath{\HOLBoundVar{E}\sp{\prime}}.
       \ensuremath{(}\HOLSymConst{\HOLTokenForall{}}\HOLBoundVar{l}.
            \ensuremath{(}\HOLSymConst{\HOLTokenForall{}}\ensuremath{\HOLBoundVar{E}\sb{\mathrm{1}}}. \HOLBoundVar{E} \HOLTokenTransBegin\HOLConst{label} \HOLBoundVar{l}\HOLTokenTransEnd \ensuremath{\HOLBoundVar{E}\sb{\mathrm{1}}} \HOLSymConst{\HOLTokenImp{}} \HOLSymConst{\HOLTokenExists{}}\ensuremath{\HOLBoundVar{E}\sb{\mathrm{2}}}. \ensuremath{\HOLBoundVar{E}\sp{\prime}} \HOLTokenWeakTransBegin\HOLConst{label} \HOLBoundVar{l}\HOLTokenWeakTransEnd \ensuremath{\HOLBoundVar{E}\sb{\mathrm{2}}} \HOLSymConst{\HOLTokenConj{}} \ensuremath{\HOLBoundVar{E}\sb{\mathrm{1}}} \HOLSymConst{\HOLTokenWeakEQ} \ensuremath{\HOLBoundVar{E}\sb{\mathrm{2}}}\ensuremath{)} \HOLSymConst{\HOLTokenConj{}}
            \HOLSymConst{\HOLTokenForall{}}\ensuremath{\HOLBoundVar{E}\sb{\mathrm{2}}}. \ensuremath{\HOLBoundVar{E}\sp{\prime}} \HOLTokenTransBegin\HOLConst{label} \HOLBoundVar{l}\HOLTokenTransEnd \ensuremath{\HOLBoundVar{E}\sb{\mathrm{2}}} \HOLSymConst{\HOLTokenImp{}} \HOLSymConst{\HOLTokenExists{}}\ensuremath{\HOLBoundVar{E}\sb{\mathrm{1}}}. \HOLBoundVar{E} \HOLTokenWeakTransBegin\HOLConst{label} \HOLBoundVar{l}\HOLTokenWeakTransEnd \ensuremath{\HOLBoundVar{E}\sb{\mathrm{1}}} \HOLSymConst{\HOLTokenConj{}} \ensuremath{\HOLBoundVar{E}\sb{\mathrm{1}}} \HOLSymConst{\HOLTokenWeakEQ} \ensuremath{\HOLBoundVar{E}\sb{\mathrm{2}}}\ensuremath{)} \HOLSymConst{\HOLTokenConj{}}
       \ensuremath{(}\HOLSymConst{\HOLTokenForall{}}\ensuremath{\HOLBoundVar{E}\sb{\mathrm{1}}}. \HOLBoundVar{E} \HOLTokenTransBegin\HOLSymConst{\ensuremath{\tau}}\HOLTokenTransEnd \ensuremath{\HOLBoundVar{E}\sb{\mathrm{1}}} \HOLSymConst{\HOLTokenImp{}} \HOLSymConst{\HOLTokenExists{}}\ensuremath{\HOLBoundVar{E}\sb{\mathrm{2}}}. \ensuremath{\HOLBoundVar{E}\sp{\prime}} \HOLSymConst{\HOLTokenEPS} \ensuremath{\HOLBoundVar{E}\sb{\mathrm{2}}} \HOLSymConst{\HOLTokenConj{}} \ensuremath{\HOLBoundVar{E}\sb{\mathrm{1}}} \HOLSymConst{\HOLTokenWeakEQ} \ensuremath{\HOLBoundVar{E}\sb{\mathrm{2}}}\ensuremath{)} \HOLSymConst{\HOLTokenConj{}}
       \ensuremath{(}\HOLSymConst{\HOLTokenForall{}}\ensuremath{\HOLBoundVar{E}\sb{\mathrm{2}}}. \ensuremath{\HOLBoundVar{E}\sp{\prime}} \HOLTokenTransBegin\HOLSymConst{\ensuremath{\tau}}\HOLTokenTransEnd \ensuremath{\HOLBoundVar{E}\sb{\mathrm{2}}} \HOLSymConst{\HOLTokenImp{}} \HOLSymConst{\HOLTokenExists{}}\ensuremath{\HOLBoundVar{E}\sb{\mathrm{1}}}. \HOLBoundVar{E} \HOLSymConst{\HOLTokenEPS} \ensuremath{\HOLBoundVar{E}\sb{\mathrm{1}}} \HOLSymConst{\HOLTokenConj{}} \ensuremath{\HOLBoundVar{E}\sb{\mathrm{1}}} \HOLSymConst{\HOLTokenWeakEQ} \ensuremath{\HOLBoundVar{E}\sb{\mathrm{2}}}\ensuremath{)} \HOLSymConst{\HOLTokenImp{}}
       \HOLBoundVar{E} \HOLSymConst{\HOLTokenWeakEQ} \ensuremath{\HOLBoundVar{E}\sp{\prime}}\hfill{[WEAK_EQUIV_rules]}
\end{alltt}
\item \emph{coind} is the coinduction principle for the relation.
\begin{alltt}
\HOLTokenTurnstile{} \HOLSymConst{\HOLTokenForall{}}\ensuremath{\HOLBoundVar{WEAK\HOLTokenUnderscore{}EQUIV}\sp{\prime}}.
       \ensuremath{(}\HOLSymConst{\HOLTokenForall{}}\ensuremath{\HOLBoundVar{a}\sb{\mathrm{0}}} \ensuremath{\HOLBoundVar{a}\sb{\mathrm{1}}}.
            \ensuremath{\HOLBoundVar{WEAK\HOLTokenUnderscore{}EQUIV}\sp{\prime}} \ensuremath{\HOLBoundVar{a}\sb{\mathrm{0}}} \ensuremath{\HOLBoundVar{a}\sb{\mathrm{1}}} \HOLSymConst{\HOLTokenImp{}}
            \ensuremath{(}\HOLSymConst{\HOLTokenForall{}}\HOLBoundVar{l}.
                 \ensuremath{(}\HOLSymConst{\HOLTokenForall{}}\ensuremath{\HOLBoundVar{E}\sb{\mathrm{1}}}.
                      \ensuremath{\HOLBoundVar{a}\sb{\mathrm{0}}} \HOLTokenTransBegin\HOLConst{label} \HOLBoundVar{l}\HOLTokenTransEnd \ensuremath{\HOLBoundVar{E}\sb{\mathrm{1}}} \HOLSymConst{\HOLTokenImp{}}
                      \HOLSymConst{\HOLTokenExists{}}\ensuremath{\HOLBoundVar{E}\sb{\mathrm{2}}}. \ensuremath{\HOLBoundVar{a}\sb{\mathrm{1}}} \HOLTokenWeakTransBegin\HOLConst{label} \HOLBoundVar{l}\HOLTokenWeakTransEnd \ensuremath{\HOLBoundVar{E}\sb{\mathrm{2}}} \HOLSymConst{\HOLTokenConj{}} \ensuremath{\HOLBoundVar{WEAK\HOLTokenUnderscore{}EQUIV}\sp{\prime}} \ensuremath{\HOLBoundVar{E}\sb{\mathrm{1}}} \ensuremath{\HOLBoundVar{E}\sb{\mathrm{2}}}\ensuremath{)} \HOLSymConst{\HOLTokenConj{}}
                 \HOLSymConst{\HOLTokenForall{}}\ensuremath{\HOLBoundVar{E}\sb{\mathrm{2}}}.
                     \ensuremath{\HOLBoundVar{a}\sb{\mathrm{1}}} \HOLTokenTransBegin\HOLConst{label} \HOLBoundVar{l}\HOLTokenTransEnd \ensuremath{\HOLBoundVar{E}\sb{\mathrm{2}}} \HOLSymConst{\HOLTokenImp{}}
                     \HOLSymConst{\HOLTokenExists{}}\ensuremath{\HOLBoundVar{E}\sb{\mathrm{1}}}. \ensuremath{\HOLBoundVar{a}\sb{\mathrm{0}}} \HOLTokenWeakTransBegin\HOLConst{label} \HOLBoundVar{l}\HOLTokenWeakTransEnd \ensuremath{\HOLBoundVar{E}\sb{\mathrm{1}}} \HOLSymConst{\HOLTokenConj{}} \ensuremath{\HOLBoundVar{WEAK\HOLTokenUnderscore{}EQUIV}\sp{\prime}} \ensuremath{\HOLBoundVar{E}\sb{\mathrm{1}}} \ensuremath{\HOLBoundVar{E}\sb{\mathrm{2}}}\ensuremath{)} \HOLSymConst{\HOLTokenConj{}}
            \ensuremath{(}\HOLSymConst{\HOLTokenForall{}}\ensuremath{\HOLBoundVar{E}\sb{\mathrm{1}}}. \ensuremath{\HOLBoundVar{a}\sb{\mathrm{0}}} \HOLTokenTransBegin\HOLSymConst{\ensuremath{\tau}}\HOLTokenTransEnd \ensuremath{\HOLBoundVar{E}\sb{\mathrm{1}}} \HOLSymConst{\HOLTokenImp{}} \HOLSymConst{\HOLTokenExists{}}\ensuremath{\HOLBoundVar{E}\sb{\mathrm{2}}}. \ensuremath{\HOLBoundVar{a}\sb{\mathrm{1}}} \HOLSymConst{\HOLTokenEPS} \ensuremath{\HOLBoundVar{E}\sb{\mathrm{2}}} \HOLSymConst{\HOLTokenConj{}} \ensuremath{\HOLBoundVar{WEAK\HOLTokenUnderscore{}EQUIV}\sp{\prime}} \ensuremath{\HOLBoundVar{E}\sb{\mathrm{1}}} \ensuremath{\HOLBoundVar{E}\sb{\mathrm{2}}}\ensuremath{)} \HOLSymConst{\HOLTokenConj{}}
            \HOLSymConst{\HOLTokenForall{}}\ensuremath{\HOLBoundVar{E}\sb{\mathrm{2}}}. \ensuremath{\HOLBoundVar{a}\sb{\mathrm{1}}} \HOLTokenTransBegin\HOLSymConst{\ensuremath{\tau}}\HOLTokenTransEnd \ensuremath{\HOLBoundVar{E}\sb{\mathrm{2}}} \HOLSymConst{\HOLTokenImp{}} \HOLSymConst{\HOLTokenExists{}}\ensuremath{\HOLBoundVar{E}\sb{\mathrm{1}}}. \ensuremath{\HOLBoundVar{a}\sb{\mathrm{0}}} \HOLSymConst{\HOLTokenEPS} \ensuremath{\HOLBoundVar{E}\sb{\mathrm{1}}} \HOLSymConst{\HOLTokenConj{}} \ensuremath{\HOLBoundVar{WEAK\HOLTokenUnderscore{}EQUIV}\sp{\prime}} \ensuremath{\HOLBoundVar{E}\sb{\mathrm{1}}} \ensuremath{\HOLBoundVar{E}\sb{\mathrm{2}}}\ensuremath{)} \HOLSymConst{\HOLTokenImp{}}
       \HOLSymConst{\HOLTokenForall{}}\ensuremath{\HOLBoundVar{a}\sb{\mathrm{0}}} \ensuremath{\HOLBoundVar{a}\sb{\mathrm{1}}}. \ensuremath{\HOLBoundVar{WEAK\HOLTokenUnderscore{}EQUIV}\sp{\prime}} \ensuremath{\HOLBoundVar{a}\sb{\mathrm{0}}} \ensuremath{\HOLBoundVar{a}\sb{\mathrm{1}}} \HOLSymConst{\HOLTokenImp{}} \ensuremath{\HOLBoundVar{a}\sb{\mathrm{0}}} \HOLSymConst{\HOLTokenWeakEQ} \ensuremath{\HOLBoundVar{a}\sb{\mathrm{1}}}\hfill{[WEAK_EQUIV_coind]}
\end{alltt}
\item \emph{cases} is the so-called `cases' or `inversion' theorem for
  the relations, and is used to decompose an element in the relation into the possible ways of
  obtaining it by the rules.
\begin{alltt}
\HOLTokenTurnstile{} \HOLSymConst{\HOLTokenForall{}}\ensuremath{\HOLBoundVar{a}\sb{\mathrm{0}}} \ensuremath{\HOLBoundVar{a}\sb{\mathrm{1}}}.
       \ensuremath{\HOLBoundVar{a}\sb{\mathrm{0}}} \HOLSymConst{\HOLTokenWeakEQ} \ensuremath{\HOLBoundVar{a}\sb{\mathrm{1}}} \HOLSymConst{\HOLTokenEquiv{}}
       \ensuremath{(}\HOLSymConst{\HOLTokenForall{}}\HOLBoundVar{l}.
            \ensuremath{(}\HOLSymConst{\HOLTokenForall{}}\ensuremath{\HOLBoundVar{E}\sb{\mathrm{1}}}. \ensuremath{\HOLBoundVar{a}\sb{\mathrm{0}}} \HOLTokenTransBegin\HOLConst{label} \HOLBoundVar{l}\HOLTokenTransEnd \ensuremath{\HOLBoundVar{E}\sb{\mathrm{1}}} \HOLSymConst{\HOLTokenImp{}} \HOLSymConst{\HOLTokenExists{}}\ensuremath{\HOLBoundVar{E}\sb{\mathrm{2}}}. \ensuremath{\HOLBoundVar{a}\sb{\mathrm{1}}} \HOLTokenWeakTransBegin\HOLConst{label} \HOLBoundVar{l}\HOLTokenWeakTransEnd \ensuremath{\HOLBoundVar{E}\sb{\mathrm{2}}} \HOLSymConst{\HOLTokenConj{}} \ensuremath{\HOLBoundVar{E}\sb{\mathrm{1}}} \HOLSymConst{\HOLTokenWeakEQ} \ensuremath{\HOLBoundVar{E}\sb{\mathrm{2}}}\ensuremath{)} \HOLSymConst{\HOLTokenConj{}}
            \HOLSymConst{\HOLTokenForall{}}\ensuremath{\HOLBoundVar{E}\sb{\mathrm{2}}}. \ensuremath{\HOLBoundVar{a}\sb{\mathrm{1}}} \HOLTokenTransBegin\HOLConst{label} \HOLBoundVar{l}\HOLTokenTransEnd \ensuremath{\HOLBoundVar{E}\sb{\mathrm{2}}} \HOLSymConst{\HOLTokenImp{}} \HOLSymConst{\HOLTokenExists{}}\ensuremath{\HOLBoundVar{E}\sb{\mathrm{1}}}. \ensuremath{\HOLBoundVar{a}\sb{\mathrm{0}}} \HOLTokenWeakTransBegin\HOLConst{label} \HOLBoundVar{l}\HOLTokenWeakTransEnd \ensuremath{\HOLBoundVar{E}\sb{\mathrm{1}}} \HOLSymConst{\HOLTokenConj{}} \ensuremath{\HOLBoundVar{E}\sb{\mathrm{1}}} \HOLSymConst{\HOLTokenWeakEQ} \ensuremath{\HOLBoundVar{E}\sb{\mathrm{2}}}\ensuremath{)} \HOLSymConst{\HOLTokenConj{}}
       \ensuremath{(}\HOLSymConst{\HOLTokenForall{}}\ensuremath{\HOLBoundVar{E}\sb{\mathrm{1}}}. \ensuremath{\HOLBoundVar{a}\sb{\mathrm{0}}} \HOLTokenTransBegin\HOLSymConst{\ensuremath{\tau}}\HOLTokenTransEnd \ensuremath{\HOLBoundVar{E}\sb{\mathrm{1}}} \HOLSymConst{\HOLTokenImp{}} \HOLSymConst{\HOLTokenExists{}}\ensuremath{\HOLBoundVar{E}\sb{\mathrm{2}}}. \ensuremath{\HOLBoundVar{a}\sb{\mathrm{1}}} \HOLSymConst{\HOLTokenEPS} \ensuremath{\HOLBoundVar{E}\sb{\mathrm{2}}} \HOLSymConst{\HOLTokenConj{}} \ensuremath{\HOLBoundVar{E}\sb{\mathrm{1}}} \HOLSymConst{\HOLTokenWeakEQ} \ensuremath{\HOLBoundVar{E}\sb{\mathrm{2}}}\ensuremath{)} \HOLSymConst{\HOLTokenConj{}}
       \HOLSymConst{\HOLTokenForall{}}\ensuremath{\HOLBoundVar{E}\sb{\mathrm{2}}}. \ensuremath{\HOLBoundVar{a}\sb{\mathrm{1}}} \HOLTokenTransBegin\HOLSymConst{\ensuremath{\tau}}\HOLTokenTransEnd \ensuremath{\HOLBoundVar{E}\sb{\mathrm{2}}} \HOLSymConst{\HOLTokenImp{}} \HOLSymConst{\HOLTokenExists{}}\ensuremath{\HOLBoundVar{E}\sb{\mathrm{1}}}. \ensuremath{\HOLBoundVar{a}\sb{\mathrm{0}}} \HOLSymConst{\HOLTokenEPS} \ensuremath{\HOLBoundVar{E}\sb{\mathrm{1}}} \HOLSymConst{\HOLTokenConj{}} \ensuremath{\HOLBoundVar{E}\sb{\mathrm{1}}} \HOLSymConst{\HOLTokenWeakEQ} \ensuremath{\HOLBoundVar{E}\sb{\mathrm{2}}}\hfill{[WEAK_EQUIV_cases]}
\end{alltt}
\end{itemize}

The coinduction principle \texttt{WEAK_EQUIV_coind} says that any
bisimulation is contained in the resulting relation (i.e.~it is
largest), but it didn't constrain the resulting relation in the set of
fixed points (e.g.~even the universal relation---the set of all
pairs---would fit with this theorem); the
purpose of \texttt{WEAK_EQUIV_cases} is to
further assert that the resulting relation is indeed a
fixed point. Thus \texttt{WEAK_EQUIV_coind} and \texttt{WEAK_EQUIV_cases}
together make sure that bisimilarity is the greatest
fixed point, as
the former contributes to ``greatest'' while the latter
contributes to ``fixed point''.
%
Without HOL's coinductive relation package, (weak) bisimilarity
would have to be defined by following literally
Def.~\ref{d:wb};  then other properties of bisimilarity, such
as the fixed-point property in \texttt{WEAK_EQUIV_cases}, would have to be
derived manually.

Finally, the original definition of \texttt{WEAK_EQUIV}
becomes a theorem:
\begin{alltt}
\HOLTokenTurnstile{} \HOLFreeVar{E} \HOLSymConst{\HOLTokenWeakEQ} \ensuremath{\HOLFreeVar{E}\sp{\prime}} \HOLSymConst{\HOLTokenEquiv{}} \HOLSymConst{\HOLTokenExists{}}\HOLBoundVar{Wbsm}. \HOLBoundVar{Wbsm} \HOLFreeVar{E} \ensuremath{\HOLFreeVar{E}\sp{\prime}} \HOLSymConst{\HOLTokenConj{}} \HOLConst{WEAK_BISIM} \HOLBoundVar{Wbsm}\hfill{[WEAK_EQUIV]}
\end{alltt}

The formal definition of rooted bisimilarity ($\rapprox$, \texttt{OBS_CONGR}) 
follows Definition~\ref{d:rootedBisimilarity}:
\begin{alltt}
   \HOLFreeVar{E} \HOLSymConst{\HOLTokenObsCongr} \ensuremath{\HOLFreeVar{E}\sp{\prime}} \HOLTokenDefEquality{}
     \HOLSymConst{\HOLTokenForall{}}\HOLBoundVar{u}.
         \ensuremath{(}\HOLSymConst{\HOLTokenForall{}}\ensuremath{\HOLBoundVar{E}\sb{\mathrm{1}}}. \HOLFreeVar{E} \HOLTokenTransBegin\HOLBoundVar{u}\HOLTokenTransEnd \ensuremath{\HOLBoundVar{E}\sb{\mathrm{1}}} \HOLSymConst{\HOLTokenImp{}} \HOLSymConst{\HOLTokenExists{}}\ensuremath{\HOLBoundVar{E}\sb{\mathrm{2}}}. \ensuremath{\HOLFreeVar{E}\sp{\prime}} \HOLTokenWeakTransBegin\HOLBoundVar{u}\HOLTokenWeakTransEnd \ensuremath{\HOLBoundVar{E}\sb{\mathrm{2}}} \HOLSymConst{\HOLTokenConj{}} \ensuremath{\HOLBoundVar{E}\sb{\mathrm{1}}} \HOLSymConst{\HOLTokenWeakEQ} \ensuremath{\HOLBoundVar{E}\sb{\mathrm{2}}}\ensuremath{)} \HOLSymConst{\HOLTokenConj{}}
         \HOLSymConst{\HOLTokenForall{}}\ensuremath{\HOLBoundVar{E}\sb{\mathrm{2}}}. \ensuremath{\HOLFreeVar{E}\sp{\prime}} \HOLTokenTransBegin\HOLBoundVar{u}\HOLTokenTransEnd \ensuremath{\HOLBoundVar{E}\sb{\mathrm{2}}} \HOLSymConst{\HOLTokenImp{}} \HOLSymConst{\HOLTokenExists{}}\ensuremath{\HOLBoundVar{E}\sb{\mathrm{1}}}. \HOLFreeVar{E} \HOLTokenWeakTransBegin\HOLBoundVar{u}\HOLTokenWeakTransEnd \ensuremath{\HOLBoundVar{E}\sb{\mathrm{1}}} \HOLSymConst{\HOLTokenConj{}} \ensuremath{\HOLBoundVar{E}\sb{\mathrm{1}}} \HOLSymConst{\HOLTokenWeakEQ} \ensuremath{\HOLBoundVar{E}\sb{\mathrm{2}}}\hfill{[OBS_CONGR]}
\end{alltt}
Below is the formal version of Lemma~\ref{l:obsCongrByWeakBisim}, which is needed in the proof
of Theorem~\ref{t:rcontraBisimulationU}:
\begin{alltt}
\HOLTokenTurnstile{} \HOLConst{WEAK_BISIM} \HOLFreeVar{Wbsm} \HOLSymConst{\HOLTokenImp{}}
   \HOLSymConst{\HOLTokenForall{}}\HOLBoundVar{E} \ensuremath{\HOLBoundVar{E}\sp{\prime}}.
       \ensuremath{(}\HOLSymConst{\HOLTokenForall{}}\HOLBoundVar{u}.
            \ensuremath{(}\HOLSymConst{\HOLTokenForall{}}\ensuremath{\HOLBoundVar{E}\sb{\mathrm{1}}}. \HOLBoundVar{E} \HOLTokenTransBegin\HOLBoundVar{u}\HOLTokenTransEnd \ensuremath{\HOLBoundVar{E}\sb{\mathrm{1}}} \HOLSymConst{\HOLTokenImp{}} \HOLSymConst{\HOLTokenExists{}}\ensuremath{\HOLBoundVar{E}\sb{\mathrm{2}}}. \ensuremath{\HOLBoundVar{E}\sp{\prime}} \HOLTokenWeakTransBegin\HOLBoundVar{u}\HOLTokenWeakTransEnd \ensuremath{\HOLBoundVar{E}\sb{\mathrm{2}}} \HOLSymConst{\HOLTokenConj{}} \HOLFreeVar{Wbsm} \ensuremath{\HOLBoundVar{E}\sb{\mathrm{1}}} \ensuremath{\HOLBoundVar{E}\sb{\mathrm{2}}}\ensuremath{)} \HOLSymConst{\HOLTokenConj{}}
            \HOLSymConst{\HOLTokenForall{}}\ensuremath{\HOLBoundVar{E}\sb{\mathrm{2}}}. \ensuremath{\HOLBoundVar{E}\sp{\prime}} \HOLTokenTransBegin\HOLBoundVar{u}\HOLTokenTransEnd \ensuremath{\HOLBoundVar{E}\sb{\mathrm{2}}} \HOLSymConst{\HOLTokenImp{}} \HOLSymConst{\HOLTokenExists{}}\ensuremath{\HOLBoundVar{E}\sb{\mathrm{1}}}. \HOLBoundVar{E} \HOLTokenWeakTransBegin\HOLBoundVar{u}\HOLTokenWeakTransEnd \ensuremath{\HOLBoundVar{E}\sb{\mathrm{1}}} \HOLSymConst{\HOLTokenConj{}} \HOLFreeVar{Wbsm} \ensuremath{\HOLBoundVar{E}\sb{\mathrm{1}}} \ensuremath{\HOLBoundVar{E}\sb{\mathrm{2}}}\ensuremath{)} \HOLSymConst{\HOLTokenImp{}}
       \HOLBoundVar{E} \HOLSymConst{\HOLTokenObsCongr} \ensuremath{\HOLBoundVar{E}\sp{\prime}}\hfill{[OBS_CONGR_BY_WEAK_BISIM]}
\end{alltt}

On the relationship between (weak) bisimilarity and rooted bisimilarity, 
we have proved Deng's Lemma and Hennessy's Lemma
(Lemma 4.1 and 4.2 of~\citep[p.~176,~178]{Gorrieri:2015jt}):
\begin{alltt}
\HOLTokenTurnstile{} \HOLFreeVar{p} \HOLSymConst{\HOLTokenWeakEQ} \HOLFreeVar{q} \HOLSymConst{\HOLTokenImp{}} \ensuremath{(}\HOLSymConst{\HOLTokenExists{}}\ensuremath{\HOLBoundVar{p}\sp{\prime}}. \HOLFreeVar{p} \HOLTokenTransBegin\HOLSymConst{\ensuremath{\tau}}\HOLTokenTransEnd \ensuremath{\HOLBoundVar{p}\sp{\prime}} \HOLSymConst{\HOLTokenConj{}} \ensuremath{\HOLBoundVar{p}\sp{\prime}} \HOLSymConst{\HOLTokenWeakEQ} \HOLFreeVar{q}\ensuremath{)} \HOLSymConst{\HOLTokenDisj{}} \ensuremath{(}\HOLSymConst{\HOLTokenExists{}}\ensuremath{\HOLBoundVar{q}\sp{\prime}}. \HOLFreeVar{q} \HOLTokenTransBegin\HOLSymConst{\ensuremath{\tau}}\HOLTokenTransEnd \ensuremath{\HOLBoundVar{q}\sp{\prime}} \HOLSymConst{\HOLTokenConj{}} \HOLFreeVar{p} \HOLSymConst{\HOLTokenWeakEQ} \ensuremath{\HOLBoundVar{q}\sp{\prime}}\ensuremath{)} \HOLSymConst{\HOLTokenDisj{}} \HOLFreeVar{p} \HOLSymConst{\HOLTokenObsCongr} \HOLFreeVar{q}\hfill{[DENG_LEMMA]}
  
\HOLTokenTurnstile{} \HOLFreeVar{p} \HOLSymConst{\HOLTokenWeakEQ} \HOLFreeVar{q} \HOLSymConst{\HOLTokenEquiv{}} \HOLFreeVar{p} \HOLSymConst{\HOLTokenObsCongr} \HOLFreeVar{q} \HOLSymConst{\HOLTokenDisj{}} \HOLFreeVar{p} \HOLSymConst{\HOLTokenObsCongr} \HOLSymConst{\ensuremath{\tau}}\HOLSymConst{\ensuremath{\ldotp}}\HOLFreeVar{q} \HOLSymConst{\HOLTokenDisj{}} \HOLSymConst{\ensuremath{\tau}}\HOLSymConst{\ensuremath{\ldotp}}\HOLFreeVar{p} \HOLSymConst{\HOLTokenObsCongr} \HOLFreeVar{q}\hfill{[HENNESSY_LEMMA]}
\end{alltt}

\subsection{Algebraic Laws}
\label{ss:alaws}

Having formalised the definitions of strong bisimulation and strong bisimilarity,
we can derive \emph{algebraic laws} for the 
 bisimilarities. We only report a few laws for the sum operator:
\begin{alltt}
STRONG_SUM_IDEMP:          \HOLTokenTurnstile{} \HOLFreeVar{E} \HOLSymConst{\ensuremath{+}} \HOLFreeVar{E} \HOLSymConst{\HOLTokenStrongEQ} \HOLFreeVar{E}  
STRONG_SUM_COMM:           \HOLTokenTurnstile{} \HOLFreeVar{E} \HOLSymConst{\ensuremath{+}} \ensuremath{\HOLFreeVar{E}\sp{\prime}} \HOLSymConst{\HOLTokenStrongEQ} \ensuremath{\HOLFreeVar{E}\sp{\prime}} \HOLSymConst{\ensuremath{+}} \HOLFreeVar{E}
STRONG_SUM_IDENT_L:        \HOLTokenTurnstile{} \HOLConst{\ensuremath{\mathbf{0}}} \HOLSymConst{\ensuremath{+}} \HOLFreeVar{E} \HOLSymConst{\HOLTokenStrongEQ} \HOLFreeVar{E}
STRONG_SUM_IDENT_R:        \HOLTokenTurnstile{} \HOLFreeVar{E} \HOLSymConst{\ensuremath{+}} \HOLConst{\ensuremath{\mathbf{0}}} \HOLSymConst{\HOLTokenStrongEQ} \HOLFreeVar{E}
STRONG_SUM_ASSOC_R:        \HOLTokenTurnstile{} \HOLFreeVar{E} \HOLSymConst{\ensuremath{+}} \ensuremath{\HOLFreeVar{E}\sp{\prime}} \HOLSymConst{\ensuremath{+}} \ensuremath{\HOLFreeVar{E}\sp{\prime\prime}} \HOLSymConst{\HOLTokenStrongEQ} \HOLFreeVar{E} \HOLSymConst{\ensuremath{+}} \ensuremath{(}\ensuremath{\HOLFreeVar{E}\sp{\prime}} \HOLSymConst{\ensuremath{+}} \ensuremath{\HOLFreeVar{E}\sp{\prime\prime}}\ensuremath{)}
STRONG_SUM_ASSOC_L:        \HOLTokenTurnstile{} \HOLFreeVar{E} \HOLSymConst{\ensuremath{+}} \ensuremath{(}\ensuremath{\HOLFreeVar{E}\sp{\prime}} \HOLSymConst{\ensuremath{+}} \ensuremath{\HOLFreeVar{E}\sp{\prime\prime}}\ensuremath{)} \HOLSymConst{\HOLTokenStrongEQ} \HOLFreeVar{E} \HOLSymConst{\ensuremath{+}} \ensuremath{\HOLFreeVar{E}\sp{\prime}} \HOLSymConst{\ensuremath{+}} \ensuremath{\HOLFreeVar{E}\sp{\prime\prime}}
STRONG_SUM_MID_IDEMP:      \HOLTokenTurnstile{} \HOLFreeVar{E} \HOLSymConst{\ensuremath{+}} \ensuremath{\HOLFreeVar{E}\sp{\prime}} \HOLSymConst{\ensuremath{+}} \HOLFreeVar{E} \HOLSymConst{\HOLTokenStrongEQ} \ensuremath{\HOLFreeVar{E}\sp{\prime}} \HOLSymConst{\ensuremath{+}} \HOLFreeVar{E}
STRONG_LEFT_SUM_MID_IDEMP: \HOLTokenTurnstile{} \HOLFreeVar{E} \HOLSymConst{\ensuremath{+}} \ensuremath{\HOLFreeVar{E}\sp{\prime}} \HOLSymConst{\ensuremath{+}} \ensuremath{\HOLFreeVar{E}\sp{\prime\prime}} \HOLSymConst{\ensuremath{+}} \ensuremath{\HOLFreeVar{E}\sp{\prime}} \HOLSymConst{\HOLTokenStrongEQ} \HOLFreeVar{E} \HOLSymConst{\ensuremath{+}} \ensuremath{\HOLFreeVar{E}\sp{\prime\prime}} \HOLSymConst{\ensuremath{+}} \ensuremath{\HOLFreeVar{E}\sp{\prime}}
\end{alltt}

% Not all above theorems are primitive (in the sense of providing a
% minimal axiomatization set for proving all other strong algebraic
% laws). 
The first five of them are proven by constructing appropriate bisimulations,
and their formal proofs are written in
a goal-directed manner~\citep[Chapter 4]{holdesc}. On the other hand, the
last three algebraic laws are derived in a forward manner by applications of
previous proven laws (without directly using the SOS
inference rules and the definition of bisimulation).
 These algebraic laws also hold for weak bisimilarity and rooted
  bisimilarity, as these are coarser than strong bisimilarity. But
for weak bisimilarity and rooted bisimilarity, the following so-called
$\tau$-laws are further available:
\begin{alltt}
TAU1:      \HOLTokenTurnstile{} \HOLFreeVar{u}\HOLSymConst{\ensuremath{\ldotp}}\HOLSymConst{\ensuremath{\tau}}\HOLSymConst{\ensuremath{\ldotp}}\HOLFreeVar{E} \HOLSymConst{\HOLTokenObsCongr} \HOLFreeVar{u}\HOLSymConst{\ensuremath{\ldotp}}\HOLFreeVar{E}
TAU2:      \HOLTokenTurnstile{} \HOLFreeVar{E} \HOLSymConst{\ensuremath{+}} \HOLSymConst{\ensuremath{\tau}}\HOLSymConst{\ensuremath{\ldotp}}\HOLFreeVar{E} \HOLSymConst{\HOLTokenObsCongr} \HOLSymConst{\ensuremath{\tau}}\HOLSymConst{\ensuremath{\ldotp}}\HOLFreeVar{E}
TAU3:      \HOLTokenTurnstile{} \HOLFreeVar{u}\HOLSymConst{\ensuremath{\ldotp}}\ensuremath{(}\HOLFreeVar{E} \HOLSymConst{\ensuremath{+}} \HOLSymConst{\ensuremath{\tau}}\HOLSymConst{\ensuremath{\ldotp}}\ensuremath{\HOLFreeVar{E}\sp{\prime}}\ensuremath{)} \HOLSymConst{\ensuremath{+}} \HOLFreeVar{u}\HOLSymConst{\ensuremath{\ldotp}}\ensuremath{\HOLFreeVar{E}\sp{\prime}} \HOLSymConst{\HOLTokenObsCongr} \HOLFreeVar{u}\HOLSymConst{\ensuremath{\ldotp}}\ensuremath{(}\HOLFreeVar{E} \HOLSymConst{\ensuremath{+}} \HOLSymConst{\ensuremath{\tau}}\HOLSymConst{\ensuremath{\ldotp}}\ensuremath{\HOLFreeVar{E}\sp{\prime}}\ensuremath{)}
TAU_STRAT: \HOLTokenTurnstile{} \HOLFreeVar{E} \HOLSymConst{\ensuremath{+}} \HOLSymConst{\ensuremath{\tau}}\HOLSymConst{\ensuremath{\ldotp}}\ensuremath{(}\ensuremath{\HOLFreeVar{E}\sp{\prime}} \HOLSymConst{\ensuremath{+}} \HOLFreeVar{E}\ensuremath{)} \HOLSymConst{\HOLTokenObsCongr} \HOLSymConst{\ensuremath{\tau}}\HOLSymConst{\ensuremath{\ldotp}}\ensuremath{(}\ensuremath{\HOLFreeVar{E}\sp{\prime}} \HOLSymConst{\ensuremath{+}} \HOLFreeVar{E}\ensuremath{)}
TAU_WEAK:  \HOLTokenTurnstile{} \HOLSymConst{\ensuremath{\tau}}\HOLSymConst{\ensuremath{\ldotp}}\HOLFreeVar{E} \HOLSymConst{\HOLTokenWeakEQ} \HOLFreeVar{E}
\end{alltt}

\subsection{Expansion, Contraction and Rooted Contraction}

To formally define bisimulation expansion and contraction (and their preorders), we have
followed the same ways as in the case of strong and weak bisimilarities:
\begin{alltt}
\HOLConst{EXPANSION} \HOLFreeVar{Exp} \HOLTokenDefEquality{}
  \HOLSymConst{\HOLTokenForall{}}\HOLBoundVar{E} \ensuremath{\HOLBoundVar{E}\sp{\prime}}.
      \HOLFreeVar{Exp} \HOLBoundVar{E} \ensuremath{\HOLBoundVar{E}\sp{\prime}} \HOLSymConst{\HOLTokenImp{}}
      \ensuremath{(}\HOLSymConst{\HOLTokenForall{}}\HOLBoundVar{l}.
           \ensuremath{(}\HOLSymConst{\HOLTokenForall{}}\ensuremath{\HOLBoundVar{E}\sb{\mathrm{1}}}. \HOLBoundVar{E} \HOLTokenTransBegin\HOLConst{label} \HOLBoundVar{l}\HOLTokenTransEnd \ensuremath{\HOLBoundVar{E}\sb{\mathrm{1}}} \HOLSymConst{\HOLTokenImp{}} \HOLSymConst{\HOLTokenExists{}}\ensuremath{\HOLBoundVar{E}\sb{\mathrm{2}}}. \ensuremath{\HOLBoundVar{E}\sp{\prime}} \HOLTokenTransBegin\HOLConst{label} \HOLBoundVar{l}\HOLTokenTransEnd \ensuremath{\HOLBoundVar{E}\sb{\mathrm{2}}} \HOLSymConst{\HOLTokenConj{}} \HOLFreeVar{Exp} \ensuremath{\HOLBoundVar{E}\sb{\mathrm{1}}} \ensuremath{\HOLBoundVar{E}\sb{\mathrm{2}}}\ensuremath{)} \HOLSymConst{\HOLTokenConj{}}
           \HOLSymConst{\HOLTokenForall{}}\ensuremath{\HOLBoundVar{E}\sb{\mathrm{2}}}. \ensuremath{\HOLBoundVar{E}\sp{\prime}} \HOLTokenTransBegin\HOLConst{label} \HOLBoundVar{l}\HOLTokenTransEnd \ensuremath{\HOLBoundVar{E}\sb{\mathrm{2}}} \HOLSymConst{\HOLTokenImp{}} \HOLSymConst{\HOLTokenExists{}}\ensuremath{\HOLBoundVar{E}\sb{\mathrm{1}}}. \HOLBoundVar{E} \HOLTokenWeakTransBegin\HOLConst{label} \HOLBoundVar{l}\HOLTokenWeakTransEnd \ensuremath{\HOLBoundVar{E}\sb{\mathrm{1}}} \HOLSymConst{\HOLTokenConj{}} \HOLFreeVar{Exp} \ensuremath{\HOLBoundVar{E}\sb{\mathrm{1}}} \ensuremath{\HOLBoundVar{E}\sb{\mathrm{2}}}\ensuremath{)} \HOLSymConst{\HOLTokenConj{}}
      \ensuremath{(}\HOLSymConst{\HOLTokenForall{}}\ensuremath{\HOLBoundVar{E}\sb{\mathrm{1}}}. \HOLBoundVar{E} \HOLTokenTransBegin\HOLSymConst{\ensuremath{\tau}}\HOLTokenTransEnd \ensuremath{\HOLBoundVar{E}\sb{\mathrm{1}}} \HOLSymConst{\HOLTokenImp{}} \HOLFreeVar{Exp} \ensuremath{\HOLBoundVar{E}\sb{\mathrm{1}}} \ensuremath{\HOLBoundVar{E}\sp{\prime}} \HOLSymConst{\HOLTokenDisj{}} \HOLSymConst{\HOLTokenExists{}}\ensuremath{\HOLBoundVar{E}\sb{\mathrm{2}}}. \ensuremath{\HOLBoundVar{E}\sp{\prime}} \HOLTokenTransBegin\HOLSymConst{\ensuremath{\tau}}\HOLTokenTransEnd \ensuremath{\HOLBoundVar{E}\sb{\mathrm{2}}} \HOLSymConst{\HOLTokenConj{}} \HOLFreeVar{Exp} \ensuremath{\HOLBoundVar{E}\sb{\mathrm{1}}} \ensuremath{\HOLBoundVar{E}\sb{\mathrm{2}}}\ensuremath{)} \HOLSymConst{\HOLTokenConj{}}
      \HOLSymConst{\HOLTokenForall{}}\ensuremath{\HOLBoundVar{E}\sb{\mathrm{2}}}. \ensuremath{\HOLBoundVar{E}\sp{\prime}} \HOLTokenTransBegin\HOLSymConst{\ensuremath{\tau}}\HOLTokenTransEnd \ensuremath{\HOLBoundVar{E}\sb{\mathrm{2}}} \HOLSymConst{\HOLTokenImp{}} \HOLSymConst{\HOLTokenExists{}}\ensuremath{\HOLBoundVar{E}\sb{\mathrm{1}}}. \HOLBoundVar{E} \HOLTokenWeakTransBegin\HOLSymConst{\ensuremath{\tau}}\HOLTokenWeakTransEnd \ensuremath{\HOLBoundVar{E}\sb{\mathrm{1}}} \HOLSymConst{\HOLTokenConj{}} \HOLFreeVar{Exp} \ensuremath{\HOLBoundVar{E}\sb{\mathrm{1}}} \ensuremath{\HOLBoundVar{E}\sb{\mathrm{2}}}\hfill{[EXPANSION]}

\HOLTokenTurnstile{} \HOLFreeVar{P} \HOLSymConst{\HOLTokenExpands{}} \HOLFreeVar{Q} \HOLSymConst{\HOLTokenEquiv{}} \HOLSymConst{\HOLTokenExists{}}\HOLBoundVar{Exp}. \HOLBoundVar{Exp} \HOLFreeVar{P} \HOLFreeVar{Q} \HOLSymConst{\HOLTokenConj{}} \HOLConst{EXPANSION} \HOLBoundVar{Exp}\hfill{[expands_thm]}
\end{alltt}

\begin{alltt}
\HOLConst{CONTRACTION} \HOLFreeVar{Con} \HOLTokenDefEquality{}
  \HOLSymConst{\HOLTokenForall{}}\HOLBoundVar{E} \ensuremath{\HOLBoundVar{E}\sp{\prime}}.
      \HOLFreeVar{Con} \HOLBoundVar{E} \ensuremath{\HOLBoundVar{E}\sp{\prime}} \HOLSymConst{\HOLTokenImp{}}
      \ensuremath{(}\HOLSymConst{\HOLTokenForall{}}\HOLBoundVar{l}.
           \ensuremath{(}\HOLSymConst{\HOLTokenForall{}}\ensuremath{\HOLBoundVar{E}\sb{\mathrm{1}}}. \HOLBoundVar{E} \HOLTokenTransBegin\HOLConst{label} \HOLBoundVar{l}\HOLTokenTransEnd \ensuremath{\HOLBoundVar{E}\sb{\mathrm{1}}} \HOLSymConst{\HOLTokenImp{}} \HOLSymConst{\HOLTokenExists{}}\ensuremath{\HOLBoundVar{E}\sb{\mathrm{2}}}. \ensuremath{\HOLBoundVar{E}\sp{\prime}} \HOLTokenTransBegin\HOLConst{label} \HOLBoundVar{l}\HOLTokenTransEnd \ensuremath{\HOLBoundVar{E}\sb{\mathrm{2}}} \HOLSymConst{\HOLTokenConj{}} \HOLFreeVar{Con} \ensuremath{\HOLBoundVar{E}\sb{\mathrm{1}}} \ensuremath{\HOLBoundVar{E}\sb{\mathrm{2}}}\ensuremath{)} \HOLSymConst{\HOLTokenConj{}}
           \HOLSymConst{\HOLTokenForall{}}\ensuremath{\HOLBoundVar{E}\sb{\mathrm{2}}}. \ensuremath{\HOLBoundVar{E}\sp{\prime}} \HOLTokenTransBegin\HOLConst{label} \HOLBoundVar{l}\HOLTokenTransEnd \ensuremath{\HOLBoundVar{E}\sb{\mathrm{2}}} \HOLSymConst{\HOLTokenImp{}} \HOLSymConst{\HOLTokenExists{}}\ensuremath{\HOLBoundVar{E}\sb{\mathrm{1}}}. \HOLBoundVar{E} \HOLTokenWeakTransBegin\HOLConst{label} \HOLBoundVar{l}\HOLTokenWeakTransEnd \ensuremath{\HOLBoundVar{E}\sb{\mathrm{1}}} \HOLSymConst{\HOLTokenConj{}} \ensuremath{\HOLBoundVar{E}\sb{\mathrm{1}}} \HOLSymConst{\HOLTokenWeakEQ} \ensuremath{\HOLBoundVar{E}\sb{\mathrm{2}}}\ensuremath{)} \HOLSymConst{\HOLTokenConj{}}
      \ensuremath{(}\HOLSymConst{\HOLTokenForall{}}\ensuremath{\HOLBoundVar{E}\sb{\mathrm{1}}}. \HOLBoundVar{E} \HOLTokenTransBegin\HOLSymConst{\ensuremath{\tau}}\HOLTokenTransEnd \ensuremath{\HOLBoundVar{E}\sb{\mathrm{1}}} \HOLSymConst{\HOLTokenImp{}} \HOLFreeVar{Con} \ensuremath{\HOLBoundVar{E}\sb{\mathrm{1}}} \ensuremath{\HOLBoundVar{E}\sp{\prime}} \HOLSymConst{\HOLTokenDisj{}} \HOLSymConst{\HOLTokenExists{}}\ensuremath{\HOLBoundVar{E}\sb{\mathrm{2}}}. \ensuremath{\HOLBoundVar{E}\sp{\prime}} \HOLTokenTransBegin\HOLSymConst{\ensuremath{\tau}}\HOLTokenTransEnd \ensuremath{\HOLBoundVar{E}\sb{\mathrm{2}}} \HOLSymConst{\HOLTokenConj{}} \HOLFreeVar{Con} \ensuremath{\HOLBoundVar{E}\sb{\mathrm{1}}} \ensuremath{\HOLBoundVar{E}\sb{\mathrm{2}}}\ensuremath{)} \HOLSymConst{\HOLTokenConj{}}
      \HOLSymConst{\HOLTokenForall{}}\ensuremath{\HOLBoundVar{E}\sb{\mathrm{2}}}. \ensuremath{\HOLBoundVar{E}\sp{\prime}} \HOLTokenTransBegin\HOLSymConst{\ensuremath{\tau}}\HOLTokenTransEnd \ensuremath{\HOLBoundVar{E}\sb{\mathrm{2}}} \HOLSymConst{\HOLTokenImp{}} \HOLSymConst{\HOLTokenExists{}}\ensuremath{\HOLBoundVar{E}\sb{\mathrm{1}}}. \HOLBoundVar{E} \HOLSymConst{\HOLTokenEPS} \ensuremath{\HOLBoundVar{E}\sb{\mathrm{1}}} \HOLSymConst{\HOLTokenConj{}} \ensuremath{\HOLBoundVar{E}\sb{\mathrm{1}}} \HOLSymConst{\HOLTokenWeakEQ} \ensuremath{\HOLBoundVar{E}\sb{\mathrm{2}}}\hfill{[CONTRACTION]}

\HOLTokenTurnstile{} \HOLFreeVar{P} \HOLSymConst{\HOLTokenContracts{}} \HOLFreeVar{Q} \HOLSymConst{\HOLTokenEquiv{}} \HOLSymConst{\HOLTokenExists{}}\HOLBoundVar{Con}. \HOLBoundVar{Con} \HOLFreeVar{P} \HOLFreeVar{Q} \HOLSymConst{\HOLTokenConj{}} \HOLConst{CONTRACTION} \HOLBoundVar{Con}\hfill{[contracts_thm]}
\end{alltt}

We can prove that the contraction preorder is contained in weak bisimilarity,
and contains the expansion preorder.
\begin{proposition}{(Relationships between contraction preorder,
    expansion preorder and weak bisimilarity)}
\begin{enumerate}
\item (The expansion preorder implies contraction preorder)
\begin{alltt}
\HOLTokenTurnstile{} \HOLSymConst{\HOLTokenForall{}}\HOLBoundVar{P} \HOLBoundVar{Q}. \HOLBoundVar{P} \HOLSymConst{\HOLTokenExpands{}} \HOLBoundVar{Q} \HOLSymConst{\HOLTokenImp{}} \HOLBoundVar{P} \HOLSymConst{\HOLTokenContracts{}} \HOLBoundVar{Q}\hfill[expands_IMP_contracts]
\end{alltt}
\item (The contraction preorder implies weak bisimilarity)
\begin{alltt}
\HOLTokenTurnstile{} \HOLSymConst{\HOLTokenForall{}}\HOLBoundVar{P} \HOLBoundVar{Q}. \HOLBoundVar{P} \HOLSymConst{\HOLTokenContracts{}} \HOLBoundVar{Q} \HOLSymConst{\HOLTokenImp{}} \HOLBoundVar{P} \HOLSymConst{\HOLTokenWeakEQ} \HOLBoundVar{Q}\hfill[contracts_IMP_WEAK_EQUIV]
\end{alltt}
\end{enumerate}
\end{proposition}
In general a proof of a property for the contraction (and the contraction preorder) is
harder than that for the cases of expansion: this is mostly due to the (surprising) fact
  that, although the contraction preorder $\mcontrBIS$ is contained in
  bisimilarity ($\wb$), in general it won't be true that
if $\R$ is a contraction then $\R$ itself is a bisimulation,
i.e.~the following proposition does not hold: (what holds is that $\R\ \cup \wb$ will be a bisimulation)
\begin{alltt}
\HOLinline{   \HOLSymConst{\HOLTokenForall{}}\HOLBoundVar{Con}. \HOLConst{CONTRACTION} \HOLBoundVar{Con} \HOLSymConst{\HOLTokenImp{}} \HOLConst{WEAK_BISIM} \HOLBoundVar{Con}}
\end{alltt}
For instance, in the proof of \texttt{contracts_IMP_WEAK_EQUIV},
we can prove it by constructing a bisimulation \HOLinline{\HOLFreeVar{Wbsm}} containing two processes
$P$ and $Q$, given that they are in $Con$ (a contraction):
\begin{alltt}
        \HOLinline{\HOLSymConst{\HOLTokenExists{}}\HOLBoundVar{Wbsm}. \HOLBoundVar{Wbsm} \HOLFreeVar{P} \HOLFreeVar{Q} \HOLSymConst{\HOLTokenConj{}} \HOLConst{WEAK_BISIM} \HOLBoundVar{Wbsm}}
   ------------------------------------
    0.  \HOLinline{\HOLFreeVar{Con} \HOLFreeVar{P} \HOLFreeVar{Q}}
    1.  \HOLinline{\HOLConst{CONTRACTION} \HOLFreeVar{Con}}
\end{alltt}
We cannot show that $Con$ itself is a bisimulation, but rather
that the union of $Con$ and $\wb$ is a bisimulation.
The corresponding lemma for the expansion preorder is rather
straightforward (just use $Con$).

The rooted contraction ($\rcontr$, \texttt{OBS_contracts}) is formally
defined as follows:
\begin{alltt}
   \HOLFreeVar{E} \HOLSymConst{\HOLTokenObsContracts} \ensuremath{\HOLFreeVar{E}\sp{\prime}} \HOLTokenDefEquality{}
     \HOLSymConst{\HOLTokenForall{}}\HOLBoundVar{u}.
         \ensuremath{(}\HOLSymConst{\HOLTokenForall{}}\ensuremath{\HOLBoundVar{E}\sb{\mathrm{1}}}. \HOLFreeVar{E} \HOLTokenTransBegin\HOLBoundVar{u}\HOLTokenTransEnd \ensuremath{\HOLBoundVar{E}\sb{\mathrm{1}}} \HOLSymConst{\HOLTokenImp{}} \HOLSymConst{\HOLTokenExists{}}\ensuremath{\HOLBoundVar{E}\sb{\mathrm{2}}}. \ensuremath{\HOLFreeVar{E}\sp{\prime}} \HOLTokenTransBegin\HOLBoundVar{u}\HOLTokenTransEnd \ensuremath{\HOLBoundVar{E}\sb{\mathrm{2}}} \HOLSymConst{\HOLTokenConj{}} \ensuremath{\HOLBoundVar{E}\sb{\mathrm{1}}} \HOLSymConst{\HOLTokenContracts{}} \ensuremath{\HOLBoundVar{E}\sb{\mathrm{2}}}\ensuremath{)} \HOLSymConst{\HOLTokenConj{}}
         \HOLSymConst{\HOLTokenForall{}}\ensuremath{\HOLBoundVar{E}\sb{\mathrm{2}}}. \ensuremath{\HOLFreeVar{E}\sp{\prime}} \HOLTokenTransBegin\HOLBoundVar{u}\HOLTokenTransEnd \ensuremath{\HOLBoundVar{E}\sb{\mathrm{2}}} \HOLSymConst{\HOLTokenImp{}} \HOLSymConst{\HOLTokenExists{}}\ensuremath{\HOLBoundVar{E}\sb{\mathrm{1}}}. \HOLFreeVar{E} \HOLTokenWeakTransBegin\HOLBoundVar{u}\HOLTokenWeakTransEnd \ensuremath{\HOLBoundVar{E}\sb{\mathrm{1}}} \HOLSymConst{\HOLTokenConj{}} \ensuremath{\HOLBoundVar{E}\sb{\mathrm{1}}} \HOLSymConst{\HOLTokenWeakEQ} \ensuremath{\HOLBoundVar{E}\sb{\mathrm{2}}}\hfill{[OBS_contracts]}
\end{alltt}

\subsection{The formalisation of ``bisimulation up to bisimilarity''}

``Bisimulation up to'' is a family of powerful proof techniques
\hl{for reducing the sizes of relations needed for defining bisimulations}.
By definition, two processes are bisimilar if there exists a
bisimulation relation containing them. However, in practice
this definition is hard \hl{to be} followed plainly. \hl{I}nstead, to reduce
the \hl{sizes of the exhibited relations}, one prefers to define relations
which are bisimulations only when closed up under some specific and
privileged relation, so to relieve the proof work needed. \hl{These}
\emph{``up-to'' techniques} \hl{are general devices allowing} a great variety of possibilities.

Recall that we often write $P \;\R\; Q$ to denote
$(P, Q) \in \R$ for any binary relation $\R$. 
Moreover, 
 $\sim \mathcal{S} \sim$ is the composition of three binary
relations: $\sim$, $\S$ and $\sim$. Hence $P \sim \S \sim Q$ means that,
there exist $P'$ and $Q'$ such that $P \sim P'$, $P' \;\S\; Q'$ and $Q' \sim Q$.
\begin{definition}[Bisimulation up to $\sim$]
  \label{def:bisimUptoSim}
$\S$ is a ``\emph{bisimulation up to $\sim$}'' if $P\ \S\ Q$ implies, for all $\mu$,
\begin{enumerate}
\item Whenever $P \overset{\mu}{\rightarrow} P'$ then, for some
  $Q'$, $Q \overset{\mu}{\rightarrow} Q'$ and $P' \sim \S
  \sim Q'$,
\item Whenever $Q \overset{\mu}{\rightarrow} Q'$ then, for some
  $P'$, $P \overset{\mu}{\rightarrow} P'$ and $P' \sim \S
  \sim Q'$.
\end{enumerate}
\end{definition}

\begin{theorem}
If $\mathcal{S}$ is a ``bisimulation up to $\sim$'', then
$\mathcal{S} \subseteq\;\sim$:
\begin{alltt}
\HOLTokenTurnstile{} \HOLConst{STRONG_BISIM_UPTO} \HOLFreeVar{Bsm} \HOLSymConst{\HOLTokenConj{}} \HOLFreeVar{Bsm} \HOLFreeVar{P} \HOLFreeVar{Q} \HOLSymConst{\HOLTokenImp{}} \HOLFreeVar{P} \HOLSymConst{\HOLTokenStrongEQ} \HOLFreeVar{Q}\hfill{[STRONG_EQUIV_BY_BISIM_UPTO]}
\end{alltt}
\end{theorem}
Hence, to prove $P \sim Q$, one only needs to find a bisimulation
up to $\sim$ that contains $(P, Q)$.

For weak bisimilarity, the \emph{naive} weak bisimulation up to weak bisimilarity
is unsound: if one simply replaces all $\sim$ in
Def.~\ref{def:bisimUptoSim} with $\wb$, the resulting ``weak
bisimulation up`` is not contained in $\wb$.~\cite{sangiorgi1992problem}
There are a few ways to fix this problem, one is the following:

\begin{definition}{(Bisimulation up to $\approx$)}
  $\S$ is a ``\emph{bisimulation up to $\approx$}'' if
  $P\ \S\ Q$ implies, for all $\mu$,
\begin{enumerate}
\item Whenever $P \arr{\mu} P'$ then, for some
  $Q'$, $Q \Arcap{\mu} Q'$ and $P' \sim \S \approx Q'$,
\item Whenever $Q \arr{\mu} Q'$ then, for some
  $P'$, $P \Arcap{\mu} P'$ and $P' \approx \S \sim Q'$.
\end{enumerate}
or formally (for illustrating purposes below),
\begin{alltt}
    \HOLConst{WEAK_BISIM_UPTO} \HOLFreeVar{Wbsm} \HOLTokenDefEquality{}
      \HOLSymConst{\HOLTokenForall{}}\HOLBoundVar{E} \ensuremath{\HOLBoundVar{E}\sp{\prime}}.
          \HOLFreeVar{Wbsm} \HOLBoundVar{E} \ensuremath{\HOLBoundVar{E}\sp{\prime}} \HOLSymConst{\HOLTokenImp{}}
          \ensuremath{(}\HOLSymConst{\HOLTokenForall{}}\HOLBoundVar{l}.
               \ensuremath{(}\HOLSymConst{\HOLTokenForall{}}\ensuremath{\HOLBoundVar{E}\sb{\mathrm{1}}}.
                    \HOLBoundVar{E} \HOLTokenTransBegin\HOLConst{label} \HOLBoundVar{l}\HOLTokenTransEnd \ensuremath{\HOLBoundVar{E}\sb{\mathrm{1}}} \HOLSymConst{\HOLTokenImp{}}
                    \HOLSymConst{\HOLTokenExists{}}\ensuremath{\HOLBoundVar{E}\sb{\mathrm{2}}}.
                        \ensuremath{\HOLBoundVar{E}\sp{\prime}} \HOLTokenWeakTransBegin\HOLConst{label} \HOLBoundVar{l}\HOLTokenWeakTransEnd \ensuremath{\HOLBoundVar{E}\sb{\mathrm{2}}} \HOLSymConst{\HOLTokenConj{}}
                        \ensuremath{(}\HOLConst{WEAK_EQUIV} \HOLSymConst{\HOLTokenRCompose{}} \HOLFreeVar{Wbsm} \HOLSymConst{\HOLTokenRCompose{}} \HOLConst{STRONG_EQUIV}\ensuremath{)} \ensuremath{\HOLBoundVar{E}\sb{\mathrm{1}}} \ensuremath{\HOLBoundVar{E}\sb{\mathrm{2}}}\ensuremath{)} \HOLSymConst{\HOLTokenConj{}}
               \HOLSymConst{\HOLTokenForall{}}\ensuremath{\HOLBoundVar{E}\sb{\mathrm{2}}}.
                   \ensuremath{\HOLBoundVar{E}\sp{\prime}} \HOLTokenTransBegin\HOLConst{label} \HOLBoundVar{l}\HOLTokenTransEnd \ensuremath{\HOLBoundVar{E}\sb{\mathrm{2}}} \HOLSymConst{\HOLTokenImp{}}
                   \HOLSymConst{\HOLTokenExists{}}\ensuremath{\HOLBoundVar{E}\sb{\mathrm{1}}}.
                       \HOLBoundVar{E} \HOLTokenWeakTransBegin\HOLConst{label} \HOLBoundVar{l}\HOLTokenWeakTransEnd \ensuremath{\HOLBoundVar{E}\sb{\mathrm{1}}} \HOLSymConst{\HOLTokenConj{}}
                       \ensuremath{(}\HOLConst{STRONG_EQUIV} \HOLSymConst{\HOLTokenRCompose{}} \HOLFreeVar{Wbsm} \HOLSymConst{\HOLTokenRCompose{}} \HOLConst{WEAK_EQUIV}\ensuremath{)} \ensuremath{\HOLBoundVar{E}\sb{\mathrm{1}}} \ensuremath{\HOLBoundVar{E}\sb{\mathrm{2}}}\ensuremath{)} \HOLSymConst{\HOLTokenConj{}}
          \ensuremath{(}\HOLSymConst{\HOLTokenForall{}}\ensuremath{\HOLBoundVar{E}\sb{\mathrm{1}}}.
               \HOLBoundVar{E} \HOLTokenTransBegin\HOLSymConst{\ensuremath{\tau}}\HOLTokenTransEnd \ensuremath{\HOLBoundVar{E}\sb{\mathrm{1}}} \HOLSymConst{\HOLTokenImp{}}
               \HOLSymConst{\HOLTokenExists{}}\ensuremath{\HOLBoundVar{E}\sb{\mathrm{2}}}. \ensuremath{\HOLBoundVar{E}\sp{\prime}} \HOLSymConst{\HOLTokenEPS} \ensuremath{\HOLBoundVar{E}\sb{\mathrm{2}}} \HOLSymConst{\HOLTokenConj{}} \ensuremath{(}\HOLConst{WEAK_EQUIV} \HOLSymConst{\HOLTokenRCompose{}} \HOLFreeVar{Wbsm} \HOLSymConst{\HOLTokenRCompose{}} \HOLConst{STRONG_EQUIV}\ensuremath{)} \ensuremath{\HOLBoundVar{E}\sb{\mathrm{1}}} \ensuremath{\HOLBoundVar{E}\sb{\mathrm{2}}}\ensuremath{)} \HOLSymConst{\HOLTokenConj{}}
          \HOLSymConst{\HOLTokenForall{}}\ensuremath{\HOLBoundVar{E}\sb{\mathrm{2}}}.
              \ensuremath{\HOLBoundVar{E}\sp{\prime}} \HOLTokenTransBegin\HOLSymConst{\ensuremath{\tau}}\HOLTokenTransEnd \ensuremath{\HOLBoundVar{E}\sb{\mathrm{2}}} \HOLSymConst{\HOLTokenImp{}}
              \HOLSymConst{\HOLTokenExists{}}\ensuremath{\HOLBoundVar{E}\sb{\mathrm{1}}}. \HOLBoundVar{E} \HOLSymConst{\HOLTokenEPS} \ensuremath{\HOLBoundVar{E}\sb{\mathrm{1}}} \HOLSymConst{\HOLTokenConj{}} \ensuremath{(}\HOLConst{STRONG_EQUIV} \HOLSymConst{\HOLTokenRCompose{}} \HOLFreeVar{Wbsm} \HOLSymConst{\HOLTokenRCompose{}} \HOLConst{WEAK_EQUIV}\ensuremath{)} \ensuremath{\HOLBoundVar{E}\sb{\mathrm{1}}} \ensuremath{\HOLBoundVar{E}\sb{\mathrm{2}}}
\end{alltt}
\end{definition}
Note that the \hl{HOL term corresponding to} $\sim \R \approx$ is
``\HOLinline{\HOLConst{WEAK_EQUIV} \HOLSymConst{\HOLTokenRCompose{}} \HOLFreeVar{R} \HOLSymConst{\HOLTokenRCompose{}} \HOLConst{STRONG_EQUIV}}'' where the order of
$\sim$ and $\approx$ seems reverted. This is because, in HOL's
notation, the rightmost relation (\HOLinline{\HOLConst{STRONG_EQUIV}} or $\sim$) in the relational composition is
applied first.

\begin{theorem}
If $\mathcal{S}$ is a bisimulation up to $\approx$, then
$\mathcal{S} \subseteq\;\approx$:
\begin{alltt}
\HOLTokenTurnstile{} \HOLConst{WEAK_BISIM_UPTO} \HOLFreeVar{Bsm} \HOLSymConst{\HOLTokenConj{}} \HOLFreeVar{Bsm} \HOLFreeVar{P} \HOLFreeVar{Q} \HOLSymConst{\HOLTokenImp{}} \HOLFreeVar{P} \HOLSymConst{\HOLTokenWeakEQ} \HOLFreeVar{Q}\hfill{[WEAK_EQUIV_BY_BISIM_UPTO]}
\end{alltt}
\end{theorem}

The above version of ``bisimulation up to $\wb$'' 
is not powerful \hl{enough} to prove Milner's ``unique solution of equations''
theorem for $\wb$ (c.f.~\cite{sangiorgi1992problem} for more details).
To complete the proof, the following version \hl{\emph{with weak arrows}} is
actually used:
\begin{definition}{(Bisimulation up to $\approx$ with weak arrows}
  \label{def:doubleweak}
$\mathcal{S}$ is a ``\emph{bisimulation up to $\approx$}'' if $P \;
  \mathcal{S} \; Q$ implies, for all $\mu$,
\begin{enumerate}
\item Whenever $P \Arr{\mu} P'$ then, for some $Q'$, $Q \Arcap{\mu} Q'$ and $P' \wb \S \wb Q'$,
\item Whenever $Q \Arr{\mu} Q'$ then, for some $P'$, $P \Arcap{\mu} P'$ and $P' \wb \S \wb Q'$.
\end{enumerate}
\end{definition}

\begin{theorem}
If $\mathcal{S}$ is a bisimulation up to $\approx$ with weak arrows, then
$\mathcal{S} \subseteq\;\approx$:
\begin{alltt}
\HOLTokenTurnstile{} \HOLConst{WEAK_BISIM_UPTO_ALT} \HOLFreeVar{Bsm} \HOLSymConst{\HOLTokenConj{}} \HOLFreeVar{Bsm} \HOLFreeVar{P} \HOLFreeVar{Q} \HOLSymConst{\HOLTokenImp{}} \HOLFreeVar{P} \HOLSymConst{\HOLTokenWeakEQ} \HOLFreeVar{Q}\hfill{[WEAK_EQUIV_BY_BISIM_UPTO_ALT]}
\end{alltt}
\end{theorem}

%  next file: coarsest.htex

%%%% -*- Mode: LaTeX -*-
%%
%% This is the draft of the 2nd part of EXPRESS/SOS 2018 paper, co-authored by
%% Prof. Davide Sangiorgi and Chun Tian.

\subsection{Coarsest (pre)congruence contained in $\approx$ ($\succeq_{\mathrm{bis}}$)}
\label{s:coarsest}

In this section we give a proof of the second part of
Theorem~\ref{t:rapproxCongruence}, i.e. $\rapprox$ is the coarsest
congruence contained in $\wb$. 
The general form of this theorem is the following
one~\cite{van2005characterisation,Gorrieri:2015jt,Mil89}:
\begin{proposition}
\label{prop:coarsest}
  Rooted bisimilarity ($\rapprox$) is the coarsest congruence
    contained in weak bisimilarity ($\wb$):
  \begin{equation}
    \label{eq:coarsest}
\forall p\ \ q.\ p\ \rapprox\ \! q\ \Longleftrightarrow\ ( \forall r.\ p\ +\
r\ \approx\ q\ +\ r )\enspace.
\end{equation}
\end{proposition}
From left to right (\ref{eq:coarsest}) trivially holds, due to the substitutivity of
$\rapprox$ for summation and the fact that $\rapprox$ implies $\wb$: (Thus we are only interested in (\ref{eq:coarsest}) from right to left.)
\begin{alltt}
\HOLTokenTurnstile{} \HOLSymConst{\HOLTokenForall{}}\HOLBoundVar{p} \HOLBoundVar{q}. \HOLBoundVar{p} \HOLSymConst{\HOLTokenObsCongr} \HOLBoundVar{q} \HOLSymConst{\HOLTokenImp{}} \HOLSymConst{\HOLTokenForall{}}\HOLBoundVar{r}. \HOLBoundVar{p} \HOLSymConst{\ensuremath{+}} \HOLBoundVar{r} \HOLSymConst{\HOLTokenWeakEQ} \HOLBoundVar{q} \HOLSymConst{\ensuremath{+}} \HOLBoundVar{r}\hfill{[COARSEST_CONGR_LR]}
\end{alltt}

The formalisation of this theorem presents some 
delicate aspects. For instance, 
within our CCS syntax which supports only binary sums,
one way to prove Proposition~\ref{prop:coarsest} is
to add an hypothesis that
 the involved processes do not use all available labels.
 Indeed, this is the standard argument by Milner~\citep[p.~153]{Mil89}.
%
 Formalising this result, however, requires a detailed treatment of
 free and bound names (of labels) of CCS
processes, with the restriction operator acting as a binder.
In our actual formalisation, instead,
we assume the weaker hypothesis that all \emph{immediate weak} derivatives of
$p$ and $q$ do not use all available labels.
We call this the \emph{free action} property:
\begin{alltt}
   \HOLConst{free_action} \HOLFreeVar{p} \HOLTokenDefEquality{} \HOLSymConst{\HOLTokenExists{}}\HOLBoundVar{a}. \HOLSymConst{\HOLTokenForall{}}\ensuremath{\HOLBoundVar{p}\sp{\prime}}. \HOLSymConst{\HOLTokenNeg{}}\ensuremath{(}\HOLFreeVar{p} \HOLTokenWeakTransBegin\HOLConst{label} \HOLBoundVar{a}\HOLTokenWeakTransEnd \ensuremath{\HOLBoundVar{p}\sp{\prime}}\ensuremath{)}\hfill{[free_action_def]}
\end{alltt}

Now we show how (\ref{eq:coarsest}) is connected with
the statement of Proposition~\ref{prop:coarsest}, and prove it under
the free action assumptions.
%
The coarsest congruence
contained in (weak) bisimilarity, namely \emph{bisimilarity
  congruence} (\texttt{WEAK_CONGR} in HOL), is
the \emph{composition closure} (\texttt{CC}) of (weak) bisimilarity:
\begin{alltt}
   \HOLConst{CC} \HOLFreeVar{R} \HOLTokenDefEquality{} \ensuremath{(}\HOLTokenLambda{}\HOLBoundVar{g} \HOLBoundVar{h}. \HOLSymConst{\HOLTokenForall{}}\HOLBoundVar{c}. \HOLConst{CONTEXT} \HOLBoundVar{c} \HOLSymConst{\HOLTokenImp{}} \HOLFreeVar{R} \ensuremath{(}\HOLBoundVar{c} \HOLBoundVar{g}\ensuremath{)} \ensuremath{(}\HOLBoundVar{c} \HOLBoundVar{h}\ensuremath{)}\ensuremath{)}\hfill{[CC_def]}
   \HOLConst{WEAK_CONGR} \HOLTokenDefEquality{} \HOLConst{CC} \HOLConst{WEAK_EQUIV}\hfill{[WEAK_CONGR]}
\end{alltt}
We do not need to put $R\ g\ h$ into the antecedents of
  \texttt{CC\_def}, as this is anyhow obtained from the trivial context $(\lambda x.\,x)$.
The next result shows that, for any binary relation $R$ 
on CCS processes, the composition closure of $R$ is always at least as
fine as $R$ (here $\subseteq_r$ stands for \emph{relational subset}):
\begin{alltt}
\HOLTokenTurnstile{} \HOLSymConst{\HOLTokenForall{}}\HOLBoundVar{R}. \HOLConst{CC} \HOLBoundVar{R} \HOLSymConst{\HOLTokenRSubset{}} \HOLBoundVar{R}\hfill{[CC_is_finer]}
\end{alltt}
Furthermore, we prove that any (pre)congruence contained in $R$,
that itself needs not to be a (pre)congruence,
is contained in the composition closure of $R$
(hence the composition closure is indeed the coarsest one):
\begin{alltt}
\HOLTokenTurnstile{} \HOLSymConst{\HOLTokenForall{}}\HOLBoundVar{R} \ensuremath{\HOLBoundVar{R}\sp{\prime}}. \HOLConst{congruence} \ensuremath{\HOLBoundVar{R}\sp{\prime}} \HOLSymConst{\HOLTokenConj{}} \ensuremath{\HOLBoundVar{R}\sp{\prime}} \HOLSymConst{\HOLTokenRSubset{}} \HOLBoundVar{R} \HOLSymConst{\HOLTokenImp{}} \ensuremath{\HOLBoundVar{R}\sp{\prime}} \HOLSymConst{\HOLTokenRSubset{}} \HOLConst{CC} \HOLBoundVar{R}\hfill{[CC_is_coarsest]}
\HOLTokenTurnstile{} \HOLSymConst{\HOLTokenForall{}}\HOLBoundVar{R} \ensuremath{\HOLBoundVar{R}\sp{\prime}}. \HOLConst{precongruence} \ensuremath{\HOLBoundVar{R}\sp{\prime}} \HOLSymConst{\HOLTokenConj{}} \ensuremath{\HOLBoundVar{R}\sp{\prime}} \HOLSymConst{\HOLTokenRSubset{}} \HOLBoundVar{R} \HOLSymConst{\HOLTokenImp{}} \ensuremath{\HOLBoundVar{R}\sp{\prime}} \HOLSymConst{\HOLTokenRSubset{}} \HOLConst{CC} \HOLBoundVar{R}\hfill{[PCC_is_coarsest]}
\end{alltt}

Given the central role of
 summation, we also consider the relation closure of bisimilarity
 with respect to summation, called \emph{equivalence compatible with summation}
(\texttt{SUM_EQUIV}): %%, denoted by $\approx^+$: (not used this symbol)
\begin{alltt}
   \HOLConst{SUM_EQUIV} \HOLTokenDefEquality{} \ensuremath{(}\HOLTokenLambda{}\HOLBoundVar{p} \HOLBoundVar{q}. \HOLSymConst{\HOLTokenForall{}}\HOLBoundVar{r}. \HOLBoundVar{p} \HOLSymConst{\ensuremath{+}} \HOLBoundVar{r} \HOLSymConst{\HOLTokenWeakEQ} \HOLBoundVar{q} \HOLSymConst{\ensuremath{+}} \HOLBoundVar{r}\ensuremath{)}\hfill{[SUM_EQUIV]}
\end{alltt}

Rooted bisimilarity $\rapprox$ (as a congruence contained in
$\wb$) is now contained in \texttt{WEAK_CONGR},
which in turn is trivially contained in \texttt{SUM_EQUIV}, as shown
in Fig.~\ref{fig:relationship}. Thus, to prove Proposition~\ref{prop:coarsest},
the crux is to prove that \texttt{SUM_EQUIV} is contained in
$\rapprox$,
making all three relations
($\rapprox$, \texttt{WEAK_CONGR} and \texttt{SUM_EQUIV}) coincide:
\begin{equation}
\label{equa:pq}
\forall p\ \ q.\ ( \forall r.\ p\ +\ r \;\approx\; q\ +\ r ) \
\Longrightarrow\ p\ \rapprox\ \! q\enspace.
\end{equation}

\begin{figure}[ht]
\begin{displaymath}
\xymatrix{
{\textrm{Weak bisimilarity } (\approx)} & {\textrm{Equiv.
    compatible with summation (\texttt{SUM\_EQUIV})}}
\ar@/^3ex/[ldd]^{\supseteq\; ?}\\
{\textrm{Bisimilarity congruence (\texttt{WEAK\_CONGR})}}
\ar[u]^{\subseteq} \ar[ru]^{\subseteq} \\
{\textrm{Rooted bisimilarity } (\rapprox)} \ar[u]^{\subseteq}
}
\end{displaymath}
\caption{Relationships between several equivalences and $\wb$}
\label{fig:relationship}
\end{figure}

Here is the formalisation of (\ref{equa:pq}) under free action hypothesis:
\begin{theorem}[\texttt{COARSEST_CONGR_RL}]
  \label{thm:coarsestR}
  Under the free action hypothesis, $\rapprox$ is coarsest congruence contained in $\wb$.
\begin{alltt}
\HOLTokenTurnstile{} \HOLSymConst{\HOLTokenForall{}}\HOLBoundVar{p} \HOLBoundVar{q}. \HOLConst{free_action} \HOLBoundVar{p} \HOLSymConst{\HOLTokenConj{}} \HOLConst{free_action} \HOLBoundVar{q} \HOLSymConst{\HOLTokenImp{}} \ensuremath{(}\HOLSymConst{\HOLTokenForall{}}\HOLBoundVar{r}. \HOLBoundVar{p} \HOLSymConst{\ensuremath{+}} \HOLBoundVar{r} \HOLSymConst{\HOLTokenWeakEQ} \HOLBoundVar{q} \HOLSymConst{\ensuremath{+}} \HOLBoundVar{r}\ensuremath{)} \HOLSymConst{\HOLTokenImp{}} \HOLBoundVar{p} \HOLSymConst{\HOLTokenObsCongr} \HOLBoundVar{q}
\end{alltt}
\end{theorem}

With an almost identical proof, rooted contraction
($\rcontr$) is also the coarsest
precongruence contained in the bisimilarity contraction ($\mcontrBIS$):
\begin{theorem}[\texttt{COARSEST_PRECONGR_RL}]
  \label{thm:coarsestPre}
  Under the free action hypothesis, $\mcontrBIS$ is the coarsest precongruence contained in $\contr$.
\begin{alltt}
\HOLTokenTurnstile{} \HOLSymConst{\HOLTokenForall{}}\HOLBoundVar{p} \HOLBoundVar{q}. \HOLConst{free_action} \HOLBoundVar{p} \HOLSymConst{\HOLTokenConj{}} \HOLConst{free_action} \HOLBoundVar{q} \HOLSymConst{\HOLTokenImp{}} \ensuremath{(}\HOLSymConst{\HOLTokenForall{}}\HOLBoundVar{r}. \HOLBoundVar{p} \HOLSymConst{\ensuremath{+}} \HOLBoundVar{r} \HOLSymConst{\HOLTokenContracts{}} \HOLBoundVar{q} \HOLSymConst{\ensuremath{+}} \HOLBoundVar{r}\ensuremath{)} \HOLSymConst{\HOLTokenImp{}} \HOLBoundVar{p} \HOLSymConst{\HOLTokenObsContracts} \HOLBoundVar{q}
\end{alltt}
\end{theorem}

The formal proofs of Theorem~\ref{thm:coarsestR} and
\ref{thm:coarsestPre} precisely follow Milner~\citep[p.~153--154]{Mil89}.
Although Milner requires a stronger hypothesis: $\mathrm{fn}(p) \cup
\mathrm{fn}(q) \neq \mathscr{L}$ (here $\mathrm{fn}$ stands for \emph{free
  names}), the actual proof essentially requires only the above
free action property.
Indeed, in the proof one only looks at the immediate weak
derivatives of $p$ and $q$, and only requires that there is an input
or output label that never occurs as a label of the involved transitions.

\subsection{Arbitrarily many non-bisimilar processes}
\label{ss:arbitrarily}

As the type ``\HOLinline{\ensuremath{(}\ensuremath{\alpha}, \ensuremath{\beta}\ensuremath{)} \HOLTyOp{CCS}}'' is parameterized with two
type variables, if the type of all label
names $\beta$ has a  small cardinality (a singleton in the
extreme case), it is possible that the processes of
Proposition~\ref{prop:coarsest} use all available labels, and thus the free
action hypothesis does not hold. In this case, it is still possible to prove
Proposition~\ref{prop:coarsest}; however, due to some
limitations of HOL itself we have to assume that the
processes are \emph{finite-state}, i.e. the set of all their derivatives is
finite. The original proof, due to van
Glabbeek~\cite{van2005characterisation}, does not require finite-state CCS.
Here is the main theorem:
\begin{theorem}[\texttt{COARSEST_CONGR_FINITE}]
    \label{thm:coarsestfiniteState}
    For finite-state CCS, $\rapprox$ is the coarsest congruence contained in $\wb$:
\begin{alltt}
\HOLTokenTurnstile{} \HOLSymConst{\HOLTokenForall{}}\HOLBoundVar{p} \HOLBoundVar{q}. \HOLConst{finite_state} \HOLBoundVar{p} \HOLSymConst{\HOLTokenConj{}} \HOLConst{finite_state} \HOLBoundVar{q} \HOLSymConst{\HOLTokenImp{}} \ensuremath{(}\HOLBoundVar{p} \HOLSymConst{\HOLTokenObsCongr} \HOLBoundVar{q} \HOLSymConst{\HOLTokenEquiv{}} \HOLSymConst{\HOLTokenForall{}}\HOLBoundVar{r}. \HOLBoundVar{p} \HOLSymConst{\ensuremath{+}} \HOLBoundVar{r} \HOLSymConst{\HOLTokenWeakEQ} \HOLBoundVar{q} \HOLSymConst{\ensuremath{+}} \HOLBoundVar{r}\ensuremath{)}
\end{alltt}
\end{theorem}

The precise definition of \texttt{finite_state} used in above theorem
will be given later. We start with a core lemma (\texttt{PROP3_COMMON}) saying that, for
any two processes $p$ and $q$, if there exists a \emph{stable}
(i.e.~without $\tau$ transitions)
 process which is not bisimilar with any weak derivative of $p$ and
 $q$, then \HOLinline{\HOLConst{SUM_EQUIV}} indeed implies rooted bisimilarity
 ($\rapprox$)~\cite{van2005characterisation,Tian:2017wrba}:
\begin{alltt}
\HOLTokenTurnstile{} \HOLSymConst{\HOLTokenForall{}}\HOLBoundVar{p} \HOLBoundVar{q}.
       \ensuremath{(}\HOLSymConst{\HOLTokenExists{}}\HOLBoundVar{k}.
            \HOLConst{STABLE} \HOLBoundVar{k} \HOLSymConst{\HOLTokenConj{}} \ensuremath{(}\HOLSymConst{\HOLTokenForall{}}\ensuremath{\HOLBoundVar{p}\sp{\prime}} \HOLBoundVar{u}. \HOLBoundVar{p} \HOLTokenWeakTransBegin\HOLBoundVar{u}\HOLTokenWeakTransEnd \ensuremath{\HOLBoundVar{p}\sp{\prime}} \HOLSymConst{\HOLTokenImp{}} \HOLSymConst{\HOLTokenNeg{}}\ensuremath{(}\ensuremath{\HOLBoundVar{p}\sp{\prime}} \HOLSymConst{\HOLTokenWeakEQ} \HOLBoundVar{k}\ensuremath{)}\ensuremath{)} \HOLSymConst{\HOLTokenConj{}}
            \HOLSymConst{\HOLTokenForall{}}\ensuremath{\HOLBoundVar{q}\sp{\prime}} \HOLBoundVar{u}. \HOLBoundVar{q} \HOLTokenWeakTransBegin\HOLBoundVar{u}\HOLTokenWeakTransEnd \ensuremath{\HOLBoundVar{q}\sp{\prime}} \HOLSymConst{\HOLTokenImp{}} \HOLSymConst{\HOLTokenNeg{}}\ensuremath{(}\ensuremath{\HOLBoundVar{q}\sp{\prime}} \HOLSymConst{\HOLTokenWeakEQ} \HOLBoundVar{k}\ensuremath{)}\ensuremath{)} \HOLSymConst{\HOLTokenImp{}}
       \ensuremath{(}\HOLSymConst{\HOLTokenForall{}}\HOLBoundVar{r}. \HOLBoundVar{p} \HOLSymConst{\ensuremath{+}} \HOLBoundVar{r} \HOLSymConst{\HOLTokenWeakEQ} \HOLBoundVar{q} \HOLSymConst{\ensuremath{+}} \HOLBoundVar{r}\ensuremath{)} \HOLSymConst{\HOLTokenImp{}}
       \HOLBoundVar{p} \HOLSymConst{\HOLTokenObsCongr} \HOLBoundVar{q}\hfill{[PROP3_COMMON]}
\end{alltt}
\begin{alltt}
   \HOLConst{STABLE} \HOLFreeVar{p} \HOLTokenDefEquality{} \HOLSymConst{\HOLTokenForall{}}\HOLBoundVar{u} \ensuremath{\HOLBoundVar{p}\sp{\prime}}. \HOLFreeVar{p} \HOLTokenTransBegin\HOLBoundVar{u}\HOLTokenTransEnd \ensuremath{\HOLBoundVar{p}\sp{\prime}} \HOLSymConst{\HOLTokenImp{}} \HOLBoundVar{u} \HOLSymConst{\HOLTokenNotEqual{}} \HOLSymConst{\ensuremath{\tau}}\hfill{[STABLE]}
\end{alltt}

To prove Theorem~\ref{thm:coarsestfiniteState}, it only remains to construct such stable
process $k$ for any two finite-state processes $p$ and $q$.
For arbitrary CCS processes, this construction relies on
arbitrary infinite sums of processes (not within our CCS syntax) and
transfinite induction to obtain
an arbitrary large sequence of processes that are all pairwise
non-bisimilar,
which was firstly introduced by Jan
Willem Klop~(see \cite{van2005characterisation} for some historical notes).
We have only partially formalised
van Glabbeek's proof, mostly because our CCS syntax does not allow infinite
summation (and it is not easy to extend it with this support).
Another more important reason is that the typed logic
implemented in various HOL systems (including Isabelle/HOL) is not
strong enough to define a type for all possible
ordinals~\cite{norrish2013ordinals} which is required in van
Glabbeek's proof. As the consequence, the formalisation
(Theorem~\ref{thm:coarsestfiniteState})
can only apply to finite-state CCS.

The above core lemma (\texttt{PROP3_COMMON}) requires the
existence of a special CCS process, which is not weakly bisimilar to
any weak derivative of the two root processes.
There could be infinitely many such subprocesses, even on finitely
branching processes.
We can, however, consider the equivalence classes of CCS processes
modulo weak bisimilarity.
If there are infinitely many such classes, 
then it will be 
possible to choose one that is distinct from all the (finitely many) states in the
transition graphs of the two given processes.
This can be done by following Klop's contruction.
 We call the processes in this construction the ``Klop processes'':
\begin{definition}[Klop processes]
For each ordinal $\lambda$, and an arbitrary chosen action $a \neq \tau$,
define a CCS process $k_\lambda$ as follows:
\begin{itemize}
\item $k_0 = 0$,
\item $k_{\lambda+1} = k_\lambda + a.k_\lambda$ and
\item for $\lambda$ a limit ordinal, $k_\lambda = \sum_{\mu < \lambda}
  k_\mu$ (meaning that $k_\lambda$ is constructed from all graphs
  $k_\mu$ for $\mu < \lambda$ by identifying their root).
\end{itemize}
\end{definition}
When processes are finite-state, that is,
the number of  states in which a process may evolve by performing
transitions is finite, 
we can use  the following subset of Klop processes, 
defined as a recursive function (on natural numbers) in HOL4:
\begin{definition}{(Klop processes as recursive function on natural numbers)}
\begin{alltt}
   \HOLConst{KLOP} \HOLFreeVar{a} \HOLNumLit{0} \HOLTokenDefEquality{} \HOLConst{\ensuremath{\mathbf{0}}}
   \HOLConst{KLOP} \HOLFreeVar{a} \ensuremath{(}\HOLConst{SUC} \HOLFreeVar{n}\ensuremath{)} \HOLTokenDefEquality{} \HOLConst{KLOP} \HOLFreeVar{a} \HOLFreeVar{n} \HOLSymConst{\ensuremath{+}} \HOLConst{label} \HOLFreeVar{a}\HOLSymConst{\ensuremath{\ldotp}}\HOLConst{KLOP} \HOLFreeVar{a} \HOLFreeVar{n}\hfill{[KLOP_def]}
\end{alltt}
\end{definition}

Following the inductive structure of the above definition,
  and using the SOS rules
  ($\mathrm{Sum}_1$) and ($\mathrm{Sum}_2$), we can prove
the following properties of Klop functions:
\begin{proposition}{(Properties of Klop functions and processes)}
\begin{enumerate}
\item (All Klop processes are stable)
\begin{alltt}
\HOLTokenTurnstile{} \HOLConst{STABLE} \ensuremath{(}\HOLConst{KLOP} \HOLFreeVar{a} \HOLFreeVar{n}\ensuremath{)}\hfill[KLOP_PROP0]
\end{alltt}
\item (Any transition from a Klop process leads to a smaller Klop
  process, and conversely)
\begin{alltt}
\HOLTokenTurnstile{} \HOLConst{KLOP} \HOLFreeVar{a} \HOLFreeVar{n} \HOLTokenTransBegin\HOLConst{label} \HOLFreeVar{a}\HOLTokenTransEnd \HOLFreeVar{E} \HOLSymConst{\HOLTokenEquiv{}} \HOLSymConst{\HOLTokenExists{}}\HOLBoundVar{m}. \HOLBoundVar{m} \HOLSymConst{\HOLTokenLt{}} \HOLFreeVar{n} \HOLSymConst{\HOLTokenConj{}} \HOLFreeVar{E} \HOLSymConst{\ensuremath{=}} \HOLConst{KLOP} \HOLFreeVar{a} \HOLBoundVar{m}\hfill{[KLOP_PROP1]}
\end{alltt}
\item (The weak version of the previous property)
\begin{alltt}
\HOLTokenTurnstile{} \HOLConst{KLOP} \HOLFreeVar{a} \HOLFreeVar{n} \HOLTokenWeakTransBegin\HOLConst{label} \HOLFreeVar{a}\HOLTokenWeakTransEnd \HOLFreeVar{E} \HOLSymConst{\HOLTokenEquiv{}} \HOLSymConst{\HOLTokenExists{}}\HOLBoundVar{m}. \HOLBoundVar{m} \HOLSymConst{\HOLTokenLt{}} \HOLFreeVar{n} \HOLSymConst{\HOLTokenConj{}} \HOLFreeVar{E} \HOLSymConst{\ensuremath{=}} \HOLConst{KLOP} \HOLFreeVar{a} \HOLBoundVar{m}\hfill{[KLOP_PROP1']}
\end{alltt}
\item (All Klop processes are distinct according to strong bisimilarity)
\begin{alltt}
\HOLTokenTurnstile{} \HOLFreeVar{m} \HOLSymConst{\HOLTokenLt{}} \HOLFreeVar{n} \HOLSymConst{\HOLTokenImp{}} \HOLSymConst{\HOLTokenNeg{}}\ensuremath{(}\HOLConst{KLOP} \HOLFreeVar{a} \HOLFreeVar{m} \HOLSymConst{\HOLTokenStrongEQ} \HOLConst{KLOP} \HOLFreeVar{a} \HOLFreeVar{n}\ensuremath{)}\hfill{[KLOP_PROP2]}
\end{alltt}
\item (All Klop processes are distinct according to weak bisimilarity)
\begin{alltt}
\HOLTokenTurnstile{} \HOLFreeVar{m} \HOLSymConst{\HOLTokenLt{}} \HOLFreeVar{n} \HOLSymConst{\HOLTokenImp{}} \HOLSymConst{\HOLTokenNeg{}}\ensuremath{(}\HOLConst{KLOP} \HOLFreeVar{a} \HOLFreeVar{m} \HOLSymConst{\HOLTokenWeakEQ} \HOLConst{KLOP} \HOLFreeVar{a} \HOLFreeVar{n}\ensuremath{)}\hfill{[KLOP_PROP2']}
\end{alltt}
\item (Klop functions are one-one)
\begin{alltt}
\HOLTokenTurnstile{} \HOLConst{ONE_ONE} \ensuremath{(}\HOLConst{KLOP} \HOLFreeVar{a}\ensuremath{)}\hfill{[KLOP_ONE_ONE]}
\end{alltt}
\end{enumerate}
\end{proposition}

For any \HOLinline{\HOLConst{label}\;\HOLFreeVar{a}}, having the function ``\HOLinline{\HOLConst{KLOP}\;\HOLFreeVar{a}}'' (of type ``\HOLinline{\HOLTyOp{num} \HOLTokenTransEnd \ensuremath{(}\ensuremath{\alpha}, \ensuremath{\beta}\ensuremath{)} \HOLTyOp{CCS}}'')
defined on the natural numbers, we obtain a countable set of Klop processes built from the same label.
As the number of all Klop processes in this set is (countably) infinite, and
as they are all pairwise non-bisimiar,
we can always choose a number, corresponding to a Klop process,
that is non-bisimilar with any derivative of two given 
 (finite-state) processes $p$ and $q$,
even when $a$ is the only element of type $\beta$, i.e. the only label name
in $\mathscr{L}$.
This property is captured by appealing to the following set-theoreric
lamma (see~\cite{Tian:2017wrba} for its proof):
\begin{lemma}
Given an equivalence relation $R$ defined on a type, and two sets $A, B$
of elements in this type, 
if $A$ is finite, $B$ is infinite, and all elements
in $B$ belong to distinct equivalence classes, then there exists an element $k$ in $B$
which is not equivalent to any element in $A$:
\begin{alltt}
\HOLTokenTurnstile{} \HOLConst{equivalence} \HOLFreeVar{R} \HOLSymConst{\HOLTokenImp{}}
   \HOLConst{FINITE} \HOLFreeVar{A} \HOLSymConst{\HOLTokenConj{}} \HOLConst{INFINITE} \HOLFreeVar{B} \HOLSymConst{\HOLTokenConj{}} \ensuremath{(}\HOLSymConst{\HOLTokenForall{}}\HOLBoundVar{x} \HOLBoundVar{y}. \HOLBoundVar{x} \HOLSymConst{\HOLTokenIn{}} \HOLFreeVar{B} \HOLSymConst{\HOLTokenConj{}} \HOLBoundVar{y} \HOLSymConst{\HOLTokenIn{}} \HOLFreeVar{B} \HOLSymConst{\HOLTokenConj{}} \HOLBoundVar{x} \HOLSymConst{\HOLTokenNotEqual{}} \HOLBoundVar{y} \HOLSymConst{\HOLTokenImp{}} \HOLSymConst{\HOLTokenNeg{}}\HOLFreeVar{R} \HOLBoundVar{x} \HOLBoundVar{y}\ensuremath{)} \HOLSymConst{\HOLTokenImp{}}
   \HOLSymConst{\HOLTokenExists{}}\HOLBoundVar{k}. \HOLBoundVar{k} \HOLSymConst{\HOLTokenIn{}} \HOLFreeVar{B} \HOLSymConst{\HOLTokenConj{}} \HOLSymConst{\HOLTokenForall{}}\HOLBoundVar{n}. \HOLBoundVar{n} \HOLSymConst{\HOLTokenIn{}} \HOLFreeVar{A} \HOLSymConst{\HOLTokenImp{}} \HOLSymConst{\HOLTokenNeg{}}\HOLFreeVar{R} \HOLBoundVar{n} \HOLBoundVar{k}\hfill[INFINITE_EXISTS_LEMMA]
\end{alltt}
\end{lemma}
% \begin{proof}
%   We built an explicit mapping $f$ from $A$ to $B$\footnote{There're
%     multiple ways to prove this lemma, a simpler proof is to make a
%     reverse mapping from $B$ to the power set of $A$ (or further use
%     the Axiom of Choice (AC) to make a mapping from $B$ to $A$), then
%     the non-injectivity of this mapping will contradict the fact that
%     all elements in the infinite set are distinct. Our proof doesn't
%     need AC, and it relies on very simple truths about sets.}, for all
%   $x \in A$, $y = f(x)$ if $y \in B$ and $y$ is equivalent with
%   $x$. But it's possible that no element in $B$ is equivalent with
%   $x$, and in this case we just choose an arbitrary element as
%   $f(x)$. Such a mapping is to make sure the range of $f$ always fall
%   into $B$.

%   Now we can map $A$ to a subset of $B$, say $B_0$, and the
%   cardinality of $B_0$ must be equal or smaller than the cardinality
%   of $A$, thus finite. Now we choose an element $k$ from the rest part
%   of $B$, this element is the desire one, because for any element
%   $x \in A$, if it's equivalent with $k$, consider two cases for
%   $y = f(x) \in B_0$:
%   \begin{enumerate}
%   \item $y$ is equivalent with $x$. In this case by transitivity of
%     $R$, we have two distinct elements $y$ and $k$, one in $B_0$, the
%     other in $B\setminus B_0$, they're equivalent. This violates the
%     assumption that all elements in $B$ are distinct.
%   \item $y$ is arbitrary chosen because there's no equivalent element
%     for $x$ in $B$. But we already know one: $k$.
%   \end{enumerate}
%   Thus there's no element $x$ (in $A$) which is equivalent with $k$.
% \end{proof}

To reason about finite-state CCS, we also need to define the concept
of ``finite-state CCS'' as a predicate on CCS processes:
\begin{definition}[finite-state CCS]
  \mbox{}
\begin{enumerate}
\item A binary relation \texttt{Reach} is the RTC (reflexive and transitive
  closure) of a relation indicating the existence of a transition between two processes:
\begin{alltt}
\HOLConst{Reach} \HOLTokenDefEquality{} \ensuremath{(}\HOLTokenLambda{}\HOLBoundVar{E} \ensuremath{\HOLBoundVar{E}\sp{\prime}}. \HOLSymConst{\HOLTokenExists{}}\HOLBoundVar{u}. \HOLBoundVar{E} \HOLTokenTransBegin\HOLBoundVar{u}\HOLTokenTransEnd \ensuremath{\HOLBoundVar{E}\sp{\prime}}\ensuremath{)}\HOLSymConst{\HOLTokenSupStar{}}\hfill[Reach_def]
\end{alltt}
\item The set of all derivatives (\texttt{NODES}) of a process is the
  set of all processes reachable from it:
\begin{alltt}
\HOLConst{NODES} \HOLFreeVar{p} \HOLTokenDefEquality{} \HOLTokenLeftbrace{}\HOLBoundVar{q} \HOLTokenBar{} \HOLConst{Reach} \HOLFreeVar{p} \HOLBoundVar{q}\HOLTokenRightbrace{}\hfill[NODES_def]
\end{alltt}
\item A process is \texttt{finite-state} if the set of all derivatives is finite:
\begin{alltt}
\HOLConst{finite_state} \HOLFreeVar{p} \HOLTokenDefEquality{} \HOLConst{FINITE} \ensuremath{(}\HOLConst{NODES} \HOLFreeVar{p}\ensuremath{)}\hfill[finite_state_def]
\end{alltt}
\end{enumerate}
\end{definition}
We rely on various 
 properties of the above definitions, such as the following one:
\begin{proposition}
If $p$ has a weak transition to $q$, then $q$ is among the derivatives of $p$:
\begin{alltt}
\HOLTokenTurnstile{} \HOLFreeVar{p} \HOLTokenWeakTransBegin\HOLFreeVar{u}\HOLTokenWeakTransEnd \HOLFreeVar{q} \HOLSymConst{\HOLTokenImp{}} \HOLFreeVar{q} \HOLSymConst{\HOLTokenIn{}} \HOLConst{NODES} \HOLFreeVar{p}\hfill[WEAK_TRANS_IN_NODES]
\end{alltt}
\end{proposition}

Using all the above results, now we can prove the following finite-state
version of ``Klop lemma'':
\begin{lemma}[Klop lemma for finite-state CCS]
\label{lem:klop-lemma-finite}
For any two finite-state CCS $p$ and $q$, there is another process
$k$, which is not weakly bisimilar 
with any weak derivative  of $p$
and $q$ (i.e., any process reachable from $p$ or $q$ by means of transitions):
\begin{alltt}
\HOLTokenTurnstile{} \HOLSymConst{\HOLTokenForall{}}\HOLBoundVar{p} \HOLBoundVar{q}.
       \HOLConst{finite_state} \HOLBoundVar{p} \HOLSymConst{\HOLTokenConj{}} \HOLConst{finite_state} \HOLBoundVar{q} \HOLSymConst{\HOLTokenImp{}}
       \HOLSymConst{\HOLTokenExists{}}\HOLBoundVar{k}.
           \HOLConst{STABLE} \HOLBoundVar{k} \HOLSymConst{\HOLTokenConj{}} \ensuremath{(}\HOLSymConst{\HOLTokenForall{}}\ensuremath{\HOLBoundVar{p}\sp{\prime}} \HOLBoundVar{u}. \HOLBoundVar{p} \HOLTokenWeakTransBegin\HOLBoundVar{u}\HOLTokenWeakTransEnd \ensuremath{\HOLBoundVar{p}\sp{\prime}} \HOLSymConst{\HOLTokenImp{}} \HOLSymConst{\HOLTokenNeg{}}\ensuremath{(}\ensuremath{\HOLBoundVar{p}\sp{\prime}} \HOLSymConst{\HOLTokenWeakEQ} \HOLBoundVar{k}\ensuremath{)}\ensuremath{)} \HOLSymConst{\HOLTokenConj{}}
           \HOLSymConst{\HOLTokenForall{}}\ensuremath{\HOLBoundVar{q}\sp{\prime}} \HOLBoundVar{u}. \HOLBoundVar{q} \HOLTokenWeakTransBegin\HOLBoundVar{u}\HOLTokenWeakTransEnd \ensuremath{\HOLBoundVar{q}\sp{\prime}} \HOLSymConst{\HOLTokenImp{}} \HOLSymConst{\HOLTokenNeg{}}\ensuremath{(}\ensuremath{\HOLBoundVar{q}\sp{\prime}} \HOLSymConst{\HOLTokenWeakEQ} \HOLBoundVar{k}\ensuremath{)}\hfill{[KLOP_LEMMA_FINITE]}
\end{alltt}
\end{lemma}
Combining the above lemma with the core lemma
(\texttt{PROP3_COMMON}) and Theorem~\ref{thm:coarsestR} (\texttt{COARSEST_CONGR_RL}),
yields the proof of Theorem~\ref{thm:coarsestfiniteState}
(\texttt{COARSEST_CONGR_FINITE}). The same proof idea can also be used with
 contraction and rooted contration.

% next file: part2.htex

%%%% -*- Mode: LaTeX -*-
%%
%% This is the draft of the 2nd part of EXPRESS/SOS 2018 paper, co-authored by
%% Prof. Davide Sangiorgi and Chun Tian.

\subsection{Unique solution of contractions}

A delicate point in the formalisation of the results about unique solution of
contractions are the proof of Lemma~\ref{l:ruptocon} and lemmas alike;
in particular, there is
 an induction on the length of weak transitions. 
For this, rather than 
 introducing a refined form of weak transition relation
enriched with its length, 
we found it more elegant  to  work with traces
(a motivation for this is to set the ground for extensions of this
formalisation work to trace equivalence in place of bisimilarity).

% but such a non-standard relation finds no
% other uses beside proving our target theorem. Another way is to use 
% traces instead, as it shows more clearly all passing actions inside a
% trace, making formal reasoning easier.

% We represent a trace by the initial process, the final derivative, and
% the list of actions performed. 
% To formalise this, 
% we first introduce 
% the Reflexive Transitive Closure with a
% List (LRTC);

A trace is represented by the initial and final processes, plus
a list of actions  so performed.
For this, we first 
 define \hl{the concept of label-accumulated reflexive transitive closure
 (\texttt{LRTC})}.
Given a labeled transition relation \texttt{R} on CCS, \texttt{LRTC R} is
a label-accumulated relation representing the trace of transitions:
\begin{alltt}
\HOLConst{LRTC} \HOLFreeVar{R} \HOLFreeVar{a} \HOLFreeVar{l} \HOLFreeVar{b} \HOLSymConst{\HOLTokenEquiv{}}
\HOLSymConst{\HOLTokenForall{}}\HOLBoundVar{P}.
    (\HOLSymConst{\HOLTokenForall{}}\HOLBoundVar{x}. \HOLBoundVar{P} \HOLBoundVar{x} [] \HOLBoundVar{x}) \HOLSymConst{\HOLTokenConj{}}
    (\HOLSymConst{\HOLTokenForall{}}\HOLBoundVar{x} \HOLBoundVar{h} \HOLBoundVar{y} \HOLBoundVar{t} \HOLBoundVar{z}. \HOLFreeVar{R} \HOLBoundVar{x} \HOLBoundVar{h} \HOLBoundVar{y} \HOLSymConst{\HOLTokenConj{}} \HOLBoundVar{P} \HOLBoundVar{y} \HOLBoundVar{t} \HOLBoundVar{z} \HOLSymConst{\HOLTokenImp{}} \HOLBoundVar{P} \HOLBoundVar{x} (\HOLBoundVar{h}\HOLSymConst{::}\HOLBoundVar{t}) \HOLBoundVar{z}) \HOLSymConst{\HOLTokenImp{}}
    \HOLBoundVar{P} \HOLFreeVar{a} \HOLFreeVar{l} \HOLFreeVar{b}\hfill{[LRTC_DEF]}
\end{alltt}
\hl{The trace relation for CCS can be then obtained
 by calling \texttt{LRTC} on the (strong, or single-step) labeled transition
 relation \texttt{TRANS} ($\overset{\mu}{\rightarrow}$) defined by SOS rules}:
\begin{alltt}
\HOLConst{TRACE} \HOLSymConst{=} \HOLConst{LRTC} \HOLConst{TRANS}\hfill{[TRACE_def]}
\end{alltt}

\hl{If the list of actions is empty, that means that there is no transition and hence,}
if there is at most one visible action (i.e., a label) in the list of actions,
then the trace is also a weak transition. Here
we have to distinguish between two cases: no label and unique label (in
the list of actions). The definition of ``no
label'' in an action list is easy (here \texttt{MEM} tests if a given element is a member of a list):
\begin{alltt}
\HOLConst{NO_LABEL} \HOLFreeVar{L} \HOLSymConst{\HOLTokenEquiv{}} \HOLSymConst{\HOLTokenNeg{}}\HOLSymConst{\HOLTokenExists{}}\HOLBoundVar{l}. \HOLConst{MEM} (\HOLConst{label} \HOLBoundVar{l}) \HOLFreeVar{L}\hfill{[NO_LABEL_def]}
\end{alltt}

The definition of ``unique label'' \hl{can be done in many ways, the
following definition (a suggestion from Robert Beers)
avoides any counting or filtering in the list.}
It says that a label is unique in a list of actions if and only if there is no
label in the rest of list:
\begin{alltt}
\HOLConst{UNIQUE_LABEL} \HOLFreeVar{u} \HOLFreeVar{L} \HOLSymConst{\HOLTokenEquiv{}}
\HOLSymConst{\HOLTokenExists{}}\HOLBoundVar{L\sb{\mathrm{1}}} \HOLBoundVar{L\sb{\mathrm{2}}}. (\HOLBoundVar{L\sb{\mathrm{1}}} \HOLSymConst{\HOLTokenDoublePlus} [\HOLFreeVar{u}] \HOLSymConst{\HOLTokenDoublePlus} \HOLBoundVar{L\sb{\mathrm{2}}} \HOLSymConst{=} \HOLFreeVar{L}) \HOLSymConst{\HOLTokenConj{}} \HOLConst{NO_LABEL} \HOLBoundVar{L\sb{\mathrm{1}}} \HOLSymConst{\HOLTokenConj{}} \HOLConst{NO_LABEL} \HOLBoundVar{L\sb{\mathrm{2}}}\hfill{[UNIQUE_LABEL_def]}
\end{alltt}

The final relationship between traces and weak transitions is stated
and proved in the following theorem
(where the  variable $acts$ stands for
a list of actions); 
it says, a weak transition $P\overset{u}{\Rightarrow}P'$ is also a
trace $P\overset{acts}{\longrightarrow}P'$ with a
 non-empty action list $acts$, in which either there is no label (for $u = \tau$), or 
$u$ is the unique label (for $u \neq \tau$):
%\begin{lemma}
% A weak transition $P\overset{u}{\Rightarrow}P'$ is a just trace with non
% empty action list: 1) without any visible label, if $u = \tau$, or 2)
% $u$ is the unique label in the list, if $u \neq \tau$.
\begin{alltt}
\HOLTokenTurnstile{} \HOLFreeVar{P} \HOLTokenWeakTransBegin\HOLFreeVar{u}\HOLTokenWeakTransEnd \HOLFreeVar{P\sp{\prime}} \HOLSymConst{\HOLTokenEquiv{}}
   \HOLSymConst{\HOLTokenExists{}}\HOLBoundVar{acts}.
       \HOLConst{TRACE} \HOLFreeVar{P} \HOLBoundVar{acts} \HOLFreeVar{P\sp{\prime}} \HOLSymConst{\HOLTokenConj{}} \HOLSymConst{\HOLTokenNeg{}}\HOLConst{NULL} \HOLBoundVar{acts} \HOLSymConst{\HOLTokenConj{}}
       \HOLKeyword{if} \HOLFreeVar{u} \HOLSymConst{=} \HOLSymConst{\ensuremath{\tau}} \HOLKeyword{then} \HOLConst{NO_LABEL} \HOLBoundVar{acts} \HOLKeyword{else} \HOLConst{UNIQUE_LABEL} \HOLFreeVar{u} \HOLBoundVar{acts}\hfill{[WEAK_TRANS_AND_TRACE]}
\end{alltt}
%\end{lemma}

Now the formalised version of Lemma~\ref{l:uptocon}:
\hfill{\texttt{[UNIQUE_SOLUTION_OF_CONTRACTIONS_LEMMA]}\vspace{-1em}
\begin{alltt}
\begin{small}
\HOLTokenTurnstile{} (\HOLSymConst{\HOLTokenExists{}}\HOLBoundVar{E}. \HOLConst{WGS} \HOLBoundVar{E} \HOLSymConst{\HOLTokenConj{}} \HOLFreeVar{P} \HOLSymConst{\HOLTokenContracts{}} \HOLBoundVar{E} \HOLFreeVar{P} \HOLSymConst{\HOLTokenConj{}} \HOLFreeVar{Q} \HOLSymConst{\HOLTokenContracts{}} \HOLBoundVar{E} \HOLFreeVar{Q}) \HOLSymConst{\HOLTokenImp{}}
   \HOLSymConst{\HOLTokenForall{}}\HOLBoundVar{C}.
       \HOLConst{GCONTEXT} \HOLBoundVar{C} \HOLSymConst{\HOLTokenImp{}}
       (\HOLSymConst{\HOLTokenForall{}}\HOLBoundVar{l} \HOLBoundVar{R}.
            \HOLBoundVar{C} \HOLFreeVar{P} \HOLTokenWeakTransBegin\HOLConst{label} \HOLBoundVar{l}\HOLTokenWeakTransEnd \HOLBoundVar{R} \HOLSymConst{\HOLTokenImp{}}
            \HOLSymConst{\HOLTokenExists{}}\HOLBoundVar{C\sp{\prime}}.
                \HOLConst{GCONTEXT} \HOLBoundVar{C\sp{\prime}} \HOLSymConst{\HOLTokenConj{}} \HOLBoundVar{R} \HOLSymConst{\HOLTokenContracts{}} \HOLBoundVar{C\sp{\prime}} \HOLFreeVar{P} \HOLSymConst{\HOLTokenConj{}}
                (\HOLConst{WEAK_EQUIV} \HOLSymConst{\HOLTokenRCompose{}} (\HOLTokenLambda{}\HOLBoundVar{x} \HOLBoundVar{y}. \HOLBoundVar{x} \HOLTokenWeakTransBegin\HOLConst{label} \HOLBoundVar{l}\HOLTokenWeakTransEnd \HOLBoundVar{y})) (\HOLBoundVar{C} \HOLFreeVar{Q})
                  (\HOLBoundVar{C\sp{\prime}} \HOLFreeVar{Q})) \HOLSymConst{\HOLTokenConj{}}
       \HOLSymConst{\HOLTokenForall{}}\HOLBoundVar{R}.
           \HOLBoundVar{C} \HOLFreeVar{P} \HOLTokenWeakTransBegin\HOLSymConst{\ensuremath{\tau}}\HOLTokenWeakTransEnd \HOLBoundVar{R} \HOLSymConst{\HOLTokenImp{}}
           \HOLSymConst{\HOLTokenExists{}}\HOLBoundVar{C\sp{\prime}}.
               \HOLConst{GCONTEXT} \HOLBoundVar{C\sp{\prime}} \HOLSymConst{\HOLTokenConj{}} \HOLBoundVar{R} \HOLSymConst{\HOLTokenContracts{}} \HOLBoundVar{C\sp{\prime}} \HOLFreeVar{P} \HOLSymConst{\HOLTokenConj{}}
               (\HOLConst{WEAK_EQUIV} \HOLSymConst{\HOLTokenRCompose{}} \HOLConst{EPS}) (\HOLBoundVar{C} \HOLFreeVar{Q}) (\HOLBoundVar{C\sp{\prime}} \HOLFreeVar{Q})
\end{small}
\end{alltt}
\vspace{-1em}
Traces are actually used in the proof of above lemma via 
the following ``unfolding lemma'':\vspace{-1em}
\begin{alltt}
\begin{small}
\HOLTokenTurnstile{} \HOLConst{GCONTEXT} \HOLFreeVar{C} \HOLSymConst{\HOLTokenConj{}} \HOLConst{WGS} \HOLFreeVar{E} \HOLSymConst{\HOLTokenConj{}} \HOLConst{TRACE} ((\HOLFreeVar{C} \HOLSymConst{\HOLTokenCompose} \HOLConst{FUNPOW} \HOLFreeVar{E} \HOLFreeVar{n}) \HOLFreeVar{P}) \HOLFreeVar{xs} \HOLFreeVar{P\sp{\prime}} \HOLSymConst{\HOLTokenConj{}}
   \HOLConst{LENGTH} \HOLFreeVar{xs} \HOLSymConst{\HOLTokenLeq{}} \HOLFreeVar{n} \HOLSymConst{\HOLTokenImp{}}
   \HOLSymConst{\HOLTokenExists{}}\HOLBoundVar{C\sp{\prime}}.
       \HOLConst{GCONTEXT} \HOLBoundVar{C\sp{\prime}} \HOLSymConst{\HOLTokenConj{}} (\HOLFreeVar{P\sp{\prime}} \HOLSymConst{=} \HOLBoundVar{C\sp{\prime}} \HOLFreeVar{P}) \HOLSymConst{\HOLTokenConj{}}
       \HOLSymConst{\HOLTokenForall{}}\HOLBoundVar{Q}. \HOLConst{TRACE} ((\HOLFreeVar{C} \HOLSymConst{\HOLTokenCompose} \HOLConst{FUNPOW} \HOLFreeVar{E} \HOLFreeVar{n}) \HOLBoundVar{Q}) \HOLFreeVar{xs} (\HOLBoundVar{C\sp{\prime}} \HOLBoundVar{Q})\hfill{[unfolding_lemma4]}
\end{small}
\end{alltt}
\vspace{-1em}
It roughly says, for any context $C$ and weakly-guarded context
$E$, if $C [\, E^n[P]\,] \overset{xs}{\Longrightarrow} P'$ and the length
of actions $xs \leqslant n$, then $P$ has the form of $C'[P]$ (meaning
that $P$ is not touched during the transitions). Traces are used for
reasoning about the \hl{number} of intermediate actions in weak
transitions. For instance, from Def.~\ref{d:BisCon}, \hl{it is easy
to see that, a weak transition either becomes shorter
or remains the same when moving between $\mcontrBIS$-related processes}.
\hl{This property is essential} in the proof of
Lemma~\ref{l:uptocon}. We show only one such lemma, for the case of
$\tau$-transitions passing into $\mcontrBIS$ (from left to right):
\begin{alltt}
\HOLTokenTurnstile{} \HOLFreeVar{P} \HOLSymConst{\HOLTokenContracts{}} \HOLFreeVar{Q} \HOLSymConst{\HOLTokenImp{}}
   \HOLSymConst{\HOLTokenForall{}}\HOLBoundVar{xs} \HOLBoundVar{P\sp{\prime}}.
       \HOLConst{TRACE} \HOLFreeVar{P} \HOLBoundVar{xs} \HOLBoundVar{P\sp{\prime}} \HOLSymConst{\HOLTokenConj{}} \HOLConst{NO_LABEL} \HOLBoundVar{xs} \HOLSymConst{\HOLTokenImp{}}
       \HOLSymConst{\HOLTokenExists{}}\HOLBoundVar{xs\sp{\prime}} \HOLBoundVar{Q\sp{\prime}}.
           \HOLConst{TRACE} \HOLFreeVar{Q} \HOLBoundVar{xs\sp{\prime}} \HOLBoundVar{Q\sp{\prime}} \HOLSymConst{\HOLTokenConj{}} \HOLBoundVar{P\sp{\prime}} \HOLSymConst{\HOLTokenContracts{}} \HOLBoundVar{Q\sp{\prime}} \HOLSymConst{\HOLTokenConj{}} \HOLConst{LENGTH} \HOLBoundVar{xs\sp{\prime}} \HOLSymConst{\HOLTokenLeq{}} \HOLConst{LENGTH} \HOLBoundVar{xs} \HOLSymConst{\HOLTokenConj{}}
           \HOLConst{NO_LABEL} \HOLBoundVar{xs\sp{\prime}}\hfill{[contracts_AND_TRACE_tau]}
\end{alltt}

\hl{With all above lemmas, we can thus finally prove Theorem~\ref{t:contraBisimulationU}:}
\begin{alltt}
\HOLTokenTurnstile{} \HOLConst{WGS} \HOLFreeVar{E} \HOLSymConst{\HOLTokenImp{}} \HOLSymConst{\HOLTokenForall{}}\HOLBoundVar{P} \HOLBoundVar{Q}. \HOLBoundVar{P} \HOLSymConst{\HOLTokenContracts{}} \HOLFreeVar{E} \HOLBoundVar{P} \HOLSymConst{\HOLTokenConj{}} \HOLBoundVar{Q} \HOLSymConst{\HOLTokenContracts{}} \HOLFreeVar{E} \HOLBoundVar{Q} \HOLSymConst{\HOLTokenImp{}} \HOLBoundVar{P} \HOLSymConst{\HOLTokenWeakEQ} \HOLBoundVar{Q}
\hfill{[UNIQUE_SOLUTION_OF_CONTRACTIONS]}
\end{alltt}
\vspace{-2ex}

\subsection{Unique solution of rooted contractions}

The formal proof of ``unique solution of rooted contractions theorem''
(Theorem~\ref{t:rcontraBisimulationU}) has the
same initial proof steps as Theorem~\ref{t:contraBisimulationU}; 
it then requires a
few more steps to handle  rooted bisimilarity in the conclusion. 
Overall  the
two proofs are very similar, mostly because the only property we need
from (rooted) contraction is its precongruence. 
 Below is the formally verified version of
Theorem~\ref{t:rcontraBisimulationU}
(having proved
the precongruence of rooted contraction, 
we can use  weakly-guarded expressions,   without constraints on  sums;
that is, \texttt{WG} in place of \texttt{WGS}):
\begin{alltt}
\HOLTokenTurnstile{} \HOLConst{WG} \HOLFreeVar{E} \HOLSymConst{\HOLTokenImp{}} \HOLSymConst{\HOLTokenForall{}}\HOLBoundVar{P} \HOLBoundVar{Q}. \HOLBoundVar{P} \HOLSymConst{\HOLTokenObsContracts} \HOLFreeVar{E} \HOLBoundVar{P} \HOLSymConst{\HOLTokenConj{}} \HOLBoundVar{Q} \HOLSymConst{\HOLTokenObsContracts} \HOLFreeVar{E} \HOLBoundVar{Q} \HOLSymConst{\HOLTokenImp{}} \HOLBoundVar{P} \HOLSymConst{\HOLTokenObsCongr} \HOLBoundVar{Q}
\hfill{[UNIQUE_SOLUTION_OF_ROOTED_CONTRACTIONS]}
\end{alltt}

Having removed the  constraints on sums, the result is
 similar to Milner's original `unique solution of
equations' theorem for \emph{strong} bisimilarity ($\sim$)~--- 
the same weakly guarded context (\texttt{WG}) is required:
\begin{alltt}
\HOLTokenTurnstile{} \HOLConst{WG} \HOLFreeVar{E} \HOLSymConst{\HOLTokenImp{}} \HOLSymConst{\HOLTokenForall{}}\HOLBoundVar{P} \HOLBoundVar{Q}. \HOLBoundVar{P} \HOLSymConst{\HOLTokenStrongEQ} \HOLFreeVar{E} \HOLBoundVar{P} \HOLSymConst{\HOLTokenConj{}} \HOLBoundVar{Q} \HOLSymConst{\HOLTokenStrongEQ} \HOLFreeVar{E} \HOLBoundVar{Q} \HOLSymConst{\HOLTokenImp{}} \HOLBoundVar{P} \HOLSymConst{\HOLTokenStrongEQ} \HOLBoundVar{Q}\hfill{[STRONG_UNIQUE_SOLUTION]}
\end{alltt}
In contrast, Milner's ``unique solution of
equations'' theorem for rooted bisimilarity ($\rapprox$)
has more severe constraints (both strongly guarded and sequential):
% Or our Theorem~\ref{t:rcontraBisimulationU} can be seen as a more
% applicable version of Milner's ``unique solution of
% equations'' theorem for rooted bisimilarity ($\rapprox$), which has more
% restrictions on equations:
\begin{alltt}
\HOLTokenTurnstile{} \HOLConst{SG} \HOLFreeVar{E} \HOLSymConst{\HOLTokenConj{}} \HOLConst{SEQ} \HOLFreeVar{E} \HOLSymConst{\HOLTokenImp{}} \HOLSymConst{\HOLTokenForall{}}\HOLBoundVar{P} \HOLBoundVar{Q}. \HOLBoundVar{P} \HOLSymConst{\HOLTokenObsCongr} \HOLFreeVar{E} \HOLBoundVar{P} \HOLSymConst{\HOLTokenConj{}} \HOLBoundVar{Q} \HOLSymConst{\HOLTokenObsCongr} \HOLFreeVar{E} \HOLBoundVar{Q} \HOLSymConst{\HOLTokenImp{}} \HOLBoundVar{P} \HOLSymConst{\HOLTokenObsCongr} \HOLBoundVar{Q}
\hfill{[OBS_UNIQUE_SOLUTION]}
\end{alltt}
\vspace{-4ex}


% part 3
\section{Related work on formalisation}
\label{s:rel}

Monica Nesi did the first CCS formalisations for both pure and
value-passing CCS \cite{Nesi:1992ve,Nesi:2017wo} using early versions of the HOL
theorem prover.\footnote{Part of this work can now be found at
  \url{https://github.com/binghe/HOL-CCS/tree/master/CCS-Nesi}.}
Her main focus was on implementing decision procedures (as a ML
program, e.g.~\cite{cleaveland1993concurrency}) for
automatically proving bisimilarities of CCS processes.
Her work has been a basis for ours~\cite{Tian:2017wrba}.
However, the differences are substantial, especially in the way of defining
bisimilarities. We greatly benefited from new features and standard
libraries in recent versions of HOL4, and our formalisation has
covered a  larger spectrum of the (pure) CCS theory.

Bengtson, Parrow and Weber did a substantial formalisation work
on CCS, $\pi$-calculi
and $\psi$-calculi 
using Isabelle/HOL and its nominal logic, with the main focus on the handling of
name binders \cite{bengtson2007completeness,parrow2009formalising}.
More details can be found in Bengtson's PhD thesis~\cite{bengtson2010formalising}. For CCS, 
he has formalised basic properties for strong/weak equivalence (congruence, basic % editorial
 algebraic laws); the CCS syntax does not have constants
or recursion, using instead replication.
% The formalisation effort has then been continued 
%
% On the other side, Jesper
% Bengtson and Joachim Parrow made great progress on $\pi$-calculus
% formalization and 
% proved that the algebraic axiomatization of bisimulation
% equivalence in the $\pi$-calculi is sound and
% complete. \cite{bengtson2007completeness} 
Other formalisations in this area include the early work of T.F.~Melham
\cite{melham1994mechanized} and O.A.~Mohamed
\cite{mohamed1995mechanizing} in HOL, Compton
\cite{compton2005embedding} in Isabelle/HOL,
Solange\footnote{\url{https://github.com/coq-contribs/ccs}} in Coq
and Chaudhuri et al.\;\cite{chaudhuri2015lightweight} in Abella, the latter
focuses on ``bisimulation up-to'' techniques (for strong bisimilarity)
for CCS and $\pi$-calculus.
Damien Pous \cite{pous2007new} also formalised up-to techniques and some CCS examples in
Coq.
Formalisations less related to ours
include Kahsai and Miculan \cite{kahsai2008implementing} for the spi
calculus in Isabelle/HOL, and Hirschkoff \cite{hirschkoff1997full} for the $\pi$-calculus in Coq.

% next file: conclusions.htex

%%%% -*- Mode: LaTeX -*-

\section{Conclusions and future work}
\label{s:concl}

In this paper, we have highlighted a formalisation of the theory of CCS in the 
HOL4 theorem prover.
%  (for lack of space we have not discussed 
% the formalisation of some basic algebraic theory, of the basic
% properties of the expansion preorder,   and of a few
%  versions of `bisimulation up to'
% techniques). %
%such as  bisimulation up-to bisimilarity). 
The formalisation supports 4 methods for establishing (strong and weak) bisimilarity
results: 
\begin{enumerate}
\item
 constructing a bisimulation (the standard bisimulation proof
method);
\item constructing a `bisimulation up-to'; 
\item employing algebraic laws;
\item defining a system of equations or contractions
(i.e., the `unique-solution' method)
\end{enumerate}

The formalisation has actually focused on the theory of
unique solution of equations and contractions. It    
 has also allowed us to further develop the theory,
notably the basic properties of rooted contraction, and the unique
solution theorem for it with respect to rooted bisimilarity. 
The formalisation brings up and exploits similarities between results
and proofs for different equivalences and preorders. Indeed we have
considered several `unique-solution' results (for various equivalences
and preorders); they share many parts of the proofs, but present a few
delicate and subtle differences in a few points. In a paper-pencil
proof, checking all details would be long and \hl{error-prone},
\hl{especially in cases where the proofs are similar to each other or
  when there are long case analyses to be carried out.}
Some of the textbook proofs are even wrong, e.g. Milner's
proof of the unique-solution theorem for $wb$
(Theorem~\ref{t:Mil89s3}).
For some other textbook proofs, even they are correct, sometimes they actually
do not need all information from the antecedents, which can be further
weakened. This kind of improvements can be easily found during the
formalisation work.
For our CCS formalisation, we believe that all the definitions and
theorem statements are easy to read and
understand, as they are very close to their original statements in
textbooks~\cite{Gorrieri:2015jt,Mil89}.

Formalising other equivalences and preorders could also be considered,
notably the \hl{trace-based} equivalences, as well as more refined process
calculi such as value-passing CCS.
% (e.g.~exploiting the type variable
%of actions).}
%
On another research line, one could examine the formalisation of a different
approach \cite{DurierHS17} to unique
solutions, in which the use of contraction is
replaced by semantic conditions on process transitions such as
divergence. 
%We hope this work also inspires new formalisations on other process calculi.

% Further plan on the formalisation mainly includes: 1) the extension to
% multi-variable equation/contractions. 2) the support of recursion
% operators in 
% CCS context and expressions.

% We believe that, beside the discovery of the rooted contraction $\rcontr$
% with a more elegant unique solution theorem,
% our CCS formalisation in HOL4 has also
% provided a solid formal framework for future theoretical developments in
% Concurrency Theory, particularly for process algebras like CCS. It's
% easily understandable, with statements extremely close to the original
% textbook. The logic foundations of HOL makes the whole work (or individual
% parts) easily portable to other theorem provers.

\paragraph{Acknowledgements}

We have  benefitted from suggestions and comments 
from  the \hl{anonymous reviewers} and several people from the HOL
community, including (in \hl{alphabetic} order) Robert Beers, Jeremy Dawson,
Ramana Kumar,
Michael Norrish, 
Konrad Slind, and
Thomas T\"{u}rk.
%
The paper was written in memory of Michael J.~C.~Gordon, the creator of the HOL theorem prover.


%\newpage

\bibliography{generic}
\end{document}
