%%%% -*- Mode: LaTeX -*-
%%
%% This is the draft of the 2nd part of EXPRESS/SOS 2018 paper, co-authored by
%% Prof. Davide Sangiorgi and Chun Tian.

\subsection{Unique solution of equations}
\label{ss:part2}

\hl{In this section we describe the formalisation of Milner's
``unique solution of equations'' theorems, limited to} \univariate equations. (The
\multivariate extension is described in Section~\ref{sec:multivariate}.)

\subsubsection{The version for strong bisimilarity}

\hl{Using ``bisimulation up to $\sim$'' technique}, we first proved the following
key lemma, which states that, if $X$ is weakly guarded
in $E$, then the ``first move'' of $E$ is independent of the agent
substituted for $X$:
\begin{lemma}[\texttt{STRONG_UNIQUE_SOLUTION_LEMMA}, Lemma 3.13
  of~\cite{Mil89}, \univariate version]
\label{lem:313}
If the variable $X$ is weakly guarded in $E$, and
$E\{P/X\}\overset{\alpha}{\rightarrow} P'$, then $P'$ takes the form
$E'\{P/X\}$ (for some expression $E'$), and moreover, for any $Q$,
$E\{Q/X\}\overset{\alpha}{\rightarrow} E'\{Q/X\}$:
\begin{alltt}
\HOLTokenTurnstile{} \HOLConst{WG} \HOLFreeVar{E} \HOLSymConst{\HOLTokenImp{}}
   \HOLSymConst{\HOLTokenForall{}}\HOLBoundVar{P} \HOLBoundVar{a} \ensuremath{\HOLBoundVar{P}\sp{\prime}}. \HOLFreeVar{E} \HOLBoundVar{P} \HOLTokenTransBegin\HOLBoundVar{a}\HOLTokenTransEnd \ensuremath{\HOLBoundVar{P}\sp{\prime}} \HOLSymConst{\HOLTokenImp{}} \HOLSymConst{\HOLTokenExists{}}\ensuremath{\HOLBoundVar{E}\sp{\prime}}. \HOLConst{CONTEXT} \ensuremath{\HOLBoundVar{E}\sp{\prime}} \HOLSymConst{\HOLTokenConj{}} \ensuremath{\HOLBoundVar{P}\sp{\prime}} \HOLSymConst{\ensuremath{=}} \ensuremath{\HOLBoundVar{E}\sp{\prime}} \HOLBoundVar{P} \HOLSymConst{\HOLTokenConj{}} \HOLSymConst{\HOLTokenForall{}}\HOLBoundVar{Q}. \HOLFreeVar{E} \HOLBoundVar{Q} \HOLTokenTransBegin\HOLBoundVar{a}\HOLTokenTransEnd \ensuremath{\HOLBoundVar{E}\sp{\prime}} \HOLBoundVar{Q}
\end{alltt}
\end{lemma}

Then, by structural induction on weakly guarded contexts
(\texttt{WG}), we have proved Theorem~\ref{t:Mil89s1}
in the \univariate case. The formal proof basically follows the outline of its
informal version~\citep[p.~102--103]{Mil89}, which is tedious due to
large amounts of case analyses on each CCS operator.
\begin{theorem}[\texttt{STRONG_UNIQUE_SOLUTION}, \univariate version
  of Theorem~\ref{t:Mil89s1}]
  \label{thm:Mil89s1f}
Suppose the expression $E$ contains at most the variable $X$, and let $X$ be weakly
guarded in $E$.
\begin{equation}
\text{If } P \sim E\{P/X\} \text{ and } Q \sim E\{Q/X\} \text{ then }
P \sim Q.
\end{equation}
\begin{alltt}
\HOLTokenTurnstile{} \HOLConst{WG} \HOLFreeVar{E} \HOLSymConst{\HOLTokenConj{}} \HOLFreeVar{P} \HOLSymConst{\HOLTokenStrongEQ} \HOLFreeVar{E} \HOLFreeVar{P} \HOLSymConst{\HOLTokenConj{}} \HOLFreeVar{Q} \HOLSymConst{\HOLTokenStrongEQ} \HOLFreeVar{E} \HOLFreeVar{Q} \HOLSymConst{\HOLTokenImp{}} \HOLFreeVar{P} \HOLSymConst{\HOLTokenStrongEQ} \HOLFreeVar{Q}
\end{alltt}
\end{theorem}

\subsubsection{The version for rooted bisimilarity}

For the proof of Theorem~\ref{t:Mil89s3}, we did not use any
``bisimulation up-to'' technique.
\hl{Instead,} we have used a different
technique based on Lemma~\ref{l:obsCongrByWeakBisim}, whose formal
version is the following one (\texttt{OBS_CONGR_BY_WEAK_BISIM}):
\begin{alltt}
\HOLTokenTurnstile{} \HOLConst{WEAK_BISIM} \HOLFreeVar{Wbsm} \HOLSymConst{\HOLTokenImp{}}
   \HOLSymConst{\HOLTokenForall{}}\HOLBoundVar{E} \ensuremath{\HOLBoundVar{E}\sp{\prime}}.
       \ensuremath{(}\HOLSymConst{\HOLTokenForall{}}\HOLBoundVar{u}.
            \ensuremath{(}\HOLSymConst{\HOLTokenForall{}}\ensuremath{\HOLBoundVar{E}\sb{\mathrm{1}}}. \HOLBoundVar{E} \HOLTokenTransBegin\HOLBoundVar{u}\HOLTokenTransEnd \ensuremath{\HOLBoundVar{E}\sb{\mathrm{1}}} \HOLSymConst{\HOLTokenImp{}} \HOLSymConst{\HOLTokenExists{}}\ensuremath{\HOLBoundVar{E}\sb{\mathrm{2}}}. \ensuremath{\HOLBoundVar{E}\sp{\prime}} \HOLTokenWeakTransBegin\HOLBoundVar{u}\HOLTokenWeakTransEnd \ensuremath{\HOLBoundVar{E}\sb{\mathrm{2}}} \HOLSymConst{\HOLTokenConj{}} \HOLFreeVar{Wbsm} \ensuremath{\HOLBoundVar{E}\sb{\mathrm{1}}} \ensuremath{\HOLBoundVar{E}\sb{\mathrm{2}}}\ensuremath{)} \HOLSymConst{\HOLTokenConj{}}
            \HOLSymConst{\HOLTokenForall{}}\ensuremath{\HOLBoundVar{E}\sb{\mathrm{2}}}. \ensuremath{\HOLBoundVar{E}\sp{\prime}} \HOLTokenTransBegin\HOLBoundVar{u}\HOLTokenTransEnd \ensuremath{\HOLBoundVar{E}\sb{\mathrm{2}}} \HOLSymConst{\HOLTokenImp{}} \HOLSymConst{\HOLTokenExists{}}\ensuremath{\HOLBoundVar{E}\sb{\mathrm{1}}}. \HOLBoundVar{E} \HOLTokenWeakTransBegin\HOLBoundVar{u}\HOLTokenWeakTransEnd \ensuremath{\HOLBoundVar{E}\sb{\mathrm{1}}} \HOLSymConst{\HOLTokenConj{}} \HOLFreeVar{Wbsm} \ensuremath{\HOLBoundVar{E}\sb{\mathrm{1}}} \ensuremath{\HOLBoundVar{E}\sb{\mathrm{2}}}\ensuremath{)} \HOLSymConst{\HOLTokenImp{}}
       \HOLBoundVar{E} \HOLSymConst{\HOLTokenObsCongr} \ensuremath{\HOLBoundVar{E}\sp{\prime}}
\end{alltt}
\hl{By using Lemma}~\ref{l:obsCongrByWeakBisim}, \hl{the next two results} can be proved by 
directly constructing the required bisimulation:
(see~\cite{Tian:2017wrba} for more details)
\begin{lemma}[\texttt{OBS_UNIQUE_SOLUTION_LEMMA}]
If the variable $X$ is guarded and sequential in $G$, and
$G\{P/X\}\overset{\alpha}{\rightarrow} P'$, then $P'$ takes the form
$H\{P/X\}$, for some  $H$, and for any $Q$, we also have 
$G\{Q/X\}\overset{\alpha}{\rightarrow} H\{Q/X\}$. Moreover $H$ is
sequential, and if $\alpha = \tau$, then $H$ is also guarded.
\begin{alltt}
\HOLTokenTurnstile{} \HOLConst{SG} \HOLFreeVar{G} \HOLSymConst{\HOLTokenConj{}} \HOLConst{SEQ} \HOLFreeVar{G} \HOLSymConst{\HOLTokenImp{}}
   \HOLSymConst{\HOLTokenForall{}}\HOLBoundVar{P} \HOLBoundVar{a} \ensuremath{\HOLBoundVar{P}\sp{\prime}}.
       \HOLFreeVar{G} \HOLBoundVar{P} \HOLTokenTransBegin\HOLBoundVar{a}\HOLTokenTransEnd \ensuremath{\HOLBoundVar{P}\sp{\prime}} \HOLSymConst{\HOLTokenImp{}}
       \HOLSymConst{\HOLTokenExists{}}\HOLBoundVar{H}. \HOLConst{SEQ} \HOLBoundVar{H} \HOLSymConst{\HOLTokenConj{}} \ensuremath{(}\HOLBoundVar{a} \HOLSymConst{\ensuremath{=}} \HOLSymConst{\ensuremath{\tau}} \HOLSymConst{\HOLTokenImp{}} \HOLConst{SG} \HOLBoundVar{H}\ensuremath{)} \HOLSymConst{\HOLTokenConj{}} \ensuremath{\HOLBoundVar{P}\sp{\prime}} \HOLSymConst{\ensuremath{=}} \HOLBoundVar{H} \HOLBoundVar{P} \HOLSymConst{\HOLTokenConj{}} \HOLSymConst{\HOLTokenForall{}}\HOLBoundVar{Q}. \HOLFreeVar{G} \HOLBoundVar{Q} \HOLTokenTransBegin\HOLBoundVar{a}\HOLTokenTransEnd \HOLBoundVar{H} \HOLBoundVar{Q}
\end{alltt}
\end{lemma}

\begin{theorem}[\texttt{OBS_UNIQUE_SOLUTION}, \univariate version of
  Theorem~\ref{t:Mil89s3}]
  \label{thm:Mil89s3f}
Let $E$ be guarded and sequential expressions, and let $P \rapprox
E\{P/X\}$,
$Q \rapprox E\{Q/X\}$. Then $P \rapprox Q$.
\begin{alltt}
\HOLTokenTurnstile{} \HOLConst{SG} \HOLFreeVar{E} \HOLSymConst{\HOLTokenConj{}} \HOLConst{SEQ} \HOLFreeVar{E} \HOLSymConst{\HOLTokenConj{}} \HOLFreeVar{P} \HOLSymConst{\HOLTokenObsCongr} \HOLFreeVar{E} \HOLFreeVar{P} \HOLSymConst{\HOLTokenConj{}} \HOLFreeVar{Q} \HOLSymConst{\HOLTokenObsCongr} \HOLFreeVar{E} \HOLFreeVar{Q} \HOLSymConst{\HOLTokenImp{}} \HOLFreeVar{P} \HOLSymConst{\HOLTokenObsCongr} \HOLFreeVar{Q}
\end{alltt}
\end{theorem}

\subsubsection{The version for weak bisimilarity}

Milner~\cite{Mil89} \emph{only mentioned} two ``unique solution of
equations'' theorems, one for strong equivalence, the other for
rooted bisimilarity.   There is, however, another
version for weak bisimilarity (Theorem~\ref{t:Mil89}) that shares a large portion of proof
steps with the proof for \hl{rooted bisimilarity.}
As weak bisimilarity is not a congruence, we have to be more restrictive
on the syntax of the equation, \hl{with only guarded sums.}
The related formal proofs are tedious but closely follow the informal
proof~\citep[p.~158--159]{Mil89}, \hl{except for the uses of} ``bisimulation up
to with weak arrows'' (Def.~\ref{def:doubleweak}) instead of the
original version (Def.~\ref{def:singleweak}).

\begin{lemma}[\texttt{WEAK_UNIQUE_SOLUTION_LEMMA}]
If the variable $X$ is guarded and sequential (with only guarded sums) in $G$, and
$G\{P/X\}\overset{\alpha}{\rightarrow} P'$, then $P'$ takes the form
$H\{P/X\}$, for some expression $H$, and for any $Q$, we have
$G\{Q/X\}\overset{\alpha}{\rightarrow} H\{Q/X\}$. Moreover $H$ is
sequential, and if $\alpha = \tau$ then $H$ is also guarded.
\begin{alltt}
\HOLTokenTurnstile{} \HOLConst{SG} \HOLFreeVar{G} \HOLSymConst{\HOLTokenConj{}} \HOLConst{GSEQ} \HOLFreeVar{G} \HOLSymConst{\HOLTokenImp{}}
   \HOLSymConst{\HOLTokenForall{}}\HOLBoundVar{P} \HOLBoundVar{a} \ensuremath{\HOLBoundVar{P}\sp{\prime}}.
       \HOLFreeVar{G} \HOLBoundVar{P} \HOLTokenTransBegin\HOLBoundVar{a}\HOLTokenTransEnd \ensuremath{\HOLBoundVar{P}\sp{\prime}} \HOLSymConst{\HOLTokenImp{}}
       \HOLSymConst{\HOLTokenExists{}}\HOLBoundVar{H}. \HOLConst{GSEQ} \HOLBoundVar{H} \HOLSymConst{\HOLTokenConj{}} \ensuremath{(}\HOLBoundVar{a} \HOLSymConst{\ensuremath{=}} \HOLSymConst{\ensuremath{\tau}} \HOLSymConst{\HOLTokenImp{}} \HOLConst{SG} \HOLBoundVar{H}\ensuremath{)} \HOLSymConst{\HOLTokenConj{}} \ensuremath{\HOLBoundVar{P}\sp{\prime}} \HOLSymConst{\ensuremath{=}} \HOLBoundVar{H} \HOLBoundVar{P} \HOLSymConst{\HOLTokenConj{}} \HOLSymConst{\HOLTokenForall{}}\HOLBoundVar{Q}. \HOLFreeVar{G} \HOLBoundVar{Q} \HOLTokenTransBegin\HOLBoundVar{a}\HOLTokenTransEnd \HOLBoundVar{H} \HOLBoundVar{Q}
\end{alltt}
\end{lemma}

\begin{theorem}[\texttt{WEAK_UNIQUE_SOLUTION}, \univariate version of
  Theorem~\ref{t:Mil89}]
  \label{thm:Mil89f}
Let $E$ be guarded and sequential expressions, and let $P \wb
E\{P/X\}$,
$Q \wb E\{Q/X\}$. Then $P \wb Q$.
\begin{alltt}
\HOLTokenTurnstile{} \HOLConst{SG} \HOLFreeVar{E} \HOLSymConst{\HOLTokenConj{}} \HOLConst{GSEQ} \HOLFreeVar{E} \HOLSymConst{\HOLTokenConj{}} \HOLFreeVar{P} \HOLSymConst{\HOLTokenWeakEQ} \HOLFreeVar{E} \HOLFreeVar{P} \HOLSymConst{\HOLTokenConj{}} \HOLFreeVar{Q} \HOLSymConst{\HOLTokenWeakEQ} \HOLFreeVar{E} \HOLFreeVar{Q} \HOLSymConst{\HOLTokenImp{}} \HOLFreeVar{P} \HOLSymConst{\HOLTokenWeakEQ} \HOLFreeVar{Q}
\end{alltt}
\end{theorem}

\subsection{Unique solution of contractions}

A delicate point in the formalisation of the results about unique solution of
contractions \hl{is} the proof of Lemma~\ref{l:ruptocon} and lemmas alike.
\hl{In particular, we have do} induction on the length of weak transitions.
For this, rather than introducing a refined form of weak transition relation
enriched with its length, we found it more elegant to work with
\emph{traces}, \hl{i.e. weak transitions with all intermediate actions.}
(A further motivation is to set the ground for future extensions of this
formalisation work to \emph{trace equivalence} in place of bisimilarity.)

A trace is formally represented by the initial and final processes, plus
a list of actions so performed.
For this, we first define the concept of \emph{label-accumulated reflexive transitive closure}
 (\HOLinline{\HOLConst{LRTC}}).
\hl{Then,} given any labeled transition relation \HOLinline{\HOLFreeVar{R}} (of the type
``\HOLinline{\ensuremath{\alpha} \HOLTokenTransEnd \ensuremath{\beta} \HOLTokenTransEnd \ensuremath{\alpha} \HOLTokenTransEnd \HOLTyOp{bool}}''), \HOLinline{\HOLConst{LRTC} \HOLFreeVar{R}} is
a relation representing \hl{traces} (of the type
``\HOLinline{\ensuremath{\alpha} \HOLTokenTransEnd \ensuremath{\beta} \HOLTyOp{list} \HOLTokenTransEnd \ensuremath{\alpha} \HOLTokenTransEnd \HOLTyOp{bool}}''):
\begin{alltt}
   \HOLConst{LRTC} \HOLFreeVar{R} \HOLFreeVar{a} \HOLFreeVar{l} \HOLFreeVar{b} \HOLTokenDefEquality{}
     \HOLSymConst{\HOLTokenForall{}}\HOLBoundVar{P}.
         \ensuremath{(}\HOLSymConst{\HOLTokenForall{}}\HOLBoundVar{x}. \HOLBoundVar{P} \HOLBoundVar{x} \ensuremath{[}\ensuremath{]} \HOLBoundVar{x}\ensuremath{)} \HOLSymConst{\HOLTokenConj{}}
         \ensuremath{(}\HOLSymConst{\HOLTokenForall{}}\HOLBoundVar{x} \HOLBoundVar{h} \HOLBoundVar{y} \HOLBoundVar{t} \HOLBoundVar{z}. \HOLFreeVar{R} \HOLBoundVar{x} \HOLBoundVar{h} \HOLBoundVar{y} \HOLSymConst{\HOLTokenConj{}} \HOLBoundVar{P} \HOLBoundVar{y} \HOLBoundVar{t} \HOLBoundVar{z} \HOLSymConst{\HOLTokenImp{}} \HOLBoundVar{P} \HOLBoundVar{x} \ensuremath{(}\HOLBoundVar{h}\HOLSymConst{::}\HOLBoundVar{t}\ensuremath{)} \HOLBoundVar{z}\ensuremath{)} \HOLSymConst{\HOLTokenImp{}}
         \HOLBoundVar{P} \HOLFreeVar{a} \HOLFreeVar{l} \HOLFreeVar{b}\hfill{[LRTC_DEF]}
\end{alltt}
The trace relation for CCS, \HOLinline{\HOLConst{TRACE}} (of the
type ``\HOLinline{\ensuremath{(}\ensuremath{\alpha}, \ensuremath{\beta}\ensuremath{)} \HOLTyOp{CCS} \HOLTokenTransEnd \ensuremath{\beta} \HOLTyOp{Action} \HOLTyOp{list} \HOLTokenTransEnd \ensuremath{(}\ensuremath{\alpha}, \ensuremath{\beta}\ensuremath{)} \HOLTyOp{CCS} \HOLTokenTransEnd \HOLTyOp{bool}}''),
can be then obtained
 by \hl{combining} \texttt{LRTC} and the \texttt{TRANS}
 ($\overset{\mu}{\rightarrow}$) relation inductively defined by SOS rules:
\begin{alltt}
   \HOLConst{TRACE} \HOLTokenDefEquality{} \HOLConst{LRTC} \HOLConst{TRANS}\hfill{[TRACE_def]}
\end{alltt}

\hl{Let the trace be $P\overset{acts}{\longrightarrow}Q$.}
If the list of actions $acts$ is empty, then there is no transition
actually: the initial and final processes are the same, i.e. $P = Q$.
\hl{On the other hand, if there is at most one visible action (i.e. a
label) $\mu$ in $acts$ (all other actions are $\tau$, if they exist),}
then the trace is actually a weak transition: $P \Arr{\mu} Q$.
Here we have to distinguish between two cases: no label and unique label (in
the list of actions). The definition of ``no
label'' in an action list is easy (here \texttt{MEM} tests if \hl{an} element is a member of a list):
\begin{alltt}
   \HOLConst{NO_LABEL} \HOLFreeVar{L} \HOLTokenDefEquality{} \HOLSymConst{\HOLTokenNeg{}}\HOLSymConst{\HOLTokenExists{}}\HOLBoundVar{l}. \HOLConst{MEM} \ensuremath{(}\HOLConst{label} \HOLBoundVar{l}\ensuremath{)} \HOLFreeVar{L}\hfill{[NO_LABEL_def]}
\end{alltt}

The definition of ``unique label'' can be done in many ways. The
following definition (due to a suggestion from Robert Beers)
avoids any \hl{kind} of counting or filtering in the list.
It says that a label is unique in a list of actions if and only if there is no
label in the rest of list (here $\HOLTokenDoublePlus$ \hl{denotes the concatenation
of lists}):
\begin{alltt}
   \HOLConst{UNIQUE_LABEL} \HOLFreeVar{u} \HOLFreeVar{L} \HOLTokenDefEquality{}
     \HOLSymConst{\HOLTokenExists{}}\ensuremath{\HOLBoundVar{L}\sb{\mathrm{1}}} \ensuremath{\HOLBoundVar{L}\sb{\mathrm{2}}}. \ensuremath{\HOLBoundVar{L}\sb{\mathrm{1}}} \HOLSymConst{\HOLTokenDoublePlus} \ensuremath{[}\HOLFreeVar{u}\ensuremath{]} \HOLSymConst{\HOLTokenDoublePlus} \ensuremath{\HOLBoundVar{L}\sb{\mathrm{2}}} \HOLSymConst{\ensuremath{=}} \HOLFreeVar{L} \HOLSymConst{\HOLTokenConj{}} \HOLConst{NO_LABEL} \ensuremath{\HOLBoundVar{L}\sb{\mathrm{1}}} \HOLSymConst{\HOLTokenConj{}} \HOLConst{NO_LABEL} \ensuremath{\HOLBoundVar{L}\sb{\mathrm{2}}}\hfill{[UNIQUE_LABEL_def]}
\end{alltt}

\hl{Finally, the connection} between traces and weak transitions is stated
and proved in the following theorem, \hl{which} says that a weak transition $P \Arr{u} P'$ is also a
trace $P\overset{acts}{\longrightarrow}P'$ with a
 non-empty action list $acts$, in which either there is no label (for $u = \tau$), or 
$u$ is the unique label (for $u \neq \tau$):
\begin{alltt}
\HOLTokenTurnstile{} \HOLFreeVar{P} \HOLTokenWeakTransBegin\HOLFreeVar{u}\HOLTokenWeakTransEnd \ensuremath{\HOLFreeVar{P}\sp{\prime}} \HOLSymConst{\HOLTokenEquiv{}}
   \HOLSymConst{\HOLTokenExists{}}\HOLBoundVar{acts}.
       \HOLConst{TRACE} \HOLFreeVar{P} \HOLBoundVar{acts} \ensuremath{\HOLFreeVar{P}\sp{\prime}} \HOLSymConst{\HOLTokenConj{}} \HOLSymConst{\HOLTokenNeg{}}\HOLConst{NULL} \HOLBoundVar{acts} \HOLSymConst{\HOLTokenConj{}}
       \HOLKeyword{if} \HOLFreeVar{u} \HOLSymConst{\ensuremath{=}} \HOLSymConst{\ensuremath{\tau}} \HOLKeyword{then} \HOLConst{NO_LABEL} \HOLBoundVar{acts} \HOLKeyword{else} \HOLConst{UNIQUE_LABEL} \HOLFreeVar{u} \HOLBoundVar{acts}\hfill{[WEAK_TRANS_AND_TRACE]}
\end{alltt}

Now the formal version of Lemma~\ref{l:uptocon} (\texttt{UNIQUE_SOLUTION_OF_CONTRACTIONS_LEMMA}):
\begin{alltt}
\HOLTokenTurnstile{} \ensuremath{(}\HOLSymConst{\HOLTokenExists{}}\HOLBoundVar{E}. \HOLConst{WGS} \HOLBoundVar{E} \HOLSymConst{\HOLTokenConj{}} \HOLFreeVar{P} \HOLSymConst{\HOLTokenContracts{}} \HOLBoundVar{E} \HOLFreeVar{P} \HOLSymConst{\HOLTokenConj{}} \HOLFreeVar{Q} \HOLSymConst{\HOLTokenContracts{}} \HOLBoundVar{E} \HOLFreeVar{Q}\ensuremath{)} \HOLSymConst{\HOLTokenImp{}}
   \HOLSymConst{\HOLTokenForall{}}\HOLBoundVar{C}.
       \HOLConst{GCONTEXT} \HOLBoundVar{C} \HOLSymConst{\HOLTokenImp{}}
       \ensuremath{(}\HOLSymConst{\HOLTokenForall{}}\HOLBoundVar{l} \HOLBoundVar{R}.
            \HOLBoundVar{C} \HOLFreeVar{P} \HOLTokenWeakTransBegin\HOLConst{label} \HOLBoundVar{l}\HOLTokenWeakTransEnd \HOLBoundVar{R} \HOLSymConst{\HOLTokenImp{}}
            \HOLSymConst{\HOLTokenExists{}}\ensuremath{\HOLBoundVar{C}\sp{\prime}}.
                \HOLConst{GCONTEXT} \ensuremath{\HOLBoundVar{C}\sp{\prime}} \HOLSymConst{\HOLTokenConj{}} \HOLBoundVar{R} \HOLSymConst{\HOLTokenContracts{}} \ensuremath{\HOLBoundVar{C}\sp{\prime}} \HOLFreeVar{P} \HOLSymConst{\HOLTokenConj{}}
                \ensuremath{(}\HOLConst{WEAK_EQUIV} \HOLSymConst{\HOLTokenRCompose{}} \ensuremath{(}\HOLTokenLambda{}\HOLBoundVar{x} \HOLBoundVar{y}. \HOLBoundVar{x} \HOLTokenWeakTransBegin\HOLConst{label} \HOLBoundVar{l}\HOLTokenWeakTransEnd \HOLBoundVar{y}\ensuremath{)}\ensuremath{)} \ensuremath{(}\HOLBoundVar{C} \HOLFreeVar{Q}\ensuremath{)} \ensuremath{(}\ensuremath{\HOLBoundVar{C}\sp{\prime}} \HOLFreeVar{Q}\ensuremath{)}\ensuremath{)} \HOLSymConst{\HOLTokenConj{}}
       \HOLSymConst{\HOLTokenForall{}}\HOLBoundVar{R}.
           \HOLBoundVar{C} \HOLFreeVar{P} \HOLTokenWeakTransBegin\HOLSymConst{\ensuremath{\tau}}\HOLTokenWeakTransEnd \HOLBoundVar{R} \HOLSymConst{\HOLTokenImp{}}
           \HOLSymConst{\HOLTokenExists{}}\ensuremath{\HOLBoundVar{C}\sp{\prime}}. \HOLConst{GCONTEXT} \ensuremath{\HOLBoundVar{C}\sp{\prime}} \HOLSymConst{\HOLTokenConj{}} \HOLBoundVar{R} \HOLSymConst{\HOLTokenContracts{}} \ensuremath{\HOLBoundVar{C}\sp{\prime}} \HOLFreeVar{P} \HOLSymConst{\HOLTokenConj{}} \ensuremath{(}\HOLConst{WEAK_EQUIV} \HOLSymConst{\HOLTokenRCompose{}} \HOLConst{EPS}\ensuremath{)} \ensuremath{(}\HOLBoundVar{C} \HOLFreeVar{Q}\ensuremath{)} \ensuremath{(}\ensuremath{\HOLBoundVar{C}\sp{\prime}} \HOLFreeVar{Q}\ensuremath{)}
\end{alltt}

Traces are actually used in the proof of above lemma via 
the following lemma (\texttt{unfolding_lemma4}):
\begin{alltt}
\HOLTokenTurnstile{} \HOLConst{GCONTEXT} \HOLFreeVar{C} \HOLSymConst{\HOLTokenConj{}} \HOLConst{WGS} \HOLFreeVar{E} \HOLSymConst{\HOLTokenConj{}} \HOLConst{TRACE} \ensuremath{(}\ensuremath{(}\HOLFreeVar{C} \HOLSymConst{\HOLTokenCompose} \HOLConst{FUNPOW} \HOLFreeVar{E} \HOLFreeVar{n}\ensuremath{)} \HOLFreeVar{P}\ensuremath{)} \HOLFreeVar{xs} \ensuremath{\HOLFreeVar{P}\sp{\prime}} \HOLSymConst{\HOLTokenConj{}} \HOLConst{LENGTH} \HOLFreeVar{xs} \HOLSymConst{\HOLTokenLeq{}} \HOLFreeVar{n} \HOLSymConst{\HOLTokenImp{}}
   \HOLSymConst{\HOLTokenExists{}}\ensuremath{\HOLBoundVar{C}\sp{\prime}}. \HOLConst{GCONTEXT} \ensuremath{\HOLBoundVar{C}\sp{\prime}} \HOLSymConst{\HOLTokenConj{}} \ensuremath{\HOLFreeVar{P}\sp{\prime}} \HOLSymConst{\ensuremath{=}} \ensuremath{\HOLBoundVar{C}\sp{\prime}} \HOLFreeVar{P} \HOLSymConst{\HOLTokenConj{}} \HOLSymConst{\HOLTokenForall{}}\HOLBoundVar{Q}. \HOLConst{TRACE} \ensuremath{(}\ensuremath{(}\HOLFreeVar{C} \HOLSymConst{\HOLTokenCompose} \HOLConst{FUNPOW} \HOLFreeVar{E} \HOLFreeVar{n}\ensuremath{)} \HOLBoundVar{Q}\ensuremath{)} \HOLFreeVar{xs} \ensuremath{(}\ensuremath{\HOLBoundVar{C}\sp{\prime}} \HOLBoundVar{Q}\ensuremath{)}
\end{alltt}
which roughly says that, for any context $C$ and weakly guarded context
$E$, if $C [\, E^n[P]\,] \overset{xs}{\longrightarrow} P'$ and the length
of actions $xs \leqslant n$, then $P'$ has the form $C'[P]$.
Traces are used in reasoning about the number of intermediate actions in weak
transitions. For instance, from Def.~\ref{d:BisCon}, it is easy
to see that, a weak transition either becomes shorter
or remains the same when moving between $\mcontrBIS$-related processes.
This property is essential in the proof of
Lemma~\ref{l:uptocon}. We show only one such lemma, for the case of
visible weak transitions:
%passing into $\mcontrBIS$ (from left to right):
\begin{alltt}
\HOLTokenTurnstile{} \HOLFreeVar{P} \HOLSymConst{\HOLTokenContracts{}} \HOLFreeVar{Q} \HOLSymConst{\HOLTokenImp{}}
   \HOLSymConst{\HOLTokenForall{}}\HOLBoundVar{xs} \HOLBoundVar{l} \ensuremath{\HOLBoundVar{P}\sp{\prime}}.
       \HOLConst{TRACE} \HOLFreeVar{P} \HOLBoundVar{xs} \ensuremath{\HOLBoundVar{P}\sp{\prime}} \HOLSymConst{\HOLTokenConj{}} \HOLConst{UNIQUE_LABEL} \ensuremath{(}\HOLConst{label} \HOLBoundVar{l}\ensuremath{)} \HOLBoundVar{xs} \HOLSymConst{\HOLTokenImp{}}
       \HOLSymConst{\HOLTokenExists{}}\ensuremath{\HOLBoundVar{xs}\sp{\prime}} \ensuremath{\HOLBoundVar{Q}\sp{\prime}}.
           \HOLConst{TRACE} \HOLFreeVar{Q} \ensuremath{\HOLBoundVar{xs}\sp{\prime}} \ensuremath{\HOLBoundVar{Q}\sp{\prime}} \HOLSymConst{\HOLTokenConj{}} \HOLFreeVar{P} \HOLSymConst{\HOLTokenContracts{}} \HOLFreeVar{Q} \HOLSymConst{\HOLTokenConj{}} \HOLConst{LENGTH} \ensuremath{\HOLBoundVar{xs}\sp{\prime}} \HOLSymConst{\HOLTokenLeq{}} \HOLConst{LENGTH} \HOLBoundVar{xs} \HOLSymConst{\HOLTokenConj{}}
           \HOLConst{UNIQUE_LABEL} \ensuremath{(}\HOLConst{label} \HOLBoundVar{l}\ensuremath{)} \ensuremath{\HOLBoundVar{xs}\sp{\prime}}\hfill{[contracts_AND_TRACE_label]}
\end{alltt}

With all above lemmas, we can thus finally prove Theorem~\ref{t:contraBisimulationU}:
\begin{alltt}
\HOLTokenTurnstile{} \HOLConst{WGS} \HOLFreeVar{E} \HOLSymConst{\HOLTokenConj{}} \HOLFreeVar{P} \HOLSymConst{\HOLTokenContracts{}} \HOLFreeVar{E} \HOLFreeVar{P} \HOLSymConst{\HOLTokenConj{}} \HOLFreeVar{Q} \HOLSymConst{\HOLTokenContracts{}} \HOLFreeVar{E} \HOLFreeVar{Q} \HOLSymConst{\HOLTokenImp{}} \HOLFreeVar{P} \HOLSymConst{\HOLTokenWeakEQ} \HOLFreeVar{Q}\hfill{[UNIQUE_SOLUTION_OF_CONTRACTIONS]}
\end{alltt}

\subsection{Unique solution of rooted contractions}

The formal proof of ``unique solution of rooted contractions theorem''
(Theorem~\ref{t:rcontraBisimulationU}, \univariate version) shares the
same initial proof steps as Theorem~\ref{t:contraBisimulationU}.
It then requires a
few more steps to handle rooted bisimilarity in the conclusion. 
Overall \hl{speaking}, the two proofs are very similar, mostly because
the only property needed from \hl{contraction and rooted contraction is their} precongruence.
Below is the formal version of Theorem~\ref{t:rcontraBisimulationU}:
\begin{alltt}
\HOLTokenTurnstile{} \HOLConst{WG} \HOLFreeVar{E} \HOLSymConst{\HOLTokenConj{}} \HOLFreeVar{P} \HOLSymConst{\HOLTokenObsContracts} \HOLFreeVar{E} \HOLFreeVar{P} \HOLSymConst{\HOLTokenConj{}} \HOLFreeVar{Q} \HOLSymConst{\HOLTokenObsContracts} \HOLFreeVar{E} \HOLFreeVar{Q} \HOLSymConst{\HOLTokenImp{}} \HOLFreeVar{P} \HOLSymConst{\HOLTokenObsCongr} \HOLFreeVar{Q}\hfill{[UNIQUE_SOLUTION_OF_ROOTED_CONTRACTIONS]}
\end{alltt}

Having proved the precongruence property of rooted contraction ($\rcontr$), 
now we can use weakly guarded expressions in the above theorem,
which has no more constraints on summation, that is, having
\texttt{WG} in place of \texttt{WGS} (cf.~\texttt{UNIQUE_SOLUTION_OF_CONTRACTIONS}).

On the other hand, having removed the constraints on \hl{summation}, now
this result is very similar to Milner's original `unique solution of
equations' theorem for \emph{strong} bisimilarity ($\sim$) (Theorem~\ref{t:Mil89s1})~--- 
the same weakly guarded context (\texttt{WG}) is required
(cf.~Theorem~\ref{thm:Mil89s1f}, \texttt{STRONG_UNIQUE_SOLUTION}).
In contrast, Milner's ``unique solution of
equations'' theorem for rooted bisimilarity ($\rapprox$) (Theorem~\ref{t:Mil89s3}),
has more rigid constraints, as the \emph{equations} must be both guarded and
sequential (cf.~Theorem~\ref{thm:Mil89s3f}, \texttt{OBS_UNIQUE_SOLUTION}).

% next file: multivariate.tex
