%%%% -*- Mode: LaTeX; -*-

\newcommand\fun{{\to}}
\newcommand\prd{{\times}}
\newcommand{\ty}[1]{\textsl{#1}}
\newcommand\conj{\ \wedge\ }
\newcommand\disj{\ \vee\ }
\newcommand\imp{ \Rightarrow }
\newcommand\eqv{\ \equiv\ }
\newcommand\vbar{\mid}
\newcommand\turn{\ \vdash\ } % FIXME: "\ " resultgs in extra space
\newcommand\hilbert{\varepsilon}
\newcommand{\uquant}[1]{\forall #1.\ }
\newcommand{\equant}[1]{\exists #1.\ }
\newcommand{\hquant}[1]{\hilbert #1.\ }
\newcommand{\iquant}[1]{\exists ! #1.\ }
\newcommand{\lquant}[1]{\lambda #1.\ }
\newcommand{\ml}[1]{\mbox{{\def\_{\char'137}\texttt{#1}}}}
\newcommand{\con}[1]{\mathrm{#1}}

\newcommand\bool{\ty{bool}}
\newcommand\num{\ty{num}}
\newcommand\ind{\ty{ind}}
\newcommand\lst{\ty{list}}

\providecommand{\T}{\con{T}}
\renewcommand{\T}{\con{T}}
\newcommand\F{\con{F}}
\newcommand\OneOne{\con{One\_One}}
\newcommand\OntoSubset{\con{Onto\_Subset}}
\newcommand\Onto{\con{Onto}}
\newcommand\TyDef{\con{Type\_Definition}}

\section{The formalisation}
\label{s:for}

We highlight here a formalisation of CCS in the HOL theorem prover
(HOL4)~\cite{Melham:1993vl,slind2008brief}, \hl{with a focus towards the
theory (and formal proofs) of the unique solution of
equations/contractions theorems} mentioned in Section~\ref{s:eq} and
\ref{s:mcontr}.
%  The main purpose is to convince the readers that, there's no flaw
%  in the informal proofs. 
\hl{The univariate versions of these theorems, apart from minor fixes and
extensions in this paper, was already described in}
\cite{Tian:2017wrba}. The multivariate proofs and the related results
are new, and will be described in Section~\ref{sec:multivariate}.
All proof scripts are available as part of HOL's official
examples\footnote{\url{https://github.com/HOL-Theorem-Prover/HOL/tree/master/examples/CCS}}.
The work so far consists of about 24,000 lines (1MB) of code in total,
in which about 5,000 lines were derived from the early work of Monica
Nesi~\cite{Nesi:1992ve} on HOL88, with major modifications.

Higher Order Logic (HOL)~\cite{hollogic} traces
its roots back to LCF
\cite{gordon1979edinburgh,milner1972logic} by Robin Milner and others
since \hl{1972. It} is a variant of
Church’s Simple Theory of Types (STT)~\cite{church1940formulation},
plus a higher order version of Hilbert's choice operator $\varepsilon$,
Axiom of Infinity, and Rank-1 (prenex) polymorphism.
HOL4 has implemented the original HOL, 
while some other theorem provers in HOL family (e.g. Isabelle/HOL) have
certain extensions.
%  (they made the formal language more powerful,
% but they also bring the possibilities that the entire logic becomes
% inconsistent). 
Indeed the HOL has \hl{considerably} simpler logical
foundations than most other theorem provers. %, e.g. Coq. 
\hl{As a consequence, theories and proofs verified in HOL are relatively more
convincible to non-experts who are not familar with more advanced
dependent type theories.}

HOL4 is implemented in Standard ML, the same programming language who
plays three different roles:
\begin{enumerate}
\item the underlying implementation language for the core HOL engine;
\item \hl{the language in which proof tactics are implemented;}
\item \hl{the interface (outer) language of the HOL proof scripts and
  interactive shell.}
\end{enumerate}
Moreover, using the same language HOL4 users can write complex automatic
verification tools by calling HOL's theorem proving
facilities. The formal proofs of theorems in CCS theory
are mostly done by an \emph{interactive process} closely following
their informal proofs, with minimal automatic proof searching.

\subsection{Higher Order Logic}

HOL is a formal system of typed logical terms. The types are expressions that denote sets (in the
universe $\mathcal{U}$). HOL's type system is much simpler than those
based on dependent types and other type theories. There are four kinds of types in the HOL
logic, as illustrated in Fig. \ref{fig:hol-types} for its BNF
grammar. Noticed that, in HOL the standard atomic types \emph{bool} and \emph{ind}
 denote, respectively, the distinguished two-element set 2 and the
distinguished infinite set $I$.

\newlength{\ttX}
\settowidth{\ttX}{\tt X}
\newcommand{\tyvar}{\setlength{\unitlength}{\ttX}\begin{picture}(1,6)
\put(.5,0){\makebox(0,0)[b]{\footnotesize type variables}}
\put(0,1.5){\vector(0,1){4.5}}
\end{picture}}
\newcommand{\tyatom}{\setlength{\unitlength}{\ttX}\begin{picture}(1,6)
\put(.5,2.3){\makebox(0,0)[b]{\footnotesize atomic types}}
\put(.5,3.3){\vector(0,1){2.6}}
\end{picture}}
\newcommand{\funty}{\setlength{\unitlength}{\ttX}\begin{picture}(1,6)
\put(.5,1.5){\makebox(0,0)[b]{\footnotesize function types}}
\put(.5,0){\makebox(0,0)[b]{\footnotesize (domain $\sigma_1$, codomain $\sigma_2$)}}
\put(1,2.5){\vector(0,1){3.5}}
\end{picture}}
\newcommand{\cmpty}{\setlength{\unitlength}{\ttX}\begin{picture}(1,6)
\put(2,3.3){\makebox(0,0)[b]{\footnotesize compound types}}
\put(1.9,4.5){\vector(0,1){1.5}}
\end{picture}}

\begin{figure}[h]
\begin{equation*}
\sigma\quad ::=\quad {\mathord{\mathop{\alpha}\limits_{\tyvar}}}
        \quad\mid\quad{\mathord{\mathop{c}\limits_{\tyatom}}}
        \quad\mid\quad\underbrace{(\sigma_1, \ldots , \sigma_n){op}}_{\cmpty}
        \quad\mid\quad\underbrace{\sigma_1\fun\sigma_2}_{\funty}
\end{equation*}
   \caption{HOL's type grammar}
   \label{fig:hol-types}
\end{figure}

The terms of the HOL logic are expressions that denote elements of the
sets denoted by types. There're four kinds of terms in the HOL
logic. There can be described approximately by the BNF grammar in
Fig. \ref{fig:hol-terms}.

\settowidth{\ttX}{\tt X}
\newcommand{\var}{\setlength{\unitlength}{\ttX}\begin{picture}(1,6)
\put(.5,0){\makebox(0,0)[b]{\footnotesize variables}}
\put(0,1.5){\vector(0,1){4.5}}
\end{picture}}
\newcommand{\const}{\setlength{\unitlength}{\ttX}\begin{picture}(1,6)
\put(.5,2.3){\makebox(0,0)[b]{\footnotesize constants}}
\put(.5,3.5){\vector(0,1){2.4}}
\end{picture}}
\newcommand{\app}{\setlength{\unitlength}{\ttX}\begin{picture}(1,6)
\put(.5,1.5){\makebox(0,0)[b]{\footnotesize function applications}}
\put(.5,0){\makebox(0,0)[b]{\footnotesize (function $t$, argument $t'$)}}
\put(0.5,2.5){\vector(0,1){3.5}}
\end{picture}}
\newcommand{\abs}{\setlength{\unitlength}{\ttX}\begin{picture}(1,6)
\put(1,3.3){\makebox(0,0)[b]{\footnotesize $\lambda$-abstractions}}
\put(0.7,4.5){\vector(0,1){1.5}}
\end{picture}}

\begin{figure}[h]
\begin{equation*}
t \quad ::=\quad {\mathord{\mathop{x}\limits_{\var}}}
        \quad\mid\quad{\mathord{\mathop{c}\limits_{\const}}}
        \quad\mid\quad\underbrace{t\ t'}_{\app}
        \quad\mid\quad\underbrace{\lambda x .\ t}_{\abs}
\end{equation*}
   \caption{HOL's term grammar}
   \label{fig:hol-terms}
 \end{figure}

The deductive system of the HOL is specified by eight primitive
derivation rules: (c.f.~\cite{hollogic} for more details.)
\begin{enumerate}
\item Assumption introduction (\texttt{ASSUME});
\item Reflexivity (\texttt{REFL});
\item $\beta$-conversion (\texttt{BETA\_CONV});
\item Substitution (\texttt{SUBST});
\item Abstraction (\texttt{ABS});
\item Type instantiation (\texttt{INST\_TYPE});
\item Discharging an assumption (\texttt{DISCH});
\item Modus Ponens (\texttt{MP}).
\end{enumerate}
All proofs are eventually reduced to applications of the above rules,
which also give the semantics of two foundamental
logical connectives, the equality ($=$) and implication
($\Rightarrow$). The rest logical connectives and first-order
quantifiers, including the logical true (\HOLinline{\HOLConst{T}}) and false (\HOLinline{\HOLConst{F}}), are
further defined as $\lambda$-functions: (one rarely needs to care
things at this level, however.)
\begin{equation*}
\begin{array}{l}
\turn \T       =  ((\lquant{x_{\ty{bool}}}x) =
               (\lquant{x_{\ty{bool}}}x))    \\
\turn \forall  =  \lquant{P_{\alpha\fun\ty{bool}}} P =
                    (\lquant{x}\T ) \\
\turn \exists  =  \lquant{P_{\alpha\fun\ty{bool}}} P({\hilbert}\, P) \\
\turn \F  =  \uquant{b_{\ty{bool}}} b  \\
\turn \neg    =  \lquant{b} b \imp \F \\
\turn {\wedge}  =  \lquant{b_1\ b_2}\uquant{b} (b_1\imp (b_2 \imp b)) \imp b \\
\turn {\vee}  =  \lquant{b_1\ b_2}\uquant{b} (b_1 \imp b)\imp ((b_2 \imp b) \imp b) \\
\turn \OneOne  =  \lquant{f_{\alpha \fun\beta}}\uquant{x_1\ x_2}
                    (f\ x_1 = f\ x_2)  \imp (x_1 = x_2) \\
\turn \Onto  =  \lquant{f_{\alpha\fun\beta}}
                  \uquant{y}\equant{x} y = f\ x \\
\turn \TyDef  =  \lambda P_{\alpha\fun\ty{bool}}\
                  rep_{\beta\fun\alpha}.\;
                  \OneOne\ rep \ \wedge{}\  (\uquant{x}P x = (\equant{y} x = rep\ y))
\end{array}
\end{equation*}
It deserves to mention that, the last logical constant above,
$\TyDef$, can be used to define new HOL types as bijections of
subsets of existing types. (c.f.~\cite{Melham:1989dk} for more details.)
HOL's \texttt{Datatype} package~\cite{Melham:1991, holdesc} automates
this tedious process, and can be used to define needed types in CCS.

Finally, the whole HOL \emph{standard} theory is based on the
following four axioms, from which the almost\footnote{HOL is strictly
  weaker than ZFC (the Zermelo-Frankel set theory with the
Axiom of Choice), thus not all theorems valid in ZFC can be formalised
in HOL. (c.f.~\cite{hollogic} for more details.)} entire mathematics
can be formalised:
\begin{equation*}
\begin{array}{@{}l@{\qquad}l}
\mbox{\texttt{BOOL\_CASES\_AX}} &\vdash \uquant{b} (b = \T )\ \vee \ (b = \F )\\
\mbox{\texttt{ETA\_AX}} &
\vdash \uquant{f_{\alpha\fun\beta}}(\lquant{x}f\ x) = f\\
\mbox{\texttt{SELECT\_AX}} &
\vdash \uquant{P_{\alpha\fun\ty{bool}}\ x} P\ x \imp P({\hilbert}\ P)\\
\mbox{\texttt{INFINITY\_AX}}&
\vdash \equant{f_{\ind\fun \ind}} \OneOne \ f \conj \neg(\Onto \ f)\\
\end{array}
\end{equation*}

Notice that, usually the above four axioms are the only axioms allowed
in conventional formalisation projects in HOL: adding new axioms may
either break the logic consistencies or make the whole work less
convincible. The theorem prover (in general) guarantees that,
without adding new axioms, if both the definitions and the statements
of theorems are correct, then any proof which successfully completes must
be correct.

% next file: sos.htex
