%%%% -*- Mode: LaTeX -*-
%%
%% This is the draft of the 2nd part of EXPRESS/SOS 2018 paper, co-authored by
%% Prof. Davide Sangiorgi and Chun Tian.

\subsection{The CCS formalisation}
\label{ss:formalCCS}

The CCS formalisation starts with type definitions for action,
relabeling and then the  processes. We use the type ``\HOLinline{\ensuremath{\beta} \HOLTyOp{Label}}''
for all labels (i.e. visible actions), where the type variable $\beta$ corresponds
to $\mathscr{L}$ (the set of names for labels) \hl{mentioned} at the beginning
of Section~\ref{ss:ccs}. (Thus the
cardinality of ``\HOLinline{\ensuremath{\beta} \HOLTyOp{Label}}'' depend on \hl{its type variable:
\mbox{when $\beta$ is finite or countable, ``\HOLinline{\ensuremath{\beta} \HOLTyOp{Label}}'' is countable.}})
All labels are divided into input and output ones.
The type ``\HOLinline{\ensuremath{\beta} \HOLTyOp{Label}}'' is defined by HOL's \texttt{Datatype} package in the
following syntax:
\begin{lstlisting}
Datatype: Label = name 'b | coname 'b
End
\end{lstlisting}
Intuitively, ``\HOLinline{\ensuremath{\beta} \HOLTyOp{Label}}'' turns names in $\mathscr{L}$ into \emph{input and output labels}.
For instance, if the type $\beta$ is instantiated to \texttt{string}
(the type of all ASCII strings), then the HOL terms \HOLinline{\HOLConst{name} \HOLStringLit{a}} and
\HOLinline{\HOLConst{coname} \HOLStringLit{b}} denote the input \emph{label} $a$ and output
\emph{label} $\overline{b}$, respectively.
The type ``\HOLinline{\ensuremath{\beta} \HOLTyOp{Action}}'' is the
union of \hl{all visible actions (input and output labels) and the invisible action} $\tau$
(\texttt{tau}). For instace, the input \emph{action} $a$ and output
\emph{action} $\overline{b}$ of the type ``\HOLinline{\HOLTyOp{string} \HOLTyOp{Action}}''
are denoted by \HOLinline{\HOLConst{label} \ensuremath{(}\HOLConst{name} \HOLStringLit{a}\ensuremath{)}} and \HOLinline{\HOLConst{label} \ensuremath{(}\HOLConst{coname} \HOLStringLit{b}\ensuremath{)}}, respectively. \hl{On the
other hand,} \HOLinline{\HOLConst{label} \ensuremath{(}\HOLConst{name} \HOLNumLit{1}\ensuremath{)}} and \HOLinline{\HOLConst{label} \ensuremath{(}\HOLConst{coname} \HOLNumLit{2}\ensuremath{)}} could be actions of the type ``\HOLinline{\HOLTyOp{num} \HOLTyOp{Action}}'', where
\HOLinline{\HOLTyOp{num}} is the type of natural numbers in HOL.

The type ``\HOLinline{\ensuremath{(}\ensuremath{\alpha}, \ensuremath{\beta}\ensuremath{)} \HOLTyOp{CCS}}'', accounting for all CCS terms,
has two type variables $\alpha$ and $\beta$ corresponding
to the set of agent variables $\mathscr{X}$
and the set of names $\mathscr{L}$,
respectively. (Indeed the CCS syntax in Section~\ref{ss:ccs} is parametric
w.r.t. the choice of these two sets.)
%
The type ``\HOLinline{\ensuremath{(}\ensuremath{\alpha}, \ensuremath{\beta}\ensuremath{)} \HOLTyOp{CCS}}'' is defined inductively by the
\texttt{Datatype} package
(here ``\HOLinline{\ensuremath{\beta} \HOLTyOp{Relabeling}}'' is the type of all relabeling
  functions; we have also formalised relabeling, though it is not
  discussed below):
\begin{lstlisting}
Datatype: CCS = nil
              | var 'a
              | prefix ('b Action) CCS
              | sum CCS CCS
              | par CCS CCS
              | restr (('b Label) set) CCS
              | relab CCS ('b Relabeling)
              | rec 'a CCS
End
\end{lstlisting}
The above \hl{definition allows us to write} terms like \texttt{nil} and
\texttt{sum P Q} in HOL4. Their correspondences with conventional CCS notations
in the literature are given in Table~\ref{tab:ccsoperator}, where
most CCS operators have also more readable \hl{abbreviated} forms,
either for end users or for \TeX{} outputs. (All
  formal theorems and definitions in this paper
  are generated from HOL4. Also, by default, all theorems are fully
  specialised with outermost universal quantifiers \hl{removed.})

\begin{table}[h]
\begin{center}
\begin{tabular}{|c|c|c|c|c|}
\hline
\textbf{CCS concept} & \textbf{Notation} & \textbf{HOL term} &
                                             \textbf{HOL abbrev.} & \textbf{\TeX{} outputs} \\
\hline
nil & $\textbf{0}$ & \texttt{nil} & \texttt{nil} & \HOLinline{\HOLConst{\ensuremath{\mathbf{0}}}} \\
prefix & $\mu.P$ & \texttt{prefix u P} & \texttt{u..P} & \HOLinline{\HOLFreeVar{u}\HOLSymConst{\ensuremath{\ldotp}}\HOLFreeVar{P}} \\
summation & $P + Q$ & \texttt{sum P Q} & \texttt{P + Q} & \HOLinline{\HOLFreeVar{P} \HOLSymConst{\ensuremath{+}} \HOLFreeVar{Q}} \\
parallel composition & $P \,\mid\, Q$ & \texttt{par P Q} & \texttt{P || Q} & \HOLinline{\HOLFreeVar{P} \HOLSymConst{\ensuremath{\mid}} \HOLFreeVar{Q}} \\
restriction & $(\nu{}L)\;P$ & \texttt{restr L P} & \texttt{(nu L) P} & \HOLinline{\ensuremath{(\nu}\HOLFreeVar{L}\ensuremath{)} \HOLFreeVar{P}}  \\
recursion & $\recu A P$ & \texttt{rec A P} & \texttt{rec A P} & \HOLinline{\HOLConst{rec} \HOLFreeVar{A} \HOLFreeVar{P}}  \\
relabeling & $P\;[r\!f]$ & \texttt{relab P rf} &\texttt{relab P rf} & \HOLinline{\HOLConst{relab} \HOLFreeVar{P} \HOLFreeVar{rf}}  \\
constant/variable & $A$ & \texttt{var A} &\texttt{var A} & \HOLinline{\HOLConst{var} \HOLFreeVar{A}} \\
\hline
invisible action & $\tau$ & \texttt{tau} & \texttt{tau} & \HOLinline{\HOLSymConst{\ensuremath{\tau}}} \\
input action & $a$ & \texttt{label (name a)} &\texttt{In(a)} & \HOLinline{\HOLConst{\HOLTokenInputAct} \HOLFreeVar{a}} \\
output action & $\outC a$ & \texttt{label (coname a)} & \texttt{Out(a)} & \HOLinline{\HOLConst{\HOLTokenOutputAct} \HOLFreeVar{a}} \\
\hline
variable substitution & $E\{E'/X\}$ & \texttt{CCS_Subst E E' X} &
                                           \texttt{[E'/X]  E} & \HOLinline{\ensuremath{[}\ensuremath{\HOLFreeVar{E}\sp{\prime}}\ensuremath{/}\HOLFreeVar{X}\ensuremath{]} \HOLFreeVar{E}} \\
transition & $P\overset{\mu}{\longrightarrow}Q$
                       & \texttt{TRANS P u Q} & \texttt{P ---u-> Q} & \HOLinline{\HOLFreeVar{P} \HOLTokenTransBegin\HOLFreeVar{u}\HOLTokenTransEnd \HOLFreeVar{Q}} \\
weak transition & $P\overset{\mu}{\Longrightarrow}Q$
                       & \texttt{WEAK\_TRANS P u Q} & \texttt{P ==u=> Q} & \HOLinline{\HOLFreeVar{P} \HOLTokenWeakTransBegin\HOLFreeVar{u}\HOLTokenWeakTransEnd \HOLFreeVar{Q}} \\
$\epsilon$--transition & $P\overset{\epsilon}{\Longrightarrow}Q$
                       & \texttt{EPS P Q} & \texttt{EPS P Q} & \HOLinline{\HOLFreeVar{P} \HOLSymConst{\HOLTokenEPS} \HOLFreeVar{Q}} \\
\hline
\end{tabular}
\end{center}
   \caption{Syntax of some CCS concepts in HOL}
   \label{tab:ccsoperator}
\end{table}

The transition semantics of CCS processes strictly follows the SOS rules given
in Fig.~\ref{f:LTSCCS}. However, they are not axioms but
consequences of an \emph{inductive relation} definition of
\HOLinline{\HOLConst{TRANS}} by the \texttt{HOL_reln} function of HOL4
(see~\citep[p.~219]{holdesc} for more details).
A successful invocation of the definitional principle returns three
important theorems (\emph{rules}, \emph{ind} and \emph{cases}):
\begin{itemize}
  \item \emph{rules} is a conjunction of implications that will be the
    same as the input term. In fact, the following formal versions of SOS rules are
    extracted from the \emph{rules} theorem:
\begin{alltt}
\HOLTokenTurnstile{} \HOLFreeVar{u}\HOLSymConst{\ensuremath{\ldotp}}\HOLFreeVar{E} \HOLTokenTransBegin\HOLFreeVar{u}\HOLTokenTransEnd \HOLFreeVar{E}\hfill\texttt{[PREFIX]}
\HOLTokenTurnstile{} \HOLFreeVar{E} \HOLTokenTransBegin\HOLFreeVar{u}\HOLTokenTransEnd \ensuremath{\HOLFreeVar{E}\sb{\mathrm{1}}} \HOLSymConst{\HOLTokenImp{}} \HOLFreeVar{E} \HOLSymConst{\ensuremath{+}} \ensuremath{\HOLFreeVar{E}\sp{\prime}} \HOLTokenTransBegin\HOLFreeVar{u}\HOLTokenTransEnd \ensuremath{\HOLFreeVar{E}\sb{\mathrm{1}}}\hfill\texttt{[SUM1]}
\HOLTokenTurnstile{} \HOLFreeVar{E} \HOLTokenTransBegin\HOLFreeVar{u}\HOLTokenTransEnd \ensuremath{\HOLFreeVar{E}\sb{\mathrm{1}}} \HOLSymConst{\HOLTokenImp{}} \ensuremath{\HOLFreeVar{E}\sp{\prime}} \HOLSymConst{\ensuremath{+}} \HOLFreeVar{E} \HOLTokenTransBegin\HOLFreeVar{u}\HOLTokenTransEnd \ensuremath{\HOLFreeVar{E}\sb{\mathrm{1}}}\hfill\texttt{[SUM2]}
\HOLTokenTurnstile{} \HOLFreeVar{E} \HOLTokenTransBegin\HOLFreeVar{u}\HOLTokenTransEnd \ensuremath{\HOLFreeVar{E}\sb{\mathrm{1}}} \HOLSymConst{\HOLTokenImp{}} \HOLFreeVar{E} \HOLSymConst{\ensuremath{\mid}} \ensuremath{\HOLFreeVar{E}\sp{\prime}} \HOLTokenTransBegin\HOLFreeVar{u}\HOLTokenTransEnd \ensuremath{\HOLFreeVar{E}\sb{\mathrm{1}}} \HOLSymConst{\ensuremath{\mid}} \ensuremath{\HOLFreeVar{E}\sp{\prime}}\hfill\texttt{[PAR1]}
\HOLTokenTurnstile{} \HOLFreeVar{E} \HOLTokenTransBegin\HOLFreeVar{u}\HOLTokenTransEnd \ensuremath{\HOLFreeVar{E}\sb{\mathrm{1}}} \HOLSymConst{\HOLTokenImp{}} \ensuremath{\HOLFreeVar{E}\sp{\prime}} \HOLSymConst{\ensuremath{\mid}} \HOLFreeVar{E} \HOLTokenTransBegin\HOLFreeVar{u}\HOLTokenTransEnd \ensuremath{\HOLFreeVar{E}\sp{\prime}} \HOLSymConst{\ensuremath{\mid}} \ensuremath{\HOLFreeVar{E}\sb{\mathrm{1}}}\hfill\texttt{[PAR2]}
\end{alltt}
\begin{equation*}
\infer[\HOLRuleName{PAR3}]{\HOLinline{\HOLFreeVar{E} \HOLSymConst{\ensuremath{\mid}} \ensuremath{\HOLFreeVar{E}\sp{\prime}} \HOLTokenTransBegin\HOLSymConst{\ensuremath{\tau}}\HOLTokenTransEnd \ensuremath{\HOLFreeVar{E}\sb{\mathrm{1}}} \HOLSymConst{\ensuremath{\mid}} \ensuremath{\HOLFreeVar{E}\sb{\mathrm{2}}}}}{\HOLinline{\HOLFreeVar{E} \HOLTokenTransBegin\HOLConst{label} \HOLFreeVar{l}\HOLTokenTransEnd \ensuremath{\HOLFreeVar{E}\sb{\mathrm{1}}}}&\HOLinline{\ensuremath{\HOLFreeVar{E}\sp{\prime}} \HOLTokenTransBegin\HOLConst{label} \ensuremath{(}\HOLConst{COMPL} \HOLFreeVar{l}\ensuremath{)}\HOLTokenTransEnd \ensuremath{\HOLFreeVar{E}\sb{\mathrm{2}}}}}
\end{equation*}
\begin{equation*}
\infer[\HOLRuleName{RESTR}]{\HOLinline{\ensuremath{(\nu}\HOLFreeVar{L}\ensuremath{)} \HOLFreeVar{E} \HOLTokenTransBegin\HOLFreeVar{u}\HOLTokenTransEnd \ensuremath{(\nu}\HOLFreeVar{L}\ensuremath{)} \ensuremath{\HOLFreeVar{E}\sp{\prime}}}}{\HOLinline{\HOLFreeVar{E} \HOLTokenTransBegin\HOLFreeVar{u}\HOLTokenTransEnd \ensuremath{\HOLFreeVar{E}\sp{\prime}}}&\HOLinline{\HOLFreeVar{u} \HOLSymConst{\ensuremath{=}} \HOLSymConst{\ensuremath{\tau}} \HOLSymConst{\HOLTokenDisj{}} \HOLFreeVar{u} \HOLSymConst{\ensuremath{=}} \HOLConst{label} \HOLFreeVar{l} \HOLSymConst{\HOLTokenConj{}} \HOLFreeVar{l} \HOLSymConst{\HOLTokenNotIn{}} \HOLFreeVar{L} \HOLSymConst{\HOLTokenConj{}} \HOLConst{COMPL} \HOLFreeVar{l} \HOLSymConst{\HOLTokenNotIn{}} \HOLFreeVar{L}}}
\end{equation*}
\begin{equation*}
\infer[\HOLRuleName{RELABELING}]{\HOLinline{\HOLConst{relab} \HOLFreeVar{E} \HOLFreeVar{rf} \HOLTokenTransBegin\HOLConst{relabel} \HOLFreeVar{rf} \HOLFreeVar{u}\HOLTokenTransEnd \HOLConst{relab} \ensuremath{\HOLFreeVar{E}\sp{\prime}} \HOLFreeVar{rf}}}{\HOLinline{\HOLFreeVar{E} \HOLTokenTransBegin\HOLFreeVar{u}\HOLTokenTransEnd \ensuremath{\HOLFreeVar{E}\sp{\prime}}}}
\end{equation*}
\begin{equation*}
\infer[\HOLRuleName{REC}]{\HOLinline{\HOLConst{rec} \HOLFreeVar{X} \HOLFreeVar{E} \HOLTokenTransBegin\HOLFreeVar{u}\HOLTokenTransEnd \ensuremath{\HOLFreeVar{E}\sb{\mathrm{1}}}}}{\HOLinline{\ensuremath{[}\HOLConst{rec} \HOLFreeVar{X} \HOLFreeVar{E}\ensuremath{/}\HOLFreeVar{X}\ensuremath{]} \HOLFreeVar{E} \HOLTokenTransBegin\HOLFreeVar{u}\HOLTokenTransEnd \ensuremath{\HOLFreeVar{E}\sb{\mathrm{1}}}}}
\end{equation*}
\item \emph{ind} is the induction principle for the relation
  (see Section~\ref{sec:multivariate} for the exact form and its application).
\item \emph{cases} is the so-called `cases' or `inversion' theorem for
  the relation, and is used to decompose an element in the relation into the possible ways of
  obtaining it by the rules:
\begin{alltt}
\HOLTokenTurnstile{} \HOLSymConst{\HOLTokenForall{}}\ensuremath{\HOLBoundVar{a}\sb{\mathrm{0}}} \ensuremath{\HOLBoundVar{a}\sb{\mathrm{1}}} \ensuremath{\HOLBoundVar{a}\sb{\mathrm{2}}}.
       \ensuremath{\HOLBoundVar{a}\sb{\mathrm{0}}} \HOLTokenTransBegin\ensuremath{\HOLBoundVar{a}\sb{\mathrm{1}}}\HOLTokenTransEnd \ensuremath{\HOLBoundVar{a}\sb{\mathrm{2}}} \HOLSymConst{\HOLTokenEquiv{}}
       \ensuremath{\HOLBoundVar{a}\sb{\mathrm{0}}} \HOLSymConst{\ensuremath{=}} \ensuremath{\HOLBoundVar{a}\sb{\mathrm{1}}}\HOLSymConst{\ensuremath{\ldotp}}\ensuremath{\HOLBoundVar{a}\sb{\mathrm{2}}} \HOLSymConst{\HOLTokenDisj{}} \ensuremath{(}\HOLSymConst{\HOLTokenExists{}}\HOLBoundVar{E} \ensuremath{\HOLBoundVar{E}\sp{\prime}}. \ensuremath{\HOLBoundVar{a}\sb{\mathrm{0}}} \HOLSymConst{\ensuremath{=}} \HOLBoundVar{E} \HOLSymConst{\ensuremath{+}} \ensuremath{\HOLBoundVar{E}\sp{\prime}} \HOLSymConst{\HOLTokenConj{}} \HOLBoundVar{E} \HOLTokenTransBegin\ensuremath{\HOLBoundVar{a}\sb{\mathrm{1}}}\HOLTokenTransEnd \ensuremath{\HOLBoundVar{a}\sb{\mathrm{2}}}\ensuremath{)} \HOLSymConst{\HOLTokenDisj{}}
       \ensuremath{(}\HOLSymConst{\HOLTokenExists{}}\HOLBoundVar{E} \ensuremath{\HOLBoundVar{E}\sp{\prime}}. \ensuremath{\HOLBoundVar{a}\sb{\mathrm{0}}} \HOLSymConst{\ensuremath{=}} \ensuremath{\HOLBoundVar{E}\sp{\prime}} \HOLSymConst{\ensuremath{+}} \HOLBoundVar{E} \HOLSymConst{\HOLTokenConj{}} \HOLBoundVar{E} \HOLTokenTransBegin\ensuremath{\HOLBoundVar{a}\sb{\mathrm{1}}}\HOLTokenTransEnd \ensuremath{\HOLBoundVar{a}\sb{\mathrm{2}}}\ensuremath{)} \HOLSymConst{\HOLTokenDisj{}}
       \ensuremath{(}\HOLSymConst{\HOLTokenExists{}}\HOLBoundVar{E} \ensuremath{\HOLBoundVar{E}\sb{\mathrm{1}}} \ensuremath{\HOLBoundVar{E}\sp{\prime}}. \ensuremath{\HOLBoundVar{a}\sb{\mathrm{0}}} \HOLSymConst{\ensuremath{=}} \HOLBoundVar{E} \HOLSymConst{\ensuremath{\mid}} \ensuremath{\HOLBoundVar{E}\sp{\prime}} \HOLSymConst{\HOLTokenConj{}} \ensuremath{\HOLBoundVar{a}\sb{\mathrm{2}}} \HOLSymConst{\ensuremath{=}} \ensuremath{\HOLBoundVar{E}\sb{\mathrm{1}}} \HOLSymConst{\ensuremath{\mid}} \ensuremath{\HOLBoundVar{E}\sp{\prime}} \HOLSymConst{\HOLTokenConj{}} \HOLBoundVar{E} \HOLTokenTransBegin\ensuremath{\HOLBoundVar{a}\sb{\mathrm{1}}}\HOLTokenTransEnd \ensuremath{\HOLBoundVar{E}\sb{\mathrm{1}}}\ensuremath{)} \HOLSymConst{\HOLTokenDisj{}}
       \ensuremath{(}\HOLSymConst{\HOLTokenExists{}}\HOLBoundVar{E} \ensuremath{\HOLBoundVar{E}\sb{\mathrm{1}}} \ensuremath{\HOLBoundVar{E}\sp{\prime}}. \ensuremath{\HOLBoundVar{a}\sb{\mathrm{0}}} \HOLSymConst{\ensuremath{=}} \ensuremath{\HOLBoundVar{E}\sp{\prime}} \HOLSymConst{\ensuremath{\mid}} \HOLBoundVar{E} \HOLSymConst{\HOLTokenConj{}} \ensuremath{\HOLBoundVar{a}\sb{\mathrm{2}}} \HOLSymConst{\ensuremath{=}} \ensuremath{\HOLBoundVar{E}\sp{\prime}} \HOLSymConst{\ensuremath{\mid}} \ensuremath{\HOLBoundVar{E}\sb{\mathrm{1}}} \HOLSymConst{\HOLTokenConj{}} \HOLBoundVar{E} \HOLTokenTransBegin\ensuremath{\HOLBoundVar{a}\sb{\mathrm{1}}}\HOLTokenTransEnd \ensuremath{\HOLBoundVar{E}\sb{\mathrm{1}}}\ensuremath{)} \HOLSymConst{\HOLTokenDisj{}}
       \ensuremath{(}\HOLSymConst{\HOLTokenExists{}}\HOLBoundVar{E} \HOLBoundVar{l} \ensuremath{\HOLBoundVar{E}\sb{\mathrm{1}}} \ensuremath{\HOLBoundVar{E}\sp{\prime}} \ensuremath{\HOLBoundVar{E}\sb{\mathrm{2}}}.
            \ensuremath{\HOLBoundVar{a}\sb{\mathrm{0}}} \HOLSymConst{\ensuremath{=}} \HOLBoundVar{E} \HOLSymConst{\ensuremath{\mid}} \ensuremath{\HOLBoundVar{E}\sp{\prime}} \HOLSymConst{\HOLTokenConj{}} \ensuremath{\HOLBoundVar{a}\sb{\mathrm{1}}} \HOLSymConst{\ensuremath{=}} \HOLSymConst{\ensuremath{\tau}} \HOLSymConst{\HOLTokenConj{}} \ensuremath{\HOLBoundVar{a}\sb{\mathrm{2}}} \HOLSymConst{\ensuremath{=}} \ensuremath{\HOLBoundVar{E}\sb{\mathrm{1}}} \HOLSymConst{\ensuremath{\mid}} \ensuremath{\HOLBoundVar{E}\sb{\mathrm{2}}} \HOLSymConst{\HOLTokenConj{}} \HOLBoundVar{E} \HOLTokenTransBegin\HOLConst{label} \HOLBoundVar{l}\HOLTokenTransEnd \ensuremath{\HOLBoundVar{E}\sb{\mathrm{1}}} \HOLSymConst{\HOLTokenConj{}}
            \ensuremath{\HOLBoundVar{E}\sp{\prime}} \HOLTokenTransBegin\HOLConst{label} \ensuremath{(}\HOLConst{COMPL} \HOLBoundVar{l}\ensuremath{)}\HOLTokenTransEnd \ensuremath{\HOLBoundVar{E}\sb{\mathrm{2}}}\ensuremath{)} \HOLSymConst{\HOLTokenDisj{}}
       \ensuremath{(}\HOLSymConst{\HOLTokenExists{}}\HOLBoundVar{E} \ensuremath{\HOLBoundVar{E}\sp{\prime}} \HOLBoundVar{l} \HOLBoundVar{L}.
            \ensuremath{\HOLBoundVar{a}\sb{\mathrm{0}}} \HOLSymConst{\ensuremath{=}} \ensuremath{(\nu}\HOLBoundVar{L}\ensuremath{)} \HOLBoundVar{E} \HOLSymConst{\HOLTokenConj{}} \ensuremath{\HOLBoundVar{a}\sb{\mathrm{2}}} \HOLSymConst{\ensuremath{=}} \ensuremath{(\nu}\HOLBoundVar{L}\ensuremath{)} \ensuremath{\HOLBoundVar{E}\sp{\prime}} \HOLSymConst{\HOLTokenConj{}} \HOLBoundVar{E} \HOLTokenTransBegin\ensuremath{\HOLBoundVar{a}\sb{\mathrm{1}}}\HOLTokenTransEnd \ensuremath{\HOLBoundVar{E}\sp{\prime}} \HOLSymConst{\HOLTokenConj{}}
            \ensuremath{(}\ensuremath{\HOLBoundVar{a}\sb{\mathrm{1}}} \HOLSymConst{\ensuremath{=}} \HOLSymConst{\ensuremath{\tau}} \HOLSymConst{\HOLTokenDisj{}} \ensuremath{\HOLBoundVar{a}\sb{\mathrm{1}}} \HOLSymConst{\ensuremath{=}} \HOLConst{label} \HOLBoundVar{l} \HOLSymConst{\HOLTokenConj{}} \HOLBoundVar{l} \HOLSymConst{\HOLTokenNotIn{}} \HOLBoundVar{L} \HOLSymConst{\HOLTokenConj{}} \HOLConst{COMPL} \HOLBoundVar{l} \HOLSymConst{\HOLTokenNotIn{}} \HOLBoundVar{L}\ensuremath{)}\ensuremath{)} \HOLSymConst{\HOLTokenDisj{}}
       \ensuremath{(}\HOLSymConst{\HOLTokenExists{}}\HOLBoundVar{E} \HOLBoundVar{u} \ensuremath{\HOLBoundVar{E}\sp{\prime}} \HOLBoundVar{rf}.
            \ensuremath{\HOLBoundVar{a}\sb{\mathrm{0}}} \HOLSymConst{\ensuremath{=}} \HOLConst{relab} \HOLBoundVar{E} \HOLBoundVar{rf} \HOLSymConst{\HOLTokenConj{}} \ensuremath{\HOLBoundVar{a}\sb{\mathrm{1}}} \HOLSymConst{\ensuremath{=}} \HOLConst{relabel} \HOLBoundVar{rf} \HOLBoundVar{u} \HOLSymConst{\HOLTokenConj{}} \ensuremath{\HOLBoundVar{a}\sb{\mathrm{2}}} \HOLSymConst{\ensuremath{=}} \HOLConst{relab} \ensuremath{\HOLBoundVar{E}\sp{\prime}} \HOLBoundVar{rf} \HOLSymConst{\HOLTokenConj{}}
            \HOLBoundVar{E} \HOLTokenTransBegin\HOLBoundVar{u}\HOLTokenTransEnd \ensuremath{\HOLBoundVar{E}\sp{\prime}}\ensuremath{)} \HOLSymConst{\HOLTokenDisj{}} \HOLSymConst{\HOLTokenExists{}}\HOLBoundVar{E} \HOLBoundVar{X}. \ensuremath{\HOLBoundVar{a}\sb{\mathrm{0}}} \HOLSymConst{\ensuremath{=}} \HOLConst{rec} \HOLBoundVar{X} \HOLBoundVar{E} \HOLSymConst{\HOLTokenConj{}} \ensuremath{[}\HOLConst{rec} \HOLBoundVar{X} \HOLBoundVar{E}\ensuremath{/}\HOLBoundVar{X}\ensuremath{]} \HOLBoundVar{E} \HOLTokenTransBegin\ensuremath{\HOLBoundVar{a}\sb{\mathrm{1}}}\HOLTokenTransEnd \ensuremath{\HOLBoundVar{a}\sb{\mathrm{2}}}    
\end{alltt}  
For instance, \hl{the following proposition
can only be proven by using the above \emph{cases} theorem}:
\begin{alltt}
\HOLTokenTurnstile{} \HOLFreeVar{E} \HOLSymConst{\ensuremath{+}} \ensuremath{\HOLFreeVar{E}\sp{\prime}} \HOLTokenTransBegin\HOLFreeVar{u}\HOLTokenTransEnd \ensuremath{\HOLFreeVar{E}\sp{\prime\prime}} \HOLSymConst{\HOLTokenEquiv{}} \HOLFreeVar{E} \HOLTokenTransBegin\HOLFreeVar{u}\HOLTokenTransEnd \ensuremath{\HOLFreeVar{E}\sp{\prime\prime}} \HOLSymConst{\HOLTokenDisj{}} \ensuremath{\HOLFreeVar{E}\sp{\prime}} \HOLTokenTransBegin\HOLFreeVar{u}\HOLTokenTransEnd \ensuremath{\HOLFreeVar{E}\sp{\prime\prime}}\hfill{[TRANS_SUM_EQ]}
\end{alltt}
\end{itemize}

\hl{In particular,} the SOS rule \texttt{REC} (Recursion)
 says that, if we substitute all occurrences of the variable $A$ in $P$ to
$(\recu A P)$ and the resulting process has a transition to $P'$
with an action $u$, then $(\recu A P)$ has the same
transition. Here ``\HOLinline{\ensuremath{[}\HOLConst{rec}\;\;\HOLFreeVar{X}\;\;\HOLFreeVar{E}\ensuremath{/}\HOLFreeVar{X}\ensuremath{]}\;\;\HOLFreeVar{E}}'' is an abbreviation
for ``\HOLinline{\HOLConst{CCS_Subst}\;\;\HOLFreeVar{E}\;\;\ensuremath{(}\HOLConst{rec}\;\;\HOLFreeVar{X}\;\;\HOLFreeVar{E}\ensuremath{)}\;\;\HOLFreeVar{X}}'', where \HOLinline{\HOLConst{CCS_Subst}} is a
recursive function substituting all \hl{occurrences} of a free variable with a CCS term.
For most CCS operators \HOLinline{\HOLConst{CCS_Subst}} just recursively goes into a deeper
level without changing anything, e.g.
\begin{alltt}
\HOLTokenTurnstile{} \ensuremath{[}\ensuremath{\HOLFreeVar{E}\sp{\prime}}\ensuremath{/}\HOLFreeVar{X}\ensuremath{]} \ensuremath{(}\ensuremath{\HOLFreeVar{E}\sb{\mathrm{1}}} \HOLSymConst{\ensuremath{+}} \ensuremath{\HOLFreeVar{E}\sb{\mathrm{2}}}\ensuremath{)} \HOLSymConst{\ensuremath{=}} \ensuremath{[}\ensuremath{\HOLFreeVar{E}\sp{\prime}}\ensuremath{/}\HOLFreeVar{X}\ensuremath{]} \ensuremath{\HOLFreeVar{E}\sb{\mathrm{1}}} \HOLSymConst{\ensuremath{+}} \ensuremath{[}\ensuremath{\HOLFreeVar{E}\sp{\prime}}\ensuremath{/}\HOLFreeVar{X}\ensuremath{]} \ensuremath{\HOLFreeVar{E}\sb{\mathrm{2}}}\hfill{[CCS_Subst_sum]}
\end{alltt}
The only two insteresting cases are those for agent variables and recursion:
\begin{alltt}
\HOLTokenTurnstile{} \ensuremath{[}\ensuremath{\HOLFreeVar{E}\sp{\prime}}\ensuremath{/}\HOLFreeVar{X}\ensuremath{]} \ensuremath{(}\HOLConst{var} \HOLFreeVar{Y}\ensuremath{)} \HOLSymConst{\ensuremath{=}} \HOLKeyword{if} \HOLFreeVar{Y} \HOLSymConst{\ensuremath{=}} \HOLFreeVar{X} \HOLKeyword{then} \ensuremath{\HOLFreeVar{E}\sp{\prime}} \HOLKeyword{else} \HOLConst{var} \HOLFreeVar{Y}\hfill{[CCS_Subst_var]}
\HOLTokenTurnstile{} \ensuremath{[}\ensuremath{\HOLFreeVar{E}\sp{\prime}}\ensuremath{/}\HOLFreeVar{X}\ensuremath{]} \ensuremath{(}\HOLConst{rec} \HOLFreeVar{Y} \HOLFreeVar{E}\ensuremath{)} \HOLSymConst{\ensuremath{=}} \HOLKeyword{if} \HOLFreeVar{Y} \HOLSymConst{\ensuremath{=}} \HOLFreeVar{X} \HOLKeyword{then} \HOLConst{rec} \HOLFreeVar{Y} \HOLFreeVar{E} \HOLKeyword{else} \HOLConst{rec} \HOLFreeVar{Y} \ensuremath{(}\ensuremath{[}\ensuremath{\HOLFreeVar{E}\sp{\prime}}\ensuremath{/}\HOLFreeVar{X}\ensuremath{]} \HOLFreeVar{E}\ensuremath{)}\hfill{[CCS_Subst_rec]}
\end{alltt}
Notice that, variable substitutions only affect free
variables. For instance, the variable $Y$ in ``\HOLinline{\HOLConst{rec}\!\;\HOLFreeVar{Y}\!\;\HOLFreeVar{E}}'' is
bound and therefore the substitution ignores it.

A useful facility exploiting the interplay
between HOL4 and Standard ML (\hl{which} follows an idea of Nesi~\cite{Nesi:1992ve})
 is a recursive ML function that takes a CCS process and returns a theorem indicating all
  direct transitions of the process. (If the input process is
 infinitely branching, the function will not terminate, however.)
For instance, we know that the process $(a.\nil | \bar{a}.\nil)$ has three
\hl{immediate derivatives given by the following transitions}:
$(a.\nil | \bar{a}.\nil) \overset{a}{\longrightarrow}
(\nil | \bar{a}.\nil)$, $(a.\nil | \bar{a}.\nil)
\overset{\bar{a}}{\longrightarrow} (a.\nil | \nil)$ and $(a.\nil | \bar{a}.\nil)
\overset{\tau}{\longrightarrow} (\nil | \nil)$.
To completely describe all possible transitions of the process, the
following two facts must be proved: (1) there are indeed the three
  transitions mentioned above; (2) there is no other
  transition. For large CCS processes it is surprisingly tedious to manually derive
all the possible transitions and prove the non-existence of other transitions.
This shows the usefulness of appealing to an ML function
\texttt{CCS\_TRANS\_CONV} that is designed to automate the whole process.
For instance, taking the input $(a.\nil | \bar{a}.\nil)$ the function returns the
following theorem which describes all its one-step transitions:
\begin{alltt}
\HOLTokenTurnstile{} \HOLConst{\HOLTokenInputAct} \HOLStringLit{a}\HOLSymConst{\ensuremath{\ldotp}}\HOLConst{\ensuremath{\mathbf{0}}} \HOLSymConst{\ensuremath{\mid}} \HOLConst{\HOLTokenOutputAct} \HOLStringLit{a}\HOLSymConst{\ensuremath{\ldotp}}\HOLConst{\ensuremath{\mathbf{0}}} \HOLTokenTransBegin\HOLFreeVar{u}\HOLTokenTransEnd \HOLFreeVar{E} \HOLSymConst{\HOLTokenEquiv{}}
   \ensuremath{(}\HOLFreeVar{u} \HOLSymConst{\ensuremath{=}} \HOLConst{\HOLTokenInputAct} \HOLStringLit{a} \HOLSymConst{\HOLTokenConj{}} \HOLFreeVar{E} \HOLSymConst{\ensuremath{=}} \HOLConst{\ensuremath{\mathbf{0}}} \HOLSymConst{\ensuremath{\mid}} \HOLConst{\HOLTokenOutputAct} \HOLStringLit{a}\HOLSymConst{\ensuremath{\ldotp}}\HOLConst{\ensuremath{\mathbf{0}}} \HOLSymConst{\HOLTokenDisj{}} \HOLFreeVar{u} \HOLSymConst{\ensuremath{=}} \HOLConst{\HOLTokenOutputAct} \HOLStringLit{a} \HOLSymConst{\HOLTokenConj{}} \HOLFreeVar{E} \HOLSymConst{\ensuremath{=}} \HOLConst{\HOLTokenInputAct} \HOLStringLit{a}\HOLSymConst{\ensuremath{\ldotp}}\HOLConst{\ensuremath{\mathbf{0}}} \HOLSymConst{\ensuremath{\mid}} \HOLConst{\ensuremath{\mathbf{0}}}\ensuremath{)} \HOLSymConst{\HOLTokenDisj{}}
   \HOLFreeVar{u} \HOLSymConst{\ensuremath{=}} \HOLSymConst{\ensuremath{\tau}} \HOLSymConst{\HOLTokenConj{}} \HOLFreeVar{E} \HOLSymConst{\ensuremath{=}} \HOLConst{\ensuremath{\mathbf{0}}} \HOLSymConst{\ensuremath{\mid}} \HOLConst{\ensuremath{\mathbf{0}}}
\end{alltt}

% next file: bisim.htex
