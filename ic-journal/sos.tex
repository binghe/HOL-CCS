%%%% -*- Mode: LaTeX -*-
%%
%% This is the draft of the 2nd part of EXPRESS/SOS 2018 paper, co-authored by
%% Prof. Davide Sangiorgi and Chun Tian.

\subsection{CCS: the formal theory}
\label{ss:formalCCS}

The CCS formalisation starts with type definitions for action,
relabeling and the CCS term itself. We use the type ``\HOLinline{\ensuremath{\beta} \HOLTyOp{Label}}'' (\texttt{'b} or
$\beta$ is a \emph{type variable}) for all visible actions. The
cardinality and the exact elements depend on the type variable $\beta$
supplied by the end user.
All actions are divided into input and $\overline{\text{output}}$ ones.
This type is defined by HOL's \texttt{Datatype} package in the
following syntax:
\begin{lstlisting}
Datatype: Label = name 'b | coname 'b
End
\end{lstlisting}
For instance, if the type $\beta$ is instantiated by the type
\texttt{string}, then HOL terms \HOLinline{\HOLConst{name} \HOLStringLit{a}} and
\HOLinline{\HOLConst{coname} \HOLStringLit{b}} denote the input action $a$ and output
action \hl{$\overline{b}$}, respectively.
The type ``\HOLinline{\ensuremath{\beta} \HOLTyOp{Action}}'' is the
union of all visible actions, plus the invisible action $\tau$
(\texttt{tau}). For instace, the input action $a$ and output
action \hl{$\overline{b}$} of type \hl{\mbox{``\HOLinline{\HOLTyOp{string} \HOLTyOp{Action}}''}}
are denoted by \HOLinline{\HOLConst{label} \ensuremath{(}\HOLConst{name} \HOLStringLit{a}\ensuremath{)}} and \HOLinline{\HOLConst{label} \ensuremath{(}\HOLConst{coname} \HOLStringLit{b}\ensuremath{)}}, respectively.

The type ``\HOLinline{\ensuremath{(}\ensuremath{\alpha}, \ensuremath{\beta}\ensuremath{)} \HOLTyOp{CCS}}'', accounting for the CCS
expressions, where \texttt{'a} (or $\alpha$) is the type variable\footnote{The order of type variables $\alpha$ and $\beta$
    is arbitrary. Our choice is aligned with  the literature.
For instance, $\mathrm{CCS}(h,k)$ is the CCS subcalculus that can use at most $h$ constants
and $k$ actions~\cite{gorrieri2017ccs}. To formalize theorems on
such $\mathrm{CCS}(h,k)$, the required CCS type can be retrieved by instantiating the type
variables $\alpha$ and $\beta$ in ``\HOLinline{\ensuremath{(}\ensuremath{\alpha}, \ensuremath{\beta}\ensuremath{)} \HOLTyOp{CCS}}'' with types
having the corresponding cardinalities $h$ and $k$.} for
agent variables, is then defined inductively: (\HOLinline{\ensuremath{\beta} \HOLTyOp{Relabeling}} is the type of all relabeling functions. They (and
  related operations) are not discussed in this paper.)
\begin{lstlisting}
Datatype: CCS = nil
              | var 'a
              | prefix ('b Action) CCS
              | sum CCS CCS
              | par CCS CCS
              | restr (('b Label) set) CCS
              | relab CCS ('b Relabeling)
              | rec 'a CCS
End
\end{lstlisting}
The above definitions allow us to use terms like \texttt{nil} and
\texttt{sum P Q} in HOL. Their correspondences with conventional CCS notations
in the literature, are given in Table~\ref{tab:ccsoperator}, where
most CCS operators have also the more readable abbreviation forms,
either for end user or for \TeX{} outputs. (All
  formal theorems and definitions in this paper
  are generated from HOL4. Also, by default, all theorems are fully
  specialised, removing outermost universal quantifiers.)

\begin{table}[h]
\begin{center}
\begin{tabular}{|c|c|c|c|c|}
\hline
\textbf{CCS concept} & \textbf{Notation} & \textbf{HOL term} &
                                             \textbf{HOL abbrev.} & \textbf{\TeX{} outputs} \\
\hline
nil & $\textbf{0}$ & \texttt{nil} & \texttt{nil} & \HOLinline{\HOLConst{\ensuremath{\mathbf{0}}}} \\
prefix & $\mu.P$ & \texttt{prefix u P} & \texttt{u..P} & \HOLinline{\HOLFreeVar{u}\HOLSymConst{\ensuremath{\ldotp}}\HOLFreeVar{P}} \\
summation & $P + Q$ & \texttt{sum P Q} & \texttt{P + Q} & \HOLinline{\HOLFreeVar{P} \HOLSymConst{\ensuremath{+}} \HOLFreeVar{Q}} \\
parallel composition & $P \,\mid\, Q$ & \texttt{par P Q} & \texttt{P || Q} & \HOLinline{\HOLFreeVar{P} \HOLSymConst{\ensuremath{\mid}} \HOLFreeVar{Q}} \\
restriction & $(\nu{}L)\;P$ & \texttt{restr L P} & \texttt{(nu L) P} & \HOLinline{\ensuremath{(\nu}\HOLFreeVar{L}\ensuremath{)} \HOLFreeVar{P}}  \\
recursion & $\recu A P$ & \texttt{rec A P} & \texttt{rec A P} & \HOLinline{\HOLConst{rec} \HOLFreeVar{A} \HOLFreeVar{P}}  \\
relabeling & $P\;[r\!f]$ & \texttt{relab P rf} &\texttt{relab P rf} & \HOLinline{\HOLConst{relab} \HOLFreeVar{P} \HOLFreeVar{rf}}  \\
constant/variable & $A$ & \texttt{var A} &\texttt{var A} & \HOLinline{\HOLConst{var} \HOLFreeVar{A}} \\
\hline
invisible action & $\tau$ & \texttt{tau} & \texttt{tau} & \HOLinline{\HOLSymConst{\ensuremath{\tau}}} \\
input action & $a$ & \texttt{label (name a)} &\texttt{In(a)} & \HOLinline{\HOLConst{\HOLTokenInputAct} \HOLFreeVar{a}} \\
output action & $\outC a$ & \texttt{label (coname a)} & \texttt{Out(a)} & \HOLinline{\HOLConst{\HOLTokenOutputAct} \HOLFreeVar{a}} \\
\hline
variable substitution & $E\{E'/X\}$ & \texttt{CCS_Subst E E' X} &
                                           \texttt{[E'/X]  E} & \HOLinline{\ensuremath{[}\ensuremath{\HOLFreeVar{E}\sp{\prime}}\ensuremath{/}\HOLFreeVar{X}\ensuremath{]} \HOLFreeVar{E}} \\
transition & $P\overset{\mu}{\longrightarrow}Q$
                       & \texttt{TRANS P u Q} & \texttt{P ---u-> Q} & \HOLinline{\HOLFreeVar{P} \HOLTokenTransBegin\HOLFreeVar{u}\HOLTokenTransEnd \HOLFreeVar{Q}} \\
weak transition & $P\overset{\mu}{\Longrightarrow}Q$
                       & \texttt{WEAK\_TRANS P u Q} & \texttt{P ==u=> Q} & \HOLinline{\HOLFreeVar{P} \HOLTokenWeakTransBegin\HOLFreeVar{u}\HOLTokenWeakTransEnd \HOLFreeVar{Q}} \\
$\epsilon$--transition & $P\overset{\epsilon}{\Longrightarrow}Q$
                       & \texttt{EPS P Q} & \texttt{EPS P Q} & \HOLinline{\HOLFreeVar{P} \HOLSymConst{\HOLTokenEPS} \HOLFreeVar{Q}} \\
\hline
\end{tabular}
\end{center}
   \caption{Syntax of some CCS concepts in HOL}
   \label{tab:ccsoperator}
\end{table}

The transition semantics of CCS processes strictly follows the SOS rules given
in Fig.~\ref{f:LTSCCS}. However, they are not axioms, but the
consequence of an \emph{inductive relation} definition of
\HOLinline{\HOLConst{TRANS}} by HOL's \texttt{HOL_reln} function.~(c.f.~\citep[p.~219]{holdesc} for more details.)
A successful invocation of the definitional principle returns three
important theorems (\emph{rules}, \emph{ind} and \emph{cases}):
\begin{itemize}
  \item \emph{rules} is a conjunction of implications that will be the
    same as the input term. In fact, the following formal versions of SOS rules are
    extracted from the \emph{rules} theorem:
\begin{alltt}
\HOLTokenTurnstile{} \HOLFreeVar{u}\HOLSymConst{\ensuremath{\ldotp}}\HOLFreeVar{E} \HOLTokenTransBegin\HOLFreeVar{u}\HOLTokenTransEnd \HOLFreeVar{E}\hfill\texttt{[PREFIX]}
\HOLTokenTurnstile{} \HOLFreeVar{E} \HOLTokenTransBegin\HOLFreeVar{u}\HOLTokenTransEnd \ensuremath{\HOLFreeVar{E}\sb{\mathrm{1}}} \HOLSymConst{\HOLTokenImp{}} \HOLFreeVar{E} \HOLSymConst{\ensuremath{+}} \ensuremath{\HOLFreeVar{E}\sp{\prime}} \HOLTokenTransBegin\HOLFreeVar{u}\HOLTokenTransEnd \ensuremath{\HOLFreeVar{E}\sb{\mathrm{1}}}\hfill\texttt{[SUM1]}
\HOLTokenTurnstile{} \HOLFreeVar{E} \HOLTokenTransBegin\HOLFreeVar{u}\HOLTokenTransEnd \ensuremath{\HOLFreeVar{E}\sb{\mathrm{1}}} \HOLSymConst{\HOLTokenImp{}} \ensuremath{\HOLFreeVar{E}\sp{\prime}} \HOLSymConst{\ensuremath{+}} \HOLFreeVar{E} \HOLTokenTransBegin\HOLFreeVar{u}\HOLTokenTransEnd \ensuremath{\HOLFreeVar{E}\sb{\mathrm{1}}}\hfill\texttt{[SUM2]}
\HOLTokenTurnstile{} \HOLFreeVar{E} \HOLTokenTransBegin\HOLFreeVar{u}\HOLTokenTransEnd \ensuremath{\HOLFreeVar{E}\sb{\mathrm{1}}} \HOLSymConst{\HOLTokenImp{}} \HOLFreeVar{E} \HOLSymConst{\ensuremath{\mid}} \ensuremath{\HOLFreeVar{E}\sp{\prime}} \HOLTokenTransBegin\HOLFreeVar{u}\HOLTokenTransEnd \ensuremath{\HOLFreeVar{E}\sb{\mathrm{1}}} \HOLSymConst{\ensuremath{\mid}} \ensuremath{\HOLFreeVar{E}\sp{\prime}}\hfill\texttt{[PAR1]}
\HOLTokenTurnstile{} \HOLFreeVar{E} \HOLTokenTransBegin\HOLFreeVar{u}\HOLTokenTransEnd \ensuremath{\HOLFreeVar{E}\sb{\mathrm{1}}} \HOLSymConst{\HOLTokenImp{}} \ensuremath{\HOLFreeVar{E}\sp{\prime}} \HOLSymConst{\ensuremath{\mid}} \HOLFreeVar{E} \HOLTokenTransBegin\HOLFreeVar{u}\HOLTokenTransEnd \ensuremath{\HOLFreeVar{E}\sp{\prime}} \HOLSymConst{\ensuremath{\mid}} \ensuremath{\HOLFreeVar{E}\sb{\mathrm{1}}}\hfill\texttt{[PAR2]}
\end{alltt}
\begin{equation*}
\infer[\HOLRuleName{PAR3}]{\HOLinline{\HOLFreeVar{E} \HOLSymConst{\ensuremath{\mid}} \ensuremath{\HOLFreeVar{E}\sp{\prime}} \HOLTokenTransBegin\HOLSymConst{\ensuremath{\tau}}\HOLTokenTransEnd \ensuremath{\HOLFreeVar{E}\sb{\mathrm{1}}} \HOLSymConst{\ensuremath{\mid}} \ensuremath{\HOLFreeVar{E}\sb{\mathrm{2}}}}}{\HOLinline{\HOLFreeVar{E} \HOLTokenTransBegin\HOLConst{label} \HOLFreeVar{l}\HOLTokenTransEnd \ensuremath{\HOLFreeVar{E}\sb{\mathrm{1}}}}&\HOLinline{\ensuremath{\HOLFreeVar{E}\sp{\prime}} \HOLTokenTransBegin\HOLConst{label} \ensuremath{(}\HOLConst{COMPL} \HOLFreeVar{l}\ensuremath{)}\HOLTokenTransEnd \ensuremath{\HOLFreeVar{E}\sb{\mathrm{2}}}}}
\end{equation*}
\begin{equation*}
\infer[\HOLRuleName{RESTR}]{\HOLinline{\ensuremath{(\nu}\HOLFreeVar{L}\ensuremath{)} \HOLFreeVar{E} \HOLTokenTransBegin\HOLFreeVar{u}\HOLTokenTransEnd \ensuremath{(\nu}\HOLFreeVar{L}\ensuremath{)} \ensuremath{\HOLFreeVar{E}\sp{\prime}}}}{\HOLinline{\HOLFreeVar{E} \HOLTokenTransBegin\HOLFreeVar{u}\HOLTokenTransEnd \ensuremath{\HOLFreeVar{E}\sp{\prime}}}&\HOLinline{\HOLFreeVar{u} \HOLSymConst{\ensuremath{=}} \HOLSymConst{\ensuremath{\tau}} \HOLSymConst{\HOLTokenDisj{}} \HOLFreeVar{u} \HOLSymConst{\ensuremath{=}} \HOLConst{label} \HOLFreeVar{l} \HOLSymConst{\HOLTokenConj{}} \HOLFreeVar{l} \HOLSymConst{\HOLTokenNotIn{}} \HOLFreeVar{L} \HOLSymConst{\HOLTokenConj{}} \HOLConst{COMPL} \HOLFreeVar{l} \HOLSymConst{\HOLTokenNotIn{}} \HOLFreeVar{L}}}
\end{equation*}
\begin{equation*}
\infer[\HOLRuleName{RELABELING}]{\HOLinline{\HOLConst{relab} \HOLFreeVar{E} \HOLFreeVar{rf} \HOLTokenTransBegin\HOLConst{relabel} \HOLFreeVar{rf} \HOLFreeVar{u}\HOLTokenTransEnd \HOLConst{relab} \ensuremath{\HOLFreeVar{E}\sp{\prime}} \HOLFreeVar{rf}}}{\HOLinline{\HOLFreeVar{E} \HOLTokenTransBegin\HOLFreeVar{u}\HOLTokenTransEnd \ensuremath{\HOLFreeVar{E}\sp{\prime}}}}
\end{equation*}
\begin{equation*}
\infer[\HOLRuleName{REC}]{\HOLinline{\HOLConst{rec} \HOLFreeVar{X} \HOLFreeVar{E} \HOLTokenTransBegin\HOLFreeVar{u}\HOLTokenTransEnd \ensuremath{\HOLFreeVar{E}\sb{\mathrm{1}}}}}{\HOLinline{\ensuremath{[}\HOLConst{rec} \HOLFreeVar{X} \HOLFreeVar{E}\ensuremath{/}\HOLFreeVar{X}\ensuremath{]} \HOLFreeVar{E} \HOLTokenTransBegin\HOLFreeVar{u}\HOLTokenTransEnd \ensuremath{\HOLFreeVar{E}\sb{\mathrm{1}}}}}
\end{equation*}
\item \emph{ind} is the induction principle for the relation
  (see Section~\ref{sec:multivariate} for the exact form and its application).
\item \emph{cases} is the so-called `cases' or `inversion' theorem for
  the relations, and is used to decompose an element in the relation into the possible ways of
  obtaining it by the rules. For instance, it leads to the following
  theorem:
\begin{alltt}
\HOLTokenTurnstile{} \HOLFreeVar{E} \HOLSymConst{\ensuremath{+}} \ensuremath{\HOLFreeVar{E}\sp{\prime}} \HOLTokenTransBegin\HOLFreeVar{u}\HOLTokenTransEnd \ensuremath{\HOLFreeVar{E}\sp{\prime\prime}} \HOLSymConst{\HOLTokenEquiv{}} \HOLFreeVar{E} \HOLTokenTransBegin\HOLFreeVar{u}\HOLTokenTransEnd \ensuremath{\HOLFreeVar{E}\sp{\prime\prime}} \HOLSymConst{\HOLTokenDisj{}} \ensuremath{\HOLFreeVar{E}\sp{\prime}} \HOLTokenTransBegin\HOLFreeVar{u}\HOLTokenTransEnd \ensuremath{\HOLFreeVar{E}\sp{\prime\prime}}\hfill{[TRANS_SUM_EQ]}
\end{alltt}
\end{itemize}

The last SOS rule \texttt{REC} (Recursion)
 says that if we substitute all occurrences of the variable $A$ in $P$ to
$(\recu A P)$ and the resulting process has a transition to $P'$
with action $u$, then $(\recu A P)$ has the same
transition. Here, ``\HOLinline{\ensuremath{[}\HOLConst{rec}\;\;\HOLFreeVar{X}\;\;\HOLFreeVar{E}\ensuremath{/}\HOLFreeVar{X}\ensuremath{]}\;\;\HOLFreeVar{E}}'' is an abbreviation
for ``\HOLinline{\HOLConst{CCS_Subst}\;\;\HOLFreeVar{E}\;\;\ensuremath{(}\HOLConst{rec}\;\;\HOLFreeVar{X}\;\;\HOLFreeVar{E}\ensuremath{)}\;\;\HOLFreeVar{X}}'', where \HOLinline{\HOLConst{CCS_Subst}} is a
recursive function substituting all appearences of a free variable \hl{with} another CCS term.
For most CCS operators \HOLinline{\HOLConst{CCS_Subst}} just recursively goes into a deeper
level without changing anything, e.g.
\begin{alltt}
\HOLTokenTurnstile{} \ensuremath{[}\ensuremath{\HOLFreeVar{E}\sp{\prime}}\ensuremath{/}\HOLFreeVar{X}\ensuremath{]} \ensuremath{(}\ensuremath{\HOLFreeVar{E}\sb{\mathrm{1}}} \HOLSymConst{\ensuremath{+}} \ensuremath{\HOLFreeVar{E}\sb{\mathrm{2}}}\ensuremath{)} \HOLSymConst{\ensuremath{=}} \ensuremath{[}\ensuremath{\HOLFreeVar{E}\sp{\prime}}\ensuremath{/}\HOLFreeVar{X}\ensuremath{]} \ensuremath{\HOLFreeVar{E}\sb{\mathrm{1}}} \HOLSymConst{\ensuremath{+}} \ensuremath{[}\ensuremath{\HOLFreeVar{E}\sp{\prime}}\ensuremath{/}\HOLFreeVar{X}\ensuremath{]} \ensuremath{\HOLFreeVar{E}\sb{\mathrm{2}}}\hfill{[CCS_Subst_sum]}
\end{alltt}
The only two insteresting cases are at agent variables and recursion:
\begin{alltt}
\HOLTokenTurnstile{} \ensuremath{[}\ensuremath{\HOLFreeVar{E}\sp{\prime}}\ensuremath{/}\HOLFreeVar{X}\ensuremath{]} \ensuremath{(}\HOLConst{var} \HOLFreeVar{Y}\ensuremath{)} \HOLSymConst{\ensuremath{=}} \HOLKeyword{if} \HOLFreeVar{Y} \HOLSymConst{\ensuremath{=}} \HOLFreeVar{X} \HOLKeyword{then} \ensuremath{\HOLFreeVar{E}\sp{\prime}} \HOLKeyword{else} \HOLConst{var} \HOLFreeVar{Y}\hfill{[CCS_Subst_var]}
\HOLTokenTurnstile{} \ensuremath{[}\ensuremath{\HOLFreeVar{E}\sp{\prime}}\ensuremath{/}\HOLFreeVar{X}\ensuremath{]} \ensuremath{(}\HOLConst{rec} \HOLFreeVar{Y} \HOLFreeVar{E}\ensuremath{)} \HOLSymConst{\ensuremath{=}} \HOLKeyword{if} \HOLFreeVar{Y} \HOLSymConst{\ensuremath{=}} \HOLFreeVar{X} \HOLKeyword{then} \HOLConst{rec} \HOLFreeVar{Y} \HOLFreeVar{E} \HOLKeyword{else} \HOLConst{rec} \HOLFreeVar{Y} \ensuremath{(}\ensuremath{[}\ensuremath{\HOLFreeVar{E}\sp{\prime}}\ensuremath{/}\HOLFreeVar{X}\ensuremath{]} \HOLFreeVar{E}\ensuremath{)}\hfill{[CCS_Subst_rec]}
\end{alltt}
The variable substitutions only affects free
variables: the variable $Y$ in ``\HOLinline{\HOLConst{rec}\!\;\HOLFreeVar{Y}\!\;\HOLFreeVar{E}}'' is
bound and therefore the substitution ignores it.
% The actual substitution happens in case of
% ``\HOLinline{\HOLConst{var} \HOLFreeVar{Y}}'' if $Y$ is being
% substituted (and if \HOLinline{\HOLConst{CCS_Subst}} can actually reach it).

A useful facility exploiting the interplay
between HOL4 and Standard ML (that follows an idea by Nesi \cite{Nesi:1992ve})
 is a \hl{recursive} ML function \hl{which takes} a CCS process and \hl{returns} a theorem indicating all
  direct transitions of the process. \hl{(If the input process is
 infinite branching, the function will not terminate.)}
For instance, we know that the process $(a.\nil | \bar{a}.\nil)$ has three
possible transitions: $(a.\nil | \bar{a}.\nil) \overset{a}{\longrightarrow}
(\nil | \bar{a}.\nil)$, $(a.\nil | \bar{a}.\nil)
\overset{\bar{a}}{\longrightarrow} (a.\nil | \nil)$ and $(a.\nil | \bar{a}.\nil)
\overset{\tau}{\longrightarrow} (\nil | \nil)$.
To completely describe all possible transitions of a process, if done manually, the
following two facts must be proved: (1) there are indeed the three
  transitions mentioned above; (2) there are no other
  transitions. For large CCS processes it is surprisingly tedious to manually derive
all the possible transitions and prove the non-existence of others.
This shows the usefulness of appealing to an ML function
\texttt{CCS\_TRANS\_CONV} that is designed to automate the whole process.
For instance, calling it on $(a.\nil | \bar{a}.\nil)$ the function returns the
following theorem which describes all the transitions (just one-step) of the process:
\begin{alltt}
\HOLTokenTurnstile{} \HOLConst{\HOLTokenInputAct} \HOLStringLit{a}\HOLSymConst{\ensuremath{\ldotp}}\HOLConst{\ensuremath{\mathbf{0}}} \HOLSymConst{\ensuremath{\mid}} \HOLConst{\HOLTokenOutputAct} \HOLStringLit{a}\HOLSymConst{\ensuremath{\ldotp}}\HOLConst{\ensuremath{\mathbf{0}}} \HOLTokenTransBegin\HOLFreeVar{u}\HOLTokenTransEnd \HOLFreeVar{E} \HOLSymConst{\HOLTokenEquiv{}}
   \ensuremath{(}\HOLFreeVar{u} \HOLSymConst{\ensuremath{=}} \HOLConst{\HOLTokenInputAct} \HOLStringLit{a} \HOLSymConst{\HOLTokenConj{}} \HOLFreeVar{E} \HOLSymConst{\ensuremath{=}} \HOLConst{\ensuremath{\mathbf{0}}} \HOLSymConst{\ensuremath{\mid}} \HOLConst{\HOLTokenOutputAct} \HOLStringLit{a}\HOLSymConst{\ensuremath{\ldotp}}\HOLConst{\ensuremath{\mathbf{0}}} \HOLSymConst{\HOLTokenDisj{}} \HOLFreeVar{u} \HOLSymConst{\ensuremath{=}} \HOLConst{\HOLTokenOutputAct} \HOLStringLit{a} \HOLSymConst{\HOLTokenConj{}} \HOLFreeVar{E} \HOLSymConst{\ensuremath{=}} \HOLConst{\HOLTokenInputAct} \HOLStringLit{a}\HOLSymConst{\ensuremath{\ldotp}}\HOLConst{\ensuremath{\mathbf{0}}} \HOLSymConst{\ensuremath{\mid}} \HOLConst{\ensuremath{\mathbf{0}}}\ensuremath{)} \HOLSymConst{\HOLTokenDisj{}}
   \HOLFreeVar{u} \HOLSymConst{\ensuremath{=}} \HOLSymConst{\ensuremath{\tau}} \HOLSymConst{\HOLTokenConj{}} \HOLFreeVar{E} \HOLSymConst{\ensuremath{=}} \HOLConst{\ensuremath{\mathbf{0}}} \HOLSymConst{\ensuremath{\mid}} \HOLConst{\ensuremath{\mathbf{0}}}
\end{alltt}

% next file: bisim.htex
