\subsection{Expansions and Contractions}
\label{s:mcontr}

Milner's ``unique solution of equations'' theorem for $\wb$
(Theorem~\ref{t:Mil89})
brings a new proof technique for proving (weak) bisimilarities. However, it has
 limitations: the equations must be guarded and sequential (moreover
all sums where equation variables appear must be guarded sums).
This limits the usefulness of the technique, since
the occurrences of other operators above the variables, such as parallel
composition and restriction, in general cannot be removed.
The constraints of Theorem~\ref{t:Mil89}, however, can be 
weakened if we move from equations to a special kind of inequations called
  \emph{contractions}.

Intuitively, the bisimilarity contraction $\mcontrBIS$ is a preorder
such that, $P \mcontrBIS Q$ holds if $P \wb Q$ and, in addition, 
$Q$ \emph{has the possibility of being at least as efficient as $P$} (as far as
$\tau$-actions are performed).
Process $Q$, however, may be non-deterministic and may have other ways
of doing the same work, and these could be slow (i.e., involving
more $\tau$-actions than those performed by $P$).
% Thus, in contrast with expansion,  we cannot really say that `$Q$ is more efficient than
% $P$'.

\begin{definition}[contraction]
\label{d:BisCon}
A process relation ${\R}$ 
 is a \emph{(bisimulation) contraction} if, whenever
 $P\RR Q$, %for all $\mu$ 

\begin{enumerate}
\item   $P \arr\mu P'$ implies there is $Q'$ such that $Q \arcap \mu
  Q'$ and $P' \RR Q'$;
\item $Q \arr\mu Q'$   implies there is $P'$ such that $P \Arcap \mu
 P'$ and $P' \wb Q'$\enspace.
\end{enumerate}
Two processes $P$ and $Q$ are in the \emph{bisimilarity
contraction}, written as $P \mcontrBIS Q$ ($P$ \emph{contracts to} $Q$),
if $P\ \R\ Q$ for some contraction $\R$.
Sometimes we write $\mexpaBIS$ for the inverse of $\mcontrBIS$.
\end{definition}

In the first clause $Q$ is required to match $P$'s challenge
transition with at most one transition.
This makes sure that $Q$ is capable of mimicking % verb: mimic
$P$'s work at least as efficiently as $P$. 
In contrast, the second clause of Definition~\ref{d:BisCon}, on the
challenges from $Q$, entirely ignores efficiency: it is the same
clause of  weak bisimulation~---the final derivatives are indeed required
to be related  by $\wb$, rather than by $\R$.

Bisimilarity contraction is coarser than 
the \emph{expansion relation}
$\expa$~\cite{arun1992efficiency,sangiorgi2015equations}:
\begin{definition}[expansion]
\label{d:expa}
A process relation ${\R}$
  is an \emph{expansion} if, whenever
we have $P\RR Q$, for all $\mu$
 \begin{enumerate}
 \item   $P \arr\mu P'$ implies that there is $Q'$ with $Q \arcap \mu  Q'$
  and $P' \RR Q'$;
 \item $Q \arr\mu Q'$   implies that there is $P'$ with $P \Arr \mu P'$ and $P' \RR Q'$.
 \end{enumerate}
Two processes $P$ and $Q$ are in the \emph{bisimilarity
  expansion}, written as $P \expa Q$ ($P$ \emph{expands} $Q$),
 if $P \RR Q$ for some expansion $\R$.
 \end{definition}
The expansion preorder is widely used in proof techniques for bisimilarity and
intuitively refines bisimilarity by 
formalising the idea of ``efficiency'' between processes.
Clause (1) is the same in the both preorders, while in clause (2) expansion uses 
$P \Arr \mu P'$, rather than $P \Arcap \mu P'$; 
moreover in clause (2) of Def.~\ref{d:BisCon} the final derivatives
are simply required to be bisimilar ($P' \wb Q'$).
Intuitively, $P \expa Q$ holds if $P\wb Q$ and, in addition, \emph{$Q$
  is always at least as efficient as $P$}.

\begin{example}
\label{exa:contr}
We have %\mcontrBIS a + \tau^n . a $
 $ a \not  \mcontrBIS \tau. a$. However,
$a+ \tau . a \mcontrBIS a$, as well as its converse, 
$  a \mcontrBIS a +
\tau . a $. Indeed, if $P \wb Q$ then 
$  P  \mcontrBIS P +Q$. The last two relations do not hold with 
$\expa$, which explains the strictness of the inclusion
 ${\expa} \subset {\mcontrBIS}$. 
% The inclusion is strict: for instance
% $a+ \tau . a \mcontrBIS a$, where $\mcontrBIS$ cannot be replaced by
%  $\contr$. Also the converse of  $a+ \tau . a \mcontrBIS a$ holds, namely
% $  a \mcontrBIS a +
% \tau . a $. However, we have %\mcontrBIS a + \tau^n . a $
%  $ a \not  \mcontrBIS \tau. a$
\end{example} 

In the same way as bisimilarity, both the expansion and the
contraction preorders are preserved by all CCS operators except the
summation. The proofs are similar, see e.g.~\cite{sangiorgi2017equations}.

% next file: unique.tex
