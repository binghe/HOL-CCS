%%%% -*- Mode: LaTeX; -*-

% 30 minutes presentation of EXPRESS/SOS 2018 paper.

% Target: 25 slides (exclude title and thanks pages)

% Table of Contents: (just write what I did in this paper!!!!!)

% 1. CCS.
% 2. Unique solution of equations (Milner)
% 3. Contraction and unique solution theorem (Sangiorgi)
% 4. Rooted contraction and unique solution theorem
% 5. Coarsest (pre)congruence contained in \approx and \mcontrBIS

% N (conclusions):
%% 1. "sometimes the greatest advantage of formalisation of theories
%% with theorem provers is the refinement of previously know results".
\begin{frame}[fragile]\frametitle{Project Motivation}
\begin{itemize}
\item Concurrency Theory is important 
  for understanding concurrent and reactive systems;
\item Milner's Calculus of Communicating Systems (CCS) is
  simple, elegant process calculi widely adopted in Concurrency Theory courses, yet textbooks
  cannot provide all proof details;
\item The CCS formalisation project is a good chance for learning Interactive Theorm
  Proving (ITP), with minimal dependencies on other formal theories.
\end{itemize}
\begin{block}{Project summary}
\begin{itemize}
\item ~20,000 lines of Standard ML code;
\item ~500 \emph{manually} proved lemmas/theorems.
\end{itemize}
Availabile in HOL official examples:
\begin{color}{blue}
\begin{texttt}
https://github.com/HOL-Theorem-Prover/HOL/tree/master/examples/CCS
\end{texttt}
\end{color}
\end{block}
\end{frame}

%4
\begin{frame}[fragile]
\frametitle{Calculus of Communicating Systems (CCS)}
\vspace{-2ex}
\begin{definition}[Actions and CCS processes]
\vspace{-2ex}
\begin{equation*}
\begin{small}
\begin{array}{ccl}
\mu  & := &  \tau \hspace{.3pt} \midd a  \midd  \outC a  \\
P  & := &  \nil \midd  \mu.P \midd  P_1 |  P_2 \midd
P_1 + P_2 \midd  (\res a\!)\, P  \midd  P\; [rf]  \midd A \midd \recu A  P
\end{array}
\end{small}
\end{equation*}
\end{definition}
\begin{definition}[Structural Operational Semantics]
\begin{center}
\begin{small}
%\vskip .1cm
 $\displaystyle{  \over  \mu.  P    \arr\mu
P } $  $ \hb$   
%\hskip .5cm
 $\displaystyle{   P \arr\mu   P' \over   P + Q   \arr\mu
P'  } $  $ \hb$   
%\hskip .5cm
 $\displaystyle{   Q \arr\mu   Q' \over   P + Q   \arr\mu
Q'  } $  $ \hb$ 
%\hskip .5cm
 $\displaystyle{   P \arr\mu   P' \over   P | Q   \arr\mu
P' | Q } $  $ \hb$   
\\
\vspace{.2cm}
 $\displaystyle{   Q \arr\mu   Q' \over   P | Q   \arr\mu
P | Q' } $  $ \hb$   
%\hskip .3cm
$\displaystyle{ P \arr{ a}P' \hk \hk  Q
\arr{\outC a }Q'  \over     P|  Q \arr{ \tau} P'
|  Q'  }$ 
\hskip .3cm
$\displaystyle{ P \arr{\mu}P' \over
 (\res a\!)\, P   \arr{\mu} (\res a\!)\, P'} $ $ \mu \neq a, \outC a$
$ \hb$
\\
\vspace{.2cm}
$\displaystyle{ P \arr{\mu} P' \over
 P \;[rf] \arr{rf(\mu)} P' \;[rf]} $ $\forall a.\, rf(\outC a) = \overline{rf(a)}$
$ \hb$
\hskip .2cm
$\displaystyle{ P \sub {\recu A P} A \arr{\mu}P' \over
 \recu A P   \arr{ \mu} P'  } $
\end{small}
\end{center}
\end{definition}
\end{frame}

\begin{frame}
\frametitle{Bisimulation and Bisimiarity}
\begin{small}
\begin{definition}
A process relation ${\R}$ is a \emph{strong bisimulation} if, whenever
 $P\RR Q$, we have:
\begin{enumerate}
\item $P \arr\mu P'$ implies that there is $Q'$ such that $Q \arr \mu Q'$ and $P' \RR Q'$;
\item $Q \arr\mu Q'$,implies that there is $P'$ such that $P \arr \mu P'$ and $P' \RR Q'$\enspace.
\end{enumerate}  
 $P$ and $Q$ are \emph{bisimilar} ($P \sim Q$), if $P \RR Q$ for some bisimulation $\R$.
\end{definition}
\begin{definition}
A process relation ${\R}$ is a \emph{weak bisimulation} if, whenever
 $P\RR Q$, we have:
\begin{enumerate}
\item $P \arr\mu P'$ implies that there is $Q'$ such that $Q \Arcap \mu Q'$ and $P' \RR Q'$%\vspace{-4pt}
\item $Q \arr\mu Q'$,implies that there is $P'$ such that $P \Arcap
  \mu P'$ and $P' \RR Q'$\enspace.
\end{enumerate}  
 $P$ and $Q$ are \emph{weakly bisimilar} ($P \wb Q$), if $P \RR Q$ for some bisimulation $\R$.
\end{definition}
\end{small}
\end{frame}

\begin{frame}
\frametitle{Rooted Bisimiarity (Observation Congruence)}
\begin{definition}
Two processes $P$ and $Q$ are \textbf{rooted bisimilar} ($P\rapprox\! Q$), if we have:
\begin{enumerate}
 \item  $P \arr\mu P'$ implies that there is $Q'$ such that $Q
   \Arr\mu Q'$ and $P' \wb Q'$;
 \item  $Q \arr\mu Q'$ implies that there is $P'$ such that $P
   \Arr\mu P'$ and $P' \wb Q'$\enspace.
\end{enumerate}
\end{definition}
\begin{theorem}
\begin{enumerate}
\item $P \rapprox Q \Rightarrow P \wb Q$;
\item $\sim$ and $\rapprox$ are preserved by all CCS operators; \hfill
  (Congruence)
\item $\approx$ is preserved by all CCS operators but direct sums;\hfill
  (``Congruence'')
\item $P \approx Q\ \Leftrightarrow\ (P \rapprox\! Q
  \vee P \rapprox\! \tau.Q \vee \tau.P \rapprox\! Q)$.\hfill (Hennessy Lemma)
\item $P \rapprox\! Q\ \Leftrightarrow\ ( \forall R.\
  P + R \approx Q + R)$.\hfill (Coarsest Congruence in $\approx$)
\end{enumerate}
\end{theorem}
\end{frame}

\begin{frame}
\frametitle{Unique Solution of Equations (Robin Milner, 1989)}
\begin{theorem}[for $\sim$]
Let $\tilde{E}$ be weakly guarded with free variables at most
$\tilde{X}$, and let $\tilde{P}\sim\tilde{E}\{\tilde{P}/\tilde{X}\}$,
$\tilde{Q}\sim\tilde{E}\{\tilde{Q}/\tilde{X}\}$, then $\tilde{P}\sim\tilde{Q}$.
\end{theorem}

\begin{theorem}[for $\approx$, not explicitly appeared]
Let $\tilde{E}$ be guarded and sequential with only guarded sums and
free variables at most $\tilde{X}$, and let
$\tilde{P}\approx\tilde{E}\{\tilde{P}/\tilde{X}\}$,
$\tilde{Q}\approx\tilde{E}\{\tilde{Q}/\tilde{X}\}$, then $\tilde{P}\approx\tilde{Q}$.
\end{theorem}

\begin{theorem}[for $\rapprox$]
Let $\tilde{E}$ be guarded and sequential with free variables at most
$\tilde{X}$, and let $\tilde{P}\rapprox\!\tilde{E}\{\tilde{P}/\tilde{X}\}$,
$\tilde{Q}\rapprox\!\tilde{E}\{\tilde{Q}/\tilde{X}\}$, then $\tilde{P}\rapprox\!\tilde{Q}$.
\end{theorem}
\end{frame}

\begin{frame}
\frametitle{Conditions required by Milner's theorems (for $\sim$, $\rapprox$)}
\begin{definition}
$X$ is \emph{weakly guarded} in $E$ if each occurrence of $X$ is
within some subexpression $\mu.F$ of $E$.
\end{definition}

\begin{definition}
$X$ is \emph{(strongly) guarded} in $E$ if each occurrence of $X$ is
within some subexpression $l.F$ of $E$.

$X$ is \emph{sequential} in $E$ if every subexpression of $E$ which
contains $X$, apart from $X$ itself, is of the form $\mu.F$ or
$\Sigma \tilde{F}$.
\begin{small}
\begin{enumerate}
\item Any $P$ (without $X$) is guarded and sequential;
\item If $E$ is sequential, then $l.E$ is guarded and sequential;
\item If $E$ is guarded and sequential, so is $\mu.E$;
\item If $E_1$ and $E_2$ are both guarded and sequential, so is $E_1 + E_2$.
\end{enumerate}
\end{small}
\end{definition}
\end{frame}

\begin{frame}
\frametitle{Conditions required by Milner's theorem (for $\approx$)}
\begin{definition}
$X$ is \emph{sequential with only guarded sums} in $E$ if every subexpression of $E$ which
contains $X$, apart from $X$ itself, is of the form $\mu.F$ or
$\Sigma \mu_i.F_i$.
\begin{small}
\begin{enumerate}
\item Any $P$ (without $X$) is guarded and ``sequential'';
\item If $E$ is ``sequential'', then $l.E$ is guarded and
  ``sequential'';
\item If $E$ is guarded and ``sequential'', so is $\mu.E$;
\item If $E_1$ and $E_2$ are guarded and ``sequential'', so is
  $\tau.E_1 + \tau.E_2$;
\item If $E_1$ is guarded and ``sequential'', $E_2$ is ``sequential'',
  then $\tau.E_1 + l.E_2$ is guarded and ``sequential'';
\item If $E_1$ is ``sequential'', $E_2$ is guarded and ``sequential'',
  then $l.E_1 + \tau.E_2$ is guarded and ``sequential'';
\item If $E_1$ and $E_2$ are ``sequential'', $l_1.E_1 +
  l_2.E_2$ is guarded and ``sequential''.
\end{enumerate}
\end{small}
\end{definition}
\end{frame}

\begin{frame}
\frametitle{A refinement of Milner's technique (D. Sangiorgi, 2015)}
\begin{small}
\begin{definition}[Contraction]
A process relation ${\R}$ 
 is a \emph{(bisimulation) contraction} if whenever
 $P\RR Q$, %for all $\mu$ 
\begin{enumerate}
\item  $P \arr\mu P'$ implies there is $Q'$ such that $Q \arcap \mu
  Q'$ and $P' \RR Q'$;
\item $Q \arr\mu Q'$   implies there is $P'$ such that $P \Arcap \mu
 P'$ and $P' \wb Q'$\enspace.
\end{enumerate}
\emph{Bisimilarity contraction}, written as $P \mcontrBIS Q$, if $P\ \R\ Q$ for some contraction $\R$.
%
\end{definition}
\end{small}
\begin{lemma}[Precongruence of $\mcontrBIS$ in CCS]
$\mcontrBIS$ is a preorder (reflexive, transitive) and is preserved by
all CCS operators but direct sums.
\end{lemma}
\begin{theorem}[Unique Solution of Contractions]
Let $\tilde{E}$ be weakly guarded with only guarded sums and
free variables at most $\tilde{X}$, and let
$\tilde{P}\mcontrBIS\tilde{E}\{\tilde{P}/\tilde{X}\}$,
$\tilde{Q}\mcontrBIS\tilde{E}\{\tilde{Q}/\tilde{X}\}$, then $\tilde{P}\approx\tilde{Q}$.
\end{theorem}
\end{frame}

\begin{frame}
\frametitle{Further refinements (C. Tian, 2017; the current paper)}
\begin{definition}[Rooted contraction]
Two processes $P$ and $Q$ are in \emph{rooted contraction}, written as
 $P\rcontr Q$, if
\begin{enumerate}
\item $P \arr\mu P'$ implies that there is $Q'$ with $Q \arr \mu Q'$
 and $P'\mcontrBIS Q'$;
\item $Q \arr\mu Q'$   implies that there is $P'$ with $P \Arr \mu
 P'$ and $P' \wb Q'$\enspace.
\end{enumerate}
\end{definition}
\begin{lemma}[Precongruence of $\rcontr$ in CCS]
$\mcontrBIS$ is a preorder (reflexive, transitive) and is preserved by
all CCS operators.
\end{lemma}
\begin{theorem}[Unique Solution of Rooted Contractions]
Let $\tilde{E}$ be weakly guarded with free variables at most $\tilde{X}$, and let
$\tilde{P}\rcontr\tilde{E}\{\tilde{P}/\tilde{X}\}$,
$\tilde{Q}\rcontr\tilde{E}\{\tilde{Q}/\tilde{X}\}$, then
$\tilde{P}\rapprox\tilde{Q}$ (thus also $\tilde{P}\approx\tilde{Q}$).
\end{theorem}
\end{frame}

\begin{frame}
\frametitle{CCS Formalisation in HOL (Monica Nesi and Chun Tian)}
\begin{center}
\begin{small}
\begin{tabular}{|l|l|r|}
\hline
\textbf{Name} & \textbf{Summary} & \textbf{Lines} \\
\hline
CCSTheory & Basic CCS definitions, SOS rules & 1009 \\
CCSConv & Decision procedure of CCS transitions & 562 \\
StrongEQTheory & Strong bisimulation and bisimilarity ($\sim$) & 634 \\
StrongLawsTheory & Algebraic laws of $\sim$; expansion law & 2002 \\
WeakEQTheory & Weak bisimulation and bisimilarity & 1974 \\
WeakLawsTheory & Algebraic laws of $\wb$ & 335 \\
ObsCongrTheory & Observation congruence ($\rapprox$) & 697 \\
ObsCongrLawsTheory & Algebraic laws of $\rapprox$ & 402 \\
\hline
BisimulationUptoTheory & Bisimulation up to $\sim$, $\wb$ and $\rapprox$ & 1180 \\
CongruenceTheory & Context, guardedness and congruence & 1457 \\
CoarsestCongrTheory & Deep results between $\wb$ and $\rapprox$ & 872 \\
TraceTheory & Trace and relationship with weak transition & 753 \\
ExpansionTheory & Expansion preorder, precongruence & 976 \\
ContractionTheory & Contraction preorder, precongruence & 2068 \\
UniqueSolutionTheory & Unique solution of equations/contractions & 2386 \\
\hline
\end{tabular}
\end{small}
\end{center}
\end{frame}

%3
\def\HOL{\textsc{Hol}}
\newcommand\fun{{\to}}
\newcommand\prd{{\times}}
\newcommand{\ty}[1]{\textsl{#1}}
\newcommand\conj{\ \wedge\ }
\newcommand\disj{\ \vee\ }
\newcommand\imp{ \Rightarrow }
\newcommand\eqv{\ \equiv\ }
\newcommand\vbar{\mid}
\newcommand\turn{\ \vdash\ } % FIXME: "\ " resultgs in extra space
\newcommand\hilbert{\varepsilon}
\newcommand{\uquant}[1]{\forall #1.\ }
\newcommand{\equant}[1]{\exists #1.\ }
\newcommand{\hquant}[1]{\hilbert #1.\ }
\newcommand{\iquant}[1]{\exists ! #1.\ }
\newcommand{\lquant}[1]{\lambda #1.\ }
\newcommand{\ml}[1]{\mbox{{\def\_{\char'137}\texttt{#1}}}}
\newcommand{\con}[1]{\mathrm{#1}}

\newcommand\bool{\ty{bool}}
\newcommand\num{\ty{num}}
\newcommand\ind{\ty{ind}}
\newcommand\lst{\ty{list}}

\providecommand{\T}{\con{T}}
\renewcommand{\T}{\con{T}}
\newcommand\F{\con{F}}
\newcommand\OneOne{\con{One\_One}}
\newcommand\OntoSubset{\con{Onto\_Subset}}
\newcommand\Onto{\con{Onto}}
\newcommand\TyDef{\con{Type\_Definition}}

\newlength{\ttX}
\settowidth{\ttX}{\tt X}
\newcommand{\tyvar}{\setlength{\unitlength}{\ttX}\begin{picture}(1,6)
\put(.5,0){\makebox(0,0)[b]{\footnotesize type variables}}
\put(0,1.5){\vector(0,1){4.5}}
\end{picture}}
\newcommand{\tyatom}{\setlength{\unitlength}{\ttX}\begin{picture}(1,6)
\put(.5,2.3){\makebox(0,0)[b]{\footnotesize atomic types}}
\put(.5,3.3){\vector(0,1){2.6}}
\end{picture}}
\newcommand{\funty}{\setlength{\unitlength}{\ttX}\begin{picture}(1,6)
\put(.5,1.5){\makebox(0,0)[b]{\footnotesize function types}}
\put(.5,0){\makebox(0,0)[b]{\footnotesize (domain, codomain)}}
\put(1,2.5){\vector(0,1){3.5}}
\end{picture}}
\newcommand{\cmpty}{\setlength{\unitlength}{\ttX}\begin{picture}(1,6)
\put(2,3.3){\makebox(0,0)[b]{\footnotesize compound types}}
\put(1.9,4.5){\vector(0,1){1.5}}
\end{picture}}

\settowidth{\ttX}{\tt X}
\newcommand{\var}{\setlength{\unitlength}{\ttX}\begin{picture}(1,6)
\put(.5,0){\makebox(0,0)[b]{\footnotesize variables}}
\put(0,1.5){\vector(0,1){4.5}}
\end{picture}}
\newcommand{\const}{\setlength{\unitlength}{\ttX}\begin{picture}(1,6)
\put(.5,2.3){\makebox(0,0)[b]{\footnotesize constants}}
\put(.5,3.5){\vector(0,1){2.4}}
\end{picture}}
\newcommand{\app}{\setlength{\unitlength}{\ttX}\begin{picture}(1,6)
\put(.5,1.5){\makebox(0,0)[b]{\footnotesize function applications}}
\put(.5,0){\makebox(0,0)[b]{\footnotesize (function $t$, argument $t'$)}}
\put(0.5,2.5){\vector(0,1){3.5}}
\end{picture}}
\newcommand{\abs}{\setlength{\unitlength}{\ttX}\begin{picture}(1,6)
\put(1,3.3){\makebox(0,0)[b]{\footnotesize $\lambda$-abstractions}}
\put(0.7,4.5){\vector(0,1){1.5}}
\end{picture}}

\begin{frame}[fragile]\frametitle{Higher Order Logic (HOL)}
\begin{footnotesize}
\vspace*{-1em}
\begin{columns}
\column{.73\textwidth}
\begin{definition}[Type in HOL]
\vspace*{-1.5em}
\begin{equation*}
\sigma\quad ::=\quad {\mathord{\mathop{\alpha}\limits_{\tyvar}}}
        \quad\mid\quad{\mathord{\mathop{c}\limits_{\tyatom}}}
        \quad\mid\quad\underbrace{(\sigma_1, \ldots , \sigma_n){op}}_{\cmpty}
        \quad\mid\quad\underbrace{\sigma_1\fun\sigma_2}_{\funty}
\end{equation*}
\end{definition}
\vspace*{-10pt}
\begin{definition}[Term in HOL]
\vspace*{-1em}
\begin{equation*}
t \quad ::=\quad {\mathord{\mathop{x}\limits_{\var}}}
        \quad\mid\quad{\mathord{\mathop{c}\limits_{\const}}}
        \quad\mid\quad\underbrace{t\ t'}_{\app}
        \quad\mid\quad\underbrace{\lambda x .\ t}_{\abs}
\end{equation*}
\end{definition}
\column{.23\textwidth}
\begin{block}{Primitive rules}
\begin{tiny}
\begin{enumerate}
\item Assumption introduction [{\tt ASSUME}],
\item Reflexivity [{\tt
  REFL}],
\item $\beta$-conversion [{\tt BETA\_CONV}],
\item Substitution [{\tt
SUBST}],
\item Abstraction [{\tt ABS}],
\item Type instantiation [{\tt
INST\_TYPE}], 
\item Discharging an assumption [{\tt DISCH}], 
\item Modus Ponens [{\tt MP}]
\end{enumerate}
\end{tiny}
\end{block}
\end{columns}
\begin{columns}
\column{.45\textwidth}
\vspace*{-1.2em}
\begin{block}{Logical constants}
\vspace*{-1em}
\begin{tiny}
\[
\begin{array}{l}
\turn \T       =  ((\lquant{x_{\ty{bool}}}x) =
               (\lquant{x_{\ty{bool}}}x))    \\
\turn \forall  =  \lquant{P_{\alpha\fun\ty{bool}}}\ P =
                    (\lquant{x}\T ) \\
\turn \exists  =  \lquant{P_{\alpha\fun\ty{bool}}}\
                    P({\hilbert}\ P) \\
\turn \F       =  \uquant{b_{\ty{bool}}}\ b  \\
\turn \neg    =  \lquant{b}\ b \imp \F \\
\turn {\wedge}  =  \lquant{b_1\ b_2}\uquant{b}
                     (b_1\imp (b_2 \imp b)) \imp b \\
\turn {\vee}  =  \lquant{b_1\ b_2}\uquant{b}
                   (b_1 \imp b)\imp ((b_2 \imp b) \imp b) \\
\end{array}
\]
\end{tiny}
\end{block}
\column{.5\textwidth}
\begin{block}{Axioms}
\vspace*{-1em}
\begin{tiny}
\[
\begin{array}{@{}l@{\qquad}l}
\mbox{\tiny\tt BOOL\_CASES\_AX}&\vdash \uquant{b} (b = \T ) \vee (b = \F )\\
% \\
\mbox{\tiny\tt ETA\_AX}&
\vdash \uquant{f_{\alpha\fun\beta}}(\lquant{x}f\ x) = f\\
% \\
\mbox{\tiny\tt SELECT\_AX}&
\vdash \uquant{P_{\alpha\fun\ty{bool}}\ x} P\ x \imp
P({\hilbert}\ P)\\
%  \\
\mbox{\tiny\tt INFINITY\_AX}&
\vdash \equant{f_{\ind\fun \ind}} \OneOne \ f \conj \neg(\Onto \ f)\\
\end{array}
\]
\end{tiny}
\end{block}
\end{columns}
\end{footnotesize}
\end{frame}

\begin{frame}[fragile]
\frametitle{Syntax of CCS operators, constant and actions}
New types: ``\HOLinline{(\ensuremath{\alpha}, \ensuremath{\beta}) \HOLTyOp{CCS}}'', ``\HOLinline{\ensuremath{\beta} \HOLTyOp{Action}}'',
``\HOLinline{\ensuremath{\beta} \HOLTyOp{Label}}'' and ``\HOLinline{\ensuremath{\beta} \HOLTyOp{Relabeling}}''.
\begin{center}
\begin{small}
\begin{tabular}{|c|c|c|c|}
\hline
\textbf{Operator} & \textbf{CCS Notation} & \textbf{HOL term} &
                                                                \textbf{HOL (abbrev.)}\\
\hline
nil & $\textbf{0}$ & \HOLinline{\HOLConst{nil}} & \HOLinline{\HOLConst{nil}} \\
prefix & $u.P$ & \texttt{prefix u P} & \HOLinline{\HOLFreeVar{u}\HOLSymConst{..}\HOLFreeVar{P}} \\
sum & $P + Q$ & \texttt{sum P Q} & \HOLinline{\HOLFreeVar{P} \HOLSymConst{\ensuremath{+}} \HOLFreeVar{Q}} \\
parallel & $P \,\mid\, Q$ & \texttt{par P Q} & \HOLinline{\HOLFreeVar{P} \HOLSymConst{\ensuremath{\parallel}} \HOLFreeVar{Q}} \\
restriction & $(\nu\;L)\;P$ & \texttt{restr L P} & \HOLinline{\HOLSymConst{\ensuremath{\nu}} \HOLFreeVar{L} \HOLFreeVar{P}}
  \\
recursion & $\recu A P$ & \texttt{rec A P} & \HOLinline{\HOLConst{rec} \HOLFreeVar{A} \HOLFreeVar{P}}
  \\
relabeling & $P\;[rf]$ & \texttt{relab P rf} & \HOLinline{\HOLConst{relab} \HOLFreeVar{P} \HOLFreeVar{rf}}
  \\
\hline
constant & $A$ & \texttt{var A} & \HOLinline{\HOLConst{var} \HOLFreeVar{A}} \\
invisible action & $\tau$ & \texttt{tau} & \HOLinline{\HOLSymConst{\ensuremath{\tau}}} \\
input action & $a$ & \texttt{label (name a)} & \HOLinline{\HOLConst{In} \HOLFreeVar{a}} \\
output action & $\outC a$ & \texttt{label (coname a)} & \HOLinline{\HOLConst{Out} \HOLFreeVar{a}} \\
\hline
\end{tabular}
\end{small}
\end{center}
\end{frame}

\begin{frame}[fragile]
\frametitle{CCS transitions: an inductive relation}
3-ary inductive relation \texttt{TRANS}: ``\texttt{TRANS P u Q}'' or ``\HOLinline{\HOLFreeVar{P} \HOLTokenTransBegin\HOLFreeVar{u}\HOLTokenTransEnd \HOLFreeVar{Q}}''.
\begin{small}
\begin{alltt}
\HOLTokenTurnstile{} \HOLFreeVar{u}\HOLSymConst{..}\HOLFreeVar{P} \HOLTokenTransBegin\HOLFreeVar{u}\HOLTokenTransEnd \HOLFreeVar{P}\hfill\texttt{[PREFIX]}
\HOLTokenTurnstile{} \HOLFreeVar{P} \HOLTokenTransBegin\HOLFreeVar{u}\HOLTokenTransEnd \HOLFreeVar{P\sp{\prime}} \HOLSymConst{\HOLTokenImp{}} \HOLFreeVar{P} \HOLSymConst{\ensuremath{+}} \HOLFreeVar{Q} \HOLTokenTransBegin\HOLFreeVar{u}\HOLTokenTransEnd \HOLFreeVar{P\sp{\prime}}\hfill\texttt{[SUM1]}
\HOLTokenTurnstile{} \HOLFreeVar{P} \HOLTokenTransBegin\HOLFreeVar{u}\HOLTokenTransEnd \HOLFreeVar{P\sp{\prime}} \HOLSymConst{\HOLTokenImp{}} \HOLFreeVar{Q} \HOLSymConst{\ensuremath{+}} \HOLFreeVar{P} \HOLTokenTransBegin\HOLFreeVar{u}\HOLTokenTransEnd \HOLFreeVar{P\sp{\prime}}\hfill\texttt{[SUM2]}
\HOLTokenTurnstile{} \HOLFreeVar{P} \HOLTokenTransBegin\HOLFreeVar{u}\HOLTokenTransEnd \HOLFreeVar{P\sp{\prime}} \HOLSymConst{\HOLTokenImp{}} \HOLFreeVar{P} \HOLSymConst{\ensuremath{\parallel}} \HOLFreeVar{Q} \HOLTokenTransBegin\HOLFreeVar{u}\HOLTokenTransEnd \HOLFreeVar{P\sp{\prime}} \HOLSymConst{\ensuremath{\parallel}} \HOLFreeVar{Q}\hfill\texttt{[PAR1]}
\HOLTokenTurnstile{} \HOLFreeVar{P} \HOLTokenTransBegin\HOLFreeVar{u}\HOLTokenTransEnd \HOLFreeVar{P\sp{\prime}} \HOLSymConst{\HOLTokenImp{}} \HOLFreeVar{Q} \HOLSymConst{\ensuremath{\parallel}} \HOLFreeVar{P} \HOLTokenTransBegin\HOLFreeVar{u}\HOLTokenTransEnd \HOLFreeVar{Q} \HOLSymConst{\ensuremath{\parallel}} \HOLFreeVar{P\sp{\prime}}\hfill\texttt{[PAR2]}
\end{alltt}
\end{small}
\begin{scriptsize}
\begin{alltt}
\HOLTokenTurnstile{} \HOLFreeVar{P} \HOLTokenTransBegin\HOLConst{label} \HOLFreeVar{l}\HOLTokenTransEnd \HOLFreeVar{P\sp{\prime}} \HOLSymConst{\HOLTokenConj{}} \HOLFreeVar{Q} \HOLTokenTransBegin\HOLConst{label} (\HOLConst{COMPL} \HOLFreeVar{l})\HOLTokenTransEnd \HOLFreeVar{Q\sp{\prime}} \HOLSymConst{\HOLTokenImp{}} \HOLFreeVar{P} \HOLSymConst{\ensuremath{\parallel}} \HOLFreeVar{Q} \HOLTokenTransBegin\HOLSymConst{\ensuremath{\tau}}\HOLTokenTransEnd \HOLFreeVar{P\sp{\prime}} \HOLSymConst{\ensuremath{\parallel}} \HOLFreeVar{Q\sp{\prime}}\hfill\begin{small}\texttt{[PAR3]}\end{small}

\HOLTokenTurnstile{} \HOLFreeVar{P} \HOLTokenTransBegin\HOLFreeVar{u}\HOLTokenTransEnd \HOLFreeVar{Q} \HOLSymConst{\HOLTokenConj{}} ((\HOLFreeVar{u} \HOLSymConst{=} \HOLSymConst{\ensuremath{\tau}}) \HOLSymConst{\HOLTokenDisj{}} (\HOLFreeVar{u} \HOLSymConst{=} \HOLConst{label} \HOLFreeVar{l}) \HOLSymConst{\HOLTokenConj{}} \HOLFreeVar{l} \HOLSymConst{\HOLTokenNotIn{}} \HOLFreeVar{L} \HOLSymConst{\HOLTokenConj{}} \HOLConst{COMPL} \HOLFreeVar{l} \HOLSymConst{\HOLTokenNotIn{}} \HOLFreeVar{L}) \HOLSymConst{\HOLTokenImp{}}
   \HOLSymConst{\ensuremath{\nu}} \HOLFreeVar{L} \HOLFreeVar{P} \HOLTokenTransBegin\HOLFreeVar{u}\HOLTokenTransEnd \HOLSymConst{\ensuremath{\nu}} \HOLFreeVar{L} \HOLFreeVar{Q}\hfill\begin{small}\texttt{[RESTR]}\end{small}

\HOLTokenTurnstile{} \HOLFreeVar{P} \HOLTokenTransBegin\HOLFreeVar{u}\HOLTokenTransEnd \HOLFreeVar{Q} \HOLSymConst{\HOLTokenImp{}} \HOLConst{relab} \HOLFreeVar{P} \HOLFreeVar{rf} \HOLTokenTransBegin\HOLConst{relabel} \HOLFreeVar{rf} \HOLFreeVar{u}\HOLTokenTransEnd \HOLConst{relab} \HOLFreeVar{Q} \HOLFreeVar{rf}\hfill\begin{small}\texttt{[RELABELING]}\end{small}
\end{alltt}
\end{scriptsize}
\begin{small}
\begin{alltt}
\HOLTokenTurnstile{} \HOLConst{CCS_Subst} \HOLFreeVar{P} (\HOLConst{rec} \HOLFreeVar{A} \HOLFreeVar{P}) \HOLFreeVar{A} \HOLTokenTransBegin\HOLFreeVar{u}\HOLTokenTransEnd \HOLFreeVar{P\sp{\prime}} \HOLSymConst{\HOLTokenImp{}} \HOLConst{rec} \HOLFreeVar{A} \HOLFreeVar{P} \HOLTokenTransBegin\HOLFreeVar{u}\HOLTokenTransEnd \HOLFreeVar{P\sp{\prime}}\hfill\texttt{[REC]}
\end{alltt}
\end{small}
\end{frame}

\begin{frame}[fragile]
\frametitle{Relabeling and Substution}
\begin{small}
\begin{alltt}
\HOLConst{Is_Relabeling} (\HOLFreeVar{f} :\ensuremath{\beta} \HOLTyOp{Label} \HOLTokenTransEnd \ensuremath{\beta} \HOLTyOp{Label}) \HOLSymConst{\HOLTokenEquiv{}}
\HOLSymConst{\HOLTokenForall{}}(\HOLBoundVar{s} :\ensuremath{\beta}). \HOLFreeVar{f} (\HOLConst{coname} \HOLBoundVar{s}) \HOLSymConst{=} \HOLConst{COMPL} (\HOLFreeVar{f} (\HOLConst{name} \HOLBoundVar{s}))

\HOLConst{relabel} \HOLFreeVar{rf} \HOLSymConst{\ensuremath{\tau}} \HOLSymConst{=} \HOLSymConst{\ensuremath{\tau}}
\HOLConst{relabel} \HOLFreeVar{rf} (\HOLConst{label} \HOLFreeVar{l}) \HOLSymConst{=} \HOLConst{label} (\HOLConst{REP_Relabeling} \HOLFreeVar{rf} \HOLFreeVar{l})
\end{alltt}
\end{small}
\begin{scriptsize}
\begin{alltt}
\HOLConst{CCS_Subst} \HOLConst{nil} \HOLFreeVar{E\sp{\prime}} \HOLFreeVar{X} \HOLSymConst{=} \HOLConst{nil}
\HOLConst{CCS_Subst} (\HOLFreeVar{u}\HOLSymConst{..}\HOLFreeVar{E}) \HOLFreeVar{E\sp{\prime}} \HOLFreeVar{X} \HOLSymConst{=} \HOLFreeVar{u}\HOLSymConst{..}\HOLConst{CCS_Subst} \HOLFreeVar{E} \HOLFreeVar{E\sp{\prime}} \HOLFreeVar{X}
\HOLConst{CCS_Subst} (\HOLFreeVar{E\sb{\mathrm{1}}} \HOLSymConst{\ensuremath{+}} \HOLFreeVar{E\sb{\mathrm{2}}}) \HOLFreeVar{E\sp{\prime}} \HOLFreeVar{X} \HOLSymConst{=}
\HOLConst{CCS_Subst} \HOLFreeVar{E\sb{\mathrm{1}}} \HOLFreeVar{E\sp{\prime}} \HOLFreeVar{X} \HOLSymConst{\ensuremath{+}} \HOLConst{CCS_Subst} \HOLFreeVar{E\sb{\mathrm{2}}} \HOLFreeVar{E\sp{\prime}} \HOLFreeVar{X}
\HOLConst{CCS_Subst} (\HOLFreeVar{E\sb{\mathrm{1}}} \HOLSymConst{\ensuremath{\parallel}} \HOLFreeVar{E\sb{\mathrm{2}}}) \HOLFreeVar{E\sp{\prime}} \HOLFreeVar{X} \HOLSymConst{=}
\HOLConst{CCS_Subst} \HOLFreeVar{E\sb{\mathrm{1}}} \HOLFreeVar{E\sp{\prime}} \HOLFreeVar{X} \HOLSymConst{\ensuremath{\parallel}} \HOLConst{CCS_Subst} \HOLFreeVar{E\sb{\mathrm{2}}} \HOLFreeVar{E\sp{\prime}} \HOLFreeVar{X}
\HOLConst{CCS_Subst} (\HOLSymConst{\ensuremath{\nu}} \HOLFreeVar{L} \HOLFreeVar{E}) \HOLFreeVar{E\sp{\prime}} \HOLFreeVar{X} \HOLSymConst{=} \HOLSymConst{\ensuremath{\nu}} \HOLFreeVar{L} (\HOLConst{CCS_Subst} \HOLFreeVar{E} \HOLFreeVar{E\sp{\prime}} \HOLFreeVar{X})
\HOLConst{CCS_Subst} (\HOLConst{relab} \HOLFreeVar{E} \HOLFreeVar{f}) \HOLFreeVar{E\sp{\prime}} \HOLFreeVar{X} \HOLSymConst{=} \HOLConst{relab} (\HOLConst{CCS_Subst} \HOLFreeVar{E} \HOLFreeVar{E\sp{\prime}} \HOLFreeVar{X}) \HOLFreeVar{f}
\HOLConst{CCS_Subst} (\HOLConst{var} \HOLFreeVar{Y}) \HOLFreeVar{E\sp{\prime}} \HOLFreeVar{X} \HOLSymConst{=} \HOLKeyword{if} \HOLFreeVar{Y} \HOLSymConst{=} \HOLFreeVar{X} \HOLKeyword{then} \HOLFreeVar{E\sp{\prime}} \HOLKeyword{else} \HOLConst{var} \HOLFreeVar{Y}
\HOLConst{CCS_Subst} (\HOLConst{rec} \HOLFreeVar{Y} \HOLFreeVar{E}) \HOLFreeVar{E\sp{\prime}} \HOLFreeVar{X} \HOLSymConst{=}
\HOLKeyword{if} \HOLFreeVar{Y} \HOLSymConst{=} \HOLFreeVar{X} \HOLKeyword{then} \HOLConst{rec} \HOLFreeVar{Y} \HOLFreeVar{E}
\HOLKeyword{else} \HOLConst{rec} \HOLFreeVar{Y} (\HOLConst{CCS_Subst} \HOLFreeVar{E} \HOLFreeVar{E\sp{\prime}} \HOLFreeVar{X})\hfill\texttt{[CCS_Subst_def]}
\end{alltt}
\end{scriptsize}
\end{frame}

\begin{frame}[fragile]
\frametitle{Bisimulation and Bisimilarity}
\begin{definition}
\begin{small}
\begin{enumerate}
\item \begin{alltt}
\HOLinline{\HOLFreeVar{E} \HOLSymConst{\HOLTokenEPS} \HOLFreeVar{E\sp{\prime}}} (EPS E E'), \HOLConst{EPS} \HOLSymConst{=} (\HOLTokenLambda{}\HOLBoundVar{E} \HOLBoundVar{E\sp{\prime}}. \HOLBoundVar{E} \HOLTokenTransBegin\HOLSymConst{\ensuremath{\tau}}\HOLTokenTransEnd \HOLBoundVar{E\sp{\prime}})\HOLSymConst{\HOLTokenSupStar{}}
\end{alltt}
\item \begin{alltt}
\HOLFreeVar{E} \HOLTokenWeakTransBegin\HOLFreeVar{u}\HOLTokenWeakTransEnd \HOLFreeVar{E\sp{\prime}} \HOLSymConst{\HOLTokenEquiv{}} \HOLSymConst{\HOLTokenExists{}}\HOLBoundVar{E\sb{\mathrm{1}}} \HOLBoundVar{E\sb{\mathrm{2}}}. \HOLFreeVar{E} \HOLSymConst{\HOLTokenEPS} \HOLBoundVar{E\sb{\mathrm{1}}} \HOLSymConst{\HOLTokenConj{}} \HOLBoundVar{E\sb{\mathrm{1}}} \HOLTokenTransBegin\HOLFreeVar{u}\HOLTokenTransEnd \HOLBoundVar{E\sb{\mathrm{2}}} \HOLSymConst{\HOLTokenConj{}} \HOLBoundVar{E\sb{\mathrm{2}}} \HOLSymConst{\HOLTokenEPS} \HOLFreeVar{E\sp{\prime}}
\end{alltt}
\item \begin{scriptsize}\begin{alltt}
\HOLConst{WEAK_BISIM} \HOLFreeVar{Wbsm} \HOLSymConst{\HOLTokenEquiv{}}
\HOLSymConst{\HOLTokenForall{}}\HOLBoundVar{E} \HOLBoundVar{E\sp{\prime}}.
    \HOLFreeVar{Wbsm} \HOLBoundVar{E} \HOLBoundVar{E\sp{\prime}} \HOLSymConst{\HOLTokenImp{}}
    (\HOLSymConst{\HOLTokenForall{}}\HOLBoundVar{l}.
         (\HOLSymConst{\HOLTokenForall{}}\HOLBoundVar{E\sb{\mathrm{1}}}.
              \HOLBoundVar{E} \HOLTokenTransBegin\HOLConst{label} \HOLBoundVar{l}\HOLTokenTransEnd \HOLBoundVar{E\sb{\mathrm{1}}} \HOLSymConst{\HOLTokenImp{}}
              \HOLSymConst{\HOLTokenExists{}}\HOLBoundVar{E\sb{\mathrm{2}}}. \HOLBoundVar{E\sp{\prime}} \HOLTokenWeakTransBegin\HOLConst{label} \HOLBoundVar{l}\HOLTokenWeakTransEnd \HOLBoundVar{E\sb{\mathrm{2}}} \HOLSymConst{\HOLTokenConj{}} \HOLFreeVar{Wbsm} \HOLBoundVar{E\sb{\mathrm{1}}} \HOLBoundVar{E\sb{\mathrm{2}}}) \HOLSymConst{\HOLTokenConj{}}
         \HOLSymConst{\HOLTokenForall{}}\HOLBoundVar{E\sb{\mathrm{2}}}.
             \HOLBoundVar{E\sp{\prime}} \HOLTokenTransBegin\HOLConst{label} \HOLBoundVar{l}\HOLTokenTransEnd \HOLBoundVar{E\sb{\mathrm{2}}} \HOLSymConst{\HOLTokenImp{}} \HOLSymConst{\HOLTokenExists{}}\HOLBoundVar{E\sb{\mathrm{1}}}. \HOLBoundVar{E} \HOLTokenWeakTransBegin\HOLConst{label} \HOLBoundVar{l}\HOLTokenWeakTransEnd \HOLBoundVar{E\sb{\mathrm{1}}} \HOLSymConst{\HOLTokenConj{}} \HOLFreeVar{Wbsm} \HOLBoundVar{E\sb{\mathrm{1}}} \HOLBoundVar{E\sb{\mathrm{2}}}) \HOLSymConst{\HOLTokenConj{}}
    (\HOLSymConst{\HOLTokenForall{}}\HOLBoundVar{E\sb{\mathrm{1}}}. \HOLBoundVar{E} \HOLTokenTransBegin\HOLSymConst{\ensuremath{\tau}}\HOLTokenTransEnd \HOLBoundVar{E\sb{\mathrm{1}}} \HOLSymConst{\HOLTokenImp{}} \HOLSymConst{\HOLTokenExists{}}\HOLBoundVar{E\sb{\mathrm{2}}}. \HOLBoundVar{E\sp{\prime}} \HOLSymConst{\HOLTokenEPS} \HOLBoundVar{E\sb{\mathrm{2}}} \HOLSymConst{\HOLTokenConj{}} \HOLFreeVar{Wbsm} \HOLBoundVar{E\sb{\mathrm{1}}} \HOLBoundVar{E\sb{\mathrm{2}}}) \HOLSymConst{\HOLTokenConj{}}
    \HOLSymConst{\HOLTokenForall{}}\HOLBoundVar{E\sb{\mathrm{2}}}. \HOLBoundVar{E\sp{\prime}} \HOLTokenTransBegin\HOLSymConst{\ensuremath{\tau}}\HOLTokenTransEnd \HOLBoundVar{E\sb{\mathrm{2}}} \HOLSymConst{\HOLTokenImp{}} \HOLSymConst{\HOLTokenExists{}}\HOLBoundVar{E\sb{\mathrm{1}}}. \HOLBoundVar{E} \HOLSymConst{\HOLTokenEPS} \HOLBoundVar{E\sb{\mathrm{1}}} \HOLSymConst{\HOLTokenConj{}} \HOLFreeVar{Wbsm} \HOLBoundVar{E\sb{\mathrm{1}}} \HOLBoundVar{E\sb{\mathrm{2}}}
\end{alltt}\end{scriptsize}
\item \begin{alltt}
\HOLFreeVar{E} \HOLSymConst{\HOLTokenWeakEQ} \HOLFreeVar{E\sp{\prime}} \HOLSymConst{\HOLTokenEquiv{}} \HOLSymConst{\HOLTokenExists{}}\HOLBoundVar{Wbsm}. \HOLBoundVar{Wbsm} \HOLFreeVar{E} \HOLFreeVar{E\sp{\prime}} \HOLSymConst{\HOLTokenConj{}} \HOLConst{WEAK_BISIM} \HOLBoundVar{Wbsm}
\end{alltt}
\end{enumerate}
\end{small}
\end{definition}
\end{frame}

\begin{frame}
\frametitle{Bisimilarity as a fixed point}
\vspace{-2ex}
The actual definition of $\wb$ is automatically built by HOL4's
\texttt{Hol_coreln}, coinducitve relation package:
\lstinputlisting{WeakEquiv.sml}
\texttt{Hol_coreln} returns 3 theorems:
\begin{enumerate}
\item \texttt{WEAK_EQUIV_rule}: input ``rules'' proved as a theorem.
\item \texttt{WEAK_EQUIV_coind}: the resulting relation is maximal.
\item \texttt{WEAK_EQUIV_cases}: the resulting relation is a fix point.
\end{enumerate}
\end{frame}

\begin{frame}[fragile]
\frametitle{Multi-hole contexts: inductive unary relation}
\vspace{-2ex}
\begin{definition}
\begin{small}
\begin{alltt}
\HOLConst{CONTEXT} (\HOLTokenLambda{}\HOLBoundVar{t}. \HOLBoundVar{t})
\HOLConst{CONTEXT} (\HOLTokenLambda{}\HOLBoundVar{t}. \HOLFreeVar{p})
\HOLConst{CONTEXT} \HOLFreeVar{e} \HOLSymConst{\HOLTokenImp{}} \HOLConst{CONTEXT} (\HOLTokenLambda{}\HOLBoundVar{t}. \HOLFreeVar{a}\HOLSymConst{..}\HOLFreeVar{e} \HOLBoundVar{t})
\HOLConst{CONTEXT} \HOLFreeVar{e\sb{\mathrm{1}}} \HOLSymConst{\HOLTokenConj{}} \HOLConst{CONTEXT} \HOLFreeVar{e\sb{\mathrm{2}}} \HOLSymConst{\HOLTokenImp{}} \HOLConst{CONTEXT} (\HOLTokenLambda{}\HOLBoundVar{t}. \HOLFreeVar{e\sb{\mathrm{1}}} \HOLBoundVar{t} \HOLSymConst{\ensuremath{+}} \HOLFreeVar{e\sb{\mathrm{2}}} \HOLBoundVar{t})
\HOLConst{CONTEXT} \HOLFreeVar{e\sb{\mathrm{1}}} \HOLSymConst{\HOLTokenConj{}} \HOLConst{CONTEXT} \HOLFreeVar{e\sb{\mathrm{2}}} \HOLSymConst{\HOLTokenImp{}} \HOLConst{CONTEXT} (\HOLTokenLambda{}\HOLBoundVar{t}. \HOLFreeVar{e\sb{\mathrm{1}}} \HOLBoundVar{t} \HOLSymConst{\ensuremath{\parallel}} \HOLFreeVar{e\sb{\mathrm{2}}} \HOLBoundVar{t})
\HOLConst{CONTEXT} \HOLFreeVar{e} \HOLSymConst{\HOLTokenImp{}} \HOLConst{CONTEXT} (\HOLTokenLambda{}\HOLBoundVar{t}. \HOLSymConst{\ensuremath{\nu}} \HOLFreeVar{L} (\HOLFreeVar{e} \HOLBoundVar{t}))
\HOLConst{CONTEXT} \HOLFreeVar{e} \HOLSymConst{\HOLTokenImp{}} \HOLConst{CONTEXT} (\HOLTokenLambda{}\HOLBoundVar{t}. \HOLConst{relab} (\HOLFreeVar{e} \HOLBoundVar{t}) \HOLFreeVar{rf})\hfill{[CONTEXT_rules]}
\end{alltt}
The composition of two contexts is still a context:
\vspace*{-10pt}
\begin{alltt}
\HOLTokenTurnstile{} \HOLConst{CONTEXT} \HOLFreeVar{c\sb{\mathrm{1}}} \HOLSymConst{\HOLTokenConj{}} \HOLConst{CONTEXT} \HOLFreeVar{c\sb{\mathrm{2}}} \HOLSymConst{\HOLTokenImp{}} \HOLConst{CONTEXT} (\HOLFreeVar{c\sb{\mathrm{1}}} \HOLSymConst{\HOLTokenCompose} \HOLFreeVar{c\sb{\mathrm{2}}})\hfill{[CONTEXT_combin]}
\end{alltt}
\end{small}
\end{definition}
\begin{examples}
$E[X] = a.X + b.X$ is presented as ``\HOLinline{\HOLTokenLambda{}\HOLBoundVar{t}. \HOLFreeVar{a}\HOLSymConst{..}\HOLBoundVar{t} \HOLSymConst{\ensuremath{+}} \HOLFreeVar{b}\HOLSymConst{..}\HOLBoundVar{t}}'' ($a.[] + b.[]$)).
\end{examples}
\end{frame}

\begin{frame}[fragile]
\frametitle{Multi-hole contexts with only direct sums}
\vspace{-2ex}
\begin{definition}
\begin{small}
\begin{alltt}
\HOLConst{GCONTEXT} (\HOLTokenLambda{}\HOLBoundVar{t}. \HOLBoundVar{t})
\HOLConst{GCONTEXT} (\HOLTokenLambda{}\HOLBoundVar{t}. \HOLFreeVar{p})
\HOLConst{GCONTEXT} \HOLFreeVar{e} \HOLSymConst{\HOLTokenImp{}} \HOLConst{GCONTEXT} (\HOLTokenLambda{}\HOLBoundVar{t}. \HOLFreeVar{a}\HOLSymConst{..}\HOLFreeVar{e} \HOLBoundVar{t})
\HOLConst{GCONTEXT} \HOLFreeVar{e\sb{\mathrm{1}}} \HOLSymConst{\HOLTokenConj{}} \HOLConst{GCONTEXT} \HOLFreeVar{e\sb{\mathrm{2}}} \HOLSymConst{\HOLTokenImp{}} \HOLConst{GCONTEXT} (\HOLTokenLambda{}\HOLBoundVar{t}. \HOLFreeVar{a\sb{\mathrm{1}}}\HOLSymConst{..}\HOLFreeVar{e\sb{\mathrm{1}}} \HOLBoundVar{t} \HOLSymConst{\ensuremath{+}} \HOLFreeVar{a\sb{\mathrm{2}}}\HOLSymConst{..}\HOLFreeVar{e\sb{\mathrm{2}}} \HOLBoundVar{t})
\HOLConst{GCONTEXT} \HOLFreeVar{e\sb{\mathrm{1}}} \HOLSymConst{\HOLTokenConj{}} \HOLConst{GCONTEXT} \HOLFreeVar{e\sb{\mathrm{2}}} \HOLSymConst{\HOLTokenImp{}} \HOLConst{GCONTEXT} (\HOLTokenLambda{}\HOLBoundVar{t}. \HOLFreeVar{e\sb{\mathrm{1}}} \HOLBoundVar{t} \HOLSymConst{\ensuremath{\parallel}} \HOLFreeVar{e\sb{\mathrm{2}}} \HOLBoundVar{t})
\HOLConst{GCONTEXT} \HOLFreeVar{e} \HOLSymConst{\HOLTokenImp{}} \HOLConst{GCONTEXT} (\HOLTokenLambda{}\HOLBoundVar{t}. \HOLSymConst{\ensuremath{\nu}} \HOLFreeVar{L} (\HOLFreeVar{e} \HOLBoundVar{t}))
\HOLConst{GCONTEXT} \HOLFreeVar{e} \HOLSymConst{\HOLTokenImp{}} \HOLConst{GCONTEXT} (\HOLTokenLambda{}\HOLBoundVar{t}. \HOLConst{relab} (\HOLFreeVar{e} \HOLBoundVar{t}) \HOLFreeVar{rf})\hfill{[GCONTEXT_rules]}
\end{alltt}
\end{small}
\end{definition}
(\texttt{GCONTEXT} can be also seen as a normal context under special
CCS syntax with only guarded sums $\Sigma \mu_i.p_i$.)
\end{frame}

\begin{frame}[fragile]
\frametitle{Congruence and precongruence}
\begin{scriptsize}
\begin{alltt}
\HOLTokenTurnstile{} \HOLConst{PreOrder} \HOLFreeVar{R} \HOLSymConst{\HOLTokenEquiv{}} \HOLConst{reflexive} \HOLFreeVar{R} \HOLSymConst{\HOLTokenConj{}} \HOLConst{transitive} \HOLFreeVar{R}
\HOLTokenTurnstile{} \HOLConst{equivalence} \HOLFreeVar{R} \HOLSymConst{\HOLTokenEquiv{}} \HOLConst{reflexive} \HOLFreeVar{R} \HOLSymConst{\HOLTokenConj{}} \HOLConst{symmetric} \HOLFreeVar{R} \HOLSymConst{\HOLTokenConj{}} \HOLConst{transitive} \HOLFreeVar{R}
\end{alltt}
\begin{alltt}
\HOLTokenTurnstile{} \HOLConst{precongruence} \HOLFreeVar{R} \HOLSymConst{\HOLTokenEquiv{}}
   \HOLConst{PreOrder} \HOLFreeVar{R} \HOLSymConst{\HOLTokenConj{}}
   \HOLSymConst{\HOLTokenForall{}}\HOLBoundVar{x} \HOLBoundVar{y} \HOLBoundVar{ctx}. \HOLConst{CONTEXT} \HOLBoundVar{ctx} \HOLSymConst{\HOLTokenImp{}} \HOLFreeVar{R} \HOLBoundVar{x} \HOLBoundVar{y} \HOLSymConst{\HOLTokenImp{}} \HOLFreeVar{R} (\HOLBoundVar{ctx} \HOLBoundVar{x}) (\HOLBoundVar{ctx} \HOLBoundVar{y})
\HOLTokenTurnstile{} \HOLConst{precongruence1} \HOLFreeVar{R} \HOLSymConst{\HOLTokenEquiv{}}
   \HOLConst{PreOrder} \HOLFreeVar{R} \HOLSymConst{\HOLTokenConj{}}
   \HOLSymConst{\HOLTokenForall{}}\HOLBoundVar{x} \HOLBoundVar{y} \HOLBoundVar{ctx}. \HOLConst{GCONTEXT} \HOLBoundVar{ctx} \HOLSymConst{\HOLTokenImp{}} \HOLFreeVar{R} \HOLBoundVar{x} \HOLBoundVar{y} \HOLSymConst{\HOLTokenImp{}} \HOLFreeVar{R} (\HOLBoundVar{ctx} \HOLBoundVar{x}) (\HOLBoundVar{ctx} \HOLBoundVar{y})
\HOLTokenTurnstile{} \HOLConst{congruence} \HOLFreeVar{R} \HOLSymConst{\HOLTokenEquiv{}}
   \HOLConst{equivalence} \HOLFreeVar{R} \HOLSymConst{\HOLTokenConj{}}
   \HOLSymConst{\HOLTokenForall{}}\HOLBoundVar{x} \HOLBoundVar{y} \HOLBoundVar{ctx}. \HOLConst{CONTEXT} \HOLBoundVar{ctx} \HOLSymConst{\HOLTokenImp{}} \HOLFreeVar{R} \HOLBoundVar{x} \HOLBoundVar{y} \HOLSymConst{\HOLTokenImp{}} \HOLFreeVar{R} (\HOLBoundVar{ctx} \HOLBoundVar{x}) (\HOLBoundVar{ctx} \HOLBoundVar{y})
\HOLTokenTurnstile{} \HOLConst{congruence1} \HOLFreeVar{R} \HOLSymConst{\HOLTokenEquiv{}}
   \HOLConst{equivalence} \HOLFreeVar{R} \HOLSymConst{\HOLTokenConj{}}
   \HOLSymConst{\HOLTokenForall{}}\HOLBoundVar{x} \HOLBoundVar{y} \HOLBoundVar{ctx}. \HOLConst{GCONTEXT} \HOLBoundVar{ctx} \HOLSymConst{\HOLTokenImp{}} \HOLFreeVar{R} \HOLBoundVar{x} \HOLBoundVar{y} \HOLSymConst{\HOLTokenImp{}} \HOLFreeVar{R} (\HOLBoundVar{ctx} \HOLBoundVar{x}) (\HOLBoundVar{ctx} \HOLBoundVar{y})
\end{alltt}
\begin{alltt}
\HOLTokenTurnstile{} \HOLConst{congruence} \HOLConst{STRONG_EQUIV}
\HOLTokenTurnstile{} \HOLConst{congruence1} \HOLConst{WEAK_EQUIV}
\HOLTokenTurnstile{} \HOLConst{congruence} \HOLConst{OBS_CONGR}
\end{alltt}
\end{scriptsize}
\end{frame}

\begin{frame}[fragile]
\frametitle{Weakly-guarded contexts: with and without direct sums}
\begin{alltt}
\HOLConst{WG} (\HOLTokenLambda{}\HOLBoundVar{t}. \HOLFreeVar{p})
\HOLConst{CONTEXT} \HOLFreeVar{e} \HOLSymConst{\HOLTokenImp{}} \HOLConst{WG} (\HOLTokenLambda{}\HOLBoundVar{t}. \HOLFreeVar{a}\HOLSymConst{..}\HOLFreeVar{e} \HOLBoundVar{t})
\HOLConst{WG} \HOLFreeVar{e\sb{\mathrm{1}}} \HOLSymConst{\HOLTokenConj{}} \HOLConst{WG} \HOLFreeVar{e\sb{\mathrm{2}}} \HOLSymConst{\HOLTokenImp{}} \HOLConst{WG} (\HOLTokenLambda{}\HOLBoundVar{t}. \HOLFreeVar{e\sb{\mathrm{1}}} \HOLBoundVar{t} \HOLSymConst{\ensuremath{+}} \HOLFreeVar{e\sb{\mathrm{2}}} \HOLBoundVar{t})
\HOLConst{WG} \HOLFreeVar{e\sb{\mathrm{1}}} \HOLSymConst{\HOLTokenConj{}} \HOLConst{WG} \HOLFreeVar{e\sb{\mathrm{2}}} \HOLSymConst{\HOLTokenImp{}} \HOLConst{WG} (\HOLTokenLambda{}\HOLBoundVar{t}. \HOLFreeVar{e\sb{\mathrm{1}}} \HOLBoundVar{t} \HOLSymConst{\ensuremath{\parallel}} \HOLFreeVar{e\sb{\mathrm{2}}} \HOLBoundVar{t})
\HOLConst{WG} \HOLFreeVar{e} \HOLSymConst{\HOLTokenImp{}} \HOLConst{WG} (\HOLTokenLambda{}\HOLBoundVar{t}. \HOLSymConst{\ensuremath{\nu}} \HOLFreeVar{L} (\HOLFreeVar{e} \HOLBoundVar{t}))
\HOLConst{WG} \HOLFreeVar{e} \HOLSymConst{\HOLTokenImp{}} \HOLConst{WG} (\HOLTokenLambda{}\HOLBoundVar{t}. \HOLConst{relab} (\HOLFreeVar{e} \HOLBoundVar{t}) \HOLFreeVar{rf})\hfill{[WG_rules]}
\end{alltt}
\begin{alltt}
\HOLConst{WGS} (\HOLTokenLambda{}\HOLBoundVar{t}. \HOLFreeVar{p})
\HOLConst{GCONTEXT} \HOLFreeVar{e} \HOLSymConst{\HOLTokenImp{}} \HOLConst{WGS} (\HOLTokenLambda{}\HOLBoundVar{t}. \HOLFreeVar{a}\HOLSymConst{..}\HOLFreeVar{e} \HOLBoundVar{t})
\HOLConst{GCONTEXT} \HOLFreeVar{e\sb{\mathrm{1}}} \HOLSymConst{\HOLTokenConj{}} \HOLConst{GCONTEXT} \HOLFreeVar{e\sb{\mathrm{2}}} \HOLSymConst{\HOLTokenImp{}} \HOLConst{WGS} (\HOLTokenLambda{}\HOLBoundVar{t}. \HOLFreeVar{a\sb{\mathrm{1}}}\HOLSymConst{..}\HOLFreeVar{e\sb{\mathrm{1}}} \HOLBoundVar{t} \HOLSymConst{\ensuremath{+}} \HOLFreeVar{a\sb{\mathrm{2}}}\HOLSymConst{..}\HOLFreeVar{e\sb{\mathrm{2}}} \HOLBoundVar{t})
\HOLConst{WGS} \HOLFreeVar{e\sb{\mathrm{1}}} \HOLSymConst{\HOLTokenConj{}} \HOLConst{WGS} \HOLFreeVar{e\sb{\mathrm{2}}} \HOLSymConst{\HOLTokenImp{}} \HOLConst{WGS} (\HOLTokenLambda{}\HOLBoundVar{t}. \HOLFreeVar{e\sb{\mathrm{1}}} \HOLBoundVar{t} \HOLSymConst{\ensuremath{\parallel}} \HOLFreeVar{e\sb{\mathrm{2}}} \HOLBoundVar{t})
\HOLConst{WGS} \HOLFreeVar{e} \HOLSymConst{\HOLTokenImp{}} \HOLConst{WGS} (\HOLTokenLambda{}\HOLBoundVar{t}. \HOLSymConst{\ensuremath{\nu}} \HOLFreeVar{L} (\HOLFreeVar{e} \HOLBoundVar{t}))
\HOLConst{WGS} \HOLFreeVar{e} \HOLSymConst{\HOLTokenImp{}} \HOLConst{WGS} (\HOLTokenLambda{}\HOLBoundVar{t}. \HOLConst{relab} (\HOLFreeVar{e} \HOLBoundVar{t}) \HOLFreeVar{rf})\hfill{[WGS_rules]}
\end{alltt}
\end{frame}

\begin{frame}[fragile]
\frametitle{(Strongly) guarded contexts}
\begin{alltt}
\HOLConst{SG} (\HOLTokenLambda{}\HOLBoundVar{t}. \HOLFreeVar{p})
\HOLConst{CONTEXT} \HOLFreeVar{e} \HOLSymConst{\HOLTokenImp{}} \HOLConst{SG} (\HOLTokenLambda{}\HOLBoundVar{t}. \HOLConst{label} \HOLFreeVar{l}\HOLSymConst{..}\HOLFreeVar{e} \HOLBoundVar{t})
\HOLConst{SG} \HOLFreeVar{e} \HOLSymConst{\HOLTokenImp{}} \HOLConst{SG} (\HOLTokenLambda{}\HOLBoundVar{t}. \HOLFreeVar{a}\HOLSymConst{..}\HOLFreeVar{e} \HOLBoundVar{t})
\HOLConst{SG} \HOLFreeVar{e\sb{\mathrm{1}}} \HOLSymConst{\HOLTokenConj{}} \HOLConst{SG} \HOLFreeVar{e\sb{\mathrm{2}}} \HOLSymConst{\HOLTokenImp{}} \HOLConst{SG} (\HOLTokenLambda{}\HOLBoundVar{t}. \HOLFreeVar{e\sb{\mathrm{1}}} \HOLBoundVar{t} \HOLSymConst{\ensuremath{+}} \HOLFreeVar{e\sb{\mathrm{2}}} \HOLBoundVar{t})
\HOLConst{SG} \HOLFreeVar{e\sb{\mathrm{1}}} \HOLSymConst{\HOLTokenConj{}} \HOLConst{SG} \HOLFreeVar{e\sb{\mathrm{2}}} \HOLSymConst{\HOLTokenImp{}} \HOLConst{SG} (\HOLTokenLambda{}\HOLBoundVar{t}. \HOLFreeVar{e\sb{\mathrm{1}}} \HOLBoundVar{t} \HOLSymConst{\ensuremath{\parallel}} \HOLFreeVar{e\sb{\mathrm{2}}} \HOLBoundVar{t})
\HOLConst{SG} \HOLFreeVar{e} \HOLSymConst{\HOLTokenImp{}} \HOLConst{SG} (\HOLTokenLambda{}\HOLBoundVar{t}. \HOLSymConst{\ensuremath{\nu}} \HOLFreeVar{L} (\HOLFreeVar{e} \HOLBoundVar{t}))
\HOLConst{SG} \HOLFreeVar{e} \HOLSymConst{\HOLTokenImp{}} \HOLConst{SG} (\HOLTokenLambda{}\HOLBoundVar{t}. \HOLConst{relab} (\HOLFreeVar{e} \HOLBoundVar{t}) \HOLFreeVar{rf})\hfill{[SG_rules]}
\end{alltt}
There's no need to define special version of \texttt{SG} with only
guarded sums, as ``guarded and sequential'' always appears together
(but a single relation defnition is too complex).
\end{frame}

\begin{frame}[fragile]
\frametitle{Sequential contexts: with and without direct sums}
``$X$ is \emph{sequential} in $E$ if every subexpression of $E$ which
contains $X$, apart from $X$ itself, is of the form $\mu.F$ or
$\Sigma \tilde{F}$.''
\begin{alltt}
\HOLConst{SEQ} (\HOLTokenLambda{}\HOLBoundVar{t}. \HOLBoundVar{t})
\HOLConst{SEQ} (\HOLTokenLambda{}\HOLBoundVar{t}. \HOLFreeVar{p})
\HOLConst{SEQ} \HOLFreeVar{e} \HOLSymConst{\HOLTokenImp{}} \HOLConst{SEQ} (\HOLTokenLambda{}\HOLBoundVar{t}. \HOLFreeVar{a}\HOLSymConst{..}\HOLFreeVar{e} \HOLBoundVar{t})
\HOLConst{SEQ} \HOLFreeVar{e\sb{\mathrm{1}}} \HOLSymConst{\HOLTokenConj{}} \HOLConst{SEQ} \HOLFreeVar{e\sb{\mathrm{2}}} \HOLSymConst{\HOLTokenImp{}} \HOLConst{SEQ} (\HOLTokenLambda{}\HOLBoundVar{t}. \HOLFreeVar{e\sb{\mathrm{1}}} \HOLBoundVar{t} \HOLSymConst{\ensuremath{+}} \HOLFreeVar{e\sb{\mathrm{2}}} \HOLBoundVar{t})\hfill{[SEQ_rules]}
\end{alltt}
\begin{alltt}
\HOLConst{GSEQ} (\HOLTokenLambda{}\HOLBoundVar{t}. \HOLBoundVar{t})
\HOLConst{GSEQ} (\HOLTokenLambda{}\HOLBoundVar{t}. \HOLFreeVar{p})
\HOLConst{GSEQ} \HOLFreeVar{e} \HOLSymConst{\HOLTokenImp{}} \HOLConst{GSEQ} (\HOLTokenLambda{}\HOLBoundVar{t}. \HOLFreeVar{a}\HOLSymConst{..}\HOLFreeVar{e} \HOLBoundVar{t})
\HOLConst{GSEQ} \HOLFreeVar{e\sb{\mathrm{1}}} \HOLSymConst{\HOLTokenConj{}} \HOLConst{GSEQ} \HOLFreeVar{e\sb{\mathrm{2}}} \HOLSymConst{\HOLTokenImp{}} \HOLConst{GSEQ} (\HOLTokenLambda{}\HOLBoundVar{t}. \HOLFreeVar{a\sb{\mathrm{1}}}\HOLSymConst{..}\HOLFreeVar{e\sb{\mathrm{1}}} \HOLBoundVar{t} \HOLSymConst{\ensuremath{+}} \HOLFreeVar{a\sb{\mathrm{2}}}\HOLSymConst{..}\HOLFreeVar{e\sb{\mathrm{2}}} \HOLBoundVar{t})
\hfill{[GSEQ_rules]}
\end{alltt}
\end{frame}

\begin{frame}[fragile]
\frametitle{Contraction: formal definition}
\begin{scriptsize}
\begin{alltt}
\HOLTokenTurnstile{} \HOLConst{CONTRACTION} \HOLFreeVar{Con} \HOLSymConst{\HOLTokenEquiv{}}
   \HOLSymConst{\HOLTokenForall{}}\HOLBoundVar{E} \HOLBoundVar{E\sp{\prime}}.
       \HOLFreeVar{Con} \HOLBoundVar{E} \HOLBoundVar{E\sp{\prime}} \HOLSymConst{\HOLTokenImp{}}
       (\HOLSymConst{\HOLTokenForall{}}\HOLBoundVar{l}.
            (\HOLSymConst{\HOLTokenForall{}}\HOLBoundVar{E\sb{\mathrm{1}}}.
                 \HOLBoundVar{E} \HOLTokenTransBegin\HOLConst{label} \HOLBoundVar{l}\HOLTokenTransEnd \HOLBoundVar{E\sb{\mathrm{1}}} \HOLSymConst{\HOLTokenImp{}}
                 \HOLSymConst{\HOLTokenExists{}}\HOLBoundVar{E\sb{\mathrm{2}}}. \HOLBoundVar{E\sp{\prime}} \HOLTokenTransBegin\HOLConst{label} \HOLBoundVar{l}\HOLTokenTransEnd \HOLBoundVar{E\sb{\mathrm{2}}} \HOLSymConst{\HOLTokenConj{}} \HOLFreeVar{Con} \HOLBoundVar{E\sb{\mathrm{1}}} \HOLBoundVar{E\sb{\mathrm{2}}}) \HOLSymConst{\HOLTokenConj{}}
            \HOLSymConst{\HOLTokenForall{}}\HOLBoundVar{E\sb{\mathrm{2}}}.
                \HOLBoundVar{E\sp{\prime}} \HOLTokenTransBegin\HOLConst{label} \HOLBoundVar{l}\HOLTokenTransEnd \HOLBoundVar{E\sb{\mathrm{2}}} \HOLSymConst{\HOLTokenImp{}} \HOLSymConst{\HOLTokenExists{}}\HOLBoundVar{E\sb{\mathrm{1}}}. \HOLBoundVar{E} \HOLTokenWeakTransBegin\HOLConst{label} \HOLBoundVar{l}\HOLTokenWeakTransEnd \HOLBoundVar{E\sb{\mathrm{1}}} \HOLSymConst{\HOLTokenConj{}} \HOLBoundVar{E\sb{\mathrm{1}}} \HOLSymConst{\HOLTokenWeakEQ} \HOLBoundVar{E\sb{\mathrm{2}}}) \HOLSymConst{\HOLTokenConj{}}
       (\HOLSymConst{\HOLTokenForall{}}\HOLBoundVar{E\sb{\mathrm{1}}}.
            \HOLBoundVar{E} \HOLTokenTransBegin\HOLSymConst{\ensuremath{\tau}}\HOLTokenTransEnd \HOLBoundVar{E\sb{\mathrm{1}}} \HOLSymConst{\HOLTokenImp{}} \HOLFreeVar{Con} \HOLBoundVar{E\sb{\mathrm{1}}} \HOLBoundVar{E\sp{\prime}} \HOLSymConst{\HOLTokenDisj{}} \HOLSymConst{\HOLTokenExists{}}\HOLBoundVar{E\sb{\mathrm{2}}}. \HOLBoundVar{E\sp{\prime}} \HOLTokenTransBegin\HOLSymConst{\ensuremath{\tau}}\HOLTokenTransEnd \HOLBoundVar{E\sb{\mathrm{2}}} \HOLSymConst{\HOLTokenConj{}} \HOLFreeVar{Con} \HOLBoundVar{E\sb{\mathrm{1}}} \HOLBoundVar{E\sb{\mathrm{2}}}) \HOLSymConst{\HOLTokenConj{}}
       \HOLSymConst{\HOLTokenForall{}}\HOLBoundVar{E\sb{\mathrm{2}}}. \HOLBoundVar{E\sp{\prime}} \HOLTokenTransBegin\HOLSymConst{\ensuremath{\tau}}\HOLTokenTransEnd \HOLBoundVar{E\sb{\mathrm{2}}} \HOLSymConst{\HOLTokenImp{}} \HOLSymConst{\HOLTokenExists{}}\HOLBoundVar{E\sb{\mathrm{1}}}. \HOLBoundVar{E} \HOLSymConst{\HOLTokenEPS} \HOLBoundVar{E\sb{\mathrm{1}}} \HOLSymConst{\HOLTokenConj{}} \HOLBoundVar{E\sb{\mathrm{1}}} \HOLSymConst{\HOLTokenWeakEQ} \HOLBoundVar{E\sb{\mathrm{2}}}\hfill{[CONTRACTION]}
\end{alltt}

\begin{alltt}
\HOLTokenTurnstile{} \HOLFreeVar{P} \HOLSymConst{\HOLTokenContracts{}} \HOLFreeVar{Q} \HOLSymConst{\HOLTokenEquiv{}} \HOLSymConst{\HOLTokenExists{}}\HOLBoundVar{Con}. \HOLBoundVar{Con} \HOLFreeVar{P} \HOLFreeVar{Q} \HOLSymConst{\HOLTokenConj{}} \HOLConst{CONTRACTION} \HOLBoundVar{Con}\hfill{[contracts_thm]}
\HOLTokenTurnstile{} \HOLFreeVar{P} \HOLSymConst{\HOLTokenContracts{}} \HOLFreeVar{Q} \HOLSymConst{\HOLTokenImp{}} \HOLFreeVar{P} \HOLSymConst{\HOLTokenWeakEQ} \HOLFreeVar{Q}\hfill{[contracts_IMP_WEAK_EQUIV]}
\end{alltt}

$\mcontrBIS$ is preorder and precongruence:
\begin{alltt}
\HOLTokenTurnstile{} \HOLConst{PreOrder} (\HOLSymConst{contracts})\hfill{[contracts_PreOrder]}
\HOLTokenTurnstile{} \HOLConst{precongruence1} (\HOLSymConst{contracts})\hfill{[contracts_precongruence]}
\end{alltt}
\end{scriptsize}
\end{frame}

\begin{frame}[fragile]
\frametitle{Rooted contraction: formal definition}
\begin{small}
\begin{alltt}
\HOLTokenTurnstile{} \HOLFreeVar{E} \HOLSymConst{\HOLTokenObsContracts} \HOLFreeVar{E\sp{\prime}} \HOLSymConst{\HOLTokenEquiv{}}
   \HOLSymConst{\HOLTokenForall{}}\HOLBoundVar{u}.
       (\HOLSymConst{\HOLTokenForall{}}\HOLBoundVar{E\sb{\mathrm{1}}}. \HOLFreeVar{E} \HOLTokenTransBegin\HOLBoundVar{u}\HOLTokenTransEnd \HOLBoundVar{E\sb{\mathrm{1}}} \HOLSymConst{\HOLTokenImp{}} \HOLSymConst{\HOLTokenExists{}}\HOLBoundVar{E\sb{\mathrm{2}}}. \HOLFreeVar{E\sp{\prime}} \HOLTokenTransBegin\HOLBoundVar{u}\HOLTokenTransEnd \HOLBoundVar{E\sb{\mathrm{2}}} \HOLSymConst{\HOLTokenConj{}} \HOLBoundVar{E\sb{\mathrm{1}}} \HOLSymConst{\HOLTokenContracts{}} \HOLBoundVar{E\sb{\mathrm{2}}}) \HOLSymConst{\HOLTokenConj{}}
       \HOLSymConst{\HOLTokenForall{}}\HOLBoundVar{E\sb{\mathrm{2}}}. \HOLFreeVar{E\sp{\prime}} \HOLTokenTransBegin\HOLBoundVar{u}\HOLTokenTransEnd \HOLBoundVar{E\sb{\mathrm{2}}} \HOLSymConst{\HOLTokenImp{}} \HOLSymConst{\HOLTokenExists{}}\HOLBoundVar{E\sb{\mathrm{1}}}. \HOLFreeVar{E} \HOLTokenWeakTransBegin\HOLBoundVar{u}\HOLTokenWeakTransEnd \HOLBoundVar{E\sb{\mathrm{1}}} \HOLSymConst{\HOLTokenConj{}} \HOLBoundVar{E\sb{\mathrm{1}}} \HOLSymConst{\HOLTokenWeakEQ} \HOLBoundVar{E\sb{\mathrm{2}}}
\hfill{[OBS_contracts]}
\HOLTokenTurnstile{} \HOLFreeVar{E} \HOLSymConst{\HOLTokenObsContracts} \HOLFreeVar{E\sp{\prime}} \HOLSymConst{\HOLTokenImp{}} \HOLFreeVar{E} \HOLSymConst{\HOLTokenContracts{}} \HOLFreeVar{E\sp{\prime}}\hfill{[OBS_contracts_IMP_contracts]}
\HOLTokenTurnstile{} \HOLConst{PreOrder} \HOLConst{OBS_contracts}\hfill{[OBS_contracts_PreOrder]}
\HOLTokenTurnstile{} \HOLConst{precongruence} \HOLConst{OBS_contracts}\hfill{[OBS_contracts_precongruence]}
\end{alltt}

Inspired by the definition rooted bisimilarity (not recursive, built
upon non-rooted relation), with candidates quickly checked by
theorem prover on its transitivity)
\begin{alltt}
\HOLTokenTurnstile{} \HOLFreeVar{E} \HOLSymConst{\HOLTokenObsCongr} \HOLFreeVar{E\sp{\prime}} \HOLSymConst{\HOLTokenEquiv{}}
   \HOLSymConst{\HOLTokenForall{}}\HOLBoundVar{u}.
       (\HOLSymConst{\HOLTokenForall{}}\HOLBoundVar{E\sb{\mathrm{1}}}. \HOLFreeVar{E} \HOLTokenTransBegin\HOLBoundVar{u}\HOLTokenTransEnd \HOLBoundVar{E\sb{\mathrm{1}}} \HOLSymConst{\HOLTokenImp{}} \HOLSymConst{\HOLTokenExists{}}\HOLBoundVar{E\sb{\mathrm{2}}}. \HOLFreeVar{E\sp{\prime}} \HOLTokenWeakTransBegin\HOLBoundVar{u}\HOLTokenWeakTransEnd \HOLBoundVar{E\sb{\mathrm{2}}} \HOLSymConst{\HOLTokenConj{}} \HOLBoundVar{E\sb{\mathrm{1}}} \HOLSymConst{\HOLTokenWeakEQ} \HOLBoundVar{E\sb{\mathrm{2}}}) \HOLSymConst{\HOLTokenConj{}}
       \HOLSymConst{\HOLTokenForall{}}\HOLBoundVar{E\sb{\mathrm{2}}}. \HOLFreeVar{E\sp{\prime}} \HOLTokenTransBegin\HOLBoundVar{u}\HOLTokenTransEnd \HOLBoundVar{E\sb{\mathrm{2}}} \HOLSymConst{\HOLTokenImp{}} \HOLSymConst{\HOLTokenExists{}}\HOLBoundVar{E\sb{\mathrm{1}}}. \HOLFreeVar{E} \HOLTokenWeakTransBegin\HOLBoundVar{u}\HOLTokenWeakTransEnd \HOLBoundVar{E\sb{\mathrm{1}}} \HOLSymConst{\HOLTokenConj{}} \HOLBoundVar{E\sb{\mathrm{1}}} \HOLSymConst{\HOLTokenWeakEQ} \HOLBoundVar{E\sb{\mathrm{2}}}
\end{alltt}
\end{small}
\end{frame}

\begin{frame}
\frametitle{All five unique solution theorems}
\begin{small}
\begin{enumerate}
\item \texttt{STRONG_UNIQUE_SOLUTION:}
\begin{alltt}
\HOLTokenTurnstile{} \HOLConst{WG} \HOLFreeVar{E} \HOLSymConst{\HOLTokenImp{}} \HOLSymConst{\HOLTokenForall{}}\HOLBoundVar{P} \HOLBoundVar{Q}. \HOLBoundVar{P} \HOLSymConst{\HOLTokenStrongEQ} \HOLFreeVar{E} \HOLBoundVar{P} \HOLSymConst{\HOLTokenConj{}} \HOLBoundVar{Q} \HOLSymConst{\HOLTokenStrongEQ} \HOLFreeVar{E} \HOLBoundVar{Q} \HOLSymConst{\HOLTokenImp{}} \HOLBoundVar{P} \HOLSymConst{\HOLTokenStrongEQ} \HOLBoundVar{Q}
\end{alltt}
\item \texttt{WEAK_UNIQUE_SOLUTION:}
\begin{alltt}
\HOLTokenTurnstile{} \HOLConst{SG} \HOLFreeVar{E} \HOLSymConst{\HOLTokenConj{}} \HOLConst{GSEQ} \HOLFreeVar{E} \HOLSymConst{\HOLTokenImp{}} \HOLSymConst{\HOLTokenForall{}}\HOLBoundVar{P} \HOLBoundVar{Q}. \HOLBoundVar{P} \HOLSymConst{\HOLTokenWeakEQ} \HOLFreeVar{E} \HOLBoundVar{P} \HOLSymConst{\HOLTokenConj{}} \HOLBoundVar{Q} \HOLSymConst{\HOLTokenWeakEQ} \HOLFreeVar{E} \HOLBoundVar{Q} \HOLSymConst{\HOLTokenImp{}} \HOLBoundVar{P} \HOLSymConst{\HOLTokenWeakEQ} \HOLBoundVar{Q}
\end{alltt}
\item \texttt{OBS_UNIQUE_SOLUTION:}
\begin{alltt}
\HOLTokenTurnstile{} \HOLConst{SG} \HOLFreeVar{E} \HOLSymConst{\HOLTokenConj{}} \HOLConst{SEQ} \HOLFreeVar{E} \HOLSymConst{\HOLTokenImp{}} \HOLSymConst{\HOLTokenForall{}}\HOLBoundVar{P} \HOLBoundVar{Q}. \HOLBoundVar{P} \HOLSymConst{\HOLTokenObsCongr} \HOLFreeVar{E} \HOLBoundVar{P} \HOLSymConst{\HOLTokenConj{}} \HOLBoundVar{Q} \HOLSymConst{\HOLTokenObsCongr} \HOLFreeVar{E} \HOLBoundVar{Q} \HOLSymConst{\HOLTokenImp{}} \HOLBoundVar{P} \HOLSymConst{\HOLTokenObsCongr} \HOLBoundVar{Q}
\end{alltt}
\item \texttt{UNIQUE_SOLUTION_OF_CONTRACTIONS:}
\begin{alltt}
\HOLTokenTurnstile{} \HOLConst{WGS} \HOLFreeVar{E} \HOLSymConst{\HOLTokenImp{}} \HOLSymConst{\HOLTokenForall{}}\HOLBoundVar{P} \HOLBoundVar{Q}. \HOLBoundVar{P} \HOLSymConst{\HOLTokenContracts{}} \HOLFreeVar{E} \HOLBoundVar{P} \HOLSymConst{\HOLTokenConj{}} \HOLBoundVar{Q} \HOLSymConst{\HOLTokenContracts{}} \HOLFreeVar{E} \HOLBoundVar{Q} \HOLSymConst{\HOLTokenImp{}} \HOLBoundVar{P} \HOLSymConst{\HOLTokenWeakEQ} \HOLBoundVar{Q}
\end{alltt}
\item \texttt{UNIQUE_SOLUTION_OF_ROOTED_CONTRACTIONS:}
\begin{alltt}
\HOLTokenTurnstile{} \HOLConst{WG} \HOLFreeVar{E} \HOLSymConst{\HOLTokenImp{}} \HOLSymConst{\HOLTokenForall{}}\HOLBoundVar{P} \HOLBoundVar{Q}. \HOLBoundVar{P} \HOLSymConst{\HOLTokenObsContracts} \HOLFreeVar{E} \HOLBoundVar{P} \HOLSymConst{\HOLTokenConj{}} \HOLBoundVar{Q} \HOLSymConst{\HOLTokenObsContracts} \HOLFreeVar{E} \HOLBoundVar{Q} \HOLSymConst{\HOLTokenImp{}} \HOLBoundVar{P} \HOLSymConst{\HOLTokenObsCongr} \HOLBoundVar{Q}
\end{alltt}
\end{enumerate}
\end{small}
\end{frame}

\begin{frame}
\frametitle{Coarsest congruence contained in $\wb$}
\begin{theorem}
Assumping $p$ and $q$ do not use all labels, i.e. $\mathrm{fn}(p) \cup
\mathrm{fn}(q) \neq \mathscr{L}$,
\begin{equation*}
p\ \rapprox\ \! q\ \Longleftrightarrow\ ( \forall r.\ p\ +\
r\ \approx\ q\ +\ r )\enspace.
\end{equation*}
\end{theorem}
Our formalised version (with slightly weaker assumptions):
\begin{small}
\begin{alltt}
\HOLTokenTurnstile{} \HOLConst{free_action} \HOLFreeVar{p} \HOLSymConst{\HOLTokenConj{}} \HOLConst{free_action} \HOLFreeVar{q} \HOLSymConst{\HOLTokenImp{}}
   (\HOLFreeVar{p} \HOLSymConst{\HOLTokenObsCongr} \HOLFreeVar{q} \HOLSymConst{\HOLTokenEquiv{}} \HOLSymConst{\HOLTokenForall{}}\HOLBoundVar{r}. \HOLFreeVar{p} \HOLSymConst{\ensuremath{+}} \HOLBoundVar{r} \HOLSymConst{\HOLTokenWeakEQ} \HOLFreeVar{q} \HOLSymConst{\ensuremath{+}} \HOLBoundVar{r})\hfill{[COARSEST_CONGR_THM]}
\end{alltt}
where
\begin{alltt}
\HOLConst{free_action} \HOLFreeVar{p} \HOLSymConst{\HOLTokenEquiv{}} \HOLSymConst{\HOLTokenExists{}}\HOLBoundVar{a}. \HOLSymConst{\HOLTokenForall{}}\HOLBoundVar{p\sp{\prime}}. \HOLSymConst{\HOLTokenNeg{}}(\HOLFreeVar{p} \HOLTokenWeakTransBegin\HOLConst{label} \HOLBoundVar{a}\HOLTokenWeakTransEnd \HOLBoundVar{p\sp{\prime}})\hfill{[free_action_def]}
\end{alltt}
\end{small}
\end{frame}

\begin{frame}
\frametitle{Coarsest congruence contained in $\wb$ (van Glabbeek's method)}
\begin{scriptsize}
\begin{lemma}
Given processes $p$ and $q$, if there's a special process $k(p,q)$,
then the hard part ($\leftarrow$) of ``coarsest congruence'' theorem
holds without classic assumption.
\begin{alltt}
\HOLTokenTurnstile{} \HOLSymConst{\HOLTokenForall{}}\HOLBoundVar{p} \HOLBoundVar{q}.
       (\HOLSymConst{\HOLTokenExists{}}\HOLBoundVar{k}.
            \HOLConst{STABLE} \HOLBoundVar{k} \HOLSymConst{\HOLTokenConj{}} (\HOLSymConst{\HOLTokenForall{}}\HOLBoundVar{p\sp{\prime}} \HOLBoundVar{u}. \HOLBoundVar{p} \HOLTokenWeakTransBegin\HOLBoundVar{u}\HOLTokenWeakTransEnd \HOLBoundVar{p\sp{\prime}} \HOLSymConst{\HOLTokenImp{}} \HOLSymConst{\HOLTokenNeg{}}(\HOLBoundVar{p\sp{\prime}} \HOLSymConst{\HOLTokenWeakEQ} \HOLBoundVar{k})) \HOLSymConst{\HOLTokenConj{}}
            \HOLSymConst{\HOLTokenForall{}}\HOLBoundVar{q\sp{\prime}} \HOLBoundVar{u}. \HOLBoundVar{q} \HOLTokenWeakTransBegin\HOLBoundVar{u}\HOLTokenWeakTransEnd \HOLBoundVar{q\sp{\prime}} \HOLSymConst{\HOLTokenImp{}} \HOLSymConst{\HOLTokenNeg{}}(\HOLBoundVar{q\sp{\prime}} \HOLSymConst{\HOLTokenWeakEQ} \HOLBoundVar{k})) \HOLSymConst{\HOLTokenImp{}}
       (\HOLSymConst{\HOLTokenForall{}}\HOLBoundVar{r}. \HOLBoundVar{p} \HOLSymConst{\ensuremath{+}} \HOLBoundVar{r} \HOLSymConst{\HOLTokenWeakEQ} \HOLBoundVar{q} \HOLSymConst{\ensuremath{+}} \HOLBoundVar{r}) \HOLSymConst{\HOLTokenImp{}}
       \HOLBoundVar{p} \HOLSymConst{\HOLTokenObsCongr} \HOLBoundVar{q}\hfill{[PROP3_COMMON]}
\end{alltt}
\begin{alltt}
\HOLConst{STABLE} \HOLFreeVar{E} \HOLSymConst{\HOLTokenEquiv{}} \HOLSymConst{\HOLTokenForall{}}\HOLBoundVar{u} \HOLBoundVar{E\sp{\prime}}. \HOLFreeVar{E} \HOLTokenTransBegin\HOLBoundVar{u}\HOLTokenTransEnd \HOLBoundVar{E\sp{\prime}} \HOLSymConst{\HOLTokenImp{}} \HOLBoundVar{u} \HOLSymConst{\HOLTokenNotEqual{}} \HOLSymConst{\ensuremath{\tau}}\hfill{[STABLE]}
\end{alltt}
\end{lemma}
\begin{definition}[Arbitrary non-bisimir processes (Klop) - finite version]
\begin{alltt}
KLOP_def:\\
\HOLTokenTurnstile{} (\HOLSymConst{\HOLTokenForall{}}\HOLBoundVar{a}. \HOLConst{KLOP} \HOLBoundVar{a} \HOLNumLit{0} \HOLSymConst{=} \HOLConst{nil}) \HOLSymConst{\HOLTokenConj{}}
   \HOLSymConst{\HOLTokenForall{}}\HOLBoundVar{a} \HOLBoundVar{n}. \HOLConst{KLOP} \HOLBoundVar{a} (\HOLConst{SUC} \HOLBoundVar{n}) \HOLSymConst{=} \HOLConst{KLOP} \HOLBoundVar{a} \HOLBoundVar{n} \HOLSymConst{\ensuremath{+}} \HOLConst{label} \HOLBoundVar{a}\HOLSymConst{..}\HOLConst{KLOP} \HOLBoundVar{a} \HOLBoundVar{n}

KLOP_PROP2':\\
\HOLTokenTurnstile{} \HOLFreeVar{m} \HOLSymConst{\HOLTokenLt{}} \HOLFreeVar{n} \HOLSymConst{\HOLTokenImp{}} \HOLSymConst{\HOLTokenNeg{}}(\HOLConst{KLOP} \HOLFreeVar{a} \HOLFreeVar{m} \HOLSymConst{\HOLTokenWeakEQ} \HOLConst{KLOP} \HOLFreeVar{a} \HOLFreeVar{n})
\end{alltt}
\end{definition}
\begin{lemma}
\begin{alltt}
\HOLTokenTurnstile{} \HOLConst{finite_state} \HOLFreeVar{p} \HOLSymConst{\HOLTokenConj{}} \HOLConst{finite_state} \HOLFreeVar{q} \HOLSymConst{\HOLTokenImp{}}
   \HOLSymConst{\HOLTokenExists{}}\HOLBoundVar{k}.
       \HOLConst{STABLE} \HOLBoundVar{k} \HOLSymConst{\HOLTokenConj{}} (\HOLSymConst{\HOLTokenForall{}}\HOLBoundVar{p\sp{\prime}} \HOLBoundVar{u}. \HOLFreeVar{p} \HOLTokenWeakTransBegin\HOLBoundVar{u}\HOLTokenWeakTransEnd \HOLBoundVar{p\sp{\prime}} \HOLSymConst{\HOLTokenImp{}} \HOLSymConst{\HOLTokenNeg{}}(\HOLBoundVar{p\sp{\prime}} \HOLSymConst{\HOLTokenWeakEQ} \HOLBoundVar{k})) \HOLSymConst{\HOLTokenConj{}}
       \HOLSymConst{\HOLTokenForall{}}\HOLBoundVar{q\sp{\prime}} \HOLBoundVar{u}. \HOLFreeVar{q} \HOLTokenWeakTransBegin\HOLBoundVar{u}\HOLTokenWeakTransEnd \HOLBoundVar{q\sp{\prime}} \HOLSymConst{\HOLTokenImp{}} \HOLSymConst{\HOLTokenNeg{}}(\HOLBoundVar{q\sp{\prime}} \HOLSymConst{\HOLTokenWeakEQ} \HOLBoundVar{k})\hfill{[KLOP_LEMMA_FINITE]}
\end{alltt}
\end{lemma}
\end{scriptsize}
\end{frame}

\begin{frame}
\frametitle{Coarsest precongruence contained in $\rcontr$}
\begin{theorem}
\begin{alltt}
\HOLTokenTurnstile{} \HOLConst{free_action} \HOLFreeVar{p} \HOLSymConst{\HOLTokenConj{}} \HOLConst{free_action} \HOLFreeVar{q} \HOLSymConst{\HOLTokenImp{}}
   (\HOLFreeVar{p} \HOLSymConst{\HOLTokenObsContracts} \HOLFreeVar{q} \HOLSymConst{\HOLTokenEquiv{}} \HOLSymConst{\HOLTokenForall{}}\HOLBoundVar{r}. \HOLFreeVar{p} \HOLSymConst{\ensuremath{+}} \HOLBoundVar{r} \HOLSymConst{\HOLTokenContracts{}} \HOLFreeVar{q} \HOLSymConst{\ensuremath{+}} \HOLBoundVar{r})\hfill{[COARSEST_PRECONGR_THM']}
\end{alltt}
Thus the current definition of $\rcontr$ is the best possible one.
\end{theorem}

Another version following van Glabbeek's proof:
\begin{alltt}
\HOLTokenTurnstile{} \HOLConst{finite_state} \HOLFreeVar{p} \HOLSymConst{\HOLTokenConj{}} \HOLConst{finite_state} \HOLFreeVar{q} \HOLSymConst{\HOLTokenImp{}}
   (\HOLFreeVar{p} \HOLSymConst{\HOLTokenObsContracts} \HOLFreeVar{q} \HOLSymConst{\HOLTokenEquiv{}} \HOLSymConst{\HOLTokenForall{}}\HOLBoundVar{r}. \HOLFreeVar{p} \HOLSymConst{\ensuremath{+}} \HOLBoundVar{r} \HOLSymConst{\HOLTokenContracts{}} \HOLFreeVar{q} \HOLSymConst{\ensuremath{+}} \HOLBoundVar{r})\hfill{[COARSEST_PRECONGR_FINITE]}
\end{alltt}
\begin{small}
\begin{alltt}
\HOLConst{finite_state} \HOLFreeVar{p} \HOLSymConst{\HOLTokenEquiv{}} \HOLConst{FINITE} (\HOLConst{NODES} \HOLFreeVar{p})\hfill{[finite_state_def]}

\HOLConst{NODES} \HOLFreeVar{p} \HOLSymConst{=} \HOLTokenLeftbrace{}\HOLBoundVar{q} \HOLTokenBar{} \HOLConst{Reach} \HOLFreeVar{p} \HOLBoundVar{q}\HOLTokenRightbrace{}\hfill{[NODES_def]}

\HOLConst{Reach} \HOLSymConst{=} (\HOLTokenLambda{}\HOLBoundVar{E} \HOLBoundVar{E\sp{\prime}}. \HOLSymConst{\HOLTokenExists{}}\HOLBoundVar{u}. \HOLBoundVar{E} \HOLTokenTransBegin\HOLBoundVar{u}\HOLTokenTransEnd \HOLBoundVar{E\sp{\prime}})\HOLSymConst{\HOLTokenSupStar{}}\hfill{[Reach_def]}
\end{alltt}
\end{small}
\end{frame}

\begin{frame}
\frametitle{Toward multi-variable equations}
\begin{small}
The multi-variable case is a ``routine'' adaptation in informal
proofs; theorems on multi-variable equation cannot be proved
by their single-variable version.

Current idea: reuse the guardedness definition of single-variable equation
(i.e. multi-hole contexts)
\begin{examples}
\vspace{-3ex}
\begin{align*}
E[X; Y] &= a.X + b.X + c.Y + d.Y \\
E_1[\cdot] &= a.[] + b.X + c.Y + d.Y \\
E_2[\cdot] &= a.X + b.[] + c.Y + d.Y \\
E_3[\cdot] &= a.[] + b.[] + c.Y + d.Y \\
E[X; Y] &= E_1[X] = E_2[X] = E_3[X]
\end{align*}
\end{examples}
\begin{definition}
\begin{alltt}
\HOLConst{weakly_guarded1} \HOLFreeVar{E} \HOLSymConst{\HOLTokenEquiv{}}
\HOLSymConst{\HOLTokenForall{}}\HOLBoundVar{X}. \HOLBoundVar{X} \HOLSymConst{\HOLTokenIn{}} \HOLConst{FV} \HOLFreeVar{E} \HOLSymConst{\HOLTokenImp{}} \HOLSymConst{\HOLTokenForall{}}\HOLBoundVar{e}. \HOLConst{CONTEXT} \HOLBoundVar{e} \HOLSymConst{\HOLTokenConj{}} (\HOLBoundVar{e} (\HOLConst{var} \HOLBoundVar{X}) \HOLSymConst{=} \HOLFreeVar{E}) \HOLSymConst{\HOLTokenImp{}} \HOLConst{WG} \HOLBoundVar{e}
\hfill{[weakly_guaded1_def]}
\end{alltt}
\end{definition}
\end{small}
\end{frame}

%\begin{frame}
%\frametitle{Open problems}
%
%\end{frame}
\begin{frame}
\frametitle{Future directions}
\begin{enumerate}
\item Formalizing unique-solution theorems under multi-variable
  equations/contractions (represented as CCS terms with free variables).
\item Formalizing CCS theorems related to free/bound names and
  free/bound variables. (learning from experiences from $\pi$-calculi formalisations)
\item Formalizing Sangiorgi's 2017 work (Divergence and Unique
  Solution of Equations) and other CCS results in frontier.
\item Decision procedure for various bisimilarity. (Concurrency
  Workbench in HOL4)
\item CCS with arbitrary sums; Deeper look at the ``coarsest (pre)congruence'' theorem without
 classic assumptions.
\item Connecting CCS Theory with Graph Theory (LTS as digraph) and
  Probability Theory.
\end{enumerate}
\end{frame}

\begin{frame}
\frametitle{Conclusions}
\begin{enumerate}
\item Now we have a \emph{more} complete formalisation of CCS with an
  archive of rigous formal proofs of related theorems and lemmas in textbook.
\item Sangiorgi's theorem on unique solution of contractions is formally verified and
  slightly extended. (frontier)
\item \emph{Sometimes} formalising a theory helps in finding
  new interesting results or refining previously known results.
\item This work could be a template or working basis for CCS extensions or other process calculi, even in other
  theorem provers than HOL.
\end{enumerate}

This CCS formalisation will be continuously maintained as part of HOL4
official examples to make sure its long-term availability.
\end{frame}

%% Proof examples:

% 1. NIL has no transition (2 proofs, with, w/o TRANS_ind)
% 2. Transitivity of ObsCongr. (EPS_ind_right and EPS_ind)

%% discovery of new theorem and definitions.
